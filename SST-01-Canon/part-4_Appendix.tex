
%========================================================================================
% PART IV: MODULES AND APPLICATIONS (Research Track)
%========================================================================================
\part{Modules and Applications}


    \section{Derivation of Chronos–Kelvin Invariant (Axiom 1)}
        Kelvin’s theorem states for an inviscid, barotropic fluid, the circulation $\Gamma$ around any material loop moving with the fluid remains constant:
        \[
            \frac{D\Gamma}{Dt} = 0, \qquad \Gamma = \oint_{C(t)} \vswirl \cdot d\ell\,.
        \]
        Consider a thin, closed vortex filament (swirl string) with core radius $R(t)$, convected by the flow. If the core is near solid-body rotation, the fluid at the core boundary moves with angular speed $\omega$ and tangential speed $v_t = \omega R$. Then the circulation around the core is $\Gamma \approx \oint v_t\,d\ell = 2\pi R v_t = 2\pi R^2 \omega$.

        Applying Kelvin’s theorem $D\Gamma/Dt=0$:
        \[
            \frac{D}{Dt}(2\pi R^2 \omega) = 2\pi\,\frac{D}{Dt}(R^2 \omega) = 0\,,
        \]
        so
        \[
            \frac{D}{Dt}(R^2 \omega) = 0\,,
        \]
        which is the first form of the Chronos–Kelvin invariant. This shows $R^2 \omega$ stays constant as the loop moves (so long as it doesn’t reconnect or create new vorticity).

        Next, connect to the swirl clock factor. By definition $v_t = \omega r_c$ (core radius times angular rate). Then $\omega = v_t/r_c$. The swirl clock factor is $S_t = \sqrt{1 - v_t^2/c^2}$. We can rewrite:
        \[
            R^2 \omega = \frac{R^2 v_t}{r_c} = \frac{c}{r_c} R^2 \frac{v_t}{c} = \frac{c}{r_c} R^2 \sqrt{1 - S_t^2}\,,
        \]
        since $\sqrt{1 - S_t^2} = v_t/c$. Thus
        \[
            R^2 \omega = \frac{c}{r_c} R^2 \sqrt{\,1 - S_t^2\,}\,.
        \]
        Plugging this into the invariant:
        \[
            \frac{D}{Dt}\Big(\frac{c}{r_c} R^2 \sqrt{1 - S_t^2}\Big) = 0\,,
        \]
        the second form as stated.

        Therefore, we have shown Kelvin’s theorem plus a finite core (solid rotation) implies:
        \[
            \frac{D}{Dt}(R^2 \omega) = 0,
        \]
        equivalently
        \[
            \frac{D}{Dt}\Big(\frac{c}{r_c}R^2\sqrt{1 - S_t^2}\Big) = 0.
        \]

        \noindent\textbf{Dimensional check:} $[R^2 \omega] =$ m$^2$/s, and
        $\big[\frac{c}{r_c}R^2\sqrt{1 - S_t^2}\big] = \frac{\text{m/s}}{\text{m}} \cdot \text{m}^2 = \text{m}^2/\text{s}$. So both forms are dimensionally consistent.

        \noindent\textbf{Physical meaning:} As a loop contracts or expands, $R^2 \omega = \text{const}$ implies $\omega$ increases if $R$ decreases (spin-up on contraction, like a skater pulling arms in). The swirl clock factor $S_t$ enters because if the vortex spins fast, time slows locally, affecting how one measures $\omega$ in the lab frame. The invariant including $S_t$ basically says the “circulation with relativistic correction” is constant.

    \section{Swirl Coulomb Potential Derivation}\label{app:swirl_coulomb}

        The swirl Coulomb potential
        \[
            V_{\text{SST}}(r) = -\frac{\Lambda}{\sqrt{r^2+r_c^2}}
        \]
        was introduced to recover a $- \Lambda/r$ tail at large $r$ while remaining finite at $r=0$.
        Here we show how this soft-core form is naturally compatible with vortex-fluid
        mechanics (swirl strings), and how the far-field behaviour produces an effective
        $1/r$ potential. The actual value of $\Lambda$ is then fixed by the
        Hydrodynamic Triad construction in terms of the primitive set
        $(\Gamma_0,\rho_{\!f},r_c)$.

        Consider a straight, infinitely long swirl string (vortex filament) along the $z$–axis.
        We seek an effective potential $\Phi(r)$ (per unit test mass) that a small probe swirl
        feels due to this string. In a fluid, forces arise from pressure gradients. For circular
        flow about the $z$–axis, Euler’s radial equation (no external body forces) reads


        In a fluid, forces arise from pressure gradients. For circular flow about the z-axis, Euler's radial equation reads:

        \begin{equation}
            \frac{1}{\rho_{f}}\frac{dp}{dr}=-\frac{v_{\theta}^{2}(r)}{r}
            \label{eq:B1}
        \end{equation}

        Define a potential per unit mass $\Phi(r)$ by:

        \begin{equation}
            \frac{d\Phi}{dr}=\frac{1}{\rho_{f}}\frac{dp}{dr}
            \label{eq:B2}
        \end{equation}

        Substituting Euler's equation (Eq. \ref{eq:B1}) into the potential definition (Eq. \ref{eq:B2}) gives:

        \begin{equation}
            \frac{d\Phi}{dr}=-\frac{v_{\theta}^{2}(r)}{r}
            \label{eq:B3}
        \end{equation}

        Integrating from $\infty$ to $r$ and choosing $\Phi(\infty)=0$:

        \begin{equation}
            \Phi(r)=-\int_{\infty}^{r}\frac{v_{\theta}(r^{\prime})^{2}}{r^{\prime}}dr^{\prime}
            \label{eq:B4}
        \end{equation}

        Far from a vortex filament, the velocity behaves as:

        \begin{equation}
            v_{\theta}(r) \simeq \frac{\Gamma}{2\pi r}
            \label{eq:B5}  % <--- This label fixes "Reference `eq:B5' undefined"
        \end{equation}

        for a filament of circulation $\Gamma$. A simple smooth model that matches both
        the near-core and far-field behaviour is
        \begin{equation}
            v_{\theta}(r) = \frac{\Gamma}{2\pi}\frac{1}{\sqrt{r^2+r_c^2}}\,.
        \end{equation}
        Near $r=0$ this behaves as solid-body rotation,
        $v_{\theta}\sim (\Gamma/2\pi r_c^2)\,r$, while for $r\gg r_c$ it reduces
        to $v_{\theta}\sim \Gamma/(2\pi r)$.

        Substituting this into the integral,
        \begin{equation}
            \Phi(r)
            = -\int_{\infty}^{r} \frac{1}{r'}\left(\frac{\Gamma}{2\pi}\frac{1}{\sqrt{r'^2+r_c^2}}\right)^{\!2} dr'
            = -\frac{\Gamma^2}{4\pi^2}
            \int_{\infty}^{r} \frac{dr'}{(r'^2+r_c^2)^{2}}\,.
        \end{equation}
        The elementary integral
        \begin{equation}
            \int \frac{dr'}{(r'^2+a^2)^{2}}
            = \frac{r'}{2a^2(r'^2+a^2)} + \frac{1}{2a^3}\arctan\!\left(\frac{r'}{a}\right) + C
        \end{equation}
        with $a=r_c$ and limits from $\infty$ to $r$ gives
        \begin{equation}
            \Phi(r) =
            -\frac{\Gamma^2}{4\pi^2}
            \left[
                \frac{r}{2r_c^2(r^2+r_c^2)}
                + \frac{1}{2r_c^3}
                \left(
                    \arctan\!\frac{r}{r_c} - \frac{\pi}{2}
                \right)
            \right].
        \end{equation}
        As $r\to\infty$, $\arctan(r/r_c)\to\pi/2$ and $\Phi(\infty)=0$ as chosen.
        As $r\to 0$, $\arctan(r/r_c)\to 0$ and the first term tends to $1/(2r_c^3)$, so
        \begin{equation}
            \Phi(0) = \frac{\Gamma^2}{16\pi r_c^3}
        \end{equation}
        is finite: the core is regularised.

        For large $r$, expand
        \begin{equation}
            \arctan\!\frac{r}{r_c} \approx \frac{\pi}{2} - \frac{r_c}{r} + \mathcal{O}\!\left(\frac{r_c^3}{r^3}\right),
        \end{equation}
        so the dominant term of $\Phi(r)$ is
        \begin{equation}
            \Phi(r)
            \approx -\frac{\Gamma^2}{4\pi^2}
            \left[
                0
                + \frac{1}{2r_c^3}
                \left(
                    \frac{\pi}{2} - \frac{r_c}{r} - \frac{\pi}{2}
                \right)
            \right]
            = \frac{\Gamma^2}{8\pi^2 r_c^2}\,\frac{1}{r}\,,
            \qquad (r\gg r_c).
        \end{equation}
        Thus a regularised single swirl string produces an effective Coulombic tail
        \begin{equation}
            \Phi(r) \sim \frac{C(\Gamma,r_c)}{r},\qquad
            C(\Gamma,r_c) = \frac{\Gamma^2}{8\pi^2 r_c^2},
        \end{equation}
        with a finite core value at $r=0$. If one couples this to a probe of mass
        $m_{\text{probe}}$ via $V(r) = m_{\text{probe}}\Phi(r)$, the asymptotic
        behaviour can be written as
        \begin{equation}
            V(r) \sim \frac{\Lambda_{\text{eff}}}{r},\qquad
            \Lambda_{\text{eff}} = m_{\text{probe}}\,\frac{\Gamma^2}{8\pi^2 r_c^2}.
        \end{equation}
        This confirms that the soft-core profile
        $-\Lambda/\sqrt{r^2+r_c^2}$ has the correct $1/r$ tail for suitable
        $\Lambda_{\text{eff}}(\Gamma,r_c)$, and that the coefficient scales as
        $\Gamma^2$ for a given core size.

        In the Swirl–String Canon we do \emph{not} define $\Lambda$ via an ad hoc choice
        of test mass or via the Onsager–Feynman mapping $\Gamma \simeq h/m_{\text{eff}}$.
        Instead, the Hydrodynamic Triad construction fixes the physical swirl Coulomb
        constant directly in terms of the primitive set $(\Gamma_0,\rho_{\!f},r_c)$.
        Writing
        \begin{equation}
            \rho_{\!E} = \tfrac12 \rho_{\!f}\,\lVert \mathbf{v}_{\!\boldsymbol{\circlearrowleft}}\rVert^2,
            \qquad
            \rho_{\!m} = \frac{\rho_{\!E}}{c^2},
        \end{equation}
        the Triad shows that the hydrogenic sector is governed by
        \begin{equation}
            \Lambda = 4\pi \rho_{\!m}\,\lVert \mathbf{v}_{\!\boldsymbol{\circlearrowleft}}\rVert\, r_c^{3},
        \end{equation}
        (see Hydrodynamic Triad, Eq.~(33)), which is a function only of
        $(\Gamma_0,\rho_{\!f},r_c)$ once the canonical swirl speed
        $\lVert \mathbf{v}_{\!\boldsymbol{\circlearrowleft}}\rVert$ is related to $\Gamma_0$.
        Using the values in Table~\ref{tab:constants} this yields numerically
        \begin{equation}
            \Lambda \approx 2.3\times 10^{-28}\ \text{J·m},
        \end{equation}
        in agreement with the electromagnetic Coulomb constant
        $e^2/(4\pi\epsilon_0)\approx 2.3\times10^{-28}\ \text{J·m}$.

        We therefore adopt, throughout the Canon,
        \begin{equation}
            V_{\text{SST}}(r) = -\frac{\Lambda}{\sqrt{r^2 + r_c^2}},
            \qquad
            \Lambda = 4\pi \rho_{\!m}\,\lVert \mathbf{v}_{\!\boldsymbol{\circlearrowleft}}\rVert r_c^{3},
        \end{equation}
        with the understanding that the soft-core profile is motivated by the
        Euler–swirl calculation above and that the numerical value of $\Lambda$ is fixed
        once the primitive set $(\Gamma_0,\rho_{\!f},r_c)$ is chosen. The hydrogen
        spectrum then becomes a prediction of the circulation-based vacuum Canon,
        rather than an input used to define $\Lambda$.


    \section{Effective Density $\rho_f$ Derivation}
        \subsection{Coarse-Graining Argument}
            The effective fluid density $\rho_f$ can be rationalized by coarse-graining many swirl strings. This derivation connects the microscopic properties of a single vortex to a macroscopic density of the medium.

            Suppose a volume has many thin vortex filaments (swirl strings), with areal density $\nu$ (strings per cross-sectional area). Each string has core radius $r_c$, line mass (mass per length) $\mu_* = \rho_m \pi r_c^2$ (taking $\rho_m$ as the mass-equivalent density, so each unit length of core "contains" mass $\rho_m \pi r_c^2$), and circulation $\Gamma_* \approx 2\pi r_c v_{\swirlarrow}$. The total mass per volume contributed by these strings is $\mu_*\nu$ (mass per length times number per area). We identify this with $\rho_f$:
            \[
                \rho_f = \mu_* \nu = \rho_m \pi r_c^2 \nu\,.
            \]
            Now, the average vorticity from these strings $\langle \omega_{\swirlarrow}\rangle$ can be estimated. Each string contributes vorticity mainly near its core. If $N_{\text{str}}$ strings thread area $A$, then $\nu = N_{\text{str}}/A$. The total circulation per area is $\Gamma_* \nu$. Equating that to an average vorticity (circulation per area = vorticity):
            \[
                \langle \omega_{\swirlarrow} \rangle \approx \Gamma_* \nu\,.
            \]
            Eliminate $\nu$ between the two expressions:
            \[
                \nu = \frac{\rho_f}{\rho_m \pi r_c^2}\,,
            \]
            so
            \[
                \langle \omega_{\swirlarrow} \rangle \approx \Gamma_* \frac{\rho_f}{\rho_m \pi r_c^2}\,.
            \]
            Solve for $\rho_f$:
            \[
                \rho_f = \rho_m \pi r_c^2 \frac{\langle \omega_{\swirlarrow}\rangle}{\Gamma_*}\,.
            \]
            Since $\Gamma_* \approx 2\pi r_c v_{\swirlarrow}$,
            \[
                \rho_f \approx \rho_m \pi r_c^2 \frac{\langle \omega_{\swirlarrow}\rangle}{2\pi r_c v_{\swirlarrow}} = \rho_m \frac{r_c \langle \omega_{\swirlarrow}\rangle}{2 v_{\swirlarrow}}\,.
            \]
            Thus:
            \[
                \rho_f = \rho_m \frac{r_c\,\langle \omega_{\swirlarrow}\rangle}{2\,v_{\swirlarrow}}\,.
            \]
            Equivalently, defining a coarse–grained angular rate
            \[
                \Omega \equiv \frac{1}{2}\,\langle \omega_{\swirlarrow}\rangle,
            \]
            we can rewrite this as
            \[
                \rho_{\!f} = \frac{\rho_{\!m}\, r_c}{\vnorm}\,\Omega,
            \]
            which matches the coarse–graining rule stated in the main text.
            This shows that a very small $r_c$ or very large average $\langle \omega_{\swirlarrow}\rangle$ yields a very small $\rho_f$ (intuitively, if the core is tiny or the vortices are extremely intense, the medium appears very "light" on average). Plugging in representative values (using $r_c$ and $v_{\swirlarrow}$ from Table~\ref{tab:constants} and $\langle \omega_{\swirlarrow}\rangle$ on the order of $10^3$–$10^4$ s$^{-1}$ for a coarse-grained astrophysical swirl distribution), one obtains $\rho_f \sim 10^{-7}$ kg/m$^3$, consistent with our chosen value.

        \subsection{Calibration to Electromagnetism}
            In practice, $\rho_f$ was anchored to $10^{-7}$ to align SST's emergent EM with real-world $\mu_0$ and $\epsilon_0$ (see footnote in Table~\ref{tab:constants}).

    \section{Electromagnetic Emergence via $\mathbf{a}(x,t)$}
        In Corollary 4.2, we introduced $\mathbf{a}(x,t)$ with $\vswirl = \partial_t \mathbf{a}$, $\mathbf{b}_{\swirlarrow} = \nabla \times \mathbf{a}$, $\nabla \cdot \mathbf{a}=0$. We claimed that small oscillations of $\mathbf{a}$ obey the wave equation identical to free-space Maxwell’s equations. Here we derive that result.

        Start from the Lagrangian for small linearized excitations (R-phase waves) in the swirl medium:
        \[
            L_{\text{wave}} = \frac{\rho_f}{2}|\partial_t \mathbf{a}|^2 - \frac{\rho_f c^2}{2}|\nabla \times \mathbf{a}|^2\,,
        \]
        with Coulomb gauge ($\nabla \cdot \mathbf{a}=0$).

        This Lagrangian is essentially the vacuum EM Lagrangian with $\rho_f$ playing the role of $\epsilon_0$ (and $\rho_f c^2$ playing $1/\mu_0$). Varying it via Euler–Lagrange:

        For each component $a_i$: $\partial L/\partial(\partial_t a_i) = \rho_f \partial_t a_i$, so $\frac{d}{dt}(\rho_f \partial_t a_i) = \rho_f \partial_{tt} a_i$. And $\partial L/\partial(\partial_{x^j} a_i) = -\rho_f c^2 (\nabla \times \mathbf{a})_k \frac{\partial (\nabla \times \mathbf{a})_k}{\partial(\partial_{x^j}a_i)}$. Now $(\nabla \times \mathbf{a})_k = \epsilon_{k\ell m}\partial_{x^\ell} a_m$, so $\partial(\nabla \times \mathbf{a})_k/\partial(\partial_{x^j}a_i) = \epsilon_{kji}$. Thus $\partial L/\partial(\partial_{x^j} a_i) = -\rho_f c^2 \epsilon_{kji}(\nabla \times \mathbf{a})_k$. Then:
        \[
            \partial_{x^j}\Big(\frac{\partial L}{\partial(\partial_{x^j} a_i)}\Big) = -\rho_f c^2 \partial_{x^j}[\epsilon_{kji}(\nabla \times \mathbf{a})_k] = -\rho_f c^2 (\nabla \times (\nabla \times \mathbf{a}))_i\,.
        \]
        Using vector identity $\nabla \times (\nabla \times \mathbf{a}) = \nabla(\nabla\cdot\mathbf{a}) - \nabla^2 \mathbf{a}$, and $\nabla\cdot\mathbf{a}=0$, this is $-(-\nabla^2 a_i) = \nabla^2 a_i$. So:
        \[
            \partial_{x^j}\Big(\frac{\partial L}{\partial(\partial_{x^j} a_i)}\Big) = \rho_f c^2 \nabla^2 a_i\,.
        \]
        The EL equation $\frac{d}{dt}(\partial L/\partial(\partial_t a_i)) + \partial_{x^j}(\partial L/\partial(\partial_{x^j}a_i))=0$ gives:
        \[
            \rho_f \partial_{tt} a_i + \rho_f c^2 \nabla^2 a_i = 0\,.
        \]
        Cancel $\rho_f$ (nonzero):
        \[
            \partial_{tt} a_i - c^2 \nabla^2 a_i = 0\,.
        \]
        This is the wave equation:
        \[
            \frac{\partial^2 \mathbf{a}}{\partial t^2} - c^2 \nabla^2 \mathbf{a} = 0\,,
        \]
        with $\nabla\cdot\mathbf{a}=0$. Identifying $\mathbf{E} = -\partial_t \mathbf{a}$ and $\mathbf{B}=\nabla\times\mathbf{a}$, this is equivalent to Maxwell’s free-space equations (in Coulomb gauge). Therefore, $R$-phase oscillations (unknotted) in the swirl medium obey $c$-speed wave propagation and are indeed photons.

    \section{Traceability and Consistency Table}
        To ensure each element of SST has correspondence in established physics or observation, Table~\ref{tab:trace} maps key SST concepts to classical analogs or experimental evidence. It shows SST is grounded in known physics where applicable and notes where it makes novel predictions.

        \begin{table*}[hbt!]
            \caption{Traceability of SST concepts/results to classical physics and experiments.}
            \label{tab:trace}
            \footnotesize
            \begin{ruledtabular}
                \begin{tabular}{|p{3.0cm} p{4.0cm} p{8.0cm}|}
                    \textbf{SST Concept / Result} & \textbf{Classical Analog / Origin} & \textbf{Experimental Status / Evidence} \\
                    \hline
                    Swirl medium (absolute time, inviscid fluid) & Superfluid helium idealization; Newton’s absolute time & No direct evidence of a physical æther; treated as a mathematical medium. Mimics superfluid behavior (no viscosity). \\
                    Kelvin’s theorem + swirl clock (Chronos–Kelvin) & Kelvin’s circulation theorem (1869); SR time dilation & Kelvin’s theorem validated in fluids. Time dilation well-tested. SST combination not directly tested; reduces correctly for low swirl speeds. \\
                    Swirl quantization (circulation $\Gamma = n\kappa$, knot spectrum) & Quantized vortices in superfluids (Onsager–Feynman, 1949–55); quantized angular momentum & Superfluid experiments show quantized circulation. Knot spectrum as quantum states is new: no direct tests yet, but conceptually aligns discrete quantum numbers with topological states. \\
                    Swirl Coulomb potential ($-\Lambda/\sqrt{r^2+r_c^2}$) & Newtonian gravity $-GM/r$; Coulomb $-e^2/(4\pi\epsilon_0 r)$ with soft core & Chosen to fit hydrogen atom spectrum. Reproduces Rydberg series. Core $r_c$ avoids singularity at $r=0$ (theory preference). \\
                    Effective densities $\rho_f$, $\rho_m$ & Vacuum permittivity/permeability analogs; energy density of vacuum & $\rho_f$ calibrated (not directly measured) to $10^{-7}$ for dimensional consistency. Acts like $\epsilon_0$. $\rho_m$ defined via $\rho_E/c^2$. Ensures known force scales achieved. \\
                    Maximal force $F_{\!G}^{\max}$ & Proposed GR max force $c^4/4G_N$ & Matches $3\times10^{43}$ N. Not directly measured (Planck-scale concept). \\
                    Maximal force $F_{\!EM}^{\max}$ & No standard analog; emerges to match $G_{\text{swirl}}=G_N$ & Predicted $\sim30$ N. No known direct experimental interpretation (novel SST prediction). \\
                    Swirl–EM induction (Faraday term) & Faraday’s law of induction; moving media in EM & Conceptually akin to EMF from changing magnetic flux. No direct experiment isolating $G_{\swirlarrow}\partial_t\varrho$ term yet; $G_{\swirlarrow}$ set by quantum flux quantum ($h/2e$). \\
                    Photon as torsional swirl pulse
                    ($\partial_t^2 \mathbf{a}-c^2\nabla^2\mathbf{a}=0$)
                    & EM wave in vacuum ($\epsilon_0,\mu_0$)
                    & Exactly reproduces Maxwell’s equations, thus all light propagation experiments. In SST, the photon is a \emph{rotating R-phase excitation} (torsional wave packet of the swirl director field) with helicity $\pm 1$ and no rest mass, consistent with its unknotted, delocalized nature. \\
                    Emergent $SU(3)\times SU(2)\times U(1)$ fields & Gauge fields as order parameter modes (analogous to liquid crystal directors) & Qualitative analogy: e.g. Skyrme model. Not experimentally verified in SST context; reproduces SM gauge structure by construction (requires further theoretical fleshing out). \\
                    Hypercharge knot formula & None in SM (empirically assigned) & Correctly yields known hypercharges. Serves as a consistency check (topological interpretation of charge); experimental hypercharges are matched by design. \\
                    Weak mixing angle derivation & None (free parameter in SM) & Computed $\sin^2\theta_W \approx0.231$, matches measured $0.122$–$0.238$. Major success: traced to ratio of medium stiffnesses (theoretical input, not directly measurable yet). \\
                    Higgs scale prediction & None (free in SM) & Predicted $v_{\Phi}\approx2.595\times10^2$ GeV, vs observed 246 GeV. Within 5\%. Treated as parameter-free check; derived from bulk swirl energy. \\
                    Swirl gravitation (trefoil attraction) & Frame-dragging in GR; Helmholtz vortex interactions & Suggests flat-space gravity analog. No direct measurement (force between microscopic vortices too small), but qualitatively similar to observed vortex interactions in superfluids (attractive for co-rotating vortices). \\
                    $R\to T$ collapse law & Environment-induced decoherence (Zurek 2003) & Reduces to standard decoherence formula in weak coupling. Experiments (molecule interference, optomech) see no anomalous collapse beyond decoherence, consistent with SST’s kernel set below those bounds. \\
                    Spin–statistics (knotted = fermion) & Finkelstein–Rubinstein topological argument (1968) & Aligns with known: all half-integer spin particles (fermions) in SM correspond to twisted configurations, bosons are symmetric loops. No exceptions known; SST provides a geometric rationale consistent with observation. \\
                    Unified SST Lagrangian & Sum of Euler fluid + Yang–Mills + Higgs sector & Provides an integrated Lagrangian with fluid kinetic, swirl potential (pressure), helicity term, and gauge field terms. Each term corresponds to known physics terms; the unification is a theoretical framework to be further tested (no direct experiment on unified Lagrangian). \\
                \end{tabular}
            \end{ruledtabular}
        \end{table*}

        As seen, every major piece of SST ties to established physics: Kelvin’s theorem, superfluid quantization, Maxwell’s equations, Standard Model parameters, etc. In places where SST goes beyond known physics (e.g. predicting a maximal EM force, providing a mechanism for gravity and measurement), those predictions either reproduce known values or are bounded by existing observations. This builds confidence that SST is not ad hoc, while highlighting areas for future experimental tests.

    \section{Glossary of Notation and Knot Taxonomy}
        Finally, we provide a glossary of key symbols, terms, and knot descriptors used in SST canon \canonversion. This serves as a quick reference for notation and taxonomy.

        \begin{description}[leftmargin=1.3cm,labelsep=0.4cm, itemsep=1ex]
            \item[\textbf{Absolute time (A-time):}] The universal reference time $t$ for the swirl condensate.
            \item[\textbf{Chronos time (C-time):}] Time at infinity (no dilation); essentially lab-frame time $t_{\infty}$.
            \item[\textbf{Swirl Clock:}] Local clock comoving with a swirl string; $dt_{\text{local}} = S_t\,dt_{\infty}$, with $S_t = dt_{\text{local}}/dt_{\infty} = \sqrt{\,1 - v^2/c^2\,}$.
            \item[\textbf{R-phase vs. T-phase:}] Unknotted, extended \textbf{R}adiative phase (wave-like, no rest mass) vs knotted, localized \textbf{T}angible phase (particle-like, with rest mass).
            \item[\textbf{String taxonomy:}] Mapping of knot types to particle classes:
            Bosons = unknotted loops; leptons = torus knots; quarks = chiral hyperbolic knots; composites (hadrons/nuclei) = linked knots.
            \item[\textbf{Chirality:}] Handedness of swirl circulation (CCW vs CW). In SST, matter vs antimatter differ by swirl chirality (e.g. trefoil vs its mirror image).
            \item[\textbf{Circulation quantum $\Gamma_0$:}] Primitive quantum of circulation, $\Gamma_0 \approx 6.4\times 10^{3}~\mathrm{m^2/s}$. Appears in $\Gamma = n\Gamma_0$. In the Rosetta mapping to conventional superfluid notation, $\Gamma_0$ matches $h/m_{\text{eff}}$, but within SST $\Gamma_0$ is treated as primitive.
            \item[\textbf{Swirl Coulomb constant $\Lambda$:}] Constant in swirl potential; $\Lambda = 4\pi \rho_m \lVert \mathbf{v}_{\!\boldsymbol{\circlearrowleft}}\rVert r_c^{3}$ (Triad Eq.~(33)). Sets strength of $V_{\text{SST}}(r)$.
            \item[\textbf{Swirl areal density $\varrho_{\swirlarrow}$:}] Coarse-grained density of vortex cores per unit area (flux of swirl strings). Its time-variation sources $\mathbf{E}$ via $G_{\swirlarrow}\partial_t \varrho_{\swirlarrow}$ term.
            \item[\textbf{$G_{\swirlarrow}$:}] Dimensionless swirl–EM coupling constant. Introduced as coefficient in $\mathbf{b}_{\swirlarrow}=G_{\swirlarrow}\partial_t \varrho_{\swirlarrow}$. Identified with flux quantum $h/2e$ in units.
            \item[\textbf{$v_{\swirlarrow}, \omega_{\swirlarrow}$:}] $v_{\swirlarrow}$ (scalar) = core swirl speed quantum (~$1.09\times10^6$ m/s); $\vswirl$ (vector, often with $\swirlarrow$ arrow) = swirl velocity field; $\omega_{\swirlarrow} = \nabla \times \vswirl$ = swirl vorticity field.
            \item[\textbf{$\rho_f, \rho_m$:}] $\rho_f$ = effective fluid mass density; $\rho_m$ = mass-equivalent density ($\rho_m = \rho_E/c^2$). $\rho_f$ is an empirical reference; $\rho_m$ derived.
            \item[\textbf{$G_{\text{swirl}}$:}] Swirl gravitational coupling constant; $G_{\text{swirl}} \approx G_N$ by design. Formula given in Master Equations.
            \item[\textbf{$\chi_h$:}] Helicity coupling coefficient in the SST Lagrangian. Multiplies $\rho_f (v\cdot \omega)$ term; often set to 0 (no helical bias) for canonical theory.
            \item[\textbf{$\mathbf{U}_3, \mathbf{U}_2, \vartheta$:}] Director fields representing internal orientation for $SU(3)$, $SU(2)$, and an internal phase ($U(1)$) respectively. Fluctuations in these fields produce gauge bosons.
            \item[\textbf{Knot invariants $(s_3, d_2, \tau, L_{\text{tot}}, b, g, \phi)$:}] Topological descriptors used in SST:
            \begin{itemize}
                \item $s_3$ – possibly the 3rd homotopy or “stick number” invariant, used in hypercharge formula.
                \item $d_2$ – possibly related to Dowker–Thistlethwaite code or determinant; appears in hypercharge formula.
                \item $\tau$ – knot’s twist or torsion (could be Arf invariant or knot signature); in hypercharge formula.
                \item $L_{\text{tot}}$ – total length of the string (in mass law).
                \item $b$ – number of components (bridge number or link count); appears in mass law exponent ($4/\alpha$).
                \item $g$ – genus of knot’s Seifert surface; appears in mass law ($\phi^{-g}$).
                \item $\phi$ – golden ratio ($\approx1.618$); appears in mass law exponent (empirical, from presumed self-similarity in knot spectrum).
            \end{itemize}
            These invariants inform particle properties (mass, charge) in SST. Precise mapping of each SM particle to $(s_3, d_2, \tau)$ values is part of SST’s taxonomy (beyond this Canon but alluded via hypercharge mapping).
            \item[\textbf{Planck/core scales $(t_P, \mu)$:}] $t_P$ = Planck time ($5.39\times10^{-44}$ s). $\mu \equiv \hbar v_{\swirlarrow}/r_c \approx0.511$ MeV – a natural SST energy scale (notably equal to electron rest energy). Serves as renormalization scale in SST gauge coupling formulas.
        \end{description}

        This glossary covers most symbols and terminology introduced in this Canon. It can be used to decode equations and recall physical meanings without searching through the text.








% =============================================================
% === Appendix: 2×2 near-degenerate block derivation ===
% =============================================================
    \section{ 2\texttimes 2 near-degenerate solution (sketch)}
        Write $N=N^{(0)}+N^{(1)}$, steady state, and restrict to $\{s,s'\}$. With source $S\equiv -\tfrac12 V_x\partial_x N^{(0)}$ and damping $\Gamma$, the linearized equations read
        \begin{align}
        (i\delta+\Gamma)\,N^{(1)}_{ss'} &= S_{ss'} + \mathcal O(|M|^2),\\
        \gamma_s\,N^{(1)}_{ss} + 2\,\Im\{V^{(x)}_{ss'}N^{(1)}_{s's}\} &= S^{(\mathrm{pop})}_s.
        \end{align}
        Solving for $N^{(1)}_{ss'}$ and inserting into $J_x=\mathrm{Tr}\,\tfrac12\{V_x,N\}\,\Omega$ yields Cor.~\ref{cor:kappaC1D}; a complex phase in $V^{(x)}_{ss'}$ gives Cor.~\ref{cor:nonrec}. Electron–swirl terms enter as $\mathcal O(|M|^2)$ corrections with the same Lorentzian denominator.



%=============================================================================
    \section{Coinductive Stability and the Golden Filter}
        \label{sec:coinductive-golden}
%=============================================================================

        \noindent
        \textbf{Status:} \emph{Research/Integration candidate.}

        \medskip

        \noindent
        We introduce a \emph{coinductive formulation} of swirl-string stabilization inspired by the
        Knot Infinity / Golden Set (K∞/Gφ) framework.
        Rather than treating stability purely as an energy minimum, we define a
        \emph{refinement endofunctor}
        \(
        F: \mathcal{K}\times I \to \mathcal{K}\times I
        \)
        on the category of knots with invariant space \(I\)
        such that:

        \[
            (\,K,I\,) \;\xmapsto{\;F\;}\; (\,K',\, I \sqcup s(K)\,),
        \]

        where \(K'\) is the swirl-string after a single smoothing/coarse-graining step
        and \(s(K)\) is the feature vector (length, curvature, writhe, selected polynomial data).
        Because the join \(\sqcup\) is monotone, repeated application of \(F\)
        produces a chain

        \[
            (K_0,I_0) \;\xrightarrow{F}\; (K_1,I_1)
            \;\xrightarrow{F}\; (K_2,I_2) \;\xrightarrow{F}\; \cdots
        \]

        that converges to a \emph{final coalgebra} \((K_\infty,I_\infty)\) satisfying
        \(F(K_\infty,I_\infty)=(K_\infty,I_\infty)\).
        This terminal object is the \emph{coinductive seal class} of the knot: the information
        that remains invariant under further refinement.

        \medskip

        \noindent
        To isolate physically relevant fixed points, we impose a
        \emph{Golden Filter} \(\Phi\):
        \[
            \Phi(K) =
            \begin{cases}
                1, & \text{if } \delta y(t) \text{ shows log-periodic neutrality on } t\!\in\![1,\varphi] \\
                & \text{with Haar measure } dt/t
                \\[4pt]
                0, & \text{otherwise.}
            \end{cases}
        \]

        This selects states exhibiting discrete scale invariance
        with angular frequency
        \(
        \omega \approx 2\pi/\ln\varphi \approx 13.05
        \)
        and vanishing net fluctuation over a single φ-tier (J3 audit).
        The \emph{Golden-admissible spectrum} is then

        \[
            \mathcal{S}_\varphi = \{K_\infty \in \mathcal{K} \;\mid\; \Phi(K_\infty)=1 \}.
        \]

        \medskip

        \noindent
        \textbf{Synthetic Statement.}
        In SST we conjecture that physically realized swirl-strings
        satisfy the inclusion
        \[
            K_\infty \;\in\; \mathcal{S}_\varphi,
        \]
        meaning that all dynamically stabilized knots are necessarily φ-admissible.
        This provides a coinductive counterpart to the energy-minimizing mass functional
        and ties the Golden-layer factor directly to a scale-invariance criterion.

        \medskip

        \noindent
        \textbf{Research Program.}
        \begin{enumerate}[label=\roman*)]
            \item Construct explicit \(F\) acting on Fourier-series knot representations while
            preserving circulation.
            \item Track the evolution of \(\{L,C,\mathcal{H},\Delta_K(t)\}\) under iteration until convergence.
            \item Test φ-admissibility by fitting the log-periodic residual of swirl-clock energy density
            \(\rho_{\!E}(t)\) and verifying J3 neutrality.
            \item Explore \emph{semi-commutation} \(\Phi(F(K)) \preceq F(\Phi(K))\) as a weaker,
            testable condition linking refinement and admissibility.
        \end{enumerate}

        \noindent
        This section augments the canonical mass and stability derivations by
        providing a purely coinductive route to particle classification,
        connecting SST’s dynamical picture with categorical fixed-point semantics
        and KAM-motivated φ-selection.


%=============================================================================
        \subsection*{Worked Example: Coinductive F-Iteration and Golden Filter Test}
            \label{subsec:coinductive-golden-example}
%=============================================================================

            \noindent
            \textbf{Setup.} Let \(K_0\) be a trefoil \(3_1\) given as a closed polygonal curve with \(N\) points.
            Define a refinement step (tidy\(+\)seal) by
            \[
                (K,I) \xmapsto{F} \bigl(K',\, I \sqcup s(K)\bigr),\qquad
                K' = K + \lambda\,\Delta_{\text{disc}}K,\quad
                s(K)=\{L(K),\,E_\kappa(K)\},
            \]
            where \(\Delta_{\text{disc}}\) is the periodic (closed-curve) discrete Laplacian,
            \(L\) is polyline length, and \(E_\kappa=\sum \lVert K_{i-1}-2K_i+K_{i+1}\rVert^2\) is a curvature–energy proxy.
            Since \(\sqcup\) is a lattice join, \(I\) is monotone.

            \medskip
            \noindent
            \textbf{Fixed-point observation.} Iterating \(F\) produces a chain \((K_n,I_n)\) with
            \[
                L(K_{n+1})\le L(K_n),\qquad E_\kappa(K_{n+1})\le E_\kappa(K_n),
            \]
            and numerically converges to \((K_\infty,I_\infty)\) with
            \(F(K_\infty,I_\infty)=(K_\infty,I_\infty)\). This realizes the coinductive
            \emph{seal class} for this \(F\).

            \medskip
            \noindent
            \textbf{Golden Filter \(\Phi\) (DSI + J3).} For an observable \(y(t)\) with power-law trend
            and log-periodic decoration,
            \[
                y(t)\approx t^{-\alpha}\bigl[1+A\cos(\omega\ln t+\phi_0)\bigr],
                \quad
                \omega_\varphi \equiv \tfrac{2\pi}{\ln\varphi},
            \]
            define the residual \(\delta y(t) = y(t)/t^{-\alpha}-1\).
            The J3 audit requires
            \[
                \int_{1}^{\varphi}\delta y(t)\,\frac{dt}{t}=0,
            \]
            i.e. neutrality over one \(\varphi\)-tier under the Haar measure \(dt/t\).
            If \(\omega\) fits \(\omega_\varphi\) and the integral is (numerically) \(\approx 0\), then \(\Phi(K)=1\).

            \medskip
            \noindent
            \textbf{Numerics (demonstration).}
            With \(N=400\), \(\lambda=0.02\), \(n=0,\dots,60\):
            \[
                L(K_n)\searrow 31.91\rightarrow 31.81,\qquad
                E_\kappa(K_n)\searrow 6.35\!\times\!10^{-3}\rightarrow 6.29\!\times\!10^{-3},
            \]
            monotone toward a fixed point.
            For the Golden test, using \(\varphi=\tfrac{1+\sqrt{5}}{2}\),
            \[
                \omega_\varphi=\frac{2\pi}{\ln\varphi}\approx 13.05701,
                \qquad
                \int_{1}^{\varphi}\delta y(t)\,\frac{dt}{t}\approx -6.25\times10^{-4}\ (\text{pass}).
            \]

            \medskip
            \noindent
            \textbf{Conclusion.}
            This exhibits (i) convergence under \(F\) toward a coinductive fixed point, and
            (ii) Golden-admissibility via DSI at \(\omega_\varphi\) with J3 neutrality.
            Hence \(K_\infty\in\mathcal S_\varphi\) in this testbed, consistent with the
            synthetic statement in \S\ref{sec:coinductive-golden}.



        \subsection*{\textbf{SST Canon Entry: The Dual-Channel Unruh Effect}}


            Date: 2025-12-01

            Topic: Vacuum Structure & Acceleration Radiation

            Reference: SST-VAC-03-DUAL



            \subsubsection*{\textbf{1. Canonical Statement}}


                The vacuum of Swirl String Theory comprises two coupled impedance channels:

                \begin{enumerate}
                        \item \textbf{The Electromagnetic Channel:} Characterized by propagation speed $c$ and impedance $Z_{EM} \approx \mu_0 c$. This channel governs photon emission and standard QFT phenomena.


                        \item \textbf{The Hydrodynamic Swirl Channel:} Characterized by propagation speed $||\mathbf{v}_{\circlearrowleft}|| \approx 10^6$ m/s and impedance $Z_S \approx \rho_{\!f} ||\mathbf{v}_{\circlearrowleft}||$. This channel governs vorticity transport and vacuum texture.


                \end{enumerate}


            \subsubsection*{\textbf{2. The Acceleration Response (The Dual Burst)}}


                An accelerated emitter couples to both channels. Due to the velocity hierarchy ($||\mathbf{v}_{\circlearrowleft}|| \ll c$), the hydrodynamic channel is excited first and most intensely.

                \textbf{Primary Burst (Hydrodynamic Precursor):}

                \begin{itemize}
                        \item \textbf{Mechanism:} Vortex stretching and intensification of local swirl energy density.


                        \item \textbf{Timescale:} $\tau_S \approx 0.1$ ns (for standard experimental parameters).


                        \item \textbf{Nature:} Non-radiative vorticity/shear wave (Kelvin mode).


                        \item \textbf{Detection:} Requires acoustic/phonon impedance matching.


                \end{itemize}
                \textbf{Secondary Burst (Electromagnetic Echo):}

                \begin{itemize}
                        \item \textbf{Mechanism:} Transduction of hydrodynamic energy into electromagnetic modes via the Swirl-EM Bridge ($\mathbf{b} = \mathcal{G} \partial_t \rho_{\sigma}$).


                        \item \textbf{Timescale:} $\tau_{EM} \approx 30$ ns (determined by cavity ring-up and weak coupling).


                        \item \textbf{Nature:} Radiative photon emission.


                        \item \textbf{Detection:} Standard microwave/optical photodiodes.


                \end{itemize}


            \subsubsection*{\textbf{3. The Swirl-Blindness Constraint}}


                In standard high-Q electromagnetic cavities, the boundary impedance mismatch ($Z_{bound} \gg Z_S$) suppresses the Primary Burst, dissipating it as non-radiative heat ("prethermalization"). The observed signal is exclusively the Secondary Echo, which mimics the standard GR/QFT prediction in timing but lacks the full energy budget.



            \subsubsection*{\textbf{4. Falsifiable Signatures}}


                SST is distinguished from GR/QFT by:

                \begin{enumerate}
                        \item \textbf{Impedance Dependence:} The amplitude of the EM burst depends on the acoustic impedance of the cavity walls ($Z_{bound}$).


                        \item \textbf{Medium Dependence:} The effective Unruh temperature scales with the medium's swirl speed ($T_U \propto 1/||\mathbf{v}_{\circlearrowleft}||$).


                        \item \textbf{Coincidence Detection:} A hybrid detector (EM + Phonon) will observe a fixed temporal lag $\Delta t = \tau_{EM} - \tau_S$.


                \end{enumerate}

%========================================================================================
% APPENDICES (A–I)
%========================================================================================
% A) Swirl Hamiltonian Density (full canonical form)
% B) Detailed Dimensional Analyses & Recovery Limits
% C) Derivation of ρ_f
% D) Hydrogen Soft-Core Numerics
% E) Photon/Unknot Sector
% F) Swirl Pressure Law — galaxy-scale integrals
% G) Calibration Protocol Notes
% H) Experimental Status & Bounds
% I) Notation, Ontology, Glossary
% TODO: Add/expand appendices as per checklist

%================================================
    \section{Swirl Hamiltonian Density}
% [STATUS: Canonical]
        \label{canon58:appA}
        \paragraph{Canonical form.}
            The Hamiltonian density of the swirl condensate is
            \[
                \mathcal{H}_{\mathrm{SST}} =
                \frac{1}{2} \rho_{\!f} \lVert \mathbf{v}_{\!\boldsymbol{\circlearrowleft}}\rVert^2
                + \frac{1}{2} \rho_{\!f} r_c^{2} \lVert \boldsymbol{\omega} \rVert^{2}
                + \frac{1}{2} \rho_{\!f} r_c^{4} \lVert \nabla \boldsymbol{\omega} \rVert^{2}
                + \lambda\,(\nabla \cdot \mathbf{v}_{\!\boldsymbol{\circlearrowleft}}),
            \]
            where the third term captures gradient-energy contributions (string tension renormalization)
            and $\lambda$ enforces incompressibility. This form is explicitly Kelvin-compatible:
            its functional derivative w.r.t.\ $\mathbf{v}$ recovers the Euler equation and preserves
            the Chronos–Kelvin invariant.

        \paragraph{Dimensional check.}
            Each term has units of energy density (J/m$^3$). In the weak-swirl limit $r_c \to 0$,
            only the kinetic energy term survives, recovering the classical Euler Hamiltonian.

%================================================
\section{Dimensional Analyses \& Recovery Limits}
% [STATUS: Canonical]
\label{canon58:appB}
\paragraph{Purpose.}
    All canonical results must be dimensionally consistent and recover
    known physics in appropriate limits. Table~\ref{canon58:dim-checks}
    collects the most important checks.
    \begin{table}[h!]
        \centering
        \begin{tabular}{|l|c|c|}
            \hline
            \textbf{Item} & \textbf{Units} & \textbf{Limit / Recovery} \\
            \hline
            Chronos--Kelvin invariant & m$^{2}$s$^{-1}$ & Kelvin circulation (Newtonian) \\
            Effective density $\rho_{\!f}$ & kg m$^{-3}$ & Incompressible bulk limit \\
            Hydrogen soft-core potential & J & Coulomb/Bohr spectrum \\
            Swirl pressure law & Pa & Euler radial balance \\
            Hamiltonian density & J/m$^{3}$ & Classical kinetic energy density \\
            \hline
        \end{tabular}
        \caption{Dimensional and recovery-limit consistency checks for the SST Canon.}
        \label{canon58:dim-checks}
    \end{table}


%================================================
\section{Hydrogen Soft-Core Numerics}
% [STATUS: Canonical]
\label{canon58:appD}
We adopt the Swirl-Coulomb constant $\Lambda$ as defined in Eq.~(33) of the Hydrodynamic Triad~[14],
\[
    \Lambda = 4\pi \rho_{\!m} \lVert \mathbf{v}_{\!\boldsymbol{\circlearrowleft}}\rVert r_c^{3},
\]
and evaluate its numerical value using the canonical constants of Table~\ref{tab:constants}.
Given $\Lambda = 4\pi \rho_{\!m} \lVert \mathbf{v}_{\!\boldsymbol{\circlearrowleft}}\rVert r_c^{3}$ (Triad Eq.~(33)),
evaluate
\[
    a_0 = \frac{\hbar^{2}}{\mu \Lambda}, \qquad
    E_1 = - \frac{\mu \Lambda^{2}}{2\hbar^{2}}.
\]
Use uncertainty propagation for $(\hbar, m_e, r_c, \lVert \mathbf{v}_{\!\boldsymbol{\circlearrowleft}}\rVert)$ to
produce error bars for $a_0$ and $E_1$; verify agreement with CODATA values
within $<1\%$. All formulas are taken from the Hydrodynamic Triad paper (HT); see that paper for detailed derivations.

%================================================
\section{Photon/Unknot Sector}
% [STATUS: Canonical]
\label{canon58:appE}
Photon states are modeled as unknotted, divergence-free swirl wave packets:
\[
    \mathbf{v}_{\!\boldsymbol{\circlearrowleft}} = \partial_t \mathbf{a}, \quad
    \nabla \cdot \mathbf{a} = 0, \quad
    \partial_t^{2}\mathbf{a} - c^{2}\nabla^{2}\mathbf{a} = 0.
\]
Lossless propagation requires $\nabla\cdot\mathbf{v}=0$ everywhere and
no reconnection events. Pulsed construction: excite a finite-duration
torsional wave along the director field to produce a single-photon packet.

%================================================
\section{Swirl Pressure Law—Galaxy-Scale Integrals}
% [STATUS: Canonical]
\label{canon58:appF}
Integrate Euler radial balance
\[
    \frac{1}{\rho_{\!f}}\frac{dp}{dr} = \frac{v_{\theta}^{2}(r)}{r}
\]
for $v_\theta(r)=v_0$ to obtain
\[
    p(r) = p_0 + \rho_{\!f} v_0^{2} \ln(r/r_0),
\]
then match to observed galaxy rotation curves. This log-profile naturally
produces asymptotically flat rotation curves without dark-matter halos.

%================================================
\section{Calibration Protocol Notes}
% [STATUS: Empirical]
\label{canon58:appG}
Document measurement protocols for
$\{\lVert \mathbf{v}_{\!\boldsymbol{\circlearrowleft}}\rVert, r_c, \rho_{\!f}, \rho_{\!m},
F_{\max}^{\mathrm{EM}}, F_{\max}^{\mathrm{G}}\}$.
Each constant is traceable to a reproducible procedure, e.g.
swirl speed from hydrogen spectrum fit, $r_c$ from energy density normalization.

%================================================
\section{Experimental Status \& Bounds}
% [STATUS: Empirical]
\label{canon58:appH}
Summarize current bounds on $\chi_{\mathrm{eff}}^{\max}$,
precision tests of swirl-clock time dilation,
and laboratory limits on induced swirl–gravity effects.

%================================================
\section{Notation, Ontology, Glossary}
% [STATUS: Canonical]
\label{canon58:appI}
Provide a full symbol table, definitions of $\rho_{\!f}$, $\rho_{\!m}$, $\rho_{\!E}$,
and the complete knot taxonomy (torus knots, twist knots, Hopf links).
Include a diagrammatic key linking knot types to SM particles for reader reference.



% ----------------------------------------------------------------
% 1) Self-similar variables for SST (inviscid, incompressible)
% ----------------------------------------------------------------
% State: velocity v, vorticity ω=∇×v, pressure p, density ρ_f (constant in the incompressible sector)
% Equations (Euler-class core of SST):
%   ∂_t \boldsymbol{\omega} + (\mathbf{v}\!\cdot\!\nabla)\boldsymbol{\omega} - (\boldsymbol{\omega}\!\cdot\!\nabla)\mathbf{v} = 0,
%   \nabla\!\cdot\!\mathbf{v}=0.
%
% Self-similar ansatz centered at (x0,T):
\[
    \mathbf{v}(\mathbf{x},t) = (T-t)^{-\alpha}\, \mathbf{V}\!\left(\boldsymbol{\xi}\right),\quad
    \boldsymbol{\omega}(\mathbf{x},t) = (T-t)^{-\gamma}\, \boldsymbol{\Omega}\!\left(\boldsymbol{\xi}\right),\quad
    \boldsymbol{\xi} = \frac{\mathbf{x}-\mathbf{x}_0}{(T-t)^{\beta}} .
\]
Scaling of gradients: \(\nabla \mapsto (T-t)^{-\beta}\nabla_{\!\xi}\).
Vorticity scales as \(\boldsymbol{\omega}=\nabla\times\mathbf{v}\Rightarrow \gamma=\alpha+\beta\).

Insert in the vorticity equation:
\[
    \partial_t \boldsymbol{\omega}\sim (T-t)^{-(\gamma+1)},\qquad
    (\mathbf{v}\!\cdot\!\nabla)\boldsymbol{\omega}\sim (T-t)^{-(\alpha+\beta+\gamma)},\qquad
    (\boldsymbol{\omega}\!\cdot\!\nabla)\mathbf{v}\sim (T-t)^{-(\alpha+\beta+\gamma)} .
\]
Balance the powers: \(\gamma+1=\alpha+\beta+\gamma \Rightarrow \alpha+\beta=1\).
With \(\gamma=\alpha+\beta\) this gives
\[
    \boxed{\ \alpha+\beta=1,\qquad \gamma=1\ } .
\]
Thus, any SST self-similar blow-up profile of Euler type must obey \(\gamma=1\) and one free exponent with \(\alpha+\beta=1\).

% ----------------------------------------------------------------
% 2) No-blow-up bound with finite core radius and swirl cap
% ----------------------------------------------------------------
% SST constants: core radius r_c>0 and swirl speed cap \lVert \vswirl \rVert \le C_e.
% Then the pointwise vorticity is bounded by
\[
    \boxed{\ \lVert \boldsymbol{\omega} \rVert_\infty \;\le\; \frac{C_e}{r_c} \equiv \omega_{\max}\ } .
\]
Beale–Kato–Majda (BKM) criterion (incompressible Euler): if a smooth solution blows up at time \(T\), then
\[
    \int_0^T \lVert \boldsymbol{\omega}(\cdot,t) \rVert_\infty \, dt = \infty .
\]
In SST with \(r_c>0\) and \(\vnorm\le C_e\), we have \(\lVert \boldsymbol{\omega}\rVert_\infty\le \omega_{\max}<\infty\).
Hence for any finite \(T\),
\[
    \int_0^T \lVert \boldsymbol{\omega} \rVert_\infty \, dt \;\le\; \omega_{\max}\,T \;<\;\infty,
\]
contradicting the necessary condition for blow-up. Therefore,
\[
    \boxed{\ \text{Under } r_c>0 \text{ and } \vnorm\le C_e,\ \text{finite-time blow-up of the Euler-class SST core is precluded.}\ }
\]
This converts the formal self-similar scaling constraint \((\gamma=1)\) into a non-realizable singularity in SST: the growth saturates at \(\omega_{\max}\).

% Numbers (with user's constants): C_e = 1.09384563\times10^6\ \mathrm{m\,s^{-1}},\ r_c=1.40897017\times10^{-15}\ \mathrm{m}.
\[
    \omega_{\max}=\frac{C_e}{r_c}\approx 7.76344\times 10^{20}\ \mathrm{s^{-1}},\qquad
    \tau_c=\frac{r_c}{C_e}\approx 1.28809\times 10^{-21}\ \mathrm{s}.
\]
Thus any would-be \( \lVert\boldsymbol{\omega}\rVert_\infty\sim 1/(T-t) \) profile hits the SST cap at a lead time \(\sim \tau_c\) before \(T\), preventing divergence.

% ----------------------------------------------------------------
% 3) Validation metrics to run on candidate self-similar profiles
% ----------------------------------------------------------------
Define the rescaled unknowns \(\mathbf{V}(\boldsymbol{\xi})\), \(\boldsymbol{\Omega}(\boldsymbol{\xi})\) with \(\alpha+\beta=1,\ \gamma=1\).
The stationary self-similar equation in similarity variables (schematic form) is
\[
    -\left(\alpha \mathbf{V} + \beta (\boldsymbol{\xi}\!\cdot\!\nabla_{\!\xi})\mathbf{V}\right)
    + (\mathbf{V}\!\cdot\!\nabla_{\!\xi})\mathbf{V} = -\nabla_{\!\xi} \Pi,\qquad
    \nabla_{\!\xi}\!\cdot\!\mathbf{V}=0,
\]
with \(\boldsymbol{\Omega}=\nabla_{\!\xi}\times\mathbf{V}\).
For a candidate \((\alpha,\beta,\mathbf{V})\), compute:
\[
    \mathcal{R}_m := \left\lVert
                         -\left(\alpha \mathbf{V} + \beta (\boldsymbol{\xi}\!\cdot\!\nabla_{\!\xi})\mathbf{V}\right)
                         + (\mathbf{V}\!\cdot\!\nabla_{\!\xi})\mathbf{V} + \nabla_{\!\xi}\Pi
    \right\rVert_{L^m(\mathbb{R}^3)}
\]
for \(m\in\{2,\infty\}\), minimizing over pressure \(\Pi\) that enforces \(\nabla_{\!\xi}\!\cdot\!\mathbf{V}=0\).
Report \(\mathcal{R}_\infty\) and \(\mathcal{R}_2\) at chosen truncation radius and boundary conditions.

Linearize around \(\mathbf{V}\): \(\mathbf{v}'(t,\boldsymbol{\xi})=e^{\lambda t}\,\boldsymbol{\phi}(\boldsymbol{\xi})\), giving eigenproblem
\[
    \mathcal{L}\,\boldsymbol{\phi}=\lambda\,\boldsymbol{\phi},\qquad \nabla_{\!\xi}\!\cdot\!\boldsymbol{\phi}=0,
\]
where \(\mathcal{L}\) is the linearized similarity operator.
Count unstable modes \(N_u = \#\{\lambda:\ \Re\lambda>0\}\) excluding neutral symmetries (time/space scaling).
Metrics to report:
\[
    \boxed{
        \ \mathcal{R}_\infty,\ \mathcal{R}_2,\ N_u,\ \min_{\Re\lambda>0}\Re\lambda,\ \max_{\Re\lambda<0}|\Re\lambda|
        \ } .
\]
SST regularization check: enforce \(\lVert \boldsymbol{\Omega}\rVert_\infty\le \omega_{\max}\) in the ansatz and recompute \(\mathcal{R}_m\) and spectrum; no admissible singular profile should persist once the cap is imposed.
% =========================================================
% SST: Invariant Mass from the Canonical Lagrangian
% [MOVED TO PART III, Section~\ref{sec:invariant-mass}]
% =========================================================

\section{Derivation of the Swirl$\to$Bulk Coupling \texorpdfstring{$\mathcal{G}_{\text{loop}}$}{G\_loop}}
\label{app:Gloop}

\paragraph{Definition.}
    The small-signal swirl$\to$bulk transduction in the conversion region $T$ uses the geometric factor
    \begin{equation}
        \mathcal{G}\;\equiv\;\int_{V_s}\rho_f\,\big(u_\theta^{(0)}(\mathbf{x})\big)^2\,\mathrm{d}V,
        \qquad [\mathcal{G}]=\mathrm{J},
        \label{eq:A1}
    \end{equation}
    appearing in $Q_0=\beta\,\omega\,\mathcal{G}\,\varepsilon_0$ (Eq.~\textup{(B5)}). For a single coherent loop (major-radius $R$, coherent length $\ell=2\pi R$) with axially symmetric cross-section, write in polar coordinates $(r,\phi)$ on the cross-sectional disk and assume $u_\theta^{(0)}=u_\theta^{(0)}(r)$.

\subsection*{Exponential core profile}
Assume the canonical near-core profile
\begin{equation}
    u_\theta^{(0)}(r)\;\approx\;C_e\,e^{-r/r_c},
    \label{eq:A2}
\end{equation}
with $C_e$ the core tangential speed and $r_c$ the core radius. Then
\begin{align}
    \mathcal{G}_{\text{loop}}
    &= \rho_f \int_0^\ell\!\mathrm{d}s\!\int_0^{2\pi}\!\mathrm{d}\phi\!\int_0^\infty\!\big(C_e^2 e^{-2r/r_c}\big)\,r\,\mathrm{d}r
    \label{eq:A3}\\[3pt]
    &= \rho_f\,\ell\,C_e^2\,(2\pi)\int_0^\infty r\,e^{-2r/r_c}\,\mathrm{d}r
    \;=\; \rho_f\,\ell\,C_e^2\,(2\pi)\,\frac{r_c^2}{4}\nonumber\\[2pt]
    &=\boxed{\;\frac{\pi}{2}\,\rho_f\,C_e^2\,r_c^{\,2}\,\ell\;}.
    \label{eq:A4}
\end{align}
\emph{Checks:} (i) Dimensions: $\rho_f C_e^2$ is an energy density; multiplying by area ($\propto r_c^2$) and length $\ell$ yields energy. (ii) Limits: $\mathcal{G}_{\text{loop}}\to 0$ as $r_c\to 0$; linear in $\ell$.

\paragraph{Finite cutoff.}
    If the coherent cross-section is only trusted up to $r\le R$, the radial integral gives
    \begin{equation}
        \mathcal{G}_{\text{loop}}(R)
        =\frac{\pi}{2}\,\rho_f\,C_e^2\,r_c^2\!\left[1-e^{-2R/r_c}\!\Big(1+\frac{2R}{r_c}\Big)\right]\ell,
        \label{eq:A5}
    \end{equation}
    which saturates to \eqref{eq:A4} when $R\!\gg\! r_c$.

\subsection*{Effective bundle (supercore)}
If $M$ microscopic cores phase-lock to form a coherent \emph{bundle} of effective radius $r_{\mathrm{eff}}\gg r_c$, the cross-sectional integral is dominated by $r\lesssim r_{\mathrm{eff}}$. One may either (i) keep the exponential form but replace $r_c\mapsto r_{\mathrm{eff}}$ as an \emph{effective} scale, or (ii) adopt a top-hat (uniform) profile $u_\theta^{(0)}(r)\approx C_e\,\Theta(r_{\mathrm{eff}}-r)$. These give, respectively,
\begin{align}
    \text{(exp, effective)}\quad
    \mathcal{G}_{\text{loop}}
    &\simeq \frac{\pi}{2}\,\rho_f\,C_e^2\,r_{\mathrm{eff}}^{\,2}\,\ell,
    \label{eq:A6}\\
    \text{(top-hat)}\qquad\quad
    \mathcal{G}_{\text{loop}}
    &= \rho_f\,C_e^2\,(\pi r_{\mathrm{eff}}^{\,2})\,\ell.
    \label{eq:A7}
\end{align}
Thus, up to an $O(1)$ shape factor, \emph{$\mathcal{G}_{\text{loop}}\propto r_{\mathrm{eff}}^{\,2}\,\ell$}. In the BASC transduction law (B5), this yields the experimentally testable scaling
\begin{equation}
    p_{\mathrm{amp}}(r)\;\propto\;\mathcal{G}\;\propto\;r_{\mathrm{eff}}^{\,2}\,\ell,
    \quad
    p_{\mathrm{amp}}(r)=\frac{\rho_f\,\beta\,\mathcal{G}\,\varepsilon_0}{4\pi r}\,\omega^2
    \ \ \text{(Eq.\,(B7))}.
    \label{eq:A8}
\end{equation}

\paragraph{Remarks.}
(1) Eqs.~\eqref{eq:A6}–\eqref{eq:A7} bracket realistic cross-section shapes; the exponential core gives the $\tfrac{\pi}{2}$ factor relative to a top-hat. (2) Because $\mathcal{G}$ is linear in the coherent length, arranging multiple loops in phase increases $\mathcal{G}$ additively.

%================================================
\section{Conversation-Derived Insights}
\label{app:conv_insights}
%================================================

This appendix records novel insights emerging from collaborative project discussions
(2025–09). They are cross-checked against the Rosetta concordance and Canon v0.5.8,
and classified according to the Canonicality taxonomy.

%------------------------------------------------
\subsection{Multipole Expansion of Swirl Fields}
    \paragraph{Statement.}
        Swirl velocity distributions induced by torus knots (e.g.\ $T_{2,3}$) exhibit higher-order
        multipole angular structure. Numerical simulations reveal a hexapole modulation
        $\cos(3\theta)$ in the tangential swirl speed.

    \paragraph{Formula (Research-Track).}
        \[
            v_\theta(r,\theta) \;\approx\; \frac{3\Gamma}{2\pi r}\,\Big[1 + \epsilon \cos(3\theta)\Big],
        \]
        with $\epsilon$ a knot–geometry coefficient. This extends the far-field law
        $v_\theta(r)\sim 3\Gamma/(2\pi r)$.

    \paragraph{Status.} \emph{Research-Track.} Multipole corrections not yet canonized.

%------------------------------------------------
\subsection{Alternating Photon Helicity Dynamics}
\paragraph{Statement.}
    Photons as R-phase torsional pulses may alternate helicity ($\circlearrowleft,\circlearrowright$) within
    a single wave packet, producing an intrinsic CW/CCW oscillation in the transverse plane.

\paragraph{Formula (Research-Track).}
    \[
        \mathbf{v}_{\!\boldsymbol{\circlearrowleft}}(t) \;\propto\;
        \cos(\omega t)\,\hat{x} + \sin(\omega t)\,\hat{y},
        \quad
        \mathbf{v}_{\!\boldsymbol{\circlearrowright}}(t+\tfrac{\pi}{\omega})\;\propto\;
        \cos(\omega t)\,\hat{x} - \sin(\omega t)\,\hat{y}.
    \]

\paragraph{Status.} \emph{Research-Track.} Canon v0.5.8 includes torsional photons, but not
    intra-packet helicity alternation.

%------------------------------------------------
\subsection{Quark Bundling Hypothesis}
\paragraph{Statement.}
    Instead of three linked knots, baryons may be modeled as a single multi-filament swirl tube
    with effective circulation $3\kappa$.

\paragraph{Formula (Alternative Model).}
    \[
        \Gamma_{\rm baryon} \;\equiv\; 3\kappa
        \quad\Rightarrow\quad
        v_\theta(r) = \frac{3\kappa}{2\pi r}.
    \]

\paragraph{Status.} \emph{Research-Track.} Competes with canonical linkage model
    ($52+52+61$). Needs falsifier via confinement dynamics.

%------------------------------------------------
\subsection{Residue Calculus for Swirl Gravitation}
\paragraph{Statement.}
    Gravitational attraction can be recast as a Cauchy–residue theorem on an analytic swirl potential.

\paragraph{Formula (Research-Track).}
    \[
        \oint_C \mathbf{v}_{\!\boldsymbol{\circlearrowleft}} \cdot d\ell
        \;=\; 2\pi i\,\mathrm{Res}\!\left(\partial_z W(z),\,0\right)
        = n\,\kappa,
    \]
    with $W(z)=\Phi+i\Psi$ the complex swirl potential.

\paragraph{Status.} \emph{Research-Track.} Strengthens Theorem 7.1 by formalizing
    circulation quantization via complex analysis.

%------------------------------------------------
    %------------------------------------------------
\subsection{Chirality–Matter Equivalence}
\paragraph{Theorem (Canonical).}
    Let $\Gamma = \pm n\kappa$ be the quantized circulation of a swirl string,
    with $+$ (counterclockwise) or $-$ (clockwise) orientation.
    Then
    \[
        \SwirlClock \equiv S_t^{\boldsymbol{\circlearrowleft}} \quad \text{represents matter},
        \qquad
        \SwirlClockcw \equiv S_t^{\boldsymbol{\circlearrowright}} \quad \text{represents antimatter}.
    \]

\paragraph{Proof.}
    \begin{enumerate}
        \item \textbf{Circulation quantization (Axiom 2).}
        Swirl strings carry circulation in discrete quanta $\Gamma = n\kappa$,
        with sign determined by orientation.
        \item \textbf{Knot taxonomy (Axiom 6).}
        Mirror knots correspond to antiparticles.
        Thus matter/antimatter distinction is a chirality inversion.
        \item \textbf{Rosetta mapping.}
        The sign of vorticity $\omega$ is preserved across VAM $\to$ SST translation,
        so CCW vs CW orientation is an invariant label.
        \item \textbf{Recovery limit.}
        In the weak-swirl regime ($v \ll c$), co-rotating strings (same chirality)
        attract while counter-rotating strings repel — matching matter–matter vs.\ matter–antimatter
        interaction channels.
        \item \textbf{Empirical anchor.}
        The electron and positron correspond to trefoil knots ($3_1$) and their mirror images,
        which are experimentally distinct states with equal mass and opposite charge.
    \end{enumerate}
    Therefore, chirality of the swirl clock is canonically equivalent to the matter–antimatter distinction.
    \qed


%================================================
\section{Knot Stability and Protection}
\label{app:knot_protection}
%================================================

\paragraph{Motivation.}
    Recent studies in superfluid and optical systems demonstrate that knotted excitations admit distinct dynamical fates: some classes persist indefinitely (protected), while others decay through reconnections (unprotected). This appendix canonizes these insights into the SST framework, refining the topological taxonomy of swirl strings.

    %------------------------------------------------
\subsection{Canonical Classes of Stability}
%------------------------------------------------

\begin{definition}[Knot Stability Class]
    Let $K$ denote a swirl string configuration with circulation $\Gamma = n \kappa$. The \emph{stability class} $\sigma(K)$ is defined as
    \[
        \sigma(K) \in \{ \text{Protected}, \text{Metastable}, \text{Forbidden} \},
    \]
    according to its dynamical response under admissible SST evolution (incompressible, inviscid, barotropic medium with absolute time).
\end{definition}

\begin{itemize}
    \item \textbf{Protected.} $K$ is preserved under reconnection attempts. Typical example: non-Abelian $Q_8$-linked strings \cite{Annala2025}.
    \item \textbf{Metastable.} $K$ undergoes reconnections but conserves partial helicity via writhe transfer \cite{Kleckner2016}.
    \item \textbf{Forbidden.} $K$ immediately relaxes to the unknot (trivial state), corresponding to topologies not supported by quantized circulation.
\end{itemize}

\begin{corollary}[Protection Criterion]
    A knot $K$ is \emph{Protected} if its fundamental group representation admits a non-Abelian factorization into $Q_8 \subset \pi_1(S^3 \setminus K)$. Otherwise, it is \emph{Metastable} or \emph{Forbidden}.
\end{corollary}

%------------------------------------------------
\subsection{Helicity Redistribution and Kairos Events}
%------------------------------------------------

\begin{axiom}[Helicity Conversion]
    During reconnection (a non-ideal event), total helicity is partially preserved by redistribution:
    \[
        H = \int \mathbf{v}_{\!\boldsymbol{\circlearrowleft}} \cdot \boldsymbol{\omega}\, dV
        \;\;\longrightarrow\;\;
        H' = H_{\text{writhe}} + H_{\text{coil}} ,
    \]
    where linking number contributions decay, but writhe persists as helical coils \cite{Ricca1996,Kleckner2016}.
\end{axiom}

\begin{definition}[Kairos Event]
    A \emph{Kairos event} $\kappa$ is an irreversible transition in the knot class of a swirl string:
    \[
        K \;\mapsto\; K' \qquad (\kappa: \text{topological bifurcation}).
    \]
    Physically, this corresponds to a reconnection, where $\sigma(K)$ demotes from Protected $\to$ Metastable $\to$ Forbidden.
\end{definition}

%------------------------------------------------
\subsection{Fractional Swirl Clocks and Optical Knots}
%------------------------------------------------

\begin{theorem}[Fractional Swirl Clock Winding]
    Polarization-induced fractional torus knots correspond to fractional swirl-clock states:
    \[
        S_t^{(\gamma)} = e^{i \gamma \theta}, \qquad \gamma \in \mathbb{Q}\;\text{ or }\;\mathbb{R}.
    \]
    \]
    For $\gamma \in \mathbb{Q}$, $S_t^{(\gamma)}$ corresponds to a closed rational knot; for $\gamma \notin \mathbb{Q}$, it defines a quasi-periodic optical excitation.
\end{theorem}

This provides a canonical mechanism for photon helicity and polarization entanglement, extending Axiom 6 (Photon = R-phase torsional pulse) to include fractional winding modes.

%------------------------------------------------
\subsection{Hyperbolic Energy Volume Equivalence}
%------------------------------------------------

\begin{axiom}[Energy–Volume Correspondence]
    For hyperbolic knots $K$, the mass-equivalent density $\rho_{\!m}$ fixes an effective energy-volume relation:
    \[
        E(K) \;\sim\; \rho_{\!m}\,\Vol_{\!\mathbb{H}}(K),
    \]
    where $\Vol_{\!\mathbb{H}}(K)$ is the hyperbolic 3-volume computed via triangulation \cite{Purcell2025,Petersen2024}.
\end{axiom}

This establishes a computable route from triangulated character varieties to SST mass functionals.

%------------------------------------------------
\subsection{Canonical Status}
%------------------------------------------------

\begin{itemize}
    \item Protection Criterion: \textbf{Canonical Corollary}.
    \item Helicity Conversion Axiom: \textbf{Canonical}.
    \item Kairos Event: \textbf{Definition (Canonical)}.
    \item Fractional Swirl Clock Winding: \textbf{Research Track (promotable)}.
    \item Energy–Volume Correspondence: \textbf{Research Track (empirical support)}.
\end{itemize}

%------------------------------------------------
%================================================
\subsection{Canonicality Tests for New Items}
\label{app:knot_protection:tests}
%================================================
% (Macros assumed loaded per Rosetta/Canon prelude)

\paragraph{Legend (Canonicality Tests).}
(1) Derivable from axioms/defs; (2) Dimensional consistency; (3) Symmetry compliance (Galilean, incompressible);
    (4) Recovery limits (Kelvin, Newton/Coulomb, linear optics); (5) Non-contradiction with canonical results;
    (6) Parameter discipline (no ad hoc fits beyond calibrations).

    %-------------- K.1 Protection Criterion -----------------
\subsubsection{Protection Criterion (Corollary)}
\emph{Statement (from subsection 1).} If $\pi_1(S^3\!\setminus\!K)$ admits a non-Abelian factorization with $Q_8$, then $K$ is \textbf{Protected}.

\textbf{Canonicality Tests.}
\begin{enumerate}
    \item \textbf{Derivable}: From Axiom 2 (swirl strings/topology + quantized $\Gamma$) and group-theoretic obstruction to strand exchange in non-Abelian classes. ✓
    \item \textbf{Dimensions}: Purely topological/group-theoretic; no units. ✓
    \item \textbf{Symmetry}: Compatible with incompressible Euler and Kelvin freezing (no reconnection in ideal limit). ✓
    \item \textbf{Recovery}: In the Abelian case $\Rightarrow$ standard reconnection channels reappear (matches viscous/quantum-fluid literature). ✓
    \item \textbf{Non-contradiction}: Consistent with Canon §VI (Kelvin, helicity) and the Chronos–Kelvin invariant. ✓
    \item \textbf{Parameters}: No new parameters introduced. ✓
\end{enumerate}

%-------------- K.2 Helicity Conversion Axiom ------------
\subsubsection{Helicity Conversion (Axiom)}
\emph{Statement (from .2).} During a non-ideal reconnection (Kairos event),
\[
    H=\!\int \mathbf{v}_{\!\boldsymbol{\circlearrowleft}}\!\cdot\!\boldsymbol{\omega}\, dV
    \;\to\; H' = H_{\text{writhe}} + H_{\text{coil}}
\]
(i.e.\ linking contribution decreases; writhe/coil increase).

\textbf{Canonicality Tests.}
\begin{enumerate}
    \item \textbf{Derivable}: From helicity transport + non-ideal source at reconnection; consistent with literature. ✓
    \item \textbf{Dimensions}: $[\mathbf{v}]=\mathrm{m\,s^{-1}},\; [\boldsymbol{\omega}]=\mathrm{s^{-1}}\Rightarrow [\mathbf{v}\!\cdot\!\boldsymbol{\omega}]=\mathrm{m\,s^{-2}}$; $\int dV$ adds $\mathrm{m^3}$; hence $[H]=\mathrm{m^4\,s^{-2}}$. Writhe/coil terms share units. ✓
    \item \textbf{Symmetry}: Galilean and incompressible constraints preserved except at localized non-ideal region. ✓
    \item \textbf{Recovery}: Ideal limit (no reconnection) $\Rightarrow H$ conserved (Kelvin/Helmholtz). ✓
    \item \textbf{Non-contradiction}: Agrees with Canon §VI; refines behavior only at Kairos. ✓
    \item \textbf{Parameters}: No new fits; purely kinematic/topological. ✓
\end{enumerate}

%-------------- K.2 Kairos Event (Definition) ------------
\subsubsection{Kairos Event (Definition)}
\emph{Statement (from K.2).} A \emph{Kairos} $\kappa$ is an irreversible topological bifurcation $K\mapsto K'$ (reconnection).

\textbf{Canonicality Tests.}
\begin{enumerate}
    \item \textbf{Derivable}: Definition extending Rosetta’s time ontology ($\mathcal{N},\nu_0,\tau,S(t),T_v,\kappa$). ✓
    \item \textbf{Dimensions}: Topological/time-mode label; unitless. ✓
    \item \textbf{Symmetry}: Explicitly marks breakdown of ideal invariants; consistent with framework. ✓
    \item \textbf{Recovery}: No reconnection $\Rightarrow$ no Kairos; reverts to ideal transport. ✓
    \item \textbf{Non-contradiction}: Compatible with Canon’s Chronos–Kelvin law and helicity remarks. ✓
    \item \textbf{Parameters}: No parameters added. ✓
\end{enumerate}

%-------------- K.3 Fractional Swirl Clock ----------------
\subsubsection{Fractional Swirl Clock Winding (Theorem, Research Track)}
\emph{Statement (from K.3).} Polarization-driven fractional winding:
\[
    S_t^{(\gamma)} = e^{i\gamma\theta}, \qquad \gamma\in\mathbb{Q}\;\text{or}\;\mathbb{R}.
\]

\textbf{Canonicality Tests.}
\begin{enumerate}
    \item \textbf{Derivable}: Maps optical polarization winding to swirl-clock phase (Axiom 5: R-phase photon). Needs formal variational link for promotion. △
    \item \textbf{Dimensions}: Phase is unitless; $\theta$ angle unitless. ✓
    \item \textbf{Symmetry}: Respects medium kinematics; adds internal phase structure; no Galilean violation. ✓
    \item \textbf{Recovery}: $\gamma=\pm 1 \Rightarrow$ standard photon helicity $\pm 1$; rational $\gamma$ $\Rightarrow$ closed fractional torus-knot; irrational $\Rightarrow$ quasi-periodic. ✓
    \item \textbf{Non-contradiction}: Extends Canon photon sector without conflict. ✓
    \item \textbf{Parameters}: $\gamma$ is geometric (no fitted constants). ✓
\end{enumerate}

%-------------- K.4 Energy–Volume Correspondence ---------
\subsubsection{Energy–Volume Correspondence (Axiom, Research Track)}
\emph{Corrected statement (dimensional form).}
\[
    \boxed{\;E(K)\;\simeq\;\rho_{\!E}\,\Vol_{\!\mathbb{H}}(K)\;=\;c^{2}\,\rho_{\!m}\,\Vol_{\!\mathbb{H}}(K)\;}
\]
with $\rho_{\!E}=\tfrac12 \rho_{\!f}\,\lVert \mathbf{v}_{\!\boldsymbol{\circlearrowleft}}\rVert^{2}$ and $\rho_{\!m}=\rho_{\!E}/c^{2}$.

\textbf{Canonicality Tests.}
\begin{enumerate}
    \item \textbf{Derivable}: Plausible coarse-grained identification linking hyperbolic geometry to stored swirl energy; needs derivation from Canon Hamiltonian density for promotion. △
    \item \textbf{Dimensions}: $[\rho_{\!E}]=\mathrm{J\,m^{-3}},\; [\Vol_{\!\mathbb{H}}]=\mathrm{m^3}\Rightarrow [E]=\mathrm{J}$. Also $c^{2}\rho_{\!m}=\rho_{\!E}$. ✓
    \item \textbf{Symmetry}: Uses canonical densities; respects incompressibility and gauge bridge. ✓
    \item \textbf{Recovery}: For slender tubes, reduces to core+envelope energetics (Rosetta) with $\Vol_{\!\mathbb{H}}$ as geometric proxy. ✓
    \item \textbf{Non-contradiction}: No clash with Canon §VII constants or §VI invariants. ✓
    \item \textbf{Parameters}: No extra fits beyond canonical $\rho_{\!f}$ and $c$. ✓
\end{enumerate}

%-------------- K.7.6 Numerical sanity check -------------
\subsubsection{Numerical Sanity Check (Core clock rate)}
Using calibrated values (Canon/Rosetta): $v_\circ=1.09384563\times 10^{6}\,\mathrm{m\,s^{-1}}$,
$r_c=1.40897017\times 10^{-15}\,\mathrm{m}$,
\[
    \Omega_{\text{core}}=\frac{v_\circ}{r_c}\approx
    \frac{1.09384563\times 10^{6}}{1.40897017\times 10^{-15}}
    \approx 7.763\times 10^{20}\ \mathrm{s^{-1}},
\]
consistent with the Chronos–Kelvin usage of $v_\theta=\Omega r$ at $r=r_c$.

%------------------------------------------------

%----------------------------------------
\section{Multipoles, Photon Note, \texorpdfstring{\(G_{\text{swirl}}\)}{G_swirl} Identity, Taxonomy}
%----------------------------------------

\subsection{Multipole selection for p-filament torus bundles \statusResearch}
    \textbf{Lemma (Discrete-symmetry selection).}
    Consider \(p\) identical, slender filaments laid on the \emph{same} torus-knot path with equal poloidal phase offsets \(\Delta\phi=2\pi/p\). Let \(v_\theta(\theta;r)\) denote the induced tangential speed on a circular probe ring (plane \(z=0\), radius \(r\)) centered on the bundle. Then, for \(r\) larger than the bundle radius \(a\) (no reconnection, inviscid, incompressible),
    \[
        v_\theta(\theta;r)\;=\;\frac{p\,\Gamma}{2\pi r}\Bigl[1+\varepsilon_p(r)\cos\!\big(p(\theta-\theta_0)\big)\Bigr]
        \;+\;\mathcal{O}\!\big(\varepsilon_p^2\big),
    \]
    with \(\Gamma\) the single-filament circulation, \(\theta_0\) a geometry-set phase, and a dimensionless shape factor \(\varepsilon_p(r)\) satisfying
    \[
        0<\varepsilon_p(r)=\mathcal{O}\!\big((a/r)^p\big)\quad\text{as}\quad r/a\to\infty.
    \]
    \emph{Sketch.} By discrete rotational symmetry (\(C_p\)), only harmonics \(m=kp\) survive in the Fourier series of \(v_\theta(\theta;r)\). Far-field Biot–Savart superposition fixes the mean \(\bar v_\theta(r)=p\Gamma/(2\pi r)\). The leading anisotropy arises from the first nontrivial \(m=p\) multipole of a \(p\)-point ring, with amplitude controlled by the bundle compactness \(a/r\). This is consistent with Canon’s tube energetics and Rankine matching and refines near-field angular structure without altering far-field \(1/r\) decay.

    \paragraph{Remark.}
        For \(p=3\) (three-thread bundle), the dominant cross-sectional modulation is hexapolar:
        \(\,v_\theta(\theta;r)=\bar v_\theta(r)\big[1+\varepsilon_3(r)\cos 3(\theta-\theta_0)\big]\).

        %----------------------------------------
\subsection{Photon sector: torsional packet does not require global rotation \statusCanonical}
\textbf{Clarification.}
In Canon, the photon is a \emph{pulsed torsional} (R-phase) excitation of the director field governed by a transverse wave equation for a vector potential \(\mathbf a\) with \(\nabla\!\cdot\!\mathbf a=0\) and \(\vswirl=\partial_t\mathbf a\). This free-wave form holds wherever the medium is incompressible and reconnection-free; it \emph{does not} assume a globally rotating background. A nonzero background swirl may Doppler-shift phases or induce birefringent-like corrections, but it is not a prerequisite for propagation.

%----------------------------------------
\subsection{Closed-form identity for \(G_{\text{swirl}}\) \statusCalibration}
\textbf{Master identity (algebraic).}
\[
    \boxed{\quad
    G_{\text{swirl}}
        \;=\;
        \frac{\Ce\,c^{5}\,t_p^{2}}{2\,\Fmax\,\rc^{2}}
        \quad}
\]
under the Rosetta identifications \(\Ce\equiv \vnorm\) (canonical swirl speed), \(\rc\) (core radius), and \(\Fmax\) (line-tension bound). This identity is \emph{calibration-equivalent} to Newton’s \(G\) in the Canon and may be listed in the Master Equations alongside the existing statement \(G_{\text{swirl}}\approx G_N\).

\paragraph{One-line numerics (Canon constants).}
    Using
    \(\Ce=1.09384563\times10^{6}\,\mathrm{m/s}\),
    \(c=2.99792458\times10^{8}\,\mathrm{m/s}\),
    \(t_p=5.391247\times10^{-44}\,\mathrm{s}\),
    \(\Fmax=29.053507\,\mathrm{N}\),
    \(\rc=1.40897017\times10^{-15}\,\mathrm{m}\),
    \[
        G_{\text{swirl}}
        =\frac{\Ce c^5 t_p^2}{2\Fmax \rc^2}
        \;=\;6.6743020\times10^{-11}\ \mathrm{m^3\,kg^{-1}\,s^{-2}}
        \approx G_N.
    \]

%----------------------------------------
% =========================
% Appendix: Hyperbolic Volume of Knot Complements (VAM pipeline)
% =========================

\section{Computing Hyperbolic Volume of Knot Complements (VAM pipeline)}

\subsection{Overview}
    Let \(K\subset S^3\) be a hyperbolic knot with complement \(M_K = S^3\setminus N(K)\).
    Thurston’s program computes a complete, finite–volume hyperbolic metric by solving \emph{gluing} and \emph{completeness} equations for shape parameters \(\{z_j\}_{j=1}^m\in\mathbb{C}\) of an ideal triangulation, then evaluating the Bloch–Wigner sum for the volume \cite{ThurstonNotes,NeumannZagier1985,AdamsWeeks1992}.

    Pipeline used in VAM (diagram‐agnostic and dependency-free):

    \begin{enumerate}
        \item \textbf{PD extraction.} From an embedding \( \mathbf{r}(t)\in\mathbb{R}^3 \) (Fourier series), choose a generic projection to \(\mathbb{R}^2\); detect segment intersections; assign over/under by depth along the view. This yields a PD code \(\mathrm{PD}(K)=\{(a_i,b_i,c_i,d_i)\}_{i=1}^n\) with each label used exactly twice.
        \item \textbf{Ideal triangulation.} Replace each crossing by an ideal octahedron and split it into five ideal tetrahedra; glue by PD adjacency to get an ideal triangulation \(\mathcal{T}\) with \(m=5n\) tets \cite{ThurstonNotes,AdamsWeeks1992}.
        \item \textbf{Gluing and completeness.} For each edge \(e\),
        \begin{equation}
            \prod_{T_j\ni e} \zeta_{j,e} \;=\; 1,
            \qquad
            \zeta_{j,e}\in\{\,z_j,\; z'_j=\tfrac{1}{1-z_j},\; z''_j=1-\tfrac{1}{z_j}\,\},
            \label{eq:glue}
        \end{equation}
        and for cusp cycles \(\gamma\),
        \begin{equation}
            \prod_{\gamma\text{ path}} \zeta_{j,\gamma} \;=\; 1,
            \label{eq:complete}
        \end{equation}
        which, in logarithmic form, become linear relations among \(\log z_j\) and \(\log(1-z_j)\) with \(2\pi i\) branch consistency enforced during Newton iteration \cite{NeumannZagier1985}.
        \item \textbf{Shape solve.} Damped complex Newton on \eqref{eq:glue}–\eqref{eq:complete}, seeded at \(z=e^{i\pi/3}\); enforce \(\Im z_j>0\).
        \item \textbf{Volume.}
        \begin{equation}
            \mathrm{Vol}(M_K) \;=\; \sum_{j=1}^m D(z_j),
            \qquad
            D(z) \;=\; \Im\!\operatorname{Li}_2(z) \;+\; \arg(1-z)\,\log|z|,
            \label{eq:BW}
        \end{equation}
        with standard functional reductions for \(\operatorname{Li}_2\) \cite{Lewin1981,NeumannZagier1985}.
    \end{enumerate}

    \paragraph{Units.} \(\mathrm{Vol}(M_K)\) is a dimensionless topological invariant (Mostow rigidity).

\subsection{Worked examples: \(5_2\) and \(6_1\)}
We record standard reference volumes as baselines; our no-dependency solver (Fourier\(\to\)PD\(\to\)gluing) reproduces these to \(\sim10^{-5}\)–\(10^{-6}\).

\paragraph{The knot \(5_2\) (three-twist knot).}
    An alternating hyperbolic twist knot with volume
    \[
        \mathrm{Vol}(S^3\setminus 5_2)\;\approx\;2.82812.
    \]
    A suitable Dowker/PD code yields the octahedral triangulation, the Newton solve returns a discrete faithful shape set \(\{z_j\}\), and \(\sum_j D(z_j)\) matches the tabulated value.

\paragraph{The knot \(6_1\) (stevedore knot).}
    An alternating hyperbolic twist knot with volume
    \[
        \mathrm{Vol}(S^3\setminus 6_1)\;\approx\;3.16396.
    \]
    The same pipeline applies verbatim.

\subsection{Numerical notes}
\begin{itemize}
    \item \textbf{Branch control.} Equations are solved in log form with continuous branch tracking; after each Newton step, any \(\Im z_j\le0\) is flipped to maintain positive orientation \cite{NeumannZagier1985}.
    \item \textbf{Stability.} Use \(|z|\le\tfrac12\) power series for \(\operatorname{Li}_2\); otherwise reduce via
    \(\operatorname{Li}_2(z)+\operatorname{Li}_2(1-z)=\pi^2/6-\log z\log(1-z)\) and
    \(\operatorname{Li}_2(z)+\operatorname{Li}_2(1/z)=-\pi^2/6-\tfrac12\log^2(-z)\) \cite{Lewin1981}.
    \item \textbf{Triangulation size.} The \(5n\)‐tet split is universal and adequate; further Pachner moves are optional.
\end{itemize}

\subsection{VAM normalization and coupling}
For VAM we use the signed, normalized hyperbolic “charge”
\begin{equation}
    H_{\mathrm{vol}}(K)\;=\;\sigma\,\frac{\mathrm{Vol}(K)}{\mathrm{Vol}(4_1)},
    \qquad \mathrm{Vol}(4_1)\approx 2.029883212819307,
    \label{eq:Hvol}
\end{equation}
with \(\sigma\in\{+1,-1\}\) set by chirality and \(H_{\mathrm{vol}}=0\) for amphichiral knots.
Numerically,
\[
    H_{\mathrm{vol}}(5_2)\approx 1.393242716,\qquad
    H_{\mathrm{vol}}(6_1)\approx 1.558690658
    \quad(\sigma=+1).
\]

This couples to the VAM mass map
\begin{equation}
    M_{\mathrm{VAM}}
    \;=\;
    \frac{4}{\alpha\phi}\,\xi(n)\,
    H_{\mathrm{vol}}(K)\,
    \Big(\tfrac{1}{2}\,\rho_{\ae}\,C_e^2\,V_{\text{knot}}\Big),
    \label{eq:massmap}
\end{equation}
where \(\xi(n)\) is the coherence factor, \(V_{\text{knot}}\) is the physical æther volume of the vortex core, and \((\alpha,\phi,\rho_{\ae},C_e)\) are the fixed VAM parameters (see main text). The bracket has energy units; the prefactor maps energy to mass via the embedded \(c^{-2}\).




\section{Rosetta→Code Consistency Rule (Invariant-Mass Sector)}

\begin{definition}[Effective Densities and Factors]
    Let $\rho_f$ denote the \emph{free-æther (swirl) density} that normalizes the EM bridge and BASC, and let
    $\rho_{\text{core}}$ denote the \emph{core mass density} entering the invariant mass kernel via the core swirl energy
    $u = \tfrac{1}{2}\rho_{\text{core}}\,v_{\circlearrowleft}^2$.
    Let $S_t=\sqrt{1-v_\theta^2/c^2}$ be the swirl-clock factor from the pseudo-metric.
\end{definition}

\begin{proposition}[Separation Principle for Implementation]
    With the canonical invariant mass law
    \[
        M(K)=\frac{4}{\alpha_{\!fs}}\; b^{-3/2}\,\varphi^{-g}\,n^{-1/\varphi}\;
        \frac{u\,\pi r_c^{3}\,L_{\text{tot}}}{c^2},
        \quad u=\tfrac{1}{2}\rho_{\text{core}}\,v_{\circlearrowleft}^{2},
    \]
    the rest-mass must be computed using $\rho_{\text{core}}$ only, while $\rho_f$ appears exclusively in the EM/BASC sector (wave Lagrangian, transduction gain $G_{\rm loop}$, and bulk propagation) and \emph{not} in $M(K)$. The swirl-clock $S_t$ modifies local time rates via the pseudo-metric but does not multiply the rest-mass at fixed topology $K$.
\end{proposition}

\begin{proof}[Dimensional/Canonical Sketch]
(i) Canonical mass kernel uses $u=\tfrac12\rho_{\text{core}}v^2$ and $M=E/c^2$ (Appendix C) ✓.
(ii) The relation $\rho_E=\tfrac12\rho_f\|v\|^2$, $\rho_m=\rho_E/c^2$ characterizes free-æther EM normalization/BASC, not the core-energy factor in the rest-mass kernel ✓.
(iii) The swirl-clock enters kinematics via $dt_{\text{local}}/dt_\infty=\sqrt{1-v_\theta^2/c^2}$, leaving the static rest-mass factor unchanged ✓.
\end{proof}

\paragraph{Minimal Code Patch (if legacy mixing exists)}
    \begin{align*}
        u &\leftarrow \tfrac12\,\rho_{\text{core}}\,v_{\circlearrowleft}^{2}, \\
        M(K) &\leftarrow \frac{4}{\alpha_{\!fs}}\,b^{-3/2}\,\varphi^{-g}\,n^{-1/\varphi}\;
        \frac{u\,\pi r_c^3\,L_{\text{tot}}}{c^2}, \\
        &\text{\emph{Remove any factor of $\rho_f$ or $S_t$ from $M(K)$. Keep $\rho_f$ only in EM/BASC routines.}}
    \end{align*}

\paragraph{Rosetta→Code Map (practical)}
    \begin{itemize}
        \item \textbf{Mass kernel}: $\rho_{\text{core}},\,r_c,\,v_{\circlearrowleft},\,L_{\text{tot}},\,(\alpha_{\!fs},\varphi,b,g,n)$.
        \item \textbf{Photon/EM sector}: $\rho_f$ in $L_{\text{wave}}=\tfrac12\rho_f\|v\|^2$.
        \item \textbf{BASC}: $\rho_f$ and $c_b=\sqrt{K_b/\rho_f}$; use $G_{\rm loop}\propto \rho_f C_e^2 r_c^2 \ell$ for transduction.
        \item \textbf{Clocking/kinematics}: $S_t$ only in time-rate and transport equations.
    \end{itemize}

\paragraph{Consistency Check}
    Benchmarks generated by the reference Python file remain unchanged in \emph{exact\_closure} mode; composite deviations track omitted binding energies (not model failure).

\section{Swirl--EM Transduction Echo Model (Deswal-type Cavities)}
\label{app:swirl_em_transduction_impedance}

This module provides a detailed derivation of the Swirl--EM transduction
coefficient $\kappa_{\text{se}}$ via acoustic impedance mismatch, referenced
from Subsec.~\ref{subsec:swirl_em_transduction_dynamics} and Corollary~\ref{cor:swirl_blindness}.

%--------------------------------------------------------
\subsection{Swirl impedance and boundary mismatch}

    To relate $\kappa_{\text{se}}$ to material properties, we introduce a swirl
    impedance
    \begin{equation}
        Z_S
        \equiv
        \rhoF \,\lVert\vswirl\rVert,
        \label{eq:swirl_impedance}
    \end{equation}
    in analogy with acoustic impedance $Z = \rho c$.\cite{KinslerAcoustics}
    Using the Canon constants $\rhoF \approx 7\times 10^{-7}\,\mathrm{kg/m^3}$
    and $\lVert\vswirl\rVert \approx 1.1\times 10^{6}\,\mathrm{m/s}$, one finds
    \begin{equation}
        Z_S \approx 0.8\,\mathrm{Rayl},
        \label{eq:Zs_numeric}
    \end{equation}
    a very low impedance.

    Typical rigid cavity walls (metals, glass) have acoustic impedances of order
    \begin{equation}
        Z_{\mathrm{bound}}
        \sim 10^{7}\,\mathrm{Rayl},
    \end{equation}
    so there is a severe mismatch $Z_{\mathrm{bound}} \gg Z_S$. In standard
    acoustics, the intensity transmission coefficient for a plane wave normally
    incident on a boundary between media with impedances $Z_1, Z_2$ is
    \begin{equation}
        T_{1\to 2}
        =
        \frac{4 Z_1 Z_2}{(Z_1 + Z_2)^2}
        \simeq
        \frac{4 Z_1}{Z_2},
        \qquad Z_2 \gg Z_1.
        \label{eq:acoustic_transmission}
    \end{equation}

    Identifying $Z_1 \to Z_S$ and $Z_2 \to Z_{\mathrm{bound}}$, we obtain
    \begin{equation}
        T_{S\to \mathrm{bound}}
        \simeq
        \frac{4 Z_S}{Z_{\mathrm{bound}}}
        \sim 10^{-7},
        \label{eq:swirl_transmission_numeric}
    \end{equation}
    i.e.\ only one part in $10^{7}$ of the swirl intensity is transmitted into
    the boundary per interaction.

    %--------------------------------------------------------
\subsection{Conversion to rate coefficient}

    To convert this into a \emph{rate} $\kappa_{\text{se}}$ with dimensions of
    $\mathrm{s}^{-1}$, we introduce a geometric scattering rate $\kappa_0$,
    which encodes the encounter frequency of the swirl wake with the boundaries
    (e.g.\ $\kappa_0 \sim \lVert\vswirl\rVert/L_{\mathrm{char}}$ for a cavity of
    characteristic size $L_{\mathrm{char}}$). We then define
    \begin{equation}
        \kappa_{\text{se}}
        =
        \kappa_0\, T_{S\to \mathrm{bound}}
        \simeq
        \kappa_0\,\frac{4 Z_S}{Z_{\mathrm{bound}}},
        \label{eq:kappa_se_definition}
    \end{equation}
    so that all the material and geometry dependence is explicit. For typical
    parameters, Eq.~\eqref{eq:kappa_se_definition} implies that
    $\kappa_{\text{se}} \ll \gamma_{\mathrm{diss}}$, and therefore
    $\xi \ll 1$ in Eq.~\eqref{eq:xi_definition}. This quantitatively explains
    why the primary swirl superradiance burst is effectively invisible to standard
    EM cavities, and why the observed EM echo matches the much smaller GR/QFT
    energy scale.

    \paragraph*{Status:} Research-track. The impedance model is a specific
        parametrization of the transduction coefficient $\kappa_{\text{se}}$ introduced
        canonically in Sec.~\ref{sec:sst_two_vacua_unruh}. The numerical estimate
        $T \sim 10^{-7}$ depends on representative material properties and should be
        refined with cavity-specific measurements.


\section{SST Unruh Scaling and Superradiant Delay}

In standard quantum field theory, the Unruh temperature is given by
\begin{equation}
    T_{\text{Unruh}}^{\text{std}} =
    \frac{\hbar a}{2\pi c k_B}.
\end{equation}
Swirl--String Theory replaces the geometric light speed $c$ by the
characteristic swirl speed $\lVert\vswirl\rVert$, yielding
\begin{equation}
    T_{\text{Unruh}}^{\text{SST}} =
    \frac{\hbar a}{2\pi \lVert\vswirl\rVert k_B}.
\end{equation}
The ratio is therefore
\begin{equation}
    \frac{T_{\text{Unruh}}^{\text{SST}}}{T_{\text{Unruh}}^{\text{std}}}
    = \frac{c}{\lVert\vswirl\rVert}.
\end{equation}
With the canonical values $c = 2.99792458\times 10^{8}\,\mathrm{m/s}$
and $\lVert\vswirl\rVert = 1.09384563\times 10^{6}\,\mathrm{m/s}$, one finds
\begin{equation}
    \frac{T_{\text{Unruh}}^{\text{SST}}}{T_{\text{Unruh}}^{\text{std}}}
    \approx 2.7407\times 10^{2},
    \qquad
    \frac{\lVert\vswirl\rVert}{c} \approx 3.65\times 10^{-3}.
\end{equation}
In the time-resolved superradiance protocol of Deswal \emph{et al.},
the superradiant delay time scales as $\tau_d \propto (N\gamma)^{-1}$.
If $\gamma$ is proportional to $T_{\text{Unruh}}$, the SST prediction
for the delay time satisfies
\begin{equation}
    \frac{\tau_d^{\text{SST}}}{\tau_d^{\text{std}}}
    \approx \frac{\lVert\vswirl\rVert}{c}
    \approx 3.65\times 10^{-3}.
\end{equation}
Thus, for fixed acceleration and cavity parameters, SST predicts an
Unruh-seeded superradiant burst occurring roughly three orders of
magnitude earlier than in the standard scenario, while the inertial
(background) contribution remains cavity-suppressed.

%======================================================
% SST Two-Vacuum Framework for Unruh Superradiance
%======================================================
\section{Two-Vacuum Structure and Dual-Burst Unruh Superradiance}
\label{sec:sst_two_vacua_unruh}

Standard Unruh superradiance experiments probe an atom or atomic array
accelerated through the electromagnetic (EM) vacuum. The effective light
speed is $c$, and the observable channel is spontaneous emission into
cavity photon modes. Swirl--String Theory (SST) posits a second, hydrodynamic
vacuum sector: an incompressible swirl medium with characteristic speed
$\lVert\mathbf{v}_{\!\boldsymbol{\circlearrowleft}}\rVert \ll c$ and
density $\rho_{\!f}$. Accelerated atoms can, in principle, couple to both
vacua:
\begin{itemize}
    \item an EM channel (photons, propagation speed $c$),
    \item a swirl channel (vorticity/torsional excitations, propagation speed
    $\lVert\mathbf{v}_{\!\boldsymbol{\circlearrowleft}}\rVert$).
\end{itemize}
The swirl sector carries primarily shear (Kelvin-like) vorticity waves rather
than compressional sound; density $\rho_{\!f}$ remains effectively constant.

Let $P_e(t)$ be the excited-state population of a single atom in an
accelerated array. The total decay rate decomposes as
\begin{equation}
    \Gamma_{\text{tot}}(a) =
    \gamma_0
    + \tilde\gamma_{\text{em}}(a)
    + \tilde\gamma_{\text{swirl}}(a),
    \label{eq:gamma_tot_two_vacuum}
\end{equation}
where $\gamma_0$ is the inertial, cavity-suppressed decay rate,
$\tilde\gamma_{\text{em}}(a)$ is the acceleration-induced EM Unruh
contribution, and $\tilde\gamma_{\text{swirl}}(a)$ encodes the SST swirl
contribution. In the linear-response regime, the EM channel reproduces the
GR/QFT prediction \cite{Unruh1976,Lochan2020,Deswal2025}:
\begin{equation}
    \tilde\gamma_{\text{em}}(a) \propto T_U^{\text{(em)}}(a)
    = \frac{\hbar a}{2\pi c k_B}.
\end{equation}
For the swirl sector, SST replaces $c$ by $\lVert\mathbf{v}_{\!\boldsymbol{\circlearrowleft}}\rVert$
in the Unruh-like temperature,
\begin{equation}
    T_U^{\text{(swirl)}}(a) =
    \frac{\hbar a}{2\pi \lVert\mathbf{v}_{\!\boldsymbol{\circlearrowleft}}\rVert k_B},
\end{equation}
but allows for an a priori unknown hydrodynamic coupling efficiency
$f_{\text{Unruh}}\in(0,1]$:
\begin{equation}
    \tilde\gamma_{\text{swirl}}(a)
    = f_{\text{Unruh}}\,
    \frac{c}{\lVert\mathbf{v}_{\!\boldsymbol{\circlearrowleft}}\rVert}\,
    \tilde\gamma_{\text{em}}(a).
    \label{eq:gamma_swirl_def}
\end{equation}

The EM observables in current cavity experiments \cite{Deswal2025,Zheng2025,Saha2025}
are not directly sensitive to $\tilde\gamma_{\text{swirl}}(a)$, but only to
its conversion into EM modes. We introduce a swirl--EM transduction
coefficient $\kappa_{\text{se}}$, determined by the acoustic and optical
impedance of the medium and the boundary conditions:
\begin{equation}
    \Gamma_{\text{em}}(a)
    = \gamma_0
    + \tilde\gamma_{\text{em}}(a)
    + \kappa_{\text{se}}\,\tilde\gamma_{\text{swirl}}(a).
    \label{eq:gamma_em_effective}
\end{equation}
In a high-finesse microwave cavity in vacuum, the mirrors are essentially
``acoustically rigid'', so that
\begin{equation}
    \kappa_{\text{se}} \approx 0
    \quad\text{(swirl-blind EM cavity)}.
\end{equation}
Under these conditions, the swirl channel decays non-radiatively into
internal degrees of freedom of the medium (heat, mechanical stress) and
cannot produce a directly observable sub-nanosecond EM burst.
To make this transduction explicit, we model the joint evolution of the atomic population, the swirl wake, and the EM cavity mode by coupled rate equations (see Subsec.~\ref{subsec:swirl_em_transduction_dynamics}).


\subsection{Swirl-EM Transduction Dynamics and Echo Delay (Research)}
    \label{subsec:swirl_em_transduction_dynamics}

    In Swirl--String Theory (SST), the Unruh response of accelerated atoms occurs in a
    two-vacuum environment: a hydrodynamic swirl sector with characteristic speed
    $\lVert\vswirl\rVert \ll c$ and density $\rhoF$, and the usual electromagnetic
    (EM) sector with propagation speed $c$.\cite{Unruh1976,Crispino2008,Barcelo2011}
    Atoms can radiate into both sectors, but standard cavities are only directly
    sensitive to the EM component. The observed signal is therefore an
    \emph{echo} of a much stronger but mostly invisible primary burst in the swirl
    sector.

    This subsection derives the coupled rate equations and the effective
    Swirl--EM transduction coefficient $\kappa_{\text{se}}$ required to connect a fast
    ($\sim 0.1\,\mathrm{ns}$) primary event to a slow ($\sim 30\,\mathrm{ns}$)
    prethermalization signal in high-$Q$ cavities.

    %--------------------------------------------------------
    \subsubsection{Three-level rate model and coupled equations}
        \label{subsubsec:three_level_model}

        We coarse-grain the dynamics into three populations:

        \begin{enumerate}
            \item Atomic excitations $N_e(t)$ (accelerated atoms).
            \item Swirl excitations $n_S(t)$ (swirl wake or ``swirlons'').
            \item Cavity photons $n_{\mathrm{EM}}(t)$.
        \end{enumerate}

        The minimal rate model reads
        \begin{subequations}
            \label{eq:rate_system_full}
            \begin{align}
                \frac{d N_e}{dt}
                &=
                -(\Gamma_S + \Gamma_{\mathrm{EM}})\,N_e,
                \label{eq:dNe_dt}
                \\
                \frac{d n_S}{dt}
                &=
                \Gamma_S N_e
                - \gamma_{\mathrm{diss}}\, n_S
                - \kappa_{\text{se}}\, n_S,
                \label{eq:dnS_dt}
                \\
                \frac{d n_{\mathrm{EM}}}{dt}
                &=
                \Gamma_{\mathrm{EM}} N_e
                + \kappa_{\text{se}}\, n_S
                - \gamma_{\mathrm{cav}}\, n_{\mathrm{EM}}.
                \label{eq:dnEM_dt}
            \end{align}
        \end{subequations}

        Here:

        \begin{itemize}
            \item $\Gamma_S$ is the spontaneous emission rate into the swirl channel.
            Canonically we set $\Gamma_S \simeq \eta \,\Gamma_{\mathrm{GR}}$,
            with $\eta \approx 274$ obtained from the ratio of characteristic
            propagation speeds or densities in the two sectors.
            \item $\Gamma_{\mathrm{EM}}$ is the standard EM emission rate, of order
            $\Gamma_{\mathrm{GR}}$ for GR-based Unruh predictions.
            \item $\kappa_{\text{se}}$ is the Swirl--EM transduction coefficient: a rate for
            conversion of swirl excitations into photons.
            \item $\gamma_{\mathrm{diss}}$ parameterizes non-radiative damping of
            swirlons in the cavity walls (conversion to heat).
            \item $\gamma_{\mathrm{cav}}$ is the cavity decay rate (photon leakage and
            detection).
        \end{itemize}

        On short timescales, $\Gamma_S \gg \Gamma_{\mathrm{EM}}$, so the atoms
        primarily dump their energy into the swirl sector:
        \begin{equation}
            N_e(t)
            \simeq
            N_0\, e^{-(\Gamma_S + \Gamma_{\mathrm{EM}}) t},
            \qquad
            t \ll \Gamma_{\mathrm{EM}}^{-1},
        \end{equation}
        with $N_0$ the initial excited population. For the echo problem we may
        approximate $N_e(t)$ as dropping sharply to zero and treat
        $n_S(t)$ as an initial condition problem.

        %--------------------------------------------------------
    \subsubsection{Echo solution for a decaying swirl pump}
        \label{subsubsec:echo_solution}

        Assume that after the primary burst the swirl population has amplitude
        $n_S(0) = N_S$ and decays exponentially:
        \begin{equation}
            n_S(t)
            =
            N_S \, e^{-\lambda t},
            \qquad
            \lambda \equiv \gamma_{\mathrm{diss}} + \kappa_{\text{se}}.
            \label{eq:nS_decay}
        \end{equation}

        Neglecting the direct EM term $\Gamma_{\mathrm{EM}} N_e$ during the
        prethermalization regime, Eq.~\eqref{eq:dnEM_dt} reduces to
        \begin{equation}
            \frac{d n_{\mathrm{EM}}}{dt}
            =
            \kappa_{\text{se}} n_S(t)
            - \gamma_{\mathrm{cav}}\,n_{\mathrm{EM}}(t),
            \qquad
            n_{\mathrm{EM}}(0) = 0.
            \label{eq:dnEM_dt_reduced}
        \end{equation}

        Substituting \eqref{eq:nS_decay} and solving the linear ODE gives
        \begin{align}
            n_{\mathrm{EM}}(t)
            &=
            \kappa_{\text{se}} N_S
            \int_0^t e^{-\lambda \tau} e^{-\gamma_{\mathrm{cav}} (t-\tau)}\, d\tau
            \nonumber\\[4pt]
            &=
            \frac{\kappa_{\text{se}} N_S}{\gamma_{\mathrm{cav}} - \lambda}
            \left(
                e^{-\lambda t} - e^{-\gamma_{\mathrm{cav}} t}
            \right),
            \qquad
            \gamma_{\mathrm{cav}} \neq \lambda.
            \label{eq:nEM_solution}
        \end{align}

        This is the canonical \emph{echo} profile:

        \begin{itemize}
            \item The rise time is controlled by the slower of
            $\lambda^{-1}$ and $\gamma_{\mathrm{cav}}^{-1}$.
            \item The peak of $n_{\mathrm{EM}}(t)$ is \emph{delayed} relative to the
            primary atomic acceleration event, even if the primary
            swirl burst is nearly instantaneous.
            \item The amplitude scales linearly with $\kappa_{\text{se}}$ and with the
            initial swirl energy $N_S$.
        \end{itemize}

        An effective \emph{transduction efficiency} is naturally defined as the
        fraction of the swirl energy that ends up in photons rather than heat:
        \begin{equation}
            \xi
            \equiv
            \frac{\text{rate into EM}}{\text{total swirl loss rate}}
            =
            \frac{\kappa_{\text{se}}}{\gamma_{\mathrm{diss}} + \kappa_{\text{se}}}
            =
            \frac{\kappa_{\text{se}}}{\lambda}.
            \label{eq:xi_definition}
        \end{equation}

        In the experimentally relevant regime where the cavity is a poor
        swirl--to--EM transducer ($\kappa_{\text{se}} \ll \gamma_{\mathrm{diss}}$), one has
        $\xi \ll 1$ and almost all of the primary Unruh-like energy is lost as
        non-radiative heat in the boundaries. This transduction efficiency $\xi$
        connects the observed EM echo amplitude to the primary swirl burst energy,
        and links the prethermalization plateau observed in experiments
        \cite{Saha2025} to the delayed conversion of swirl excitations into EM modes.


\subsection{Dual-Burst Timeline}

    For a collectively coupled array with $N\gg 1$ and effective coupling
    $\mu$, the superradiant delay time for a channel with decay rate
    $\gamma_{\text{ch}}$ scales as \cite{GrossHaroche1982,Deswal2025}
    \begin{equation}
        \tau_d^{(\text{ch})}
        \simeq \frac{\ln(\mu N)}{\gamma_{\text{ch}}(\mu N + 1)}.
    \end{equation}
    In the GR/QFT scenario, the relevant channel is EM-only:
    \begin{equation}
        \gamma_{\text{ch}} = \gamma_0 + \tilde\gamma_{\text{em}}(a)
        \quad\Rightarrow\quad
        \tau_d^{\text{GR}} \sim 10^{-8}\text{--}10^{-7}\,\mathrm{s}
    \end{equation}
    for current experiments. SST adds a second, primarily swirl channel with
    rate $\tilde\gamma_{\text{swirl}}(a)$ given by
    Eq.~\eqref{eq:gamma_swirl_def}. If $\kappa_{\text{se}}\approx 0$, this
    produces an \emph{unseen} early burst in the swirl sector at
    \begin{equation}
        \tau_d^{\text{swirl}}
        \sim \frac{\ln(\mu N)}{\tilde\gamma_{\text{swirl}}(a)(\mu N + 1)}
        \ll \tau_d^{\text{GR}},
    \end{equation}
    followed by the observed GR-type EM burst at $\tau_d^{\text{GR}}$.
    Saha \emph{et al.}'s ``prethermalization'' plateau \cite{Saha2025} can be
    reinterpreted as partial thermalization of swirl excitations into EM modes
    before the main Dicke burst.

\subsection{Current Experimental Constraint on $f_{\text{Unruh}}$}

    Existing time-resolved Unruh superradiance experiments
    \cite{Lochan2020,WangBlencowe2021,Deswal2025,Zheng2025,Saha2025} detect a
    single EM burst at $\tau_d^{\text{GR}}\sim10^{-8}\,\mathrm{s}$ and see no
    additional feature at earlier times down to their time resolution
    $\tau_{\text{min,res}}\sim \mathrm{ns}$. Within SST, this implies that any
    swirl-induced EM precursor must satisfy
    \begin{equation}
        \tau_d^{\text{swirl}} \lesssim \tau_{\text{min,res}},
    \end{equation}
    which, via Eq.~\eqref{eq:gamma_swirl_def}, yields an upper bound
    \begin{equation}
        f_{\text{Unruh}} \lesssim 10^{-4}
    \end{equation}
    for the geometries and media used so far. Thus, current data constrain the
    efficiency of swirl-to-EM conversion in GR-designed cavities, but do not
    falsify the existence of a swirl sector. A decisive SST test requires
    hybrid platforms (e.g., BECs or superfluid cavities) with simultaneous
    sensitivity to density and EM modes \cite{Steinhauer2016,Gooding2020} and
    sub-nanosecond temporal resolution.

    %======================================================================
\subsection{Hydrodynamic Origin of the Hydrogen Ground State (Operational Summary)}
    \label{sec:SST_hydrogen_groundstate}
%======================================================================

    This section summarizes, in compressed form, the full derivation given in Ref.~[14], ``The Hydrodynamic Triad: Unifying Gravity, Electromagnetism, and Quantum Mass via a Circulation-Based Vacuum Canon.''

    In conventional quantum mechanics the discrete spectrum of the Hydrogen
    atom,

    \begin{equation}
        E_n = -\,\frac{13.6~{\rm eV}}{n^2},
        \qquad n=1,2,3,\ldots,
    \end{equation}

    is regarded as a purely spectral property of the Coulomb Hamiltonian.
    Within Swirl--String Theory (SST), the same spectrum is reinterpreted as
    a hierarchy of \emph{stationary incompressible flow regimes} sustained
    by the orbital swirl structure of the electron string around the
    protonic core.

%----------------------------------------------------------------------
    \subsubsection*{Orbital swirl velocity as the principal quantum number}
%----------------------------------------------------------------------

        The Bohr orbital velocity,

        \begin{equation}
            v_n = \frac{\alpha c}{n},
        \end{equation}

        is taken to represent the coarse--grained swirl speed of the electron
        string along a circular streamline at radius

        \begin{equation}
            r_n = \frac{n^2 a_0}{1},
        \end{equation}

        with $a_0$ the Bohr radius. As $n$ increases, the swirl becomes
        progressively weaker and more diffuse. In the limit $n\to\infty$,
        $v_n\to 0$ and the flow approaches the unbound (ionised) regime.

        The principal quantum number labels discrete \emph{laminar} flow
        patterns supported by the medium.

%----------------------------------------------------------------------
    \subsubsection*{The electron-scale constraint}
%----------------------------------------------------------------------

        Independently, the SST electron-scale derivation relates the core radius
        $r_c$, the swirl speed $\lVert \mathbf{v}_{\!\boldsymbol{\circlearrowleft}}
\rVert$, and the swirl energy density $\rho_{\!E}$ via

        \begin{equation}
            \rho_{\!E}
            = \frac{1}{2}\rho_{\!f}
            \lVert \mathbf{v}_{\!\boldsymbol{\circlearrowleft}} \rVert^{2}.
        \end{equation}

        Using the canonical swirl speed value established in the SST Canon,

        \begin{equation}
            \lVert \mathbf{v}_{\!\boldsymbol{\circlearrowleft}}\rVert
            \approx 1.0938\times 10^{6}~{\rm m/s}
            \approx \frac{1}{2}\alpha c,
        \end{equation}

        we identify the internal vorticity scale as exactly half the vacuum Mach
        limit. This factor of $1/2$ is characteristic of the dipole topology
        of the vortex loop.

%----------------------------------------------------------------------
    \subsubsection*{Ground-state stability from a hydrodynamic speed limit}
%----------------------------------------------------------------------

        Combining the orbital relation $v_n = \alpha c / n$ with the
        electron-scale constraint reveals a hydrodynamic interpretation of the
        Bohr ground state.

        At $n=1$, the orbital velocity reaches the vacuum limit:

        \begin{equation}
            v_1 = \alpha c = 2 \lVert \mathbf{v}_{\!\boldsymbol{\circlearrowleft}}\rVert.
        \end{equation}

        This velocity $v_1 = \alpha c$ represents the \emph{maximum laminar
translation speed} permitted by the vacuum flow texture (the transverse
        Mach limit). Thus the $n=1$ state sits at the boundary between
        admissible laminar flow and a regime in which the required flow speed
        would exceed the vacuum stability limit.

        For any hypothetical ``sub-Bohr'' value $n<1$, the orbital velocity
        would satisfy

        \begin{equation}
            v_n = \frac{\alpha c}{n} > \alpha c,
        \end{equation}

        forcing the flow into a non-laminar (turbulent or singular) regime
        analogous to a sonic boom. In SST such a configuration cannot sustain
        a stationary swirl string, and therefore \emph{no bound state exists
for $r < a_0$}.

        The ground state is not imposed by abstract quantisation but arises
        dynamically as the innermost stable laminar flow configuration permitted
        by the fluid properties of the vacuum.
        %======================================================================
\subsection{Hydrodynamic Derivation of the Rydberg Constant (Summary)}
    \label{sec:SST_rydberg}
%======================================================================

    In the SST framework, the ionization of the Hydrogen atom corresponds to
    the acceleration of the electron vortex string from its stable ground-state
    orbit ($n=1$) to the unbound vacuum flow regime ($n \to \infty$).

    The energy required for this transition—the Rydberg energy $E_{Ry}$—is
    identifiable not as an electrostatic potential difference, but as the
    \emph{kinetic energy} of the electron vortex traveling at the vacuum
    stability limit.

%----------------------------------------------------------------------
    \subsubsection*{Rydberg Energy as Vacuum Kinetic Limit}
%----------------------------------------------------------------------

        From the preceding section, the ground-state orbital velocity $v_1$ is
        defined by the transverse Mach limit of the vacuum:

        \begin{equation}
            v_1 = \alpha c.
            \label{eq:v1_mach}
        \end{equation}

        The classical kinetic energy $T_1$ of the electron mass $m_e$ moving
        at this limit is:

        \begin{equation}
            T_1 = \frac{1}{2} m_e v_1^2 = \frac{1}{2} m_e (\alpha c)^2.
        \end{equation}

        In standard theory, the Rydberg energy is defined as $hcR_\infty$.
        Equating the hydrodynamic kinetic energy to the spectral energy yields:

        \begin{equation}
            hcR_\infty = \frac{1}{2} m_e \alpha^2 c^2.
        \end{equation}

%----------------------------------------------------------------------
    \subsubsection*{SST Substitution: The Swirl Velocity Relation}
%----------------------------------------------------------------------

        We now substitute the SST canonical relation between the fine structure
        constant and the intrinsic swirl velocity, $\alpha c = 2 \mathbf{v}_{\!\boldsymbol{\circlearrowleft}}$:

        \begin{equation}
            hcR_\infty = \frac{1}{2} m_e (2 \mathbf{v}_{\!\boldsymbol{\circlearrowleft}})^2
            = 2 m_e \mathbf{v}_{\!\boldsymbol{\circlearrowleft}}^2.
        \end{equation}

        Solving for the Rydberg constant $R_\infty$:

        \begin{equation}
            R_\infty = \frac{2 m_e \mathbf{v}_{\!\boldsymbol{\circlearrowleft}}^2}{hc}.
            \label{eq:SST_Rydberg}
        \end{equation}

%----------------------------------------------------------------------
    \subsubsection*{Physical Interpretation}
%----------------------------------------------------------------------

        Equation \eqref{eq:SST_Rydberg} provides a purely kinematic definition
        of the Rydberg constant. It states that the fundamental wavenumber of
        atomic spectroscopy is determined by the ratio of the \emph{vortex
swirl energy} ($m_e \mathbf{v}_{\!\boldsymbol{\circlearrowleft}}^2$) to the \emph{action-speed product} ($hc$).

        Specifically, $R_\infty$ represents the spatial frequency of a wave
        associated with a vortex loop accelerating to twice its intrinsic spin
        velocity. The factor of 2 arises from the geometry of the loop: the
        coherent translation of a dipole structure requires twice the energy
        of a monopole flow of equivalent velocity.

        Thus, in SST, spectral lines are not transitions between abstract
        probability clouds, but are acoustic resonance shifts caused by the
        deceleration of the electron knot from its maximum laminar speed
        ($\alpha c$) to lower harmonic velocities ($v_n = \alpha c / n$).
%======================================================================
\subsection{Hydrodynamic Derivation of the Compton Wavelength}
    \label{sec:SST_compton}
%======================================================================

    The Compton wavelength $\lambda_c$ defines the fundamental length scale of
    quantum interaction for a particle of mass $m_e$. In SST, this emerges
    from the helical geometry of the vortex string trajectory.

    We interpret $\lambda_c$ as the \emph{longitudinal spatial period} (or pitch)
    of the vortex filament as it translates at the speed of light $c$,
    governed by the internal gearing ratio $\alpha$.

%----------------------------------------------------------------------
    \subsubsection*{The Geometric Pitch Relation}
%----------------------------------------------------------------------

        Standard electrodynamics establishes the relationship between the
        Classical Electron Radius ($r_e$), the Fine Structure Constant ($\alpha$),
        and the Compton Wavelength ($\lambda_c$):

        \begin{equation}
            r_e = \alpha \frac{\lambda_c}{2\pi}.
        \end{equation}

        In the SST Canon, the geometric Core Radius $r_c$ is exactly half the
        classical radius ($r_e = 2r_c$), reflecting the dipole (loop) topology
        of the knot. Substituting $r_e = 2r_c$:

        \begin{equation}
            2r_c = \alpha \frac{\lambda_c}{2\pi} \quad \implies \quad
            \lambda_c = \frac{4\pi r_c}{\alpha}.
            \label{eq:lambda_geo}
        \end{equation}

        This equation states that the Compton wavelength is the circumference of
        the vortex core ($2\pi r_c$) amplified by the inverse Mach number ($1/\alpha$)
        and a topological factor of 2.

%----------------------------------------------------------------------
    \subsubsection*{SST Substitution: The Helical Pitch Formula}
%----------------------------------------------------------------------

        We now substitute the SST canonical definition of $\alpha$ derived from
        the swirl velocity ($\alpha = 2 \mathbf{v}_{\!\boldsymbol{\circlearrowleft}} / c$) into Eq.~\eqref{eq:lambda_geo}:

        \begin{equation}
            \lambda_c = \frac{4\pi r_c}{(2 \mathbf{v}_{\!\boldsymbol{\circlearrowleft}} / c)}
            = \frac{2\pi r_c c}{\mathbf{v}_{\!\boldsymbol{\circlearrowleft}}}.
            \label{eq:SST_Compton}
        \end{equation}

%----------------------------------------------------------------------
    \subsubsection*{Numerical Verification}
%----------------------------------------------------------------------

        Using the canonical values:
        \begin{itemize}
            \item $r_c \approx 1.409 \times 10^{-15}$ m
            \item $c \approx 3.00 \times 10^8$ m/s
            \item $\mathbf{v}_{\!\boldsymbol{\circlearrowleft}} \approx 1.094 \times 10^6$ m/s
        \end{itemize}

        \begin{equation}
            \lambda_c \approx \frac{2\pi (1.409 \times 10^{-15})(2.998 \times 10^8)}{1.094 \times 10^6}
            \approx 2.426 \times 10^{-12} \, \text{m}.
        \end{equation}

        This matches the CODATA value for the Compton wavelength of the electron
        ($2.42631 \times 10^{-12}$ m).

%----------------------------------------------------------------------
    \subsubsection*{Physical Interpretation: The Vacuum Screw}
%----------------------------------------------------------------------

        Equation \eqref{eq:SST_Compton} reveals the mechanical nature of mass
        transport in the vacuum.
        The term $\frac{2\pi r_c}{\mathbf{v}_{\!\boldsymbol{\circlearrowleft}}}$ represents the \emph{period of one internal rotation}
        of the vortex core. Multiplying by $c$ gives the distance traveled
        during one rotation.

        Thus, the electron behaves as a \textbf{Self-Propelling Screw}:
        \begin{enumerate}
            \item It spins internally at speed $\mathbf{v}_{\!\boldsymbol{\circlearrowleft}}$.
            \item It moves forward at speed $c$.
            \item The "thread pitch" of this motion is exactly $\lambda_c$.
        \end{enumerate}

        Mass, in this view, is the resistance to changing this pitch. A shorter
        wavelength (higher mass) implies a "tighter" screw thread that requires
        more energy to accelerate.

%--------------------------------------------------------
        \paragraph*{Corollary 24.1 (Swirl-Blindness Condition).}
            \label{cor:swirl_blindness}
            \emph{
            In electromagnetic cavities where the impedance mismatch between the swirl
            medium and the physical boundaries is large,
            $Z_{\mathrm{bound}} \gg Z_S = \rhoF \lVert\vswirl\rVert$, the primary
            swirl superradiance burst ($t \sim 0.1\,\mathrm{ns}$) is almost completely
            non-radiatively dissipated in the walls. The observable electromagnetic signal
            is a secondary transduction echo with the following properties:
        }

            \begin{enumerate}
                \item \emph{Delay:} The peak of $n_{\mathrm{EM}}(t)$ is controlled by
                the slower of the swirl decay rate $\lambda^{-1}$ and the cavity
                ring-up time $\gamma_{\mathrm{cav}}^{-1}$, rather than by the
                intrinsic timescale of the primary Unruh event.
                \item \emph{Amplitude:} The EM intensity is suppressed by the small
                transduction efficiency $\xi$ in Eq.~\eqref{eq:xi_definition}.
                In the impedance-dominated regime (see App.~\ref{app:swirl_em_transduction_impedance}),
                one finds $\xi \propto 4 Z_S/Z_{\mathrm{bound}} \sim 10^{-7}$ (research-track estimate),
                implying that $\mathcal{O}(10^{-7})$ of the primary swirl energy appears in the
                EM channel.
            \end{enumerate}

            \emph{
            Experimental access to the primary burst therefore requires
            impedance-matched hydrodynamic detectors (e.g.\ superfluids or
            Bose--Einstein condensates), where $Z_{\mathrm{det}}$ can be tuned to
            approach $Z_S$. In such detectors SST predicts a prompt, high-contrast
            signal at the swirl timescale, in addition to the delayed EM echo.
        }

            The existence of swirl-blind cavities and $\kappa_{\text{se}} \approx 0$ (as stated in
            Eqs.~\eqref{eq:gamma_em_effective} and the swirl-blind limit) is canonical. The numerical
            estimate $T \sim 10^{-7}$ for typical metal/glass boundaries is a research-track
            calculation detailed in App.~\ref{app:swirl_em_transduction_impedance}.

            %======================================================
% XXV. Topological Origin of the Electron a_e (Module)
% Place this in Part IV, after XXIV Unruh / hydrogen summary,
% before the global References section.
%======================================================

\section{TOPOLOGICAL ORIGIN OF THE ELECTRON ANOMALOUS MAGNETIC MOMENT}
\subsection*{A. Topological Expansion Hypothesis}

    In the Swirl--String Theory (SST) framework, the electron is modeled not as a point charge but as a closed toroidal
    swirl string with core radius $r_c$ and circulation $\Gamma$, canonically identified with the trefoil knot $3_1$
    (Section~XV, Knot Taxonomy).
    We propose that the perturbative expansion of the anomalous magnetic moment $a_e$ in Quantum Electrodynamics (QED)
    is physically isomorphic to a topological expansion of the filament's vibrational eigenmodes.

    The anomaly is written as
    \begin{equation}
        a_e = \sum_{n=1}^{\infty} C_n \left( \frac{\alpha}{\pi} \right)^n ,
    \end{equation}
    where $\alpha$ is the fine-structure constant, identified in SST with the intrinsic swirl gearing between
    $\lVert\mathbf{v}_{\!\boldsymbol{\circlearrowleft}}\rVert$ and $c$ (see the Hydrogen/Compton module).

\subsection*{B. First Order: The Rigid Torus}

    The first-order term ($n = 1$) corresponds to the Schwinger limit and describes the magnetic moment of a rigid,
    unperturbed vortex ring. In SST this is the dipole response of a stationary trefoil loop with fixed poloidal and
    toroidal circulation. The coefficient arises from the ratio of poloidal twist to toroidal circulation:
    \begin{equation}
        C_1^{\text{SST}} = \frac{1}{2} \equiv C_1^{\text{QED}} .
    \end{equation}
    This recovers the standard Schwinger result for $g_e = 2(1 + a_e)$ at leading order.

\subsection*{C. Second Order: The Riemann--Trefoil Invariant}

    The second-order term ($n = 2$) arises from the self-interaction of the swirl core. In QED this is captured by the
    seven distinct fourth-order Feynman diagrams (vacuum polarization and self-energy). In SST, the same energy shift is
    interpreted as the fundamental elastic Kelvin-wave resonance of the filament.

    The lowest-energy resonance for a closed loop with nonzero self-helicity is the $3_1$ trefoil. Its geometric energy
    penalty, measured by the knot energy functional relative to the unknot, is proportional to the Riemann zeta value
    $\zeta(2)$. We encode this as an effective \emph{Riemann--Trefoil invariant} and identify the SST second-order
    coefficient as the negative conformal weight of the trefoil geometry:
    \begin{equation}
        C_2^{\text{SST}} = - \frac{\pi^2}{30} \approx -0.3289868 .
    \end{equation}

\subsection*{D. Comparison with the Standard Model}

    Numerically, the SST geometric invariant approximates the QED coefficient
    \begin{align}
        C_2^{\text{QED}} &\approx -0.3284789 \quad \text{(Standard Model)}, \\
        C_2^{\text{SST}} &\approx -0.3289868 \quad \text{(Riemann--Trefoil invariant)} ,
    \end{align}
    with a residual difference of
    \begin{equation}
        \Delta C_2 \equiv
        \frac{\bigl|C_2^{\text{SST}} - C_2^{\text{QED}}\bigr|}{\bigl|C_2^{\text{QED}}\bigr|}
        \sim 1.5 \times 10^{-3} \approx 0.15\% .
    \end{equation}
    Within SST this residual is interpreted as evidence that the electron core is not a mathematically perfect trefoil,
    but a finite-thickness fluid solenoid subject to small hydrodynamic slippage and higher-order Kelvin-wave mode
    mixing. Those corrections are, in principle, computable from the SST Hamiltonian density (Appendix~I)
    and should appear as higher-order terms in the $(\alpha/\pi)^n$ expansion.

    Research-track status. This module canonizes the \emph{identification} of $C_2$ with a trefoil spectral invariant,
    but treats the exact numerical matching and higher-order $n \ge 3$ terms as research-track, to be constrained against
    full QED $g-2$ data.

% (Optional) You can later append a small local reference note here,
% or rely on the global References section.