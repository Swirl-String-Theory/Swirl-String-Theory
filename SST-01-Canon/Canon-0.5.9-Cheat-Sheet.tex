%! Author = Omar Iskandarani
%! Title = Swirl String Theory (SST) Canon v0.5
%! Date = Sept 4, 2025
%! Affiliation = Independent Researcher, Groningen, The Netherlands
%! License = © 2025 Omar Iskandarani. All rights reserved. This manuscript is made available for academic reading and citation only. No republication, redistribution, or derivative works are permitted without explicit written permission from the author. Contact: info@omariskandarani.com
%! ORCID = 0009-0006-1686-3961
%! DOI = 10.5281/zenodo.17155748

\newcommand{\canonversion}{\textbf{v0.5.9}} % Semantic versioning
\newcommand{\papertitle}{Swirl String Theory (SST) Canon \canonversion}
\newcommand{\paperdoi}{10.5281/zenodo.17155748}

%========================================================================================
% PACKAGES AND DOCUMENT CONFIGURATION
%========================================================================================
\documentclass[10pt,reprint,aps,onecolumn,nofootinbib]{revtex4-2}

\usepackage{amsmath,amssymb,amsfonts}
\usepackage{bm}
\usepackage{physics}
\usepackage{microtype}
\usepackage{tcolorbox}
\usepackage{hyperref}
\hypersetup{colorlinks=true,linkcolor=blue,citecolor=blue,urlcolor=blue}
\usepackage[T1]{fontenc}
\usepackage{lmodern}
\usepackage{booktabs}
\usepackage[utf8]{inputenc}
\usepackage{graphicx}
\usepackage{siunitx}
\sisetup{per-mode=symbol,detect-all=true}

% ===== Gauge sector macros =====
\newcommand{\Tr}{\mathrm{Tr}}
\newcommand{\ii}{\mathrm{i}}
\newcommand{\GsA}{G^a_{\mu\nu}}
\newcommand{\WsI}{W^i_{\mu\nu}}
\newcommand{\Bmn}{B_{\mu\nu}}

% ===============================
% Macros (canonicalized)
% ===============================

% swirl arrows (context-aware)
\newcommand{\swirlarrow}{%
    \mathchoice{\mkern-2mu\scriptstyle\boldsymbol{\circlearrowleft}}%
    {\mkern-2mu\scriptstyle\boldsymbol{\circlearrowleft}}%
    {\mkern-2mu\scriptscriptstyle\boldsymbol{\circlearrowleft}}%
    {\mkern-2mu\scriptscriptstyle\boldsymbol{\circlearrowleft}}%
}
\newcommand{\swirlarrowcw}{%
    \mathchoice{\mkern-2mu\scriptstyle\boldsymbol{\circlearrowright}}%
    {\mkern-2mu\scriptstyle\boldsymbol{\circlearrowright}}%
    {\mkern-2mu\scriptscriptstyle\boldsymbol{\circlearrowright}}%
    {\mkern-2mu\scriptscriptstyle\boldsymbol{\circlearrowright}}%
}

% Canonical symbols
\newcommand{\vswirl}{\mathbf{v}_{\swirlarrow}}
\newcommand{\vswirlcw}{\mathbf{v}_{\swirlarrowcw}}
\newcommand{\SwirlClock}{S_{(t)}^{\swirlarrow}}
\newcommand{\SwirlClockcw}{S_{(t)}^{\swirlarrowcw}}
\newcommand{\omegas}{\boldsymbol{\omega}_{\swirlarrow}}
\newcommand{\vscore}{v_{\swirlarrow}}
\newcommand{\vnorm}{\lVert \mathbf{v}_{\mkern-2mu\scriptscriptstyle\boldsymbol{\circlearrowleft}} \rVert}
\newcommand{\rhof}{\rho_{\!f}}
\newcommand{\rhoE}{\rho_{\!E}}
\newcommand{\rhom}{\rho_{\!m}}
\newcommand{\rc}{r_c}
\newcommand{\FmaxEM}{F_{\mathrm{EM}}^{\max}}
\newcommand{\FmaxG}{F_{\mathrm{G}}^{\max}}
\newcommand{\Lam}{\Lambda}
\newcommand{\Om}{\Omega_{\swirlarrow}}
\newcommand{\alpg}{\alpha_g}
\newcommand{\omegaVec}{\boldsymbol{\omega}}
\newcommand{\rhoF}{\rho_{\!f}}
\newcommand{\rhoM}{\rho_{\!m}}
\newcommand{\OmegaCore}{\Omega_{\mathrm{core}}}
\newcommand{\bg}{\mathrm{bg}}
\newcommand{\core}{\mathrm{core}}
\newcommand{\Vol}{\operatorname{Vol}}

% Golden policy (hyperbolic declaration)
\newcommand{\xig}{\operatorname{asinh}\!\left(\tfrac{1}{2}\right)}
\newcommand{\phig}{\exp(\xig)}
\newcommand{\phialg}{\bigl(1+\sqrt{5}\bigr)/2}
\newcommand{\xigold}{\tfrac{3}{2}\,\xig}
\newcommand{\GoldenDeclare}{%
    \textbf{Golden (hyperbolic)}:\ \(\ln\phi=\xig\), hence \(\phi=\phig\).
    \ \emph{(Algebraic form \(\phi=\phialg\) is equivalent.)}%
}

% ===== Canonical constants (SI; numeric macros for in-text checks) =====
\newcommand{\vswirlval}{1.09384563\times 10^{6}\ \si{m.s^{-1}}}
\newcommand{\rcval}{1.40897017\times 10^{-15}\ \si{m}}
\newcommand{\rhofval}{7.0\times 10^{-7}\ \si{kg.m^{-3}}}
\newcommand{\rhocoreval}{3.8934358266918687\times 10^{18}\ \si{kg.m^{-3}}}
\newcommand{\cval}{2.99792458\times 10^{8}\ \si{m.s^{-1}}}
\newcommand{\alphafsval}{7.2973525693\times 10^{-3}}


% ===== Reusable boxes for analogies / algorithms =====
\newtcolorbox{analogynote}[1][]{colback=white,colframe=black,
    title={Analogy (fluid picture)}, fonttitle=\bfseries, #1}
\newtcolorbox{algobox}[1][]{colback=white,colframe=black,
    title={Algorithmic recipe}, fonttitle=\bfseries, #1}
\newtcolorbox{edgebox}[1][]{colback=white,colframe=black,
    title={Edge cases \& uncertainties}, fonttitle=\bfseries, #1}


% ===============================
% DOCUMENT
% ===============================

\begin{document}

\title{Cosmological Foundations in Swirl String Theory}
\author{Omar Iskandarani}
\affiliation{Independent Researcher, Groningen, The Netherlands}
\thanks{ORCID: 0009-0006-1686-3961, DOI: \paperdoi}
\date{\today}

\begin{abstract}
This Canon Cheat-Sheet condenses \emph{Swirl String Theory (SST)} for cosmology: definitions, constants, boxed master equations, and notational conventions. It emphasizes dimensional consistency, known-limit checks, and minimal assumptions.
\end{abstract}

\maketitle

% =======================
% Foundations
% =======================
\section*{Foundations}
\begin{itemize}
  \item \textbf{Arena:} Flat \( \mathbb{R}^3 \) with absolute (Chronos) time.
  \item \textbf{Medium:} Homogeneous, incompressible swirl condensate of density \( \rhof \); circulation quantized in closed filaments (``swirl strings'').
  \item \textbf{Gravity:} Emergent from swirl-pressure and clock-rate gradients; no curved spacetime.
\end{itemize}

% =======================
% Cosmogony
% =======================
\section*{Swirl Cosmogony (Genesis via Knots)}
\begin{itemize}
  \item \textbf{Primordial:} Uniform, laminar state (topologically trivial).
  \item \textbf{Instability:} Fluctuations/reconnections nucleate closed loops (unknots).
  \item \textbf{Knot genesis:} Reconnection cascades stabilize nontrivial knots; topology protects excitation.
  \item \textbf{Freeze-in:} Energy is inherited via line-length and local topology.
  \item \textbf{Causal asymmetry:} Arrow of time measured by monotone growth of knot complexity and coherent volume fraction.
  \item \textbf{Inflation-like era:} Burst of coherence and reconnection leads to exponential growth of coherent domains.
  \item \textbf{Post-era:} Knots seed matter; coherence zones act as gravitational attractors.
\end{itemize}

    % =======================
% Cosmogony — governing dynamics and observables
% =======================
\section*{Cosmogony: Governing Dynamics, Freeze-out, and Observables}

    \subsection*{Primary scales at Big Condensation}
        Define the quantum of circulation and an initial correlation length:
        \[
            \kappa \;\equiv\; 2\pi\,\rc\,\lVert \mathbf{v}_{\!\boldsymbol{\circlearrowleft}}\rVert,
            \qquad
            \xi_0 \;\sim\; \rc.
        \]
        \paragraph*{Units and numeric check.}
            \([ \kappa ] = \si{m^2.s^{-1}}\).
            With \(\rc=\rcval\) and \(\lVert \mathbf{v}_{\!\boldsymbol{\circlearrowleft}}\rVert=\vswirlval\),
            \[
                \kappa \approx 9.684\times 10^{-9}\ \si{m^2.s^{-1}}.
            \]
            This ensures continuity with the core-scale swirl speed: \( \kappa/(2\pi \rc)=\lVert \mathbf{v}_{\!\boldsymbol{\circlearrowleft}}\rVert\).

            \begin{analogynote}
            \textbf{Kid picture:} \(\kappa\) is how much ``spin'' one tiny loop carries—like a fixed twist baked into every small rubber band.
            \end{analogynote}

% -----------------------
\subsection*{Freeze-out of coherence (Kibble--Zurek–type scaling)}
    When the condensate forms under a finite quench time \(\tau_Q\), domains freeze out at a scale
    \begin{equation}
    \xi_{\mathrm{fr}} \;\simeq\; \xi_0\,
    \Big(\frac{\tau_Q}{\tau_0}\Big)^{\nu/(1+\nu z)},
    \label{eq:kz}
    \end{equation}
    with static exponent \(\nu\) and dynamic exponent \(z\) appropriate to the SST universality class (to be measured). % \cite{Kibble1976,Zurek1985,Zurek1996}
    \paragraph*{Dimensions.} \(\xi_0\) has units of length; the ratio \((\tau_Q/\tau_0)^{\nu/(1+\nu z)}\) is dimensionless, so \([\xi_{\mathrm{fr}}]=\si{m}\).

        \begin{edgebox}
        \textbf{Uncertainties.} The exponents \((\nu,z)\) are \emph{not} assumed from external systems; SST must calibrate them from coherence-growth data (BAO + CMB phase + low-\(z\) structure) to avoid importing priors.
        \end{edgebox}

        \begin{analogynote}
        \textbf{Kid picture:} If you cool soup too fast, many small fat-islands form; cool it slower, and you get fewer, bigger islands. \(\xi_{\mathrm{fr}}\) is the island size at the instant the pattern ``freezes.''
        \end{analogynote}

% -----------------------
\subsection*{Coherence fraction dynamics (logistic locking vs. scrambling)}
Let \(f(t)\in[0,1]\) be the fraction of volume in phase-locked (coherent) swirl.
We model competition between locking and scrambling by
\begin{equation}
\frac{df}{dt} \;=\; \big(\Gamma_{\rm lock}-\Gamma_{\rm scr}\big)\, f\,\big(1-f\big),
\label{eq:logistic}
\end{equation}
with rates \(\Gamma_{\rm lock},\Gamma_{\rm scr}\) in \(\si{s^{-1}}\).
A minimal parametric form consistent with SST kinematics is
\[
    \Gamma_{\rm lock} \;=\; \chi\,\frac{\kappa}{2\pi\,\xi^2},
    \qquad
    \Gamma_{\rm scr} \;=\; \eta\,\Gamma_{\rm rec},
\]
where \(\xi(t)\) is the instantaneous correlation length, \(\chi,\eta\) are dimensionless efficiencies, and \(\Gamma_{\rm rec}\) is the reconnection rate density (research-calibrated).
\paragraph*{Solution.} For piecewise-constant \(\Gamma_{\rm eff}\!\equiv\!\Gamma_{\rm lock}-\Gamma_{\rm scr}\),
    \[
        f(t)=\frac{1}{1+\Big(\frac{1-f_0}{f_0}\Big)e^{-\Gamma_{\rm eff}(t-t_0)}}.
    \]
    A transient epoch with \(\Gamma_{\rm eff}>0\) is the SST analogue of ``inflation-like'' coherence burst (rapid ordering without metric expansion).

    \begin{algobox}
    \textbf{Recipe to fit \(f(t)\):}
    (1) Choose \(\xi(t)\) model (next subsection).
    (2) Set priors on \(\chi,\eta\) and \(\Gamma_{\rm rec}\) from simulations.
    (3) Fit \(\Gamma_{\rm eff}(t)\) to SN\,Ia \(H_{\rm eff}(z)\) + BAO AP anisotropy.
    (4) Cross-check with CMB acoustic phase (stage-locking imprint).
    \end{algobox}

% -----------------------
\subsection*{Correlation-length growth (reconnection-limited coarsening)}
Coarse-grained swirl speed at scale \(\xi\) follows Biot–Savart scaling:
\[
    v_{\rm coarse}(\xi)\;\simeq\;\frac{\kappa}{2\pi\,\xi}.
\]
A reconnection-limited coarsening law that respects dimensions is
\begin{equation}
\frac{d\xi}{dt} \;=\; A\,\frac{\kappa}{2\pi\,\xi}\;-\;B\,\Gamma_{\rm rec}\,\xi,
\label{eq:xi-growth}
\end{equation}
with \(A,B\) dimensionless. The first term grows domains via advective coalescence; the second shrinks them when reconnections dominate.
\paragraph*{Limits.}
(i) \(\Gamma_{\rm rec}\!\to\!0\): \(\xi^2(t)\) grows linearly: \(\xi^2(t)=\xi_i^2+\frac{A\kappa}{\pi}(t-t_i)\).
    (ii) Strong reconnections: steady state \(\xi_\star=\sqrt{\frac{A\kappa}{2\pi B\,\Gamma_{\rm rec}}}\).

    \begin{edgebox}
    \textbf{Calibration targets.}
    CMB peak spacing \(\Rightarrow \xi_{\rm fr}\); BAO scale \(\Rightarrow \xi\) at \(z\!\sim\!0.5\!-\!1\);
    Weak-lensing two-point \(\Rightarrow\) late-time \(\xi\) anisotropy;
    Peculiar-velocity flows \(\Rightarrow v_{\rm coarse}(\xi)\) normalization.
    \end{edgebox}

    \begin{analogynote}
    \textbf{Kid picture:} Small whirlpools merge into bigger ones unless they keep cutting each other. The first term makes bigger pools; the second keeps chopping them up.
    \end{analogynote}

% -----------------------
\subsection*{Swirl-clock background and effective Hubble rate}
SST uses the clock ratio as the distance–redshift engine:
\[
    1+z \;=\; \frac{S_t^{-1}({\rm emit})}{S_t^{-1}({\rm obs})},
    \qquad
    H_{\rm eff}(t)\equiv -\frac{d}{dt}\ln S_t.
\]
During a coherence burst (\(\Gamma_{\rm eff}>0\)) Eq.~\eqref{eq:logistic} pushes \(S_t\) toward uniformity, which drives a phase of \(\dot H_{\rm eff}\!<\!0\) mimicking accelerated expansion without metric growth (already referenced in your \(\Lambda\)CDM dictionary).

% -----------------------
\subsection*{Topological spectrum lock-in at freeze-out}
At \(t_{\rm fr}\) with \(\xi_{\rm fr}\) from Eq.~\eqref{eq:kz}, the knot density spectrum freezes.
The mass law (Eq.~\eqref{eq:mass-law}) then fixes species energy densities once \(L_{\rm tot}(K)\) and \((b,g,n)\) are set.
Hopf-charge stabilization provides a topological lower bound on energy for linked sectors (research calibration of the bound’s SST coefficient). % \cite{FaddeevNiemi1997}

% -----------------------
\subsection*{Falsifiable cosmogony signals}
\begin{tcolorbox}[colback=white,colframe=black,title=Predictions specific to cosmogony]
\begin{itemize}\itemsep2pt
\item \textbf{KZ scaling in LSS:} The inferred \(\xi_{\rm fr}\) from CMB should obey a power law in an independently estimated \(\tau_Q\) proxy (duration of condensation epoch).
\item \textbf{Phase shift of acoustic peaks:} Ordering dynamics impart a calculable phase offset in the CMB acoustic series distinct from standard \(\Lambda\)CDM (sign fixed by \(\Gamma_{\rm eff}(t)\)).
\item \textbf{BAO AP anisotropy vs. environment:} \(\xi\) and \(S_t\) gradients predict \(\mathcal{O}(10^{-3}\!-\!10^{-2})\) directional distortions correlated with large-scale shear.
\item \textbf{Redshift drift:} The combination \(\dot z = H_{{\rm eff},0}-H_{\rm eff}(z)/(1+z)\) deviates at \(z\!\lesssim\!1\) if \(f(t)\) is still evolving.
\end{itemize}
\end{tcolorbox}

% =======================
% Swirl clock
% =======================
\section*{Swirl Clock, Time Dilation, and Redshift}
Define the swirl-clock factor
\[
S_t \equiv \sqrt{1-\frac{\vnorm^2}{c^2}}\!,
\qquad
dt_{\mathrm{local}} = S_t\, dt_{\infty}.
\]
Cosmological redshift is interpreted as a clock-ratio:
\[
1+z \;=\; \frac{S_t^{-1}(\mathrm{emit})}{S_t^{-1}(\mathrm{obs})}
\quad\text{(line-of-sight shear gives subleading corrections).}
\]
    \begin{analogynote}
    A clock is a leaf on water. Where the water swirls fast, the leaf wobbles and ticks slower. Light leaving the slow-water zone looks slightly ``stretched'' (redder).
    \end{analogynote}

    \paragraph*{Known-limit + numeric check.}
        With \(v=\lVert \mathbf{v}_{\!\boldsymbol{\circlearrowleft}}\rVert=\vswirlval\) and \(c=\cval\),
        \[
            \frac{v}{c}\approx 3.65\times10^{-3},\qquad
            S_t=\sqrt{1-v^2/c^2}\approx 0.9999933,
        \]
        so local clock-slowdown at the characteristic swirl speed is small (consistent with weak-field behavior).  \cite{Einstein1905}


% =======================
% Emergent gravity
% =======================
\section*{Emergent Gravity from Swirl Pressure}
For axisymmetric swirl with azimuthal speed \(v_\theta(r)\), steady Euler balance gives
\[
\frac{1}{\rhof}\,\frac{dp_{\text{swirl}}}{dr} \;=\; \frac{v_\theta^2}{r},
\]
so an effective inward acceleration \(g_{\text{eff}}(r)=v_\theta^2/r\), approximating \(1/r^2\) attraction when \(v_\theta\propto r^{-1/2}\). % \cite{Batchelor1967,Saffman1992}

\begin{analogynote}
\textbf{Fluid picture:} Swirl makes a pressure ``dip.'' Marbles (test masses) roll toward the dip; the radial balance \( \rhof^{-1} dp_{\rm swirl}/dr = v_\theta^2/r \) is just the slope the marble feels. If \(v_\theta\!\propto\! r^{-1/2}\), the inward pull behaves like \(1/r^2\).
\end{analogynote}


% =======================
% Vacuum/core energy scale + numeric
% =======================
\section*{Vacuum (Core) Energy Density Scale}
Assuming the core carries the characteristic swirl speed \( \vnorm\approx \vscore \),
\[
u \;=\; \tfrac{1}{2}\,\rho_{\text{core}}\,\vnorm^2.
\]
\textbf{Numerical check (SI):}
\[
\rho_{\text{core}}=\rhocoreval,\quad \vnorm=\vswirlval
\;\Rightarrow\;
u \approx 2.329\times 10^{30}\ \si{J.m^{-3}}.
\]

% =======================
% Mass law (dimensionally correct) + numeric scale
% =======================
\section*{Invariant Mass Law for Knotted Excitations (Canonical)}
Let \(L_{\text{tot}}(K)\) be a \emph{dimensionless} ropelength of knot \(K\). The dimensionally correct SST mass law used in particle fits is
\begin{equation}
\boxed{\
M(K)\;=\;\Big(\frac{4}{\alpha_{\mathrm{fs}}}\Big)\, b(K)^{-3/2}\,\phi^{-g(K)}\,n(K)^{-1/\phi}\;
\frac{u\,\big(\pi\,\rc^{3}\,L_{\text{tot}}(K)\big)}{c^{2}}\
}
\label{eq:mass-law}
\end{equation}
with \(b\) (braid index proxy), \(g\) (genus proxy), \(n\) (component count), and \(\phi=\phig\) per the Golden policy.
\paragraph*{Units check.} \(u[\si{J.m^{-3}}]\cdot(\pi \rc^3 L_{\text{tot}})[\si{m^3}]/c^2\to \si{kg}\).
\paragraph*{Mass scale per unit \(L_{\text{tot}}\) (numerical).}
\[
\frac{u\,\pi\rc^3}{c^2}
=
\frac{(2.329\times 10^{30})\,[\si{J.m^{-3}}]\cdot \pi(1.40897\times 10^{-15}\ \si{m})^{3}}{(2.9979\times 10^{8}\ \si{m.s^{-1}})^{2}}
\approx 2.28\times 10^{-31}\ \si{kg}.
\]
Including \(4/\alpha_{\mathrm{fs}}\approx 5.48\times 10^{2}\) sets the observed lepton/baryon scale once \(L_{\text{tot}}(e)\) is calibrated.
    \begin{analogynote}
    Topological knots are like rubber bands tied in different ways; tighter or more tangled bands store more ``swirl energy,'' which we weigh as mass.
    \end{analogynote}


% =======================
% Particle classes
% =======================
\section*{Knot Topologies for Standard Particles}
\begin{table}[h]
\centering
\begin{tabular}{@{}llccc@{}}
\toprule
Designation & Representative knot & \(b\) & \(g\) & \(n\) \\
\midrule
Electron \(e^-\) & Trefoil (\(3_1\), torus) & 3 & 1 & 1 \\
Muon \(\mu^-\) & Cinquefoil (\(5_1\), torus) & 5 & 2 & 1 \\
Proton \(p\) & 3-component chiral compound & 3 & 2 & 3 \\
Neutron \(n\) & as proton, different core strengths & 3 & 2 & 3 \\
Photon \(\gamma\) & Unknot (closed loop) & 1 & 0 & 1 \\
\bottomrule
\end{tabular}
\caption{SST classification parameters \((b,g,n)\) used in Eq.~\eqref{eq:mass-law}.}
\label{tab:sst-classes}
\end{table}

\paragraph*{Proton–neutron split (internal geometry).}
Let \(s_u\approx 2.828\), \(s_d\approx 3.164\) denote geometric swirl volumes (e.g., from hyperbolic data of candidate subknots \(5_2,6_1\)). With global scale \(2\pi^2\kappa_R\) (e.g., \(\kappa_R\!\approx\!2\)):
\begin{align*}
L_{\text{tot}}^{(p)} &= \lambda_b\,(2s_u+s_d)\,(2\pi^2\kappa_R),\\
L_{\text{tot}}^{(n)} &= \lambda_b\,(s_u+2s_d)\,(2\pi^2\kappa_R),
\end{align*}
preserving \((b,g,n)\) while shifting masses via internal geometry.

% =======================
% Rosetta: Lambda-CDM dictionary
% =======================
\section*{SST \(\leftrightarrow\) \(\Lambda\)CDM: Minimal Dictionary}
\begin{itemize}
  \item \textbf{Effective Hubble rate:}
  \(
  1+z = S_t^{-1}(\mathrm{em})/S_t^{-1}(\mathrm{obs})
  \Rightarrow
  H_{\text{eff}}(t) \equiv \frac{d}{dt}\ln(1+z) = -\frac{d}{dt}\ln S_t.
  \)
  \item \textbf{Distances:} Use \(H_{\text{eff}}(z)\) in FRW distance integrals,
  \(D_L(z)=(1+z)\int_0^z \frac{c\,dz'}{H_{\text{eff}}(z')}\),
  with small corrections if \(S_t\) varies along the line of sight.
  \item \textbf{BAO/CMB:} Coherence correlation length plays the role of a standard ruler; freeze-out of swirl modes maps to acoustic peaks.
  \item \textbf{Growth:} Growth rate \(f\sigma_8\) encodes build-up of coherent domains under reconnection and shear of \(\vswirl\).
\end{itemize}

% =======================
% Falsifiability
% =======================
\section*{Observational Consequences and Falsifiers}
\begin{tcolorbox}[colback=white,colframe=black,title=Falsifiable predictions]
\begin{itemize}
  \item \textbf{SN\,Ia host dependence:} After standardization, Hubble residuals correlate with local density (voids vs. clusters) via \(\Delta S_t\).
  \item \textbf{Strong-lens time delays:} Inferred \(H_0\) shifts with environmental \(S_t\); joint modeling predicts a sign/magnitude.
  \item \textbf{Redshift drift (Sandage test):} \(\dot z=H_{\text{eff},0}-H_{\text{eff}}(z)/(1+z)\). SST curves differ if \(S_t\) evolves non-FRW-like.
  \item \textbf{BAO AP anisotropy:} Directional \(S_t\) gradients generate Alcock–Paczyński distortions at \(10^{-3}\!-\!10^{-2}\).
  \item \textbf{GW speed:} \(c_{\mathrm{GW}}=c\) (baseline \(c_{13}=0\)); persistent \(c_{\mathrm{GW}}\neq c\) falsifies this sector. % \cite{GW170817}
\end{itemize}
\end{tcolorbox}

% =======================
% Canonical constants table
% =======================
\section*{Canonical Constants (SI)}
\begin{table}[h]
\centering
\begin{tabular}{@{}lll@{}}
\toprule
Quantity & Symbol & Value \\
\midrule
Swirl core radius & \(\rc\) & \rcval \\
Effective density & \(\rhof\) & \rhofval \\
Core density & \(\rho_{\text{core}}\) & \rhocoreval \\
Swirl speed (char.) & \(\vnorm\) & \vswirlval \\
Speed of light & \(c\) & \cval \\
Fine structure const. & \(\alpha_{\mathrm{fs}}\) & \alphafsval \\
\bottomrule
\end{tabular}
\end{table}

% =======================
% Implementation notes
% =======================
\section*{Implementation Notes (Data Fits)}
\begin{enumerate}
  \item Calibrate \(L_{\text{tot}}(e)\) from \(M_e\) using Eq.~\eqref{eq:mass-law}.
  \item Fix \(\lambda_b,\kappa_R\) on \((e,\mu,p)\); predict remaining leptons/hadrons and isotope splittings.
  \item Infer \(H_{\text{eff}}(z)\) non-parametrically from SN\,Ia; compare with BAO ruler from coherence correlation length.
  \item Cross-validate with time-delay lenses and CMB acoustic scale to bound line-of-sight variations in \(S_t\).
\end{enumerate}

% =======================
% Bibliography notes (non-original ideas; copy into .bib)
% =======================
% @book{Batchelor1967,
%   author = {Batchelor, G. K.},
%   title = {An Introduction to Fluid Dynamics},
%   publisher = {Cambridge Univ. Press},
%   year = {1967}
% }
% @book{Saffman1992,
%   author = {Saffman, Philip G.},
%   title = {Vortex Dynamics},
%   publisher = {Cambridge Univ. Press},
%   year = {1992}
% }
% @article{Einstein1905,
%   author = {Einstein, A.},
%   title = {Zur Elektrodynamik bewegter K{\"o}rper},
%   journal = {Annalen der Physik},
%   year = {1905},
%   volume = {17},
%   pages = {891--921},
%   doi = {10.1002/andp.19053221004}
% }
% @article{GW170817,
%   author = {Abbott, B. P. and others (LIGO/Virgo)},
%   title = {GW170817: Observation of Gravitational Waves from a Binary Neutron Star Inspiral},
%   journal = {Phys. Rev. Lett.},
%   year = {2017},
%   volume = {119},
%   pages = {161101},
%   doi = {10.1103/PhysRevLett.119.161101}
% }

\end{document}