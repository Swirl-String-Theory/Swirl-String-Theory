%! Author = Omar Iskandarani
%! Title = Swirl String Theory (SST) Canon v0.5.4
%! Date = Sept 9, 2025
%! Affiliation = Independent Researcher, Groningen, The Netherlands
%! License = © 2025 Omar Iskandarani. All rights reserved. This manuscript is made available for academic reading and citation only. No republication, redistribution, or derivative works are permitted without explicit written permission from the author. Contact: info@omariskandarani.com
%! ORCID = 0009-0006-1686-3961
%! DOI = 10.5281/zenodo.17052966 % Placeholder DOI, update upon finalization

\newcommand{\canonversion}{\textbf{v0.5.4}} % Semantic versioning: vMAJOR.MINOR.PATCH
\newcommand{\papertitle}{Swirl String Theory (SST) Canon \canonversion}
\newcommand{\paperdoi}{10.5281/zenodo.17052966}



%========================================================================================
% PACKAGES AND DOCUMENT CONFIGURATION
%========================================================================================
\documentclass[11pt]{article}
\usepackage{subfiles}
\input{../template/SSTstyle.sty} % Assuming a local style file
\input{../template/SST_appendix_setup.sty} % Assuming a local appendix setup
\usepackage[margin=1in]{geometry}
\usepackage{amsmath,amssymb,amsfonts,amsthm}
\usepackage{tcolorbox}
\usetikzlibrary{knots,intersections,decorations.pathreplacing,3d,calc,arrows.meta,positioning,decorations.pathmorphing}
\usepackage{pgfmath}
\usepackage{pgfplots}
\pgfplotsset{compat=1.18}
\usepackage{ulem}


% ==== Packages ====
\usepackage[T1]{fontenc}
\usepackage{lmodern}
\usepackage{microtype}

\geometry{margin=1in}
\usepackage{ bm, mathtools}
\usepackage{siunitx}
\sisetup{per-mode=symbol,round-mode=figures,round-precision=6}
\usepackage{physics}
\usepackage{upgreek}
\usepackage{graphicx}
\usepackage{booktabs}
\usepackage{hyperref}
\hypersetup{colorlinks=true, linkcolor=blue!60!black, citecolor=blue!60!black, urlcolor=blue!60!black}


% ===== Gauge sector macros =====
\newcommand{\Tr}{\mathrm{Tr}}
\newcommand{\ii}{\mathrm{i}}
% Gauge fields (adjoints; indices a=1..8, i=1..3)
\newcommand{\GsA}{G^a_{\mu\nu}}
\newcommand{\WsI}{W^i_{\mu\nu}}
\newcommand{\Bmn}{B_{\mu\nu}}



% ===============================
% Macros (canonicalized)
% ===============================

% swirl arrows (context-aware)
\newcommand{\swirlarrow}{%
    \mathchoice{\mkern-2mu\scriptstyle\boldsymbol{\circlearrowleft}}%
    {\mkern-2mu\scriptstyle\boldsymbol{\circlearrowleft}}%
    {\mkern-2mu\scriptscriptstyle\boldsymbol{\circlearrowleft}}%
    {\mkern-2mu\scriptscriptstyle\boldsymbol{\circlearrowleft}}%
}
\newcommand{\swirlarrowcw}{%
    \mathchoice{\mkern-2mu\scriptstyle\boldsymbol{\circlearrowright}}%
    {\mkern-2mu\scriptstyle\boldsymbol{\circlearrowright}}%
    {\mkern-2mu\scriptscriptstyle\boldsymbol{\circlearrowright}}%
    {\mkern-2mu\scriptscriptstyle\boldsymbol{\circlearrowright}}%
}


% Canonical symbols
\newcommand{\vswirl}{\mathbf{v}_{\swirlarrow}}
\newcommand{\vswirlcw}{\mathbf{v}_{\swirlarrowcw}}
\newcommand{\SwirlClock}{S_{(t)}^{\swirlarrow}}
\newcommand{\SwirlClockcw}{S_{(t)}^{\swirlarrowcw}}
\newcommand{\omegas}{\boldsymbol{\omega}_{\swirlarrow}}  % swirl vorticity
\newcommand{\vscore}{v_{\swirlarrow}}                    % shorthand: |v_swirl| at r=r_c
\newcommand{\vnorm}{\lVert \vswirl \rVert}               % swirl speed magnitude
\newcommand{\rhof}{\rho_{\!f}}                           % effective fluid density
\newcommand{\rhoE}{\rho_{\!E}}                           % swirl energy density
\newcommand{\rhom}{\rho_{\!m}}                           % mass-equivalent density
\newcommand{\rc}{r_c}                                    % string core radius (swirl string radius)
\newcommand{\FmaxEM}{F_{\mathrm{EM}}^{\max}}             % EM-like maximal force scale
\newcommand{\FmaxG}{F_{\mathrm{G}}^{\max}}               % G-like maximal force scale
\newcommand{\Lam}{\Lambda}                               % Swirl Coulomb constant
\newcommand{\Om}{\Omega_{\swirlarrow}}                   % swirl angular frequency profile
\newcommand{\alpg}{\alpha_g}                             % gravitational fine-structure analogue

% Policy: the golden constant is only allowed via hyperbolic functions.
\newcommand{\xig}{\operatorname{asinh}\!\left(\tfrac{1}{2}\right)}
\newcommand{\phig}{\exp(\xig)}
\newcommand{\phialg}{\bigl(1+\sqrt{5}\bigr)/2}
\newcommand{\xigold}{\tfrac{3}{2}\,\xig}
\newcommand{\GoldenDeclare}{%
    \textbf{Golden (hyperbolic)}:\ \(\ln\phi=\xig\), hence \(\phi=\phig\).
    \ \emph{(Equivalently, \(\phi=\phialg\); the algebraic form is derivative.)}%
}

% Theorem-like environments
\newtheorem{identity}{Identity}
\newtheorem{axiom}{Axiom}
\newtheorem{theorem}{Theorem}[section]
\newtheorem{lemma}[theorem]{Lemma}
\newtheorem{corollary}[theorem]{Corollary}
\newtheorem{definition}{Definition}[section]



%========================================================================================
% DOCUMENT START
%========================================================================================
\begin{document}

%========================================================================================
% TITLE PAGE
%========================================================================================

\titlepageOpen
\begin{abstract}
This Canon is the single source of truth for \emph{Swirl String Theory (SST)}: definitions, constants, boxed master equations, and notational conventions. It unifies the core hydrodynamic, electromagnetic, and gauge principles of the theory. This version canonizes the following principles:
\begin{itemize}
\item[\textbf{I}] The foundational hydrodynamic laws, including the Chronos-Kelvin Invariant and Swirl Coulomb constant $\Lam$.
\item[\textbf{II}] The Swirl-Electromagnetic Bridge, linking swirl dynamics directly to Maxwell's equations.
\item[\textbf{III}] The emergence of the $\mathrm{SU}(3)\times\mathrm{SU}(2)\times\mathrm{U}(1)$ gauge sector and a first-principles derivation of the weak mixing angle $\theta_W$.
\item[\textbf{IV}] A parameter-free prediction for the Electroweak Symmetry Breaking (EWSB) scale.
\item[\textbf{V}] A formal dynamical rule for quantum measurement via R↔T phase transitions.
\end{itemize}

\paragraph{Core Axioms (SST)}
    \begin{enumerate}
    \item \textbf{Swirl Medium:} Physics is formulated on $\mathbb{R}^3$ with absolute reference time. Dynamics occur in a frictionless, incompressible \emph{swirl condensate}, which serves as a universal substrate.
    \item \textbf{Swirl Strings (Circulation and Topology):} Particles and field quanta correspond to closed vortex filaments (\emph{swirl strings}). The circulation of the swirl velocity around any closed loop is quantized:
    \[
        \Gamma = \oint \vswirl \cdot d\boldsymbol{\ell} = n\,\kappa,\qquad n\in\mathbb{Z},\qquad \kappa = \frac{h}{m_{\text{eff}}}.
    \]
    Discrete quantum numbers (mass, charge, spin) track to the topological invariants of the swirl string.
    \item \textbf{String-induced gravitation:} Macroscopic attraction emerges from coherent swirl flows and swirl-pressure gradients. The effective gravitational coupling $G_{\text{swirl}}$ is fixed by canonical constants.
    \item \textbf{Swirl Clocks:} Local proper-time rate depends on tangential swirl speed $v$, ticking slower by the factor $S_t=\sqrt{\,1-v^2/c^2\,}$ relative to an observer at rest in the medium.
    \item \textbf{Dual Phases (Wave–Particle):} Each swirl string has two limiting phases: an extended \emph{R-phase} (unknotted, wave-like) and a localized \emph{T-phase} (knotted, particle-like). Measurement is a dynamical transition between these phases.
    \item \textbf{Taxonomy:} Unknotted excitations are bosonic modes; chiral hyperbolic knots map to quarks; torus knots map to leptons. The particle-knot dictionary is canonical.
    \end{enumerate}
\end{abstract}
\vfill
\titlepageClose

%================================================
\section{Canon Governance and Formal System}
%================================================
    \label{sec:canon_governance}
    The SST formal system is $\mathcal{S}=(\mathcal{P},\mathcal{D},\mathcal{R})$, comprising axioms $\mathcal{P}$, definitions $\mathcal{D}$, and admissible inference rules $\mathcal{R}$. A statement is \emph{canonical} if it is derivable within $\mathcal{S}$ and consistent with all prior canonical results.
    \begin{itemize}
    \item \textbf{Axiom (Canonical):} A primitive assumption of SST (e.g., swirl medium).
    \item \textbf{Definition (Canonical):} Introduces a symbol by construction (e.g., Swirl Coulomb constant $\Lambda$).
    \item \textbf{Theorem/Corollary (Canonical):} A proven consequence within $\mathcal{S}$.
    \item \textbf{Calibration (Empirical):} Recommended numerical values for canonical symbols, used to anchor the theory but not as premises in proofs.
    \item \textbf{Research Track (Non-canonical):} Conjectures or alternatives pending proof or axiomatization.
    \end{itemize}

%================================================
\section{Classical Invariants and Swirl Quantization}
%================================================
    \label{sec:classical_invariants}
    Under Axiom 1, the Euler equations yield standard vortex invariants.
    \begin{itemize}
    \item \textbf{Kelvin's circulation theorem:} $\frac{d\Gamma}{dt}=0, \quad \Gamma=\oint_{\mathcal{C}(t)} \vswirl\cdot d\boldsymbol{\ell}$.
    \item \textbf{Vorticity transport:} $\pdv{\omegas}{t} = \nabla\times(\vswirl\times \omegas)$.
    \end{itemize}

    \begin{axiom}[Chronos–Kelvin Invariant]
    \label{ax:chronos-kelvin}
    For any thin, closed swirl loop (material core radius $R(t)$), the material invariant holds:
    \begin{equation}
    \boxed{\;
    \frac{D}{Dt}\!\Big(R^2\,\omega\Big)=0,
        \;} \quad \text{equivalently,} \quad
    \boxed{\;
    \frac{D}{Dt}\!\Big(
    \frac{c}{r_c}\,R^2 \sqrt{\,1-S_t^2\,}
    \Big)=0\,,
        \;}
    \label{eq:CK}
    \end{equation}
    where $\omega=\|\omegas\|$ on the loop and $S_t=\sqrt{\,1-(\omega r_c/c)^2\,}$ is the local Swirl Clock factor.
    \end{axiom}

    \subsection{Swirl Quantization Principle}
        \label{sec:swirl_quantization}
        The joint discreteness of circulation ($\Gamma = n\kappa$) and topology (knot invariants) is the \emph{Swirl Quantization Principle}, which replaces canonical commutation relations as the origin of quantum phenomena.

%===============================================================
\section{Canonical Constants and Effective Densities}
%===============================================================
    \label{sec:canonical_constants}
    \subsection*{Primary SST Constants (SI units)}
        \begin{itemize}
        \item \textbf{Swirl speed scale (core):} $\vscore = \num{1.09384563e6}\ \si{m/s}$.
        \item \textbf{String core radius:} $\rc = \num{1.40897017e-15}\ \si{m}$.
        \item \textbf{Effective fluid density:} $\rhof = \num{7.0e-7}\ \si{kg/m^3}$.
        \item \textbf{Mass-equivalent density:} $\rhom = \num{3.8934358266918687e18}\ \si{kg/m^3}$.
        \item \textbf{EM-like maximal force:} $\FmaxEM = \num{2.9053507e1}\ \si{N}$.
        \item \textbf{Gravitational maximal force:} $\FmaxG = \num{3.02563e43}\ \si{N}$.
        \item \textbf{Swirl-EM Transduction Constant:} $G_{\swirlarrow}$ (derived from flux quantum, dimensionless).
        \end{itemize}

    \subsection*{Universal Constants Used}
        \begin{itemize}
        \item $c=\num{299792458}\ \si{m/s}$, \quad $t_p=\num{5.391247e-44}\ \si{s}$ (Planck time), \quad $\alpha \approx \num{7.29735256e-3}$.
        \end{itemize}

    \subsection{Derivation of Effective Densities}
        We define the swirl energy density $\rhoE = \tfrac{1}{2}\,\rhof\,\vscore^2$ and mass-equivalent density $\rhom = \rhoE/c^2$. The physical origin of $\rhof$ is derived from coarse-graining an ensemble of swirl strings. For an ensemble with average vorticity $\langle\omega_{\swirlarrow}\rangle$:
        \begin{equation}
        \boxed{\,\rhof = \frac{\rhom\,\rc}{2\,\vscore}\,\langle\omega_{\swirlarrow}\rangle\,}\,.
        \label{eq:rhof_from_omega}
        \end{equation}

%================================================
\section{The Swirl-Electromagnetic Bridge}
%================================================
    \label{sec:swirl_em_bridge}
    Electromagnetism emerges from the dynamics of the swirl condensate. Local changes in the density and configuration of swirl strings act as sources for electric and magnetic fields.

    \begin{definition}[Swirl Areal Density]
    The swirl areal density, $\varrho_{\swirlarrow}$, is the coarse-grained number of swirl string cores per unit area normal to their average orientation.
    \end{definition}

    \begin{theorem}[Swirl-Induced Electromotive Force]
    A time-varying swirl areal density generates an effective source term in Faraday's law of induction. This links swirl reconnection dynamics directly to the electromotive force.
    \[ \boxed{ \nabla \times \mathbf{E} = -\,\partial_t \mathbf{B} \;-\; \mathbf{b}_{\swirlarrow}, \qquad \mathbf{b}_{\swirlarrow} = G_{\swirlarrow}\,\partial_t \varrho_{\swirlarrow} } \]
    Here, $G_{\swirlarrow}$ is the canonical Swirl–EM transduction constant, dimensionally equivalent to a magnetic flux quantum. This law provides the mechanism by which kinetic energy from swirl reconnections is converted into electromagnetic field energy.
    \end{theorem}

%================================================
\section{Master Equations (Boxed Canonical Relations)}
%================================================
    \label{sec:master_equations}

    \subsection*{Hydrodynamics and Swirl Mechanics}
        \begin{itemize}
        \item \textbf{Swirl Coulomb Potential:} $V_{\text{SST}}(r)=-\,\frac{\Lambda}{\sqrt{r^2+\rc^2}}$, with $\Lambda = 4\pi\,\rhom\,\vscore^2\,\rc^4$. This potential correctly reproduces the hydrogen spectrum.
        \item \textbf{Swirl Pressure Law:} $\frac{1}{\rhof}\frac{dp_{\text{swirl}}}{dr}=\frac{v_\theta(r)^2}{r}$.
        \item \textbf{Swirl Clock:} $\frac{dt_{\text{local}}}{dt_{\infty}} = \sqrt{\,1 - \frac{\|\vswirl\|^2}{c^2}\,}$ for $r=\rc$.
        \item \textbf{Swirl Hamiltonian Density:} $\mathcal{H}_{\text{SST}} = \frac{1}{2}\rhof\,\|\vswirl\|^2 + \frac{1}{2}\rhof\,\rc^2\,\|\omegas\|^2 + \lambda\,(\nabla\cdot \vswirl)$.
        \end{itemize}

    \subsection*{Gravitation and Mass}
        \begin{itemize}
        \item \textbf{Swirl–Gravity Coupling:} $\boxed{\,G_{\text{swirl}} = \frac{\vscore\,c^5\,t_p^2}{2\,\FmaxEM\,\rc^2}\, \approx G_{\text{Newton}}}\,$.
        \item \textbf{Topology–Driven Mass Law:} $\boxed{\,M(K) = \left(\frac{4}{\alpha}\right) b^{-3/2}\,\varphi^{-g}\,n^{-1/\varphi} \left(\frac{1}{2}\,\rhof \vscore^2\right) \frac{\pi\,\rc^3\,\mathcal{L}_{\text{tot}}(K)}{c^2}\,}$.
        \end{itemize}

%================================================
\section{The Standard Gauge Sector}
%================================================
    \label{sec:gauge_core}
    The gauge structure of the Standard Model emerges from the collective dynamics of swirl-string directors.

    \begin{theorem}[Emergent Yang–Mills from Swirl Directors]
    The elasticity of local swirl director fields ($U_3, U_2, \vartheta$) gives rise to the Yang-Mills Lagrangian upon coarse-graining:
    \[ \mathcal L_{\rm dir} \quad \Longrightarrow \quad \boxed{\ \mathcal L_{\rm YM}^{\rm eff}=-\tfrac14\sum_{i=1}^3 g_i^{-2}\,F^{(i)}_{\mu\nu}F^{(i)\,\mu\nu},\qquad g_i^{-2} \propto \kappa_i\ } \]
    where $\kappa_i$ are the stiffness constants of the director fields.
    \end{theorem}

    \subsection{Knot-to-Representation Map and Particle Taxonomy}
        Quantum numbers are identified with topological invariants of swirl strings. The canonical mapping is summarized below.

        \begin{definition}[Hypercharge from Swirl Indices]
        For a knot $K$ with color sign $s_3$, doublet indicator $d_2$, and twist sign $\tau$:
        \[ \boxed{ Y(K)=\tfrac{1}{2}+\tfrac{2}{3}s_3(K)-d_2(K)-\tfrac{1}{2}\tau(K) } \]
        \end{definition}

        \begin{table}[h!]\centering\small
        \caption{Canonical Particle-Knot Taxonomy (One Generation)}
        \begin{tabular}{@{}lllll@{}}\toprule
        Field & Rep & $Y$ & $Q$ example & Knot Class Analogue\\ \midrule
        $Q_L^{(i)}=(u_L,d_L)$ & $(\mathbf 3,\mathbf 2)$ & $+1/6$ & $(+2/3,-1/3)$ & Chiral Hyperbolic\\
        $u_R^{(i)}$ & $(\mathbf 3,\mathbf 1)$ & $+2/3$ & $+2/3$ & Chiral Hyperbolic\\
        $d_R^{(i)}$ & $(\mathbf 3,\mathbf 1)$ & $-1/3$ & $-1/3$ & Chiral Hyperbolic\\
        $L_L^{(i)}=(\nu_L,e_L)$ & $(\mathbf 1,\mathbf 2)$ & $-1/2$ & $(0,-1)$ & Torus Knot \\
        $e_R^{(i)}$ & $(\mathbf 1,\mathbf 1)$ & $-1$ & $-1$ & Torus Knot\\
        \bottomrule
        \end{tabular}
        \end{table}

        \begin{theorem}[Per-Generation Anomaly Cancellation]
        The spectrum of left-chiral fermions generated by the knot-to-representation map is free of all gauge and mixed gravitational anomalies.
        \end{theorem}

    \subsection{Coupling Constants, EWSB, and the Weak Mixing Angle}
        \paragraph{Canonical Renormalization Scale and Couplings.}
            The theory defines a natural energy scale $\mu_\* \equiv \hbar\vscore/\rc \approx 0.511~\mathrm{MeV}$, where the gauge couplings are determined from first principles via a dimensionless core modulus $\Sigma_{\rm core} = 1/\pi$ and topological weights $W_i$.
            \[ \boxed{\ g_i^{-2}(\mu_\*) = \kappa_i\;\Sigma_{\rm core}\;W_i\ }. \]

            \begin{theorem}[Emergence of the Weak Mixing Angle]
            The weak mixing angle $\theta_W$ is not a fundamental constant but is determined by the ratio of the underlying director field stiffnesses for the U(1) and SU(2) sectors.
            \[ \boxed{ \tan^2\theta_W = \frac{g'^2}{g^2} = \frac{\kappa_2}{\kappa_1} } \]
            This ratio is, in principle, computable from the topology of the swirl condensate.
            \end{theorem}

        \paragraph{Electroweak Symmetry Breaking.}
            The EWSB scale $v_\Phi$ is determined by the bulk swirl energy density $u_{\rm swirl} = \frac{1}{2}\rhof\vscore^2$.
            \[ \boxed{\ v_\Phi\ =\ u_{\rm swirl}^{1/4}\;\big(W_1 W_2 W_3\big)^{1/4}\ \approx 259.5\ \mathrm{GeV}\ }. \]
            This parameter-free prediction is within 5.4% of the measured value.

%================================================
\section{Wave–Particle Duality and Quantum Measurement}
%================================================
\label{sec:wave_particle_duality}
\paragraph{Dual Phases and de Broglie Wavelength.}
    An unknotted R-phase string is a delocalized wave, while a knotted T-phase is a localized particle. The de Broglie relation $\lambda = h/p$ emerges from the wave coherence condition for a closed R-phase loop.

\paragraph{Formalizing Measurement.}
    The transition from wave-like to particle-like behavior (measurement) is a physical, dynamical process.
    \begin{theorem}[R↔T Transition Dynamics]
    The transition rate $\Gamma_{R \to T}$ for an R-phase string to collapse into a T-phase is proportional to the energy density of the interacting field, $\rho_{E, \text{int}}$, and a function $f(\Delta\mathcal{K})$ of the change in topological invariants (knot complexity).
    \[ \boxed{ \Gamma_{R \to T} = C_{\text{int}} \cdot \rho_{E, \text{int}} \cdot f(\Delta\mathcal{K}) } \]
    where $C_{\text{int}}$ is an interaction-specific coupling constant. This law replaces the measurement postulate with a predictive, dynamical equation for collapse.
    \end{theorem}

\paragraph{Unknot Bosons and Photons.}
    The photon is identified with a delocalized, unknotted swirl oscillation, whose dynamics are equivalent to Maxwell's equations.
    \[ \boxed{\,\partial_t^2 \mathbf{a} - c^2\,\nabla\times(\nabla\times \mathbf{a}) = 0, \qquad \nabla\cdot \mathbf{a}=0\,}\,. \]

%================================================
% APPENDICES
%================================================
    \appendix

\section{Unified SST Lagrangian (Definitive Form)}
\label{sec:lagrangian}
The total dynamics of swirl and emergent gauge fields are described by:
\[
    \boxed{\,\mathcal{L}_{\text{SST+Gauge}}
        =
        \underbrace{\frac{1}{2}\rhof\,\|\vswirl\|^2
        - \rhof\,\Phi_{\text{swirl}}
            + \lambda(\nabla\cdot\vswirl)
            + \chi_h\,\rhof\,(\vswirl\cdot\omegas)}_{\text{SST Hydrodynamics}}
        \;+\;
        \underbrace{\mathcal{L}_{\text{YM}}}_{\text{Gauge Fields}}
        \;+\;
        \underbrace{\mathcal{L}_{\text{Matter}}}_{\text{Gauge-charged Matter}}
        \,}\,.
\]
All terms have units of energy density ($\mathrm{J\,m^{-3}}$).

\section{Canon 4R: Research Extensions (Non-Canonical)}
\begin{itemize}
\item \textbf{Blackbody Swirl Temperature:} A longstanding research goal is to derive the blackbody spectrum from the thermodynamics of a "gas" of unknotted swirl strings. This requires formalizing a swirl temperature and achieving thermal equilibrium, which remains an open challenge.
\item \textbf{Swirl-Helicity and Chern-Simons Couplings:} Potential couplings between the swirl helicity term and topological terms in the gauge sector.
\item \textbf{Detailed Knot-to-Mass Spectrum:} A complete mapping from specific knot classes to the full particle mass spectrum, including binding energies.
\end{itemize}

\section{Persona Prompts and Session Checklist}
\subsection*{Reviewer Persona}
    \scriptsize
    You are a peer reviewer for an SST paper. Use only the definitions and constants in the "SST Canon (\canonversion)".
    Check dimensional consistency, limiting behavior, and numerical validation. Flag any use of non-canonical
    constants or equations unless equivalence is proved.

\subsection*{Theorist Persona}
    You are a theoretical physicist specialized in Swirl String Theory (SST). Base all reasoning on the attached
    "SST Canon (\canonversion)". Your task: derive [TARGET RESULT] from the canonical axioms and equations.

%================================================
% References
%================================================
    \begin{thebibliography}{9}
        % Standard references would be included here.
    \bibitem{Weinberg1967} S. Weinberg, \emph{A Model of Leptons}, Phys. Rev. Lett. 19, 1264 (1967).
    \bibitem{PeskinSchroeder} M. E. Peskin and D. V. Schroeder, \emph{An Introduction to Quantum Field Theory}, Addison-Wesley (1995).
    \bibitem{PDG2024} R.L. Workman et al. (Particle Data Group), \emph{Review of Particle Physics}, Prog. Theor. Exp. Phys. 2022, 083C01 (2022) and 2023 update.
    \end{thebibliography}

\end{document}
