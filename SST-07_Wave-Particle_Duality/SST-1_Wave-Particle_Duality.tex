\documentclass[11pt,a4paper]{article}
\usepackage[utf8]{inputenc}
\usepackage{amsmath,amssymb,amsthm}
\usepackage{graphicx}
\usepackage{geometry}
\usepackage{hyperref}
\usepackage{tikz}
\usepackage[T1]{fontenc}
\geometry{margin=1in}
% ===============================
% Macros (canonicalized)
% ===============================

% swirl arrows (context-aware)
\newcommand{\swirlarrow}{%
    \mathchoice{\mkern-2mu\scriptstyle\boldsymbol{\circlearrowleft}}%
    {\mkern-2mu\scriptstyle\boldsymbol{\circlearrowleft}}%
    {\mkern-2mu\scriptscriptstyle\boldsymbol{\circlearrowleft}}%
    {\mkern-2mu\scriptscriptstyle\boldsymbol{\circlearrowleft}}%
}
\newcommand{\swirlarrowcw}{%
    \mathchoice{\mkern-2mu\scriptstyle\boldsymbol{\circlearrowright}}%
    {\mkern-2mu\scriptstyle\boldsymbol{\circlearrowright}}%
    {\mkern-2mu\scriptscriptstyle\boldsymbol{\circlearrowright}}%
    {\mkern-2mu\scriptscriptstyle\boldsymbol{\circlearrowright}}%
}

% Canonical symbols
\newcommand{\vswirl}{\mathbf{v}_{\swirlarrow}}
\newcommand{\vswirlcw}{\mathbf{v}_{\swirlarrowcw}}
\newcommand{\SwirlClock}{S_{(t)}^{\swirlarrow}}
\newcommand{\SwirlClockcw}{S_{(t)}^{\swirlarrowcw}}
\newcommand{\omegas}{\boldsymbol{\omega}_{\swirlarrow}}  % swirl vorticity
\newcommand{\vscore}{v_{\swirlarrow}}                    % shorthand: |v_swirl| at r=r_c
\newcommand{\vnorm}{\lVert \vswirl \rVert}               % swirl speed magnitude
\newcommand{\rhof}{\rho_{\!f}}                           % effective fluid density
\newcommand{\rhoE}{\rho_{\!E}}                           % swirl energy density
\newcommand{\rhom}{\rho_{\!m}}                           % mass-equivalent density
\newcommand{\rc}{r_c}                                    % string core radius (swirl string radius)
\newcommand{\FmaxEM}{F_{\mathrm{EM}}^{\max}}             % EM-like maximal force scale
\newcommand{\FmaxG}{F_{\mathrm{G}}^{\max}}               % G-like maximal force scale
\newcommand{\Lam}{\Lambda}                               % Swirl Coulomb constant
\newcommand{\Om}{\Omega_{\swirlarrow}}                   % swirl angular frequency profile
\newcommand{\alpg}{\alpha_g}                             % gravitational fine-structure analogue

% Policy: the golden constant is only allowed via hyperbolic functions.
\newcommand{\xig}{\operatorname{asinh}\!\left(\tfrac{1}{2}\right)}
\newcommand{\phig}{\exp(\xig)}
\newcommand{\phialg}{\bigl(1+\sqrt{5}\bigr)/2}
\newcommand{\xigold}{\tfrac{3}{2}\,\xig}
\newcommand{\GoldenDeclare}{%
    \textbf{Golden (hyperbolic)}:\ \(\ln\phi=\xig\), hence \(\phi=\phig\).
    \ \emph{(Equivalently, \(\phi=\phialg\); the algebraic form is derivative.)}%
}

\title{Wave--Particle Duality in Swirl--String Theory: \\
Toroidal Circulation, Knot Collapse, and Photon-Induced Transitions}

\author{Omar Iskandarani \\ Independent Researcher, Groningen, The Netherlands}
\date{\today}

\begin{document}
\maketitle

\begin{abstract}
    We present a Swirl--String Theory (SST) interpretation of the electron’s wave--particle duality based on Canon v0.3.1. The electron is modeled as a swirl-string admitting two phases: an unknotted, delocalized toroidal circulation $\mathcal{R}$ (wave-like) and a localized, knotted soliton (trefoil) $\mathcal{T}$ (particle-like). We define an effective energy functional including bulk swirl density, line tension, near-contact interactions, and helicity terms. Photon-induced transitions $\mathcal{R} \rightleftarrows \mathcal{T}$ occur at resonant frequencies determined by impedance matching between electromagnetic modes and topological excitations. We show how circulation quantization on $\mathcal{R}$ recovers de Broglie wave relations, while $\mathcal{T}$ provides a stable, localized excitation. This framework yields falsifiable predictions in atomic absorption spectra, Rydberg scaling, and polarization-dependent selection rules.
\end{abstract}

\section{Introduction}

    Wave--particle duality remains a central puzzle in quantum mechanics. The hydrodynamic representation of the Schrödinger equation by Madelung~\cite{Madelung1927}, the matter-wave hypothesis of de Broglie~\cite{deBroglie1925}, and Bohm’s hidden-variable interpretation~\cite{Bohm1952a,Bohm1952b} all highlight fluid analogies. In superfluids, Onsager’s circulation quantization~\cite{Onsager1949} and Feynman’s analysis of helium vortices~\cite{Feynman1955} demonstrate how phase coherence enforces quantized flow. Earlier, Helmholtz~\cite{Helmholtz1858} and Kelvin~\cite{Kelvin1869} established the conservation of vorticity and proposed atoms as vortex rings.

    Within SST, the Canon postulates a universal swirl condensate with effective density $\rhof$ and characteristic swirl velocity $\|\mathbf{v}_{\circlearrowleft}\|$. Electrons are interpreted not as point particles but as filamentary swirl-strings. We propose here that the dual character of the electron arises from two distinct phases of the same underlying string.

\section{Two Phases of the Electron String}

    \subsection{Unknotted toroidal ring R}

        The delocalized phase is an unknotted ring circulation $\mathcal{R}$ with radius $R$ and length $L(\mathcal{R})=2\pi R$. Its circulation is quantized:
\begin{equation}
            \Gamma_n = \oint_{\mathcal{R}} \mathbf{v} \cdot d\boldsymbol{\ell} = n \frac{h}{m_e}, \qquad n\in \mathbb{Z} .
\end{equation}
Hence the tangential speed is
\begin{equation}
    v_\theta(R) = \frac{\Gamma_n}{2\pi R},
\end{equation}
        and the phase around the ring is $e^{i n\theta}$, yielding the de Broglie relation
\begin{equation}
    \lambda_{\textrm ring} = \frac{2\pi R}{n} = \frac{h}{p_\theta}, \qquad p_\theta = m_e v_\theta .
\end{equation}
        Thus $\mathcal{R}$ naturally supports interference and standing-wave behavior.

    \subsection{Knotted trefoil soliton }

        The localized phase $\mathcal{T}$ is a knotted filament (e.g.\ trefoil $3_1$) with enhanced curvature and helicity. Its invariants satisfy $C(\mathcal{T})>0$, $\mathcal{H}(\mathcal{T})\neq 0$, and typically $L(\mathcal{T})>L(\mathcal{R})$ for equal scale.

\subsection{Effective energy functional}

We postulate an effective energy functional
\begin{equation}
    \boxed{
        \mathcal{E}_{\textrm eff}[K] =
        \underbrace{\epsilon_0 A L(K)}_{\text{bulk swirl energy}} +
        \underbrace{\beta L(K)}_{\text{line tension}} +
        \underbrace{\alpha C(K)}_{\text{near-contact}} +
        \underbrace{\gamma \mathcal{H}(K)}_{\text{helicity}}
    }
\end{equation}
        where $K$ is the filament curve, $A=\pi r_c^2$ is the core cross-sectional area, and $\epsilon_0$ is the Canon bulk energy density
\begin{equation}
            \epsilon_0 = \frac{4}{\alpha_{\textrm fs}\,\varphi}\left(\tfrac12 \rhof \|\mathbf{v}_{\circlearrowleft}\|^2\right).
\end{equation}

\subsection{Dimensional analysis}

\begin{itemize}
    \item $[\epsilon_0]=\text{J\,m}^{-3}$, $[A]=\text{m}^2$, $[L]=\text{m}$ $\Rightarrow [\epsilon_0 A L]=\text{J}$.
    \item $[\beta]=\text{J\,m}^{-1}$, hence $[\beta L]=\text{J}$.
    \item $\alpha C$, $\gamma \mathcal H$ contribute as energy terms.
\end{itemize}

Numerically, with Canon constants,
\begin{equation}
    \epsilon_0 \approx 1.4187\times10^{8}\ \text{J\,m}^{-3}, \quad
    A\approx 6.24\times10^{-30}\ \text{m}^2,
\end{equation}
so bulk energy per length is
\begin{equation}
    \epsilon_0 A \approx 8.85\times10^{-22}\ \text{J\,m}^{-1}.
\end{equation}

\section{Photon-Driven Transitions}

\subsection{Resonance condition}

The transition $\mathcal R\to \mathcal T$ changes the energy by
\begin{equation}
    \Delta E = (\epsilon_0 A + \beta)\Delta L + \alpha C(\mathcal T) + \gamma \mathcal H(\mathcal T).
\end{equation}
Resonance occurs when
\begin{equation}
    \boxed{\ \hbar \omega \approx \Delta E\ }.
\end{equation}

\subsection{Selection rules}

\begin{itemize}
    \item Angular momentum matching: photon helicity must match the winding number (e.g.\ $\Delta n=\pm 3$ for trefoil).
    \item Radius dependence: larger $R$ reduces $\Delta L$, lowering resonance energy (Rydberg scaling).
    \item Polarization dependence: circular polarization aligned with knot chirality enhances coupling.
\end{itemize}

\section{Effective Field Theory Formulation}

At the EFT level, the string world-sheet $\Sigma$ enters a Lagrangian
\begin{equation}
\mathcal L = \tfrac12 \rhof \|\mathbf v\|^2 - \rho_E
- \beta \ell[\Sigma] - \alpha \mathcal C[\Sigma]
- \gamma \mathcal H[\Sigma] + \mathcal L_{\textrm EM}^{\textrm int}[A_\mu;\Sigma].
\end{equation}
In the static limit, $\mathcal L \to -\mathcal E_{\textrm eff}$.
Time-dependent solutions yield Rabi-like oscillations between $\mathcal R$ and $\mathcal T$ 
under monochromatic drive $\omega\approx \Delta E/\hbar$.


%==============================================================================
\section{Photoelectric and Compton Effects from Canonical Photon Modes}
%==============================================================================

    \subsection{Canonical Photon Mode Recap}
    From Sec.~X (Canonical photon derivation), the photon is described as a
    \emph{pulsed unknot swirl-string}, with action
    \[
    S[\xi] = \tfrac{1}{2}\,\rhof A_{\mathrm{eff}}
    \int dt \int_0^L ds \left[(\partial_t \xi)^2 - c^2 (\partial_s \xi)^2 \right],
    \]
    normal modes $\xi_m(s,t)$ of frequency $\omega_m=c k_m$,
    and single-quantum amplitude
    \begin{equation}
    a_m = \sqrt{\frac{\hbar}{\rhof\,A_{\mathrm{eff}}\,L\,\omega_m}}.
    \label{eq:canonical-photon-amplitude}
    \end{equation}
    This constitutes the \emph{canonical} photon description in SST:
    delocalized circulation with no rest-mass density ($\rhom=0$),
    but finite energy density $\rhoE$.

    \subsection{Swirl--EM Mapping (Empirical Consistency)}
    To compare with classical electrodynamics, we adopt the
    empirical swirl--EM map
    \begin{equation}
    \mathbf{E}=\sqrt{\frac{\rhof}{\varepsilon_0}}\,\mathbf{v},\qquad
    \mathbf{B}=\sqrt{\frac{\rhof}{\varepsilon_0}}\,\mathbf{b},\qquad
    \omega=ck,
    \label{eq:swirl-em-map}
    \end{equation}
    so that the quadratic energy--momentum balance reproduces Maxwell:
    \[
    u=\tfrac{\varepsilon_0}{2}E^2+\tfrac{1}{2\mu_0}B^2,\qquad
    \mathbf{S}=\mu_0^{-1}\mathbf{E}\!\times\!\mathbf{B},
    \]
    with $c^{-2}=\varepsilon_0\mu_0$ \cite{Jackson1999}.
    Equation~\eqref{eq:canonical-photon-amplitude} is then consistent with the
    cavity-QED single-photon field amplitudes \cite{HarocheRaimond2006,ScullyZubairy1997}.

    \subsection{Photoelectric Effect as $\mathcal R\!\to\!\mathcal T$ Transition}
    Within the $\mathcal R/\mathcal T$ two-phase electron model (SST-1),
    absorption of a canonical photon mode by a delocalized electron $\mathcal R$
    can trigger localization to a knotted $\mathcal T$ state if
    \begin{equation}
    \hbar\omega \;\ge\; \Phi + \Delta E_{RT},
    \end{equation}
    where $\Phi$ is the material work function and
    $\Delta E_{RT}=E[\mathcal T]-E[\mathcal R]$ the localization gap.
    In metals, $\Delta E_{RT}$ is negligible or renormalized into
    an effective $\Phi_{\textrm eff}$, giving Einstein’s law
    \begin{equation}
    K_{\max}=\hbar\omega-\Phi_{\textrm eff}.
    \end{equation}
    This is thus recovered as an \emph{empirical consequence} of the canonical photon.

    \subsection{Compton Scattering with a Delocalized Photon Packet}
    Treating the incident photon as a canonical delocalized mode of momentum
    $(\hbar\omega/c,\ \hbar\mathbf{k})$ and the target electron as a localized $\mathcal T$
    of mass $m_e$, energy--momentum conservation gives
    \begin{equation}
    \Delta\lambda=\lambda'-\lambda=\frac{h}{m_e c}\,(1-\cos\theta),
    \end{equation}
    the standard Compton shift \cite{Compton1923}.
    The equality follows because the swirl--EM map
    \eqref{eq:swirl-em-map} preserves stress--energy flux.

    \subsection{Consistency, Limits, and Corrections}
    Finite core structure of the electron string introduces
    corrections $\mathcal{O}((kr_c)^2)$, negligible for
    $kr_c\ll 1$ (X-rays) but potentially testable at $\gamma$-ray energies.

    \paragraph{Interpretation.}
    \begin{enumerate}
    \item \textbf{Photoelectric:} canonical photon $\to$ delocalized $\mathcal R$ electron $\to$ localized $\mathcal T$ emission.
    \item \textbf{Compton:} canonical photon scattering on $\mathcal T$ electron, kinematics as in QED.
    \item \textbf{Polarization:} photon helicity corresponds to swirl-clock orientation, affecting cross sections and selection rules.
    \end{enumerate}

    \subsection{Notes on Provenance}
    \begin{itemize}
    \item Einstein’s photoelectric law \cite{Einstein1905,Millikan1916}.
    \item Compton scattering and Klein--Nishina \cite{Compton1923,KleinNishina1929}.
    \item Swirl--EM map consistency with Maxwell \cite{Jackson1999}.
    \item Cavity QED amplitudes \cite{HarocheRaimond2006,ScullyZubairy1997}.
    \end{itemize}



    \section{Predictions and Tests}

    \begin{enumerate}
    \item \textbf{Spectral lines:} new resonances in absorption spectra, not coinciding with standard atomic transitions.
    \item \textbf{Rydberg atoms:} red-shifted knotting lines with increasing principal quantum number.
    \item \textbf{Pump--probe:} suppression of interference fringes coincident with localization after resonant pump.
    \item \textbf{Polarization dependence:} transition rates sensitive to photon chirality.
    \end{enumerate}

    \section{Conclusion}

    In SST, the same electron string supports both a delocalized circulation $\mathcal R$ (wave aspect) and a localized knot $\mathcal T$ (particle aspect). Photon-induced transitions between these phases provide a dynamical explanation of wave--particle duality consistent with the Canon and with hydrodynamic analogies established since Helmholtz and Kelvin. The framework yields concrete, falsifiable predictions.

\appendix
\section*{Appendix: Popular Summary}

A swirl-ring can ripple smoothly like a hula-hoop (wave).
If twisted into a knot, the swirl bunches up in one spot (particle).
A photon of just the right frequency can flip the string between these two states.

%===============================
% Extended Analysis Sections
%===============================

\section{Quantitative Fringe Geometry and Visibility}

Consider a double slit with center-to-center separation $s$ and distance $L$ from slits to screen.
Let the electron approach in the delocalized ring phase $\mathcal R$ with mean axial momentum $p_z$ (de Broglie wavelength $\lambda=h/p_z$).
In the Fraunhofer regime ($L \gg s^2/\lambda$), the transverse intensity on the screen is well approximated by
\begin{equation}
I(x) \;\propto\; I_1(x) + I_2(x) + 2\sqrt{I_1(x)I_2(x)}\,\cos\!\Big(\tfrac{2\pi s}{\lambda}\tfrac{x}{L} + \phi_0\Big),
\end{equation}
with fringe spacing
\begin{equation}
\boxed{\ d \;=\; \frac{\lambda L}{s}\ }.
\end{equation}
Within SST, the phase $\phi_0$ is the circulation phase offset inherited from $\mathcal R$; any path-dependent coupling to the environment adds a random phase $\delta\phi$ that reduces the fringe visibility
\begin{equation}
\mathcal V \equiv \frac{I_{\max}-I_{\min}}{I_{\max}+I_{\min}} \;=\; \big|\langle e^{i \delta\phi}\rangle\big|.
\end{equation}


\begin{figure}[t]
\centering
\begin{tikzpicture}[scale=0.75, >=Latex, line cap=round, line join=round]
    % Colors (adjust to grayscale if desired)
\definecolor{flow}{RGB}{33,150,243}    % coherent flow / phase
\definecolor{emph}{RGB}{220,50,47}     % localized hit
\definecolor{screen}{gray}{0.20}

% --- Source: toroidal circulation (delocalized ring R) ---
\draw[thick] (-1,0) circle (0.6);
% circulation arrows along the ring
\foreach \a in {20,80,140,200,260,320}{
    \draw[->,flow,very thick]
    (-1,0) + ({0.6*cos(\a)},{0.6*sin(\a)})
    ++({-0.20*sin(\a)},{0.20*cos(\a)}) -- ++({0.25*sin(\a)},{-0.25*cos(\a)});
}
\node[below] at (-1,-0.8) {$\mathcal R$ (delocalized ring)};

% --- Barrier with two slits ---
\draw[very thick] (3,-2.8) -- (3,-1.3);
\draw[very thick] (3,-0.3) -- (3,0.3);
\draw[very thick] (3,1.3) -- (3,2.8);
\node[above, align=center] at (3,3.1) {barrier with\\two slits};

% --- Screen ---
\draw[very thick,screen] (11,-3) -- (11,3);
\node[above] at (11,3.2) {screen};

% --- Incident phase field toward slits (schematic parallel rays) ---
\foreach \y in {-1.8,-1.4,...,1.8}{
    \draw[flow,opacity=0.20] (-0.4,\y) -- (2.8,\y);
}

% --- Huygens-like secondary wavelets from each slit (schematic arcs) ---
\foreach \r in {0.6,1.1,1.6,2.1,2.6,3.1,3.6,4.1,4.6,5.1,5.6,6.1}{
    \draw[flow,thick,opacity=0.35] (3, 0.8) arc (0:60:\r);
    \draw[flow,thick,opacity=0.35] (3,-0.8) arc (0:-60:\r);
}

% --- Interference intensity on the screen: alternating thicker ticks ---
\foreach \yy/\wid in {-2.6/0.30,-2.2/0.10,-1.8/0.25,-1.4/0.10,-1.0/0.35,
-0.6/0.10,-0.2/0.45, 0.2/0.10, 0.6/0.35, 1.0/0.10,
1.4/0.25, 1.8/0.10, 2.2/0.30}{
    \draw[screen,line width=\wid cm] (11,\yy) -- ++(0.40,0);
}

% --- Localized detection (collapse to T) ---
\fill[emph] (11,1.0) circle (0.08);
\draw[emph,very thick] (11,1.0) circle (0.18);
\node[right,emph] at (11.2,1.0) {$\mathcal T$ (localized hit)};

% --- Two representative coherent paths (guiding arrows) ---
\draw[->,flow,thick] (2.6, 0.8) -- (6,1.2);
\draw[->,flow,thick] (2.6,-0.8) -- (6,0.4);
\end{tikzpicture}
\caption{SST double slit schematic. The electron approaches as a delocalized toroidal circulation $\mathcal R$ (ring), whose phase bifurcates at two slits to produce coherent downstream fields that interfere. At the screen, interaction triggers collapse into a localized knotted state $\mathcal T$, yielding discrete impacts while the ensemble reproduces the fringe intensity.}
\label{fig:SST-double-slit}
\end{figure}




\subsection{Ring-phase mapping}
    For the ring state $\mathcal R$ with quantized azimuthal phase $e^{i n\theta}$, the two slits act as partial projectors of this phase onto two spatially separated downstream wavefronts. The resulting interference encodes the \emph{same} azimuthal winding through the optical path difference $\Delta \ell(x) \approx s\,x/L$, hence the cosine argument above.

\section{Which-Way Coupling and Decoherence in SST}

In SST language, ``which-way'' monitoring is an \emph{EM impedance} to the ring phase, parameterized by a coupling rate $\Gamma$ (net photon or field-interaction rate that carries path information).
Let $\tau$ be the transit time through the interferometer.
For weak, memoryless monitoring the phase undergoes a random walk, giving the standard exponential visibility law~\cite{Zurek2003}
\begin{equation}
\boxed{\ \mathcal V(\Gamma,\tau) \;=\; e^{-\Gamma \tau}\ }. \label{eq:Vexp}
\end{equation}
Equivalently, define a \emph{monitoring strength} $\eta\in[0,1]$ as the single-pass which-way information; then $\mathcal V=\sqrt{1-\eta}$ (two-outcome, information-balance form). Both parameterizations are compatible at small $\eta$ via $\eta\simeq 2(1-e^{-\Gamma\tau})$.

\paragraph{SST mechanism.}
    Microscopically, the EM coupling stochastically seeds premature $\mathcal R\!\to\!\mathcal T$ collapses \emph{upstream}, terminating coherent superposition.
    Equation~\eqref{eq:Vexp} thus measures the survival probability of coherence before the knotting transition.

\subsection{Dimensional check}
$\Gamma$ has units s$^{-1}$ and $\tau$ has units s, so $\Gamma\tau$ is dimensionless, consistent with~\eqref{eq:Vexp}.

\section{Delayed-Choice and Quantum Eraser (SST View)}

In a Wheeler-type delayed choice, the interferometer is reconfigured \emph{after} the electron passes the slits~\cite{Wheeler1978}.
In SST, this reconfiguration alters whether the downstream optics \emph{allow} continued $\mathcal R$-phase recombination or instead force local $\mathcal T$-phase collapse:
\begin{itemize}
\item \textbf{Interference mode on:} downstream optics preserve the $\mathcal R$ phase coherence until the screen $\Rightarrow$ fringes.
\item \textbf{Which-way mode on:} downstream optics couple EM impedance strongly, enforcing premature $\mathcal R\!\to\!\mathcal T$ collapse $\Rightarrow$ no fringes.
\end{itemize}
A quantum eraser removes the stored which-way information, effectively \emph{post-selecting} runs where $\mathcal R$-coherence survived to the recombination stage; sorted sub-ensembles then display fringes.

\section{Spectroscopic Rabi Drive Between $\mathcal R$ and $\mathcal T$}

We model the $\mathcal R\leftrightarrow\mathcal T$ manifold as a driven two-level system with detuning $\Delta=\omega-\omega_0$ (where $\hbar\omega_0=\Delta E_{\mathcal R\to\mathcal T}$) and coupling $\Omega_R$ set by the EM impedance overlap:
\begin{align}
\dot c_{\mathcal R} &= -\tfrac{i}{2}\Omega_R c_{\mathcal T} - \tfrac{\gamma_{\mathcal R}}{2} c_{\mathcal R},\\
\dot c_{\mathcal T} &= -\tfrac{i}{2}\Omega_R c_{\mathcal R} - \Big(\tfrac{\gamma_{\mathcal T}}{2} + i\Delta\Big) c_{\mathcal T}.
\end{align}
At resonance ($\Delta=0$) and for $\gamma_{\mathcal R},\gamma_{\mathcal T}\ll \Omega_R$,
\begin{equation}
P_{\mathcal T}(t)=|c_{\mathcal T}(t)|^2 \;=\; \sin^2\!\Big(\frac{\Omega_R t}{2}\Big)\,e^{-\gamma t},\qquad \gamma\equiv\tfrac{\gamma_{\mathcal R}+\gamma_{\mathcal T}}{2}.
\end{equation}
A \emph{pump--probe} synchronized to the flight time across the interferometer can therefore \emph{gate} the knotting probability and modulate fringe visibility in a time-resolved fashion.

%===============================
% Figures (TikZ)
%===============================

\begin{figure}[t]
\centering
\begin{tikzpicture}[>=Latex,scale=1.0]
    % Axes
\draw[->] (0,0) -- (7.2,0) node[below] {$\Gamma\tau$};
\draw[->] (0,0) -- (0,4.2) node[left] {$\mathcal V$};
% Curve V = exp(-Gamma tau)
\draw[very thick,blue!60!black,domain=0:6.5,samples=200] plot(\x,{4*exp(-0.5*\x)});
% Ticks and labels
\foreach \x/\lab in {0/0,1/1,2/2,3/3,4/4,5/5,6/6}{
    \draw (\x,0) -- ++(0,0.08) node[below,yshift=-3pt] {\small \lab};
}
\foreach \y/\lab in {0/0,1/0.25,2/0.50,3/0.75,4/1.00}{
    \draw (0,\y) -- ++(0.08,0) node[left,xshift=-3pt] {\small \lab};
}
\node[above right] at (3.2,1.8) {$\mathcal V(\Gamma,\tau)=e^{-\Gamma\tau}$};
\node[below right,gray] at (0.1,0.2) {weak monitoring};
\node[below right,gray] at (4.5,0.2) {strong monitoring};
\end{tikzpicture}
\caption{Fringe visibility vs.\ which-way coupling in SST.
The parameter $\Gamma$ encodes EM impedance to the $\mathcal R$ phase (path information rate),
    $\tau$ is the transit time.
    Exponential decay of visibility reflects coherence loss prior to recombination.}
\label{fig:VvsGamma}
\end{figure}

\begin{figure}[t]
\centering
\begin{tikzpicture}[>=Latex,scale=1.0]
    % Energy levels
\draw[very thick] (0,0) -- (6,0) node[right] {$\mathcal R$};
\draw[very thick] (0,3) -- (6,3) node[right] {$\mathcal T$};
% Transition arrow
\draw[->,thick] (3,0.15) -- node[right] {$\hbar\omega_0=\Delta E_{\mathcal R\to\mathcal T}$} (3,2.85);
% Driving field
\draw[->,thick,blue!60!black] (1,-0.8) -- (5,-0.8) node[right] {drive $\omega$};
\draw[blue!60!black,decorate,decoration={coil,aspect=0.8,segment length=5pt,amplitude=2pt}] (1.2,-0.8) -- (4.8,-0.8);
% Rabi label
\node at (3,1.5) {$\Omega_R$};
% Damping
\node[gray] at (6.4,1.5) {$\gamma$};
\end{tikzpicture}
\caption{Two-level SST manifold for $\mathcal R \leftrightarrow \mathcal T$ with Rabi drive.
At resonance $\omega=\omega_0$, the knotting probability oscillates at $\Omega_R$ and decays at rate $\gamma$, enabling pump--probe control of collapse within an interferometer.}
\label{fig:RabiRT}
\end{figure}

%===============================
% Boxed executive statement
%===============================
\begin{center}
\fbox{\parbox{0.92\linewidth}{
    \textbf{Boxed Result (SST double slit).}
    An electron is a single swirl-string with two phases: $\mathcal R$ (delocalized ring) for propagation and $\mathcal T$ (knotted soliton) for detection.
    Interference arises from coherent splitting and recombination of $\mathcal R$; localized impacts result from $\mathcal R\!\to\!\mathcal T$ collapse at the screen.
    Which-way monitoring increases the EM impedance, raising the premature knotting rate $\Gamma$; the fringe visibility obeys $\mathcal V=e^{-\Gamma\tau}$ and vanishes in the strong-monitoring limit.}}
\end{center}



\bibliographystyle{unsrt}
\begin{thebibliography}{99}


\bibitem{Kelvin1867}
William Thomson (Lord Kelvin).
On Vortex Atoms.
Proceedings of the Royal Society of Edinburgh, 6:94--105, 1867.
\url{https://zapatopi.net/kelvin/papers/on_vortex_atoms.html}

\bibitem{Batchelor1967}
G.~K. Batchelor.
An Introduction to Fluid Dynamics.
Cambridge University Press, 1967.
doi:10.1017/CBO9780511800955

\bibitem{Saffman1992}
P.~G. Saffman.
Vortex Dynamics.
Cambridge University Press, 1992.
doi:10.1017/CBO9780511624063

\bibitem{Madelung1927}
E.~Madelung.
Quantentheorie in hydrodynamischer Form.
Zeitschrift f{\"u}r Physik, 40:322--326, 1927.
doi:10.1007/BF01400372

\bibitem{Fetter1967}
A.~L. Fetter.
Quantum Theory of Superfluid Vortices. I. Liquid Helium II.
Physical Review, 162:143--153, 1967.
doi:10.1103/PhysRev.162.143


\bibitem{CorradaEmmanuel1992}
Andr{\'e}s Corrada-Emmanuel.
Algebraic topology and the quantization of circulation in superfluid helium.
Physical Review B, 45:2553--2556, 1992.
doi:10.1103/PhysRevB.45.2553

% Existing entries below (do not duplicate)
\bibitem{deBroglie1925}
L.~de Broglie.
Recherches sur la th{\'e}orie des quanta.
Annales de Physique, 3(10):22--128, 1925.
doi:10.1051/anphys/192510030022.

\bibitem{Bohm1952a}
D.~Bohm.
A Suggested Interpretation of the Quantum Theory in Terms of ``Hidden'' Variables I.
Phys. Rev., 85(2):166--179, 1952.
doi:10.1103/PhysRev.85.166.

\bibitem{Bohm1952b}
D.~Bohm.
A Suggested Interpretation of the Quantum Theory in Terms of ``Hidden'' Variables II.
Phys. Rev., 85(2):180--193, 1952.
doi:10.1103/PhysRev.85.180.

\bibitem{Onsager1949}
L.~Onsager.
Statistical hydrodynamics.
Il Nuovo Cimento (Supplemento), 6:279--287, 1949.
doi:10.1007/BF02780991.

\bibitem{Feynman1955}
R.~P. Feynman.
Application of Quantum Mechanics to Liquid Helium.
In C.~J. Gorter, editor, Progress in Low Temperature Physics, Vol. I, pages 17--53. North-Holland, 1955.
doi:10.1016/S0079-6417(08)60077-3.

\bibitem{Helmholtz1858}
H.~von Helmholtz.
{\"U}ber Integrale der hydrodynamischen Gleichungen, welche den Wirbelbewegungen entsprechen.
Journal f{\"u}r die reine und angewandte Mathematik, 55:25--55, 1858.
doi:10.1515/crll.1858.55.25.

\bibitem{Kelvin1869}
W.~Thomson (Lord Kelvin).
On Vortex Motion.
Transactions of the Royal Society of Edinburgh, 25:217--260, 1869.

\bibitem{Einstein1905}
A.~Einstein.
On a Heuristic Viewpoint Concerning the Production and Transformation of Light.
Annalen der Physik, 1905.

\bibitem{Millikan1916}
R.~A. Millikan.
A Direct Photoelectric Determination of Planck's h.
Physical Review, 7:355--388, 1916.
doi:10.1103/PhysRev.7.355.

\bibitem{Compton1923}
A.~H. Compton.
A Quantum Theory of the Scattering of X-Rays by Light Elements.
Physical Review, 21:483--502, 1923.
doi:10.1103/PhysRev.21.483.

\bibitem{KleinNishina1929}
O.~Klein and Y.~Nishina.
{\"U}ber die Streuung von Strahlung durch freie Elektronen nach der neuen relativistischen Quantendynamik von Dirac.
Zeitschrift f{\"u}r Physik, 52:853--868, 1929.
doi:10.1007/BF01366453.

\bibitem{Jackson1999}
J.~D. Jackson.
Classical Electrodynamics, 3rd Edition.
Wiley, 1999.
ISBN: 9780471309321.

\bibitem{HarocheRaimond2006}
S.~Haroche and J.-M. Raimond.
Exploring the Quantum: Atoms, Cavities, and Photons.
Oxford University Press, 2006.
ISBN: 9780198509141.

\bibitem{ScullyZubairy1997}
M.~O. Scully and M.~S. Zubairy.
Quantum Optics.
Cambridge University Press, 1997.
ISBN: 9780521435956.

\bibitem{Zurek2003}
W.~H. Zurek,
Decoherence, einselection, and the quantum origins of the classical,
\emph{Rev. Mod. Phys.} \textbf{75}(3), 715--775 (2003).
doi:10.1103/RevModPhys.75.715

\bibitem{Wheeler1978}
J.~A. Wheeler,
The `Past' and the `Delayed-Choice' Double-Slit Experiment,
in \emph{Mathematical Foundations of Quantum Theory}, ed. A.~R. Marlow,
Academic Press, New York, 1978, pp.~9--48.

\end{thebibliography}

\end{document}