%! Author = Omar Iskandarani
%! Date = 12/1/2025
%! Affiliation = Independent Researcher, Groningen, The Netherlands
%! License = © 2025 Omar Iskandarani. All rights reserved. This manuscript is made available for academic reading and citation only. No republication, redistribution, or derivative works are permitted without explicit written permission from the author. Contact: info@omariskandarani.com
%! ORCID = 0009-0006-1686-3961
%! DOI = 10.5281/zenodo.xxx

\newcommand{\paperdoi}{10.5281/zenodo.18390279}
\newcommand{\papertitle}{Hydrodynamic Dual-Vacuum Unification:\\ From Swirl Superradiance to Photon Torsion and Atomic Stability}

%=========================================
% % PREAMBLE, PACKAGES AND DOCUMENT CONFIGURATION
%=========================================
\documentclass[11pt,a4paper]{article}
\usepackage{amsmath,amssymb, amsfonts,bm,graphicx}
\usepackage[hidelinks]{hyperref}
\usepackage{cite}
\usepackage[margin=1in]{geometry}

% Macros
\newcommand{\vswirl}{\mathbf{v}_{\!\boldsymbol{\circlearrowleft}}}
\newcommand{\rhof}{\rho_{\!f}}

\title{\bfseries Hydrodynamic Dual-Vacuum Unification:\\
Swirl Superradiance, Photon Torsion, and the Unruh Echo Mechanism}

\author{
        {\large Omar Iskandarani}\\[4pt]
    Independent Researcher, Groningen, The Netherlands\\
    ORCID: 0009-0006-1686-3961\\
    Email: \href{mailto:info@omariskandarani.com}{info@omariskandarani.com}
}
\date{\today}

\begin{document}
    \maketitle

    \begin{abstract}
        We develop a falsifiable hydrodynamic extension of field theory in which the vacuum consists of two interacting, incompressible continua: a fast electromagnetic sector with characteristic propagation speed $c$, and a slower rotational Swirl sector with speed $|\vswirl|\sim 10^{6}\,$m/s.  Using Helmholtz--Kelvin vortex dynamics, we show that acceleration excites vorticity stretching rather than compression, producing a primary Swirl burst with characteristic delay $\tau_S\!\sim\!0.1$\,ns.  Because electromagnetic cavities suppress shear-wave impedance by $\kappa_{se} \sim 10^{-7}$, this burst dissipates non-radiatively, leaving a secondary photon ``Unruh Echo'' at $\sim 30$~ns—the signal observed in recent cavity superradiance experiments.  We formulate a coupled-rate theory of Swirl--EM energy flow, introduce a torsion-modified Maxwell equation, derive a parity-odd linear-in-$B$ vacuum birefringence, and propose a BEC vortex-lattice experiment capable of distinguishing this dual-vacuum picture from standard quantum field theory.
    \end{abstract}

    \tableofcontents

%==============================================================
% I. INTRODUCTION
%==============================================================

    \section{Introduction}

        Quantum Field Theory (QFT) and General Relativity (GR) model the vacuum as a single Lorentz-invariant medium defined by a fixed causal speed $c$.  In contrast, the Swirl String Theory (SST) framework proposes that the vacuum comprises two interacting, incompressible substrates:

        \begin{enumerate}
            \item A fast electromagnetic (EM) sector with characteristic wave speed $c$.
            \item A slower hydrodynamic Swirl sector with rotational wave speed $|\vswirl|\sim 10^{6}$~m/s.
        \end{enumerate}

        This ``Dual-Vacuum Hypothesis'' naturally resolves an outstanding puzzle: recent cavity experiments have observed superradiant delay times $\tau_d\!\sim\!30$–$50$~ns consistent with standard QFT predictions for the Unruh effect~\cite{Deswal2025,Zheng2025}, while providing no evidence of non-standard vacuum excitations.

        We argue that these experiments probe only the photon echo of a deeper, faster, hydrodynamic event.  Because the Swirl sector supports shear/twist modes rather than compressional modes,
        \begin{equation}
            \nabla\cdot\mathbf{v}=0, \qquad \rhof=\text{const.},
            \label{eq:incompress}
        \end{equation}
        acceleration excites {\it vorticity stretching} rather than pressure waves.  The characteristic Unruh-like temperature of this excitation is amplified by the velocity ratio
        \begin{equation}
            \eta = \frac{c}{|\vswirl|} \approx 274.
        \end{equation}
        Thus the Swirl sector responds $\sim 274$ times faster than the EM sector, producing a sub-nanosecond torsional ``Unruh precursor.''  Because the cavity impedance is mismatched for shear waves by $\kappa_{se}\!\sim\!10^{-7}$, this precursor is unobservable, leaving only a secondary photon outburst—the ``Unruh Echo.''

        In this article we (i) formalize this hydrodynamic interpretation, (ii) derive the coupled Swirl--EM rate equations, (iii) introduce a torsion-modified Faraday law, and (iv) propose a vortex-lattice BEC experiment capable of falsifying the model.


%==============================================================
% II. HISTORICAL CONTEXT
%==============================================================

    \section{Historical Context: Helmholtz, Kelvin, and Unruh}

        Helmholtz~\cite{Helmholtz1858} showed that vorticity is materially conserved in an inviscid, incompressible fluid. Kelvin~\cite{Kelvin1867} extended this to vortex filaments, demonstrating that their topology and circulation $\Gamma$ remain invariant. These foundations underpin modern superfluid dynamics and motivate SST's incompressible-vacuum framework.

        Unruh's 1976 prediction~\cite{Unruh1976} that acceleration induces thermal vacuum noise is typically derived assuming a single, massless field propagating at $c$. Recent theoretical proposals~\cite{Deswal2025,Zheng2025} have shown that cavity-enhanced superradiance can amplify this effect into an observable photon burst.

        Our work unifies these ideas: the Unruh effect is reinterpreted as a vorticity-stretching instability in an incompressible rotational substrate, and photon emission is a secondary transduction event.


%==============================================================
% III. VORTICITY INTENSIFICATION IN AN INCOMPRESSIBLE MEDIUM
%==============================================================

    \section{Vorticity Intensification Under Acceleration}

        In incompressible Euler flow, vorticity evolves by the stretching equation:
        \begin{equation}
            \frac{D\boldsymbol{\omega}}{Dt} = (\boldsymbol{\omega}\cdot\nabla)\mathbf{v}.
            \label{eq:stretch}
        \end{equation}
        Acceleration of an embedded vortex filament produces background shear, which stretches the filament, reducing its core area $A_c$ while preserving circulation,
        \begin{equation}
            \Gamma = \int \boldsymbol{\omega}\cdot d\mathbf{S} = \omega A_c = \mathrm{const}.
        \end{equation}
        Thus $|\omega|$ increases as $A_c$ decreases.  The local swirl energy density rises,
        \begin{equation}
            \rho_E = \frac{1}{2}\rhof |\mathbf{v}|^2,
        \end{equation}
        without violating incompressibility.

        This intensification is the hydrodynamic analogue of ``Unruh excitation'' in standard QFT.  Because the Swirl sector's characteristic wave speed is $|\vswirl|\ll c$, the effective Unruh temperature becomes
        \begin{equation}
            T_{\rm SST} = \frac{\hbar a}{2\pi k_B|\vswirl|}
            = \eta\, T_U.
        \end{equation}
        Vorticity amplification therefore occurs on timescales
        \begin{equation}
            \tau_S \sim \frac{|\vswirl|}{c}\,\tau_\gamma \approx 0.1~\text{ns}
        \end{equation}
        for photon-burst delays $\tau_\gamma \sim 30$~ns observed in cavity experiments.


%==============================================================
% IV. DUAL-VACUUM ARCHITECTURE AND IMPEDANCE MISMATCH
%==============================================================

    \section{Dual-Vacuum Architecture: Swirl and EM Sectors}

        The Swirl sector supports torsional (Kelvin-like) waves, while the EM sector supports transverse electromagnetic waves.  Their impedances are
        \begin{equation}
            Z_S = \rhof\,|\vswirl|, \qquad Z_\gamma = Z_{\mathrm{bound}}\sim 10^{7}\ \mathrm{Rayl}.
        \end{equation}
        The Swirl--EM transduction coefficient is
        \begin{equation}
            \kappa_{se}=\frac{4 Z_S}{Z_{\mathrm{bound}}}\sim 10^{-7},
        \end{equation}
        so only one part in $10^7$ of Swirl energy couples to photons.

        This explains why cavity experiments detect only the late electromagnetic burst and not the fast Swirl precursor.


%==============================================================
% V. TORSION-MODIFIED MAXWELL EQUATIONS
%==============================================================

    \section{Torsion-Modified Electrodynamics}

        The electromagnetic field responds to time-varying vorticity via an additional torsional source term.  The modified Faraday law is
        \begin{equation}
            \nabla\times\mathbf{E}
            =-\frac{\partial\mathbf{B}}{\partial t}
            - \mathcal{G}\,\partial_t(\nabla\times\mathbf{v})\,\hat{n},
            \label{eq:torsion}
        \end{equation}
        where $\mathcal{G}$ is a coupling constant and $\hat{n}$ is the swirl-string director.  This produces:

        \begin{trivlist}
            \item (i) A linear-in-$B$ vacuum birefringence, distinct from QED's quadratic response.
            \item (ii) Parity-odd optical activity due to the chiral swirl-clock.
        \end{trivlist}


%==============================================================
% VI. COUPLED RATE EQUATIONS
%==============================================================

    \section{Coupled Swirl--Photon Rate Dynamics}

        Let $n_S$ and $n_{EM}$ be Swirl and EM excitation populations. Their evolution is
        \begin{align}
            \dot{n}_S &= - (\Gamma_S + \gamma_{\mathrm{diss}} + \kappa_{se})\, n_S, \\
            \dot{n}_{EM} &= \kappa_{se}\, n_S - \Gamma_{EM}\, n_{EM}.
        \end{align}
        Here
        \begin{itemize}
            \item $\Gamma_S = \eta\, \Gamma_{GR}$ is the Swirl superradiant rate.
            \item $\Gamma_{EM}$ is the cavity-enhanced photon decay rate.
            \item $\gamma_{\mathrm{diss}}$ encodes non-radiative Swirl damping.
        \end{itemize}

        Solving yields
        \begin{equation}
            n_{EM}(t)=
            \frac{\kappa_{se}}{\Gamma_{EM}-\Gamma_S'}
            \left(
                e^{-\Gamma_S' t}-e^{-\Gamma_{EM}t}
            \right),
        \end{equation}
        where $\Gamma_S' = \Gamma_S + \kappa_{se} + \gamma_{\mathrm{diss}}$.

        The rising edge forms a ``prethermalization plateau,'' while the EM peak at $t\!\sim\!1/\Gamma_{EM}\approx 30$~ns matches experimental data~\cite{Deswal2025}.


%==============================================================
% VII. EXPERIMENTAL PREDICTIONS
%==============================================================

    \section{Experimental Predictions}

        \begin{enumerate}
            \item \textbf{Dual-burst emission.}
            A Swirl precursor at $\sim 0.1$~ns followed by a photon echo at $\sim 30$~ns.

            \item \textbf{Impedance sensitivity.}
            Echo amplitude scales as $\kappa_{se}\propto Z_S/Z_{\mathrm{bound}}$.

            \item \textbf{Vacuum birefringence.}
            Linear-in-$B$ optical rotation due to torsion term (\ref{eq:torsion}).

            \item \textbf{Medium dependence.}
            Unruh temperature scales as $T_{\rm SST} \propto 1/|\vswirl|$.

            \item \textbf{Absence of density waves.}
            No acoustic emission whatsoever; signal is purely vorticity-based.

        \end{enumerate}


%==============================================================
% VIII. FALSIFICATION: BEC VORTEX LATTICE EXPERIMENT
%==============================================================

    \section{Falsification Experiment: Accelerated Vortex-Lattice BEC}

        A vortex-lattice Bose--Einstein condensate responds to torsional disturbances through Kelvin modes~\cite{Fetter2009}.  Accelerating such a lattice should excite a fast torsional precursor.  SST predicts a sharp Kelvin-mode pulse at $\sim 0.1$~ns; QFT predicts no such pulse.

        A null detection rules out hydrodynamic dual-vacuum theories.
        A positive detection falsifies the standard view of vacuum structure.


    \section{The SST Rosetta Stone Summary Table}

        \begin{table}[h]
            \centering
            \caption{The SST Rosetta Stone: Mapping Fluid Variables to Standard Physics}
            \label{tab:rosetta}
            \resizebox{\textwidth}{!}{%
                \begin{tabular}{lcll}
                    \toprule
                    \textbf{Standard Constant} & \textbf{Symbol} & \textbf{SST Hydrodynamic Derivation} & \textbf{Physical Interpretation} \\
                    \midrule
                    \textbf{Fine Structure} & $\alpha$ & $2 \mathbf{v}_{\circlearrowleft} / c$ & Vacuum Mach Number (Stability Limit) \\
                    \textbf{Electron Mass} & $m_e$ & $\frac{\rho_{\text{core}} \Gamma_0^2 r_c}{2\pi c^2}$ & Mass of Minimal Vortex Loop ($L=2r_c$) \\
                    \textbf{Gravitational Const.} & $G$ & $\lambda_G \frac{\Gamma_0^2}{\rho_f r_c^4}$ & Diluted Vortex Tension ($\lambda_G \approx 10^{-60}$) \\
                    \textbf{Planck's Constant} & $\hbar$ & $\sim \rho_{\text{core}} r_c^3 \Gamma_0$ & Angular Momentum of Vortex Core \\
                    \textbf{Rydberg Constant} & $R_\infty$ & $\frac{2 m_e \mathbf{v}_{\circlearrowleft}^2}{h c}$ & Kinetic Energy of Vortex at Mach Limit \\
                    \textbf{Compton Wavelength} & $\lambda_c$ & $\frac{2\pi r_c c}{\mathbf{v}_{\circlearrowleft}}$ & Helical Pitch of Vortex Trajectory \\
                    \textbf{Magnetic Flux} & $\Phi_0$ & $\propto \Gamma_0$ & Circulation Integral \\
                    \bottomrule
                \end{tabular}%
            }
        \end{table}
%==============================================================
% IX. CONCLUSION
%==============================================================

    \section{Conclusion}

        We have presented a hydrodynamic dual-vacuum theory unifying Swirl and EM excitations through vorticity stretching, cavity impedance, and torsion-modified Maxwell dynamics.  This framework explains the timing and shape of Unruh-superradiant signals observed in accelerated cavities and predicts a fast, currently undetected Swirl precursor.  The proposed BEC vortex-lattice experiment provides a clear, falsifiable test.

    \section*{Acknowledgments}

        The author thanks the analogue gravity, quantum optics, and vortex-dynamics communities for foundational insights.

%==============================================================
% REFERENCES
%==============================================================

        \begin{thebibliography}{99}

            \bibitem{Deswal2025}
            A. Deswal, N. Arya, K. Lochan, S. K. Goyal,
            ``Time-Resolved and Superradiantly Amplified Unruh Effect,''
            Phys. Rev. Lett. {\bf 135}, 183601 (2025).

            \bibitem{Zheng2025}
            H. Zheng \emph{et al.},
            ``Enhancing Analog Unruh Effect via Superradiance,''
            Phys. Rev. Research {\bf 7}, 013027 (2025).

            \bibitem{Unruh1976}
            W. G. Unruh,
            ``Notes on black-hole evaporation,''
            Phys. Rev. D {\bf 14}, 870 (1976).

            \bibitem{Helmholtz1858}
            H. von Helmholtz,
            ``Über Integrale der hydrodynamischen Gleichungen,''
            J. Reine Angew. Math. {\bf 55}, 25 (1858).

            \bibitem{Kelvin1867}
            W. Thomson (Lord Kelvin),
            ``On Vortex Motion,''
            Trans. R. Soc. Edinburgh {\bf 25}, 217 (1867).

            \bibitem{GrossHaroche}
            M. Gross and S. Haroche,
            ``Superradiance,''
            Phys. Rep. {\bf 93}, 301 (1982).

            \bibitem{Fetter2009}
            A. L. Fetter,
            ``Rotating Trapped Bose-Einstein Condensates,''
            Rev. Mod. Phys. {\bf 81}, 647 (2009).

            \bibitem{Weinfurtner2011}
            S. Weinfurtner \emph{et al.},
            ``Measurement of Stimulated Hawking Emission in an Analogue System,''
            Phys. Rev. Lett. {\bf 106}, 021302 (2011).

        \end{thebibliography}

\end{document}