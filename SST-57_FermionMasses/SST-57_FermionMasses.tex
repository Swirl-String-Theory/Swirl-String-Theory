%==============================================================================
% SST-57_FermionMasses.tex
%==============================================================================
\documentclass[11pt,a4paper]{article}

%--- Geometry & Layout ---
\usepackage[margin=1in]{geometry}
\usepackage[parfill]{parskip}

%--- Typography & Math ---
\usepackage[T1]{fontenc}
\usepackage[utf8]{inputenc}
\usepackage{newtxtext,newtxmath}
\usepackage{microtype}
\usepackage{amsmath,amssymb,bm}
\usepackage{physics}
\usepackage{siunitx}

%--- Figures & Tables ---
\usepackage{graphicx}
\usepackage{booktabs}

%--- Links ---
\usepackage[colorlinks=true,
    linkcolor=blue!40!black,
    citecolor=green!40!black,
    urlcolor=blue!40!black]{hyperref}

%==============================================================================
% Title & Metadata
%==============================================================================
\title{\textbf{Fermion Masses as Localized Energy Functionals:\\
A Geometric Interpretation of Higgs--Lepton Couplings}}

\author{Omar Iskandarani\\
\small Independent Researcher, Groningen, The Netherlands}

\date{\today}

%==============================================================================
\begin{document}
    \maketitle

%==============================================================================
    \begin{abstract}
        Recent evidence for the Higgs boson decay into a muon pair provides a direct probe
        of the Higgs coupling to second--generation charged leptons.
        In the Standard Model, such couplings are parametrized by Yukawa constants whose
        origin and hierarchical structure remain unexplained.
        In this work, we propose an alternative but fully compatible interpretation in which
        fermion masses arise from localized energy functionals associated with stable
        fermionic configurations, and Higgs couplings are identified with scalar
        susceptibilities of these configurations.
        Within this framework, Yukawa couplings scale automatically with fermion mass,
        independently of microscopic details.
        We show that this interpretation naturally accommodates the observed Higgs--muon
        coupling strength, reproduces charged--lepton mass hierarchies, and yields
        testable constraints for future precision Higgs measurements.
        The construction is purely effective, introduces no new particles or interactions,
        and remains fully consistent with the Standard Model at presently accessible energies.
    \end{abstract}

%==============================================================================
    \section{Introduction}

        The discovery of the Higgs boson completed the particle content of the Standard Model
        (SM) and provided a mechanism for fermion mass generation via Yukawa interactions.
        Despite its empirical success, the Yukawa sector remains largely phenomenological:
        fermion masses and couplings span several orders of magnitude without an underlying
        structural explanation.

        Recent experimental progress has opened a new window into this sector.
        In particular, the observation of the decay
        \(
        H \to \mu^+ \mu^-
        \)
        provides the first direct probe of the Higgs coupling to a second--generation charged
        lepton.
        The measured rate is compatible with Standard Model expectations within current
        uncertainties, reinforcing the approximate proportionality between fermion mass and
        Higgs coupling strength.

        This development motivates a reconsideration of the physical meaning of Yukawa
        couplings.
        Rather than treating them as fundamental parameters, one may ask whether they admit
        an effective interpretation in terms of more primitive, universal quantities.

        In this paper, we explore a conservative possibility:
        fermion masses are modeled as localized energy functionals associated with stable
        fermionic configurations, and Higgs couplings are identified with the response of
        these energies to a scalar background field.
        This viewpoint does not modify the Standard Model Lagrangian, but reinterprets its
        parameters in a geometric and energetic language.

%==============================================================================
    \section{Localized Energy Picture of Fermion Mass}

        We begin by adopting a general and model--independent definition of fermion mass.
        For a fermionic excitation labeled by \(f\), we write its rest energy as
        \begin{equation}
            M_f c^2
            \equiv
            \int_{V_f} \mathcal{E}_f(\mathbf{x})\, d^3x ,
            \label{eq:mass_integral}
        \end{equation}
        where \(\mathcal{E}_f(\mathbf{x})\) is an effective localized energy density and
        \(V_f\) denotes the spatial region over which the excitation is supported.

        Equation~\eqref{eq:mass_integral} is intentionally agnostic about microscopic
        structure.
        It merely encodes the assumption that fermion mass corresponds to energy stored in
        a localized, stable configuration.
        Such a viewpoint is familiar from solitonic models, bag models, and effective
        descriptions of bound states.

        For later convenience, we factorize the mass as
        \begin{equation}
            M_f = \mathcal{K}\,\mathcal{G}_f ,
            \label{eq:mass_factorization}
        \end{equation}
        where \(\mathcal{K}\) is a universal energy--density scale, common to all fermions,
        and \(\mathcal{G}_f\) is a dimensionless configuration factor encoding geometry,
        internal structure, and coherence properties of the fermionic state.

        All fermion mass hierarchies are therefore attributed to differences in
        \(\mathcal{G}_f\).

%==============================================================================
    \section{Scalar Susceptibility Interpretation of Yukawa Couplings}

        In the Standard Model, fermion masses arise through Yukawa interactions with the
        Higgs field \(H\),
        \begin{equation}
            \mathcal{L}_Y = - y_f\, \bar{\psi}_f \psi_f H .
        \end{equation}

        We propose to reinterpret the Yukawa coupling \(y_f\) as a scalar susceptibility:
        \begin{equation}
            y_f \;\equiv\; \frac{\partial M_f}{\partial \Phi},
            \label{eq:yukawa_susceptibility}
        \end{equation}
        where \(\Phi\) is a scalar background field identified, at low energies, with the
        Higgs vacuum expectation value.

        If the scalar field rescales the underlying energy density entering
        Eq.~\eqref{eq:mass_integral}, then
        \begin{equation}
            \mathcal{E}_f(\mathbf{x}) \rightarrow
            \mathcal{E}_f(\mathbf{x})\,[1 + \epsilon\,\Phi],
        \end{equation}
        and one finds
        \begin{equation}
            y_f \propto M_f .
            \label{eq:linear_scaling}
        \end{equation}

        Thus, the observed proportionality between fermion mass and Higgs coupling emerges
        as a generic consequence of energy localization, without requiring fermion--specific
        fundamental couplings.

%==============================================================================
    \section{Charged--Lepton Mass Hierarchy}

        Applying Eq.~\eqref{eq:mass_factorization} to the charged leptons
        \(
        f \in \{e,\mu,\tau\}
        \),
        we write
        \begin{equation}
            M_\ell = \mathcal{K}\,\mathcal{G}_\ell .
        \end{equation}

        Using experimental masses, one infers the dimensionless hierarchy
        \begin{equation}
            \mathcal{G}_e \ll \mathcal{G}_\mu \ll \mathcal{G}_\tau .
        \end{equation}

        Importantly, no assumption is made regarding the microscopic origin of
        \(\mathcal{G}_\ell\).
        It may reflect differences in effective support volume, internal coherence, or other
        geometric characteristics of the fermionic configuration.

        Under the susceptibility interpretation~\eqref{eq:yukawa_susceptibility}, the same
        factors \(\mathcal{G}_\ell\) govern the Higgs couplings, leading directly to
        \begin{equation}
            \frac{y_\mu}{y_e} \approx \frac{M_\mu}{M_e},
            \qquad
            \frac{y_\tau}{y_\mu} \approx \frac{M_\tau}{M_\mu}.
        \end{equation}

        This behavior is precisely what is being tested by Higgs decay measurements.

%==============================================================================
    \section{Implications of Higgs--Muon Measurements}

        The recent observation of the decay \(H \to \mu^+ \mu^-\) indicates that the Higgs
        coupling to the muon is consistent with Standard Model expectations within current
        uncertainties.
        In the present framework, this result implies that the scalar susceptibility of the
        muonic configuration is proportional to its localized energy, with no anomalous
        generation--dependent suppression or enhancement.

        This provides a nontrivial consistency check.
        Any effective theory in which fermion masses arise from localized energy structures
        must reproduce the near--linear scaling of scalar couplings with mass.
        Conversely, future deviations from this scaling would signal either a breakdown of
        the localization picture or the presence of additional dynamical structure.

%==============================================================================
    \section{Predictions and Falsifiability}

        The proposed interpretation yields several testable consequences:

        \begin{itemize}
            \item Precision measurements of \(H \to \tau^+ \tau^-\) should continue to follow
            mass--proportional scaling, up to radiative corrections.
            \item Any significant deviation from linear scaling in Higgs--fermion couplings would
            indicate generation--dependent scalar responses.
            \item The framework predicts no new particles or interactions; deviations must appear
            only as modifications of effective couplings.
        \end{itemize}

        These predictions are directly accessible to ongoing and future collider experiments.

%==============================================================================
    \section{Conclusion}

        We have presented an effective and fully orthodox reinterpretation of fermion masses
        and Higgs couplings in terms of localized energy functionals and scalar
        susceptibilities.
        Within this framework, Yukawa couplings emerge as responses of fermionic energy
        configurations to a scalar background, naturally explaining their proportionality to
        fermion masses.

        The recent observation of Higgs decay into muon pairs provides the first experimental
        anchor for this picture beyond the third generation.
        Future precision measurements will further test whether fermion masses and scalar
        couplings share a common geometric origin.

        This work does not modify the Standard Model, but offers a complementary perspective
        on its parameter structure, potentially guiding the search for a deeper underlying
        theory.

%==============================================================================
        \begin{thebibliography}{9}

            \bibitem{ATLAS_Hmumu}
            ATLAS Collaboration,
            \textit{Evidence for the Dimuon Decay of the Higgs Boson in pp Collisions with the ATLAS Detector},
            Phys.\ Rev.\ Lett.\ (2025).

            \bibitem{PDG}
            Particle Data Group,
            \textit{Review of Particle Physics},
            Prog.\ Theor.\ Exp.\ Phys.\ (annual update).

            \bibitem{Higgs1964}
            P.~W.~Higgs,
            \textit{Broken Symmetries and the Masses of Gauge Bosons},
            Phys.\ Rev.\ Lett.\ \textbf{13}, 508 (1964).

        \end{thebibliography}

\end{document}