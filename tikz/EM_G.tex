\documentclass[11pt,a4paper]{article}
\usepackage[utf8]{inputenc}
\usepackage[T1]{fontenc}
\usepackage[margin=2.2cm]{geometry}
\usepackage{amsmath,amssymb,amsfonts,bm}
\usepackage{hyperref}
\usepackage{tikz}
\usetikzlibrary{arrows.meta,positioning,calc,fit,decorations.pathmorphing}
\usepackage{cite}

\newcommand{\swirlarrow}{%
    \mathchoice{\mkern-2mu\scriptstyle\boldsymbol{\circlearrowleft}}%
    {\mkern-2mu\scriptstyle\boldsymbol{\circlearrowleft}}%
    {\mkern-2mu\scriptscriptstyle\boldsymbol{\circlearrowleft}}%
    {\mkern-2mu\scriptscriptstyle\boldsymbol{\circlearrowleft}}%
}

\title{Rotating--Frame Unification in the SST Canon: \\
From Swirl Density to Swirl--EMF, and a Canonical Derivation of the Coupling $\mathcal G_{\rm \swirlarrow}$}
\author{Omar Iskandarani}
\date{\today}

\begin{document}
\maketitle

\begin{abstract}
We derive, directly from the Swirl--String Theory (SST) Canon, a rotating--frame unification in which centrifugal and gravitational (swirl) contributions merge into a single effective source term that modifies Faraday's law in matter. The core objects are the swirl (vortex-line) areal density $\bm{\varrho}_{\rm \swirlarrow}$ and a swirl-induced electromotive source $\mathbf b_{\rm \swirlarrow}$ that appears in the curl equation of $\mathbf E$. We prove the canonical relation
\[
    \boxed{\ \nabla\times\mathbf E \;=\; -\,\partial_t\mathbf B \;-\; \mathbf b_{\rm \swirlarrow}, \qquad
    \mathbf b_{\rm \swirlarrow} \;=\; \mathcal G_{\rm \swirlarrow}\;\partial_t\bm{\varrho}_{\rm \swirlarrow}\ }
\]
with $\mathcal G_{\rm \swirlarrow}$ a material/topological transduction constant. Using SST ``electron logic'' (the toroidal ring phase $\mathcal R$), circulation quantization, and a flux-pumping pillbox argument, we show that $\mathcal G_{\rm \swirlarrow}$ is naturally quantized in Weber units and, under minimal assumptions, is fixed by a single-flux normalization $\mathcal G_{\rm \swirlarrow}\simeq \Phi_\star$ with $\Phi_\star$ a flux quantum (\emph{a priori} $h/e$; in superconducting analogues $h/2e$)~\cite{Aharonov1959,Tinkham2004,Onsager1949,Feynman1955}. We give a rotating-frame derivation, dimensional checks, and experimental predictions (EMF spikes coincident with vortex nucleation during plate-area compression; integrated EMF $\simeq \Phi_\star \Delta N$).
\end{abstract}

\section{Canonical objects and rotating foliation}
    SST adopts absolute time $t$ and Euclidean space on leaves $\Sigma_t$, with a preferred congruence $u^\mu$ orthogonal to $\Sigma_t$. The Canon’s chronos--Kelvin invariant enforces conservation of circulation at fixed topology,
    \begin{equation}
    \frac{D}{Dt}\Big(R^2\omega\Big)=0
    \quad\Longrightarrow\quad
    \Gamma\equiv\oint_{\mathcal C}\mathbf v\!\cdot d\boldsymbol\ell \;=\; N\,\kappa,\qquad N\in\mathbb Z,
    \label{eq:Kelvin}
    \end{equation}
    where $\kappa$ is the circulation quantum. Coarse graining over an area $A\subset\Sigma_t$ defines the \emph{swirl (vortex-line) areal density vector}
    \begin{equation}
    \bm{\varrho}_{\rm \swirlarrow}(\mathbf x,t) \;\equiv\; n_v(\mathbf x,t)\,\hat{\mathbf n} \;=\; \frac{1}{A}\sum_{\ell\in A}\hat{\mathbf t}_\ell,
    \qquad [\bm{\varrho}]={\rm m^{-2}},
    \label{eq:rho_def}
    \end{equation}
    whose flux counts vortex lines through $A$:
    \begin{equation}
    \Phi_{\rm \swirlarrow}(t;A)=\int_A \bm{\varrho}_{\rm \swirlarrow}\!\cdot d\mathbf A = N(A,t).
    \label{eq:swflux}
    \end{equation}

    \paragraph{Rotating frame merger.}
        In a frame rotating with angular velocity $\boldsymbol\Omega$, the standard decomposition of absolute vorticity $\boldsymbol\zeta_a=\boldsymbol\zeta_r+2\boldsymbol\Omega$ and the effective gravity $\mathbf g_{\rm eff}=\mathbf g-\boldsymbol\Omega\times(\boldsymbol\Omega\times\mathbf r)$ imply that centrifugal and gravitational contributions enter through \emph{one} potential. In SST, this translates to a long-range \emph{swirl gravity} channel: time-varying $\bm{\varrho}_{\rm \swirlarrow}$ couples to electromotive response via a single effective source $\mathbf b_{\rm \swirlarrow}$, i.e. the ``centrifugal\,+\,gravity'' merger manifests as
        \begin{equation}
        \nabla\times\mathbf E = -\,\partial_t\mathbf B - \mathbf b_{\rm \swirlarrow},
        \qquad \mathbf b_{\rm \swirlarrow}=\text{(long-range response to }\partial_t \bm{\varrho}_{\rm \swirlarrow}\text{)}.
        \label{eq:Faraday_mod}
        \end{equation}

\section{Constitutive closure in matter (local tier)}
At laboratory scales we assume two local, linear constitutive maps:
\begin{align}
\mathbf D &= \varepsilon\,\mathbf E, \qquad \mathbf B = \mu\,\mathbf H, \\
\bm{\varrho}_{\rm \swirlarrow} &= \chi_H \,\mathbf H, \qquad [\chi_H] = {\rm m^{-1}A^{-1}},
\label{eq:local_const}
\end{align}
where $\chi_H$ is a \emph{swirl susceptibility}: stronger $\mathbf H$ aligns/admits more vortex lines per area in the medium. This is the right-hand (magnetic/swirl) mirror of Ohm’s law on the left (electric/conduction) side,
\begin{equation}
\mathbf j = \sigma\,\mathbf E,
\qquad [\sigma]={\rm S\,m^{-1}}.
\end{equation}

\section{Pillbox theorem and the mixed topological coupling}
Integrate~\eqref{eq:Faraday_mod} over a surface $S\subset \Sigma_t$ with boundary $\partial S$ and time interval $[t_i,t_f]$:
\begin{align}
\int_{t_i}^{t_f}\!\!\oint_{\partial S}\mathbf E\!\cdot d\boldsymbol\ell\,dt
&= -\,\Delta\Phi_B(S) \;-\; \int_{t_i}^{t_f}\!\!\int_S \mathbf b_{\rm \swirlarrow}\!\cdot d\mathbf A\,dt.
\label{eq:pillbox1}
\end{align}
If the magnetic flux is held fixed ($\Delta\Phi_B=0$), the \emph{time-integrated EMF} equals minus the spacetime integral of $\mathbf b_{\rm \swirlarrow}$.

Now, by definition~\eqref{eq:swflux} the rate of change of swirl flux counts vortex nucleations/escapes through $S$:
\begin{equation}
\frac{d}{dt}\!\int_S \bm{\varrho}_{\rm \swirlarrow}\!\cdot d\mathbf A \;=\; \dot N(S,t).
\label{eq:count}
\end{equation}
Postulate the \emph{mixed topological coupling} (EFT level)
\begin{equation}
\boxed{\ \mathbf b_{\rm \swirlarrow} \;=\; \mathcal G_{\rm \swirlarrow}\,\partial_t\bm{\varrho}_{\rm \swirlarrow}\ },
\qquad [\mathcal G_{\rm \swirlarrow}]={\rm V\,s}= {\rm Wb},
\label{eq:mix}
\end{equation}
which is the unique linear, local-in-time map that (i) respects units ($\mathrm{V\,m^{-2}}$ on both sides of~\eqref{eq:Faraday_mod}), (ii) vanishes in steady states, and (iii) couples only to \emph{topological} changes (nucleations/reconnections) via~\eqref{eq:count}.

Inserting~\eqref{eq:mix} into~\eqref{eq:pillbox1} and using~\eqref{eq:count} gives the \emph{flux-pumping quantization}:
\begin{equation}
\boxed{\ \int_{t_i}^{t_f}\!\!\oint_{\partial S}\mathbf E\!\cdot d\boldsymbol\ell\,dt
\;=\; -\,\mathcal G_{\rm \swirlarrow}\, \Delta N(S)\ },
\qquad \Delta N(S)=\int_{t_i}^{t_f}\!\dot N(S,t)\,dt \in \mathbb Z.
\label{eq:pumping}
\end{equation}
Thus each net vortex line added/removed through $S$ produces a \emph{quantized EMF-time impulse} set by $\mathcal G_{\rm \swirlarrow}$.

\section{Electron logic: canonical normalization of $\mathcal G$}
SST models the electron in its propagation phase as a toroidal ring $\mathcal R$ with tangential speed fixed by the Canon,
\begin{equation}
\|\mathbf v_{\swirlarrow}\| \equiv C_e \approx 1.09384563\times 10^6~{\rm m\,s^{-1}},
\qquad r_c \approx 1.40897\times 10^{-15}~{\rm m},
\label{eq:canonCe}
\end{equation}
and core cross-section $A_c=\pi r_c^2$. When $\mathcal R$ knots ($\mathcal T$) or unknots, the swirl topology changes by $\Delta N=\pm 1$. The ring guides electromagnetic phase around its core; a minimal and natural normalization is to require that \emph{one topological event} corresponds to \emph{one flux impulse} of size $\Phi_\star$:
\begin{equation}
\int_{t_i}^{t_f}\!\!\oint_{\partial S_c}\mathbf E\!\cdot d\boldsymbol\ell\,dt
\stackrel{!}{=}\; \Phi_\star \,\Delta N,
\qquad S_c \sim \text{core disk}.
\label{eq:normalization}
\end{equation}
Comparing with~\eqref{eq:pumping} fixes
\begin{equation}
\boxed{\ \mathcal G_{\rm \swirlarrow} = \Phi_\star \ },
\label{eq:Gsw_fluxq}
\end{equation}
i.e. the swirl--EMF transduction constant equals a \emph{flux quantum}. For single-charged rings the Aharonov--Bohm quantum suggests $\Phi_\star=h/e$~\cite{Aharonov1959}; for Cooper-paired media, $\Phi_\star=h/2e$~\cite{Tinkham2004}. Which constant is realized is a \emph{material/topology} question; either choice preserves~\eqref{eq:pumping} and yields a falsifiable prediction.

\paragraph{Dimensional and energetic consistency.}
    Equation~\eqref{eq:Gsw_fluxq} gives $[\mathcal G_{\rm \swirlarrow}]={\rm V\,s}$ as required by~\eqref{eq:mix}. Energetically, the EM work per event is
    \(
    W=\int dt\oint \mathbf E\cdot d\boldsymbol\ell\,I_{\rm loop}(t).
    \)
    For weak backaction ($I_{\rm loop}$ set by readout),~\eqref{eq:normalization} predicts an \emph{impulse} independent of drive details—an SST counterpart of flux quantization.

\section{Rotating frame: centrifugal + gravity $\Rightarrow$ $\mathbf b$}
Let the container rotate at $\boldsymbol\Omega$ while the plate area shrinks from $A_0$ to $A$. With swirl flux frozen (disconnected electrodes), flux conservation~\eqref{eq:swflux} implies
\begin{equation}
\bm{\varrho}_{\rm \swirlarrow}(A) = \frac{N \,\hat{\mathbf n}}{A},
\quad a(A)\sim n_v^{-1/2}=\sqrt{\frac{A}{N}},
\end{equation}
and nucleation when $a\lesssim \alpha r_c$. The rate $\partial_t \bm{\varrho}_{\rm \swirlarrow}$ is nonzero during nucleation bursts, and by~\eqref{eq:mix} produces a nonzero $\mathbf b_{\rm \swirlarrow}$. In the rotating foliation, the absolute vorticity merger ensures that centrifugal forcing does not appear as a separate source: its effect is absorbed into the \emph{long-range} channel represented by $\mathbf b_{\rm \swirlarrow}$. Combining these, we obtain the \emph{two-tier symmetry}:

\medskip
\noindent\emph{Local tier (mirror):}
\[
    \boxed{\ \mathbf j=\sigma\mathbf E \quad \leftrightarrow \quad \bm{\varrho}_{\rm \swirlarrow}=\chi_H\mathbf H\ }.
\]
\noindent\emph{Long-range tier (unification):}
\[
    \boxed{\ \partial_t \bm{\varrho}_{\rm \swirlarrow} \xRightarrow[\ \ \mathcal G_{\rm \swirlarrow}\ \ ] = \ \mathbf b_{\rm \swirlarrow},
        \quad \text{centrifugal + gravity merged} }.
\]

%%%%%%%%%%%%%%%%%%%%%%%%%%%%%%%%%%%%%%%%%%%%%%%%%%%%%%%%%%%%%%%%%%%%%%%%%%%%%%%%%%%%%%%%%%%%%%%%%%%%%%%%%%
\section{Complete diagram (with units and the long-range link)}
%%%%%%%%%%%%%%%%%%%%%%%%%%%%%%%%%%%%%%%%%%%%%%%%%%%%%%%%%%%%%%%%%%%%%%%%%%%%%%%%%%%%%%%%%%%%%%
\begin{center}
\begin{tikzpicture}[
    node distance=1.5 and 2.1,
    every node/.style={draw, rounded corners, align=center, minimum height=2, font=\small},
    arrow/.style={-{Latex[length=2]}, thick},
    garrow/.style={-{Latex[length=2]}, thick, dashed}
]
% ---------------- center curl (Faraday) ----------------
\node(Faraday)
{$\nabla \times \mathbf{E} = -\,\partial_t \mathbf{B} \;-\; \mathbf{b}_{\rm \swirlarrow}$\\
\scriptsize $[\nabla\times\mathbf E]=\tfrac{V}{m^{2}},\ [\partial_t\mathbf B]=\tfrac{T}{s}$};

% ---------------- left: field E ----------------
\node[left=of Faraday]  (E)
{$\mathbf{E}$\\ \scriptsize $[\mathbf E]=\tfrac{V}{m}$};

% ---------------- right: swirl density rho ----------------
\node[right=of Faraday] (rho)
{\scriptsize swirl density $\bm{\varrho}_{\rm \swirlarrow}$\\ $[\bm{\varrho}_{\rm \swirlarrow}]=\tfrac{1}{m^{2}}$};

\node[below left=0.25 and 0 of rho] (b)
{
    \scriptsize $\mathbf b_{\rm \swirlarrow}=\mathcal G_{\rm \swirlarrow}\,\partial_t\bm{\varrho}_{\rm \swirlarrow}$\\
\scriptsize $[\mathbf b_{\rm \swirlarrow}]=\tfrac{V}{m^{2}}$
};

% ---------------- right middle: susceptibility ----------------
\node[below=of rho] (C)
{$\chi_H\,\mathbf{H} = \bm{\varrho}_{\rm \swirlarrow}$\\
\scriptsize $[\chi_H]=\tfrac{1}{Am}$};

% ---------------- left middle: conduction accumulation ----------------
\node[below=of E] (Eta)
{$\bm{\eta} = (\mathcal K_E * \mathbf E)$\\
\mathcal $K_E = \varepsilon = \frac{C}{Vm}$};

% ---------------- constitutive B,D ----------------
\node[below left=0.75 and -2 of Faraday] (D)
{$\varepsilon\,\mathbf{E} = \mathbf{D}$\\
\scriptsize $[\varepsilon]=\tfrac{F}{m},\ [\mathbf D]=\tfrac{C}{m^{2}}$};

\node[below right=2.25 and -2 of Faraday] (B)
{$\mathbf{B}=\mu\,\mathbf{H}$\\
\scriptsize $[\mathbf B]=T,\ [\mu]=\tfrac{N}{A^{2}}$};



% --- bottom-left areal density (now the visible card) ---
\node[below=of Eta] (EtaBottom)
{$\bm{\eta}$\\
\scriptsize $[\bm{\eta}]=\tfrac{C}{m^{2}}$};

% --- flying source card below (derivative step) ---
\node[above right=0.25 and -0.25 of EtaBottom] (Jsrc)
{$\mathcal G_{\rm el}\, \partial_t \bm{\eta} = \mathbf{j}$\\
\scriptsize $\mathcal G_{\rm el}=1~\tfrac{A \, s}{C},\ [\mathbf j]=\tfrac{A}{m^{2}}$};

% ---------------- Ampère curl ----------------
\node[right=of EtaBottom] (Ampere)
{$\mathbf{j} + \partial_t \mathbf{D} = \nabla \times \mathbf{H}$\\
\scriptsize $[\partial_t\mathbf D]=\tfrac{A}{m^{2}},\ [\nabla\times\mathbf H]=\tfrac{A}{m^{2}}$};

% ---------------- right bottom: H ----------------
\node[below=of C] (H)
{$\mathbf{H}$\\ \scriptsize $[\mathbf H]=\tfrac{A}{m}$};


% ---------------- arrows (same geometry as your framework) ----------------
\draw[arrow] (E) -- (D);
\draw[arrow] (C) -- (rho);
%\draw[arrow] (rho) -- (Faraday);
%\draw[arrow] (EtaBottom) -- (Ampere);
\draw[arrow] (H) -- (Ampere);
\draw[arrow] (E) -- (Faraday);
\draw[arrow] (Faraday) -- (B);
\draw[arrow] (Ampere) -- (D);
\draw[arrow] (B) -- (H);
\draw[arrow] (H) -- (C);

% --- replaced Ohm path by the mirrored left ladder ---
\draw[arrow] (E) -- (Eta);          % kernel stage
\draw[arrow] (Eta) -- (EtaBottom);  % field -> areal density (visible bottom card)
\draw[arrow] (EtaBottom) -- (Jsrc); % derivative to source (flying)
\draw[arrow] (Jsrc) -- (Ampere);    % feeds Ampère

% --- long-range mediation on the right stays ---
\draw[arrow] (rho) --  (b);
\draw[arrow] (b) -- (Faraday);


\end{tikzpicture}

\end{center}


\begin{itemize}
    \item $\mathbf{b}_{\rm \swirlarrow}=\mathcal G_{\rm \swirlarrow}\,\partial_t\bm{\varrho}_{\rm \swirlarrow}$: swirl--EMF source, with units $[\mathbf b_{\rm \swirlarrow}]$.
    \item $\bm{\varrho}_{\rm \swirlarrow}$: swirl density, with units $[\bm{\varrho}_{\rm \swirlarrow}]$.
    \item Swirl--gravity mediation: $\mathcal G_{\rm \swirlarrow}=\Phi_\star$.
    \item $\bm{\eta}$: conduction accumulation, with units $[\bm{\eta}]$.
    \item $\mathcal{K}_E$: a constitutive kernel (electric side), mapping the field $\mathbf{E}$ into an areal charge accumulation $\bm{\eta}$. In the simplest (local, isotropic) form:
    \[
        \bm{\eta} = \varepsilon\,\mathbf{E}
    \]
    but written as $(\mathcal{K}_E * \mathbf{E})$, it allows for spatial/temporal nonlocal response (like a susceptibility kernel).
    \item $\chi_H$: a swirl susceptibility (magnetic side), mapping the field $\mathbf{H}$ into the swirl density $\bm{\varrho}_{\rm \swirlarrow}$:
    \[
        \bm{\varrho}_{\rm \swirlarrow} = \chi_H\,\mathbf{H}
    \]
    Units: $[\chi_H] = \mathrm{m^{-1}A^{-1}}$. It plays the same role as an electric or magnetic susceptibility, but in the SST Canon it measures how strongly $\mathbf{H}$ seeds swirl line density.
\end{itemize}









%%%%%%%%%%%%%%%%%%%%%%%%%%%%%%%%%%%%%%%%%%%%%%%%%%%%%%%%%%%%%%%%%%%%%%%%%%%%%%%%%%%%%%%%%%%%%%%%%%%%%%%%%%

%%%%%%%%%%%%%%%%%%%%%%%%%%%%%%%%%%%%%%%%%%%%%%%%%%%%%%%%%%%%%%%%%%%%%%%%%%%%%%%%%%%%%%%%%%%%%%%%%%%%%%%%%%


\section{From the Canon to a value for $\mathcal G$}
Equation~\eqref{eq:Gsw_fluxq} sets the \emph{scale} of $\mathcal G_{\rm \swirlarrow}$; SST “electron logic” refines it:

\paragraph{(i) Topological normalization.}
    The ring $\mathcal R$ carries an integer winding $N$; knotting/unknotting changes $N\!\to\!N\pm1$. A single event thus generates an EMF-time impulse $\Phi_\star$ by~\eqref{eq:pumping}--\eqref{eq:normalization}.

\paragraph{(ii) Energetic matching.}
    The ring’s effective energy change for $\Delta N=\pm1$ is
    \begin{equation}
    \Delta E \simeq (\epsilon_0 A_c + \beta)\,\Delta L + \alpha C(\mathcal T)+\gamma \mathcal H(\mathcal T),
    \end{equation}
    with Canon bulk term $\epsilon_0$ and line/helicity/contact coefficients (as in the SST Lagrangian). A resonant photon of $\hbar\omega_0\!\approx\!\Delta E$ mediates the transition. The EMF impulse $\Phi_\star$ does \emph{no net} work without a readout current; thus energetic matching does not fix $\Phi_\star$—it fixes \emph{rates} (Rabi), while~\eqref{eq:Gsw_fluxq} fixes the \emph{topological size}. This separation is natural in a mixed topological term.

\paragraph{(iii) Choice of $\Phi_\star$.}
    For single-charge matter waves, the Aharonov--Bohm flux quantum $h/e$ is the canonical choice~\cite{Aharonov1959}; in superconducting media, $h/2e$ applies~\cite{Tinkham2004}. Measuring EMF-time impulses during controlled vortex nucleation discriminates these cases.

\section{Predictions \& experimental program}
\begin{itemize}
\item \textbf{Plate compression (levitated PG/electret stack).} With electrodes disconnected (frozen charge), shrink the effective plate area so that the swirl flux cannot escape. Monitor a pickup loop around the active region. Prediction:
\[
    \int dt\,{\rm EMF}(t) \;=\; \Phi_\star\,\Delta N, \quad \Delta N\in\mathbb Z,
\]
with bursts coincident with vortex nucleation (when $a\lesssim\alpha r_c$).
\item \textbf{Rotating frame.} Repeat while ramping $\Omega$. The threshold area $A_\star(\Omega)$ for first nucleation obeys $N/A_\star\simeq 2\Omega/\kappa$ (Feynman relation), and EMF-time impulse remains quantized by $\Phi_\star$.
\item \textbf{Pump--probe control.} A resonant optical pump at $\omega_0$ modulates the nucleation rate $\propto |\partial_t \bm{\varrho}|$; the \emph{integrated} EMF per event remains $\Phi_\star$ (topologically protected), while the \emph{temporal} profile tracks the pump.
\end{itemize}

\section{Boxed summary (SST Canon $\Rightarrow$ diagram)}
\[
    \boxed{
        \begin{gathered}
        \text{(Kelvin/Canon)}\quad \Gamma=\oint \mathbf v\!\cdot d\boldsymbol\ell = N\kappa, \quad
        \Phi_{\rm \swirlarrow}=\int_A \bm{\varrho}\!\cdot d\mathbf A=N \\
        \text{(local mirror)}\quad \mathbf j=\sigma\mathbf E \;\;\leftrightarrow\;\; \bm{\varrho}=\chi_H\mathbf H \\
        \text{(long-range unification)}\quad \nabla\times\mathbf E = -\partial_t\mathbf B - \mathbf b_{\rm \swirlarrow},\quad
        \mathbf b_{\rm \swirlarrow}=\mathcal G_{\rm \swirlarrow}\,\partial_t\bm{\varrho} \\
        \text{(electron normalization)}\quad \mathcal G_{\rm \swirlarrow}=\Phi_\star \in\{h/e,\;h/2e\},\quad
        \int dt\,\oint \mathbf E\!\cdot d\boldsymbol\ell = -\,\Phi_\star\,\Delta N
        \end{gathered}}
\]

\paragraph{Dimensional checks.}
    $[\bm{\varrho}]={\rm m^{-2}}$, $[\partial_t\bm{\varrho}]={\rm m^{-2}s^{-1}}$; $[\mathcal G_{\rm \swirlarrow}]={\rm V\,s}$ so $[\mathbf b_{\rm \swirlarrow}]={\rm V\,m^{-2}}$ matches $[\partial_t\mathbf B]={\rm T\,s^{-1}}={\rm V\,m^{-2}}$; all local maps in~\eqref{eq:local_const} use standard SI.

\section*{Acknowledgement of canonical constants}
Where numerical evaluation is desired, adopt the Canon values $C_e$, $r_c$, $\rho_{\ae}^{\rm core}$, $\rho_{\ae}$ provided in the SST Canon; these enter rate and threshold estimates (via $a\sim \sqrt{A/N}$ and $r_c$), but not the quantized \emph{magnitude} $\Phi_\star$ of the EMF-time impulse.

\bibliographystyle{plain}
\begin{thebibliography}{10}

\bibitem{Aharonov1959}
Y.~Aharonov and D.~Bohm.
\newblock Significance of electromagnetic potentials in the quantum theory.
\newblock {\em Physical Review}, 115(3):485--491, 1959.
\newblock doi:10.1103/PhysRev.115.485.

\bibitem{Tinkham2004}
M.~Tinkham.
\newblock {\em Introduction to Superconductivity}.
\newblock Dover, 2nd edition, 2004.

\bibitem{Onsager1949}
L.~Onsager.
\newblock Statistical hydrodynamics.
\newblock {\em Il Nuovo Cimento (Supplemento)}, 6:279--287, 1949.
\newblock doi:10.1007/BF02780991.

\bibitem{Feynman1955}
R.~P. Feynman.
\newblock Application of quantum mechanics to liquid helium.
\newblock In C.~J. Gorter, editor, {\em Progress in Low Temperature Physics, Vol. I}, pages 17--53. North-Holland, 1955.
\newblock doi:10.1016/S0079-6417(08)60077-3.

\end{thebibliography}

\end{document}
