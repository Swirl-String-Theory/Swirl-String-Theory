%! Author: Omar Iskandarani (SST derivation package)
%! Date: 2025-09-06
%! Canon note: uses standard MHD helicity; neutral notation (no SST macros needed).

\documentclass[11pt]{article}
\usepackage[utf8]{inputenc}
\usepackage[T1]{fontenc}
\usepackage[a4paper,margin=1in]{geometry}
\usepackage{amsmath,amssymb,amsthm,mathtools,physics,bm}
\usepackage{hyperref}

% Theorem environments
\newtheorem{theorem}{Theorem}
\newtheorem{prop}{Proposition}
\newtheorem{lemma}{Lemma}
\newtheorem{defn}{Definition}
\newtheorem{remark}{Remark}

% Notation helpers
\renewcommand{\div}{\nabla\!\cdot\!}
\newcommand{\curl}{\nabla\!\times\!}
\newcommand{\bbR}{\mathbb{R}}
\newcommand{\bbT}{\mathbb{T}}
\newcommand{\ucover}{\widetilde{\Omega}}   % universal cover
\newcommand{\mean}[1]{\langle #1\rangle}

\begin{document}

\section*{Magnetic helicity in periodic domains: gauge conditions, existence of vector potentials, and periodic winding}

    \paragraph{Setting.}
        Let $\Omega\subset\bbR^3$ be either (i) a bounded box with one or two pairs of opposite faces identified (``$1$- or $2$-periodic''), or (ii) the fully periodic $3$-torus $\bbT^3 = \bbR^3/\Lambda$ with lattice $\Lambda$. Let $\bm{B}:\Omega\to\bbR^3$ be a smooth magnetic field with $\div\bm{B}=0$.

\section*{1. Helicity, gauge, and boundary/periodicity conditions}

The (total) magnetic helicity on $\Omega$ is
\begin{equation}
H[\bm{A},\bm{B}] \;=\; \int_{\Omega} \bm{A}\cdot\bm{B}\, \mathrm{d}V,
\qquad \bm{B}=\curl\bm{A}.
\label{eq:HelicityDef}
\end{equation}
Under a gauge transform $\bm{A}\mapsto \bm{A}'=\bm{A}+\nabla\chi$,
\begin{align}
H[\bm{A}',\bm{B}] - H[\bm{A},\bm{B}]
&= \int_{\Omega} \nabla\chi \cdot \bm{B}\, \mathrm{d}V
\;=\; \int_{\Omega} \div(\chi\,\bm{B})\, \mathrm{d}V
\;-\; \int_{\Omega} \chi\,\underbrace{\div\bm{B}}_{0}\,\mathrm{d}V \nonumber\\
&= \int_{\partial\Omega} \chi\, \bm{B}\cdot \mathrm{d}\bm{S}.
\label{eq:GaugeShift}
\end{align}
\noindent Hence helicity is gauge invariant if and only if the boundary flux term vanishes.

\begin{prop}[Gauge invariance in periodic/closed settings]
\label{prop:gauge-invariance}
Suppose either:
\begin{itemize}
\item[(a)] $\Omega$ is bounded with perfectly conducting boundary ($\bm{B}\cdot \bm{n}|_{\partial\Omega}=0$), or
\item[(b)] $\Omega$ is periodic in one or two directions and the \emph{net magnetic flux} through each identified pair of faces is zero; additionally the gauge function $\chi$ respects the same periodicity,
\end{itemize}
then $H$ in \eqref{eq:HelicityDef} is gauge invariant.
\end{prop}

\begin{proof}
In case (a) the surface integral in \eqref{eq:GaugeShift} vanishes since $\bm{B}\cdot\mathrm{d}\bm{S}=0$ on $\partial\Omega$. In case (b), identify opposite faces with orientation; periodic $\chi$ has equal values on paired faces, while zero net flux through each pair implies the oriented surface integrals cancel pairwise. Thus the sum over $\partial\Omega$ is zero.
\end{proof}

\begin{prop}[Obstruction in fully periodic domains]
\label{prop:obstruction}
On $\bbT^3$, a \emph{globally periodic} vector potential $\bm{A}$ with $\curl\bm{A}=\bm{B}$ exists if and only if the mean field vanishes:
\begin{equation}
\mean{\bm{B}} \;\equiv\; \frac{1}{|\Omega|} \int_{\Omega} \bm{B}\,\mathrm{d}V \;=\; \bm{0}.
\label{eq:MeanBZero}
\end{equation}
Equivalently, any nonzero constant (harmonic) component of $\bm{B}$ produces a topological obstruction to a periodic $\bm{A}$.
\end{prop}

\begin{proof}[Sketch]
Fourier decompose on $\bbT^3$. For $k\neq 0$, $\widehat{\bm{A}}(k)$ exists with $ik\times \widehat{\bm{A}}(k)=\widehat{\bm{B}}(k)$. At $k=0$, we would require a constant $\widehat{\bm{A}}(0)$ with $\curl \widehat{\bm{A}}(0)=\bm{0}$ producing $\widehat{\bm{B}}(0)=\bm{0}$. Thus periodic solvability demands $\widehat{\bm{B}}(0)=\mean{\bm{B}}=0$.
\end{proof}

\begin{remark}[Physical dimension]
$[\bm{A}]=\mathrm{V\,s\,m^{-1}}$, $[\bm{B}]=\mathrm{T}=\mathrm{Wb\,m^{-2}}$. Hence $[\bm{A}\cdot\bm{B}]=\mathrm{Wb^2\,m^{-3}}$ and $[H]=\mathrm{Wb^2}$ after integration, as standard.
\end{remark}

\section*{2. Periodic winding and the helicity equivalence (two-periodic case)}

Consider a domain that is periodic in two directions, e.g.\ $\Omega = [0,L_x]\times [0,L_y]\times [0,L_z]$ with identifications in $x,y$ (a $2$-torus in the horizontal plane). Let $\ucover = \bbR^2\times[0,L_z]$ be its universal cover. Field lines of $\bm{B}$ in $\Omega$ lift to curves in $\ucover$.

\begin{defn}[Periodic winding of a pair of field lines]
Let $\gamma_1,\gamma_2$ be two \emph{lifted} field lines in $\ucover$, parameterized by $z\in[0,L_z]$, and let $\bm{r}(z)=\bm{x}_1(z)-\bm{x}_2(z)$ be their horizontal separation (projected to $\bbR^2$ before modding out by periods). Define the \emph{periodic winding increment}
\begin{equation}
\Delta\Theta(\gamma_1,\gamma_2)
\;=\;
\int_{0}^{L_z} \frac{\bm{r}(z)\times \dot{\bm{r}}(z)}{\|\bm{r}(z)\|^2}\cdot \bm{e}_z \,\mathrm{d}z,
\label{eq:PeriodicWindingPair}
\end{equation}
and the \emph{periodic winding} as the flux-weighted mean over pairs of field lines:
\begin{equation}
\mathcal{W}_{\mathrm{per}}
\;=\; \frac{1}{\Phi^2}\iint \Delta\Theta(\gamma_1,\gamma_2)\, \mathrm{d}\Phi(\gamma_1)\,\mathrm{d}\Phi(\gamma_2),
\label{eq:PeriodicWinding}
\end{equation}
where $\Phi$ denotes the magnetic flux measure induced by $\bm{B}$ on a transverse cross-section.
\end{defn}

\begin{remark}
Expression \eqref{eq:PeriodicWindingPair} generalizes the pairwise angular change (linking/winding) to horizontally periodic geometry by working on the universal cover and then averaging in a flux-consistent way.
\end{remark}

\begin{theorem}[Helicity--winding equivalence in two-periodic domains]
\label{thm:equivalence}
Assume: (i) ideal evolution (frozen-in field, sufficiently smooth), (ii) two-directional periodicity with zero net flux through each periodic pair, (iii) a vector potential $\bm{A}$ in a \emph{periodic winding gauge} compatible with the identifications. Then
\begin{equation}
H \;=\; \int_{\Omega} \bm{A}\cdot\bm{B}\,\mathrm{d}V
\;=\; \Phi^2 \, \mathcal{W}_{\mathrm{per}}.
\label{eq:HelicityEqualsPeriodicWinding}
\end{equation}
In particular, the right-hand side defines a gauge-invariant helicity that is conserved under ideal MHD evolution.
\end{theorem}

\begin{proof}[Proof sketch]
One constructs a vector potential by solving $\curl\bm{A}=\bm{B}$ with a gauge condition that fixes the mean horizontal rotational content to coincide with the periodic pairwise winding (the ``periodic winding gauge''). Using the Biot–Savart–type representation adapted to the periodic geometry on $\ucover$, one shows that $\bm{A}\cdot\bm{B}$ integrates to the flux-weighted average of the pairwise angular increments, yielding \eqref{eq:HelicityEqualsPeriodicWinding}. Conservation follows from ideal evolution and the boundary/periodicity assumptions (no helicity flux through identified faces).
\end{proof}

\begin{remark}[Fourier connection]
On horizontally periodic domains, the periodic winding helicity admits a Fourier representation; equivalently, in spectral space the gauge choice aligns the phase relations of $\widehat{\bm{A}}(k)$ with those of $\widehat{\bm{B}}(k)$ to encode winding density, providing computational routes consistent with \eqref{eq:HelicityEqualsPeriodicWinding}.
\end{remark}

\section*{3. Summary of conditions (existence, invariance, equivalence)}
\begin{itemize}
\item \textbf{Gauge invariance:} Eq.~\eqref{eq:GaugeShift} shows $H$ is gauge invariant if (PC) $\bm{B}\!\cdot\!\bm{n}=0$ on $\partial\Omega$, or (Per) zero net flux through each periodic face pair with periodic $\chi$.
\item \textbf{Existence of periodic $\bm{A}$:} On $\bbT^3$, a periodic $\bm{A}$ exists iff $\mean{\bm{B}}=\bm{0}$; otherwise a harmonic (mean) part obstructs periodic potentials.
\item \textbf{Helicity--winding equivalence (two-periodic):} Under assumptions of Theorem~\ref{thm:equivalence}, $H=\Phi^2\mathcal{W}_{\mathrm{per}}$.
\end{itemize}

\section*{4. Notes on dimensions and limiting behavior}
If periodicity is removed and $\Omega$ is simply connected with $\bm{B}\!\cdot\!\bm{n}=0$, Eq.~\eqref{eq:HelicityEqualsPeriodicWinding} reduces to the classical flux-weighted total winding/ linking interpretation of helicity. The dimensionality remains $[H]=\mathrm{Wb^2}$ in all cases.

\section*{5. Rosetta Translation: Magnetic Helicity $\to$ Swirl--String Theory Canon}

\paragraph{Mapping of variables.}
    In Swirl--String Theory (SST), the correspondence is:

    \begin{align*}
    \bm{B} &\;\longmapsto\; \boldsymbol{\omega}
    && \text{(swirl vorticity field, $[\boldsymbol{\omega}]=\mathrm{s}^{-1}$)}, \\
    \bm{A} &\;\longmapsto\; \bm{\Psi}
    && \text{(swirl potential / circulation density, $[\bm{\Psi}]=\mathrm{m}^2\!\,\mathrm{s}^{-1}$)}, \\
    H &= \int_\Omega \bm{A}\cdot \bm{B}\,\mathrm{d}V
    \;\longmapsto\; \mathcal{H}_{\text{swirl}}
    = \int_\Omega \bm{\Psi}\cdot\boldsymbol{\omega}\,\mathrm{d}V, \\
    \mathcal{W}_{\mathrm{per}}
    &\;\longmapsto\; \mathcal{L}_{\circlearrowleft}^{\mathrm{per}}
    && \text{(periodic swirl--linking density)}, \\
    \Phi &= \int \bm{B}\cdot\mathrm{d}\bm{S}
    \;\longmapsto\; \Gamma
    && \text{(circulation quantum, flux of vorticity)} .
    \end{align*}

\paragraph{Interpretation.}
    \begin{itemize}
    \item The gauge invariance conditions (Prop.~\ref{prop:gauge-invariance}) translate to:
    \emph{Swirl helicity $\mathcal{H}_{\text{swirl}}$ is invariant under potential shifts $\bm{\Psi}\mapsto \bm{\Psi}+\nabla\chi$ whenever the net circulation across identified boundaries vanishes.}

    \item The obstruction in $\bbT^3$ (Prop.~\ref{prop:obstruction}) becomes:
    \emph{A global swirl potential $\bm{\Psi}$ exists iff the mean vorticity $\mean{\boldsymbol{\omega}}=0$. Nonzero bias corresponds to an irreducible swirl-clock offset across the periodic foliation.}

    \item The periodic winding theorem (Thm.~\ref{thm:equivalence}) translates to:
    \begin{equation}
    \mathcal{H}_{\text{swirl}}
    = \Gamma^2 \, \mathcal{L}_{\circlearrowleft}^{\mathrm{per}},
    \label{eq:SST-Helicity-Linking}
    \end{equation}
    where $\mathcal{L}_{\circlearrowleft}^{\mathrm{per}}$ measures the pairwise phase-winding of swirl strings in the universal cover of the periodic domain.
    \end{itemize}

\paragraph{Dimensional check.}
    \[
        [\Gamma] = \mathrm{m}^2 \,\mathrm{s}^{-1},
        \quad [\mathcal{L}_{\circlearrowleft}^{\mathrm{per}}]=1,
        \quad [\mathcal{H}_{\text{swirl}}] = \mathrm{m}^4 \,\mathrm{s}^{-2},
    \]
    consistent with a conserved quadratic invariant of the swirl field.

\paragraph{Canonical status.}
    Equation~\eqref{eq:SST-Helicity-Linking} is classified as:
    \begin{center}
    \fbox{\textbf{Theorem (Rosetta)}}
    \end{center}
    \emph{In SST, the conserved swirl helicity $\mathcal{H}_{\text{swirl}}$ equals the square of the fundamental circulation quantum $\Gamma$ multiplied by the periodic swirl-linking density $\mathcal{L}_{\circlearrowleft}^{\mathrm{per}}$.}

\paragraph{Physical picture (analogy).}
    Swirl helicity is the ``knottedness'' of swirl strings. Periodicity forces us to lift the foliation to its universal cover: then, every swirl string traces a path whose \emph{relative winding} with others accumulates across layers. The conserved $\mathcal{H}_{\text{swirl}}$ counts exactly this hidden choreography of interwoven clocks.

\section*{5.1 Numerical validation (SST units, SI)}
We adopt the rigid-swirl estimate for a single core loop:
\[
    \Gamma \;=\; \oint \mathbf{v}\cdot d\boldsymbol{\ell}
    \;\approx\; 2\pi\, r_c\, C_e,
    \quad
    [\Gamma]=\mathrm{m^2\,s^{-1}}.
\]
With your constants
\( C_e = 1.09384563\times 10^{6}\,\mathrm{m\,s^{-1}}\),
\( r_c = 1.40897017\times 10^{-15}\,\mathrm{m}\),
we obtain
\[
    \Gamma \approx 9.683619203\times 10^{-9}\ \mathrm{m^2\,s^{-1}}.
\]
Setting a representative periodic swirl-linking density
\( \mathcal{L}_{\circlearrowleft}^{\mathrm{per}} = 1 \) (dimensionless),
the Rosetta identity
\[
    \mathcal{H}_{\text{swirl}} \;=\; \Gamma^2\, \mathcal{L}_{\circlearrowleft}^{\mathrm{per}}
\]
gives
\[
    \mathcal{H}_{\text{swirl}}
    \;\approx\; 9.377248088\times 10^{-17}\ \mathrm{m^4\,s^{-2}}.
\]
\paragraph{Dimensional check.}
    \([\,\Gamma^2\,]=\mathrm{m^4\,s^{-2}}\) and \([\,\mathcal{L}_{\circlearrowleft}^{\mathrm{per}}\,]=1\),
    so \([\,\mathcal{H}_{\text{swirl}}\,]=\mathrm{m^4\,s^{-2}}\) as required.

\paragraph{Scaling notes.}
    For $N$ coherently linked cores with identical $\Gamma$ and pairwise periodic winding density contributing additively, one expects
    $ \mathcal{H}_{\text{swirl}}\sim \Gamma^2 \sum_{i<j}\mathcal{L}_{ij}^{\mathrm{per}} $,
    showing quadratic growth in the circulation scale and linear growth in the effective pair count.

\section*{5.2 Sensitivity scalings}
Let $r_c \mapsto \lambda r_c$ and $C_e \mapsto \mu C_e$. Then
\[
    \Gamma(\lambda,\mu) = 2\pi (\lambda r_c)(\mu C_e) = (\lambda\mu)\,\Gamma_0,\qquad
    \mathcal{H}_{\text{swirl}}(\lambda,\mu) = \Gamma(\lambda,\mu)^2\,\mathcal{L}_{\circlearrowleft}^{\mathrm{per}}
    = (\lambda\mu)^2\,\Gamma_0^2 \,\mathcal{L}_{\circlearrowleft}^{\mathrm{per}}.
\]
Hence $\partial\ln\mathcal{H}_{\text{swirl}}/\partial\ln\lambda = 2$ and likewise for $\mu$; the invariant is \emph{quadratic} in each scale.

\paragraph{Numerics (with $\mathcal{L}_{\circlearrowleft}^{\mathrm{per}}=1$).}
    Using $C_e=1.09384563\times10^6\,\mathrm{m\,s^{-1}}$, $r_c=1.40897017\times10^{-15}\,\mathrm{m}$, we have
    \[
        \Gamma_0 \approx 9.683619203\times10^{-9}\ \mathrm{m^2\,s^{-1}},
        \quad
        \Gamma_0^2 \approx 9.377248088\times10^{-17}\ \mathrm{m^4\,s^{-2}}.
    \]
    A grid over $\lambda,\mu\in\{0.1,0.5,1,2,5,10\}$ confirms $\mathcal{H}_{\text{swirl}} \propto (\lambda\mu)^2$.

\subsection*{5.3 Multi-core aggregate (uniform pairwise winding)}
For $N$ identical cores with uniform pairwise periodic winding $L_{ij}=1$,
\[
    \mathcal{H}_{\text{swirl}}^{\text{(total)}}
    = \Gamma_0^2 \sum_{i<j} L_{ij}
    = \Gamma_0^2 \frac{N(N-1)}{2}.
\]
This shows linear growth in the number of interacting pairs and quadratic growth in the circulation scale.

\paragraph{Numerics.}
    For $N=30$ and $L_{ij}=1$,
    \[
        \mathcal{H}_{\text{swirl}}^{\text{(total)}}
        = \Gamma_0^2 \frac{30\cdot29}{2}
        \approx 4.078 \times 10^{-14}\ \mathrm{m^4\,s^{-2}}.
    \]

\paragraph{Generalization (non-uniform topology).}
    If $L_{ij}$ depends on geometry (knot type, relative phase, spacing), replace $1$ with the measured/estimated $L_{ij}$:
    \[
        \mathcal{H}_{\text{swirl}}^{\text{(total)}}
        = \Gamma_0^2 \sum_{i<j} L_{ij}.
    \]
    In trefoil or Hopf-linked arrays, $L_{ij}$ can deviate from $1$ due to phase slippage across periods; the periodic-winding construction (§2) provides the correct $L_{ij}$ on the universal cover.


\section*{6. Corollaries (SST extensions beyond original works)}

\begin{corollary}[Numerical grounding of circulation quantum]
With $C_e=1.09384563\times 10^6\,\mathrm{m\,s^{-1}}$ and $r_c=1.40897017\times 10^{-15}\,\mathrm{m}$,
the fundamental circulation quantum is
\[
    \Gamma_0 = 2\pi r_c C_e \;\approx\; 9.68\times10^{-9}\ \mathrm{m^2\,s^{-1}},
\]
yielding a base swirl helicity scale
\(\mathcal{H}_{\text{swirl}} = \Gamma_0^2 \approx 9.38\times10^{-17}\ \mathrm{m^4\,s^{-2}}\).
\end{corollary}

\begin{corollary}[Quadratic scaling in core radius and swirl speed]
Under the rescalings $r_c\mapsto \lambda r_c$, $C_e\mapsto \mu C_e$,
\[
    \mathcal{H}_{\text{swirl}}(\lambda,\mu) = (\lambda\mu)^2\,\Gamma_0^2\,
    \mathcal{L}_{\circlearrowleft}^{\mathrm{per}}.
\]
Thus helicity grows quadratically with either scale, providing explicit control parameters.
\end{corollary}

\begin{corollary}[Multi-core helicity growth]
For $N$ identical cores with uniform pairwise periodic swirl-linking $L_{ij}=1$,
\[
    \mathcal{H}_{\text{swirl}}^{\mathrm{(total)}}
    = \Gamma_0^2\,\frac{N(N-1)}{2}.
\]
This shows quadratic growth in $\Gamma$ and combinatorial growth in $N$, predicting rapid amplification in linked-knot arrays or coil bundles.
\end{corollary}

\begin{corollary}[Canonical SST theorem status]
Equation
\[
    \mathcal{H}_{\text{swirl}} \;=\; \Gamma^2 \,\mathcal{L}_{\circlearrowleft}^{\mathrm{per}}
\]
is elevated from an MHD identity to a \emph{Theorem in SST Canon},
classifying swirl helicity as a conserved invariant of foliation dynamics in periodic domains.
This provides a direct bridge from topological MHD helicity to the SST energy sector
$\rho_{\!E}$ and establishes the Rosetta dictionary (potential $\bm{\Psi}$, vorticity $\boldsymbol{\omega}$, circulation $\Gamma$).
\end{corollary}

\begin{corollary}[Helicity–energy constitutive law for SST experiments]\label{cor:helicity_energy}
There exists a constitutive coupling between conserved swirl helicity and the energy density (swirl pressure) of the foliation,
\begin{equation}
\rho_{\!E}
\;=\;
\kappa_H \,\frac{\rho_{\!f}}{L_H^{2}}\;\mathcal{H}_{\text{swirl}},
\qquad
\mathcal{H}_{\text{swirl}} \;=\; \Gamma^{2}\,\mathcal{L}_{\circlearrowleft}^{\mathrm{per}},
\tag{C5}\label{eq:C5}
\end{equation}
where $\rho_{\!f}$ is the effective (fluid) density, $\kappa_H$ is a dimensionless calibration constant, $L_H$ is a helicity coherence length (the transverse scale over which pairwise periodic winding coherently contributes), and $\mathcal{L}_{\circlearrowleft}^{\mathrm{per}}$ is the periodic swirl-linking density of Sec.~5.
\end{corollary}

\paragraph{Status.}
    \emph{Constitutive (from Canon + calibration).} The identity $\mathcal{H}_{\text{swirl}}=\Gamma^{2}\mathcal{L}_{\circlearrowleft}^{\mathrm{per}}$ is Theorem~(Rosetta). The proportionality to $\rho_{\!E}$ is a phenomenological law constrained by units and validated by experiment/simulation via $\kappa_H$ and $L_H$.

\paragraph{Dimensional audit.}
    $[\rho_{\!f}] = \mathrm{kg\,m^{-3}}$, $[\mathcal{H}_{\text{swirl}}]=\mathrm{m^{4}\,s^{-2}}$, $[L_H^{-2}] = \mathrm{m^{-2}}$.
    Thus $[\rho_{\!E}] = \mathrm{kg\,m^{-1}\,s^{-2}} = \mathrm{J\,m^{-3}}$ (a pressure), as required.

\paragraph{Scalings and limits.}
    Using $\Gamma = 2\pi r_c C_e$ and \eqref{eq:C5},
    \[
        \rho_{\!E}
        = \kappa_H\,\frac{\rho_{\!f}}{L_H^2}\,(2\pi r_c C_e)^2\,\mathcal{L}_{\circlearrowleft}^{\mathrm{per}}
        \;\propto\;
        \rho_{\!f}\,\frac{r_c^2 C_e^2}{L_H^2}\,\mathcal{L}_{\circlearrowleft}^{\mathrm{per}}.
    \]
    Hence
    $\partial\ln\rho_{\!E}/\partial\ln r_c = 2$,
    $\partial\ln\rho_{\!E}/\partial\ln C_e = 2$,
    $\partial\ln\rho_{\!E}/\partial\ln L_H = -2$,
    and $\rho_{\!E}$ grows linearly with $\mathcal{L}_{\circlearrowleft}^{\mathrm{per}}$.
    For $N$ identical cores with uniform $L_{ij}=1$,
    $\mathcal{L}_{\circlearrowleft}^{\mathrm{per}}\!\sim N(N\!-\!1)/2$
    (see Sec.~5.3), giving rapid amplification in linked arrays.

\paragraph{Numerical illustration (with your constants).}
    Let $\rho_{\!f}=7.0\times10^{-7}\ \mathrm{kg\,m^{-3}}$,
    $r_c=1.40897017\times10^{-15}\ \mathrm{m}$,
    $C_e=1.09384563\times10^{6}\ \mathrm{m\,s^{-1}}$.
    Then $\Gamma_0\approx 9.683619203\times10^{-9}\ \mathrm{m^2\,s^{-1}}$
    and $\Gamma_0^2\approx 9.377248088\times10^{-17}\ \mathrm{m^4\,s^{-2}}$.
    For a single-core, unit linking density ($\mathcal{L}_{\circlearrowleft}^{\mathrm{per}}=1$),
    \[
        \rho_{\!E}
        \;=\;
        \kappa_H\,\frac{7.0\times10^{-7}}{L_H^2}\;
        \bigl(9.377248088\times10^{-17}\bigr)
        \;\; \mathrm{J\,m^{-3}}.
    \]
    Example values:
    \begin{align*}
    L_H=10^{-2}\ \mathrm{m}:&\quad \rho_{\!E}\approx
    \kappa_H\,6.56\times10^{-19}\ \mathrm{J\,m^{-3}},\\
    L_H=10^{-4}\ \mathrm{m}:&\quad \rho_{\!E}\approx
    \kappa_H\,6.56\times10^{-15}\ \mathrm{J\,m^{-3}},\\
    L_H=10^{-6}\ \mathrm{m}:&\quad \rho_{\!E}\approx
    \kappa_H\,6.56\times10^{-11}\ \mathrm{J\,m^{-3}}.
    \end{align*}
    For $N=30$ with uniform $L_{ij}=1$ (\,$\mathcal{L}_{\circlearrowleft}^{\mathrm{per}}=435$\,),
    multiply these by $435$.

\paragraph{From energy density to thrust.}
    In the Euler–SST sector, swirl pressure equals energy density:
    $p_{\text{swirl}}=\rho_{\!E}$. A directed pressure gradient produces force
    $F \approx \Delta p \, A = \Delta \rho_{\!E}\,A$ on area $A$.
    Thus asymmetric control of $L_H$ and/or $\mathcal{L}_{\circlearrowleft}^{\mathrm{per}}$
    (e.g.\ phase-biased linking across the device) yields a net thrust:
    \[
        F \;\approx\; \kappa_H\,\frac{\rho_{\!f}}{L_H^{2}}\;\Delta\!\big(\Gamma^{2}\mathcal{L}_{\circlearrowleft}^{\mathrm{per}}\big)\,A.
    \]

\paragraph{Calibration protocol (minimal experiment).}
    \begin{enumerate}
    \item Build a multi-core array with controllable pairwise phase to tune $\mathcal{L}_{\circlearrowleft}^{\mathrm{per}}$.
    \item Fix $(r_c,C_e)$ drive; vary $N$ and the phase program to scan $\mathcal{L}_{\circlearrowleft}^{\mathrm{per}}$.
    \item Measure net force $F$ on a thrust stand while modulating a boundary layer to set $L_H$ (e.g.\ dielectric spacing or flow-ring spacing).
    \item Fit $\kappa_H$ and $L_H$ via $F \approx (\rho_{\!f}/L_H^2)\,\kappa_H\,\Delta(\Gamma^2\mathcal{L}_{\circlearrowleft}^{\mathrm{per}})\,A$.
    \end{enumerate}
    This closes the loop from the conserved topological invariant to an experimentally measurable propulsion observable.

\bibliographystyle{unsrt}
\bibliography{references}
\end{document}
