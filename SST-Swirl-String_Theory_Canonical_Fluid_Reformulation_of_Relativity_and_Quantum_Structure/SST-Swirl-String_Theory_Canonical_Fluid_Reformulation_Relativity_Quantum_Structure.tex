% =====================================================================

% Swirl–String Theory (SST)

% Canonical Fluid Reformulation of Relativity and Quantum Structure

% Skeleton v0.1 (2025-10-09)

% =====================================================================

\documentclass[11pt,a4paper]{article}


% --- Packages ---

\usepackage[utf8]{inputenc}

\usepackage[T1]{fontenc}

\usepackage[margin=2.2cm]{geometry}

\usepackage{lmodern}

\usepackage{microtype}

\usepackage{amsmath,amssymb,amsfonts,bm,mathtools}

\usepackage{physics}

\usepackage{siunitx}

\sisetup{per-mode=symbol}

\usepackage{graphicx}

\usepackage[dvipsnames]{xcolor}

\usepackage{booktabs}

\usepackage{enumitem}

\usepackage{hyperref}

\hypersetup{colorlinks=true,linkcolor=MidnightBlue,citecolor=MidnightBlue,urlcolor=MidnightBlue}

\usepackage{authblk}

\usepackage{cite}


% --- House Macros: SST (preloaded; safe for copy) ---

% Explicit (macro-free in chat), but we preload for compilable PDF.

% Symbols follow user's preferred notation mapping.


% Robust circle-arrow glyphs

\newcommand{\swirlarrow}{\mathchoice{\mkern-2mu\scriptstyle\boldsymbol{\circlearrowleft}}{\mkern-2mu\scriptstyle\boldsymbol{\circlearrowleft}}{\mkern-2mu\scriptscriptstyle\boldsymbol{\circlearrowleft}}{\mkern-2mu\scriptscriptstyle\boldsymbol{\circlearrowleft}}}

\newcommand{\swirlarrowcw}{\mathchoice{\mkern-2mu\scriptstyle\boldsymbol{\circlearrowright}}{\mkern-2mu\scriptstyle\boldsymbol{\circlearrowright}}{\mkern-2mu\scriptscriptstyle\boldsymbol{\circlearrowright}}{\mkern-2mu\scriptscriptstyle\boldsymbol{\circlearrowright}}}


% Canonical symbols

% Fixed macro definitions
\newcommand{\vswirl}{\mathbf{v}_{\swirlarrow}} % characteristic swirl velocity vector
\newcommand{\vswirlcw}{\mathbf{v}_{\swirlarrowcw}} % antimatter (clockwise) variant
\newcommand{\vnorm}{\lVert \vswirl \rVert}               % speed magnitude

\newcommand{\SwirlClock}{S_{t}^{\swirlarrow}}              % matter time scaling

\newcommand{\SwirlClockcw}{S_{t}^{\swirlarrowcw}}          % antimatter time scaling

\newcommand{\rc}{r_c}                                      % core radius

\newcommand{\rhoF}{\rho_{!f}}                            % effective fluid density

\newcommand{\rhoE}{\rho_{!E}}                            % swirl energy density

\newcommand{\rhoM}{\rho_{!m}}                            % mass-equivalent density


% Useful operators

\DeclareMathOperator{\sgn}{sgn}

\DeclareMathOperator{\sech}{sech}

\DeclareMathOperator{\arcosh}{arcosh}


% Convenience environments

\newcommand{\eqdef}{\vcentcolon=}     % definition equality

\newcommand{\defeq}{=\vcentcolon}     % reversed

\newcommand{\ddt}{\frac{\mathrm d}{\mathrm dt}}

\newcommand{\DDt}{\frac{\mathrm D}{\mathrm Dt}}         % material derivative


% Placeholder for maximum forces/constants (user to confirm values)

% Comment: keep numerical assignment in Appendix to avoid hard-coding in the core.

\newcommand{\FmaxSwirl}{F_{\text{swirl}}^{\max}}

\newcommand{\FmaxGr}{F_{\text{gr}}^{\max}}


% Chronos–Kelvin invariant (named form; used in text)

\newcommand{\CKInv}{\displaystyle \DDt ( R^2 \omega ) = 0}


% --- Title & Authors ---

\title{Swirl--String Theory: Canonical Fluid Reformulation of Relativity and Quantum Structure}

\author[1]{Omar Iskandarani}

\affil[1]{Independent Researcher}

\date{October 2025}


\begin{document}

\maketitle


\begin{abstract}

\noindent

We present Swirl--String Theory (SST) as a canonical, mechanical reformulation of relativistic and quantum structure based on an incompressible, inviscid medium endowed with quantized internal rotation (``swirl strings''). We state kinematic axioms, fix the symbol set $\{\vswirl,\rc,\rhoF,\rhoE,\rhoM\}$ and derive: (i) local time scaling from swirl speed; (ii) a stress--pressure correspondence replacing spacetime curvature in the weak field; (iii) circulation quantization leading to discrete particle attributes. We prove a Chronos--Kelvin invariant, construct an analogue metric, and show correspondence limits to GR and linear quantum wave equations. Dimensional anchors and numerical benchmarks are provided from a single constant set. We conclude with falsifiable predictions and a program for experimental probes.

\end{abstract}


\tableofcontents


% =====================================================================

\section{Introduction}

\label{sec:intro}

% -- Motivation: fragmentation GR/QFT; ontology gap.

% -- Programme: restore mechanical ontology via incompressible swirl medium.

% -- Scope: definition of canonical constants; correspondence limits; predictions.

% -- Historical lineage (Helmholtz, Kelvin, Einstein--\AE ther, Jacobson, Verlinde).


\paragraph{Motivation.}  Modern physics excels empirically yet rests on distinct ontologies for gravitation and quantum fields. SST posits a shared mechanical origin governed by incompressible kinematics, vorticity conservation, and circulation quantization.


\paragraph{Contributions.}  This paper (i) states the axioms; (ii) fixes symbols and constants; (iii) establishes GR/QFT correspondences; (iv) lists falsifiers.


% =====================================================================

\section{Kinematic Axioms and Conservation Laws}

\label{sec:axioms}

\begin{enumerate}[label=K\arabic*., leftmargin=3.5em]

\item \textbf{Incompressible, inviscid continuum:} $\nabla\cdot\mathbf v = 0$, with Euler form $\DDt \, \mathbf v = -\nabla p/\rhoF$.

\item \textbf{Swirl time scaling (Chronos/Swirl--Clock):}

\begin{equation}
 dt_{\rm local} = dt_{\infty} \sqrt{1 - \frac{\lVert\vswirl\rVert^2}{c^2}}.
 \label{eq:swirl-time}
\end{equation}

\item \textbf{Kelvin circulation and material invariance:} (\CKInv) under conditions of incompressibility, barotropy, and no reconnection.

\end{enumerate}


\noindent \textbf{Derived densities:}

\begin{equation}
 \rhoE = \tfrac{1}{2} \rhoF \lVert\vswirl\rVert^2,\qquad \rhoM = \frac{\rhoE}{c^2}.\label{eq:densities}
\end{equation}


% =====================================================================

\section{Canonical Constants and Dimensional Anchors}

\label{sec:constants}

% State symbols and (to-be-inserted) numerical benchmarks

We adopt the canonical set $\{\vswirl,\rc,\rhoF,\rhoE,\rhoM,\FmaxSwirl,\FmaxGr\}$.

Table~\ref{tab:constants} lists values and units (populated in Appendix~\ref{app:num}).


\begin{table}[h]

\centering

\caption{Canonical constants (to be numerically populated).}

\label{tab:constants}

\begin{tabular}{@{}llll@{}}
\toprule
Symbol & Meaning & Dimension & Value (SI) \\
\midrule
$\vswirl$ & characteristic swirl speed & L T$^{-1}$ & $1.09384563\times10^6\,\si{m\,s^{-1}}$ \\
$\rc$ & core radius & L & $1.40897017\times10^{-15}\,\si{m}$ \\
$\rhoF$ & effective fluid density & M L$^{-3}$ & $7.0\times10^{-7}\,\si{kg\,m^{-3}}$ \\
$\rho_{\text{core}}$ & core density & M L$^{-3}$ & $3.8934358266918687\times10^{18}\,\si{kg\,m^{-3}}$ \\
$\FmaxSwirl$ & max swirl force & M L T$^{-2}$ & $29.053507\,\si{N}$ \\
$\FmaxGr$ & max gravitational force & M L T$^{-2}$ & $3.02563\times10^{43}\,\si{N}$ \\
\bottomrule
\end{tabular}

\end{table}


% =====================================================================

\section{Relativistic Correspondence: Analogue Metric and Weak Field}

\label{sec:relativistic}

We define the effective line element

\begin{equation}

ds^2 \defeq c^2 dt^2\Big(1-\frac{\lVert\vswirl\rVert^2}{c^2}\Big) - d\mathbf x^2,

\label{eq:metric}

\end{equation}

which yields gravitational time dilation in the weak-swirl limit.


\subsection{Stress--pressure mapping to Newtonian limit}

% TODO: derive Poisson-like equation from pressure wells; compare with GR weak field.


\subsection{Constraints and known-limit checks}

% TODO: recover standard redshift and Shapiro-like delay scaling under assumptions.


% =====================================================================

\section{Quantum Correspondence: Circulation Quantization}

\label{sec:quantum}

% Kelvin circulation, topological charges, helicity as quantum numbers.

\subsection{Action principle}

\begin{equation}

\mathcal S_{\rm SST} = \int \Big[ \tfrac{1}{2}\rhoF \mathbf v^2 - \rhoF \Phi(\rhoF) - \lambda (\nabla\cdot\mathbf v) \Big] \, d^3x\, dt.

\end{equation}

Linearization about a stationary background leads to wave equations analogous to massive/massless sectors.


\subsection{Topological sectors and particle taxonomy}

% TODO: map linking/helicity to quantum numbers; note SU(3)\oplus SU(2)\oplus U(1) homomorphism (outline only).


% =====================================================================

\section{Canonical Correspondences}

\label{sec:dictionary}

\begin{table}[h]

\centering

\caption{Dictionary between relativistic/quantum objects and SST quantities.}

\begin{tabular}{@{}ll@{}}
\toprule
GR/QFT concept & SST counterpart \\
\midrule
Metric curvature $R_{\mu\nu}$ & Swirl-pressure tensor gradients \\
Stress--energy $T_{\mu\nu}$ & Energy--momentum flux of $(\rhoF,\mathbf v)$ \\
Quantum phase & String-phase circulation (quantized) \\
Planck constant $\hbar$ & Circulation quantum (to be calibrated) \\
\bottomrule
\end{tabular}

\end{table}


% =====================================================================

\section{Predictions and Falsifiers}

\label{sec:predictions}

\begin{enumerate}[leftmargin=2.5em]

\item Time dilation near rotating cores measurable via clock comparison (scaling with $\vnorm^2$).

\item Swirl--EMF coupling in rotating frames (separate dedicated paper; referenced test geometry).

\item Astrophysical bounds: compact-object limits from $\rhoE$ and $\FmaxGr$.

\end{enumerate}


% =====================================================================

\section{Discussion and Outlook}

% Connections to emergent spacetime, superfluid analogies; programmatic roadmap.


% =====================================================================

\appendix
% =====================================================================
% APPENDED: Auto-filled Numerical Benchmarks (2025-10-09)
\section*{Appendix A': Numerical Benchmarks (auto-filled)}
\addcontentsline{toc}{section}{Appendix A': Numerical Benchmarks (auto-filled)}

Using the canonical constants
$\lVert\mathbf{v}_{\!\boldsymbol{\circlearrowleft}}\rVert=1.09384563\times10^6\,\mathrm{m\,s^{-1}}$,
$\rho_{\!f}=7.0\times10^{-7}\,\mathrm{kg\,m^{-3}}$,
$c=2.99792458\times10^8\,\mathrm{m\,s^{-1}}$, and
$r_c=1.40897017\times10^{-15}\,\mathrm{m}$, we obtain:

\begin{align}
\rho_{\!E} &= \tfrac{1}{2}\,\rho_{\!f}\,\lVert\mathbf{v}_{\!\boldsymbol{\circlearrowleft}}\rVert^2
= 0.5\times(7.0\times10^{-7})\times(1.09384563\times10^6)^2 \\
&\phantom{=}\;= \boxed{4.187743917945338\times10^{5}\;\mathrm{J\,m^{-3}}} \; (=\,\mathrm{Pa}),\\[4pt]
\rho_{\!m} &= \rho_{\!E}/c^2
= \dfrac{4.187743917945338\times10^{5}}{(2.99792458\times10^{8})^2}
= \boxed{4.65949350504008\times10^{-12}\;\mathrm{kg\,m^{-3}}},\\[4pt]
\frac{\lVert\mathbf{v}_{\!\boldsymbol{\circlearrowleft}}\rVert}{c}
&= \boxed{3.648676278574026\times10^{-3}},\\[4pt]
\frac{dt_{\rm local}}{dt_{\infty}}
&= \sqrt{1-\big(\lVert\mathbf{v}_{\!\boldsymbol{\circlearrowleft}}\rVert/c\big)^2}
= \boxed{0.999993343558553},\\[4pt]
E_{\rm core} &= \rho_{\!E}\,\dfrac{4\pi}{3}\,r_c^3
= \big(4.187743917945338\times10^{5}\big)\,\dfrac{4\pi}{3}\,(1.40897017\times10^{-15})^3 \\
&\phantom{=}\;= \boxed{4.906526191489413\times10^{-39}\;\mathrm{J}}.
\end{align}

Consistency: $\rho_{\!m} c^2 = \rho_{\!E}$ numerically to floating-point precision.
Units: $[\rho_{\!E}] = \mathrm{J\,m^{-3}} = \mathrm{Pa}$, $[\rho_{\!m}] = \mathrm{kg\,m^{-3}}$.

\begin{table}[h]
\centering
\caption{Benchmark values derived from the canonical constants.}
\begin{tabular}{@{}lll@{}}
\toprule
Quantity & Expression & Value (SI) \\
\midrule
Swirl energy density & $\tfrac{1}{2}\rho_{\!f}\,\lVert\mathbf{v}_{\!\boldsymbol{\circlearrowleft}}\rVert^2$ & $4.187743917945338\times10^{5}\,\mathrm{J\,m^{-3}}$ \\
Mass-equivalent density & $\rho_{\!E}/c^2$ & $4.65949350504008\times10^{-12}\,\mathrm{kg\,m^{-3}}$ \\
Time-dilation factor & $\sqrt{1-(\lVert\mathbf{v}_{\!\boldsymbol{\circlearrowleft}}\rVert/c)^2}$ & $0.999993343558553$ \\
Core-volume energy & $\rho_{\!E}\,(4\pi/3)\,r_c^3$ & $4.906526191489413\times10^{-39}\,\mathrm{J}$ \\
\bottomrule
\end{tabular}
\end{table}
% (Auto-filled block ends.)

\section{Dimensional Analysis and Numerical Validation}

\label{app:num}

% Provide explicit SI checks and evaluate Eqs.~(\ref{eq:densities})--(\ref{eq:metric}) with the canonical constants.


\subsection{Units}

% TODO: show that \rhoE has units J/m^3, etc.


\subsection{Numerical benchmarks}

% TODO: Insert Python-generated table of values; ensure SI units.


\section{Derivation of a Swirl Coupling for $G$ (optional)}

% TODO: Show mapping to Newtonian constant under SST assumptions; record uncertainties.


% =====================================================================

\section*{Acknowledgments}

% Optional


\section*{Data and Code Availability}

All computations and figure scripts will be released upon publication.


\section*{Conflict of Interest}

The author declares no competing interests.


% =====================================================================

% Bibliography (BibTeX)

\bibliographystyle{unsrt}

\bibliography{sst_canonical}

% Add foundational entries: Helmholtz1858, Kelvin1867, Einstein1920, Jacobson1995, Verlinde2011, etc.


\end{document}

