
\begin{figure}[htbp]

\centering
\resizebox{0.9\textwidth}{!}{
    \begin{tikzpicture}[
        node distance=0.8 and 0.8,
        every node/.style={draw, rounded corners, align=center, minimum height=2},
        arrow/.style={-{Latex[length=2]}, thick},
        garrow/.style={-{Latex[length=2]}, thick, dashed}
        unittext/.style={
            font=\tiny\color{gray!60}
        }
    ]
% ---------------- TOP LAYER ----------------
% ---------------- center left curl (Faraday) ----------------
% ---------------- TOP LAYER ----------------
    \node(Faraday)
    {
        \small$\nabla \times \vec{E} = - \frac{\partial \vec{B}}{\partial t} - \vec{b}_{\mkern-2mu\scriptscriptstyle\boldsymbol{\circlearrowleft}}$\\
    \tiny \textcolor{gray}{$[\nabla\times\vec{E}]=\tfrac{V}{m^{2}},\ [\frac{\partial \vec{B}}{\partial t}]=\tfrac{T}{s}$}
    };

    \node[left=of Faraday]  (E)
    {
        \small$\vec{E}$\\
    \tiny \textcolor{gray}{$[\vec{E}]=\tfrac{V}{m}$}
    };

    \node[right=of Faraday] (b) {
        \small $\vec{b}_{\mkern-2mu\scriptscriptstyle\boldsymbol{\circlearrowleft}}
        = \mathcal{G}_{\mkern-2mu\scriptscriptstyle\boldsymbol{\circlearrowleft}}
        \, \frac{\partial \rho_{\mkern-2mu\scriptscriptstyle\boldsymbol{\circlearrowleft}}}{\partial t}
        \, \hat{n}$\\
        \tiny \textcolor{gray}{
            $[\vec{b}_{\mkern-2mu\scriptscriptstyle\boldsymbol{\circlearrowleft}}] = \tfrac{V}{m^2},\quad
            [\mathcal{G}_{\mkern-2mu\scriptscriptstyle\boldsymbol{\circlearrowleft}}] = \tfrac{V \cdot s}{N}$
        }
    };

    \node[right=of b] (rho)
    {
         \small $\rho_{\mkern-2mu\scriptscriptstyle\boldsymbol{\circlearrowleft}}$\\
    \tiny \textcolor{gray}{ $[\rho_{\mkern-2mu\scriptscriptstyle\boldsymbol{\circlearrowleft}}]=\tfrac{N}{m^{2}}$}
    };

% ---------------- MIDDLE LAYER ----------------
    \node[below=of E] (Eta)
    {
        \small $\eta = \mathcal K_E \vec{E}$\\
    \tiny \textcolor{gray}{ $[\mathcal K_E = \varepsilon] = \frac{C}{Vm}$}
    };

    \node[below=of Faraday] (D)
    {
        \small $\varepsilon \vec{E} = \vec{D}$\\
    \tiny \textcolor{gray}{ $[\varepsilon]=\tfrac{F}{m},\ [\vec{D}]=\tfrac{C}{m^{2}}$}
    };

    \node[below=of b] (B)
    {
        \small $\vec{B} = \mu \vec{H}$\\
    \tiny \textcolor{gray}{ $[\vec{B}]=T,\ [\mu]=\tfrac{N}{A^{2}}$}
    };

    \node[below=of rho] (C)
    {
        \small $\chi_H \vec{H} = \rho_{\mkern-2mu\scriptscriptstyle\boldsymbol{\circlearrowleft}}$\\
    \tiny \textcolor{gray}{ $[\chi_H]=\tfrac{N}{Am}$}
    };

% ---------------- BOTTOM LAYER ----------------
    \node[below=of Eta] (EtaBottom)
    {
        \small $\eta$\\
    \tiny \textcolor{gray}{ $[\eta]=\tfrac{C}{m^{2}}$}
    };

    \node[below=of D] (Jsrc)
    {
        \small $\mathcal{G}_{\textrm{el}} \frac{\partial \eta}{\partial t} = \vec{\jmath}$\\
    \tiny \textcolor{gray}{ $[\mathcal{G}_{\textrm{el}}]=\tfrac{A\,s}{C},\ [\vec{\jmath}]=\tfrac{A}{m^{2}}$}
    };

    \node[below=of B] (Ampere)
    {
        \small $\vec{\jmath} + \frac{\partial \vec{D}}{\partial t} = \nabla \times \vec{H}$\\
    \tiny \textcolor{gray}{ $[\frac{\partial \vec{D}}{\partial t}]=\tfrac{A}{m^{2}},\ [\nabla \times \vec{H}]=\tfrac{A}{m^{2}}$}
    };

    \node[below=of C] (H)
    {
        \small $\vec{H}$\\
    \tiny \textcolor{gray}{ $[\vec{H}]=\tfrac{A}{m}$}
    };

% ---------------- arrows  ----------------
    \draw[arrow] (E)       --  (D)             ;
    \draw[arrow] (rho)       --  (C)             ;
    \draw[arrow] (H)       --  (Ampere)        ;
    \draw[arrow] (E)       --  (Faraday)       ;
    \draw[arrow] (B)       --  (Faraday)       ;
    \draw[arrow] (D)       --  (Ampere)        ;
    \draw[arrow] (H)       --  (B)             ;
    \draw[arrow] (C)       --  (H)             ;
    \draw[arrow] (Eta)       --  (E)             ;
    \draw[arrow] (EtaBottom)       --  (Eta)           ;
    \draw[arrow] (Jsrc)       --  (EtaBottom)     ;
    \draw[arrow] (Ampere)       --  (Jsrc)          ;
    \draw[arrow] (b)       --  (rho)            ;
    \draw[arrow] (Faraday)       --  (b)             ;

    \end{tikzpicture}
}
\caption{\textbf{Canonical Swirl–Electromagnetic Coupling Diagram.}
Causal and dimensional structure of the electromagnetic sector within the
Swirl–String framework.
The top layer extends Faraday’s law with a swirl-induced backreaction term
    $\mathbf{b}_{\swirlarrow} = \mathcal{G}_{\swirlarrow} \,\partial_t \bm{\varrho}_{\swirlarrow}$,
    encoding the electromotive response to time-varying swirl density in the medium.
    The middle layer represents the constitutive closure:
    $\mathbf{D} = \bm{\varepsilon}\mathbf{E}$ and
    $\mathbf{B} = \mu\mathbf{H}$,
    together with the mechanical correspondence
    $\bm{\varrho}_{\swirlarrow} = \chi_H \mathbf{H}$.
    The bottom layer completes the circuit with areal accumulation
    $\bm{\eta}$, source current $\mathbf{j}$, and the modified Ampère curl.
    All dimensionalities are shown for canonical homology between mechanical
    (swirl) and electromagnetic sectors, establishing the
    \emph{Swirl–Electromagnetic Bridge} that underlies the
    flat-space emergence of Maxwellian dynamics.}

\label{fig:swirl_em_causal}
\end{figure}







