%! Author = Omar Iskandarani
%! Title = Swirl String Theory (SST): A Canonical Fluid Reformulation of Relativity and Quantum Structure
%! Date = Oct, 2025
%! Affiliation = Independent Researcher, Groningen, The Netherlands
%! License = © 2025 Omar Iskandarani. All rights reserved. This manuscript is made available for academic reading and citation only. No republication, redistribution, or derivative works are permitted without explicit written permission from the author. Contact: info@omariskandarani.com
%! ORCID = 0009-0006-1686-3961
%! DOI = 10.5281/zenodo.17309679

\newcommand{\paperversion}{\textbf{v1.0.0}}
\newcommand{\papertitle}{\textbf{Swirl--String Theory: A Canonical Fluid Reformulation of Relativity and Quantum Structure}}
\newcommand{\paperdoi}{10.5281/zenodo.17309679}

%========================================================================================
% PACKAGES AND DOCUMENT CONFIGURATION
%========================================================================================
\documentclass[10pt,reprint,aps,onecolumn,nofootinbib]{revtex4-2}

\usepackage[utf8]{inputenc}
\usepackage[T1]{fontenc}
\usepackage[margin=2.3cm]{geometry}
\usepackage{amsmath,amssymb,amsfonts,bm}
\usepackage{adjustbox}
\usepackage{tikz}
\usetikzlibrary{matrix,positioning}
\usetikzlibrary{arrows.meta,positioning,calc,fit,decorations.pathmorphing}
\usetikzlibrary{matrix, positioning, fit, backgrounds}
\usepackage{siunitx}
\sisetup{per-mode=symbol,detect-weight=true,detect-family=true}
\usepackage{hyperref}
\usepackage{amsopn}
\usepackage[most]{tcolorbox}

% Canonical vector and field symbols
\newcommand{\vect}[1]{\boldsymbol{#1}} % choose \mathbf or \boldsymbol; using \boldsymbol here
\newcommand{\vv}{\vect{v}}
\newcommand{\EE}{\vect{E}}
\newcommand{\BB}{\vect{B}}
\newcommand{\HH}{\vect{H}}
\newcommand{\DD}{\vect{D}}
\newcommand{\jj}{\vect{\jmath}}

% Boxed equation helper
\newcommand{\bboxeq}[1]{\boxed{\displaystyle #1}}

% Testable-claim flag
\newcommand{\testable}{\textbf{(Testable)}}

% ===============================
% Macros (canonicalized)
% ===============================
\newcommand{\swirlarrow}{%
    \mathchoice{\mkern-2mu\scriptstyle\boldsymbol{\circlearrowleft}}%
    {\mkern-2mu\scriptscriptstyle\boldsymbol{\circlearrowleft}}%
}
\newcommand{\swirlarrowcw}{%
    \mathchoice{\mkern-2mu\scriptstyle\boldsymbol{\circlearrowright}}%
    {\mkern-2mu\scriptscriptstyle\boldsymbol{\circlearrowright}}%
}

% Canonical symbols
\newcommand{\vswirl}{\mathbf{v}_{\swirlarrow}}
\newcommand{\vswirlcw}{\mathbf{v}_{\swirlarrowcw}}
\newcommand{\SwirlClock}{S_{(t)}^{\swirlarrow}}
\newcommand{\SwirlClockcw}{S_{(t)}^{\swirlarrowcw}}
\newcommand{\Fmaxswirl}{F^{\max}_{\mkern-1mu\scriptscriptstyle\boldsymbol{\circlearrowleft}}}
\newcommand{\Fmaxswirlcw}{F^{\max}_{\mkern-1mu\scriptscriptstyle\boldsymbol{\circlearrowright}}}
\newcommand{\FmaxEM}{F^{\max}_{\mathrm{EM}}}

\newcommand{\omegas}{\boldsymbol{\omega}_{\swirlarrow}}  % swirl vorticity
\newcommand{\vscore}{v_{\swirlarrow}}                    % shorthand: |v_swirl| at r=r_c
\newcommand{\vnorm}{\lVert \vswirl \rVert}               % swirl speed magnitude
\newcommand{\rhof}{\rho_{\!f}}                           % effective fluid density
\newcommand{\rhoE}{\rho_{\!E}}                           % swirl energy density
\newcommand{\rhom}{\rho_{\!m}}                           % mass-equivalent density
\newcommand{\rc}{r_c}                                    % string core radius
\newcommand{\Lam}{\Lambda}                               % Swirl Coulomb constant
\newcommand{\Om}{\Omega_{\swirlarrow}}                   % swirl angular frequency
\makeatletter
\@ifundefined{theorem}{\newtheorem{theorem}{Theorem}}{}
\@ifundefined{corollary}{\newtheorem{corollary}{Corollary}}{}
\@ifundefined{definition}{\newtheorem{definition}{Definition}}{}
\@ifundefined{lemma}{\newtheorem{lemma}{Lemma}}{}
\makeatother

\begin{document}
\title{\papertitle}
\author{Omar Iskandarani}
\affiliation{Independent Researcher, Groningen, The Netherlands}
\thanks{ORCID: 0009-0006-1686-3961, \\ DOI: \paperdoi}
\date{\today}

%==================== ABSTRACT (revised) ====================
\begin{abstract}
We present \emph{Swirl--String Theory} (SST), a fluid–topological framework in which matter and radiation are modeled as quantized vortex loops (“swirl strings”) in an incompressible, non-dissipative condensate. Within SST, classical gravitational phenomenology \emph{emerges} as a collective pressure effect in flat space, while local swirl flows \emph{recover} relativistic time-dilation kinematics. We formulate a covariant effective field theory on a preferred foliation, show how topological quantization organizes a discrete particle spectrum, and outline a route by which gauge structures may be \emph{derived} from orientational textures of the medium. A modified Faraday law is proposed in which time-varying swirl areal density sources an electromotive impulse; this yields geometry-independent, quantized flux signatures that serve as explicit falsification targets. Wave–particle duality is described via R/T phase dynamics (unknotted vs.\ knotted states), with measurement modeled as an R$\leftrightarrow$T transition. We compare SST with Kelvin’s vortex lineage, analogue-gravity programs, and emergent-gauge constructions, emphasizing where SST \emph{recovers} established limits (Newtonian gravity, Maxwell electrodynamics, quantum interference) and where it \emph{predicts} deviations that are testable in BECs, superconducting films, and attosecond spectroscopy. All equations use SI units with dimensional checks. Our aim is a parameter-light, topological account that complements standard formulations and invites direct experimental appraisal.
\emph{Keywords:} vortex dynamics; topological fluid; emergent gauge theory; time dilation; quantum measurement; attosecond spectroscopy
\end{abstract}
\maketitle

%==================== INTRODUCTION ====================
\section{Introduction}\label{sec:intro}

    Reconciling General Relativity (GR) with Quantum Mechanics (QM) remains difficult because GR treats spacetime as dynamical and curved, whereas quantum field theories (QFT) typically operate on a fixed background. Swirl–String Theory (SST) approaches this tension by positing a unified substrate: a flat-space, incompressible condensate that supports and constrains all excitations.~\footnote{Several preprints by the author are cited to reflect distinct developments of the SST framework. Each addresses separate derivational or phenomenological components. Peer-reviewed validation is ongoing.}

    \subsection*{Motivation: Recasting the GR–QM Mismatch}
        The long–standing mismatch between the Standard Model (SM) and GR suggests re-examining some starting assumptions. SST recasts curvature as effective fluid kinematics within a background Euclidean manifold. In this picture, what is interpreted as gravity \emph{emerges} from pressure and flow gradients in the condensate rather than from a fundamental geometric interaction. Treating both gravitational and quantum effects within one hydrodynamic framework \emph{aims} to clarify shared mechanisms while preserving empirical constraints from GR and the SM.

    \subsection*{Historical Lineage: From Vortex Atoms to Analogue Gravity}
        SST fits within a broader tradition that links topological fluid structures to matter. Early ideas—such as Kelvin’s vortex atoms—anticipated the use of knotted, conserved circulations as building blocks. Modern developments in knot theory, quantized circulation, and analogue gravity deepen this thread: horizons and other curved–spacetime analogues can be \emph{recovered} in superfluids and Bose–Einstein condensates. SST draws on this lineage by using the stability of vortex dynamics to \emph{derive} a durable particle spectrum.

    \subsection*{SST Proposition: A Topological Fluid Substrate for Particles and Interactions}
        SST posits a single universal medium—the “swirl” condensate—characterized by effective density, core swirl speed, and core radius. Physical observables such as mass, forces, and time dilation \emph{emerge} from topologically protected excitations: knotted swirl strings. Formally, the framework resembles a modern Lorentz–Ether–style construction: it lives on a flat 4D manifold with an absolute time parameter and a preferred foliation set by the condensate’s unit timelike flow. The appeal of this background lies in what it may \emph{predict}: topological constraints on mass generation and the \emph{recovery} of SM–like gauge structure offer a path toward fewer free parameters than many effective models. At the same time, SST accepts empirical equivalences with standard formulations wherever they obtain and treats departures as testable claims rather than assumptions.

    \subsection*{Relation to Emergent-Gravity Programs}
        Conceptually, SST aligns with approaches in which gravity \emph{derives} from statistical or informational tendencies rather than standing as a separate fundamental interaction. In SST, a time–varying swirl density acts like an entropy– or information–density field. Resulting hydrodynamic gradients track the gradient of this swirl–based entropy and \emph{recover} attractive behavior consistent with gravitational phenomenology. This analogy is used as an organizing principle rather than as a claim of superiority; points of agreement and potential deviation are presented as targets for calculation and experiment. While string theory often pursues unification through extra–dimensional vibrational modes~\cite{Susskind2003}, SST frames unification in terms of topological fluid structures in a fixed 4D background. Consistent with calls for sharper empirical criteria~\cite{Hossenfelder2018}, we emphasize falsifiable consequences (Secs.~\ref{sec:em}, \ref{sec:falsifiability}).


%==================== CORE POSTULATES ====================
\section{Core Postulates of Swirl–String Theory}\label{sec:postulates}
    SST is defined by six axioms that constrain the dynamics and properties of a universal condensate and its topological excitations. Core postulates follow from SST Canon v0.5.10~\cite{sstCanon}, with corresponding derivations in~\cite{sstLagrangian}.

    \subsection*{Axioms from the Canon v0.5.10}
        \begin{enumerate}
        \item \textbf{Incompressible swirl condensate.} Physics is modeled on Euclidean $\mathbb{R}^3$ with an absolute time $t$. The background substrate is a frictionless, incompressible fluid ($\nabla \cdot \vv=0$) that supplies the kinematic stage on which all excitations evolve.

        \item \textbf{Swirl strings as knotted topological excitations.} Particles and field quanta are represented by closed, stable vortex filaments (``swirl strings''). Their discrete quantum numbers \emph{derive} from knot topology and linking invariants, providing a topological state space for matter and radiation.

        \item \textbf{Quantized circulation ($\Gamma=n\kappa$).} The circulation of the swirl velocity $\vect{v}_{\mkern-2mu\scriptscriptstyle\boldsymbol{\circlearrowleft}}$ around any closed loop $C$ is \emph{quantized} in integer multiples of a fundamental quantum $\kappa$,
        \[
            \Gamma = n\kappa,
        \]
        with $\kappa=h/m_{\mathrm{eff}}$ linking topological class to a quantum scale, in parallel with Onsager–Feynman quantization in superfluids~ \cite{Onsager1949}.

        \item \textbf{Swirl clocks and local time dilation ($S_t$).} Local proper time is set kinematically by the local tangential swirl speed $v$, with
        \begin{equation}
        S_t \;=\; \frac{dt_\text{local}}{dt_\infty} \;=\; \sqrt{1-\frac{v^2}{c^2}},
        \label{eq:swirlclock}
        \end{equation}
        which \emph{recovers} the standard special–relativistic time-dilation factor. In the SST picture, high swirl speeds co-vary with deeper effective gravitational potentials, so clocks slow where the medium’s flow is strongest. The claim is kinematic equivalence, not empirical replacement: SST aims to match established relativistic tests while offering a medium-based account of the same effects.

        \item \textbf{Dual R-phase and T-phase states.} Swirl strings exhibit a two-phase description: an extended, unknotted R-phase (radiative, wave-like, effectively massless) and a localized, knotted T-phase (tangible, particle-like, mass-carrying). Quantum duality is modeled as a dynamical R$\leftrightarrow$T transition.

        \item \textbf{Canonical knot–particle correspondence (illustrative mapping).} Specific knot classes are proposed to \emph{correspond} to particle species. A representative assignment places charged leptons on torus knots (e.g., electron $\leftrightarrow$ trefoil $3_1$) and quarks on chiral hyperbolic knots (e.g., up quark $\leftrightarrow 5_2$, down quark $\leftrightarrow 6_1$). Massless bosons (e.g., photons) are modeled as unknotted R-phase torsional excitations. These mappings are presented as testable, calculable hypotheses anchored in topological invariants rather than as presupposed identities.
        \end{enumerate}

%==================== LAGRANGIAN ====================
\section{Lagrangian and Field-Theoretic Framework}\label{sec:lagrangian}
    The dynamical content of SST is organized as a covariant effective field theory (EFT) defined on a preferred time foliation supplied by the condensate flow~ \cite{sstLagrangian}. The formalism is intended to \emph{recover} standard phenomenology where required while making distinct, testable predictions where the medium’s structure is consequential.

    \subsection*{Preferred foliation via clock field, projectors, and Khronon sector}
        A scalar “clock” field selects a unit timelike 4-velocity that defines the preferred frame. Spatial dynamics are confined to leaves orthogonal to this flow via a projector construction. The action includes a Khronon sector with gradient terms for the time field and associated couplings. Constraints from multimessenger observations (e.g., GW170817) motivate parameter choices that \emph{fix} the tensor-mode speed to be luminal, aligning the EFT with gravitational-wave propagation bounds.

    \subsection*{Two-form vorticity, non-Abelian swirl connection, and emergent gauge structure}
        Topological degrees of freedom are described by two complementary ingredients. A two-form potential captures coherence and topological charge with a topologically conserved field strength. In parallel, an emergent non-Abelian “swirl connection” encodes coarse-grained orientational textures of the string network and takes values in a compact Lie algebra; its curvature measures defect density in the medium. Upon integrating out short-distance structure, the EFT \emph{recovers} a Yang–Mills sector whose modes correspond to the medium’s internal excitations, offering a route to SM-like gauge interactions as emergent phenomena~\cite{sstCanon,sst-Lagrangian}. The emphasis is on derivation and empirical equivalence, not on asserting prior superiority.

    \subsection*{Canonical Lagrangian formalism}
        A minimal consistent Lagrangian includes kinetic terms for the clock and vorticity sectors, matter couplings to the swirl connection, and protected topological terms. Stability follows from conserved charges and topological invariants, with contributions such as a Chern–Pontryagin density enforcing the requisite conservation laws in the action.

    \subsection*{Summary of the Canonical Lagrangian (see Appendix~\ref{app:fullL})}
        On a flat 4D background with a preferred foliation selected by a scalar clock field \(\Phi\), we define the unit timelike vector \(u^\mu \propto \partial^\mu \Phi\) and the spatial projector \(P^{\mu\nu}=\eta^{\mu\nu}+u^\mu u^\nu\). The SST action decomposes into the following sectoral terms:
        \begin{equation}
            \mathcal{L} =
            \mathcal{L}_\Phi
            + \mathcal{L}_B
            + \mathcal{L}_{\mathrm{YM}}
            + \mathcal{L}_{\mathrm{mat}}
            + \mathcal{L}_{\mathrm{top}}
            + \mathcal{L}_{\mathrm{bridge}}
            \label{eq:L_summary}
        \end{equation}

        \begin{itemize}
            \item \textbf{Clock (Khronon) sector} \(\mathcal{L}_\Phi\).\; Fixes the preferred foliation via the unit field \(u^\mu\) (constructed from \(\Phi\)). Couplings are chosen such that \(c_{13}\equiv c_1+c_3=0\), yielding a luminal tensor mode consistent with multimessenger constraints.

            \item \textbf{Two–form (vorticity) sector} \(\mathcal{L}_B\).\; Encodes coherent vorticity with a two–form potential \(B_{\mu\nu}\) and conserved field strength \(H_{\mu\nu\rho}=3\partial_{[\mu}B_{\nu\rho]}\); a topological density \(\mathcal{L}_{\mathrm{top}}\) (e.g., Chern–Pontryagin–type) protects the associated charge.

            \item \textbf{Emergent gauge (swirl connection) sector} \(\mathcal{L}_{\mathrm{YM}}\).\; Introduces a non-Abelian connection \(A_\mu\in\mathfrak{g}\) with curvature
            \(G_{\mu\nu}=\partial_\mu A_\nu-\partial_\nu A_\mu+[A_\mu,A_\nu]\); the Yang–Mills form governs the coarse-grained orientational textures of the swirl network.

            \item \textbf{Matter (knot–soliton) sector} \(\mathcal{L}_{\mathrm{mat}}\).\; Dirac fields \(\psi_K\) minimally coupled via \(D_\mu\) to the emergent connections on the spatial leaves (\(P^\mu_{\ \nu}D_\mu\)), with solitonic rest mass  \(m_K^{(\mathrm{sol})}=\mathcal{M}_0\,\Xi_K\) as in Eq.~\eqref{eq:masslaw}. ~\cite{MantonSutcliffe2004,Skyrme1962}

            \item \textbf{Bridge term} \(\mathcal{L}_{\mathrm{bridge}}\).\; Realizes the modified Faraday law by coupling the time variation of the swirl areal density \(\partial_t\rho_{\circlearrowleft}\) to the \(\mathrm{U}(1)\) field strength, yielding
            \(\nabla\times \vect{E}=-\partial_t \vect{B}-\vect{b}_{\circlearrowleft}\) with
            \(\vect{b}_{\circlearrowleft}\propto \mathcal{G}_{\circlearrowleft}\,\partial_t\rho_{\circlearrowleft}\,\hat n\) (Sec.~\ref{sec:em})~\cite{EM_G}.
        \end{itemize}

        Explicit densities, coefficient normalizations, and symmetry restrictions (foliation-preserving diffeomorphisms; topological charge conservation) are listed in Appendix~\ref{app:fullL}.

    \subsection*{Mass via a solitonic knot energy functional}
        A central working hypothesis in SST is that fermion rest masses \emph{emerge} as non-perturbative soliton energies of stable knotted excitations~ \cite{sstLagrangian}. This aims to reduce reliance on freely tuned Yukawa parameters by tying masses to topological data through a canonical scaling law:
        \begin{equation} \label{eq:masslaw}
            m_K^{(\mathrm{sol})} \;=\; \mathcal{M}_0 \,\Xi_K(m,n,s,k;V_K,\phi_{\mathrm{DSI}}),
        \end{equation}
        where the universal scale $\mathcal{M}_0$ is fixed by swirl–fluid parameters $\big(\mathbf{v}_{\!\boldsymbol{\circlearrowleft}},\,\rho_{\!f},\,r_c\big)$ and calibrated to the electron mass $m_e$. The dimensionless multiplier $\Xi_K$ encodes knot-specific invariants—e.g., crossing number $m$, symmetry class $s$, chirality index $k$, and (for quark knots) hyperbolic volume $V_K$~ \cite{sstLagrangian}.

        To incorporate helicity and torsional twist, a discrete–scale–invariance factor $\phi_{\mathrm{DSI}}^{-2k}$ enters $\Xi_K$, with
        \[
            \phi_{\mathrm{DSI}} \equiv \exp\!\left(\operatorname{asinh}\tfrac12\right),
        \]
        providing a canonical suppression of higher–chirality configurations.

        \paragraph{Normalization and canonical limits.}
            \begin{itemize}
            \item \textbf{Electron anchor:} Assign $\Xi_{3_1}=1$ for the trefoil ($3_1$), fixing $\mathcal{M}_0=m_e$.
            \item \textbf{R/T phase limit:} The R-phase (unknot) satisfies $\Xi_{\mathrm{unknot}}=0$ (massless), while T-phase knots yield $\Xi_K>0$.
            \item \textbf{Quark-sector monotonicity:} For chiral hyperbolic knots, impose $\partial \Xi_K/\partial V_K>0$ to preserve topological ordering across species.
            \end{itemize}
            Once $\mathcal{M}_0$ is calibrated, the fermion spectrum is \emph{recovered} from knot class via $\Xi_K$, yielding a parameter-light account subject to direct comparison with observed masses.

% =====================================================
% Section: Swirl–Braid Correspondence and Energy Functional
% =====================================================
\section{Swirl–Braid Correspondence and Energy Functional}\label{sec:swirlbraid}

    In the Swirl–String framework, each quantized filament is represented by a smooth, oriented embedding
    \[
        \Gamma: S^1 \to \mathbb{R}^3,\qquad \Gamma(s)=\mathbf{r}(s),\ s\in[0,1),
    \]
    with local swirl velocity field \(\mathbf{v}_{\!\boldsymbol{\circlearrowleft}}(\mathbf{r})\) and conserved fluid helicity
    \[
        \mathcal{H} \;=\; \int_{\mathbb{R}^3} \mathbf{v}_{\!\boldsymbol{\circlearrowleft}}\cdot
        \big(\nabla\times \mathbf{v}_{\!\boldsymbol{\circlearrowleft}}\big)\,d^3x,
    \]
    which is preserved in ideal (nondissipative) flow up to boundary terms.

    \paragraph*{Braid closure for knotted filaments.}
        A classical result ensures every oriented knot or link is the closure of a braid:

        \begin{theorem}[Alexander~\cite{Alexander1923}]
        Every oriented knot or link in \(\mathbb{R}^3\) is equivalent to the closure of some \(n\)-strand braid \(w\in B_n\).
        \end{theorem}

        Thus each swirl string can be associated with an element of the Artin braid group
        \[
            B_n = \big\langle \sigma_1,\dots,\sigma_{n-1}\ \big|\
            \sigma_i\sigma_j=\sigma_j\sigma_i\ \ (|i-j|>1),\ \
            \sigma_i\sigma_{i+1}\sigma_i=\sigma_{i+1}\sigma_i\sigma_{i+1}
            \big\rangle,
        \]
        where \(\sigma_i\) denotes a right-handed crossing of strands \(i\) and \(i{+}1\) (and \(\sigma_i^{-1}\) a left-handed crossing).

    \subsection*{Trefoil as a braided swirl loop}
        The canonical electron configuration in SST—the trefoil knot—admits a 3-strand braid closure, e.g.
        \[
            w_e = \sigma_1^3,
        \]
        so dynamical tightening can be viewed as a braid-word contraction driven by swirl tension. Each \(\sigma_i\) represents a localized swirl crossing; closure enforces the single-valued phase condition \(\Gamma(0)=\Gamma(1)\).

    \subsection*{Swirl–braid energy functional}
        To formalize contraction dynamics, we use a canonical effective functional on a braid representative \(w\):
        \begin{equation}\label{eq:Eeff-braid}
            \mathcal{E}_{\mathrm{eff}}[w] = \alpha\,C_{\min}(w) + \beta\,L(w) + \gamma\,\mathcal{H}(w).
        \end{equation}
        Here \(C_{\min}(w)\) is the minimal crossing number over equivalent braid representatives of the same link type (to avoid projection dependence), \(L(w)\) is the physical core-line length, and \(\mathcal{H}(w)\) is the helicity associated with the configuration. Coefficients scale with swirl-fluid parameters as
        \[
            \alpha \sim \frac{\rho_f\, r_c^3}{\|\mathbf{v}_{\!\boldsymbol{\circlearrowleft}}\|^{2}},\qquad
            \beta \sim \rho_E\, r_c,\qquad
            \gamma \sim \tfrac12\,\rho_f\, r_c^2,
        \]
        where \(\rho_f\) is the effective fluid density, \(r_c\) the core radius, and \(\rho_E\) the energy per unit length of the filament.

    \paragraph*{Markov moves as physical updates.}
        Alexander’s theorem plus Markov’s theorem implies that closures of braids related by conjugation and stabilization represent the same link type:
        \[
            w \to \sigma_i w \sigma_i^{-1}\quad\text{(conjugation)},\qquad
            w \to w\,\sigma_{n}^{\pm 1}\quad\text{(stabilization)}.
        \]
        We interpret these as local reconnections/reorderings and strand creation/annihilation on the coarse-grained worldsheet. In ideal swirl flow, helicity is conserved (Moffatt) and changes only through controlled dissipative channels; \(\mathcal{E}_{\mathrm{eff}}\) decreases monotonically under allowed relaxation.

    \subsection*{Topological quantization and swirl charge}
        The statistical treatment of knotted vortex dynamics finds lineage in early work on quantized circulation~\cite{Onsager1949}, anchoring SST’s helicity-based energy model.
        For thin, isolated filaments, the Calugăreanu–White–Fuller relation connects link, twist, and writhe of a framed curve: \(L_k = T + W\)~\cite{Calugareanu1959,White1969,Fuller1971}. In fluid dynamics, helicity decomposes into self- and mutual-linking contributions with circulation weights (Moffatt)~\cite{Moffatt1969}. In the single-filament limit with fixed circulation, we use the schematic form
        \[
            \mathcal{H} \;\propto\; T + W,
        \]
        noting that the precise coefficients depend on circulation and framing. The associated quantized swirl charge is
        \[
            Q_{\!\circlearrowleft} = \frac{\rho_f}{2\pi}\int_{\Sigma} \big(\nabla\times \mathbf{v}_{\!\boldsymbol{\circlearrowleft}}\big)\cdot d\mathbf{S},
        \]
        which, in SST, maps algebraically to braid data (e.g., braid index \(n\)) and to the particle-family tuple used in Sec.~\ref{sec:quantization}.

    \subsection*{Connection to Chern–Simons field theory}
        In a continuum description, braid closures correspond to Wilson loops in three-dimensional Chern–Simons theory:
        \[
            S_{\mathrm{CS}} = \frac{k}{4\pi}\int A\wedge dA + \tfrac{2}{3}A\wedge A\wedge A,\qquad
            \langle W(\Gamma)\rangle \ \propto\ V_K\!\big(q\big),
        \]
        where \(V_K\) is the Jones polynomial of the knot \(K\) evaluated at \(q=\exp\!\big(\frac{2\pi i}{k+2}\big)\) for the SU(2) theory~\cite{Witten1989}. This furnishes a formal bridge between the swirl-string energy landscape and topological quantum invariants.

        \begin{tcolorbox}[colback=gray!10,colframe=black,title={Physical Picture}]
            A braided swirl string behaves like multiple helical vortices twisting around each other and closing into a loop. As tension tightens the bundle, crossings smooth while preserving helicity, and the configuration settles into a quantized knotted state (e.g., the trefoil) corresponding to an elementary excitation.
        \end{tcolorbox}


%==================== GRAVITY ====================
\section{Emergent Gravity and Time Dilation}\label{sec:gravity}
In SST, gravity is reinterpreted as a hydrodynamic attraction resulting from conserved circulation in a flat background~\cite{chiralSwirl}.

    \subsection*{Swirl-Induced Pressure Gradients as Gravitational Attraction}
        Massive particles, represented by stable chiral knotted strings, maintain a persistent, non-vanishing circulation around a central axis. This circulation induces a radial pressure deficit along the axis, governed by the Euler fluid balance equation~\cite{sstCanon}. When two neutral, composite systems (e.g., two proton cores within an H$_2$ molecule) share this central line, their circulations add, intensifying the pressure well (for two protons). This shared pressure deficit draws the systems together, producing the observed long-range inverse-square gravitational attraction in flat space, known as the Hydrogen-Gravity Mechanism. We treat this as a constructive mechanism with testable scaling relations; it is not assumed to replace post-Newtonian fits \textit{a priori}.

    \subsection*{Derivation of Matching Newton's Constant}
        The effective gravitational coupling $G_\text{swirl}$ is derived from the core swirl parameters of the medium:
        \begin{equation}
            \bboxeq{
                G_{\text{swirl}}
                = \mathcal{G}_{\mkern-2mu\scriptscriptstyle\boldsymbol{\circlearrowleft}}
                = \frac{v_{\mkern-2mu\scriptscriptstyle\boldsymbol{\circlearrowleft}} \, c^5 \, t^2}{2 \, \Fmaxswirl \, r_c^2}
                \approx G_N
            }
            \label{eq:gswirl}
        \end{equation}
        By calibrating the foundational constants ($v_{\mkern-2mu\scriptscriptstyle\boldsymbol{\circlearrowleft}}$, $r_c$) and the maximum emergent electromagnetic force ($\Fmaxswirl \approx 2.9 \times 10^1$ N), $G_\text{swirl}$ is numerically consistent with Newton's constant $G_N$ within calibration uncertainty~\cite{sstCanon}. This relation frames the gravitational coupling strength as constrained by the medium’s dynamics and the maximal electromagnetic tension it can support.

    \subsection*{Composite Baryons as Merged Vortex Tubes}
        Baryons, such as the proton, are realized as composite swirl tubes formed by the merging of three quark knots (e.g., two up $5_2$ and one down $6_1$ twist-knots) at a Y-junction~\cite{chiralSwirl}. Due to Kelvin's circulation theorem, the circulation is additive. Since each constituent quark carries circulation around the central axis, the baryon core possesses a total circulation. This increased circulation translates to a significant increase in the effective tangential core velocity and, consequently, a much deeper pressure well, correlating with the baryon's larger rest mass.

    \subsection*{Swirl Clock Effects Explain Gravitational Redshift}
        The Swirl Clock factor (Equation~\ref{eq:swirlclock}) dictates that time runs slower in regions of higher swirl velocity~\cite{sstCanon}. Regions of concentrated mass (knotted strings) generate intense swirl flows. Consequently, intense gravitational potential wells are simply high-swirl-velocity regions where time dilation is pronounced. Gravitational redshift is therefore interpreted as a kinematic frequency shift arising from the difference in local time rates between a source located in a deep swirl region and an observer located in a quiescent, faster-ticking region of the medium~\cite{sstCanon}.

%==================== EMERGENCE OF EM ====================
\section{Electromagnetic Emergence: Modified Faraday Law}\label{sec:em}
We model electromagnetism as an emergent response of the swirl medium to topological dynamics~\cite{EM_G}. In this view, non-adiabatic events—nucleation, annihilation, or reconnection of swirl strings—produce localized changes in the swirl areal density $\rho_{\circlearrowleft}$ and thereby \emph{drive} an electromotive impulse in the surrounding region.

    \subsection*{Swirl String Nucleation/Annihilation Produces EM Impulses}\label{subsec:em_g}
        A central prediction is that each integer change $\Delta N$ in the number of linking swirl strings yields a geometry-independent flux impulse,
        \[
            \Delta\Phi = \mathcal{G}_{\circlearrowleft} \,\Delta N,
        \]
        with the sign fixed by chirality~\cite{EM_G}. \testable\ This provides a sharp, falsifiable signature designed for SQUIDs or fast pickup loops; independence from detector geometry is part of the prediction rather than an assumption.


        \begin{figure}[htbp]\label{fig:swirl_em_causal}
            \centering
            \resizebox{0.9\textwidth}{!}{
                \begin{tikzpicture}[
                        node distance=0.8 and 0.8,
                        every node/.style={draw, rounded corners, align=center, minimum height=2},
                        arrow/.style={-{Latex[length=2]}, thick},
                        garrow/.style={-{Latex[length=2]}, thick, dashed}
                        unittext/.style={ font=\tiny\color{gray!60}}
                    ]
                    % ---------------- TOP LAYER ----------------
                    \node(Faraday)
                    {
                        \small$\nabla \times \vec{E} = - \frac{\partial \vec{B}}{\partial t} - \vec{b}_{\mkern-2mu\scriptscriptstyle\boldsymbol{\circlearrowleft}}$\\
                        \tiny \textcolor{gray}{$[\nabla\times\vec{E}]=\tfrac{V}{m^{2}},\ [\frac{\partial \vec{B}}{\partial t}]=\tfrac{T}{s}$}
                    };

                    \node[left=of Faraday]  (E)
                    {
                        \small$\vec{E}$\\
                        \tiny \textcolor{gray}{$[\vec{E}]=\tfrac{V}{m}$}
                    };

                    \node[right=of Faraday] (b) {
                        \small $\vec{b}_{\mkern-2mu\scriptscriptstyle\boldsymbol{\circlearrowleft}}
                        = \mathcal{G}_{\mkern-2mu\scriptscriptstyle\boldsymbol{\circlearrowleft}}
                        \, \frac{\partial \rho_{\mkern-2mu\scriptscriptstyle\boldsymbol{\circlearrowleft}}}{\partial t}
                        \, \hat{n}$\\
                        \tiny \textcolor{gray}{
                            $[\vec{b}_{\mkern-2mu\scriptscriptstyle\boldsymbol{\circlearrowleft}}] = \tfrac{V}{m^2},\quad
                            [\mathcal{G}_{\mkern-2mu\scriptscriptstyle\boldsymbol{\circlearrowleft}}] = \tfrac{V \cdot s}{N}$
                        }
                    };

                    \node[right=of b] (rho)
                    {
                        \small $\rho_{\mkern-2mu\scriptscriptstyle\boldsymbol{\circlearrowleft}}$\\
                        \tiny \textcolor{gray}{ $[\rho_{\mkern-2mu\scriptscriptstyle\boldsymbol{\circlearrowleft}}]=\tfrac{N}{m^{2}}$}
                    };

                    % ---------------- MIDDLE LAYER ----------------
                    \node[below=of E] (Eta)
                    {
                        \small $\eta = \mathcal K_E \vec{E}$\\
                        \tiny \textcolor{gray}{ $[\mathcal K_E = \varepsilon] = \frac{C}{Vm}$}
                    };

                    \node[below=of Faraday] (D)
                    {
                        \small $\varepsilon \vec{E} = \vec{D}$\\
                        \tiny \textcolor{gray}{ $[\varepsilon]=\tfrac{F}{m},\ [\vec{D}]=\tfrac{C}{m^{2}}$}
                    };

                    \node[below=of b] (B)
                    {
                        \small $\vec{B} = \mu \vec{H}$\\
                        \tiny \textcolor{gray}{ $[\vec{B}]=T,\ [\mu]=\tfrac{N}{A^{2}}$}
                    };

                    \node[below=of rho] (C)
                    {
                        \small $\chi_H \vec{H} = \rho_{\mkern-2mu\scriptscriptstyle\boldsymbol{\circlearrowleft}}$\\
                        \tiny \textcolor{gray}{ $[\chi_H]=\tfrac{N}{Am}$}
                    };

                    % ---------------- BOTTOM LAYER ----------------
                    \node[below=of Eta] (EtaBottom)
                    {
                        \small $\eta$\\
                        \tiny \textcolor{gray}{ $[\eta]=\tfrac{C}{m^{2}}$}


