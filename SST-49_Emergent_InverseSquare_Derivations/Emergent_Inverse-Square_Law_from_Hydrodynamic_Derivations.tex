% ======================================================================
% Follow-up paper: Solving the 1/r (inverse-square) "open problem" in SST
% Mainstream derivations (no unpublished SST citations in main text)
% ======================================================================

\documentclass[11pt]{article}
\usepackage[margin=1in]{geometry}
\usepackage{amsmath,amssymb,amsfonts}
\usepackage{microtype}
\usepackage{hyperref}

\title{A First-Principles Origin of the Inverse-Square Law in Swirl--String Theory:\\
Three Derivations from Local Field Mediation and Momentum-Flux Conservation}
\author{Omar Iskandarani}
\date{}

\begin{document}
    \maketitle

    \begin{abstract}
        A recurrent objection to flat-background, emergent-gravity programs is that the inverse-square distance law is often imported rather than derived. We close this gap in the weak-field, static monopole sector by exhibiting three independent derivations of the $1/r$ potential and $1/r^2$ flux. (I) A Gauss-law scalar effective field theory (EFT) for the far-field mediator yields the Poisson equation whose Green's function in $\mathbb{R}^3$ is $1/r$. (II) Identifying the specific SST far-field degree of freedom as a foliation/clock scalar, we compute its quadratic EFT stress tensor $T_{ij}$ and show the associated conserved radial flux density scales as $1/r^2$, with charge $Q\propto\int\rho_m\,d^3x$. (III) We replace the Newtonian potential $\chi$ by a foliation (``khronon-like'') scalar and show that, in the static weak-field monopole limit, the EFT reduces to the same Gauss-law scalar form as in (I), so the $1/r^2$ law follows automatically. These results demonstrate that once SST commits to a local mediator in three spatial dimensions, inverse-square behavior is not an assumption but a consequence of locality, symmetry, and the $\mathbb{R}^3$ Green's function.
    \end{abstract}

    \section{Setup and scope}
        We work in a flat operational background with Minkowski causal structure and consider the weak-field, static, spherically symmetric (monopole) sector. Let $\rho_m(\mathbf{x})$ be the rest-mass density of a compact source with total mass
        \begin{equation}
            M \;=\;\int_{\mathbb{R}^3}\rho_m(\mathbf{x})\,d^3x.
        \end{equation}
        The goal is to derive, from first principles, that the far-field gravitational influence (whatever SST field represents it) must exhibit
        \begin{equation}
            \Phi(\mathbf{x}) \sim \frac{1}{r},\qquad \nabla \Phi \sim \frac{1}{r^2},\qquad r=\|\mathbf{x}\|.
        \end{equation}
        We present three derivations that differ in emphasis but agree in content.

    \section{Derivation I: Gauss-law scalar EFT $\Rightarrow$ $1/r$ Green's function}
        \subsection{Specify the mediator and write the quadratic EFT}
            In the static weak-field monopole sector, the minimal local mediator is a scalar field $\phi(\mathbf{x})$ coupled linearly to the source density. The most general rotationally invariant quadratic functional (Euclidean static limit of a Lorentzian EFT) is
            \begin{equation}\label{eq:S_static}
            S_{\rm stat}[\phi]
            \;=\;
            \int_{\mathbb{R}^3} d^3x\,
            \left[
                \frac{\kappa}{2}\,(\nabla \phi)^2 \;-\; \lambda\,\phi\,\rho_m(\mathbf{x})
            \right],
            \end{equation}
            with constants $\kappa>0$ and coupling $\lambda$.

        \subsection{Euler--Lagrange equation: Poisson form}
            Varying \eqref{eq:S_static}:
            \begin{align}
                \delta S_{\rm stat}
                &=
                \int d^3x\,
                \left[
                    \kappa\,\nabla\phi\cdot\nabla(\delta\phi) \;-\; \lambda\,\rho_m\,\delta\phi
                \right]\\
                &=
                \int d^3x\,
                \left[
                    -\kappa\,(\nabla^2\phi)\,\delta\phi \;-\; \lambda\,\rho_m\,\delta\phi
                \right]
                \quad (\text{integrate by parts, drop boundary term})
            \end{align}
            so stationarity for arbitrary $\delta\phi$ gives
            \begin{equation}\label{eq:Poisson_general}
            \kappa\,\nabla^2\phi(\mathbf{x}) \;=\; -\,\lambda\,\rho_m(\mathbf{x}).
            \end{equation}
            Define the ``Gauss-law charge density'' $\rho_Q := \rho_m$ and total charge
            \begin{equation}\label{eq:Q_def}
            Q \;:=\;\int \rho_Q\,d^3x \;=\; \int \rho_m\,d^3x \;=\; M,
            \end{equation}
            so the source is monopolar with charge $Q$.

        \subsection{Solve using the Green's function of $\nabla^2$ on $\mathbb{R}^3$}
            The Green's function $G(\mathbf{x})$ satisfying
            \begin{equation}
                \nabla^2 G(\mathbf{x}) = -4\pi\,\delta^{(3)}(\mathbf{x})
            \end{equation}
            is
            \begin{equation}\label{eq:Green_1overr}
            G(\mathbf{x}) = \frac{1}{\|\mathbf{x}\|}.
            \end{equation}
            This is a standard result: $G(r)=1/r$ is the unique (up to addition of harmonic functions) spherically symmetric fundamental solution on $\mathbb{R}^3$ \cite{Jackson1999,Arfken2013}.

            Convolving \eqref{eq:Poisson_general} with $G$ gives
            \begin{align}
                \phi(\mathbf{x})
                &=
                \frac{\lambda}{4\pi\kappa}\int d^3x'\,
                \frac{\rho_m(\mathbf{x}')}{\|\mathbf{x}-\mathbf{x}'\|}.
            \end{align}
            In the far field $r\gg$ source size, $\|\mathbf{x}-\mathbf{x}'\|\approx r$ and
            \begin{equation}\label{eq:phi_far}
            \phi(r)\;\simeq\;\frac{\lambda}{4\pi\kappa}\,\frac{1}{r}\int\rho_m(\mathbf{x}')\,d^3x'
            \;=\;\frac{\lambda}{4\pi\kappa}\,\frac{Q}{r}.
            \end{equation}
            Thus
            \begin{equation}\label{eq:gradphi_far}
            \nabla\phi(r)\;\simeq\;-\frac{\lambda Q}{4\pi\kappa}\,\frac{\hat{\mathbf{r}}}{r^2},
            \end{equation}
            which is the inverse-square flux law.

        \subsection{Gauss law from divergence theorem}
            Define the flux density
            \begin{equation}\label{eq:J_def}
            \mathbf{J} := -\,\kappa\,\nabla\phi.
            \end{equation}
            Then \eqref{eq:Poisson_general} is precisely
            \begin{equation}
                \nabla\cdot \mathbf{J} = \lambda\,\rho_m.
            \end{equation}
            Integrating over a ball $B_R$ and applying divergence theorem:
            \begin{equation}
                \oint_{S_R}\mathbf{J}\cdot d\mathbf{A}
                =
                \int_{B_R}\nabla\cdot\mathbf{J}\,d^3x
                =
                \lambda\int_{B_R}\rho_m\,d^3x
                \;\xrightarrow{R\to\infty}\;
                \lambda Q.
            \end{equation}
            Spherical symmetry implies $\mathbf{J}=J_r(r)\hat{\mathbf{r}}$, hence
            \begin{equation}\label{eq:Jr_1overr2}
            4\pi r^2\,J_r(r)=\lambda Q
            \quad\Rightarrow\quad
            J_r(r)=\frac{\lambda Q}{4\pi}\,\frac{1}{r^2}.
            \end{equation}
            Since $\mathbf{J}\propto\nabla\phi$, this is equivalent to \eqref{eq:gradphi_far}.

            \paragraph{Interpretation (10-year-old analogy).}
                Imagine ``influence'' leaving a source and spreading equally in all directions. The surface area of a sphere grows like $r^2$, so the influence per square meter must drop like $1/r^2$.

    \section{Derivation II: Identify the SST far-field carrier, compute $T_{ij}$, and extract the $1/r^2$ flux}
    \subsection{Which SST field carries far-field momentum flux?}
        In SST language, the long-range static field is taken to be carried by a \emph{clock/foliation} mode: a scalar that labels preferred-time hypersurfaces (``swirl-clock''). Denote this field by $T(x)$ and consider small perturbations about an inertial foliation:
        \begin{equation}
            T(x) = t + \tau(x),
        \end{equation}
        where $t$ is the operational background time coordinate and $\tau$ is a weak perturbation sourced by matter.

        At quadratic order, the most general Lorentz-invariant action for $\tau$ (ignoring higher derivatives) is
        \begin{equation}\label{eq:tau_Lorentz}
        S[\tau]
        =
        \int d^4x\,
        \left[
            \frac{\kappa}{2}\,\partial_\mu \tau\,\partial^\mu \tau
            -\lambda\,\tau\,\rho_m(\mathbf{x})
        \right],
        \end{equation}
        with $\rho_m$ treated as static, $\partial_t\rho_m=0$. The static sector of \eqref{eq:tau_Lorentz} reduces to \eqref{eq:S_static} with $\phi\equiv\tau$.

    \subsection{Compute the stress-energy tensor}
        From \eqref{eq:tau_Lorentz}, the symmetric stress-energy tensor (metric variation, or canonical symmetrized) for the free part is
        \begin{equation}\label{eq:Tmunu_tau}
        T_{\mu\nu}^{(\tau)}
        =
        \kappa\left(
                  \partial_\mu \tau\,\partial_\nu \tau
                  -\frac{1}{2}\eta_{\mu\nu}\,\partial_\alpha\tau\,\partial^\alpha\tau
        \right).
        \end{equation}
        In the static regime, $\partial_0\tau=0$, so $\partial_\alpha\tau\,\partial^\alpha\tau = -(\nabla\tau)^2$ and
        \begin{equation}\label{eq:Tij_static}
        T_{ij}^{(\tau)}
        =
        \kappa\left(
                  \partial_i\tau\,\partial_j\tau
                  -\frac{1}{2}\delta_{ij}(\nabla\tau)^2
        \right),
        \qquad
        T_{00}^{(\tau)}=\frac{\kappa}{2}(\nabla\tau)^2.
        \end{equation}

    \subsection{Monopole solution and the \emph{conserved} radial flux density}
        From Derivation I, for $r$ outside the source,
        \begin{equation}\label{eq:tau_mono}
        \tau(r) = \frac{\lambda Q}{4\pi\kappa}\,\frac{1}{r},
        \qquad
        \partial_r\tau(r)= -\frac{\lambda Q}{4\pi\kappa}\,\frac{1}{r^2}.
        \end{equation}

        Now define the \emph{Gauss-law flux} associated to the foliation scalar (this is the conserved radial ``momentum-like'' flux density in the static sector):
        \begin{equation}\label{eq:flux_def}
        \mathcal{F}_r(r)
        :=
        -\kappa\,\partial_r\tau(r).
        \end{equation}
        Using \eqref{eq:tau_mono},
        \begin{equation}\label{eq:flux_1overr2}
        \mathcal{F}_r(r)
        =
        \frac{\lambda Q}{4\pi}\,\frac{1}{r^2}.
        \end{equation}
        This is the requested monopole scaling:
        \[
            \boxed{\;\mathcal{F}_r(r)\propto \frac{1}{r^2},\qquad Q\propto \int \rho_m\,d^3x.\;}
        \]
        The proportionality to the enclosed mass is explicit via $Q=M$ from \eqref{eq:Q_def}. The crucial point is that this $1/r^2$ law is enforced by $\nabla\cdot(\kappa\nabla\tau)=-\lambda\rho_m$ and the divergence theorem, i.e. by local field mediation in $\mathbb{R}^3$.

    \subsection{How $T_{ij}$ encodes momentum transport}
        Although the conserved Gauss-law flux \eqref{eq:flux_def} is \emph{linear} in $\nabla\tau$ and therefore scales as $1/r^2$, the mechanical stress carried by the field is quadratic and scales as $(\nabla\tau)^2\sim 1/r^4$. For completeness, the radial traction (pressure/tension) on a sphere is
        \begin{equation}
            t_r(r)
            :=
            \hat{r}_i\,T_{ij}^{(\tau)}\,\hat{r}_j
            =
            \kappa\left[
                      (\partial_r\tau)^2-\frac{1}{2}(\partial_r\tau)^2
            \right]
            =
            \frac{\kappa}{2}(\partial_r\tau)^2
            \propto \frac{1}{r^4}.
        \end{equation}
        However, the \emph{integrated} traction force over the sphere involves the area factor:
        \begin{equation}
            F_{\rm field}(r)
            =
            \int_{S_r} t_r\,dA
            \sim
            4\pi r^2 \times \frac{1}{r^4}
            \sim
            \frac{1}{r^2}.
        \end{equation}
        Thus the field's stress tensor is consistent with the inverse-square scaling of net force, while the conserved Gauss-law flux density \eqref{eq:flux_1overr2} provides the most direct ``flux'' statement.

        \paragraph{Interpretation (10-year-old analogy).}
            The field is like a stretched rubber sheet. The \emph{slope} (gradient) gets smaller like $1/r^2$ because the ``amount of slope'' has to spread over bigger spheres. The \emph{stretching energy} depends on slope squared, so it falls off faster.

    \section{Derivation III: Replace $\chi$ by a foliation scalar; static weak-field $\Rightarrow$ Gauss-law scalar form}
    \subsection{Foliation scalar as the gravitational potential variable}
        Many covariant ``preferred foliation'' theories introduce a scalar $T(x)$ whose gradient defines a timelike direction. Define the unit timelike vector
        \begin{equation}
            u_\mu := \frac{\partial_\mu T}{\sqrt{\partial_\alpha T\,\partial^\alpha T}},
        \end{equation}
        and consider a general quadratic action in derivatives of $u_\mu$ (the usual Einstein--\emph{aether}/khronon class) \cite{JacobsonMattingly2001,BlasPujolasSibiryakov2010}.
        In the weak-field, static, monopole limit, only the scalar (spin-0) sector contributes at leading order, and one can parameterize the effective action for $\tau$ by the leading operator $(\nabla\tau)^2$.

        Concretely, expand about $T=t$:
        \begin{equation}
            T=t+\tau,\qquad \partial_\mu T = (1,\nabla\tau),\qquad
            \sqrt{\partial_\alpha T\,\partial^\alpha T}=\sqrt{1-(\nabla\tau)^2}\approx 1-\frac{1}{2}(\nabla\tau)^2.
        \end{equation}
        To quadratic order, the effective static Lagrangian necessarily contains
        \begin{equation}\label{eq:Leff_tau_static}
        \mathcal{L}_{\rm eff,stat}
        =
        \frac{\kappa}{2}\,(\nabla\tau)^2
        -\lambda\,\tau\,\rho_m
        \;+\;\text{(higher derivatives / higher powers)}.
        \end{equation}
        This is exactly the Gauss-law scalar functional \eqref{eq:S_static}. Therefore, independent of the microphysical foliation completion, the static weak-field monopole sector reduces to the same Poisson equation \eqref{eq:Poisson_general}.

    \subsection{Identify $\chi$ as a reparameterization of $\tau$}
        Define the Newtonian potential variable $\chi$ as a linear rescaling of $\tau$:
        \begin{equation}\label{eq:chi_map}
        \chi := \alpha\,\tau,
        \end{equation}
        for some constant $\alpha$ chosen so that test-particle equations match the conventional Newtonian limit. Then \eqref{eq:Leff_tau_static} becomes
        \begin{equation}
            \mathcal{L}_{\rm eff,stat}
            =
            \frac{\kappa}{2\alpha^2}\,(\nabla\chi)^2
            -\frac{\lambda}{\alpha}\,\chi\,\rho_m.
        \end{equation}
        Variation yields
        \begin{equation}
            \nabla^2\chi = -\left(\frac{\alpha\lambda}{\kappa}\right)\rho_m.
        \end{equation}
        Thus, with the identification
        \begin{equation}\label{eq:match_4piG}
        \frac{\alpha\lambda}{\kappa} = 4\pi G,
        \end{equation}
        the foliation scalar reproduces the Gauss-law scalar Poisson equation of Newtonian gravity. Its monopole solution is therefore
        \begin{equation}
            \chi(r) = -\frac{GM}{r},
            \qquad
            \nabla\chi(r)= +GM\,\frac{\hat{\mathbf{r}}}{r^2},
        \end{equation}
        which is the inverse-square law.

        \paragraph{Interpretation (10-year-old analogy).}
            Calling the field $\chi$ or calling it a ``clock field'' does not change the math. If it obeys the same Poisson equation, it must have the same $1/r$ shape.

    \section{Discussion and outlook (SST identification of $G$ kept here)}
    The three derivations show that once SST specifies a \emph{local} far-field mediator (identified here with a foliation/clock scalar in the monopole sector), the $1/r^2$ law is forced by:
    \begin{enumerate}\itemsep=0pt
    \item the form of the quadratic EFT and its Euler--Lagrange equation;
    \item the unique $\mathbb{R}^3$ Green's function $G(r)=1/r$ for $\nabla^2$;
    \item flux conservation (divergence theorem) implying $4\pi r^2 \mathcal{F}_r=\text{const}$.
    \end{enumerate}

    In SST, one additionally seeks to \emph{derive the normalization} $G$ (not merely the $1/r^2$ shape) from microphysical constants. In the SST program, this corresponds to matching the EFT coefficient combination in \eqref{eq:match_4piG} to a derived expression $G_{\rm swirl}$ constructed from SST's canonical constants. This matching fixes $\lambda/\kappa$ (and the rescaling $\alpha$) in terms of SST parameters and thereby provides a first-principles value for $G$ in the low-energy monopole limit. The present paper establishes the distance-law part independently of that normalization step.

    \appendix
    \section*{Appendix A: How this connects to other SST manuscripts (non-cited in main text)}
    This appendix is informational and does not enter the main derivations. The ``Kelvin-mode suppression'' result supports treating the far-field source as effectively rigid in ordinary conditions (so the monopole approximation is stable). Thermodynamic SST results provide a microscopic route to compute the EFT stiffness $\kappa$ and coupling $\lambda$ by coarse-graining the condensate's equation of state. Variational particle-structure results motivate that unique stable configurations select unique effective couplings, suggesting that $G$ can be obtained by matching SST microphysics to the EFT coefficients in \eqref{eq:match_4piG}.

% --------------------
% Bibliography
% --------------------
    \begin{thebibliography}{99}

        \bibitem{Jackson1999}
        J.~D.~Jackson,
        \textit{Classical Electrodynamics}, 3rd ed.,
        Wiley (1999).
        (See Poisson equation Green's function and $1/r$ fundamental solution.)

        \bibitem{Arfken2013}
        G.~B.~Arfken, H.~J.~Weber, and F.~E.~Harris,
        \textit{Mathematical Methods for Physicists}, 7th ed.,
        Academic Press (2013).
        (See Green's functions in $\mathbb{R}^3$.)

        \bibitem{JacobsonMattingly2001}
        T.~Jacobson and D.~Mattingly,
        \textit{Gravity with a dynamical preferred frame},
        Phys.\ Rev.\ D \textbf{64} (2001) 024028.
        doi:10.1103/PhysRevD.64.024028

        \bibitem{BlasPujolasSibiryakov2010}
        D.~Blas, O.~Pujol\`as, and S.~Sibiryakov,
        \textit{Consistent extension of Ho\v{r}ava gravity},
        Phys.\ Rev.\ Lett.\ \textbf{104} (2010) 181302.
        doi:10.1103/PhysRevLett.104.181302

    \end{thebibliography}

\end{document}