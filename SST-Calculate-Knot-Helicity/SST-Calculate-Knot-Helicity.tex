%! Author = Omar Iskandarani
%! Title = Helicity in Swirl-String Theory (SST) Knot Systems
%! Date = Sept 4, 2025
%! Affiliation = Independent Researcher, Groningen, The Netherlands
%! License = © 2025 Omar Iskandarani. All rights reserved. This manuscript is made available for academic reading and citation only. No republication, redistribution, or derivative works are permitted without explicit written permission from the author. Contact: info@omariskandarani.com
%! ORCID = 0009-0006-1686-3961
%! DOI = 10.5281/zenodo.17469806

%========================================================================================
\newcommand{\paperdoi}{10.5281/zenodo.17469806}

%========================================================================================
% PACKAGES AND DOCUMENT CONFIGURATION
%========================================================================================

\documentclass[aps,prb,amsmath,amssymb]{revtex4-2} % switch to aip,rsi or aps,prapplied as needed
\usepackage{siunitx}
\usepackage{graphicx}
\usepackage{bm}
\usepackage[hidelinks]{hyperref}
\usepackage[utf8]{inputenc}
\usepackage[T1]{fontenc}

% Packages
\usepackage{amsmath,amssymb}
\usepackage[margin=1in]{geometry}

\newcommand{\swirlarrow}{%
    \mathchoice{\mkern-2mu\boldsymbol{\circlearrowleft}}% display
    {\mkern-2mu\boldsymbol{\circlearrowleft}}% text
    {\mkern-2mu\scriptscriptstyle\boldsymbol{\circlearrowleft}}% script
    {\mkern-2mu\scriptscriptstyle\boldsymbol{\circlearrowleft}}% scriptscript
}
\newcommand{\vswirl}{\mathbf{v}_{\swirlarrow}}                % canonical name
\newcommand{\vnorm}{\lVert \vswirl \rVert}                    % magnitude, if needed
\newcommand{\vv}{\vswirl}                                     % (optional) alias for existing \vv uses
\newcommand{\ww}{\boldsymbol{\omega}}   % vorticity field
\newcommand{\Hcal}{\mathcal{H}}         % total helicity
\newcommand{\SL}{\mathrm{SL}}           % self-linking number
\newcommand{\Lk}{\mathrm{Lk}}           % Gauss linking number
\newcommand{\Ck}{\mathcal{C}_k}         % kth tube volume (core neighborhood)
\newcommand{\Ci}{\mathcal{C}_i}
\newcommand{\Cj}{\mathcal{C}_j}




\begin{document}
    \title{Helicity in Swirl-String Theory (SST) Knot Systems}
    \author{Omar Iskandarani}
    \affiliation{Independent Researcher, Groningen, The Netherlands}
    \thanks{ORCID: 0009-0006-1686-3961, DOI: \paperdoi}

    \date{\today}
    \begin{abstract}
        This document presents a comprehensive framework for analyzing helicity in Swirl-String Theory (SST) knot systems. We derive key expressions for total helicity, self-linking, and mutual linking, providing a robust toolset for studying topological properties of these configurations.
    \end{abstract}
    \maketitle
    \section*{Objective}
        Compute the total helicity $\Hcal$ of a knotted/linked swirl-string configuration:
        \begin{equation}
            \boxed{
                \Hcal
                = \sum_{k} \int_{\Ck} \vv_k \!\cdot\! \ww_k \, dV
                \;+\; \sum_{i<j} 2\,\Lk_{ij}\, \Gamma_i \Gamma_j
            }
        \end{equation}
        This splits naturally into:
        \begin{itemize}
            \item \textbf{Self-helicity} (per component): twist $+$ writhe within each swirl string,
            \item \textbf{Mutual helicity}: pairwise contributions from topological linking between distinct strings.
        \end{itemize}



    \section*{1. Background (SST form)}
        \subsection*{a. Swirl velocity and vorticity}
            \begin{itemize}
                \item $\vv(\mathbf{r})$: local swirl velocity,
                \item $\ww = \nabla \times \vv$: vorticity.
            \end{itemize}

        \subsection*{b. Circulation (quantized)}
            \begin{equation}
                \Gamma_k = \oint_{\partial S_k} \vv \cdot d\boldsymbol{\ell}
                \qquad
                (\,[\Gamma] = \mathrm{m}^2/\mathrm{s}\,)
            \end{equation}
            In SST, $\Gamma_k$ is conserved (Kelvin) and takes integer quanta on closed strings, $\Gamma_k = n_k \kappa$ (not needed for computations here, but often convenient).

        \subsection*{c. Helicity (invariant absent reconnection)}
            \begin{equation}
                \Hcal = \int_V \vv \cdot \ww \, dV
            \end{equation}
            Helicity is a topological invariant for inviscid, incompressible flow; it changes only by reconnection or boundary flux.

    \section*{2. Deriving the decomposition}
        Assume $N$ disjoint, thin-core swirl strings $\mathcal{K}_1,\dots,\mathcal{K}_N$ with tubular neighborhoods $\Ck$.

        \subsection*{Step 1: Split by components}
            \begin{equation}
                \Hcal = \sum_{i=1}^N \Hcal_{\text{self}}^{(i)} \;+\; \sum_{i<j} \Hcal_{\text{mutual}}^{(i,j)}.
            \end{equation}

        \subsection*{Step 2: Self-helicity of component $k$}
            \begin{equation}
                \Hcal_{\text{self}}^{(k)} = \int_{\Ck} \vv_k \!\cdot\! \ww_k \, dV \;\approx\; \Gamma_k^2 \,\SL_k,
            \end{equation}
            where $\SL_k = \mathrm{Tw}_k + \mathrm{Wr}_k$ (Călugăreanu–White). For a standard trefoil embedding, $\SL_k \approx 3$.

        \subsection*{Step 3: Mutual helicity}
            \begin{equation}
                \Hcal_{\text{mutual}}^{(i,j)} = 2\,\Lk_{ij}\,\Gamma_i \Gamma_j,
            \end{equation}
            with $\Lk_{ij}$ the Gauss linking number between components $i$ and $j$.

        \subsection*{Final form (algebraic and integral)}
            \begin{equation}
                \boxed{
                    \Hcal = \sum_{i=1}^{N} \Gamma_i^2 \, \SL_i \;+\; \sum_{i<j}^{N} 2\,\Lk_{ij}\,\Gamma_i \Gamma_j
                }
            \end{equation}
            Equivalently,
            \begin{equation}
                \boxed{
                    \Hcal = \sum_{i=1}^{N} \int_{\Ci} \vv_i \!\cdot\! \ww_i \, dV
                    \;+\; \sum_{i<j} 2\,\Lk_{ij}\,\Gamma_i \Gamma_j
                }
            \end{equation}

    \section*{3. Practical recipe}
        \begin{enumerate}
            \item Fix the knot/link type (e.g.\ torus link $T(p,q)$), with $N=\gcd(p,q)$ components.
            \item Estimate circulation (per component) from core data; e.g.\ $\Gamma \approx 2\pi r_c\, \vnorm$ in a solid-body core model.
            \item Use $\SL_k$ and $\Lk_{ij}$ for the chosen embedding (trefoil baseline: $\SL_k=3$, nearest-neighbor link $\Lk_{ij}=1$).
            \item Evaluate:
            \[
                \Hcal \;=\; N\,\Gamma^2 \cdot \SL \;+\; 2\,\binom{N}{2}\,\Gamma^2 \,.
            \]
        \end{enumerate}

    \section*{4. Example: $T(18,27)$}
        \begin{itemize}
            \item $N=\gcd(18,27)=9$, \quad $\Gamma = 2\pi r_c C_e$,
            \item $\SL=3$, \quad $\binom{9}{2}=36$.
        \end{itemize}
        \begin{equation}
            \Hcal = 9\cdot \Gamma^2 \cdot 3 \;+\; 2\cdot 36 \cdot \Gamma^2
            \;=\; 27\Gamma^2 + 72\Gamma^2
            \;=\; 99\,\Gamma^2.
        \end{equation}


    \section*{5. Canonical Stability: Helicity-Energy Correspondence}
            \label{sec:energy}

    % Core Principle Enforcement: Relativistic/Quantum effects require energy and stability consideration.

        The total helicity, $\Hcal$, is the topological invariant ($\Hcal$-conservation). The conserved energy, $\mathcal{E}$, governs the string's dynamics ($\mathcal{E}$-conservation). Their relationship dictates the stable configurations of Swirl-String matter.

        \subsection*{A. Energy of a Swirl-String Configuration}
            The total energy $\mathcal{E}$ of the induced swirl-field in the background Swirl-Space (density $\rho_0$) is given by the kinetic energy integral:
            \begin{equation}
                \mathcal{E} = \frac{1}{2} \int_V \rho_0 \, \vv \cdot \vv \, dV
            \end{equation}
            For $N$ disjoint, thin-core strings $\mathcal{K}_i$, the energy decomposes into self-energy (dominated by length and core physics) and mutual energy (dominated by interaction geometry).

            \subsubsection*{i. Self-Energy $\mathcal{E}_{\text{self}}^{(i)}$}
                The self-energy of a string $\mathcal{K}_i$ of length $L_i$ and canonical core radius $r_c$ requires regularization. It is logarithmically dependent on the ratio of a characteristic domain radius $R_0$ to $r_c$:
                \begin{equation}
                    \mathcal{E}_{\text{self}}^{(i)} \approx \frac{\rho_0 \Gamma_i^2}{4\pi} L_i \left( \ln\left(\frac{R_0}{r_c}\right) + C_{\text{geom}} \right)
                \end{equation}
                where $C_{\text{geom}}$ is a constant accounting for the curvature and local field topology of the string's embedding. This term drives the string to minimize its length $L_i$ for a fixed core radius.

            \subsubsection*{ii. Mutual Energy $\mathcal{E}_{\text{mutual}}^{(i,j)}$}
                The mutual interaction energy between two strings $\mathcal{K}_i$ and $\mathcal{K}_j$ is given by the canonical double-line integral, which is a key measure of their inductive coupling:
                \begin{equation}
                    \mathcal{E}_{\text{mutual}}^{(i,j)} = \frac{\rho_0 \Gamma_i \Gamma_j}{4\pi} \oint_{\mathcal{K}_i} \oint_{\mathcal{K}_j} \frac{d\boldsymbol{\ell}_i \cdot d\boldsymbol{\ell}_j}{|\mathbf{r}_i - \mathbf{r}_j|}
                \end{equation}
                The total energy is $\mathcal{E} = \sum_{i} \mathcal{E}_{\text{self}}^{(i)} + \sum_{i<j} \mathcal{E}_{\text{mutual}}^{(i,j)}$.

        \subsection*{B. Variational Stability Principle}
            The stable, long-lived configurations of Swirl Strings (analogous to elementary particles or stable atomic structures) are determined by a variational principle: the equilibrium shape must minimize the total energy $\mathcal{E}$ while holding the canonical topological invariant $\Hcal$ fixed.

            This is formalized using the method of Lagrange multipliers:
            \begin{equation}
                \delta (\mathcal{E} - \lambda \Hcal) = 0
                \label{eq:variational}
            \end{equation}
            The quantity $\lambda$ is the \textbf{Energy-Helicity Ratio}, serving as the Lagrange multiplier, with dimensions $[\lambda] = [\mathcal{E}/\Hcal] = \mathrm{s}^{-1}$. In the canonical SST framework, $\lambda$ is directly related to a characteristic global field frequency ($\Omega_{\text{global}}$) or velocity-squared constant, $\lambda \propto \Omega_{\text{global}} \cdot \rho_0 / \Gamma_0$.

        \subsection*{C. Canonical Interpretation of $\Hcal/\mathcal{E}$}
            The ratio $\Hcal/\mathcal{E}$ (or its inverse) provides a measure of \textbf{canonical stability} and the complexity of the topological state:
            \begin{itemize}
                \item Configurations with $\lambda \approx 0$ (low helicity for high energy) tend to be dynamically unstable or highly localized.
                \item Configurations with a maximal $\lambda$ (maximal helicity for minimal energy) represent the most \textbf{compact and topologically robust states}—these are the prime candidates for stable fundamental particles or long-lived energy structures in the Swirl-Space. The $T(18,27)$ link, with its large $\Hcal$, represents a complex structure with significant topological charge.
            \end{itemize}

            \vspace{1em}
            \noindent\textit{Canonical Note:} The minimization of $\mathcal{E}$ for fixed $\Hcal$ provides the canonical derivation for the \textbf{relativistic stability} of knotted string solutions, preventing the unphysical decay of topological charge predicted by some earlier VAM formulations.

    \section*{Literature Foundations}
        \label{sec:literature}

        Topological helicity has evolved across multiple domains—fluid mechanics, plasma physics, quantum field theory, and most recently, Swirl-String Topology (SST). This section anchors SST in a broader theoretical context.

        \subsection*{A. Historical Origins and Classical Helicity}
            The concept of helicity was first introduced by \textbf{Moffatt (1969)} as a topological measure of the linkage of vortex lines. He established its conservation in ideal fluid flows and its connection to linking numbers, forming the backbone of self and mutual helicity decomposition.

        \subsection*{B. Knot Theory in Fluid Mechanics}
            \textbf{Ricca (1998)} and \textbf{Ricca \& Berger (1996)} extended Moffatt's ideas using tools from knot theory and differential geometry. They provided models for torus knots and highlighted the importance of self-linking and Seifert surfaces in representing helicity in tangled vortex structures.

        \subsection*{C. Reconnection-Resilient Helicity (Modern Experiments)}
            Recent experimental and numerical work by \textbf{Scheeler et al. (2014)} and \textbf{Salman (2017)} demonstrated that total helicity can remain conserved even under vortex reconnection events. These results directly support the invariance of $\Hcal$ in SST even when local geometry changes due to string interactions.

        \subsection*{D. Toward Quantum and String-Field Analogies}
            \textbf{Migdal (2021)} interprets vortex-sheet turbulence using a solvable string theory model, suggesting deep analogies between fluid knots and flux tubes in string field theory. In SST, where $\Hcal$ is tied to field energy and topological stability, this dual interpretation offers a pathway to unifying topological charge and string embeddings.

        \subsection*{E. Comparative Table of Helicity Forms}

            \vspace{1em}
            \begin{table}[h]
                \centering
                \footnotesize
                \begin{tabular}{lll}
                    \hline
                    \textbf{Domain} & \textbf{Helicity Definition} & \textbf{Remarks} \\
                    \hline
                    \textbf{Classical Fluid} & $\Hcal = \int \vv \cdot \ww \, dV$ & Vorticity topology; Moffatt invariant \\
                    \textbf{Quantum Fluid} & $\Hcal_q \propto n \cdot \mathrm{Twist} + \mathrm{Link}$ & Circulation quantized; reconnection effects \\
                    \textbf{SST (Swirl-String)} & $\Hcal = \sum \Gamma_k^2 \SL_k + \sum_{i<j} 2\Lk_{ij} \Gamma_i \Gamma_j$ & Self/mutual terms; conserved unless reconnection \\
                    \textbf{String-Theoretic} & $\Hcal \sim \int A \wedge dA$ (Chern-Simons) & Flux-tube or D-string interpretation; field dual \\
                    \hline
                \end{tabular}
                \caption{Comparative table of helicity forms across different domains.}\label{tab:helicity-forms}
            \end{table}


            \vspace{1em}
            \noindent\textit{Note:} SST helicity behaves analogously to Chern–Simons invariants in gauge theory, where linking of field lines carries conserved topological charge.

        \subsection*{F. Summary}
            Together, these references justify the structure of helicity decomposition in SST and its resilience under reconnection, while connecting it to broader topological and field-theoretic frameworks. They also motivate interpreting SST knots as fundamental dual structures in a canonical field-space.

    \section*{Summary Table (SST terms)}
        \begin{tabular}{|c|l|}
            \hline
            \textbf{Symbol} & \textbf{Meaning} \\
            \hline
            $\vv \cdot \ww$  & Local helicity density \\
            $\Gamma$        & Circulation around a string core (quantized, conserved) \\
            $\SL_k$         & Self-linking of component $k$ ($\mathrm{Tw}+\mathrm{Wr}$) \\
            $\Lk_{ij}$      & Gauss linking number between components $i,j$ \\
            $\Hcal$         & Total helicity (self $+$ mutual) \\
            \hline
        \end{tabular}

        \vspace{1em}
        \noindent\textit{Canonical notes (SST):} Kelvin circulation and helicity invariance apply in an inviscid,
        incompressible medium; $\Hcal$ changes only via reconnection or boundary flux.


    \begin{thebibliography}{9}\setlength{\itemsep}{1pt}
        \bibitem{moffatt1969degree} H. K. Moffatt, The degree of knottedness of tangled vortex lines, J. Fluid Mech. \textbf{35}, 117--129 (1969). \href{https://doi.org/10.1017/S0022112069000991}{doi:10.1017/S0022112069000991}
        \bibitem{arnold1998topological} V. I. Arnold and B. A. Khesin, \textit{Topological Methods in Hydrodynamics} (Springer, 1998). \href{https://doi.org/10.1007/978-1-4612-0645-3}{doi:10.1007/978-1-4612-0645-3}
        \bibitem{scheeler2014helicity} M. W. Scheeler, D. Kleckner, D. Proment, G. L. Kindlmann, W. T. M. Irvine, Helicity conservation by flow across scales in reconnecting vortex links and knots, Proc. Natl. Acad. Sci. \textbf{111}(43), 15350--15355 (2014). \href{https://doi.org/10.1073/pnas.1407232111}{doi:10.1073/pnas.1407232111}
        \bibitem{salman2017helicity} Hayder Salman, Helicity conservation and twisted Seifert surfaces for superfluid vortices, Proc. R. Soc. A \textbf{473}(2201), 20160853 (2017). \href{https://doi.org/10.1098/rspa.2016.0853}{doi:10.1098/rspa.2016.0853}
        \bibitem{ricca1998knot} R. L. Ricca, Applications of Knot Theory in Fluid Mechanics, Banach Center Publ. \textbf{42}, 321--346 (1998). \href{https://bibliotekanauki.pl/articles/1342426.pdf}{PDF}
        \bibitem{ricca1996topological} R. L. Ricca and M. A. Berger, Topological ideas and fluid mechanics, Phys. Today \textbf{49}(12), 28--34 (1996). \href{https://doi.org/10.1063/1.881576}{doi:10.1063/1.881576}
        \bibitem{migdal2021vortex} A. A. Migdal, Vortex sheet turbulence as solvable string theory, Int. J. Mod. Phys. A \textbf{36}(17), 2150062 (2021). \href{https://doi.org/10.1142/S0217751X21500627}{doi:10.1142/S0217751X21500627}
    \end{thebibliography}

\end{document}