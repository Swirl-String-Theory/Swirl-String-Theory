% ======================================================================
% InverseSquare_SST_Followup.tex
% Follow-up paper: Solving the 1/r (inverse-square) "open problem" in SST
% Three mainstream derivations; SST-specific connections only in appendices.
% ======================================================================

\documentclass[11pt]{article}
\usepackage[margin=1in]{geometry}
\usepackage{amsmath,amssymb,amsfonts}
\usepackage{microtype}
\usepackage[hidelinks]{hyperref}

\title{A First-Principles Origin of the Inverse-Square Law in Swirl--String Theory:\\
Three Derivations from Local Field Mediation and Momentum-Flux Conservation}
\author{Omar Iskandarani}
\date{}

\newcommand{\paperdoi}{10.5281/zenodo.18388684}

\begin{document}
\maketitle

\begin{abstract}
A recurrent objection to emergent or flat-background gravity programs is that the inverse-square distance law is often imported rather than derived. In the static, weak-field, spherically symmetric (monopole) sector, we close this gap by exhibiting three independent derivations of the $1/r$ potential and $1/r^2$ flux. (I) A minimal Gauss-law scalar effective field theory (EFT) for the far-field mediator yields the Poisson equation, whose Green's function in $\mathbb{R}^3$ is $1/r$. (II) We identify the SST far-field carrier as a foliation (``Swirl-Clock'') scalar perturbation, write its quadratic EFT, compute its spatial stress tensor $T_{ij}$, and show that its conserved radial momentum-flux density satisfies $\mathcal{F}_r\propto 1/r^2$ with total charge $Q\propto\int\rho_m d^3x$. (III) We give an SST-compatible replacement of an auxiliary scalar $\chi$ by the foliation scalar, showing that the static weak-field monopole sector reduces to the Gauss-law form of (I), so the $1/r^2$ flux follows automatically. The remaining model-dependent content is the overall coupling normalization, to be matched to the SST expression $G_{\rm swirl}$ in a separate step.
\end{abstract}

\section{Setup and scope}\label{sec:setup}
We work on a flat operational background with Minkowski causal structure and consider the weak-field, static, spherically symmetric (monopole) sector. Let $\rho_m(\mathbf{x})$ be the rest-mass density of a compact source with total mass
\begin{equation}
M \;=\;\int_{\mathbb{R}^3}\rho_m(\mathbf{x})\,d^3x.
\end{equation}
The objective is to derive, from local field mediation and momentum-flux conservation, that the far-field influence generated by $M$ must exhibit
\begin{equation}
\Phi(\mathbf{x}) \sim \frac{1}{r},\qquad \nabla \Phi \sim \frac{1}{r^2},\qquad r=\|\mathbf{x}\|.
\end{equation}

\section{Derivation I: Gauss-law scalar EFT $\Rightarrow$ $1/r$ Green's function}\label{sec:der1}
\subsection{Minimal mediator and quadratic static EFT}
In the static weak-field monopole sector, the minimal local mediator is a scalar field $\phi(\mathbf{x})$ coupled linearly to the source density. The most general rotationally invariant quadratic functional is
\begin{equation}\label{eq:S_static}
S_{\rm stat}[\phi]
\;=\;
\int_{\mathbb{R}^3} d^3x\,
\left[
\frac{\kappa}{2}\,(\nabla \phi)^2 \;-\; \lambda\,\phi\,\rho_m(\mathbf{x})
\right],
\end{equation}
with constants $\kappa>0$ and coupling $\lambda$.

\subsection{Euler--Lagrange equation: Poisson form}
Varying \eqref{eq:S_static},
\begin{align}
\delta S_{\rm stat}
&=
\int d^3x\,
\left[
\kappa\,\nabla\phi\cdot\nabla(\delta\phi) \;-\; \lambda\,\rho_m\,\delta\phi
\right]\\
&=
\int d^3x\,
\left[
-\kappa\,(\nabla^2\phi)\,\delta\phi \;-\; \lambda\,\rho_m\,\delta\phi
\right]
\quad (\text{integrate by parts, drop boundary term}),
\end{align}
so stationarity for arbitrary $\delta\phi$ gives
\begin{equation}\label{eq:Poisson_general}
\kappa\,\nabla^2\phi(\mathbf{x}) \;=\; -\,\lambda\,\rho_m(\mathbf{x}).
\end{equation}

\subsection{Green's function and the far-field $1/r$ solution}
Let $G(\mathbf{x})$ satisfy
\begin{equation}\label{eq:Green_def}
\nabla^2 G(\mathbf{x}) = -4\pi\,\delta^{(3)}(\mathbf{x}).
\end{equation}
On $\mathbb{R}^3$, the spherically symmetric fundamental solution is
\begin{equation}\label{eq:Green_1overr}
G(\mathbf{x}) = \frac{1}{\|\mathbf{x}\|},
\end{equation}
a standard result in potential theory \cite{Jackson1999,Arfken2013}.
Convolving \eqref{eq:Poisson_general} with \eqref{eq:Green_def} gives
\begin{equation}\label{eq:phi_convolution}
\phi(\mathbf{x})
=
\frac{\lambda}{4\pi\kappa}\int d^3x'\,
\frac{\rho_m(\mathbf{x}')}{\|\mathbf{x}-\mathbf{x}'\|}.
\end{equation}
In the far field $r\gg$ source size,
\begin{equation}\label{eq:phi_far}
\phi(r)\;\simeq\;\frac{\lambda}{4\pi\kappa}\,\frac{1}{r}\int\rho_m(\mathbf{x}')\,d^3x'
\;=\;\frac{\lambda}{4\pi\kappa}\,\frac{M}{r}.
\end{equation}
Therefore
\begin{equation}\label{eq:gradphi_far}
\nabla\phi(r)\;\simeq\;-\frac{\lambda M}{4\pi\kappa}\,\frac{\hat{\mathbf{r}}}{r^2}.
\end{equation}

\subsection{Gauss law and inverse-square flux}
Define the flux density
\begin{equation}\label{eq:J_def}
\mathbf{J} := -\,\kappa\,\nabla\phi.
\end{equation}
Then \eqref{eq:Poisson_general} becomes $\nabla\cdot\mathbf{J}=\lambda\rho_m$. Integrating over a ball $B_R$ and using the divergence theorem:
\begin{equation}\label{eq:gauss}
\oint_{S_R}\mathbf{J}\cdot d\mathbf{A}
=
\lambda\int_{B_R}\rho_m\,d^3x
\;\xrightarrow{R\to\infty}\;
\lambda M.
\end{equation}
Spherical symmetry implies $\mathbf{J}=J_r(r)\hat{\mathbf{r}}$, so
\begin{equation}\label{eq:Jr_1overr2}
4\pi r^2\,J_r(r)=\lambda M
\quad\Rightarrow\quad
J_r(r)=\frac{\lambda M}{4\pi}\,\frac{1}{r^2}.
\end{equation}
Thus the inverse-square law follows from locality and the $\mathbb{R}^3$ Green's function structure.

\paragraph{Analogy (for a child).}
If something spreads out equally in all directions, it must cover a bigger sphere as you go farther away, so each square meter gets less and less---like $1/r^2$.

\section{Derivation II: Foliation scalar as far-field carrier; $T_{ij}$ and $\mathcal{F}_r\propto 1/r^2$}\label{sec:der2}
\subsection{Field identification and quadratic EFT}
In SST language, the long-range static degree of freedom is taken to be a foliation (``Swirl-Clock'') scalar $T(x)$ that labels preferred-time hypersurfaces. Consider perturbations about an inertial foliation:
\begin{equation}
T(x) = t + \tau(x),
\end{equation}
where $t$ is the operational background time coordinate and $\tau$ is a weak perturbation sourced by matter. At quadratic order, the minimal Lorentzian EFT is
\begin{equation}\label{eq:tau_Lorentz}
S[\tau]
=
\int d^4x\,
\left[
\frac{\kappa}{2}\,\partial_\mu \tau\,\partial^\mu \tau
-\lambda\,\tau\,\rho_m(\mathbf{x})
\right],
\end{equation}
with $\partial_t\rho_m=0$ (static sources). The static sector reduces to \eqref{eq:S_static} with $\phi\equiv\tau$.

\subsection{Stress-energy tensor}
For the free part of \eqref{eq:tau_Lorentz}, the symmetric stress-energy tensor is
\begin{equation}\label{eq:Tmunu_tau}
T_{\mu\nu}^{(\tau)}
=
\kappa\left(
\partial_\mu \tau\,\partial_\nu \tau
-\frac{1}{2}\eta_{\mu\nu}\,\partial_\alpha\tau\,\partial^\alpha\tau
\right).
\end{equation}
In the static regime $\partial_0\tau=0$, hence $\partial_\alpha\tau\,\partial^\alpha\tau = -(\nabla\tau)^2$ and
\begin{equation}\label{eq:Tij_static}
T_{ij}^{(\tau)}
=
\kappa\left(
\partial_i\tau\,\partial_j\tau
-\frac{1}{2}\delta_{ij}(\nabla\tau)^2
\right),
\qquad
T_{00}^{(\tau)}=\frac{\kappa}{2}(\nabla\tau)^2.
\end{equation}

\subsection{Monopole solution, Gauss charge, and radial momentum flux}
The field equation from \eqref{eq:tau_Lorentz} is $\kappa\nabla^2\tau=-\lambda\rho_m$, so in the exterior region the monopole solution is
\begin{equation}\label{eq:tau_mono}
\tau(r) = \frac{\lambda}{4\pi\kappa}\,\frac{Q}{r},
\qquad
Q:=\int\rho_m\,d^3x \;=\; M,
\end{equation}
and
\begin{equation}\label{eq:dtaudr}
\partial_r\tau(r)= -\frac{\lambda Q}{4\pi\kappa}\,\frac{1}{r^2}.
\end{equation}
Define the conserved far-field momentum-flux density (Gauss flux) carried by the foliation scalar:
\begin{equation}\label{eq:flux_def}
\mathcal{F}_r(r)
:=
-\kappa\,\partial_r\tau(r).
\end{equation}
Using \eqref{eq:dtaudr},
\begin{equation}\label{eq:flux_1overr2}
\boxed{\;
\mathcal{F}_r(r)
=
\frac{\lambda Q}{4\pi}\,\frac{1}{r^2},
\qquad
Q=\int\rho_m\,d^3x.
\;}
\end{equation}
This is the requested monopole scaling.

\subsection{From $T_{ij}$ to integrated inverse-square force}
The radial traction (normal stress) from \eqref{eq:Tij_static} is
\begin{equation}
t_r(r)
:=
\hat{r}_i\,T_{ij}^{(\tau)}\,\hat{r}_j
=
\frac{\kappa}{2}\,(\partial_r\tau)^2
\propto \frac{1}{r^4}.
\end{equation}
The integrated force across a sphere of radius $r$ is
\begin{equation}
F(r)=\int_{S_r} t_r\,dA
\sim
4\pi r^2\times \frac{1}{r^4}
\sim
\frac{1}{r^2},
\end{equation}
consistent with inverse-square behavior.

\paragraph{Analogy (for a child).}
The field's slope gets weaker like $1/r^2$; stress depends on slope squared, so it weakens even faster, but when you multiply by the sphere's area you get back $1/r^2$.

\section{Derivation III: SST replacement of $\chi$ by foliation scalar; static weak-field $\Rightarrow$ Gauss-law form}\label{sec:der3}
\subsection{Static EFT inevitability from symmetry and locality}
Any SST-compatible long-range mediator in the weak-field static monopole sector must be described by a local scalar functional whose leading term is quadratic in first derivatives. Denoting the relevant scalar by $\tau$, the leading static effective Lagrangian must take the Gauss-law form
\begin{equation}\label{eq:Leff_tau_static}
\mathcal{L}_{\rm eff,stat}
=
\frac{\kappa}{2}\,(\nabla\tau)^2
-\lambda\,\tau\,\rho_m
\;+\;\mathcal{O}(\nabla^4,\tau^3).
\end{equation}
Higher-derivative terms renormalize the near-zone but do not change the far-field $1/r$ Green's function behavior in the monopole sector.

\subsection{Replacement map $\chi\mapsto\tau$ and recovery of Poisson}
Let $\chi$ be an auxiliary scalar used to package the weak-field influence. Replace it by a rescaled foliation perturbation:
\begin{equation}\label{eq:chi_map}
\chi := \alpha\,\tau,
\end{equation}
where $\alpha$ is a constant fixed by matching to the Newtonian limit. Then \eqref{eq:Leff_tau_static} becomes
\begin{equation}
\mathcal{L}_{\rm eff,stat}
=
\frac{\kappa}{2\alpha^2}\,(\nabla\chi)^2
-\frac{\lambda}{\alpha}\,\chi\,\rho_m.
\end{equation}
Varying yields
\begin{equation}\label{eq:Poisson_chi}
\nabla^2\chi = -\left(\frac{\alpha\lambda}{\kappa}\right)\rho_m.
\end{equation}
Imposing the standard Poisson normalization
\begin{equation}\label{eq:match_4piG}
\frac{\alpha\lambda}{\kappa} = 4\pi G,
\end{equation}
we recover
\begin{equation}
\chi(r) = -\frac{GM}{r},
\qquad
\nabla\chi(r)= +GM\,\frac{\hat{\mathbf{r}}}{r^2}.
\end{equation}
Hence, once the foliation scalar reduces to the Gauss-law static form, the inverse-square flux follows automatically.

\paragraph{Analogy (for a child).}
If two different ``field names'' obey the same rule, they make the same shape---so both give $1/r$.

\section{Discussion and outlook (normalization and $G_{\rm swirl}$ kept separate)}\label{sec:outlook}
The distance-law problem admits a purely structural solution: locality, rotational invariance, and propagation in three spatial dimensions force the monopole solution of the Laplacian to behave as $1/r$, yielding $1/r^2$ flux. What remains model-dependent is the overall normalization: in the EFT language, the coupling combination $\alpha\lambda/\kappa$ in \eqref{eq:match_4piG}. In SST, this normalization is to be matched to a derived gravitational coupling $G_{\rm swirl}$ constructed from microphysical constants. That matching is orthogonal to the inverse-square derivation itself: it fixes the coefficient of the already-determined Green's function.

\appendix
\section*{Appendix A: Connection to other SST manuscripts (informational, not cited in main text)}
This appendix outlines how the present structural result interfaces with other SST lines of development, without relying on those manuscripts as premises. First, Kelvin-mode suppression supports treating ordinary atomic sources as effectively rigid in the far zone, justifying a monopole approximation for $\rho_m(\mathbf{x})$ in \eqref{eq:Poisson_general}. Second, SST thermodynamic formulations provide a route to compute the EFT stiffness $\kappa$ and coupling $\lambda$ by coarse-graining a microscopic equation of state and identifying the foliation mode's kinetic term. Third, variational selection principles for stable configurations can constrain admissible couplings and potentially fix $\alpha$ in \eqref{eq:chi_map}, thereby enabling an explicit match to $G_{\rm swirl}$.

\begin{thebibliography}{99}

\bibitem{Jackson1999}
J.~D.~Jackson,
\textit{Classical Electrodynamics}, 3rd ed.,
Wiley (1999).

\bibitem{Arfken2013}
G.~B.~Arfken, H.~J.~Weber, and F.~E.~Harris,
\textit{Mathematical Methods for Physicists}, 7th ed.,
Academic Press (2013).

\end{thebibliography}

\end{document}
