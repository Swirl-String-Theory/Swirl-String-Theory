\documentclass[a4paper,11pt]{letter}

\usepackage[T1]{fontenc}
\usepackage[utf8]{inputenc}
\usepackage{lmodern}
\usepackage[hidelinks]{hyperref}
\usepackage{microtype}
\usepackage[margin=1in]{geometry}
\usepackage{amstext}
\usepackage{amssymb}

% Sender info
\signature{Omar Iskandarani\\
Independent Researcher, Groningen,\\ The Netherlands\\
ORCID: 0009-0006-1686-3961\\
Email: \href{mailto:info@omariskandarani.com}{info@omariskandarani.com}}

\address{Omar Iskandarani\\
Vinkenstraat 86A\\
9713 TK Groningen\\
The Netherlands}

\date{\today}

\begin{document}

    \begin{letter}{Editors\\\textit{Foundations of Physics}}

        \opening{Dear Editor,}

        I am pleased to submit the manuscript
        \textit{Relational Time-of-Arrival as a Covariant Field Observable: From Event Currents to a Continuum Clock Limit}
        for consideration as a Research Article in \textit{Foundations of Physics}.

        The status of time observables in quantum theory remains conceptually unsettled. Pauli-type arguments obstruct the existence of a self-adjoint time operator conjugate to energy, while relativistic quantum field theory lacks a universal, observer-independent notion of arrival time. In this manuscript, I propose a complementary approach: time-of-arrival (TOA) is formulated as a \emph{relational field observable}, constructed from two conserved structures—a matter flux through a detector world-tube and an event current whose coarse-grained limit defines a physical clock field governed by an effective field theory.

        The construction is manifestly covariant and avoids operator-based difficulties by never postulating a universal time operator. Discrete event counts and smooth clock readings emerge as ultraviolet and infrared descriptions of the same underlying structure. In a worked $(1+1)$-dimensional example, the TOA distribution is obtained explicitly as a convolution of the matter flux with clock correlations, yielding a calculable arrival-time broadening and an exponential suppression \emph{envelope} for early-arrival contributions set by the clock-sector mass scale.

        \textbf{What is new and potentially of interest.}
        \begin{itemize}\setlength\itemsep{0.25em}
        \item Time-of-arrival is defined directly at the field-theoretic level as a relational functional of conserved currents and a physical clock, independent of detector microphysics up to geometry and flux.
        \item Discrete event-count clocks and continuum clock fields are unified through an explicit UV--IR coarse-graining, making time an emergent infrared observable rather than a fundamental operator.
        \item The clock sector carries its own effective dynamics, whose correlators enter TOA statistics and fix observable broadening and suppression scales.
        \item A complete $(1+1)$D worked example demonstrates explicitly how clock fluctuations modify semiclassical arrival times.
        \end{itemize}

        \textbf{Falsifiability and predictive content.}
        Given a measured flux profile $\mathcal{F}(t)$ and a clock variance $\sigma_\tau^2$, the TOA distribution is fixed by a convolution formula and predicts an excess variance beyond wavepacket dispersion. While local clock readouts produce Gaussian one-point tails, spacelike clock correlations impose an exponential suppression envelope governed by the clock mass scale $\mu_\tau$, yielding a concrete and testable bound. At sufficiently high temporal resolution, discrete event-count clocks additionally predict renewal-type deviations that vanish under coarse-graining.

        I believe this manuscript aligns well with \textit{Foundations of Physics} in its emphasis on conceptual clarity, explicit assumptions, and physically interpretable structures, while remaining close to standard quantum field-theoretic tools.

        A timestamped preprint is available on Zenodo DOI: \href{https://doi.org/10.5281/zenodo.18050157}{10.5281/zenodo.18050157}. The submission is original, not under consideration elsewhere, and involves no external funding or conflicts of interest.

        Thank you for your time and consideration.

        \closing{Sincerely,}

    \end{letter}

\end{document}