% =====================================================================
% Foundations of Physics (FoP) -- Personal Reusable Submission Template
% Omar Iskandarani (adaptable to Springer Nature sn-jnl class)
%
% Notes:
% - Single .tex file (no \input{...}); figures submitted separately.
% - Keep this template lean; add packages only when needed.
% - Prefer \cite{key} with a .bib file, OR paste .bbl content if required.
% =====================================================================

% =========================
% 1) Document class choice
% =========================
% referee option -> double line spacing for review
% \documentclass[referee,pdflatex,sn-mathphys-num]{sn-jnl}
\documentclass[pdflatex,sn-mathphys-num]{sn-jnl}

% =========================
% 2) Core packages (minimal)
% =========================
\usepackage[T1]{fontenc}
\usepackage[utf8]{inputenc}
\usepackage{lmodern}
\usepackage{microtype}
\usepackage{graphicx}
\usepackage{booktabs}
\usepackage{amsmath,amssymb,amsfonts}
\usepackage{amsthm}
\usepackage[hidelinks]{hyperref}

% Optional (uncomment only when required)
% \usepackage{physics}        % convenience macros; use with care
% \usepackage{mathtools}      % extra math tools
% \usepackage{siunitx}        % SI units
% \usepackage{bm}             % bold math
% \usepackage{xcolor}         % color (avoid in final; XML issues)
% \usepackage[title]{appendix}
% \usepackage{algorithm}
% \usepackage{algorithmicx}
% \usepackage{algpseudocode}
% \usepackage{listings}

% ====================================
% 3) Theorem environments (as needed)
% ====================================
\theoremstyle{thmstyleone}
\newtheorem{theorem}{Theorem}
\newtheorem{proposition}[theorem]{Proposition}

\theoremstyle{thmstyletwo}
\newtheorem{remark}{Remark}

\theoremstyle{thmstylethree}
\newtheorem{definition}{Definition}

% ====================================
% 4) Paper metadata macros (edit here)
% ====================================
\newcommand{\papertitle}{<FULL PAPER TITLE>}
\newcommand{\papershorttitle}{<SHORT RUNNING TITLE>}
\newcommand{\paperdoi}{10.5281/zenodo.xxx} % e.g., 10.5281/zenodo.xxxxx or arXiv:xxxx.xxxxx
\newcommand{\paperorcid}{0009-0006-1686-3961}
\newcommand{\paperemail}{info@omariskandarani.com}
\newcommand{\paperlocation}{Groningen, The Netherlands}

% Keywords (comma separated)
\newcommand{\paperkeywords}{relational time, equivalence principle, teleparallel gravity, causal structure, quantum clocks}

% ====================================
% 5) Author block (single-author default)
%    Add more authors by copying the pattern.
% ====================================
\raggedbottom
% \unnumbered % uncomment if you want unnumbered section heads (rare)

\begin{document}

    \title[\papershorttitle]{\papertitle}

% ---- Single-author default (edit affiliation as needed)
    \author*[1]{\fnm{Omar} \sur{Iskandarani}}\email{\paperemail}
% Optional: ORCID sometimes supported in the PDF metadata; keep in text as needed:
% \author*[1]{\fnm{Omar} \sur{Iskandarani}}\email{\paperemail}\orcid{\paperorcid}

    \affil*[1]{\orgname{Independent Researcher}, \orgaddress{\city{\paperlocation}, \country{The Netherlands}}}

% ====================================
% 6) Abstract (FoP typically: unstructured; no equations/citations ideally)
% ====================================
    \abstract{%
        <ABSTRACT HERE. Keep it factual, non-technical, and usually avoid citations/equations unless absolutely necessary.>
    }

    \keywords{\paperkeywords}

    \maketitle

% ============================================================
% 7) Main text skeleton (edit section names to your paper)
% ============================================================

    \section{Introduction}\label{sec:intro}
% - Motivation and context
% - Key contributions (bulleted paragraph is acceptable)
% - Paper organization

    \section{Background and Prior Work}\label{sec:background}
% - Keep concise; cite the most relevant foundational papers
% Example: \cite{PageWootters1983,Zeeman1964,Will2014}

    \section{Framework and Definitions}\label{sec:framework}
% - Assumptions, definitions, notation
% - If using multiple geometric formulations, define metric/connection conventions

    \section{Main Results}\label{sec:results}
% - State the main propositions/theorems first
% - Then derivations/proofs/arguments

        \subsection{Result 1: <name>}\label{subsec:r1}
% Derivation / proof / argument

        \subsection{Result 2: <name>}\label{subsec:r2}
% Derivation / proof / argument

    \section{Physical Interpretation and Limits}\label{sec:interpretation}
% - Regime of validity
% - Known limits (weak-field, low-energy, classical limit)
% - Edge cases and potential failure modes

    \section{Phenomenology and Experimental Outlook}\label{sec:phenom}
% - What is constrained already
% - What could be tested
% - Order-of-magnitude estimates (optional)

    \section{Discussion}\label{sec:discussion}
% - Comparison with alternatives
% - Conceptual implications
% - Open problems

    \section{Conclusion}\label{sec:conclusion}
% - Short summary of what was shown
% - What is new
% - Next steps

% ============================================================
% 8) Back matter (FoP / Springer: use \backmatter)
% ============================================================
        \backmatter

% -----------------
% Acknowledgements
% -----------------
        \bmhead{Acknowledgements}
        Not applicable.

% -----------------
% Declarations
% -----------------
        \bmhead{Declarations}

        \subsection*{Funding}
            Not applicable.

        \subsection*{Competing interests}
            The author declares no competing interests.

        \subsection*{Ethics approval}
            Not applicable.

        \subsection*{Consent to participate}
            Not applicable.

        \subsection*{Consent for publication}
            Not applicable.

        \subsection*{Data availability}
            All data generated or analyzed during this study are included in this published article and its supplementary information (if any).

        \subsection*{Code availability}
            Source code (if applicable) will be made available in a public repository upon publication / is available at: \url{<REPOSITORY URL OR ZENODO DOI>}.

        \subsection*{Author contributions}
            O.I. conceived the study, developed the analysis, performed the derivations, and wrote the manuscript.

% ============================================================
% 9) Appendices (optional)
% ============================================================
% \begin{appendices}
% \section{Technical Lemmas}\label{app:lemmas}
% \section{Supplementary Derivations}\label{app:derivations}
% \end{appendices}

% ============================================================
% 10) References
% ============================================================
% Option A: BibTeX database file (recommended for iteration)
            \bibliography{sn-bibliography} % rename to your .bib filename (without .bib)

% Option B: Paste the .bbl content here for single-file submission if required
% \begin{thebibliography}{99}
% \bibitem{PageWootters1983} ...
% \end{thebibliography}

\end{document}