% =====================================================================
% Foundations of Physics (FoP) -- Converted Manuscript to Reusable Template
% Omar Iskandarani
% Date: 2025-12-22
% Affiliation: Independent Researcher, Groningen, The Netherlands
% License: © 2025 Omar Iskandarani. All rights reserved. Academic reading/citation only.
% ORCID: 0009-0006-1686-3961
% DOI: 10.5281/zenodo.18018114
%
% Notes:
% - Single .tex file (no \input{...}); figures attached separately.
% - This file is already in the Springer Nature sn-jnl format (FoP-compatible).
% =====================================================================

% =========================
% 1) Document class choice
% =========================
% referee option -> double line spacing for review
% \documentclass[referee,pdflatex,sn-mathphys-num]{sn-jnl}
\documentclass[pdflatex,sn-mathphys-num]{sn-jnl}

% =========================
% 2) Core packages (keep lean)
% =========================
% NOTE: sn-jnl.cls already loads hyperref; to avoid option clashes we
% do not load it again here. Use sn-jnl's built-in hyperlink setup.
\usepackage[T1]{fontenc}
\usepackage[utf8]{inputenc}
\usepackage{lmodern}
\usepackage{microtype}
\usepackage{graphicx}
\usepackage{booktabs}
\usepackage{amsmath,amssymb,amsfonts}
\usepackage{amsthm}

% Optional (only if needed)
% \usepackage{bm}
% \usepackage{siunitx}
% \usepackage[title]{appendix}
% \usepackage{xcolor}

% ====================================
% 3) Theorem environments (as needed)
% ====================================
\theoremstyle{thmstyleone}
\newtheorem{theorem}{Theorem}
\newtheorem{proposition}[theorem]{Proposition}

\theoremstyle{thmstyletwo}
\newtheorem{remark}{Remark}

\theoremstyle{thmstylethree}
\newtheorem{definition}{Definition}

% ====================================
% 4) Paper metadata macros (edit here)
% ====================================
\newcommand{\paperdoi}{10.5281/zenodo.18018114}
\newcommand{\papertitle}{Emergent Equivalence Principle from Relational Time and Connection Dynamics}
\newcommand{\papershorttitle}{Emergent Equivalence Principle}
\newcommand{\paperorcid}{0009-0006-1686-3961}
\newcommand{\paperemail}{info@omariskandarani.com}
\newcommand{\papercity}{Groningen}
\newcommand{\papercountry}{The Netherlands}
\newcommand{\paperkeywords}{relational time, equivalence principle, teleparallel gravity, causal structure, quantum clocks}

% ====================================
% 5) Document starts
% ====================================
\raggedbottom

\begin{document}

    \title[\papershorttitle]{\papertitle}

% ---- Single-author block (FoP style)
    \author*[1]{\fnm{Omar} \sur{Iskandarani}}\email{\paperemail}
    \affil*[1]{\orgname{Independent Researcher}, \orgaddress{\city{\papercity}, \country{\papercountry}}}

% ====================================
% 6) Abstract and keywords
% ====================================
    \abstract{%
        The Equivalence Principle (EP) has long anchored the geometric reading of gravitation, yet its foundational status is less clear than the historical narrative suggests. Three lines of work point in the same direction: (i) relativistic kinematics can be fixed on operational grounds from causality alone; (ii) relational formulations of quantum dynamics generate effective time-evolution from clock--system correlations; and (iii) the ``geometric trinity'' shows that curvature-, torsion-, and non-metricity-based formulations are dynamically equivalent while conceptually distinct. I bring these strands together in a conservative framework. Lorentzian causal structure is treated as operationally primary. Mild deformations of a relational Hamiltonian constraint yield gravitational redshift and Newtonian-like interaction terms at low energy. In this setting, the EP reads most naturally as an \emph{emergent} organizing symmetry rather than a primitive axiom. I isolate which parts of the usual EP story are logically prior, which are recovered asymptotically, and where coherence-dependent corrections to redshift should appear. The role of the affine connection is made explicit in mediating inertial and gravitational effects, and the regime of validity is stated in operational terms amenable to quantum-clock tests.%
    }

    \keywords{\paperkeywords}

    \maketitle

% ============================================================
    \section{Introduction}\label{sec:intro}
        The Equivalence Principle (EP) is often presented as the conceptual cornerstone of General Relativity (GR), justifying the identification of gravity with spacetime geometry and the universality of free fall \cite{Will2014}. At the same time, several mature ideas now suggest a different ordering of concepts. First, special-relativistic kinematics can be derived from assumptions about observers, signals, and causal order, with no privileged role for electromagnetism \cite{Pineda2026,Zeeman1964}. Second, in relational-time approaches to quantum dynamics, effective redshift and Newtonian-like terms arise from clock--energy couplings in a global constraint, even for otherwise non-interacting subsystems \cite{SinghFriedrich2025,PageWootters1983}. Third, the geometric trinity shows that curvature-, torsion-, and non-metricity-based theories can be dynamically equivalent while remaining conceptually distinct \cite{ManciniTinoCapozziello2025,BeltranJimenez2018,AldrovandiPereira2013}.

        While the EP has traditionally anchored the geometric reading of gravity, its role is not uniformly fundamental across modern formulations. In particular, operational reconstructions of spacetime structure from causality, and quantum-relational approaches to time, both suggest that certain gravitational effects can be viewed as emergent rather than postulated. This motivates a reordering of logical assumptions: rather than taking the EP as primitive, one may ask whether it can be derived from more basic relational and dynamical principles.

        \paragraph{Aim.}
            I give a structural account that respects established phenomenology while reordering the logical dependencies: (i) causality fixes the inertial scaffold; (ii) relational time generates effective gravitational structure at low energy; and (iii) the same empirical content admits inequivalent geometric codings. Under this interpretation, the EP becomes an emergent low-energy symmetry.

        \paragraph{Scope.}
            No new long-range forces are proposed, and no composition-dependent violations are assumed. The goal is to clarify the domain in which the EP is best interpreted as emergent and to identify concrete, coherence-sensitive signatures for quantum-clock experiments.

        \paragraph{Contributions.}
            \begin{enumerate}
                \item An operational summary of how causal order fixes Lorentzian inertial structure (Sec.~\ref{sec:causality}).
                \item A compact operator statement for the low-energy expansion of the relational effective Hamiltonian and its universal energy--energy cross-term (Sec.~\ref{sec:relationaltime}, Prop.~\ref{prop:neumann}).
                \item A perspectival reading that reconciles relational time with connection-based encodings without a globally preferred clock (Sec.~\ref{sec:perspectival}).
                \item A clear account of weak-field redshift and the Newtonian limit in the relational picture, and what is still open about distance dependence (Sec.~\ref{sec:redshift_newton}).
                \item A use of the geometric trinity as geometric underdetermination of shared empirical content (Sec.~\ref{sec:trinity}).
                \item A synthesis in which the EP is an emergent organizing symmetry, with quantum-clock tests singled out as near-term probes (Secs.~\ref{sec:emergence}--\ref{sec:phenom}).
            \end{enumerate}

% ============================================================
    \section{Causality as the Origin of Relativistic Kinematics}\label{sec:causality}

    \subsection{Observers, events, and an operational start}\label{subsec:observers}
        Treat an observer as a localized system carrying a clock and the ability to send/receive signals. Events are indexed by that clock’s reading. This minimal setup captures the idea that spacetime notions are fixed by procedures of registration and communication \cite{Pineda2026}.

    \subsection{Finite signaling and a universal distance}\label{subsec:distance}
        If exchanging information takes finite time, round-trip signal times define distance. Minimizing over admissible messengers picks out a limiting signal class and a maximal transfer speed, with no need to reference a specific field \cite{Pineda2026}.

    \subsection{Causal automorphisms and the Lorentz group}\label{subsec:lorentz}
        Under mild spatial regularity (Euclidean distance in inertial frames), transformations that preserve causal order form the inhomogeneous Lorentz group (up to dilatations) by Alexandrov–Zeeman-type results \cite{Zeeman1964}. Causality thus fixes the Lorentzian kinematic backbone.

% ============================================================
    \section{Relational Time in Quantum Mechanics}\label{sec:relationaltime}

    \subsection{Page--Wootters in brief}\label{subsec:pw}
        Relational-time schemes encode dynamics in correlations between a clock and a system. A stationary global constraint coexists with conditional Schrödinger evolution at fixed clock readings \cite{PageWootters1983,Wootters1984,DeWitt1967}.

    \subsection{A simple deformation and clock--energy coupling}\label{subsec:deformations}
        To model backreaction-like effects, consider
        \begin{equation}
            \hat{J} \equiv \hat{p}_t\otimes \hat{I}_S + \hat{I}_t\otimes \hat{H}_S + \frac{1}{\Lambda}\,\hat{p}_t\otimes \hat{H}_S \approx 0,
            \label{eq:deformed_constraint}
        \end{equation}
        with deformation scale $\Lambda$ \cite{SinghFriedrich2025}. Conditioning on the clock yields an energy-dependent phase rate summarized by
        \begin{equation}
            \hat{H}_{\rm eff} = \hat{H}_S \bigl(\hat{I}_S + \hat{H}_S/\Lambda\bigr)^{-1},
            \label{eq:Heff}
        \end{equation}
        on suitable domains (App.~\ref{app:domains}).

        \begin{proposition}[Low-energy expansion of the relational effective Hamiltonian]
            \label{prop:neumann}
            Let $\hat{H}_S$ be self-adjoint and spectrally supported on a subspace with $\|\hat{H}_S\|<\Lambda$ for some $\Lambda>0$. Then
            \[
                (\hat{I}_S+\hat{H}_S/\Lambda)^{-1}=\sum_{n=0}^{\infty}(-1)^n(\hat{H}_S/\Lambda)^n,
            \]
            so
            \[
                \hat{H}_{\rm eff}=\hat{H}_S-\frac{1}{\Lambda}\hat{H}_S^2+O(\Lambda^{-2}) \qquad (\|\hat{H}_S\|/\Lambda\ll 1).
            \]
            If $\hat{H}_S=\hat{H}_A+\hat{H}_B$ with $[\hat{H}_A,\hat{H}_B]=0$ on that subspace, then the $O(\Lambda^{-1})$ term includes the universal cross-term $-(2/\Lambda)\,\hat{H}_A\otimes \hat{H}_B$.
        \end{proposition}

        \begin{remark}
            The spectral restriction can be read as a low-energy condition or enforced by a bounded-spectrum truncation of the clock/system. That is the regime in which $\Lambda^{-1}$ is a controlled perturbation \cite{SinghFriedrich2025}.
        \end{remark}

    \subsection{Composite systems and emergent interaction}\label{subsec:emergent}
        For $\hat{H}_S=\hat{H}_A+\hat{H}_B$ with $|\hat{H}_S|\ll\Lambda$, expanding \eqref{eq:Heff} gives
        \begin{equation}
            \hat{H}_{\rm eff} \approx \hat{H}_A+\hat{H}_B -\frac{1}{\Lambda}\bigl(\hat{H}_A^2+\hat{H}_B^2+2\,\hat{H}_A\otimes \hat{H}_B\bigr)+O(\Lambda^{-2}),
            \label{eq:Heff_expand}
        \end{equation}
        i.e., a universal energy--energy coupling. With an external identification of distance, this matches the Newtonian form in the weak-field limit (Sec.~\ref{sec:redshift_newton}).

% ============================================================
    \section{Perspectival Structure of Relational Time and Gravity}
    \label{sec:perspectival}
    A durable lesson from quantum foundations is that physically meaningful descriptions are local to a context. In an endo-theoretic perspectival view, perspectives come with finite resolution and context-dependent Boolean algebras; a single global Boolean valuation is unavailable \cite{KarakostasZafiris2025}. Page--Wootters-type constructions fit this picture: each clock choice defines a local temporal ordering while the global state remains stationary \cite{PageWootters1983,Wootters1984}.

    When gravitational effects are modeled by universal clock--energy couplings, proper time is best treated as an observable in correlations. Gravitational time dilation then records a mismatch between clock perspectives rather than the action of a background metric \cite{SinghFriedrich2025}. The geometric trinity is compatible with this: distinct geometric codings (curvature, torsion, non-metricity) summarize the same local dynamical content \cite{BeltranJimenez2018,AldrovandiPereira2013,ManciniTinoCapozziello2025}. In that light, the EP reads as a low-energy compatibility condition among local perspectives.

% ============================================================
    \section{Gravitational Redshift and the Newtonian Limit from Relational Dynamics}\label{sec:redshift_newton}

    \subsection{Proper time as an internal observable}\label{subsec:propertime}
        Model a subsystem as a clock and compare its conditional evolution with a coordinate time derived from the global clock. Proper time is then an observable, not a background parameter \cite{SinghFriedrich2025}.

    \subsection{Weak-field redshift}\label{subsec:weakfield}
        In the low-energy regime, the deformation scale $\Lambda$ modifies the clock’s tick rate relative to coordinate time. Identifying $\Lambda^{-1}$ with a Newtonian potential scale reproduces leading-order gravitational time dilation \cite{SinghFriedrich2025,Will2014}.

    \subsection{On distance dependence}\label{subsec:distance_dependence}
        The relational model does not itself fix a $1/r$ law unless spatial degrees of freedom and reference frames are included. What emerges directly is an energy--energy coupling and a redshift structure; mapping it to $GM/r$ requires a spatial model. Quantum reference frames consistent with the operational causal scaffold are a natural way forward \cite{Pineda2026}.

    \subsection{High-energy saturation}\label{subsec:uv}
        Because \eqref{eq:Heff} saturates phase rates at high energy, certain ultraviolet behaviors are softened. I treat this as an effective-theory feature rather than a fundamental claim \cite{SinghFriedrich2025}.

% ============================================================
    \section{Metric--Affine Gravity and the Geometric Trinity}\label{sec:trinity}

    \subsection{Metric and connection as independent}\label{subsec:metric_affine}
        In metric--affine approaches the metric $g_{\mu\nu}$ and connection $\Gamma^\lambda_{\mu\nu}$ are independent. Curvature, torsion, and non-metricity characterize different properties of $\Gamma^\lambda_{\mu\nu}$ \cite{AldrovandiPereira2013,BeltranJimenez2018}.

    \subsection{GR, TEGR, STEGR: dynamical equivalence}\label{subsec:equivalence}
        GR (curvature), TEGR (torsion), and STEGR (non-metricity) are dynamically equivalent up to boundary terms, yielding the same classical field equations in their standard forms \cite{BeltranJimenez2018,AldrovandiPereira2013}. The upshot is geometric underdetermination.

        This equivalence ensures that classical gravitational dynamics remain unchanged across these formulations. However, their conceptual underpinnings differ: in GR, inertial and gravitational effects are unified via curvature; in TEGR, torsion encodes gravitational effects; in STEGR, non-metricity does. This distinction becomes important when assessing which assumptions are structural, and which are emergent consequences of the dynamics.

    \subsection{Conceptual inequivalence and EP}\label{subsec:conceptual}
        Despite dynamical equivalence, the part played by the EP differs conceptually, especially regarding the unification of inertial and gravitational effects. See \cite{ManciniTinoCapozziello2025} for a detailed mapping.

% ============================================================
    \section{The Equivalence Principle Reconsidered}\label{sec:ep}

    \subsection{EP variants and evidence}\label{subsec:variants}
        Distinguish weak equivalence (universality of free fall), Einstein equivalence (local Lorentz and position invariance), and strong equivalence (including gravitational self-energy). Precision tests impose tight bounds in many regimes \cite{Will2014}.

    \subsection{EP as axiom in curvature-based gravity}\label{subsec:axiom}
        In standard GR pedagogy, the EP motivates locally inertial frames that remove gravitational effects by coordinates, suggesting gravity is geometry rather than force. Powerful—but not unique—especially in metric--affine or gauge formulations \cite{ManciniTinoCapozziello2025}.

    \subsection{EP as emergent in teleparallel and symmetric teleparallel}\label{subsec:teleparallel}
        TEGR and STEGR separate inertial and gravitational contributions at the level of the connection, allowing EP-like phenomenology without taking the EP as fundamental. This supports an emergent reading tied to connection dynamics \cite{ManciniTinoCapozziello2025,BeltranJimenez2018,AldrovandiPereira2013}.

% ============================================================
    \section{Emergence of the EP from Relational Time and Connection Dynamics}\label{sec:emergence}

    \subsection{Synthesis}\label{subsec:synthesis}
        Combine: (i) causality fixes the Lorentzian scaffold \cite{Pineda2026,Zeeman1964}; (ii) relational constraints yield redshift and energy--energy interactions at low energy \cite{SinghFriedrich2025,PageWootters1983}; and (iii) geometric codings are empirically equivalent \cite{ManciniTinoCapozziello2025,BeltranJimenez2018}. In this regime, the EP is an emergent organizing principle.

    \subsection{Low-energy universality}\label{subsec:universality}
        The effective coupling depends on energy operators rather than composition-specific constants. Universality of free fall and equality of inertial and gravitational mass then appear as consequences of the relational structure, not separate postulates \cite{SinghFriedrich2025,ManciniTinoCapozziello2025}.

    \subsection{When deviations should appear}\label{subsec:deviations}
        Deviations should track clock resolution, energy superpositions in the source, or departures from the perturbative regime. These are coherence-dependent corrections to conditional evolution, not classical fifth forces \cite{SinghFriedrich2025}. This points naturally to quantum-clock experiments \cite{SmithAhmadi2020,CastroRuiz2017}.

% ============================================================
    \section{Phenomenology and Outlook}\label{sec:phenom}

    \subsection{Consistency with weak-field tests}\label{subsec:consistency}
        Standard redshift and Newtonian limits are recovered once $\Lambda^{-1}$ is matched to the Newtonian potential scale \cite{SinghFriedrich2025,Will2014}. The geometric trinity ensures equivalent classical phenomenology across codings \cite{ManciniTinoCapozziello2025,BeltranJimenez2018}.

    \subsection{Quantum-clock probes}\label{subsec:qclocks}
        Quantum-clock interferometry and related setups directly target coherence-dependent redshift corrections and offer clean tests of the relational mechanism \cite{SmithAhmadi2020,CastroRuiz2017}.

    \subsection{Open problem: distance}\label{subsec:openproblem}
        A principled derivation of the $1/r$ form remains. Operational distance compatible with causal structure exists \cite{Pineda2026}; importing it into the relational constraint is a concrete next step.

% ============================================================
    \section{Discussion}\label{sec:discussion}
    The account offered here keeps phenomenology intact while reordering foundations: causality sets the inertial scaffold; relational time and mild constraint deformations supply effective gravitational structure; and different geometric codings summarize the same empirical content. On this reading, the EP is a low-energy symmetry. The framework highlights coherence-dependent redshift as an actionable experimental target.

% ============================================================
    \section{Conclusion}\label{sec:conclusion}
    Operational causality fixes relativistic kinematics \cite{Pineda2026,Zeeman1964}. Relational-time dynamics with weak deformations generate redshift and energy--energy interactions \cite{SinghFriedrich2025,PageWootters1983}. The geometric trinity reveals multiple, inequivalent geometric summaries of identical phenomenology \cite{ManciniTinoCapozziello2025,BeltranJimenez2018}. Together these motivate treating the EP as emergent and identify quantum-clock tests as the right place to look for coherence-dependent corrections. This synthesis supports a shift in perspective: rather than treating the EP as a foundational postulate, we may view it as a robust feature of the relational, causal, and geometric structures that underlie gravitational phenomenology.

% ============================================================
    \backmatter

    \bmhead{Acknowledgements}
    Not applicable.

    \bmhead{Declarations}

    \subsection*{Funding}
        Not applicable.

    \subsection*{Competing interests}
        The author declares no competing interests.

    \subsection*{Ethics approval}
        Not applicable.

    \subsection*{Consent to participate}
        Not applicable.

    \subsection*{Consent for publication}
        Not applicable.

    \subsection*{Data availability}
        Not applicable.

    \subsection*{Code availability}
        Not applicable.

    \subsection*{Author contributions}
        O.I. conceived the study, developed the analysis, performed the derivations, and wrote the manuscript.

% ============================================================
        \appendix
    \section{Operator Domains and Regularization of Effective Evolution}\label{app:domains}
    The expression $\hat{H}_{\rm eff}=\hat{H}_S(\hat{I}+\hat{H}_S/\Lambda)^{-1}$ requires care when $\hat{H}_S$ is unbounded or when the spectrum overlaps $-\Lambda$. In practice one can: (i) restrict to low-energy sectors $|\hat{H}_S|\ll \Lambda$ and use the expansion \eqref{eq:Heff_expand}; (ii) model clocks/systems with bounded spectra; or (iii) introduce spectral cutoffs or smearing representing finite clock resolution. These standard choices make the relational evolution well-defined \cite{SinghFriedrich2025,PageWootters1983}.

% ============================================================
% References
% ============================================================
    \begin{thebibliography}{99}

        \bibitem{Pineda2026}
        A.~Pineda,
        \textit{Relativity: A Matter of Causality},
        Foundations of Physics \textbf{56} (2026) 2.
        doi:10.1007/s10701-025-00897-4

        \bibitem{Zeeman1964}
        E.~C.~Zeeman,
        \textit{Causality implies the Lorentz group},
        Journal of Mathematical Physics \textbf{5} (1964) 490--493.
        doi:10.1063/1.1704140

        \bibitem{Kreinovich1994}
        V.~Kreinovich,
        \textit{Approximately measured causality implies the Lorentz group: Alexandrov--Zeeman result made more realistic},
        International Journal of Theoretical Physics \textbf{33} (1994) 1733--1747.
        doi:10.1007/BF00672697

        \bibitem{PageWootters1983}
        D.~N.~Page and W.~K.~Wootters,
        \textit{Evolution without evolution: Dynamics described by stationary observables},
        Physical Review D \textbf{27} (1983) 2885.
        doi:10.1103/PhysRevD.27.2885

        \bibitem{Wootters1984}
        W.~K.~Wootters,
        \textit{``Time'' replaced by quantum correlations},
        International Journal of Theoretical Physics \textbf{23} (1984) 701--711.
        doi:10.1007/BF02214098

        \bibitem{DeWitt1967}
        B.~S.~DeWitt,
        \textit{Quantum Theory of Gravity. II. The Manifestly Covariant Theory},
        Physical Review \textbf{162} (1967) 1195--1239.
        doi:10.1103/PhysRev.162.1195

        \bibitem{ManciniTinoCapozziello2025}
        C.~Mancini, G.~M.~Tino, and S.~Capozziello,
        \textit{Equivalent Gravities and Equivalence Principle: Foundations and Experimental Implications},
        Foundations of Physics \textbf{56} (2026) 1 (online 2025).
        doi:10.1007/s10701-025-00882-x

        \bibitem{AldrovandiPereira2013}
        R.~Aldrovandi and J.~G.~Pereira,
        \textit{Teleparallel Gravity: An Introduction},
        Springer (2013).
        doi:10.1007/978-94-007-5143-9

        \bibitem{BeltranJimenez2018}
        J.~Beltr\'an~Jim\'enez, L.~Heisenberg, and T.~S.~Koivisto,
        \textit{Coincident General Relativity},
        Physical Review D \textbf{98} (2018) 044048.
        doi:10.1103/PhysRevD.98.044048

        \bibitem{Will2014}
        C.~M.~Will,
        \textit{The Confrontation between General Relativity and Experiment},
        Living Reviews in Relativity \textbf{17} (2014) 4.
        doi:10.12942/lrr-2014-4

        \bibitem{SmithAhmadi2020}
        A.~R.~H.~Smith and M.~Ahmadi,
        \textit{Quantum clocks observe classical and quantum time dilation},
        Nature Communications \textbf{11} (2020) 5360.
        doi:10.1038/s41467-020-18264-4

        \bibitem{CastroRuiz2017}
        E.~Castro-Ruiz, F.~Giacomini, and \v{C}.~Brukner,
        \textit{Entanglement of quantum clocks through gravity},
        PNAS \textbf{114} (2017) E2303--E2309.
        doi:10.1073/pnas.1616427114

        \bibitem{SinghFriedrich2025}
        A.~Singh and O.~Friedrich,
        \textit{Emergence of Gravitational Potential and Time Dilation from Non-interacting Systems Coupled to a Global Quantum Clock},
        Foundations of Physics \textbf{55} (2025) 82.
        doi:10.1007/s10701-025-00893-8

        \bibitem{KarakostasZafiris2025}
        V.~Karakostas and E.~Zafiris,
        \textit{Contemporary Perspectivism as a Framework of Scientific Inquiry in Quantum Mechanics and Beyond},
        Foundations of Physics \textbf{55} (2025) 76.
        doi:10.1007/s10701-025-00888-5

    \end{thebibliography}

\end{document}