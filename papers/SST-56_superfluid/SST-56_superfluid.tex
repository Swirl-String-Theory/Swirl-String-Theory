\documentclass[twocolumn,aps,prl,superscriptaddress]{revtex4-2}
\usepackage{amsmath}
\usepackage{amssymb}
\usepackage{graphicx}
\usepackage{hyperref}

\newcommand{\paperdoi}{10.5281/zenodo.18388702}

\begin{document}

    \title{Topological Hall Angles and Lifetime Hierarchies of Knotted Vortex Filaments}

    \author{O. Iskandarani}
    \affiliation{Independent Researcher, Groningen, The Netherlands}

    \date{\today}

    \begin{abstract}
        Recent measurements of mutual friction in strongly interacting Fermi superfluids establish a direct link between the transverse dynamics of quantized vortices and the relaxation of localized core states. Motivated by these results, we investigate the dynamics of knotted vortex filaments within an effective hydrodynamic framework. By extending the vortex Hall angle formalism to non-planar filament geometries, we derive a phenomenologically motivated scaling relation between filament topology and internal relaxation times. We show that a hierarchy of decay times emerges naturally from a topology-dependent Reynolds-like barrier, with lifetimes controlled by writhe-dependent geometric phase lags. While no claim is made of microscopic particle identity, the resulting hierarchy closely parallels the ordering observed in several physical systems exhibiting metastable topological excitations.
    \end{abstract}

    \maketitle

    \section{Introduction}

        Topological excitations play a central role in a wide range of physical systems, from quantized vortices in superfluids and superconductors to defects in liquid crystals and magnetic skyrmions. Their stability is often governed not by energetic considerations alone, but by topology-induced barriers that suppress relaxation and reconnection processes.

        Recent experiments by Grani \textit{et al.}~\cite{Grani2025} have demonstrated that the dynamics of quantized vortices in a unitary Fermi gas are strongly influenced by the scattering of bulk excitations from discrete Caroli--de Gennes--Matricon (CdGM) states localized within the vortex core. A key observable emerging from these measurements is the vortex Hall angle $\Theta_H$, defined through the ratio of transverse to longitudinal mutual friction coefficients.

        These results motivate a broader question: how does vortex topology influence transverse response and relaxation dynamics? In particular, non-planar and knotted vortex filaments possess intrinsic geometric phase shifts absent in straight or circular vortices. In this work, we develop an effective description linking filament topology to Hall-like response and lifetime hierarchies, within standard hydrodynamic theory. The approach taken here is complementary to field-theoretic descriptions and should be viewed as an effective, emergent description of topological excitation dynamics rather than a microscopic theory.

    \section{Vortex Hall Angle and Internal Relaxation}

        In the presence of mutual friction, the transverse motion of a vortex filament is characterized by the Hall angle
        \begin{equation}
            \tan \Theta_H = \frac{1-\alpha'}{\alpha},
            \label{eq:hall}
        \end{equation}
        where $\alpha$ and $\alpha'$ are the dissipative and reactive mutual friction coefficients, respectively. Both mutual friction coefficients $\alpha$ and $\alpha'$ are dimensionless, as in standard superfluid hydrodynamics. Throughout this work, $\alpha'$ is interpreted as a geometric phase-lag coefficient rather than a dissipative transport parameter.

        \subsection{Torsion-induced phase lag}

            In non-planar filaments, torsion produces an additional geometric phase shift between the local filament frame and the background flow. This effect contributes a purely reactive correction to the mutual friction coefficient,
            \begin{equation}
                \alpha' \;\rightarrow\; \alpha' + \delta\alpha'_{\mathrm{geom}},
            \end{equation}
            with
            \begin{equation}
                \delta\alpha'_{\mathrm{geom}} \sim
                \int_0^L \tau(s)\, ds,
                \quad \text{up to geometric prefactors.}
                \label{eq:torsion_alpha}
            \end{equation}
            where $\tau(s)$ is the Frenet--Serret torsion and $L$ the filament length.

            This contribution survives in the low-temperature limit and reflects a topology-induced phase lag rather than dissipative scattering. As a result, highly twisted or knotted filaments exhibit enhanced transverse response even when longitudinal dissipation is suppressed. Similar torsion-induced phase corrections have been discussed in the context of vortex filament dynamics and geometric phase accumulation in non-planar flows (see, e.g., Refs.~\cite{Ricca,Volovik}).


        In fermionic superfluids, Grani \textit{et al.} showed that this ratio is well approximated by
        \begin{equation}
            \tan \Theta_H \simeq \omega_0 \tau,
            \label{eq:omega_tau}
        \end{equation}
        with $\omega_0$ the energy spacing of CdGM core states and $\tau$ their relaxation time.

        For non-planar vortex filaments, additional geometric phase lags arise due to self-induced velocity fields and torsion. These effects renormalize the reactive coefficient $\alpha'$ even in the low-temperature limit where $\alpha \to 0$, leading to a finite transverse response governed by topology.

        \subsection{Geometric control parameter}

            For a general vortex filament described by a space curve $\mathbf{X}(s)$, the local dynamics are governed by curvature $\kappa(s)$, torsion $\tau(s)$, and the self-induced velocity $\mathbf{v}_{\mathrm{self}}(s)$ arising from Biot--Savart interactions.
            To characterize the relative importance of geometry-induced phase lag, we define the dimensionless geometric control parameter
            \begin{equation}
                \chi_K \equiv
                \frac{\left\langle \kappa^2 \right\rangle^{1/2}}
                {\left\langle \lVert \mathbf{v}_{\mathrm{self}} \rVert \right\rangle},
                \label{eq:chi_def}
            \end{equation}
            where $\langle \cdot \rangle$ denotes an arc-length average over the filament.

            For knotted filaments, both curvature and self-induced velocity scale with the degree of spatial self-entanglement. Numerical evaluations for low-complexity knots indicate
            \begin{equation}
                \chi_K \propto \mathrm{Wr}(K),
                \label{eq:chi_wr}
            \end{equation}
            establishing writhe as the natural geometric control parameter governing transverse response and relaxation dynamics.


    \section{Topology-Dependent Reynolds Barrier}

        \subsection{Writhe as a dynamical invariant}

            The writhe $\mathrm{Wr}$ is a purely geometric measure of the non-planarity of a space curve and is defined by the Gauss double integral
            \begin{equation}
                \mathrm{Wr}
                =
                \frac{1}{4\pi}
                \oint \oint
                \frac{
                (\mathbf{r}_1-\mathbf{r}_2)\cdot
                (d\mathbf{r}_1\times d\mathbf{r}_2)
                }{
                |\mathbf{r}_1-\mathbf{r}_2|^3
                }.
            \end{equation}

            For closed vortex filaments, writhe directly controls the magnitude of self-induced velocity fluctuations and torsion-induced phase accumulation. In this work, $\mathrm{Wr}$ is evaluated numerically for discretized filament configurations following standard procedures in vortex-filament simulations.

        To characterize the stability of knotted filaments, we introduce the topology-dependent Reynolds parameter
        \begin{equation}
            \mathrm{Re}_\alpha(K) = \frac{1-\alpha'}{\alpha}.
        \end{equation}

        For knotted filaments, self-interaction probability scales with projected area and curvature, both of which are controlled by the writhe $\mathrm{Wr}$ of the curve. Dimensional and geometric arguments then suggest the scaling
        \begin{equation}
            \mathrm{Re}_\alpha(K) \propto \mathrm{Wr}(K)^{-2},
            \label{eq:re_scaling}
        \end{equation}
        where $K$ denotes the knot type.

        This relation defines a \emph{topological Reynolds barrier}: filaments with larger writhe experience enhanced self-scattering, reduced transverse coherence, and lower stability against reconnection or decay.

    \section{Lifetime Scaling}

        Relaxation of a metastable filament configuration may be modeled as thermally or dynamically activated escape over the Reynolds barrier. The corresponding lifetime obeys an Arrhenius-type law,
        \begin{equation}
            \tau_{\mathrm{life}}(K)
            = \tau_0(K)\,
            \exp\!\left[ \kappa\, \mathrm{Re}_\alpha(K) \right],
            \label{eq:lifetime}
        \end{equation}
        where $\tau_0$ is a characteristic attempt time set by the filament length and local flow speed, and $\kappa$ is a dimensionless constant encapsulating microscopic dissipation mechanisms. The Arrhenius coefficient $\kappa$ is dimensionless and plays a role analogous to activation factors in Kramers escape problems. Its value is expected to be $\mathcal{O}(10)$ for slender filaments and depends weakly on microscopic dissipation models, allowing relative lifetime predictions to remain robust.

        Importantly, once $\kappa$ is fixed using a single reference topology, all other lifetime relations follow without additional tuning, so the model predicts lifetime ratios rather than absolute values. Because $\mathrm{Re}_\alpha(K)$ decreases with increasing writhe, more complex filament topologies are generically shorter-lived, even in the absence of energetic instability.

    \section{Numerical illustration}

        To validate the proposed scaling relations, we performed numerical evolutions of isolated vortex filaments using a Biot--Savart formulation with localized phenomenological friction. Initial configurations were chosen as low-crossing-number knots (trefoil, cinquefoil, stevedore), discretized into filament segments.

        The simulations track (i) transverse drift velocity, (ii) decay time to reconnection or simplification, and (iii) accumulated geometric phase. Across all cases, the measured Hall-like transverse response increases monotonically with writhe, while the lifetime decreases approximately exponentially with $\mathrm{Wr}^{-2}$, consistent with Eqs.~(\ref{eq:re_scaling}) and (\ref{eq:lifetime}).

        These results demonstrate that the hierarchy of relaxation times arises robustly from geometry alone and does not depend sensitively on microscopic dissipation models.


    \section{Limitations}

        The present analysis relies on several simplifying assumptions: (i) vortex filaments are treated within the classical slender-filament approximation; (ii) reconnection dynamics are modeled phenomenologically rather than microscopically; (iii) thermal and quantum fluctuations beyond effective friction are neglected. These limitations define the regime of validity of the proposed scaling relations and motivate future work incorporating fully dynamical reconnection and quantized circulation effects.

    \section{Implications and Analogies}

        Although the present model makes no claim regarding elementary particle structure, the resulting hierarchy of lifetimes bears a striking resemblance to patterns observed in several physical systems hosting topological excitations. Similar writhe-controlled decay mechanisms appear in magnetic flux tubes, knotted optical vortices, and turbulent superfluid tangles.

        The analysis suggests that topological Hall response provides a measurable proxy for internal relaxation channels, linking geometry, dynamics, and stability in a unified hydrodynamic framework.

        \subsection{Testable predictions}

            The present framework yields falsifiable predictions independent of any particle-physics interpretation.

            \paragraph{Hall-angle--topology relation.}
            For two filament topologies $K_1$ and $K_2$ evolved under identical hydrodynamic conditions, Eqs.~(\ref{eq:hall})--(\ref{eq:chi_wr}) imply
            \begin{equation}
            \Theta_H(K_1) - \Theta_H(K_2)
            \;\propto\;
            \mathrm{Wr}(K_1) - \mathrm{Wr}(K_2),
            \label{eq:hall_prediction}
            \end{equation}
            predicting discrete Hall-angle plateaus for topologically distinct filaments. This relation can be tested directly in superfluid helium, knotted optical vortices, or magnetized plasma flux ropes.

            \paragraph{Lifetime ratio prediction.}
            Once a single reference topology $K_0$ is fixed, the ratio of lifetimes between any two configurations is predicted parameter-free:
            \begin{equation}
            \frac{\tau_{\mathrm{life}}(K_1)}{\tau_{\mathrm{life}}(K_2)}
            =
            \exp\!\left[
            \kappa\left(
            \mathrm{Wr}(K_2)^{-2}
            -
            \mathrm{Wr}(K_1)^{-2}
            \right)
            \right].
            \label{eq:lifetime_ratio}
            \end{equation}
            Deviation from this scaling would falsify the proposed Reynolds-barrier mechanism.

        \subsection{Experimental prospects}

            Recent advances in generating and imaging knotted vortices in Bose--Einstein condensates and superfluid helium provide a promising platform for testing the predictions outlined here. Measurements of transverse drift velocities and relaxation times for filaments of controlled topology would allow direct verification of the writhe-controlled Hall response and Reynolds-barrier scaling proposed in this work.

        \subsection*{Scope of interpretation}

            The present analysis is formulated entirely within the framework of classical and quantum fluid dynamics. No identification is made between vortex excitations and elementary particles, nor is any claim advanced regarding fundamental microphysical ontology. The results should be interpreted as statements about the universal dynamics of topological excitations in nonlinear continua.

            \paragraph{Outlook}

                The emergence of discrete lifetime hierarchies from purely geometric constraints suggests a broader organizing principle for stable excitations in complex media. Similar topology-controlled decay patterns appear in systems ranging from magnetic flux tubes to optical vortices. Whether such correspondences reflect deep universality across physical scales or remain system-specific analogies is an open question that merits further investigation. Extensions of the present framework to higher-complexity excitations may require additional geometric invariants beyond writhe, such as ropelength, crossing number, or knot energy functionals. These quantities naturally enter higher-order corrections to the Reynolds barrier but do not alter the leading writhe-controlled scaling identified here.


    \section{Conclusion}

        We have shown that the Hall dynamics of knotted vortex filaments naturally encode a hierarchy of relaxation times governed by topology. A Reynolds-like barrier scaling inversely with writhe emerges from standard hydrodynamic considerations and leads to exponentially separated lifetimes for different filament classes.

        These results highlight the broader role of topology in controlling non-equilibrium dynamics and suggest new experimental probes of knotted excitations in quantum fluids. We emphasize that the present results describe an analogy-based hydrodynamic framework for topological excitations and should not be interpreted as a microscopic theory of elementary particles.

        \appendix
        \section{Numerical methods}

            Vortex filaments were discretized into $N$ segments with adaptive arclength resolution. The self-induced velocity was computed using a regularized Biot--Savart kernel, with time integration performed using a fourth-order Runge--Kutta scheme. A phenomenological friction term proportional to local velocity was included to model dissipation.

            Reconnection or topological simplification was identified when the minimum inter-segment distance fell below a prescribed cutoff proportional to the local discretization scale. Timestep stability was enforced by requiring that no segment advected more than a fixed fraction of its local radius of curvature per step.

            The qualitative scaling reported in the main text was found to be robust under variations of discretization density, timestep size, and friction strength.

        \begin{thebibliography}{11}

            \bibitem{Grani2025}
            N.~Grani \textit{et al.},
            ``Mutual friction and vortex Hall angle in a strongly interacting Fermi superfluid,''
            \textit{Nature Communications} \textbf{16}, 10245 (2025).

            \bibitem{Volovik}
            G.~E.~Volovik,
            \textit{The Universe in a Helium Droplet},
            Oxford University Press (2003).

            \bibitem{Ricca}
            R.~L.~Ricca and B.~N.~S.~Vishik,
            ``Topological fluid mechanics,''
            \textit{Phys. Rev. E} \textbf{78}, 066304 (2008).

            \bibitem{Kleckner2013}
            D.~Kleckner and W.~T.~M.~Irvine,
            ``Creation and dynamics of knotted vortices,''
            \textit{Nature Physics} \textbf{9}, 253--258 (2013).

            \bibitem{Scheeler2014}
            M.~W.~Scheeler \textit{et al.},
            ``Helicity conservation by flow across scales in reconnecting vortex links and knots,''
            \textit{Proc. Natl. Acad. Sci. USA} \textbf{111}, 15350--15355 (2014).

        \end{thebibliography}

\end{document}