\documentclass[a4paper,10pt]{letter}

\usepackage[T1]{fontenc}
\usepackage[utf8]{inputenc}
\usepackage{lmodern}
\usepackage[hidelinks]{hyperref}
\usepackage{microtype}
\usepackage[margin=1in]{geometry}
\usepackage{amstext}
\usepackage{amssymb}

% Sender info
\signature{Omar Iskandarani\\
Independent Researcher, Groningen,\\ The Netherlands\\
ORCID: 0009-0006-1686-3961\\
Email: \href{mailto:info@omariskandarani.com}{info@omariskandarani.com}}
\address{Omar Iskandarani\\
Vinkenstraat 86A\\
9713 TK Groningen\\
The Netherlands}

\date{\today}

\begin{document}

    \begin{letter}{Editors\\\textit{Foundations of Physics}}
        \opening{Dear Editor,}

        I am pleased to submit the manuscript \textit{Kelvin Mode Suppression in Atomic Orbitals: A Vortex-Filament Gap} for consideration as a Research Article in \textit{Foundations of Physics}.

        \textbf{Summary.} In hydrodynamic models where electrons are represented by closed vortex filaments in an incompressible medium, atomic orbitals arise as equilibrium configurations. A core consistency issue is whether internal Kelvin-wave excitations could thermodynamically couple to orbital degrees of freedom and thus destabilize the hydrogenic spectrum. The paper shows that, without additional structure, these corrections would exceed spectroscopic bounds by many orders of magnitude; it then identifies a natural resolution: a \emph{topological excitation gap} \(\Delta_K\) in the Kelvin spectrum of order \(\mathcal{O}(10^2\text{--}10^3\,\mathrm{eV})\) that exponentially suppresses Kelvin contributions under ordinary conditions. In this gapped regime, the stationary Schr\"odinger equation appears as a low-energy equation of state, and Kelvin dynamics are inert except under extreme acceleration or high-energy probes.

        \textbf{What is new (concise).}
        \begin{itemize}\setlength\itemsep{0.25em}
        \item \emph{Quantified inconsistency without a gap:} Ungapped elastic estimates give \(a_n^{\mathrm{naive}}\!\sim\!10^{-39}\,\mathrm{J\,K^{-2}}\) for level shifts \(E_n^{\mathrm{eff}}=E_n^{(0)}-a_n T^2+\dots\), whereas hydrogen spectroscopy requires \(a_n\!\lesssim\!10^{-62}\,\mathrm{J\,K^{-2}}\) for low-lying states (\(>20\) orders tighter).
        \item \emph{Natural fix via topology:} A Kelvin gap \(\Delta_K\) replaces polynomial thermal scaling by exponential suppression, \(U\propto e^{-\Delta_K/(k_B T)}\), rendering Kelvin modes inert at atomic temperatures.
        \item \emph{Concrete scale setting:} An appendix estimate places the first Kelvin mode near \(E_1\!\approx\!30\,\mathrm{eV}\) for a closed ground-state loop; the consistency condition \(\Delta_K/(k_B T_{\mathrm{eff}})\!\gtrsim\!60\) with \(T_{\mathrm{eff}}\!\lesssim\!10^5\,\mathrm{K}\) yields \(\Delta_K\!\gtrsim\!5\times10^2\,\mathrm{eV}\), comfortably below \(m_e c^2\) yet well above atomic scales.
        \item \emph{Recovery of standard quantum structure:} With Kelvin modes frozen and a vortex-induced \(V(r)\propto -1/r\), the stationary Schr\"odinger equation follows as the Euler--Lagrange condition of a minimal free-energy functional, reproducing the hydrogenic \(1/n^2\) scaling.
        \end{itemize}

        \textbf{Distinctive predictions / testable consequences.}
        \begin{itemize}\setlength\itemsep{0.25em}
        \item \emph{Activation threshold:} Kelvin excitations remain silent in ordinary atoms but can activate under ultra-high accelerations (Unruh-like settings) or keV--MeV scattering, providing a sharp separation of scales.
        \item \emph{Temperature-law discriminator:} Any residual Kelvin contribution must exhibit exponential, not polynomial, temperature dependence---a falsifiable signature in precision spectroscopy and thermal shift bounds.
        \item \emph{Astro/cosmo triggers:} Vortex reconnections in astrophysical fluids offer a natural domain for Kelvin activation, suggesting cross-constraints from plasma/astrophysical observations.
        \end{itemize}

        \textbf{Why \textit{Foundations of Physics}.} The result isolates the structural ingredient (a Kelvin gap) required for hydrodynamic/vortex-filament accounts to be compatible with atomic phenomenology, linking topology, thermodynamics, and quantum structure. This directly serves the journal’s focus on conceptual clarity, consistency across theories, and testable foundational claims.

        \textbf{Availability and transparency.} A timestamped preprint is available at Zenodo (DOI: \href{https://doi.org/10.5281/zenodo.18018084}{10.5281/zenodo.18018084}). The submission is original, not under consideration elsewhere, and the authorship/affiliation information is accurate. There is no external funding and no conflicts of interest. Data and code are not applicable; the work consists of analytical derivations using standard constants, with figure assets included.

        Thank you for considering this manuscript. I would be grateful if you would consider it for publication as a Research Article in \textit{Foundations of Physics}.

        \closing{Sincerely,}

    \end{letter}
\end{document}