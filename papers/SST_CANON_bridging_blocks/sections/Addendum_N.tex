%! Author = Omar Iskandarani
%! Title = ......
%! Date = .....
%! Affiliation = Independent Researcher, Groningen, The Netherlands
%! License = © 2025 Omar Iskandarani. All rights reserved. This manuscript is made available for academic reading and citation only. No republication, redistribution, or derivative works are permitted without explicit written permission from the author. Contact: info@omariskandarani.com
%! ORCID = 0009-0006-1686-3961
%! DOI = 10.5281/zenodo.xxxxxxx

% =====================================================================
% SST Canon Addendum — Bridging Blocks
% Version tag: v0.3.1+2025-08-27 (conforms to Canon v0.3.1)
% Persona: Bridging (does not modify Core Postulates)
% =====================================================================

% === Metadata ===
\newcommand{\papertitle}{....}
\newcommand{\paperdoi}{10.5281/zenodo.xxxxxxxx}


\ifdefined\standalonechapter\else
% Standalone mode
\documentclass[11pt]{article}
\input{../../template/SSTstyle.sty}
\input{../../template/SST_appendix_setup.sty}
\begin{document}

  % === Title page ===
  \titlepageOpen

  \begin{abstract}


  \end{abstract}

  \titlepageClose
  \fi

  \ifdefined\standalonechapter
  \section{\papertitle}
  \else
  \fi
% ============= Begin of content ============


% =====================================================================
% SST Canon Addendum N — Fixed-Point Family Replication & \alpha
% Version tag: v0.3.2+2025-08-29-N (bridging-only; does not modify Core)
% =====================================================================

      \usepackage{amsmath,amssymb,bm}
      \usepackage{hyperref}
      \usepackage{physics}
      \usepackage{siunitx}
      \usepackage{cleveref}

% --------- Embed a local .bib file so this compiles standalone ----------
      \begin{filecontents*}{addendumN.bib}
      @misc{Lemmon2013FamiliesAlpha,
      author       = {John Lemmon},
      title        = {The origin of fermion families and the value of the fine structure constant},
      year         = {2013},
      eprint       = {1307.2201},
      archivePrefix= {arXiv},
      primaryClass = {hep-ph},
      version      = {v3},
      url          = {https://arxiv.org/abs/1307.2201}
      }

      @article{Feynman1949SelfEnergy,
      author  = {Richard P. Feynman},
      title   = {Space-Time Approach to Quantum Electrodynamics},
      journal = {Physical Review},
      year    = {1949},
      volume  = {76},
      pages   = {769--789},
      doi     = {10.1103/PhysRev.76.769}
      }

      @article{PauliVillars1949,
      author  = {Wolfgang Pauli and Felix Villars},
      title   = {On the Invariant Regularization in Relativistic Quantum Theory},
      journal = {Reviews of Modern Physics},
      year    = {1949},
      volume  = {21},
      pages   = {434--444},
      doi     = {10.1103/RevModPhys.21.434}
      }

      @misc{SSTCanon2025,
      author       = {Omar Iskandarani},
      title        = {Swirl String Theory (SST) Canon v0.3.2},
      year         = {2025},
      doi          = {10.5281/zenodo.16934536},
      howpublished = {Zenodo},
      url          = {https://doi.org/10.5281/zenodo.16934536}
      }

      @misc{SSTLagrangian2025,
      author       = {Omar Iskandarani},
      title        = {Swirl-String Theory as an Emergent Relativistic Effective Field Theory with Preferred Foliation},
      year         = {2025},
      doi          = {10.5281/zenodo.16956665},
      howpublished = {Zenodo},
      url          = {https://doi.org/10.5281/zenodo.16956665}
      }

      @article{KniehlSirlin2008PoleMass,
      author  = {Bernd A. Kniehl and Alberto Sirlin},
      title   = {Pole Mass, Width, and Propagators of Unstable Fermions},
      journal = {Physical Review D},
      year    = {2008},
      volume  = {77},
      pages   = {116012},
      doi     = {10.1103/PhysRevD.77.116012}
      }
      \end{filecontents*}
% ----------------------------------------------------------------------



      \paragraph{Policy status.}
          \emph{Bridging-only addendum.} This note introduces a fixed-point mechanism for fermion family triplication and an approximate determination of the fine-structure constant, adapted from Lemmon~\cite{Lemmon2013FamiliesAlpha}. It \textbf{does not modify} SST Core postulates or boxed Canon results~\cite{SSTCanon2025}, and is intended for phenomenological interfacing with the covariant EFT in~\cite{SSTLagrangian2025}.

      \section{Executive Summary (Benchmark Targets)}
      Lemmon~\cite{Lemmon2013FamiliesAlpha} proposes: (i) treating the UV cutoff as physical, (ii) resumming the divergent fermion self-mass as a geometric formal sum, yielding a self-consistency equation with potentially three fixed points (interpreted as generations), (iii) a QED-only estimate of $\alpha$, and (iv) an electroweak extension that preserves the $\alpha$ estimate while driving the cutoff to $M_P$. We port this \emph{formal structure} into SST by replacing the hard cutoff with an SST spectral regulator derived from swirl scales (no change to SST ontology).

      \section{Canon-Aligned Formal Statements}
      \subsection{Fixed-point self-mass (charged leptons)}
          The formal resummation of the mass series gives
          \begin{equation}
          \boxed{\, m \;=\; m_0 \;+\; \frac{m\,\Delta(m)}{1-\Delta(m)} \,}\,,
          \label{eq:N1}
          \end{equation}
          with one-loop QED kernel
          \begin{equation}
          \Delta^{(1)}(m)\;=\;\frac{3\alpha_0}{4\pi}\!\left(\ln\!\frac{\Lambda^2}{m^2}+\frac12\right)\;+\;\cdots\,,
          \label{eq:N2}
          \end{equation}
          as in~\cite{Lemmon2013FamiliesAlpha} (see also classic renormalization structures~\cite{Feynman1949SelfEnergy,PauliVillars1949}). For suitable $(m_0,\alpha_0,\Lambda)$, \cref{eq:N1,eq:N2} admit three fixed points $m$, interpretable as $(m_e,m_\mu,m_\tau)$~\cite{Lemmon2013FamiliesAlpha}.

      \subsection{Electroweak extension (leptons)}
          Including $W$ and $Z$ loops and a physical weak angle $\theta_W$, Lemmon writes (displaying only the integral skeleton)
          \begin{align}
          \Delta^{(1)}(m)\;&=\;\frac{3\alpha_0}{4\pi}\!\left(\ln\frac{\Lambda^2}{m^2}+\frac12\right)
          -\frac{\alpha_0}{\pi}\Bigg[\frac{1}{16}\csc^2\!\theta_W\,I_1 \nonumber\\
          &\quad\;+\left(\frac{1}{32}\cot^2\!\theta_W+\frac{5}{32}\tan^2\!\theta_W-\frac{1}{16}\right)I_2
          +\left(\frac12-\frac12\tan^2\!\theta_W\right)I_3\Bigg],
          \label{eq:N3}
          \end{align}
          with
          \begin{align}
          I_1(m,m_\nu)&=2\!\int_0^1\! z\,\ln\frac{\Lambda^2 z}{M_W^2 z-m^2z(1-z)+m_\nu^2(1-z)}\,dz,\nonumber\\
          I_2(m)&=2\!\int_0^1\! z\,\ln\frac{\Lambda^2 z}{M_Z^2 z+m^2(1-z)^2}\,dz,\nonumber\\
          I_3(m)&=\int_0^1\!\ln\frac{\Lambda^2 z}{M_Z^2 z+m^2(1-z)^2}\,dz,
          \label{eq:N4}
          \end{align}
          and a chiral averaging rule for the pole mass (see also~\cite{KniehlSirlin2008PoleMass}) as quoted in~\cite{Lemmon2013FamiliesAlpha}. The $W$ loop contributes positively to charge renormalization, shifting the preferred cutoff toward $M_P$ while preserving the $\alpha$ estimate~\cite{Lemmon2013FamiliesAlpha}.

      \subsection{Charge renormalization and $\alpha$}
          At one loop,
          \begin{equation}
          \alpha \;=\;\alpha_0\!\left[1-\frac{\alpha_0}{3\pi}\sum_{\ell=e,\mu,\tau}\ln\frac{\Lambda^2}{m_\ell^2}
                                    +\frac{11\alpha_0}{12\pi}\ln\frac{\Lambda^2}{M_W^2}\right]
          \;-\;\alpha(M_Z^2)\,\Delta\alpha_H(M_Z^2)\,,
          \label{eq:N5}
          \end{equation}
          yielding $\alpha^{-1}\!\approx\!158.7$ (QED-only) and $\alpha^{-1}\!\approx\!164.5$ with EW contributions in the scheme of~\cite{Lemmon2013FamiliesAlpha}.

      \subsection{Quark sector and mixing trigger}
          For up-type quarks (down-type analogous), a one-loop kernel of the form
          \begin{equation}
          \Delta^{(1)}_U(m_U)\;=\;\frac{\alpha_0}{3\pi}\!\left(\ln\frac{\Lambda^2}{m_U^2}+\frac12\right)
          +\frac{3\alpha_0^{(s)}}{4\pi}\!\left(\frac{4}{9}\right)\!\left(\ln\frac{\Lambda^2}{m_U^2}+\frac12\right)
          -\frac{\alpha_0}{\pi}\,\Big[\cdots\Big]
          \label{eq:N6}
          \end{equation}
          is augmented by a mixing necessity once $I_1(m_t,m_b)$ becomes complex, which Lemmon implements by CKM-weighted replacements $I_1\to\sum_{ij}\abs{V_{ij}}^2 I_1(m_i,m_j)$~\cite{Lemmon2013FamiliesAlpha}.

      \section{SST Bridging Map (No-Core-Change)}
      To interface \cref{eq:N1}--\cref{eq:N6} with SST:
      \begin{itemize}
      \item Replace the hard cutoff $\Lambda$ by an \emph{SST spectral regulator} tied to swirl scales,
      \[
          \Lambda \;\longrightarrow\; \Lambda_{\text{SST}} \sim \Omega_{\max} \equiv \frac{v_{\circlearrowleft}}{r_c}\,,
      \]
      where $v_{\circlearrowleft}$ and $r_c$ are Canon constants~\cite{SSTCanon2025}. This keeps $\Delta$ dimensionless while grounding the logs in SST kinematics.
      \item Interpret $\Delta(m)$ as nonlinear backreaction from swirl polarization (vacuum screening) around a knotted filament. The geometric sum in~\cref{eq:N1} corresponds to iterated medium feedback---consistent with the EFT structure in~\cite{SSTLagrangian2025}.
      \item Mixing/CP: complex thresholds in the SST mediator sector (swirl-gauge channels) trigger superpositions, paralleling the CKM-weighted $I_1$ replacement~\cite{Lemmon2013FamiliesAlpha}.
      \end{itemize}
      \emph{Note.} This addendum does not alter SST ontology (no point-particle fundamentals, no Planckian necessity); it supplies a calculational kernel compatible with the Canon/Lagrangian.

      \section{Immediate SST Tests}
      \begin{enumerate}
      \item \textbf{Triplication with $\Lambda_{\text{SST}}$.} Calibrate $(\alpha_0,m_0)$ on $(m_e,m_\mu)$ using $\Lambda_{\text{SST}}$ and test whether a third fixed point reproduces $m_\tau$.
      \item \textbf{$\alpha$ with SST screening.} Evaluate \cref{eq:N5} with $\Lambda\!\to\!\Lambda_{\text{SST}}$ and quantify the analogue of $\Delta\alpha_H$ via collective swirl modes.
      \item \textbf{Mixing trigger.} Identify the SST channel whose opening renders the analogue of $I_1$ complex and verify that CKM-like phase structure follows.
      \end{enumerate}

      \section*{Classification (Canon Governance)}
      \textbf{Type:} Addendum (Bridging). \quad
      \textbf{Scope:} Family replication, hierarchies, $\alpha$ estimate via fixed points. \quad
      \textbf{Core:} Unchanged (no modification to boxed Canon equations~\cite{SSTCanon2025}).

% -------------------- Bibliography --------------------
      \bibliographystyle{unsrt}
      \bibliography{addendumN}




% ============== End of content =============

% === Bibliography (only for standalone) ===
  \ifdefined\standalonechapter\else
  \bibliographystyle{unsrt}
  \bibliography{../bridging_blocks_references}
\end{document}
\fi