%! Author = Omar Iskandarani
%! Title = ......
%! Date = .....
%! Affiliation = Independent Researcher, Groningen, The Netherlands
%! License = © 2025 Omar Iskandarani. All rights reserved. This manuscript is made available for academic reading and citation only. No republication, redistribution, or derivative works are permitted without explicit written permission from the author. Contact: info@omariskandarani.com
%! ORCID = 0009-0006-1686-3961
%! DOI = 10.5281/zenodo.xxxxxxx

% =====================================================================
% SST Canon Addendum — Bridging Blocks
% Version tag: v0.3.1+2025-08-27 (conforms to Canon v0.3.1)
% Persona: Bridging (does not modify Core Postulates)
% =====================================================================

% === Metadata ===
\newcommand{\papertitle}{....}
\newcommand{\paperdoi}{10.5281/zenodo.xxxxxxxx}


\ifdefined\standalonechapter\else
% Standalone mode
\documentclass[11pt]{article}
\input{../../template/SSTstyle.sty}
\input{../../template/SST_appendix_setup.sty}
\begin{document}

  % === Title page ===
  \titlepageOpen

  \begin{abstract}


  \end{abstract}

  \titlepageClose
  \fi

  \ifdefined\standalonechapter
  \section{\papertitle}
  \else
  \fi
% ============= Begin of content ============


% =====================================================================
% SST Canon Addendum — SU(6) Grand Gauge–Higgs Unification (GHU) Kernel
% Version tag: v0.3.1+2025-08-27-M (bridging-only; does not modify Core)
% =====================================================================

  \section*{Addendum M: SU(6) Grand Gauge–Higgs Unification (GHU) kernel (bridging)}

      \subsection*{M1. Scope}
          Record the mass–generation kernel and EWSB structure from 5D SU(6) GHU on $S^1/\mathbb Z_2$, and provide an SST/VAM dictionary so it can be used as an \emph{auxiliary seed} for fermion hierarchies and an alternative EWSB parametrization without altering the Core.

      \subsection*{M2. Model primitives (captured)}
          \begin{itemize}
          \item \textbf{Orbifold breaking and Higgs as gauge:} SU(6) on $S^1/\mathbb Z_2$ with boundary parities breaking to $SU(5)\!\times\!U(1)_X$ (at $y{=}0$) and $SU(2)\!\times\!SU(4)$ (at $y{=}\pi R$). The 4D Higgs doublet is a component of $A_y$. At the unification (near compactification) scale, $g_3{=}g_2{=}\sqrt{5/3}\,g_Y \Rightarrow \sin^2\theta_W = 3/8$.
          \item \textbf{Boundary SM matter \& bulk messengers:} SM fermions live on the $y{=}0$ boundary as SU(5) multiplets and couple via boundary mass terms to massive bulk fermions (four SU(6) representations plus mirrors). Integrating out the bulk tower yields nonlocal masses that are exponentially sensitive to the bulk mass parameters.
          \item \textbf{Mass kernel (large-$\lambda$ limit):} For species $a\in\{u,d,e,\nu\}$, the physical masses scale as
          \[
              m^{\text{phys}}_a \simeq m_W\, e^{-\lambda_a},
              \qquad
              \lambda_a \equiv \pi R\,M_a,
          \]
          up to wavefunction renormalizations encoded by simple $\coth$ combinations (explicit ratio formulae below).
          \item \textbf{Explicit ratios vs.\ bulk masses:}
          \begin{align}
          \frac{m_u}{m_W}
          &= \sqrt{\,1-\coth^2\lambda_{20}\,}\;
          \sqrt{\,1+\frac{\coth\lambda_{20}}{\lambda_{20}}+\frac{\coth\lambda_{56}}{\lambda_{56}}\,}\;
          \Bigl(1+\frac{\coth\lambda_{20}}{\lambda_{20}}\Bigr),
          \\[2pt]
          \frac{m_d}{m_W}
          &= \sqrt{2}\;\sqrt{\,1-\coth^2\lambda_{56}\,}\;
          \sqrt{\,1+\varepsilon_{21}\frac{\coth\lambda_{56}}{\lambda_{56}}\,}\;
          \Bigl(1+\varepsilon_{22}\frac{\coth\lambda_{56}}{\lambda_{56}}
          +\varepsilon_{21}\frac{\coth\lambda_{20}}{\lambda_{20}}\Bigr),
          \\[2pt]
          \frac{m_e}{m_W}
          &= \sqrt{\,1-\coth^2\lambda_{15}\,}\;
          \sqrt{\,1+\frac{\coth\lambda_{15}}{\lambda_{15}}\,}\;
          \Bigl(1+\frac{\coth\lambda_{15}}{\lambda_{15}}+\frac{\coth\lambda_{21}}{\lambda_{21}}\Bigr),
          \\[2pt]
          \frac{m_\nu}{m_W}
          &= \sqrt{2}\;\sqrt{\,1-\coth^2\lambda_{21}\,}\;
          \sqrt{\,1+\frac{\coth\lambda_{21}}{\lambda_{21}}\,}\;
          \Bigl(1+\frac{\coth\lambda_{21}}{\lambda_{21}}+\frac{\coth\lambda_{15}}{\lambda_{15}}\Bigr).
          \end{align}
          \item \textbf{Illustrative fit (bulk masses):} A representative set reproducing SM masses (except top) is
          \[
              \lambda_{20}=(5.9,\,2.55,\,0.1),\quad
              \lambda_{56}=(5.65,\,4.1,\,1.1),\quad
              \lambda_{15}=(6.58,\,3.87,\,2.4),\quad
              \lambda_{21}=(13,\,10,\,10)
          \]
          for generations 1–3.
          \item \textbf{EWSB one-loop potential:} With only the four messenger sets, the Coleman–Weinberg potential has its minimum at the origin (no EWSB). Adding a pair of bulk fermions in the $\mathbf{126}$ of SU(6) triggers EWSB with a small dimensionless VEV $\alpha\sim 10^{-2}$, and a Higgs mass
          $m_H \approx 147\,g_4\,\mathrm{GeV}$ at $1/R\sim 0.8~\mathrm{TeV}$.
          \end{itemize}

      \subsection*{M3. SST/VAM dictionary}

          \paragraph{M3.1 Exponential hierarchy as swirl attenuation.}
              Identify each bulk parameter $\lambda_a$ with a dimensionless \emph{swirl attenuation length}
              $\Lambda_a\equiv \sigma_a\,L/L_*$ along the local swirl–moduli geodesic. Then
              \[
                  m_a^{(\text{ae})}=m_W\,e^{-\Lambda_a}\quad\text{(seed masses)},
              \]
              providing UV-insensitive IR seeds for Addendum~J (EJA targets $\{s,\delta\}$ and R2M2 arc–lengths). Species coefficients $\sigma_a$ absorb representation-dependent normalization.

          \paragraph{M3.2 Hypercharge and PS matching.}
              The SU(6) boundary pattern $SU(5)\times U(1)_X$ delivers $g_3{=}g_2{=}\sqrt{5/3}\,g_Y$ at the compactification scale; this is consistent with our Safe–Pati–Salam (Addendum~F) matching and Georgi–Jarlskog boundary conditions in A4.

          \paragraph{M3.3 EWSB: misalignment map.}
              The small $\alpha$ solution naturally fits the pNGB misalignment adapter (Addendum~A3): use $v=f_{\text{ae}}\sin(\Theta/f_{\text{ae}})$ with $\xi_{\text{ae}}{=}v^2/f_{\text{ae}}^2\ll 1$ and take the GHU potential as one possible IR completion generating $\Theta$.

  \subsection*{M4. Minimal calibration protocol}
      \begin{enumerate}
      \item Pick a common geometric scale $L_*\sim r_c$ and initial $\sigma_a$. Calibrate $\sigma_e$ to $m_e$ and $\sigma_\nu$ to a target $m_\nu$.
      \item Use the $\lambda$-ratio formulas as constraints on $\sigma_{u,d,e,\nu}$ (or on the effective arc lengths) so that the seeds match the observed hierarchies before R2M2 leakage.
      \item For EWSB, choose $f_{\text{ae}}$ such that $\xi_{\text{ae}}\ll 1$. If a GHU-like periodic potential is desired, encode its periodicity via a higher-representation analogue of the $\mathbf{126}$ add-on.
      \end{enumerate}

  \subsection*{M5. Consistency gates}
      \begin{itemize}
      \item \textbf{Top mass:} the baseline SU(6) setup underproduces $m_t$ unless higher-rank embeddings or localized gauge–kinetic terms are included (outside this addendum). Use our Safe–PS machinery for the third family (Addendum~F).
      \item \textbf{Custodial/EW veto:} keep $\xi_{\text{ae}}\ll 1$ (Addendum~A3) and enforce $|\Delta\rho|\lesssim 10^{-3}$ (Addendum~C).
      \item \textbf{UV: unification and proton decay:} treat SU(6)-GUT coupling running and baryon number as external constraints; this block does not import those UV commitments into the Canon.
      \end{itemize}

  \section*{References (Bib\TeX)}
  \begin{verbatim}
@article{MaruYatagai2019GHU,
  author        = {Nobuhito Maru and Yoshiki Yatagai},
  title         = {Fermion Mass Hierarchy in Grand Gauge-Higgs Unification},
  year          = {2019},
  eprint        = {1903.08359},
  archivePrefix = {arXiv},
  primaryClass  = {hep-ph}
}
@misc{Wulzer2019BehindSM,
  author        = {Andrea Wulzer},
  title         = {Behind the Standard Model},
  year          = {2019},
  eprint        = {1901.01017},
  archivePrefix = {arXiv},
  primaryClass  = {hep-ph}
}
  \end{verbatim}



% ============== End of content =============

% === Bibliography (only for standalone) ===
  \ifdefined\standalonechapter\else
  \bibliographystyle{unsrt}
  \bibliography{../bridging_blocks_references}
\end{document}
\fi