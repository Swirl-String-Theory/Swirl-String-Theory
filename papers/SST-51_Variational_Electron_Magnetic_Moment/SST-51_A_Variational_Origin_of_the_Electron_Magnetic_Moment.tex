% =====================================================================
% Foundations of Physics (FoP) -- Submission Manuscript
% A Variational Origin of the Electron Magnetic Moment
% Omar Iskandarani
% =====================================================================

\documentclass[pdflatex,sn-mathphys-num]{sn-jnl}

% =========================
% Core packages
% =========================
\usepackage[T1]{fontenc}
\usepackage[utf8]{inputenc}
\usepackage{lmodern}
\usepackage{microtype}
\usepackage{graphicx}
\usepackage{booktabs}
\usepackage{amsmath,amssymb,amsfonts}
\usepackage{amsthm}
\usepackage[hidelinks]{hyperref}

% =========================
% Theorem environments
% =========================
\theoremstyle{thmstyleone}
\newtheorem{proposition}{Proposition}

\theoremstyle{thmstyletwo}
\newtheorem{remark}{Remark}

\theoremstyle{thmstylethree}
\newtheorem{definition}{Definition}

% =========================
% Paper metadata
% =========================
\newcommand{\papertitle}{A Variational Origin of the Electron Magnetic Moment}
\newcommand{\papershorttitle}{Variational Origin of the Electron Magnetic Moment}
\newcommand{\paperdoi}{10.5281/zenodo.18388692}
\newcommand{\paperorcid}{0009-0006-1686-3961}
\newcommand{\paperemail}{info@omariskandarani.com}
\newcommand{\paperlocation}{Groningen, The Netherlands}

\newcommand{\paperkeywords}{electron magnetic moment, self-interaction, dressing, variational principles, extended particle models, Swirl-String Theory}

\raggedbottom

\begin{document}

    \title[\papershorttitle]{\papertitle}

    \author*[1]{\fnm{Omar} \sur{Iskandarani}}\email{\paperemail}

    \affil*[1]{\orgname{Independent Researcher},
        \orgaddress{\city{\paperlocation}, \country{The Netherlands}}}

% =========================
% Abstract
% =========================
    \abstract{%
        Recent work has shown that the electron’s magnetic moment is not uniquely determined at the level of the Dirac equation once electromagnetic self-interaction and dressing are treated consistently. This observation challenges the common textbook narrative in which Dirac theory fixes the Bohr magneton and quantum electrodynamics (QED) merely supplies small radiative corrections. In this paper, we argue that the resulting indeterminacy is not a deficiency of relativistic quantum theory, but rather an indication that an additional physical selection principle is required. We propose that Swirl-String Theory (SST), a hydrodynamic model of extended excitations in an incompressible effective medium, naturally supplies such a principle. Within SST, particle properties—including magnetic moment—arise as constrained extrema of a swirl-energy functional. We show how self-interaction generically produces state-dependent magnetic moments, and how universality can emerge only when the dynamics selects a unique stable configuration within a given topological and polarization class. This reframes the empirical success of QED as evidence for strong dynamical selection rather than point-particle kinematics.
    }

    \keywords{\paperkeywords}

    \maketitle

% ============================================================
    \section{Introduction}\label{sec:intro}
% ============================================================

        The magnetic moment of the electron is widely regarded as one of the most precisely tested quantities in physics. Dirac’s relativistic wave equation predicts a gyromagnetic ratio $g=2$, while quantum electrodynamics (QED) accounts for the small anomalous correction with remarkable numerical accuracy. In standard presentations, the Dirac value is treated as exact at the one-particle level, with radiative effects merely perturbing it.

        Recent analysis has clarified that this narrative rests on implicit assumptions about electromagnetic self-interaction and mass dressing. When the electron’s self-field is treated consistently—already within relativistic single-particle theory—the magnetic moment inferred from the theory becomes state-dependent rather than universal. This raises a foundational question: \emph{why does nature realize a single, sharply defined magnetic moment at all?}

        The purpose of this paper is twofold. First, we summarize the conceptual implications of this self-interaction–induced indeterminacy. Second, we argue that Swirl-String Theory (SST) offers a natural resolution by replacing point-particle attributes with dynamically selected properties of extended excitations.

% ============================================================
    \section{Self-Interaction and the Limits of Dirac-Level Predictions}
        \label{sec:selfinteraction}
% ============================================================

        Standard derivations of the electron’s magnetic moment proceed either by taking the non-relativistic limit of the Dirac equation in an external magnetic field or by decomposing the Dirac current into convective and spin contributions. Both approaches implicitly neglect the backreaction of the electron’s own electromagnetic field on its motion and current distribution.

        When this backreaction is included, the separation between ``bare'' particle properties and ``field'' contributions becomes ambiguous. The magnetic moment inferred from the theory depends on how charge, current, and field energy are distributed in space. In other words, the Dirac equation alone does not fix a unique magnetic moment once self-interaction is treated consistently.

        This observation does not contradict the numerical success of QED. Rather, it exposes a missing conceptual step: a physical mechanism is required to select one particular dressed configuration from a continuum of mathematically admissible ones.

% ============================================================
    \section{Extended Structure and Emergent Properties}
        \label{sec:extended}
% ============================================================

        The state dependence revealed by self-interaction analysis strongly suggests that magnetic moment is not a primitive kinematic attribute. Instead, it is an emergent property of a self-consistent configuration of charge, current, and field.

        This perspective aligns naturally with extended-structure models of particles. If the electron is not a point object but an excitation with internal structure, then its observable properties must be determined by the dynamics governing that structure. Universality, in this context, becomes a nontrivial achievement rather than an assumption.

        The key question becomes: \emph{what physical principle selects the observed configuration?}

% ============================================================
    \section{Swirl-String Theory: Framework and Definitions}
        \label{sec:framework}
% ============================================================

        Swirl-String Theory models particles as coherent excitations of an incompressible, inviscid effective medium. The fundamental degrees of freedom are swirl (vorticity-like) structures rather than point charges. In the ideal limit, the dynamics reduces to conservation laws familiar from fluid mechanics, while particle-like behavior arises from stable, localized configurations.

        Energy in SST is stored in rotational motion of the medium. A generic swirl-energy functional takes the form
        \begin{equation}
            E = \frac{1}{2}\int \rho_{\!f}\,\lVert \mathbf{v} \rVert^2\, dV,
        \end{equation}
        where $\rho_{\!f}$ is the effective fluid density and $\mathbf{v}$ the local swirl velocity. This formulation automatically includes both ``core'' and ``field-like'' contributions, eliminating the artificial distinction between bare and dressed quantities.

% ============================================================
    \section{Magnetic Moment as a Variationally Selected Property}
        \label{sec:variational}
% ============================================================

        Within SST, a magnetic-moment analogue arises from circulating swirl currents. As in the Dirac self-interaction analysis, different internal configurations generically produce different magnetic moments. State dependence is therefore the generic situation, not a pathology.

        Universality enters only when the system is restricted by additional constraints, including:
        \begin{itemize}
            \item conservation of circulation-like quantities,
            \item fixed topological class of the excitation,
            \item and extremization of the swirl-energy functional.
        \end{itemize}

        \begin{proposition}
            The physical electron corresponds to a dynamically stable extremum of the swirl-energy functional within a fixed topological and polarization sector.
        \end{proposition}

        All other configurations are dynamically unstable or inaccessible. In this way, a unique magnetic moment emerges not from kinematics, but from dynamical selection.

% ============================================================
    \section{Photons, Fermions, and the Role of Topology}
        \label{sec:topology}
% ============================================================

        SST naturally distinguishes between particle types by topology and internal degrees of freedom. Bosonic excitations, such as photons, correspond to unknotted swirl structures carrying internal torsional (polarization) modes. Fermionic excitations, such as electrons, correspond to knotted or topologically constrained configurations.

        Topology supplies discrete invariants where required, while variational selection fixes continuous parameters such as magnetic moment. This explains why certain particle properties are robust and universal, while others depend sensitively on dynamical stability.

% ============================================================
    \section{Reinterpreting the Success of QED}
        \label{sec:qed}
% ============================================================

        From this perspective, the empirical success of QED does not imply that particles are fundamentally structureless points. Instead, it suggests that the underlying dynamics strongly favors a unique dressed configuration. QED efficiently computes fluctuations around this configuration without explaining why it is selected in the first place.

        SST offers a complementary explanation: the observed electron corresponds to a unique, dynamically stable swirl-string configuration, and the magnetic moment reflects that stability.

% ============================================================
    \section{Conclusion}
        \label{sec:conclusion}
% ============================================================

        Self-interaction analysis at the Dirac level shows that the electron’s magnetic moment is not uniquely fixed without additional assumptions. This result does not undermine quantum electrodynamics, but it does highlight the need for a physical selection principle.

        Swirl-String Theory provides such a principle by treating particle properties as emergent features of extended excitations selected by variational dynamics and topology. In this framework, universality is not assumed but explained.

        Future work will focus on explicit computation of magnetic-moment analogues within SST and on quantitative comparison with QED predictions.

% ============================================================
        \backmatter

        \bmhead{Acknowledgements}
        Not applicable.

        \bmhead{Declarations}

        \subsection*{Funding}
            Not applicable.

        \subsection*{Competing interests}
            The author declares no competing interests.

        \subsection*{Author contributions}
            O.I. conceived the study, developed the analysis, and wrote the manuscript.

            \bibliography{sn-bibliography}

\end{document}