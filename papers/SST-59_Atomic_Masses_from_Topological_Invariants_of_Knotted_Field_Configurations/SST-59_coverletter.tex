\documentclass[a4paper,10pt]{letter}

\usepackage[T1]{fontenc}
\usepackage[utf8]{inputenc}
\usepackage{lmodern}
\usepackage[hidelinks]{hyperref}
\usepackage{microtype}
\usepackage[margin=1in]{geometry}

\signature{Omar Iskandarani\\
Independent Researcher, Groningen\\
The Netherlands\\
ORCID: 0009-0006-1686-3961\\
Email: \href{mailto:info@omariskandarani.com}{info@omariskandarani.com}}

\address{Omar Iskandarani\\
Vinkenstraat 86A\\
9713 TK Groningen\\
The Netherlands}

\date{\today}

\begin{document}

    \begin{letter}{Editors\\
    \textit{Journal of Physics A: Mathematical and Theoretical}}

\opening{Dear Editors,}

Please consider the enclosed manuscript entitled
\textbf{``Atomic Masses from Topological Invariants of Knotted Field Configurations''}
for publication in \textit{Journal of Physics A: Mathematical and Theoretical}.

The manuscript presents a classical, invariant mass formula in which rest masses of elementary particles, atomic nuclei, and molecules are expressed in terms of purely geometric and topological quantities associated with knotted field configurations. The construction is fully analytical and depends only on dimensionless topological indices (kernel index, genus, component number, and ropelength), together with fixed physical constants.

The central result is an explicit closed-form mass mapping derived from a classical energy density and a topological kernel. All equations appearing in the manuscript are implemented verbatim in a companion reference code, enabling exact numerical reproduction of the results. No adjustable parameters are introduced beyond a single calibration step that fixes the electron ropelength; all other masses follow algebraically.

For composite nuclei, standard nuclear binding energies are incorporated via a clearly identified phenomenological correction. This hybrid treatment explicitly separates the invariant constituent-mass contribution from interaction energies, ensuring conceptual transparency and avoiding reinterpretation of established nuclear models.

The work is intended as a contribution to the mathematical and theoretical study of topology-driven structure in physical models, rather than as a proposal for new interaction dynamics. It may be of interest to readers working on topological methods in physics, invariant formulations, and the mathematical structure underlying mass spectra.

This manuscript is original, has not been published elsewhere, and is not under consideration by any other journal. No figures, text, or results overlap with the author’s other submissions.

Thank you for your time and consideration.


\closing{Sincerely,}

    \end{letter}

\end{document}