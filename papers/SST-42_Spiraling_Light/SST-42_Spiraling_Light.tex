%==============================================================================
% Swirl-String Theory (SST) macro prelude (minimal, self-contained)
%==============================================================================
\providecommand{\rc}{r_c}
\newcommand{\rhoF}{\rho_{\!f}}
\newcommand{\rhoE}{\rho_{\!E}}
\newcommand{\rhoM}{\rho_{\!m}}
\newcommand{\swirlarrow}{\boldsymbol{\circlearrowleft}}
\newcommand{\vswirl}{\mathbf{v}_{\swirlarrow}}
\newcommand{\SwirlClock}{S_t^{\swirlarrow}}

%==============================================================================
\documentclass[11pt]{article}
\usepackage[a4paper,margin=1in]{geometry}
\usepackage{amsmath,amssymb,bm}
\usepackage{siunitx}
\usepackage[hidelinks]{hyperref}

\title{Spiraling Light in Swirl--String Theory:\\
Transverse Orbital Angular Momentum and Off-Axis Tweezer Traps as a Maxwell-Limit Benchmark}
\author{Omar Iskandarani}
\date{2025}

\newcommand{\paperdoi}{10.5281/zenodo.18388680}

\begin{document}
    \maketitle

    \begin{abstract}
        We provide a Swirl--String Theory (SST) interpretation of the spiraling light and optical Magnus-effect phenomenology developed by Spreeuw~\cite{Spreeuw2021SpiralingLight}. In standard electromagnetism, a tightly focused, linearly polarized beam excites a circular dipole whose radiation carries transverse orbital angular momentum (OAM), behaves as if emitted from a source displaced by $k^{-1}=\lambda/2\pi$, and produces spin-dependent off-axis equilibrium positions for atoms trapped in optical tweezers. In SST, photons are modeled as quantized swirl strings in an incompressible substrate; Maxwell electrodynamics appears as the long-wavelength ($\lambda \gg \rc$) limit of the swirl-field equations. We show that (i) the circular-dipole spiral field corresponds to an $\ell=\pm 1$ transverse swirl-string mode, (ii) the apparent source displacement is a geometric centroid of the swirl energy flux, and (iii) the off-axis tweezer displacement is a direct example of matter following gradients of the coarse-grained swirl energy density. The numerical size of possible SST corrections, scaling as $(k\rc)^2$, is shown to be $\mathcal{O}(10^{-16})$ for optical wavelengths, so that current experiments probe the pure Maxwell limit of SST while still providing a precise Rosetta case for photon topology and spin--orbit coupling.
    \end{abstract}

%==============================================================================
    \section{Motivation}

        The analysis in Ref.~\cite{Spreeuw2021SpiralingLight} revisits the angular momentum content of tightly focused beams and the resulting forces on atoms. A key result is that a nominally linearly polarized beam, when focused to large numerical aperture, acquires a nontrivial spatial structure in both polarization and phase. This generates transverse spin, transverse OAM, and an effective source displacement of order $k^{-1}$, together producing an optical Magnus effect and spin-dependent off-axis trapping in optical tweezers.

        Swirl--String Theory (SST) postulates that electromagnetic fields arise as collective excitations of quantized swirl strings in a flat, incompressible fluid-like substrate characterized by an effective density $\rhoF$, a characteristic swirl speed $\vswirl$, and a core length $\rc$. At wavelengths $\lambda \gg \rc$, the coarse-grained swirl dynamics reproduces Maxwell's equations. The spiraling light configuration is therefore an ideal Maxwell-limit benchmark: it is experimentally sharp, heavily constrained by angular-momentum conservation, and geometrically transparent.

        In this note we map the main constructions of Ref.~\cite{Spreeuw2021SpiralingLight} into SST language and identify where potential SST-specific corrections could, in principle, enter.

%==============================================================================
    \section{Maxwell-Limit Summary of Spiraling Light}

        We briefly recall the core standard-electromagnetic results; detailed derivations are given in~\cite{Spreeuw2021SpiralingLight,Allen1992OAM,Bliokh2010NonParaxial}.

        \subsection{Circular dipole and transverse OAM}

            Consider an oscillating electric dipole
            \begin{equation}
                \bm{p}(t) = p_0(\hat{\bm{x}} \pm i \hat{\bm{z}}) e^{-i\omega t},
            \end{equation}
            with angular frequency $\omega$ and wavenumber $k = \omega/c = 2\pi/\lambda$. The $\pm$ signs correspond to opposite senses of rotation. In the dipole plane ($y=0$), the radiated field is (locally) linearly polarized but acquires a spiral phase dependence of the form $e^{\pm i\theta}$, where $\theta$ is the azimuthal angle in the $xz$-plane~\cite{Spreeuw2021SpiralingLight}. This phase winding corresponds to a transverse OAM of $\ell=\pm 1$ per photon for observers in the dipole plane, while on-axis observers see predominant spin.

            The field behaves as if emitted by an effective source displaced by
            \begin{equation}
                \bm{R}_\text{eff} \simeq \pm \frac{1}{k}\,\hat{\bm{y}}
                = \pm \frac{\lambda}{2\pi}\,\hat{\bm{y}},
                \label{eq:Reff-Maxwell}
            \end{equation}
            a purely geometric offset set by the wavelength and the dipole rotation sense~\cite{Spreeuw2021SpiralingLight}. Equation~\eqref{eq:Reff-Maxwell} is dimensionally consistent, with $k^{-1}$ having units of length.

        \subsection{Off-axis tweezer trapping}

            In an optical tweezer formed by a tightly focused beam, atoms experience a conservative potential $U(\bm{r}) \propto -\alpha I(\bm{r})$, where $\alpha$ is the scalar polarizability and $I$ the local intensity. When the beam is configured to excite the circular dipole described above, the interference of the incident and re-radiated fields produces a transverse asymmetry of the intensity and Poynting vector. Ref.~\cite{Spreeuw2021SpiralingLight} shows that this leads to a stable equilibrium position displaced by $\mathcal{O}(\lambda)$ from the nominal optical axis, with opposite Zeeman sublevels trapped at opposite offsets.

            Thus, the spiraling light construction provides:
            \begin{itemize}
                \item a tightly constrained mapping between dipole helicity, transverse OAM, and apparent source displacement;
                \item a specific, experimentally accessible force pattern on trapped atoms.
            \end{itemize}

%==============================================================================
    \section{Swirl--String Photon Model in the Maxwell Limit}

        In SST, a single photon of frequency $\omega$ is modeled as a quantized swirl string: a closed filament of concentrated vorticity with core radius $\rc$ and swirl speed $\vswirl$, embedded in an incompressible background of density $\rhoF$. The swirl energy density $\rhoE$ is related to $\rhoF$ and the local swirl speed, and the mass-equivalent density is $\rhoM = \rhoE / c^2$.

        At long wavelengths ($\lambda \gg \rc$), the coarse-grained dynamics of these swirl strings reduces to linear wave equations for effective fields that can be identified with the electromagnetic field tensor $F_{\mu\nu}$. In this limit, SST predicts that:
        \begin{enumerate}
            \item the dispersion relation is $\omega = c k$ up to corrections of order $(k\rc)^2$;
            \item the field carries total angular momentum $\bm{J} = \bm{L} + \bm{S}$, with $L_z=\ell\hbar$ and $S_z=\sigma\hbar$ emerging from the topology and internal twist of the swirl string;
            \item the leading corrections to Maxwell electrodynamics are suppressed by a dimensionless factor
            \begin{equation}
                \epsilon(k) \sim (k\rc)^2 = \left(\frac{2\pi\rc}{\lambda}\right)^2.
                \label{eq:epsilon-kr}
            \end{equation}
        \end{enumerate}

        Numerically, taking $\rc = 1.40897017\times 10^{-15}\,\mathrm{m}$ and a representative optical wavelength $\lambda = 1.0\times 10^{-6}\,\mathrm{m}$, we obtain
        \begin{equation}
            \epsilon(k) = \left(\frac{2\pi\rc}{\lambda}\right)^2
            \approx 7.8\times 10^{-17},
        \end{equation}
        a dimensionless correction that is far below the sensitivity of current optical experiments. Thus the spiraling light configurations studied in~\cite{Spreeuw2021SpiralingLight} lie deep inside the Maxwell regime of SST.

%==============================================================================
    \section{SST Translation of Spiraling Light}

        \subsection{Circular dipole as a localized swirl-string source}

            In the Maxwell description, the circular dipole is specified by $\bm{p}(t)$ and produces a complex field $\bm{E}(\bm{r},t)$ with spiral phase. In SST, we instead specify a localized swirl-string source whose internal degree of freedom (the SwirlClock $\SwirlClock$) precesses with frequency $\omega$ and helicity $\sigma = \pm 1$. The far-field, after coarse-graining, is described by an effective complex scalar phase
            \begin{equation}
                \Phi_\text{SST}(\bm{r},t) = kz - \omega t + \ell\theta,
            \end{equation}
            with $\ell = \pm 1$ encoding the transverse phase winding inherited from the swirl-string topology. The emergent electromagnetic field is then reconstructed from gradients and time-derivatives of $\Phi_\text{SST}$, reproducing the same $e^{\pm i\theta}$ phase structure found in~\cite{Spreeuw2021SpiralingLight}.

            In this mapping:
            \begin{itemize}
                \item the internal swirl-string twist maps to photon spin $\sigma$;
                \item the spatial winding of the swirl string in the transverse plane maps to photon OAM $\ell$;
                \item the dipole plane corresponds to an observation slice where OAM dominates over spin in the measured angular-momentum density, consistent with the transverse OAM interpretation.
            \end{itemize}

        \subsection{Effective source displacement as a swirl energy centroid}

            Equation~\eqref{eq:Reff-Maxwell} can be rephrased as a statement about the intensity-weighted centroid of the field in the dipole plane. Let $I(\bm{r}_\perp)$ be the time-averaged Poynting flux magnitude through a transverse plane, with $\bm{r}_\perp = (x,z)$ and $y$ the transverse offset. Define the effective emission point as
            \begin{equation}
                \bm{R}_\text{eff}
                = \frac{\displaystyle \int \bm{r}_\perp\,I(\bm{r}_\perp)\,\mathrm{d}^2 r_\perp}
                {\displaystyle \int I(\bm{r}_\perp)\,\mathrm{d}^2 r_\perp}.
                \label{eq:Reff-centroid}
            \end{equation}
            In the Maxwell analysis of~\cite{Spreeuw2021SpiralingLight}, the special structure of $I(\bm{r}_\perp)$ for the circular dipole leads directly to $\bm{R}_\text{eff} \simeq \pm k^{-1}\hat{\bm{y}}$.

            In SST, $I(\bm{r}_\perp)$ is proportional to the swirl energy flux, derived from $\rhoE(\bm{r})$ and the local swirl velocity. In the $\lambda \gg \rc$ regime, the proportionality is linear and isotropic, so that the centroid definition~\eqref{eq:Reff-centroid} is unchanged. Consequently, SST predicts the same universal offset~\eqref{eq:Reff-Maxwell} in the Maxwell limit, with any deviation only entering through the tiny $(k\rc)^2$ corrections of Eq.~\eqref{eq:epsilon-kr}.

        \subsection{Off-axis tweezer displacement as motion in a swirl potential}

            The optical tweezer potential $U(\bm{r}) \propto -\alpha I(\bm{r})$ can be interpreted in SST as a potential energy landscape generated by the swirl energy density $\rhoE(\bm{r})$ and its coupling to the internal degrees of freedom of the trapped atom. The force is
            \begin{equation}
                \bm{F}(\bm{r}) = -\bm{\nabla}U(\bm{r})
                \propto \bm{\nabla} I(\bm{r}),
            \end{equation}
            identical in form to the Maxwell description. The spin-dependent displacement arises because the circular dipole component---and therefore the local swirl-string mode that is most strongly excited---changes with the internal state of the atom. Different Zeeman sublevels couple to different helicities $\sigma$, thus sampling slightly different swirl energy landscapes.

            In SST terms:
            \begin{enumerate}
                \item the internal two-level system of the atom is treated as a localized swirl-clock degree of freedom $\SwirlClock$,
                \item the focused beam prepares a structured swirl-string field with nontrivial $\ell$ and $\sigma$,
                \item the equilibrium positions are minima of the effective potential defined by the swirl energy density around the atom.
            \end{enumerate}
            Because the underlying field is in the Maxwell limit, the resulting equilibrium displacements are of order $\lambda$ and match those computed in~\cite{Spreeuw2021SpiralingLight}. SST neither modifies nor conflicts with these predictions at currently accessible optical parameters.

%==============================================================================
    \section{Outlook and SST-Specific Opportunities}

        The spiraling light construction offers a clean Rosetta example for SST:
        \begin{itemize}
            \item It provides a concrete mapping between swirl-string topology $(\ell,\sigma)$ and measurable spin--orbit effects in focused beams.
            \item It realizes a simple example of matter moving in a structured swirl energy landscape, with spin-dependent equilibrium positions.
            \item It probes the regime $\lambda \gg \rc$, where SST must reproduce Maxwell's equations to extremely high precision.
        \end{itemize}

        A natural next step is to consider regimes where the beam waist approaches microscopic scales or where multiple swirl-string modes interfere in strongly nonparaxial geometries. Within SST, higher-derivative corrections suppressed by $(k\rc)^2$ could then, in principle, generate small deviations from the pure Maxwell predictions in angular momentum flow or equilibrium positions. For $\lambda\sim \SI{1}{\micro\meter}$ and $\rc \sim 10^{-15}\,\mathrm{m}$, the estimated correction $\epsilon(k)\sim 8\times 10^{-17}$ implies that such effects are currently unobservable, but they offer a well-defined target for future high-precision tests.

        In summary, the spiraling light and optical Magnus-effect phenomenology of Ref.~\cite{Spreeuw2021SpiralingLight} is fully compatible with the SST Canon. It provides a useful benchmark problem for the SST photon model, a laboratory example of matter--swirl coupling in the Maxwell limit, and a potential springboard for constraining SST corrections beyond Maxwell electrodynamics.

%==============================================================================
        \bibliographystyle{unsrt}
        \begin{thebibliography}{99}

            \bibitem{Spreeuw2021SpiralingLight}
            R.~J.~C. Spreeuw,
            \newblock Spiraling light: from donut modes to a {Magnus}-effect analogy,
            \newblock \emph{Preprint} (2021), arXiv:2109.03937.

            \bibitem{Allen1992OAM}
            L.~Allen, M.~W. Beijersbergen, R.~J.~C. Spreeuw, and J.~P. Woerdman,
            \newblock Orbital angular momentum of light and the transformation of {Laguerre--Gaussian} laser modes,
            \newblock \emph{Phys. Rev. A} \textbf{45}, 8185--8189 (1992).
            \newblock doi:10.1103/PhysRevA.45.8185.

            \bibitem{Bliokh2010NonParaxial}
            K.~Y. Bliokh, M.~A. Alonso, E.~A. Ostrovskaya, and A.~Aiello,
            \newblock Angular momenta and spin--orbit interaction of nonparaxial light in free space,
            \newblock \emph{Phys. Rev. A} \textbf{82}, 063825 (2010).
            \newblock doi:10.1103/PhysRevA.82.063825.

        \end{thebibliography}

\end{document}