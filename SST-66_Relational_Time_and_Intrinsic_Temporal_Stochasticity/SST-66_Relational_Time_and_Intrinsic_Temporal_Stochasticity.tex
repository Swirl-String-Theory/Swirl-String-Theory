%! Author = Omar Iskandarani
%! Date = 1/29/2026
%! Affiliation = Independent Researcher, Groningen, The Netherlands
%! License = © 2025 Omar Iskandarani. All rights reserved. This manuscript is made available for academic reading and citation only. No republication, redistribution, or derivative works are permitted without explicit written permission from the author. Contact: info@omariskandarani.com
%! ORCID = 0009-0006-1686-3961
%! DOI = 10.5281/zenodo.xxx

\newcommand{\paperdoi}{10.5281/zenodo.xxx}
\newcommand{\papertitle}{Relational Time and Intrinsic Temporal Stochasticity}

%=========================================
% % PREAMBLE, PACKAGES AND DOCUMENT CONFIGURATION
%=========================================
\documentclass[11pt]{article}
\usepackage{amsmath,amssymb,amsfonts,bm}
\usepackage{siunitx}
\usepackage[hidelinks]{hyperref}
\usepackage[a4paper,margin=1in]{geometry}
\usepackage[T1]{fontenc}
\usepackage[utf8]{inputenc}

% swirl arrows (context-aware)
\newcommand{\swirlarrow}{\mkern-2mu\scriptscriptstyle\boldsymbol{\circlearrowleft}}
\newcommand{\vswirl}{\mathbf{v}_{\mkern-2mu\scriptscriptstyle\boldsymbol{\circlearrowleft}}}
\newcommand{\SwirlClock}{S_{(t)}^{\mkern-2mu\scriptscriptstyle\boldsymbol{\circlearrowleft}}}
\newcommand{\Fmaxswirl}{F^{\max}_{\mkern-1mu\scriptscriptstyle\boldsymbol{\circlearrowleft}}}
\newcommand{\Fmax}{F^{\max}_{\mkern-1mu\scriptscriptstyle\boldsymbol{\circlearrowleft}}} 
\newcommand{\FmaxEM}{F^{\max}_{\mathrm{EM}}}
\newcommand{\FmaxG}{F_{\mathrm{G}}^{\max}}               % G-like maximal force scale
\newcommand{\vscore}{v_{\swirlarrow}}                    % shorthand: |v_swirl| at r=r_c
\newcommand{\vnorm}{\lVert \mathbf{v}_{\mkern-2mu\scriptscriptstyle\boldsymbol{\circlearrowleft}} \rVert}  % swirl speed magnitude
\newcommand{\rhoF}{\rho_{\!f}}\newcommand{\rhof}{\rho_{\!f}}     % effective fluid density
\newcommand{\rhoE}{\rho_{\!E}}\newcommand{\rhoe}{\rho_{\!E}}                           % swirl energy density
\newcommand{\rhoM}{\rho_{\!m}}\newcommand{\rhom}{\rho_{\!m}}                           % mass-equivalent density
\newcommand{\omegas}{\boldsymbol{\omega}_{\swirlarrow}}  % swirl vorticity
\newcommand{\Om}{\Omega_{\swirlarrow}}                   % swirl angular frequency profile
\newcommand{\rc}{r_c}                                    % string core radius (swirl string radius)


\newcommand{\titlepageOpen}{
    \begin{titlepage}
        \thispagestyle{empty}  \centering
        \Large \bfseries \papertitle \par \vspace{1cm}
        {\Large \itshape \textbf{Omar Iskandarani}\textsuperscript{\textbf{*}} \par} \vspace{0.5cm}
        {\today \par}  \vspace{0.5cm}
}

\newcommand{\titlepageClose}{
        \vfill \raggedright \null
        \begin{picture}(0,0)
            \put(0,-45){  % Shift 200pt left, 40pt down
                \begin{minipage}[b]{0.7\textwidth} \footnotesize
                    \renewcommand{\arraystretch}{1.0} \noindent\rule{\textwidth}{0.4pt} \\[0.5em]
                    \textsuperscript{\textbf{*}} Independent Researcher, Groningen, The Netherlands \\
                    Email: \texttt{info@omariskandarani.com} \\
                    ORCID: \texttt{\href{https://orcid.org/0009-0006-1686-3961}{0009-0006-1686-3961}} \\
                    DOI: \href{https://doi.org/\paperdoi}{\paperdoi}
                \end{minipage}
            }
        \end{picture}
    \end{titlepage}
}
%=========================================
% Start Document - Title Page
%=========================================
\begin{document}
    \titlepageOpen
    \begin{abstract}
        Swirl-String Theory reformulates physical time as a relational observable
        associated with a conserved event current rather than as an external parameter.
        We explore the phenomenological consequence that clock readouts may exhibit
        intrinsic temporal broadening, independent of environmental noise or relativistic
        time dilation.
        We introduce an operational ansatz for time-of-arrival fluctuations and discuss
        how such effects could scale with gradients of the clock foliation field.
        Finally, we outline exploratory clock-comparison experiments capable of
        constraining this temporal stochasticity and discuss implications for quantum
        measurement and decoherence.
    \end{abstract}

    \titlepageClose


    \section{Relational Time in Swirl--String Theory}

        In Swirl--String Theory (SST), physical time is not introduced as a
        fundamental external parameter.
        Instead, time is defined operationally through a conserved event current
        associated with the underlying foliation field.
        Clock readings correspond to the accumulation of discrete events along
        integral curves of a preferred timelike vector field $u^\mu$.

        This construction is fixed canonically in SST--31 and Canon v0.7.7 and
        is equivalent in the infrared to hypersurface--orthogonal
        Einstein--\AE ther (khronometric) theory.
        No stochasticity is postulated at the level of the action.


    \section{Intrinsic Temporal Broadening}

        While the event current is conserved, its operational realization
        in physical clocks involves counting discrete microscopic processes.
        As a consequence, clock readouts need not be perfectly sharp even in
        the absence of environmental noise.

        We define \emph{intrinsic temporal broadening} as a fundamental variance
        in time--of--arrival measurements that persists after all known
        classical, thermal, quantum, and relativistic corrections are removed.

        Operationally, the observed time--of--arrival distribution
        $P_{\mathrm{obs}}(\Theta)$ is modeled as
        \begin{equation}
            P_{\mathrm{obs}}(\Theta)
            =
            \int dt\;
            P_{\mathrm{cl}}(t)
            \frac{1}{\sqrt{2\pi\sigma_\tau^2}}
            \exp\!\left[
                      -\frac{(\Theta - t)^2}{2\sigma_\tau^2}
            \right],
        \end{equation}
        where $\sigma_\tau^2$ parametrizes intrinsic temporal stochasticity.


    \section{Scaling Ansatz and Physical Interpretation}

        Canon v0.7.7 implies that relational time is defined relative to the
        clock--foliation field.
        Accordingly, intrinsic temporal broadening may depend on spatial
        inhomogeneities of the foliation.

        We therefore adopt the phenomenological ansatz
        \begin{equation}
            \sigma_\tau^2
            =
            \sigma_\tau^2(\nabla \chi),
        \end{equation}
        where $\chi$ denotes the scalar foliation (clock) potential.

        Physically, this dependence reflects fluctuations in the local event
        rate induced by gradients of the clock field.
        No claim is made regarding the microscopic origin or spectral structure
        of this noise, which remains an open parameterization.


    \section{Experimental Search Channels and Falsification}

        Intrinsic temporal broadening predicts observable consequences only in
        high--precision differential clock experiments.
        Suitable search channels include:
        \begin{itemize}
            \item co--located optical lattice clocks operated in differential mode,
            \item entangled clock pairs sensitive to phase decoherence,
            \item controlled modulation of local clock--field gradients.
        \end{itemize}

        A null result at sensitivity $\sigma_\tau \lesssim 10^{-19}\,\mathrm{s}/\sqrt{\mathrm{Hz}}$
        places direct bounds on the admissible stochastic sector of SST.
        Conversely, observation of clock decoherence correlated with
        $\nabla\chi$ would constitute evidence for relational time fluctuations
        distinct from standard quantum or gravitational noise.

        This work therefore defines a falsifiable experimental program rather
        than a prediction of detectability.

    \section{Discussion and Outlook}

        \subsection{Summary of Results}

            This work reformulates physical time in Swirl--String Theory (SST) as a
            relational observable associated with a conserved event current rather than
            as an external evolution parameter.
            Within this framework, clock readouts are not assumed \emph{a priori} to
            correspond to a perfectly smooth variable, even in isolated systems.

            We introduced an operational ansatz for intrinsic temporal broadening,
            modeled as a convolution of ideal clock evolution with a stochastic kernel
            characterized by a variance $\sigma_\tau^2$.
            This quantity parametrizes intrinsic clock noise that is conceptually
            distinct from environmental disturbances, relativistic time dilation, or
            measurement backaction.

            No claim has been made regarding the magnitude or detectability of
            $\sigma_\tau^2$ with existing technology.
            The purpose of this work is to define a consistent phenomenological framework
            and an associated experimental search channel.

        \subsection{Relation to Other SST Results}

            The temporal stochasticity discussed here is logically independent of
            velocity-dependent mass effects and preferred-frame corrections addressed
            elsewhere.
            In particular, gravity-sector tests of foliation-referenced mass anisotropy
            are treated separately and do not enter the present analysis.

            Accordingly, this work should be read as complementary to, but distinct from,
            operational follow-ups of the SST mass functional.
            The results presented here depend only on the relational definition of time
            fixed in the SST Canon and do not modify the gravitational sector.

        \subsection{Falsification and Outlook}

            The framework presented admits clear falsification pathways.

            A null result in high-precision differential clock experiments constrains the
            allowed stiffness and smoothness of the SST clock foliation.
            Conversely, observation of excess decoherence or time-of-arrival broadening
            that scales with controlled gradients of the clock field, while surviving
            standard noise rejection protocols, would provide evidence for intrinsic
            temporal stochasticity.

            Future work may refine the spectral properties of $\sigma_\tau^2$,
            investigate its coupling to quantum coherence, and explore its implications
            for foundational questions in quantum measurement and time observables.

            The present analysis establishes the minimal phenomenological groundwork
            required for such investigations, without extending beyond empirically
            testable statements.



\end{document}