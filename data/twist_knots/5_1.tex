\documentclass[tikz, border=10pt]{standalone}
\usepackage{tikz}
\usetikzlibrary{knots, hobby, calc, intersections, decorations.pathreplacing, decorations.markings, shapes.geometric, spath3}

% ==============================================================================
%  SST RENDER ENGINE: Twisted Dual-Strand Logic
% ==============================================================================

% -- Configuration --
\def\Wth{2.5pt}          % Strand width
\def\AMP{.1}
\def\TWISTS{5}        % amount twist per full length
\def\Samples{240}        % Resolution
\def\Core{black}         % Gap color (background color)
\def\Mask{5.0pt}         % Mask width for overlap
\def\ClrA{black!60!black}      % Strand A Color (Shadow)
\def\ClrB{red!70!black}  % Strand B Color (Main)

% -- Strand Style --
\tikzset{
    fat strand/.style={
        draw=#1, line width=\Wth,
        double=\Core, double distance=1.2*\Wth,
        line cap=round, line join=round
    }
}

% -- Coordinate Generator --
% #1: Path Definition, #2: Samples, #3: Amp, #4: Turns, #5: StrandName(A/B)
\newcommand{\MakeStrandCoords}[5]{%
    \pgfmathtruncatemacro{\Ns}{#2}
    \def\Phase{0}\ifx#5B\def\Phase{0.5}\fi
    \foreach \i in {0,...,\Ns}{%
        \pgfmathsetmacro{\s}{\i/\Ns}%
        \pgfmathsetmacro{\y}{#3*sin(360*(#4*\s + \Phase))}%
        \path[postaction={decorate, decoration={markings,
        mark=at position \s with {\coordinate (#5\i) at (0,\y);}}}] #1;%
    }%
}

% -- Segment Drawer --
\newcommand{\drawSeg}[4]{% #1:Name, #2:Index, #3:Color, #4:Over(0/1)
    \ifnum#4=1
    \draw[preaction={draw=\Core, line width=\Mask}, fat strand=#3]
    (#1#2) .. (#1\the\numexpr#2+1\relax);
    \else
        \draw[fat strand=#3]
        (#1#2) .. (#1\the\numexpr#2+1\relax);
    \fi
}

% -- Main Render Command --
% #1: Path Macro (e.g., \KPATH), #2: Turns, #3: Amplitude
\newcommand{\RenderTwist}[3]{
    \MakeStrandCoords{#1}{\Samples}{#3}{#2}{A}
    \MakeStrandCoords{#1}{\Samples}{#3}{#2}{B}
    \pgfmathtruncatemacro{\Last}{\Samples-1}
    \foreach \k in {0,...,\Last}{
        \pgfmathsetmacro{\s}{\k/\Samples}
        % Twist logic: swaps A/B depth based on sine phase
        \pgfmathtruncatemacro{\blk}{floor(2*#2*\s + 0.5)}
        \ifodd\blk
        \drawSeg{A}{\k}{\ClrA}{0}
        \drawSeg{B}{\k}{\ClrB}{1}
        \else
            \drawSeg{B}{\k}{\ClrB}{0}
            \drawSeg{A}{\k}{\ClrA}{1}
        \fi
    }
}

% -- Debug/Guide Points (Optional) --
\newcommand{\SSTGuidesPoints}[2]{
    \foreach \i in {1,...,#2}{
        \fill[blue] (P\i) circle (1.5pt);
        \node[blue,font=\tiny,above right] at (P\i) {\i};
    }
}

\begin{document}


% ==============================================================================
%  3. CINQUEFOIL (5_1)
%     "Higher Generation Resonance"
% ==============================================================================
    \begin{tikzpicture}[use Hobby shortcut]
        % Explicit Coordinates: R_outer=2.2, R_inner=-1.2 (Star Polygon)
        \coordinate (P1)  at (0, 2.2);            % Top Lobe
        \coordinate (P2)  at (0.705, -0.971);     % Inner 1
        \coordinate (P3)  at (-2.092, 0.680);     % Lobe 2
        \coordinate (P4)  at (1.141, 0.371);      % Inner 2
        \coordinate (P5)  at (-1.293, -1.779);    % Lobe 3
        \coordinate (P6)  at (0, 1.200);          % Inner 3 (Top Crossing)
        \coordinate (P7)  at (1.293, -1.779);     % Lobe 4
        \coordinate (P8)  at (-1.141, 0.371);     % Inner 4
        \coordinate (P9)  at (2.092, 0.680);      % Lobe 5
        \coordinate (P10) at (-0.705, -0.971);    % Inner 5
        \coordinate (P11) at (0, 2.2);            % Close = P1



        \begin{knot}[
            consider self intersections,
        %draft mode=crossings,              % show crossing indices while tuning
            ignore endpoint intersections=true,
%    clip width=5pt, clip radius=3pt,
            flip crossing/.list={2,4,6,8,10,12,14,16,18,20}        % same over/under pattern as your snippet
        ]

            \def\KPATH{
                ([closed] P1)..(P2)..(P3)..(P4)..(P5)..(P6)..(P7)..(P8)..(P9)..(P10)..(P11)
            }

            \RenderTwist{\KPATH}{\TWISTS}{\AMP}
            \node[font=\bfseries\sffamily] at (0,-2.5) {Cinquefoil ($5_1$)};


        \end{knot}
    \end{tikzpicture}


\end{document}