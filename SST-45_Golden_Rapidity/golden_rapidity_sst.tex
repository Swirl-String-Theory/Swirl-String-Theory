% =====================================================================
% Swirl–String Theory (SST)
% Golden Hyperbolic Swirl-Angle and Tangential Swirl Speed
% Author: Omar Iskandarani
% =====================================================================

\documentclass[11pt]{article}

\usepackage[T1]{fontenc}
\usepackage{lmodern}
\usepackage{microtype}
\usepackage{amsmath,amssymb,bm}
\usepackage{amsthm}
\usepackage{siunitx}
\usepackage[hidelinks]{hyperref}
\usepackage[margin=1in]{geometry}

\sisetup{
    scientific-notation = true,
    exponent-product = \times,
    output-product = \cdot
}

\title{Golden Hyperbolic Swirl-Angle and Tangential Swirl Speed in Swirl--String Theory (SST)}
\author{Omar Iskandarani}
\date{\today}

% ========================
% SST symbols (compile-safe)
% ========================
\newcommand{\vswirl}{\mathbf{v}_{\!\boldsymbol{\circlearrowleft}}} % canonical characteristic swirl velocity (vector)
\newcommand{\vnorm}[1]{\left\lVert #1 \right\rVert}
\newcommand{\rc}{r_c}

% Hyperbolic-first golden ratio
\newcommand{\xistar}{\operatorname{asinh}\!\left(0.5\right)}
\newcommand{\phiG}{e^{\ \xistar}}

% Golden layer swirl-angle
\newcommand{\xiG}{\tfrac{3}{2}\,\xistar}

\newtheorem{theorem}{Theorem}
\newtheorem{lemma}{Lemma}

\begin{document}
    \maketitle

    \begin{abstract}
        This paper introduces a hyperbolic, kinematics-only construction used to define a distinguished ``golden layer'' for tangential swirl motion in Swirl--String Theory (SST).
        We adopt a hyperbolic definition of the golden ratio,
        \(\phi := e^{\ \operatorname{asinh}(1/2)}\), based on standard inverse-hyperbolic identities. \cite{NISTDLMF-asinh}
        Defining the \emph{golden hyperbolic swirl-angle} \(\xi_g := \tfrac{3}{2}\operatorname{asinh}(1/2)\), we prove the closed identity
        \(\tanh(\xi_g)=e^{\ -\operatorname{asinh}(1/2)}=\phi^{-1}\) using standard hyperbolic-function relations. \cite{NISTDLMF-hyp}
        We then define a dimensionless tangential swirl fraction \(\beta:=\lVert \mathbf{v}\rVert/v_{\circlearrowleft}\in[0,1)\), where \(v_{\circlearrowleft}=\lVert\vswirl\rVert\) is the SST canonical reference swirl speed. \cite{SSTCanon2025}
        A \emph{hyperbolic swirl-angle} \(\xi\) is introduced purely as a bijective re-parameterization of \(\beta\) via \(\beta=\tanh\xi\). \cite{NISTDLMF-hyp}
        At the golden layer \(\xi=\xi_g\), the tangential fraction is fixed to \(\beta_g=\phi^{-1}\), implying
        \(\lVert\mathbf{v}\rVert_g=v_{\circlearrowleft}/\phi\) and \(\Omega_g=v_{\circlearrowleft}/(\phi r_c)\).
        Numerical evaluation is given using the SST constants \( v_{\circlearrowleft}=\SI{1.09384563e6}{m.s^{-1}}\) and \(r_c=\SI{1.40897017e-15}{m}\). \cite{SSTCanon2025}
    \end{abstract}

    \section{Introduction and motivation}
        Hyperbolic functions provide a natural coordinate system for any quantity constrained to the open interval \([0,1)\), because \(\tanh:\mathbb{R}\to(-1,1)\) is smooth, monotone, and saturating. \cite{NISTDLMF-hyp}
        In many areas of mathematical physics, a bounded ``fraction'' \(\beta\) is therefore re-parameterized by a hyperbolic angle \(\xi\) such that \(\beta=\tanh\xi\); this is a coordinate choice, not a dynamical assumption. \cite{NISTDLMF-hyp,Rindler2006}

        In SST, a canonical tangential swirl-speed scale \(v_{\circlearrowleft}\) and a core length scale \(r_c\) are treated as fundamental constants in the theory's kinematic and energetic bookkeeping. \cite{SSTCanon2025}
        When discussing tangential swirl fractions \(\beta=\lVert\mathbf{v}\rVert/v_{\circlearrowleft}\), it is therefore useful to introduce a \emph{hyperbolic swirl-angle} \(\xi\) as a compact, analytic parameterization of \(\beta\).

        This paper adds one further ingredient: a hyperbolic definition of the golden ratio and a corresponding \emph{golden swirl-angle} \(\xi_g\) at which \(\tanh\xi_g=\phi^{-1}\). Standard properties of \(\phi\) are classical and widely documented; here we emphasize a hyperbolic-first presentation. \cite{Koshy2001,Livio2002}

    \section{Definitions and notation}
        \subsection{SST kinematic quantities}
            Let \(\vswirl\) denote the SST canonical characteristic swirl velocity vector, and define the reference speed
            \begin{equation}
                v_{\circlearrowleft} \;\equiv\; \vnorm{\vswirl}.
                \label{eq:vref}
            \end{equation}
            Let \(\mathbf{v}\) denote a local tangential swirl velocity (e.g.\ associated with a swirl-string segment). Define the dimensionless tangential fraction
            \begin{equation}
                \boxed{
                    \beta \;\equiv\; \frac{\vnorm{\mathbf{v}}}{v_{\circlearrowleft}} \in [0,1).
                }
                \label{eq:beta_def}
            \end{equation}
            The interval restriction \(\beta<1\) is definitional: \(v_{\circlearrowleft}\) is used as the sector reference scale. \cite{SSTCanon2025}

        \subsection{Hyperbolic swirl-angle}
            We introduce a \emph{hyperbolic swirl-angle} \(\xi\) by the bijection
            \begin{equation}
                \boxed{
                    \beta = \tanh\xi,
                    \qquad
                    \xi = \operatorname{artanh}(\beta).
                }
                \label{eq:beta_xi}
            \end{equation}
            The functions \(\tanh\) and \(\operatorname{artanh}\) and their standard identities are classical. \cite{NISTDLMF-hyp}

            \paragraph{Limit checks.}
                From standard expansions, \(\tanh\xi\sim \xi\) as \(\xi\to 0\), so \(\beta\approx\xi\) for small tangential fractions. \cite{NISTDLMF-hyp}
                As \(\beta\to 1^{-}\), one has \(\xi=\operatorname{artanh}(\beta)\to +\infty\), consistent with the open interval \([0,1)\). \cite{NISTDLMF-hyp}

    \section{Hyperbolic definition of the golden ratio}
    We adopt a hyperbolic-first definition:
    \begin{equation}
        \boxed{
            \phi \;\equiv\; \phiG,
            \qquad
            \xistar \equiv \operatorname{asinh}\!\left(\tfrac{1}{2}\right).
        }
        \label{eq:phi_def}
    \end{equation}
    The defining inverse-hyperbolic identity used here is standard:
    \(\operatorname{asinh}(x)=\ln\!\big(x+\sqrt{x^2+1}\big)\). \cite{NISTDLMF-asinh}

    \begin{lemma}[Algebraic corollary]
        The hyperbolic definition \eqref{eq:phi_def} implies the familiar algebraic relation
        \begin{equation}
            \phi^2=\phi+1,
            \qquad
            \text{hence}\quad
            \phi=\frac{1+\sqrt{5}}{2}.
            \label{eq:phi_algebraic_cor}
        \end{equation}
    \end{lemma}

    \begin{proof}
        Let \(t:=e^{\ \xistar}>0\). Since \(\xistar=\operatorname{asinh}(1/2)\), we have \(\sinh(\xistar)=1/2\). \cite{NISTDLMF-asinh,NISTDLMF-hyp}
        Using \(\sinh(\xistar)=\frac{t-t^{-1}}{2}\), \cite{NISTDLMF-hyp} we obtain
        \[
            \frac{t-t^{-1}}{2}=\frac{1}{2}
            \;\Longrightarrow\;
            t-\frac{1}{t}=1
            \;\Longrightarrow\;
            t^2-t-1=0
            \;\Longrightarrow\;
            t^2=t+1.
        \]
        By definition \(\phi=t\), so \(\phi^2=\phi+1\). The quadratic solution gives \(\phi=(1+\sqrt5)/2\) since \(\phi>0\). Classical properties of \(\phi\) are standard. \cite{Koshy2001,Livio2002}
    \end{proof}

    \section{Golden hyperbolic swirl-angle identity}
    Define the \emph{golden hyperbolic swirl-angle}
    \begin{equation}
        \boxed{
            \xi_g \;\equiv\; \frac{3}{2}\,\xistar
            \;=\;
            \frac{3}{2}\operatorname{asinh}\!\left(\tfrac{1}{2}\right).
        }
        \label{eq:xi_g_def}
    \end{equation}

    \begin{theorem}[Golden swirl-angle implies golden tangential fraction]
        With \(\phi= e^(\ \operatorname{asinh}(1/2))\) and \(\xi_g\) defined by \eqref{eq:xi_g_def}, the following identity holds:
        \begin{equation}
            \boxed{
                \tanh(\xi_g) \;=\; e^{-\xistar} \;=\; \phi^{-1}.
            }
            \label{eq:golden_identity}
        \end{equation}
    \end{theorem}

    \begin{proof}
        Let \(t:=e^{\ \xistar}\). From Lemma~\eqref{eq:phi_algebraic_cor} we have \(t^2=t+1\).
        Using the standard identity \(\tanh y=\frac{e^{2y}-1}{e^{2y}+1}\), \cite{NISTDLMF-hyp} with \(y=\xi_g=\tfrac{3}{2}\xistar\),
        \[
            \tanh(\xi_g)
            =
            \frac{e^{\ 3\xistar}-1}{e^{\ 3\xistar}+1}
            =
            \frac{t^3-1}{t^3+1}.
        \]
        Since \(t^2=t+1\), we obtain \(t^3=t(t+1)=t^2+t=2t+1\). Therefore
        \[
            \tanh(\xi_g)
            =
            \frac{(2t+1)-1}{(2t+1)+1}
            =
            \frac{2t}{2t+2}
            =
            \frac{t}{t+1}
            =
            \frac{t}{t^2}
            =
            \frac{1}{t}
            =
            e^{\ -\xistar}.
        \]
        Finally, \(\phi=e^{\ \xistar}\) by \eqref{eq:phi_def}, hence \(\tanh(\xi_g)=\phi^{-1}\).
    \end{proof}

    \section{SST golden layer for tangential swirl kinematics}
    Combining the SST kinematic mapping \eqref{eq:beta_xi} with Theorem~\eqref{eq:golden_identity}, the golden layer is defined by \(\xi=\xi_g\) and yields the fixed fraction
    \begin{equation}
        \boxed{
            \beta_g \;\equiv\; \tanh(\xi_g)=\phi^{-1}.
        }
        \label{eq:beta_g}
    \end{equation}
    Thus, for any SST sector employing the hyperbolic swirl-angle coordinate \(\xi\), the golden layer implies the tangential speed scale
    \begin{equation}
        \boxed{
            \vnorm{\mathbf{v}}_g=\beta_g\,v_{\circlearrowleft}=\frac{v_{\circlearrowleft}}{\phi}.
        }
        \label{eq:vg}
    \end{equation}

    If \(r_c\) denotes the SST core length scale, a corresponding angular-frequency scale can be introduced by dimensional normalization,
    \begin{equation}
        \boxed{
            \Omega \equiv \frac{v_{\circlearrowleft}}{r_c},
            \qquad
            \Omega_g \equiv \frac{\vnorm{\mathbf{v}}_g}{r_c}
            =
            \frac{1}{\phi}\,\frac{v_{\circlearrowleft}}{r_c}
            =
            \frac{\Omega}{\phi}.
        }
        \label{eq:omegag}
    \end{equation}
    This step is purely dimensional: it does not assert a specific dynamical equation for \(\Omega\). The constants \(v_{\circlearrowleft}\) and \(r_c\) are taken from SST canon. \cite{SSTCanon2025}

    \paragraph{Dimensional consistency.}
        Equation \eqref{eq:vg} has units of speed, and \eqref{eq:omegag} has units of \(\si{s^{-1}}\), since \(r_c\) has units of length. \cite{SSTCanon2025}

    \section{Numerical evaluation using SST constants}
    We evaluate \(\phi\), \(\xi_g\), and the corresponding kinematic scales using
    \[
        v_{\circlearrowleft}=\SI{1.09384563e6}{m.s^{-1}},
        \qquad
        r_c=\SI{1.40897017e-15}{m},
    \]
    as specified in SST canon. \cite{SSTCanon2025}

    Using the hyperbolic definitions and standard functions, \cite{NISTDLMF-asinh,NISTDLMF-hyp} we obtain
    \begin{align}
        \phi &= e^{\ (\operatorname{asinh}(1/2))} \approx 1.618033988749895,\\
        \xistar &= \operatorname{asinh}(1/2) \approx 0.4812118250596035,\\
        \xi_g &= \tfrac{3}{2}\xistar \approx 0.7218177375894053,\\
        \beta_g &= \tanh(\xi_g) \approx 0.6180339887498948 = \phi^{-1}.
    \end{align}
    Therefore,
    \begin{align}
        \vnorm{\mathbf{v}}_g &= \frac{v_{\circlearrowleft}}{\phi}
        \approx \SI{6.760337777855416e5}{m.s^{-1}},\\
        \Omega &= \frac{v_{\circlearrowleft}}{r_c}
        \approx \SI{7.763440655383073e20}{s^{-1}},\\
        \Omega_g &= \frac{\Omega}{\phi}
        \approx \SI{4.798070194669498e20}{s^{-1}}.
    \end{align}

    \section{Interpretation and mainstream-facing remarks}
    \subsection{What is ``new'' here}
        The hyperbolic function identities and the classical properties of \(\phi\) used above are not new. \cite{NISTDLMF-asinh,NISTDLMF-hyp,Koshy2001}
        The SST-specific contribution is the \emph{interpretive packaging}:
        \begin{itemize}
            \item introduce \(\xi\) as a \emph{hyperbolic swirl-angle} coordinating the SST tangential fraction \(\beta\),
            \item define the golden layer \(\xi_g\) and note that \(\beta_g=\phi^{-1}\) becomes an SST-internal, dimensionless marker,
            \item express the associated speed and frequency scales in terms of SST canonical constants \(v_{\circlearrowleft}\) and \(r_c\). \cite{SSTCanon2025}
        \end{itemize}

    \subsection{Why use the hyperbolic coordinate at all}
        The mapping \(\beta=\tanh\xi\) has two practical advantages:
        (i) it automatically enforces \(\beta\in[0,1)\) for all real \(\xi\); \cite{NISTDLMF-hyp}
        (ii) it linearizes certain algebraic manipulations because exponentials appear directly in \(\tanh\) identities. \cite{NISTDLMF-hyp}
        In SST, this makes it convenient to talk about ``layers'' in \(\xi\) rather than repeatedly carrying bounded fractions \(\beta\).

    \subsection{Analogy (age 10)}
        Imagine a speed slider that can never go above 100\%. The hyperbolic swirl-angle is like a special knob behind the slider: turning the knob by equal steps changes the slider smoothly, and there is one special knob position that always lands exactly at the same famous fraction (about 62\%), the golden one.

    \section*{Acknowledgements}
    The numerical values used for \(v_{\circlearrowleft}\) and \(r_c\) are taken from the SST canon. \cite{SSTCanon2025}

% ========================
% Bibliography (bibitem only)
% ========================
    \begin{thebibliography}{9}

        \bibitem{NISTDLMF-asinh}
        F.~W.~J. Olver, A.~B. Olde Daalhuis, D.~W. Lozier, B.~I. Schneider, R.~F. Boisvert,
        C.~W. Clark, B.~R. Miller, and B.~V. Saunders (eds.) (2010--present),
        \textit{NIST Digital Library of Mathematical Functions},
        National Institute of Standards and Technology.
        Section 4.37 ``Inverse Hyperbolic Functions'' (definition and properties of \(\operatorname{asinh}\)).
        Permalink: \texttt{https://dlmf.nist.gov/4.37}.

        \bibitem{NISTDLMF-hyp}
        F.~W.~J. Olver, A.~B. Olde Daalhuis, D.~W. Lozier, B.~I. Schneider, R.~F. Boisvert,
        C.~W. Clark, B.~R. Miller, and B.~V. Saunders (eds.) (2010--present),
        \textit{NIST Digital Library of Mathematical Functions},
        National Institute of Standards and Technology.
        Section 4.35 ``Hyperbolic Functions'' (identities for \(\tanh\), \(\sinh\), and \(\operatorname{artanh}\)).
        Permalink: \texttt{https://dlmf.nist.gov/4.35}.

        \bibitem{Rindler2006}
        W.~Rindler (2006),
        \textit{Relativity: Special, General, and Cosmological},
        2nd ed., Oxford University Press, Oxford.
        (Reference for the mainstream use of hyperbolic-angle parameterizations of bounded velocity fractions in mathematical physics.)

        \bibitem{Koshy2001}
        T.~Koshy (2001),
        \textit{Fibonacci and Lucas Numbers with Applications},
        Wiley-Interscience, New York.
        (Reference for classical identities of the golden ratio, including \(\phi^2=\phi+1\).)

        \bibitem{Livio2002}
        M.~Livio (2002),
        \textit{The Golden Ratio: The Story of Phi, the World's Most Astonishing Number},
        Broadway Books, New York.
        (General reference for historical and mathematical context of \(\phi\).)

        \bibitem{SSTCanon2025}
        O.~Iskandarani (2025),
        \textit{Swirl String Theory (SST) Canon v0.6.0},
        Zenodo.
        DOI: \texttt{10.5281/zenodo.17899592}.

    \end{thebibliography}

\end{document}