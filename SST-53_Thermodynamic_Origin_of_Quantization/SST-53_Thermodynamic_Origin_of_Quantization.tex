% =====================================================================
% SST Paper Draft (single-file) -- Thermodynamic Origin of Quantization
% Step-2 update: triad normalized to (rho_f, r_c, ||v_swirl||),
%                Gamma0 derived, expanded numerical anchors table.
% =====================================================================

\documentclass[pdflatex,sn-mathphys-num]{sn-jnl}

% -------------------------
% Core packages
% -------------------------
\usepackage[T1]{fontenc}
\usepackage[utf8]{inputenc}
\usepackage{lmodern}
\usepackage{microtype}
\usepackage{amsmath,amssymb,mathtools}
\usepackage{physics}
\usepackage{booktabs}
\usepackage{siunitx}
\usepackage{hyperref}

% -------------------------
% SST macro prelude (house)
% -------------------------
\newcommand{\VolH}[1]{\operatorname{Vol}_{\!\mathbb{H}}(#1)}
\newcommand{\swirlarrow}{\mathchoice{\mkern-2mu\scriptstyle\boldsymbol{\circlearrowleft}}{\mkern-2mu\scriptstyle\boldsymbol{\circlearrowleft}}{\mkern-2mu\scriptscriptstyle\boldsymbol{\circlearrowleft}}{\mkern-2mu\scriptscriptstyle\boldsymbol{\circlearrowleft}}}
\newcommand{\swirlarrowcw}{\mathchoice{\mkern-2mu\scriptstyle\boldsymbol{\circlearrowright}}{\mkern-2mu\scriptstyle\boldsymbol{\circlearrowright}}{\mkern-2mu\scriptscriptstyle\boldsymbol{\circlearrowright}}{\mkern-2mu\scriptscriptstyle\boldsymbol{\circlearrowright}}}
\newcommand{\Vol}{\operatorname{Vol}}
\newcommand{\rhoF}{\rho_{\!f}}   % effective fluid density
\newcommand{\rhoE}{\rho_{\!E}}   % swirl energy density
\newcommand{\rhoM}{\rho_{\!m}}   % mass-equivalent density = rho_E/c^2
\newcommand{\rhocore}{\rho_{\text{core}}}
\newcommand{\vswirl}{\mathbf{v}_{\swirlarrow}}
\newcommand{\vswirlcw}{\mathbf{v}_{\swirlarrowcw}}
\newcommand{\SwirlClock}{S_t^{\swirlarrow}}
\newcommand{\SwirlClockcw}{S_t^{\swirlarrowcw}}
\newcommand{\vnorm}{\lVert \vswirl \rVert}
\providecommand{\rc}{r_c}

% -------------------------
% Useful constants (SI)
% NOTE: c and h are exact in SI post-2019 redefinition.
% -------------------------
\newcommand{\clight}{c}
\newcommand{\hexact}{h}
\newcommand{\hbarexact}{\hbar}

% -------------------------
% Metadata
% -------------------------
\title{Thermodynamic Origin of Quantization in Swirl--String Theory:\\
From Clausius Work--Heat Structure to Parameter-Free Constants}

\author[1]{Omar Iskandarani}
\affil[1]{Independent Researcher, Groningen, The Netherlands}

\abstract{
    We present a thermodynamic foundation for quantization within Swirl--String Theory (SST),
    a hydrodynamic model in which matter is represented by closed circulation-carrying filaments
    (``swirl strings'') embedded in an inviscid, incompressible vacuum-like medium.
    The central claim is that Planck-scale quantization need not be postulated: discretization
    arises from a topological circulation invariant together with a Clausius-consistent
    work/heat decomposition. Using the Abe--Okuyama mapping between Clausius equality and the
    Shannon/von Neumann structure, we formalize a controlled route from pure-state mechanics
    to quantum thermodynamics and identify the mechanical meaning of SST work (geometric deformation)
    and SST heat (redistribution over internal Kelvin-wave modes). We then define a minimal
    primitive set of medium parameters and show how familiar constants may be expressed as derived
    quantities under a ``zero-free-parameter'' program, with explicit dimensional checks and
    numerical anchors. Finally, we state falsifiable predictions, including a low-temperature
    heat-capacity scaling tied to filament mode structure and an acceleration-induced echo channel
    analogous to Unruh response.
}

\keywords{thermodynamics, quantization, circulation, hydrodynamic quantum analogs, Clausius equality}

\newcommand{\paperdoi}{10.5281/zenodo.18388696}

\begin{document}
    \maketitle

% ============================================================
    \section{Scope, notation, and claim taxonomy}
% ============================================================

        \subsection{Terminology}
            To minimize ambiguity we use:

            \begin{itemize}
                \item \textbf{Swirl medium:} an inviscid, incompressible condensate-like vacuum model (effective density \(\rhoF\)).
                \item \textbf{Swirl string:} a closed, circulation-carrying filament (topological defect) supporting internal wave modes.
                \item \textbf{Circulation:} \(\Gamma \equiv \oint \mathbf{v}\cdot d\mathbf{l}\).
                \item \textbf{Zero-free-parameter principle:} after fixing a minimal primitive set of mechanical inputs, all other
                constants (e.g.\ \(h,e,m_e\)) are derived without additional tuning.
            \end{itemize}

        \subsection{Claim taxonomy}
            We distinguish:
            \begin{description}
                \item[Postulates:] medium properties and invariants (incompressible, inviscid, circulation invariants).
                \item[Derivations:] results that follow from postulates (thermodynamic mapping, derived constants, scaling laws).
                \item[Speculations:] consistent hypotheses lacking direct experimental validation (e.g.\ log-periodic ``Golden Layer'' selection).
            \end{description}

% ------------------------------------------------------------
        \subsection{What is proven here (and what remains an assumption)}
        \label{sec:what_proved}

        For clarity we separate theorem-level statements (conditional on explicit assumptions) from SST-specific
        identifications that are currently postulated or calibrated.

        \paragraph{Proved statements (conditional).}
            Under the assumptions listed in Sec.~\ref{sec:ACderive} and the mediator EFT of Sec.~\ref{sec:eft_mediator},
            the paper establishes:

            \begin{enumerate}
                \item \textbf{(Monopole uniqueness)} If the static mediator equation outside the core is Laplacian,
                \(\nabla^2 \delta p=0\) for \(r>\rc\), and the far field is isotropic and decays at infinity, then the
                unique solution is \(\delta p(r)\propto 1/r\) (Appendix~\ref{app:greens}).

                \item \textbf{(Screened variant)} If the mediator has a quadratic mass term in the EFT, then the static
                equation is Helmholtz and the unique isotropic tail is Yukawa,
                \(\delta p(r)\propto e^{-r/\lambda}/r\) with \(\lambda=m_\phi^{-1}\) (Sec.~\ref{sec:eft_mediator},
                Appendix~\ref{app:greens}).

                \item \textbf{(Flux carrier)} Time-dependent sources in the quadratic EFT necessarily carry energy and momentum
                flux \(S_i=T_{0i}=Z(\partial_t\phi)(\partial_i\phi)\), enabling a concrete wave/echo channel (Appendix~\ref{app:stress}).

                \item \textbf{(Closed constant chain, given calibrated micro-inputs)} Given the calibrated micro-scale inputs
                \((\rhocore,\rc,\vnorm)\) and the identification \(\alpha=2\vnorm/\clight\), the chain
                \((\rhocore,\rc,\vnorm)\to A_C^{\mathrm{(SST)}}\) and \((A_C^{\mathrm{(SST)}},\alpha)\to(\hbar,h)\) reproduces the SI values numerically.
            \end{enumerate}

        \paragraph{Assumptions / calibrated identifications (not proven here).}
            The following are treated as SST identifications or calibrations to be justified elsewhere:
            (i) \(\alpha \equiv 2\vnorm/\clight\);
            (ii) the existence and numerical value of \(\rhocore\) as a microphysical stiffness density;
            (iii) the pressure-work interaction model \(V=-V_{\mathrm{eff}}\delta p\) as the correct coarse-grained
            two-body energy functional;
            (iv) the selection principle that maps \(\Gamma_0\) (and topology) to the observed particle mass spectrum.

% ============================================================
    \section{Primitive inputs and parameter logic (normalized triad)}
% ============================================================

        \subsection{The primitive triad}
            In this paper we normalize the SST primitive set to
            \begin{equation}
                \boxed{\quad \{\rhoF,\;\rc,\;\vnorm\}\quad}
                \label{eq:triad}
            \end{equation}
            where \(\rhoF\) is the effective inertial density of the medium, \(\rc\) is the filament core radius,
            and \(\vnorm\) is the characteristic tangential swirl speed.

            \paragraph{Canonical numerical values (SST).}
                We will use the canonical SST values (SI):
                \begin{equation}
                    \rhoF = 7.0\times 10^{-7}\ \mathrm{kg\,m^{-3}},\qquad
                    \rc = 1.40897017\times 10^{-15}\ \mathrm{m},\qquad
                    \vnorm = 1.09384563\times 10^{6}\ \mathrm{m\,s^{-1}}.
                    \label{eq:canonvals}
                \end{equation}

    \subsection{Derived circulation quantum (removes an input)}
        A derived circulation quantum consistent with \eqref{eq:triad} is
        \begin{equation}
            \boxed{\quad \Gamma_0 \;\equiv\; 2\pi \rc \,\vnorm\quad}
            \label{eq:Gamma0}
        \end{equation}
        with units \([\Gamma_0]=\mathrm{m^2\,s^{-1}}\).
        Using \eqref{eq:canonvals},
        \begin{equation}
            \Gamma_0 \;=\; 9.683619203\times 10^{-9}\ \mathrm{m^2\,s^{-1}}.
            \label{eq:Gamma0num}
        \end{equation}

        \paragraph{Legacy note.}
            Some earlier drafts used \(\Gamma_0\) as an independent primitive input. Equation~\eqref{eq:Gamma0}
            eliminates that redundancy by treating \(\Gamma_0\) as a derived invariant once \((\rc,\vnorm)\) are fixed.

    \subsection{A natural dimensionless constant: \(\alpha\) from the swirl speed}
    A central numerical anchor is the dimensionless ratio \(\vnorm/\clight\).
    In SST the fine-structure constant is identified as
    \begin{equation}
        \boxed{\quad \alpha \;\equiv\; \frac{2\vnorm}{\clight}\quad}
        \label{eq:alpha}
    \end{equation}
    so that \(\vnorm = (\alpha/2)\clight\). With \(\clight=299\,792\,458\ \mathrm{m/s}\),
    \begin{equation}
        \alpha = 7.297352557\times 10^{-3},\qquad \alpha^{-1}=137.035999312.
        \label{eq:alphanum}
    \end{equation}
    This will be used below to derive \(\hbar\) and \(h\) from a single coupling scale.

    \subsection{Two-density structure and microphysical stiffness scale}
    While \(\rhoF\) governs the far-field inertial response of the medium, SST also employs a microphysical
    core stiffness density \(\rhocore\) for the filament core.
    We treat \(\rhocore\) as a fixed (canonically calibrated) material property of the string core:
    \begin{equation}
        \rhocore = 3.8934358266918687\times 10^{18}\ \mathrm{kg\,m^{-3}}.
        \label{eq:rhocore}
    \end{equation}
    Operationally:
    \begin{itemize}
        \item \(\rhoF\) controls large-scale inertia/transport and thermodynamic bookkeeping of the medium.
        \item \(\rhocore\) controls near-core pressures/energetics relevant for the interaction coupling scale.
    \end{itemize}
    This separation is not an added free parameter once \(\rhocore\) is fixed by independent SST calibration; it is a
    two-scale constitutive structure.

% ============================================================
    \section{Mechanical theory of heat in SST}
% ============================================================

    \subsection{Clausius structure and the Abe--Okuyama mapping}
        A key structural input is the Clausius equality for reversible processes,
        \begin{equation}
            \oint \frac{\delta Q}{T} = 0,
            \label{eq:clausius}
        \end{equation}
        viewed here as a constraint on admissible state transformations of filament excitations.
        Following Abe and Okuyama, imposing a Clausius-consistent definition of ``heat'' on Shannon entropy yields a
        formal coincidence with von Neumann entropy of a canonical density matrix, providing an isomorphism-like bridge
        between pure-state mechanics and ``quantum thermodynamics'' \cite{AbeOkuyama2011PRE,AbeOkuyama2010arXiv}.

    \subsection{Work--heat decomposition from spectral data}
        Let \(\{E_n\}\) denote a discrete internal mode spectrum (e.g.\ Kelvin-wave eigenmodes on the filament) and
        \(\{p_n\}\) their occupation probabilities.
        Define
        \begin{equation}
            \delta W \equiv \sum_n p_n\, dE_n,\qquad
            \delta Q \equiv \sum_n E_n\, dp_n.
            \label{eq:WQsplit}
        \end{equation}
        This matches standard quantum-thermodynamic decompositions of energy change into ``level shift'' and
        ``population change'' parts \cite{AbeOkuyama2011PRE}.

    \subsection{SST interpretation}
        Within SST we interpret:
        \begin{itemize}
            \item \textbf{Work \(\delta W\):} geometric deformation energy required to change filament geometry
            (core radius \(\rc\), envelope size \(R\), curvature/torsion class) against medium pressure.
            \item \textbf{Heat \(\delta Q\):} redistribution of energy over internal filament modes (Kelvin-wave degrees of freedom),
            at fixed instantaneous spectrum.
        \end{itemize}

    \subsection{Temperature as geometric strain parameter}
        We introduce a swirl-temperature \(T_{\text{swirl}}\) as a parameter controlling the mode population distribution,
        and we posit it correlates with a geometric strain measure \(\epsilon\) (radial swelling or envelope deformation):
        \begin{equation}
            T_{\text{swirl}} \propto \epsilon.
            \label{eq:Tswirl}
        \end{equation}
        The equilibrium vacuum corresponds to \(\epsilon=0\) and \(T_{\text{swirl}}=0\).

% ============================================================
    \section{Derived constants: strategy and closed consistency chain}
% ============================================================

    \subsection{The Coulomb coupling scale as a near-core pressure invariant}

    \paragraph{Empirical SI comparison scale.}
        For comparison with standard SI electromagnetism, define the empirical Coulomb coupling scale
        \begin{equation}
            A_C^{\mathrm{(SI)}} \;\equiv\; \frac{e^2}{4\pi\epsilon_0},
            \label{eq:ACdef}
        \end{equation}
        with units \([\mathrm{J\,m}]\).
        This quantity is \emph{not} taken as an SST primitive; it is used only as a numerical benchmark.

    \paragraph{SST prediction (magnitude).}
        Section~\ref{sec:ACderive} yields the SST coupling magnitude
        \[
        A_C^{\mathrm{(SST)}} = 4\pi\,\rhocore\,\vnorm^2\,\rc^4,
        \]
        which is dimensionally \([\mathrm{J\,m}]\) and subsequently fixes \(\hbar\) and \(h\) via Eq.~\eqref{eq:hfromAC}
        once the identification step below is made.

    \paragraph{Identification step (comparison to nature).}
        The ``zero-free-parameter'' program asserts that after calibrating \((\rhocore,\rc,\vnorm)\),
        the predicted magnitude should match the empirical SI benchmark within uncertainty,
        \begin{equation}
            A_C^{\mathrm{(SST)}} \stackrel{!}{\approx} A_C^{\mathrm{(SI)}}.
            \label{eq:AC_identification}
        \end{equation}

    \paragraph{Numerical value (SST prediction).}
        With \eqref{eq:canonvals} and \eqref{eq:rhocore},
        \begin{equation}
            A_C^{\mathrm{(SST)}} \;=\; 2.307077328\times 10^{-28}\ \mathrm{J\,m}.
            \label{eq:ACnum}
        \end{equation}
        This is a parameter-free anchor once \((\rhocore,\rc,\vnorm)\) are fixed.

    \subsection{Far-field monopole from EFT (Laplace/Helmholtz) and core matching (fixing the \(4\pi\))}
    \label{sec:ACderive}

    We now derive Eq.~\eqref{eq:ACsst} from a minimal far-field construction.

    \paragraph{Assumptions (explicit).}
        (i) The medium is inviscid and incompressible; (ii) we consider a quasi-static interaction regime
        in which the relevant mediator is a scalar pressure/foliation perturbation \(\delta p(\mathbf{x})\)
        generated by a compact swirl-string core of radius \(\rc\); (iii) at distances \(r\gg \rc\) the closed loop
        coarse-grains to an isotropic monopole in \(\delta p\) (all multipoles beyond \(\ell=0\) are neglected).

    \paragraph{Mediator equation outside the core (from the quadratic EFT).}
        The quasi-static far-field is controlled by the scalar mediator \(\phi\) introduced in
        Sec.~\ref{sec:eft_mediator}, with \(\delta p \equiv g_p\,\phi\) [Eq.~\eqref{eq:dp_phi}].
        Outside the compact core (\(r>\rc\)) the source vanishes, \(J=0\), and the static limit of
        Eq.~\eqref{eq:eom_phi} gives
        \begin{equation}
            (\nabla^2 - m_\phi^2)\phi = 0 \qquad (r>\rc).
        \end{equation}
        Hence the physical pressure/foliation perturbation obeys
        \begin{equation}
            (\nabla^2 - \lambda^{-2})\,\delta p = 0 \qquad (r>\rc),
            \qquad \lambda \equiv m_\phi^{-1},
            \label{eq:dp_helmholtz_outside}
        \end{equation}
        with the unscreened Laplacian limit recovered as \(\lambda\to\infty\):
        \begin{equation}
            \nabla^2 \delta p = 0 \qquad (r>\rc,\ \lambda\to\infty).
            \label{eq:laplace_dp}
        \end{equation}
        The uniqueness of the isotropic \(1/r\) tail (and its screened Yukawa generalization) is shown explicitly in
        Appendix~\ref{app:greens}.

    \paragraph{Isotropic monopole tail (unscreened).}
        In the unscreened case, spherical symmetry implies the unique decaying solution
        \begin{equation}
            \delta p(r) = -\chi_1\,p_\star\,\frac{\rc}{r}, \qquad r\ge \rc,\qquad \chi_1\in\{+1,-1\},
            \label{eq:dp_monopole}
        \end{equation}
        where the discrete sign \(\chi_1\) labels the two polarity classes of the source (the ``charge'' analogue),
        and boundary matching fixes \(|\delta p(\rc)|=p_\star\).

    \paragraph{Core matching: fixing \(p_\star\) by a momentum-flux stress scale.}
        The near-core interaction scale is set by a normal stress (momentum flux density) rather than a purely kinetic-energy
        density. We therefore match to
        \begin{equation}
            p_\star \equiv \sigma_\star \equiv \rhocore\,\vnorm^2,
            \label{eq:pstar}
        \end{equation}
        which has units of pressure and is the canonical stress scale associated with a tangential swirl speed \(\vnorm\)
        in a stiff core medium of density \(\rhocore\).
        (Using the aerodynamic ``dynamic pressure'' \(\tfrac12\rhocore\vnorm^2\) would shift the final prefactor by \(1/2\);
        SST fixes the choice \eqref{eq:pstar} by matching to momentum-flux normalization.)

    \paragraph{Effective excluded volume from isotropic stress normalization.}
        For a spherical core the geometric volume is \(V_{\text{core}}=\tfrac{4\pi}{3}\rc^3\).
        In an isotropic 3D medium, the scalar pressure corresponds to the trace of the momentum-flux tensor,
        and the appropriate normalization implies an effective factor of \(3\) in the pressure-work channel
        (derivation in Appendix~\ref{app:isotropic3}):
        \begin{equation}
            V_{\text{eff}} \equiv 3\,V_{\text{core}} = 3\cdot \frac{4\pi}{3}\rc^3 = 4\pi \rc^3.
            \label{eq:Veff}
        \end{equation}

    \paragraph{Two-body sign and magnitude.}
        To separate the \emph{magnitude} of the coupling from the \emph{sign} (attraction/repulsion), we assign each
        swirl-string core a discrete polarity \(\chi\in\{+1,-1\}\).
        Core~1 generates \(\delta p_1(r)\) as in Eq.~\eqref{eq:dp_monopole}. The interaction energy of core~2 placed in that
        background is modeled by pressure-work on an excluded effective volume,
        \begin{equation}
            V_{12}(r) \;=\; -\,\chi_2\,V_{\text{eff}}\,\delta p_1(r),
            \label{eq:Vfromdp}
        \end{equation}
        so that like polarities (\(\chi_1\chi_2=+1\)) give repulsion and opposite polarities (\(\chi_1\chi_2=-1\)) give attraction.

        Using Eqs.~\eqref{eq:dp_monopole}, \eqref{eq:pstar}, and \eqref{eq:Veff} we obtain
        \begin{equation}
            V_{12}(r)
            =
            -\chi_2\Big(4\pi \rc^3\Big)\Big(-\chi_1\,\rhocore\vnorm^2\frac{\rc}{r}\Big)
            =
            \chi_1\chi_2\,\frac{A_C^{\mathrm{(SST)}}}{r},
            \label{eq:V_Coulomb_like}
        \end{equation}
        where the positive coupling magnitude is
        \begin{equation}
            \boxed{\quad A_C^{\mathrm{(SST)}} = 4\pi\,\rhocore\,\vnorm^2\,\rc^4.\quad}
            \label{eq:ACsst}
        \end{equation}

    \paragraph{Screened (Yukawa) variant.}
        If \(m_\phi>0\) (finite \(\lambda\)), the isotropic tail is Yukawa (Appendix~\ref{app:greens}), and the two-body
        potential becomes
        \begin{equation}
            V_{12}(r)=\chi_1\chi_2\,\frac{A_C^{\mathrm{(SST)}}}{r}\,e^{-r/\lambda}.
        \end{equation}

    \paragraph{Origin of the \(4\pi\) (summary).}
        The overall coefficient in Eq.~\eqref{eq:ACsst} is fixed without continuous tuning by two standard normalizations:
        (i) \(\delta p\) has an isotropic monopole tail determined by the Laplace/Helmholtz Green's function and boundary matching
        (Appendix~\ref{app:greens}); (ii) the pressure-work channel couples to the trace part of the isotropic stress,
        which yields \(V_{\mathrm{eff}}=3V_{\mathrm{core}}=4\pi \rc^3\) (Appendix~\ref{app:isotropic3}).
        Together with the core stress scale \(p_\star=\rhocore\vnorm^2\), this fixes \(A_C\) uniquely.

        \paragraph{Sign and ``charge'' assignment.}
        Equation~\eqref{eq:ACsst} fixes the positive coupling magnitude \(A_C^{\mathrm{(SST)}}\).
        The interaction sign is encoded by the discrete polarity labels \(\chi_1,\chi_2\in\{+1,-1\}\) via
        Eq.~\eqref{eq:V_Coulomb_like}. Mapping \(\chi\) to a microscopic SST chirality/polarization invariant is deferred to the
        topological selection sector.

    \subsection{Planck's constant as a derived quantity (no postulate)}
    In SI, the fine-structure constant satisfies
    \begin{equation}
        \alpha \;=\; \frac{A_C^{\mathrm{(SI)}}}{\hbar \clight}.
        \label{eq:alphaStandard}
    \end{equation}
    Using \eqref{eq:alpha} and identifying \(A_C^{\mathrm{(SST)}}\approx A_C^{\mathrm{(SI)}}\) as in
    Eq.~\eqref{eq:AC_identification}, we obtain
    \begin{equation}
        \boxed{\quad \hbar \;=\; \frac{A_C^{\mathrm{(SST)}}}{\alpha \clight}\quad},\qquad
        \boxed{\quad h \;=\; 2\pi\hbar \;=\; \frac{2\pi A_C^{\mathrm{(SST)}}}{\alpha \clight}\quad}.
        \label{eq:hfromAC}
    \end{equation}
    This is the precise sense in which SST ``bypasses Planck'': \(h\) becomes a derived constant fixed by
    (i) a topological/kinematic dimensionless ratio \(\alpha=2\vnorm/\clight\) and (ii) a mechanically predicted coupling magnitude \(A_C^{\mathrm{(SST)}}\).

    \paragraph{Numerical value.}
        Using \eqref{eq:ACnum} and \eqref{eq:alphanum},
        \begin{equation}
            \hbar \;=\; 1.0545717167\times 10^{-34}\ \mathrm{J\,s},\qquad
            h \;=\; 6.6260695157\times 10^{-34}\ \mathrm{J\,s}.
            \label{eq:hnum}
        \end{equation}

    \subsection{Action from circulation: the structural identity}
    A robust identity in superfluid circulation quantization is
    \begin{equation}
        \kappa = \frac{h}{m},
        \label{eq:superfluid}
    \end{equation}
    where \(\kappa\) is a circulation quantum \cite{Onsager1949,Feynman1955}.
    SST retains the structural form
    \begin{equation}
        h \sim m_{\text{eff}}\,\Gamma_0,
        \label{eq:hmgamma}
    \end{equation}
    but \(\Gamma_0\) is now fixed by \eqref{eq:Gamma0} and \(h\) by \eqref{eq:hfromAC}. Hence \eqref{eq:hmgamma}
    becomes an \emph{equation that fixes} \(m_{\text{eff}}\) for the relevant excitation class:
    \begin{equation}
        m_{\text{eff}} \;\sim\; \frac{h}{\Gamma_0}.
    \end{equation}
    The ``selection problem'' is then: which filament configuration realizes that \(m_{\text{eff}}\) as a stable extremum?

    \subsection{Quadratic EFT for the pressure/foliation mediator (deriving Helmholtz/Yukawa)}
    \label{sec:eft_mediator}

    To make the screened alternative of Appendix~\ref{app:greens} structurally unavoidable (rather than optional),
    we introduce a minimal quadratic effective field theory (EFT) for a scalar mediator \(\phi\) that encodes the
    relevant pressure/foliation perturbation channel.

    \paragraph{Field content and calibration.}
        We treat \(\phi\) as a coarse-grained scalar degree of freedom sourced by a compact swirl-string core.
        The physical pressure perturbation is taken to be proportional to the field,
        \begin{equation}
            \delta p \equiv g_p\,\phi,
            \label{eq:dp_phi}
        \end{equation}
        where the proportionality constant \(g_p\) is fixed by one calibration condition (e.g.\ matching the boundary
        value \(\delta p(\rc)=-p_\star\) used in Sec.~\ref{sec:ACderive}).

    \paragraph{Minimal quadratic action.}
        In a flat background and at leading order in derivatives, the most general stable quadratic Lagrangian density is
        \begin{equation}
            \mathcal{L}_\phi
            =
            \frac{Z}{2}\left(
            \frac{1}{c_\phi^{2}}(\partial_t \phi)^2
            -
            (\nabla \phi)^2
            -
            m_\phi^{2}\phi^2
            \right)
            +\phi\,J,
            \label{eq:Lphi}
        \end{equation}
        with wave speed \(c_\phi>0\), normalization \(Z>0\), mass parameter \(m_\phi\ge 0\), and a source \(J\)
        representing the compact core.

    \paragraph{Equation of motion and the static limit.}
        Varying \eqref{eq:Lphi} gives
        \begin{equation}
            \frac{1}{c_\phi^{2}}\partial_t^2\phi - \nabla^2\phi + m_\phi^2\phi = \frac{J}{Z}.
            \label{eq:eom_phi}
        \end{equation}
        In the quasi-static regime relevant for the Coulomb-like interaction,
        \(\partial_t\phi \approx 0\), so \eqref{eq:eom_phi} reduces to the screened Poisson (Helmholtz) equation
        \begin{equation}
            (\nabla^2 - m_\phi^2)\phi = -\frac{J}{Z}.
            \label{eq:helmholtz_phi}
        \end{equation}
        For a pointlike monopole source \(J=q_\phi\,\delta^{(3)}(\mathbf{x})\),
        the Green's function solution is
        \begin{equation}
            \phi(r)=\frac{q_\phi}{4\pi Z}\,\frac{e^{-m_\phi r}}{r}.
            \label{eq:phi_yukawa}
        \end{equation}
        Using \(\delta p=g_p\phi\) in \eqref{eq:dp_phi} gives precisely the Yukawa form
        \(\delta p(r)\propto e^{-r/\lambda}/r\) of Appendix~\ref{app:greens}, with identification
        \begin{equation}
            \boxed{\quad \lambda = m_\phi^{-1}\quad}
            \label{eq:lambda_m}
        \end{equation}
        and effective monopole strength \(Q_p=(g_p q_\phi/Z)\).

    \paragraph{Unscreened limit.}
        When \(m_\phi\to 0\) (equivalently \(\lambda\to\infty\)), \eqref{eq:phi_yukawa} reduces to \(\phi\propto 1/r\),
        recovering the Laplacian monopole channel and the \(1/r\) interaction potential.

    \paragraph{Why this matters for the ``echo'' channel.}
        Equation~\eqref{eq:eom_phi} is hyperbolic for time-dependent sources. Hence acceleration pulses can launch
        propagating \(\phi\)-waves that carry energy and momentum to boundaries. This supplies a concrete dynamical
        carrier for the ``Unruh-echo'' phenomenology, independent of whether \(m_\phi=0\) or \(m_\phi>0\).

    \paragraph{10-year-old analogy (brief).}
        Imagine a stretchy drum skin (\(\phi\)). If you press it and hold it (static), the dent spreads like \(1/r\).
        If the skin is heavier or stiffer (a ``mass term''), the dent fades away faster with distance (the exponential).
        If you tap it (time-dependent), waves run outward and carry the tap's energy away.

% ============================================================
    \section{Consistency anchors (numerical checks and SI bookkeeping)}
% ============================================================

    \subsection{Kinematic and pressure scales}
        Two useful derived scales from the triad are:
        \begin{equation}
            \omega_\star \equiv \frac{\vnorm}{\rc},\qquad
            q_f \equiv \frac{1}{2}\rhoF\vnorm^2,\qquad
            q_{\text{core}} \equiv \frac{1}{2}\rhocore\vnorm^2.
            \label{eq:scales}
        \end{equation}
        Numerically,
        \begin{equation}
            \omega_\star = 7.763440655\times 10^{20}\ \mathrm{s^{-1}},\qquad
            q_f = 4.187743918\times 10^{5}\ \mathrm{Pa},\qquad
            q_{\text{core}} = 2.329244600\times 10^{30}\ \mathrm{Pa}.
        \end{equation}

    \subsection{Closed chain summary}
        The central ``parameter-free'' chain, in its leanest form, is:
        \begin{equation}
        (\rc,\vnorm)\Rightarrow \Gamma_0,\ \alpha;\qquad
        (\rhocore,\rc,\vnorm)\Rightarrow A_C^{\mathrm{(SST)}};\qquad
        (A_C^{\mathrm{(SST)}},\alpha)\Rightarrow \hbar,\ h.
        \end{equation}

    \subsection{Anchor table}
        Table~\ref{tab:anchors} collects the above anchors. Where available we also list reference values for comparison.

        \begin{table}[h]
            \centering
            \caption{SST numerical anchors derived from the normalized primitive set.}
            \label{tab:anchors}
            \begin{tabular}{@{}llll@{}}
                \toprule
                Quantity & Formula & Value (SST) & Units \\ \midrule
                \(\rhoF\) & input & \(7.0\times 10^{-7}\) & \(\mathrm{kg\,m^{-3}}\) \\
                \(\rc\) & input & \(1.40897017\times 10^{-15}\) & \(\mathrm{m}\) \\
                \(\vnorm\) & input & \(1.09384563\times 10^{6}\) & \(\mathrm{m\,s^{-1}}\) \\
                \(\Gamma_0\) & \(2\pi\rc\vnorm\) & \(9.683619203\times 10^{-9}\) & \(\mathrm{m^2\,s^{-1}}\) \\
                \(\alpha\) & \(2\vnorm/\clight\) & \(7.297352557\times 10^{-3}\) & (dimensionless) \\
                \(\alpha^{-1}\) & \(1/\alpha\) & \(137.035999312\) & (dimensionless) \\
                \(\rhocore\) & input (micro-scale) & \(3.8934358267\times 10^{18}\) & \(\mathrm{kg\,m^{-3}}\) \\
                \(A_C^{\mathrm{(SST)}}\) & \(4\pi\rhocore\vnorm^2\rc^4\) & \(2.307077328\times 10^{-28}\) & \(\mathrm{J\,m}\) \\
                \(A_C^{\mathrm{(SI)}}\) & \(e^2/(4\pi\epsilon_0)\) & (benchmark) & \(\mathrm{J\,m}\) \\
                \(\hbar\) & \(A_C^{\mathrm{(SST)}}/(\alpha\clight)\) & \(1.0545717167\times 10^{-34}\) & \(\mathrm{J\,s}\) \\
                \(h\) & \(2\pi\hbar\) & \(6.6260695157\times 10^{-34}\) & \(\mathrm{J\,s}\) \\
                \(\omega_\star\) & \(\vnorm/\rc\) & \(7.763440655\times 10^{20}\) & \(\mathrm{s^{-1}}\) \\
                \(q_f\) & \(\tfrac{1}{2}\rhoF\vnorm^2\) & \(4.187743918\times 10^{5}\) & \(\mathrm{Pa}\) \\
                \(q_{\text{core}}\) & \(\tfrac{1}{2}\rhocore\vnorm^2\) & \(2.329244600\times 10^{30}\) & \(\mathrm{Pa}\) \\
                \bottomrule
            \end{tabular}
        \end{table}

    \subsection{Hydrogen ground state energy (external comparison; not an SST input)}
        For later cross-checking, standard hydrogen binding energy can be written as
        \begin{equation}
            E_B = \frac{1}{2}m_e \alpha^2 \clight^2.
        \end{equation}
        Using \(\alpha=2\vnorm/\clight\) from \eqref{eq:alpha} gives the equivalent form
        \begin{equation}
            E_B = 2 m_e \vnorm^2.
            \label{eq:EBv}
        \end{equation}
        With the standard electron mass \(m_e\) (used here \emph{only} for comparison, not as an SST primitive),
        \eqref{eq:EBv} evaluates to \(E_B \approx 13.6057\,\mathrm{eV}\), consistent with the known value.
        In SST, the objective is to derive \(m_e\) (or its effective analog) from topological/thermodynamic selection,
        at which point \eqref{eq:EBv} becomes a genuine parameter-free prediction.

        \paragraph{10-year-old analogy (brief).}
            Think of a rubber ring in water: you can stretch the ring (work) or make it wiggle (heat). The ``quantum numbers''
            are like special wiggle patterns the ring is allowed to have, because the ring is closed and cannot wiggle any random way.

% ============================================================
    \section{Predictions and experimental tests}
% ============================================================

    \subsection{Low-temperature heat capacity scaling}
        If internal excitations are effectively one-dimensional gapless modes on a filament,
        then the low-\(T\) specific heat generically scales linearly,
        \begin{equation}
            C_V^{\text{SST}} \propto T_{\text{swirl}},
            \label{eq:CVlinear}
        \end{equation}
        in contrast with exponentially suppressed heat capacity in gapped spectra.
        A decisive falsifier is the observation of exponential suppression in regimes where SST predicts a mode continuum.

    \subsection{A direct falsifier of the monopole pressure-mediator hypothesis}
        The derivation of Eq.~\eqref{eq:ACsst} assumes that the relevant far-field mediator is an isotropic monopole in
        \(\delta p\), implying \(\delta p(r)\propto r^{-1}\) and \(V(r)\propto r^{-1}\).
        A direct falsifier is any robust observation (in an SST-analog system or a proposed proxy experiment) that the
        dominant far-field scaling is instead \(r^{-2}\) (or faster), which would indicate that the leading monopole channel
        is absent or screened and that the coupling must arise from higher multipoles or a different mediator sector.

    \subsection{Acceleration-induced echo channel (Unruh-analog)}
        A uniformly accelerating detector in relativistic QFT experiences thermal response at temperature
        \begin{equation}
            T_U = \frac{\hbar a}{2\pi c k_B},
        \end{equation}
        the Unruh effect \cite{Unruh1976}.
        SST posits that an accelerating filament couples to medium degrees of freedom in a way that can produce a delayed boundary
        transduction ``echo'' under suitable impedance mismatch between characteristic speeds.
        This is presented as a phenomenological prediction; a full calculation requires the quadratic EFT of the mediating SST field
        and its stress tensor (planned follow-up).

% ============================================================
    \section{Speculative extension: Golden-layer selection}
    The ``Golden Layer'' hypothesis introduces a log-periodic weighting
    \[
        w(K)\propto \varphi^{-g(K)},
    \]
    with \(\varphi\) the golden ratio and \(g(K)\) a topological index, acting as a thermodynamic filter selecting
    stable mass islands. This is explicitly classified as speculation until a concrete mapping \(g(K)\) and an experimental discriminator are supplied.

% ============================================================
    \section{Discussion and paper map}
    This step-2 version makes the ``bypass Planck'' claim concrete:
    \(h\) is not assumed but obtained from the SST-normalized primitive set via the chain
    \((\rhocore,\rc,\vnorm)\to A_C^{\mathrm{(SST)}}\) and \((\vnorm/\clight)\to \alpha\), then \((A_C^{\mathrm{(SST)}},\alpha)\to \hbar,h\).
    Future work must (i) derive \(\rhocore\) and (ii) supply the selection principle for \(m_{\text{eff}}\)
    that ties circulation \(\Gamma_0\) to particle masses.

% ============================================================
    \appendix
    \section{Isotropic stress normalization and the factor \(3\) in \(V_{\text{eff}}\)}
    \label{app:isotropic3}

    This appendix justifies the factor \(3\) used in Eq.~\eqref{eq:Veff} using only standard continuum-mechanics
    identities (no group-theory assumptions).

    \subsection{Pressure as the trace of the momentum-flux tensor}
        In an inviscid medium the Cauchy stress tensor is
        \begin{equation}
            \sigma_{ij} = -p\,\delta_{ij},
        \end{equation}
        so the scalar pressure is related to the trace by
        \begin{equation}
            p = -\frac{1}{3}\,\mathrm{tr}(\sigma) = -\frac{1}{3}\,\sigma_{ii}.
            \label{eq:p_trace}
        \end{equation}
        Equivalently, a pressure perturbation \(\delta p\) corresponds to an isotropic perturbation in the normal stresses
        \(\delta\sigma_{ii}=-3\,\delta p\).

    \subsection{Work of inserting an excluded core: why the trace contributes}
        Consider inserting a small spherical excluded region (the core) into a background stress field.
        The incremental mechanical work can be written as a contraction of stress with an isotropic strain/volume change.
        For a purely isotropic insertion channel, the relevant scalar work is proportional to the trace part of the stress,
        hence to \(\delta\sigma_{ii}\), not to a single component \(\delta\sigma_{rr}\) alone.

        A convenient way to package this into a scalar pressure-work form is to write the interaction energy shift as
        \begin{equation}
            \Delta E \equiv -\,\delta p\,V_{\text{eff}},
        \end{equation}
        and then absorb the conversion between \(\delta p\) and \(\delta\sigma_{ii}\) into \(V_{\text{eff}}\).
        Using Eq.~\eqref{eq:p_trace}, \(\delta\sigma_{ii}=-3\,\delta p\), so the isotropic channel effectively counts three
        equal normal-stress contributions (one per spatial axis) in the trace.

    \subsection{Result: \(V_{\text{eff}}=3V_{\text{core}}\)}
        If \(V_{\text{core}}\) is the geometric excluded volume, then the trace-normalized scalar work channel corresponds to
        \begin{equation}
            V_{\text{eff}} = 3\,V_{\text{core}},
        \end{equation}
        which yields Eq.~\eqref{eq:Veff}:
        \(
        V_{\text{eff}} = 3\cdot\frac{4\pi}{3}\rc^3 = 4\pi \rc^3.
        \)

    \subsection{Scope and limitation}
        This argument fixes only the \emph{normalization} of the isotropic pressure-work channel under the assumptions of
        (i) inviscid isotropic stress, (ii) monopole far-field \(\delta p\propto 1/r\), and (iii) a compact spherical core.
        If the core is strongly anisotropic, or if higher multipoles dominate, the factorization into \(\delta p\) times a
        single \(V_{\text{eff}}\) must be revisited.

    \section{Monopole solution for \(\delta p\): Green's function derivation and screened variant}
    \label{app:greens}

    \subsection{Laplace equation and uniqueness of the \(1/r\) far field}
        Outside a compact source region (the core), the quasi-static pressure/foliation perturbation satisfies
        \begin{equation}
            \nabla^2 \delta p = 0 \qquad (r>\rc).
        \end{equation}
        The general spherically symmetric harmonic function in 3D is
        \begin{equation}
            \delta p(r)=a+\frac{b}{r}.
        \end{equation}
        Imposing \(\delta p(\infty)=0\) gives \(a=0\), hence \(\delta p(r)=b/r\).
        Matching the boundary magnitude \(|\delta p(\rc)|=p_\star\) fixes the amplitude up to a discrete polarity
        \(\chi_1\in\{+1,-1\}\), yielding
        \begin{equation}
            \delta p(r) = -\chi_1\,p_\star\,\frac{\rc}{r},\qquad r\ge \rc,\qquad \chi_1\in\{+1,-1\},
        \end{equation}
        which is Eq.~\eqref{eq:dp_monopole}. This establishes the uniqueness of the \(1/r\) tail given (i) Laplace outside,
        (ii) spherical symmetry, and (iii) decay at infinity.

    \subsection{Equivalent source formulation with a monopole strength}
        The same result can be written using the Poisson equation with a pointlike monopole source:
        \begin{equation}
            \nabla^2 \delta p(\mathbf{x}) = -Q_p\,\delta^{(3)}(\mathbf{x}),
            \label{eq:poisson_source}
        \end{equation}
        whose solution is
        \begin{equation}
            \delta p(r) = \frac{Q_p}{4\pi r}.
            \label{eq:greens_sol}
        \end{equation}
        Comparing \eqref{eq:greens_sol} with \(\delta p(r)=-\chi_1 p_\star \rc/r\) gives the effective monopole strength
        \begin{equation}
            Q_p = -4\pi\,\chi_1\,p_\star \rc.
            \label{eq:Qp}
        \end{equation}
        In this form, the \(4\pi\) appears directly from the Green's function normalization
        \(\nabla^2(1/r)=-4\pi \delta^{(3)}(\mathbf{x})\) \cite{Jackson1999}.

    \subsection{Screened mediator: Yukawa tail and a falsifiable length scale}
        If the mediator sector has a finite correlation length \(\lambda\) (e.g.\ from a mass term in an effective field theory),
        then the quasi-static equation becomes the screened Poisson (Helmholtz) equation
        \begin{equation}
            (\nabla^2-\lambda^{-2})\,\delta p(\mathbf{x}) = -Q_p\,\delta^{(3)}(\mathbf{x}).
            \label{eq:helmholtz}
        \end{equation}
        The corresponding Green's function yields the Yukawa form
        \begin{equation}
            \delta p(r)=\frac{Q_p}{4\pi r}\,e^{-r/\lambda}.
            \label{eq:yukawa_dp}
        \end{equation}
        Substituting into the two-body pressure-work model of Sec.~\ref{sec:ACderive} gives the Yukawa-screened interaction
        \begin{equation}
            V_{12}(r)= \chi_1\chi_2\,\frac{A_C^{\mathrm{(SST)}}}{r}\,e^{-r/\lambda},
            \label{eq:yukawa_V}
        \end{equation}
        with the same short-distance coupling magnitude \(A_C^{\mathrm{(SST)}}>0\) but exponential suppression beyond \(\lambda\).

        \paragraph{Falsifier (operational).}
            The unscreened monopole channel predicts
            \(\lim_{r\to\infty} r\,V_{12}(r)=\chi_1\chi_2\,A_C^{\mathrm{(SST)}}\),
            whereas the screened model predicts
            \(\lim_{r\to\infty} r\,e^{+r/\lambda}V_{12}(r)=\chi_1\chi_2\,A_C^{\mathrm{(SST)}}\).
            Equivalently, the magnitude obeys \(\lim_{r\to\infty} |rV_{12}(r)|=A_C^{\mathrm{(SST)}}\) (unscreened) and
            \(\lim_{r\to\infty} |r e^{+r/\lambda}V_{12}(r)|=A_C^{\mathrm{(SST)}}\) (screened).
            Any robust evidence for exponential suppression of the effective coupling at large
            separation would falsify the strictly Laplacian mediator assumption and instead indicate a finite \(\lambda\).

    \section{Stress tensor of the mediator field and far-field momentum flux}
    \label{app:stress}

    This appendix records the canonical stress-energy tensor for the scalar mediator \(\phi\) and identifies the
    energy/momentum flux channel relevant for pulse/echo phenomenology.

    \subsection{Stress-energy tensor}
        From the quadratic Lagrangian \eqref{eq:Lphi}, the symmetric stress-energy tensor (in flat space) can be written as
        \begin{equation}
            T_{\mu\nu}
            =
            Z\,\partial_\mu \phi\,\partial_\nu \phi
            -
            \eta_{\mu\nu}\,\mathcal{L}_\phi^{(0)},
            \label{eq:Tmunu}
        \end{equation}
        where \(\mathcal{L}_\phi^{(0)}\) denotes \(\mathcal{L}_\phi\) without the source term \(\phi J\), and \(\eta_{\mu\nu}\)
        is the Minkowski metric.

    \subsection{Energy density and flux}
        Writing explicitly in \(3+1\) form, the energy density \(u\equiv T_{00}\) and the energy flux (Poynting-like) vector
        \(S_i\equiv T_{0i}\) are
        \begin{align}
            u
            &=
            \frac{Z}{2}\left(
            \frac{1}{c_\phi^2}(\partial_t\phi)^2
            +
            (\nabla\phi)^2
            +
            m_\phi^2\phi^2
            \right),
            \label{eq:u_phi}
            \\
            S_i
            &=
            T_{0i}
            =
            Z\,(\partial_t\phi)(\partial_i\phi).
            \label{eq:flux_phi}
        \end{align}
        Thus, \emph{static} configurations \(\partial_t\phi=0\) carry no flux:
        \begin{equation}
            \partial_t\phi=0 \quad\Rightarrow\quad S_i=0.
        \end{equation}
        Time-dependent excitations (pulses) have \(S_i\neq 0\) and transport energy and momentum outward.

    \subsection{Implication for echo phenomenology}
        If a localized acceleration episode produces a transient source \(J(t,\mathbf{x})\), Eq.~\eqref{eq:eom_phi} generates
        outgoing \(\phi\)-waves whose far-field flux is given by \eqref{eq:flux_phi}. Reflection/transduction at boundaries
        can then yield delayed responses (``echoes'') without modifying the static \(1/r\) coupling in the \(m_\phi\to 0\) limit.

    \backmatter

    \section*{Data availability}
    No new experimental data are reported.

% -------------------------
% Bibliography (thebibliography / \bibitem)
% -------------------------
    \begin{thebibliography}{99}

        \bibitem{Clausius1879}
        R.~Clausius (1879).
        \textit{The Mechanical Theory of Heat: with its Applications to the Steam-Engine and to Physical Properties of Bodies}.
        Macmillan, London.

        \bibitem{Clausius1865}
        R.~Clausius (1865).
        ``Concerning Several Conveniently Applicable Forms for the Main Equations of the Mechanical Theory of Heat''.
        \textit{Annalen der Physik und der Chemie} \textbf{125}, 353--400.

        \bibitem{AbeOkuyama2010arXiv}
        S.~Abe and S.~Okuyama (2010).
        ``Similarity between quantum mechanics and thermodynamics: Entropy, temperature, and Carnot cycle''.
        \textit{arXiv:1012.5581}.
        Permalink: \texttt{https://arxiv.org/abs/1012.5581}.

        \bibitem{AbeOkuyama2011PRE}
        S.~Abe and S.~Okuyama (2011).
        ``Similarity between quantum mechanics and thermodynamics: Entropy, temperature, and Carnot cycle''.
        \textit{Physical Review E} \textbf{83}, 021121.
        DOI: 10.1103/PhysRevE.83.021121.

        \bibitem{Madelung1926}
        E.~Madelung (1926).
        ``Quantentheorie in hydrodynamischer Form''.
        \textit{Zeitschrift f\"ur Physik} \textbf{40}, 322--326.

        \bibitem{Bohm1952a}
        D.~Bohm (1952).
        ``A Suggested Interpretation of the Quantum Theory in Terms of `Hidden' Variables. I''.
        \textit{Physical Review} \textbf{85}, 166--179.
        DOI: 10.1103/PhysRev.85.166.

        \bibitem{Bohm1952b}
        D.~Bohm (1952).
        ``A Suggested Interpretation of the Quantum Theory in Terms of `Hidden' Variables. II''.
        \textit{Physical Review} \textbf{85}, 180--193.
        DOI: 10.1103/PhysRev.85.180.

        \bibitem{CouderFort2006}
        Y.~Couder, E.~Fort (2006).
        ``Single-Particle Diffraction and Interference at a Macroscopic Scale''.
        \textit{Physical Review Letters} \textbf{97}, 154101.
        DOI: 10.1103/PhysRevLett.97.154101.

        \bibitem{Onsager1949}
        L.~Onsager (1949).
        ``Statistical Hydrodynamics''.
        \textit{Nuovo Cimento} (Supplement) \textbf{6}, 279--287.

        \bibitem{Feynman1955}
        R.~P.~Feynman (1955).
        ``Application of Quantum Mechanics to Liquid Helium''.
        In: C.~J.~Gorter (ed.), \textit{Progress in Low Temperature Physics}, Vol.\ 1.
        North-Holland, Amsterdam.

        \bibitem{BIPMSI2019}
        BIPM (2019).
        \textit{The International System of Units (SI Brochure), 9th edition}.
        Bureau International des Poids et Mesures.
        Permalink: \texttt{https://www.bipm.org/en/publications/si-brochure}.

        \bibitem{Unruh1976}
        W.~G.~Unruh (1976).
        ``Notes on black-hole evaporation''.
        \textit{Physical Review D} \textbf{14}, 870--892.
        DOI: 10.1103/PhysRevD.14.870.

        \bibitem{LandauLifshitzFluid1987}
        L.~D.~Landau and E.~M.~Lifshitz (1987).
        \textit{Fluid Mechanics}, 2nd ed. (Course of Theoretical Physics, Vol.~6).
        Pergamon Press.
        (See sections on stress tensor in inviscid fluids and pressure as isotropic normal stress.)

        \bibitem{Jackson1999}
        J.~D.~Jackson (1999).
        \textit{Classical Electrodynamics}, 3rd ed.
        Wiley.
        (See the Green's function identity \(\nabla^2(1/r)=-4\pi\delta^{(3)}(\mathbf{x})\) and Yukawa/Helmholtz Green's functions.)

        \bibitem{PeskinSchroeder1995}
        M.~E.~Peskin and D.~V.~Schroeder (1995).
        \textit{An Introduction to Quantum Field Theory}.
        Westview Press.
        (For quadratic scalar field actions, Green's functions, and Yukawa/Helmholtz structure.)

        \bibitem{GoldbergerRothstein2006}
        W.~D.~Goldberger and I.~Z.~Rothstein (2006).
        ``An Effective Field Theory of Gravity for Extended Objects''.
        \textit{Physical Review D} \textbf{73}, 104029.
        DOI: 10.1103/PhysRevD.73.104029.
        (General EFT methodology; useful precedent for organizing long-distance fields sourced by compact objects.)

    \end{thebibliography}

\end{document}