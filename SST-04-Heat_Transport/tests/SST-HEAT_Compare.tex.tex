%! Author = Omar Iskandarani
%! Date = 12/3/2025
%! Affiliation = Independent Researcher, Groningen, The Netherlands
%! License = © 2025 Omar Iskandarani. All rights reserved. This manuscript is made available for academic reading and citation only. No republication, redistribution, or derivative works are permitted without explicit written permission from the author. Contact: info@omariskandarani.com
%! ORCID = 0009-0006-1686-3961
%! DOI = 10.5281/zenodo.xxx

\newcommand{\paperdoi}{10.5281/zenodo.xxx}
\newcommand{\papertitle}{Comparative Thermodynamics of Topological Defects and Quantum Ensembles:\\
Structural Divergence between Swirl--String Theory and the Abe--Okuyama Framework}
%=========================================
% PREAMBLE, PACKAGES AND DOCUMENT CONFIGURATION
%=========================================
\documentclass[11pt]{article}
\usepackage{amsmath,amssymb,amsfonts,bm}
\usepackage{siunitx}
\usepackage[hidelinks]{hyperref}
\usepackage[a4paper,margin=1in]{geometry}
\usepackage[T1]{fontenc}
\usepackage[utf8]{inputenc}
\usepackage{tikz}

% swirl arrows (context-aware)
\newcommand{\swirlarrow}{\mkern-2mu\scriptscriptstyle\boldsymbol{\circlearrowleft}}
\newcommand{\vswirl}{\mathbf{v}_{\swirlarrow}}
\newcommand{\SwirlClock}{S_{(t)}^{\swirlarrow}}
\newcommand{\Fmaxswirl}{F^{\max}_{\mkern-1mu\scriptscriptstyle\boldsymbol{\circlearrowleft}}}
% swirl arrows Counter Clockwise
\newcommand{\swirlarrowcw}{ \mkern-2mu\scriptscriptstyle\boldsymbol{\circlearrowright}}
\newcommand{\vswirlcw}{\mathbf{v}_{\swirlarrowcw}}
\newcommand{\SwirlClockcw}{S_{(t)}^{\swirlarrowcw}}
\newcommand{\Fmaxswirlcw}{F^{\max}_{\mkern-1mu\scriptscriptstyle\boldsymbol{\circlearrowright}}}

\newcommand{\Fmax}{\Fmaxswirl} % default maximal force (left swirl)
\newcommand{\FmaxEM}{F^{\max}_{\mathrm{EM}}}
\newcommand{\FmaxG}{F_{\mathrm{G}}^{\max}}               % G-like maximal force scale

\newcommand{\omegas}{\boldsymbol{\omega}_{\swirlarrow}}  % swirl vorticity
\newcommand{\Om}{\Omega_{\swirlarrow}}                   % swirl angular frequency profile

\newcommand{\vscore}{v_{\swirlarrow}}                    % shorthand: |v_swirl| at r=r_c
\newcommand{\vnorm}{\lVert \mathbf{v}_{\swirlarrow} \rVert}               % swirl speed magnitude
\newcommand{\Ce}{\vswirl}                                % canonical swirl-speed constant

\newcommand{\rhof}{\rho_{\!f}}                           % effective fluid density
\newcommand{\rhoE}{\rho_{\!E}}                           % swirl energy density
\newcommand{\rhom}{\rho_{\!m}}                           % mass-equivalent density
\newcommand{\rc}{r_c}                                    % string core radius (swirl string radius)

\newcommand{\rhoM}{\rho_{\!m}}          % mass-equivalent density
\newcommand{\rhocore}{\rho_{\text{core}}}


\newcommand{\Lam}{\Lambda}                               % Swirl Coulomb constant
\newcommand{\alpg}{\alpha_g}                             % gravitational fine-structure analogue
\newcommand{\kB}{k_{\mathrm{B}}}
\newcommand{\titlepageOpen}{
    \begin{titlepage}
    \thispagestyle{empty}
    \centering
    \Large \bfseries \papertitle \par \vspace{1cm}
    {\Large \itshape \textbf{Omar Iskandarani}\textsuperscript{\textbf{*}} \par}
    \vspace{0.5cm}
    {\today \par}
    \vspace{0.5cm}
}

\newcommand{\titlepageClose}{
    \vfill \raggedright \null
    \begin{picture}(0,0)
    \put(0,-45){  % Shift 200pt left, 40pt down
        \begin{minipage}[b]{0.7\textwidth} \footnotesize
        \renewcommand{\arraystretch}{1.0}
        \noindent\rule{\textwidth}{0.4pt} \\[0.5em]
        \textsuperscript{\textbf{*}} Independent Researcher, Groningen, The Netherlands \\
        Email: \texttt{info@omariskandarani.com} \\
        ORCID: \texttt{\href{https://orcid.org/0009-0006-1686-3961}{0009-0006-1686-3961}} \\
        DOI: \href{https://doi.org/\paperdoi}{\paperdoi}
        \end{minipage}
    }
    \end{picture}
    \end{titlepage}
}
%=========================================
% Start Document - Title Page
%=========================================
\begin{document}
    \titlepageOpen
    \begin{abstract}
        This article presents a comparative thermodynamic analysis of two conceptually distinct
        frameworks for microscopic physics: the information-theoretic thermodynamics of Abe
        and Okuyama (AO), and the hydrodynamic thermodynamics of Swirl--String Theory (SST).
        AO operates entirely within standard quantum mechanics, defining temperature and
        entropy from the Shannon information of pure-state expansion coefficients in a fixed
        Hilbert space. SST, by contrast, posits a physical, incompressible swirl condensate
        and identifies particles with knotted vortex filaments; thermodynamic variables are
        reinterpreted as geometric and topological properties of these defects. We formulate
        and compare the definitions of temperature, heat, work, and entropy, and derive the
        respective equations of state. SST predicts a quadratic energy law
        $E \propto T_{\text{swirl}}^{2}$ and a linear low-temperature heat capacity
        $C_V \propto T_{\text{swirl}}$, in contrast to the gapped/exponential or
        equipartition-type behavior of the AO particle-in-a-box model. We further discuss
        the Golden Layer hypothesis in SST, which endows the vacuum enthalpy with discrete
        scale invariance, and contrast it with the smooth Shannon entropy structure in AO.
        The comparison highlights an ontological schism: AO treats thermodynamics as an
        emergent information calculus over quantum amplitudes, whereas SST treats it as the
        elastic mechanics of a continuous fluid substrate. We identify falsifiable
        differences in low-temperature heat capacity, stability mechanisms, and possible
        temperature dependence of decay rates, and outline experimental and analogue-gravity
        tests that could discriminate between the two pictures.
    \end{abstract}

    \titlepageClose
%=========================================
% Title Page End
%=========================================

        \section{Introduction: The Ontological Crisis of the Quantum Vacuum}

            Thermodynamics and quantum mechanics use superficially similar mathematical
            structures, yet differ profoundly in ontology. In classical statistical mechanics,
            entropy and temperature arise from coarse-graining over a continuum of microstates in
            phase space. In quantum mechanics, the state is a vector in a Hilbert space, and
            pure states have vanishing von Neumann entropy. The work of Abe and Okuyama
            demonstrates that, by using the Shannon entropy of expansion coefficients in an
            energy eigenbasis and imposing the Clausius relation, one can reconstruct the
            canonical ensemble and even a quantum Carnot cycle within this framework
            \cite{AbeOkuyama2011}. Thermodynamics is thereby understood as an emergent
            information-theoretic structure on Hilbert space.

            Swirl--String Theory (SST) represents a qualitatively different response to the same
            conceptual tension. Instead of treating the vacuum as a passive geometric manifold
            or a probabilistic field, SST posits a physical, frictionless, incompressible fluid
            condensate filling three-dimensional Euclidean space. Elementary particles are
            topologically stable knotted filaments in this medium (``swirl strings'') with
            quantized circulation and well-defined geometric cores. The SST Canon derives mass,
            charge, and time dilation from the hydrodynamics of this substrate, governed by a
            primitive triplet $(\Gamma_0,\rhof,\rc)$ and a characteristic swirl velocity scale
            $\vnorm$.

            This article develops a comparative thermodynamics of these two frameworks. We ask:
            \emph{What does ``temperature'' mean in a quantum ensemble vs.\ in a knotted
            defect? How do energy, heat, and work scale with this temperature? What are the
            stability mechanisms that prevent atomic collapse?} The answers diverge sharply.

            \subsection{Abe--Okuyama baseline: thermodynamics as isomorphism}

                The AO approach assumes the standard quantum-mechanical ontology: a Hamiltonian
                $\hat{H}$, energy eigenstates $\{|u_n\rangle\}$ with eigenvalues $E_n$, and a pure
                state
                \begin{equation}
                    |\psi\rangle = \sum_{n} c_n |u_n\rangle,\qquad p_n = |c_n|^2.
                \end{equation}
                The internal energy is
                \begin{equation}
                    E = \langle\psi|\hat{H}|\psi\rangle = \sum_{n} p_n E_n.
                \end{equation}
                Because $S_{\mathrm{vN}}=0$ for pure states, AO identify the entropy with the
                Shannon functional of $\{p_n\}$:
                \begin{equation}
                    S^{\mathrm{S}} = -\kB \sum_{n} p_n\ln p_n.
                \end{equation}
                They decompose an infinitesimal change in $E$ as
                \begin{equation}
                    dE = \sum_{n} E_n\,dp_n + \sum_{n} p_n\,dE_n,
                \end{equation}
                and identify
                \begin{align}
                    \delta' Q &= \sum_{n} E_n\,dp_n,\\
                    \delta' W &= \sum_{n} p_n\,dE_n,
                \end{align}
                as heat and work, respectively, for quasi-static processes. Imposing the
                Clausius equality for reversible changes,
                \begin{equation}
                    dS^{\mathrm{S}} = \frac{\delta' Q}{T},
                \end{equation}
                forces the probabilities $p_n$ to take the canonical form
                \begin{equation}
                    p_n = \frac{1}{Z}\exp\left(-\frac{E_n}{\kB T}\right),
                \end{equation}
                with partition function $Z$. No physical medium is introduced; temperature $T$
                appears as a Lagrange multiplier implementing an information constraint. The
                ``particle-in-a-box'' provides a concrete example with $E_n\propto n^2/L^2$
                and work performed by varying $L$.

            \subsection{Swirl--String baseline: thermodynamics as mechanics}

                SST rejects an information-theoretic interpretation of thermodynamics. The vacuum
                is modeled as an incompressible fluid with effective density $\rhof$, circulation
                quantum $\Gamma_0$, and an intrinsic core radius $\rc$. A stable particle is a
                closed filament with core radius $r\simeq\rc$, embedded in this fluid; its mass
                originates from the kinetic energy of the surrounding swirl and from vacuum
                displacement energy. The primitive SST constant set fixes a characteristic swirl
                speed
                \begin{equation}
                    \vnorm \sim \frac{\Gamma_0}{2\pi\rc},
                \end{equation}
                which, together with $\rhof$ and $\rc$, calibrates the electron mass and the
                hydrogen spectrum in the Canon.

                Within this ontology:
                \begin{itemize}
                    \item \emph{Heat} is the energy exchanged via excitation and damping of Kelvin
                    modes and topological phase transitions at fixed core geometry.
                    \item \emph{Work} is the mechanical energy required to swell or compress the
                    vortex core against the ambient vacuum pressure.
                    \item \emph{Temperature} is defined as a geometric strain variable measuring
                    the deviation of the vortex radius from its equilibrium value.
                \end{itemize}
                These definitions are grounded in Euler–Bernoulli hydrodynamics and a specific
                enthalpy functional $H_{\text{swirl}}$ for the vortex configuration.

        \section{Temperature: Lagrange multiplier vs.\ geometric strain}

            \subsection{AO temperature as a spectral parameter}

                In the AO framework, temperature $T$ emerges from maximizing $S^{\mathrm{S}}$ under
                constraints on $E$ and normalization. It is not introduced as a property of space
                or matter, but as the multiplier relating the entropy gradient to energy
                redistribution:
                \begin{equation}
                    dS^{\mathrm{S}} = \frac{\delta' Q}{T},
                    \qquad
                    \delta' Q = \sum_{n} E_n\,dp_n.
                \end{equation}
                Physically, larger $T$ corresponds to broader occupation of high-energy eigenstates
                in the chosen basis; smaller $T$ corresponds to concentration near the ground
                state. A single eigenstate has no intrinsic temperature beyond this statistical
                interpretation.

            \subsection{SST swirl temperature as radial strain}

                SST introduces a two-scale description of the electron in hydrogen: a core radius
                $r$ (vortex thickness) and an orbital radius $R$ (torus major radius), with ground
                state
                \begin{equation}
                    r_0 = \rc,\qquad R_0 = a_0.
                \end{equation}
                The SST Canon relates these scales via a constitutive relation
                \begin{equation}
                    a_0 = \frac{c^2}{2\vnorm^2}\,\rc,
                \end{equation}
                so small perturbations satisfy
                \begin{equation}
                    \frac{\delta a_0}{a_0} \simeq \frac{\delta\rc}{\rc}.
                \end{equation}
                SST defines a unified strain
                \begin{equation}
                    \epsilon \equiv \frac{r-\rc}{\rc} \simeq \frac{R-a_0}{a_0},
                \end{equation}
                and sets the swirl temperature as
                \begin{equation}
                    T_{\text{swirl}} = \Theta\,\epsilon,
                \end{equation}
                with $\Theta$ an effective stiffness scale (in Kelvin) of the swirl condensate.
                Here $T_{\text{swirl}}$ is an intensive observable of a \emph{single} vortex
                configuration: any deformed state $(R,r)$ has a well-defined temperature, even
                without statistical mixing.

            \subsection{Interpretational divergence}

                The two definitions differ qualitatively:
                \begin{itemize}
                    \item AO temperature is a property of the ensemble of amplitudes $\{c_n\}$; it
                    encodes informational spread over energy eigenstates and has no direct geometric
                    meaning.
                    \item SST temperature is a property of the geometric state $(R,r)$ of a single
                    defect; it measures physical swelling of the core and envelope against vacuum
                    pressure.
                \end{itemize}
                In SST, a ``hotter'' atom is literally larger; in AO, a ``hotter'' state need not
                have any spatial deformation, only a different superposition structure.

        \section{Enthalpy and the equation of state}

            \subsection{Swirl enthalpy functional and harmonic swelling}

                In SST the relevant thermodynamic potential for a hydrogenic vortex ring in a
                background vacuum pressure $P_{\infty}$ is the enthalpy
                \begin{equation}
                    H_{\text{swirl}}(R,r) = E_{\text{kin}}(R,r) + E_{\text{vac}}(R,r)
                    + E_{\text{surf}}(R,r),
                \end{equation}
                where
                \begin{align}
                    E_{\text{kin}}(R,r)
                    &\simeq \frac{1}{2}\,\rhof\,\Gamma^2 R
                    \left[\ln\!\left(\frac{8R}{r}\right) - \alpha\right],
                    \label{eq:Ekin}\\
                    E_{\text{vac}}(R,r)
                    &= P_{\infty}\,V(R,r),\qquad
                    V(R,r) = 2\pi^2 R r^2,\\
                    E_{\text{surf}}(R,r)
                    &\sim \sigma\,4\pi^2 R r,
                \end{align}
                with $\alpha=\mathcal{O}(1)$ and an effective surface tension $\sigma$
                \cite{Saffman1992}. Expanding $H_{\text{swirl}}$ around the ground state
                $(R_0,r_0)=(a_0,\rc)$ and projecting along the unified strain direction
                $\delta R=a_0\epsilon$, $\delta r=\rc\epsilon$ yields
                \begin{equation}
                    H_{\text{swirl}}(T_{\text{swirl}})
                    \simeq H_0 + \frac{1}{2}K_{\mathrm{eff}}
                    \left(\frac{T_{\text{swirl}}}{\Theta}\right)^2,
                \end{equation}
                with $K_{\mathrm{eff}}$ a positive stiffness coefficient combining the second
                derivatives of $H_{\text{swirl}}$ with respect to $R$ and $r$. The thermodynamic
                excitation energy above the ground state is therefore
                \begin{equation}
                    \Delta E_{\text{SST}}(T_{\text{swirl}})
                    = E(T_{\text{swirl}})-E(0) \propto T_{\text{swirl}}^2.
                \end{equation}

            \subsection{AO internal energy scaling}

                In the AO particle-in-a-box, eigenvalues scale as
                \begin{equation}
                    E_n(L) = \frac{h^2 n^2}{8mL^2}.
                \end{equation}
                For a canonical distribution at temperature $T$ one has
                \begin{equation}
                    E_{\mathrm{AO}}(T) = \sum_{n} p_n(T) E_n(L),
                    \qquad
                    p_n(T) = \frac{1}{Z}\exp\left(-\frac{E_n(L)}{\kB T}\right).
                \end{equation}
                At low $T$, the internal energy is dominated by the ground state plus exponentially
                suppressed contributions from the first excited level:
                \begin{equation}
                    \Delta E_{\mathrm{AO}}(T)
                    \sim \Delta E \,\exp\!\left(-\frac{\Delta E}{\kB T}\right),
                \end{equation}
                with $\Delta E=E_2-E_1$. At high $T$, the classical limit recovers $E\propto T$.

            \subsection{Structural conflict}

                SST thus predicts a quadratic power law $\Delta E\propto T_{\text{swirl}}^2$ near
                the ground state, characteristic of a harmonic elastic mode. AO predicts either
                exponential activation (gapped discrete spectrum) at low $T$ or linear scaling
                (classical equipartition) at high $T$. This difference originates in the choice of
                microscopic degrees of freedom: a continuous breathing mode of a core radius vs.\ a
                discrete ladder of standing waves in a rigid box.

        \section{Heat capacity as a vacuum diagnostic}

            \subsection{Linear heat capacity in SST}

                From $\Delta E_{\text{SST}}\propto T_{\text{swirl}}^2$ one obtains
                \begin{equation}
                    C_V^{\text{SST}}(T_{\text{swirl}})
                    \equiv \frac{dE}{dT_{\text{swirl}}}
                    \propto T_{\text{swirl}},
                \end{equation}
                i.e.\ a heat capacity that vanishes linearly with temperature. This behavior is
                typical of gapless excitations with a linear density of states (e.g.\ certain
                fermionic or phononic systems), and suggests that the swirl condensate provides a
                continuum of low-energy deformation modes.

            \subsection{Schottky-type behavior in AO}

                For a low-temperature two-level truncation in the AO picture, the heat capacity
                follows a Schottky-type form
                \begin{equation}
                    C_V^{\mathrm{AO}}(T) \propto
                    \left(\frac{\Delta E}{\kB T}\right)^2
                    \exp\!\left(-\frac{\Delta E}{\kB T}\right),
                \end{equation}
                vanishing exponentially as $T\to 0$, and exhibiting a finite peak at intermediate
                temperatures before approaching a constant high-$T$ value.

            \subsection{Falsifiable difference}

                The qualitative difference,
                \begin{equation}
                    C_V^{\text{SST}}\sim T
                    \quad\text{vs.}\quad
                    C_V^{\mathrm{AO}}\sim e^{-\Delta E/\kB T},
                \end{equation}
                constitutes a clear empirical discriminator, in principle accessible in precision
                low-temperature spectroscopy or analogue systems designed to emulate the SST
                swelling mode.

        \section{Entropy, Golden Layer, and fractal thermodynamics}

            \subsection{Geometric entropy and Boltzmann--swirl distribution}

                SST proposes a probability density for the core radius
                \begin{equation}
                    P(r)\propto
                    \exp\!\left[-\frac{H_{\text{swirl}}(r)}{k_{\text{SST}}\Theta_{\text{vac}}}\right],
                \end{equation}
                with an effective swirl Boltzmann constant $k_{\text{SST}}$ and vacuum noise
                temperature $\Theta_{\text{vac}}$. Near equilibrium, with
                $H_{\text{swirl}}(r)\simeq H_0+\tfrac{1}{2}K_r(r-\rc)^2$, this yields Gaussian
                fluctuations
                \begin{equation}
                    \langle(r-\rc)^2\rangle =
                    \frac{k_{\text{SST}}\Theta_{\text{vac}}}{K_r},
                \end{equation}
                and entropy as a measure of geometric phase-space volume of the fluctuating
                boundary. The spatial ``fuzziness'' of the electron is interpreted as thermal
                fluctuation of a vortex interface rather than intrinsic indeterminacy.

            \subsection{Golden potential and discrete scale invariance}

                The Golden Principle in SST introduces a log-periodic vacuum potential for the
                swirl energy density $\rhoE$,
                \begin{equation}
                    V_{\phi}(\rhoE) =
                    \Lambda^4\left[
                                 1 - \cos\!\left(
                                               \frac{2\pi}{\ln\phi}\,
                                               \ln\frac{\rhoE}{\rhoE^\ast}
                        \right)
                    \right],
                \end{equation}
                where $\phi$ is the Golden Ratio and $\rhoE^\ast$ a reference density. Because
                $\rhoE\propto v^2\propto 1/r^2$, this induces a sequence of preferred radii
                corresponding to minima of $V_{\phi}$, leading to discrete scale invariance in
                core swelling. Summing over such Golden-layer states produces log-periodic
                corrections to thermodynamic quantities, including a fractal heat capacity with
                oscillatory dependence on $\ln T$ \cite{Sornette1998}.

            \subsection{Contrast with smooth Shannon entropy}

                In AO the entropy is smooth in $\{p_n\}$ and, through the canonical distribution,
                produces thermodynamic functions without log-periodic structure. There is no
                number-theoretic structure analogous to $\phi$ in the underlying Hilbert-space
                formalism. SST thereby embeds number theory directly into thermodynamics through
                the Golden Layer, whereas AO remains purely analytic and smooth.

        \section{Stability: Uncertainty principle vs.\ hydrodynamic balance}

            \subsection{Kinematic stability in AO/QM}

                In standard quantum mechanics, and implicitly in AO, atomic stability is ensured by
                Heisenberg's uncertainty relation. Confining a particle to a narrower spatial
                region increases momentum variance and kinetic energy, preventing collapse of the
                wavefunction into the nucleus. The potential well is imposed externally; stability
                is kinematic and probabilistic.

            \subsection{Mechanical stability and Golden filter in SST}

                SST explains stability through the equilibrium of pressures:
                \begin{itemize}
                    \item Centrifugal swirl pressure $P_{\text{cent}}(r)$ from circulation around
                    the core, tending to expand the loop.
                    \item Vacuum confining pressure $P_{\text{vac}}$ from the background condensate,
                    tending to contract it.
                \end{itemize}
                The core equilibrium radius $\rc$ satisfies $P_{\text{cent}}(\rc)=P_{\text{vac}}$.
                For composite knots such as the proton, deeper layers of topology are protected by
                Golden suppression factors in the mass functional, rendering inner structures
                thermally inaccessible and dynamically stable over cosmological times.

        \section{Chronometric dimension and inverse-time cooling}

            SST relates local proper time $\tau$ to swirl speed via the Swirl Clock,
            \begin{equation}
                \SwirlClock = \frac{d\tau}{dt} =
                \sqrt{1-\frac{v^2}{c^2}},
                \qquad v=v_{\theta}(r),
            \end{equation}
            with $v_{\theta}(r)\sim \Gamma/(2\pi r)$. Cooling a bound state drives $r\to\rc$,
            increasing $v$ and decreasing $\SwirlClock$: cold, compact states have more
            time dilation and slower internal clocks. Heating swells the core, reduces $v$,
            and moves $\SwirlClock$ toward unity; excited states age faster.

            AO, in contrast, uses an external time parameter $t$ independent of temperature or
            internal deformation. No chronometric thermodynamics is present.

        \section{From particle-in-a-box to harmonic swelling}

            In AO the confinement length $L$ is an externally set parameter of the potential;
            work is defined by moving the box walls via an external agent. In SST, the
            confinement scale $a_0$ is tied to the core radius $\rc$ and arises from
            hydrodynamic balance. One cannot change the ``box'' without deforming the
            ``particle'': the system is self-confining.

            The SST harmonic swelling model thus internalizes the boundary conditions that in
            AO are imposed by hand. Confinement and mass emerge from the same enthalpy
            functional, rather than being split into separate kinematic and potential terms.

        \section{Outlook and experimental directions}

            The structural differences between AO and SST thermodynamics offer concrete,
            though challenging, experimental tests:
            \begin{itemize}
                \item \textbf{Low-temperature specific heat}: SST predicts $C_V\propto T$ for
                the swelling mode, while AO predicts exponentially small $C_V$ for isolated,
                gapped atomic degrees of freedom.
                \item \textbf{Log-periodic modulations}: Golden layering implies subtle,
                log-periodic oscillations of $C_V(T)$ on a logarithmic temperature scale; AO
                predicts smooth, monotonic dependence.
                \item \textbf{Temperature-dependent decay}: SST's inverse-time cooling implies
                that decay rates of unstable states should depend on their geometric swelling,
                and hence on temperature; AO ties decay primarily to coupling strengths and
                spectral gaps with no explicit geometric strain.
                \item \textbf{Analogue systems}: Superfluid and BEC vortices, as well as
                high-$Q$ cavities probing Unruh-type excitations, may provide experimental
                platforms where SST-inspired swelling and transduction dynamics can be tested.
            \end{itemize}
            A decisive discrimination between information-theoretic and hydrodynamic
            thermodynamics at the microscopic scale would have far-reaching consequences for
            our understanding of the quantum vacuum.

            \appendix

        \section*{Appendix: Summary comparison table}

            \begin{center}
                \begin{tabular}{lll}
                    \hline
                    Feature &
                    Abe--Okuyama (AO) &
                    Swirl--String Theory (SST) \\
                    \hline
                    Ontology &
                    Hilbert space, pure states &
                    Fluid condensate, vortex defects \\
                    Temperature &
                    Lagrange multiplier on $S^{\mathrm{S}}$ &
                    Geometric strain $T_{\text{swirl}}=\Theta\epsilon$ \\
                    Heat $Q$ &
                    Energy from $dp_n$ at fixed $E_n$ &
                    Mode/topology excitation at fixed $\rc$ \\
                    Work $W$ &
                    Energy from $dE_n$ at fixed $p_n$ &
                    Mechanical swelling/compression of core \\
                    Equation of state &
                    $\Delta E\sim e^{-\Delta E/\kB T}$ (low $T$) &
                    $\Delta E\propto T_{\text{swirl}}^2$ \\
                    Heat capacity $C_V$ &
                    Schottky/exponential, then constant &
                    Linear: $C_V\propto T_{\text{swirl}}$ \\
                    Entropy &
                    Shannon entropy of $\{p_n\}$ &
                    Geometric phase-space, Golden layering \\
                    Stability &
                    Uncertainty principle, external well &
                    Pressure balance, Golden filter \\
                    Time &
                    External parameter $t$ &
                    Swirl Clock $\SwirlClock=\sqrt{1-v^2/c^2}$ \\
                    Model motif &
                    Particle-in-a-box &
                    Harmonic swelling of self-confined ring \\
                    \hline
                \end{tabular}
            \end{center}

            \begin{thebibliography}{9}

                \bibitem{AbeOkuyama2011}
                S.~Abe and S.~Okuyama,
                Similarity between quantum mechanics and thermodynamics:
                Entropy, temperature, and Carnot cycle,
                Phys.\ Rev.\ E \textbf{83}, 021121 (2011).
                \newblock \href{https://doi.org/10.1103/PhysRevE.83.021121}{doi:10.1103/PhysRevE.83.021121}.

                \bibitem{Saffman1992}
                P.~G.~Saffman,
                \emph{Vortex Dynamics},
                Cambridge University Press, Cambridge (1992).

                \bibitem{Sornette1998}
                D.~Sornette,
                Discrete scale invariance and complex dimensions,
                Phys.\ Rep.\ \textbf{297}, 239--270 (1998).
                \newblock \href{https://doi.org/10.1016/S0370-1573(97)00076-8}{doi:10.1016/S0370-1573(97)00076-8}.

            \end{thebibliography}

    \end{document}