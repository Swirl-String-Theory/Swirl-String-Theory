%! Author = Omar Iskandarani
%! Title = Swirl String Theory (SST) Canon v0.5
%! Date = Sept 9, 2025
%! Affiliation = Independent Researcher, Groningen, The Netherlands
%! License = © 2025 Omar Iskandarani. All rights reserved. This manuscript is made available for academic reading and citation only. No republication, redistribution, or derivative works are permitted without explicit written permission from the author. Contact: info@omariskandarani.com
%! ORCID = 0009-0006-1686-3961
%! DOI = 10.5281/zenodo.17052966 % Placeholder DOI, update upon finalization

\newcommand{\canonversion}{\textbf{v0.5}} % Semantic versioning: vMAJOR.MINOR.PATCH
\newcommand{\papertitle}{Swirl String Theory (SST) Canon \canonversion}
\newcommand{\paperdoi}{10.5281/zenodo.17052966}



%========================================================================================
% PACKAGES AND DOCUMENT CONFIGURATION
%========================================================================================
\documentclass[11pt]{article}
\usepackage{subfiles}
% sststyle.sty
\NeedsTeXFormat{LaTeX2e}
\ProvidesPackage{sststyle}[2025/07/01 SST unified style]



% === Draft Options ===
\newif\ifsstdraft
% \sstdrafttrue
\ifsstdraft
\RequirePackage{showframe}
\fi

% === Load Once ===
\RequirePackage{ifthen}
\newboolean{sststyleloaded}
\ifthenelse{\boolean{sststyleloaded}}{}{\setboolean{sststyleloaded}{true}

% === Page ===
\RequirePackage[a4paper, margin=2.5cm]{geometry}

% === Fonts ===
\RequirePackage[T1]{fontenc}
\RequirePackage[utf8]{inputenc}
\RequirePackage[english]{babel}
\RequirePackage{textgreek}
\RequirePackage{mathpazo}
\RequirePackage[scaled=0.95]{inconsolata}
\RequirePackage{helvet}

% === Math ===
\RequirePackage{amsmath, amssymb, mathrsfs, physics}
\RequirePackage{siunitx}
\sisetup{per-mode=symbol}

% === Tables ===
\RequirePackage{graphicx, float, booktabs}
\RequirePackage{array, tabularx, multirow, makecell}
\newcolumntype{Y}{>{\centering\arraybackslash}X}
\newenvironment{tighttable}[1][]{\begin{table}[H]\centering\renewcommand{\arraystretch}{1.3}\begin{tabularx}{\textwidth}{#1}}{\end{tabularx}\end{table}}
\RequirePackage{etoolbox}
\newcommand{\fitbox}[2][\linewidth]{\makebox[#1]{\resizebox{#1}{!}{#2}}}

% === Graphics ===
\RequirePackage{tikz}
\usetikzlibrary{3d, calc, arrows.meta, positioning}
\RequirePackage{pgfplots}
\pgfplotsset{compat=1.18}
\RequirePackage{xcolor}

% === Code ===
\RequirePackage{listings}
\lstset{basicstyle=\ttfamily\footnotesize, breaklines=true}

% === Theorems ===
\newtheorem{theorem}{Theorem}[section]
\newtheorem{lemma}[theorem]{Lemma}

% === TOC ===
\RequirePackage{tocloft}
\setcounter{tocdepth}{2}
\renewcommand{\cftsecfont}{\bfseries}
\renewcommand{\cftsubsecfont}{\itshape}
\renewcommand{\cftsecleader}{\cftdotfill{.}}
\renewcommand{\contentsname}{\centering \Huge\textbf{Contents}}

% === Sections ===
\RequirePackage{sectsty}
\sectionfont{\Large\bfseries\sffamily}
\subsectionfont{\large\bfseries\sffamily}

% === Bibliography ===


% === Links ===
\RequirePackage{hyperref}
\hypersetup{
    colorlinks=true,
    linkcolor=blue,
    citecolor=blue,
    urlcolor=blue,
    pdftitle={The Vortex \AE ther Model},
    pdfauthor={Omar Iskandarani},
    pdfkeywords={vorticity, gravity, \ae ther, fluid dynamics, time dilation, SST}
}
\urlstyle{same}
\RequirePackage{bookmark}

% === Misc ===
\RequirePackage[none]{hyphenat}
\sloppy
\RequirePackage{empheq}
\RequirePackage[most]{tcolorbox}
\newtcolorbox{eqbox}{colback=blue!5!white, colframe=blue!75!black, boxrule=0.6pt, arc=4pt, left=6pt, right=6pt, top=4pt, bottom=4pt}
\RequirePackage{titling}
\RequirePackage{amsfonts}
\RequirePackage{titlesec}
\RequirePackage{enumitem}

\AtBeginDocument{\RenewCommandCopy\qty\SI}

\pretitle{\begin{center}\LARGE\bfseries}
\posttitle{\par\end{center}\vskip 0.5em}
\preauthor{\begin{center}\large}
\postauthor{\end{center}}
\predate{\begin{center}\small}
\postdate{\end{center}}


\endinput
} % Assuming a local style file
% sstappendixsetup.sty

\newcommand{\titlepageOpen}{
  \begin{titlepage}
  \thispagestyle{empty}
  \centering
  \ifdefined\standalonechapter
  {\Large\bfseries \appendixtitle \par}
  \else
  {\Large\bfseries \papertitle \par}
    \fi
  \vspace{1cm}
  {\Large\itshape \textbf{Omar Iskandarani}\textsuperscript{\textbf{*}} \par}
  \vspace{0.5cm}
  {\today \par}
  \vspace{0.5cm}
}

% here comes abstract
\newcommand{\titlepageClose}{
  \vfill
  \raggedright % <-- fixes left alignment
  \null
  \begin{picture}(0,0)
  % Adjust position: (x,y) = (left, bottom)
  \put(0,-45){  % Shift 200pt left, 40pt down
    \begin{minipage}[b]{0.7\textwidth}
    \footnotesize % One step bigger than \tiny \scriptsize
    \renewcommand{\arraystretch}{1.0}
    \noindent\rule{\textwidth}{0.4pt} \\[0.5em]  % ← horizontal line
    \textsuperscript{\textbf{*}} Independent Researcher, Groningen, The Netherlands \\
    Email: \texttt{info@omariskandarani.com} \\
    ORCID: \texttt{\href{https://orcid.org/0009-0006-1686-3961}{0009-0006-1686-3961}} \\
    DOI: \href{https://doi.org/\paperdoi}{\paperdoi} \\
    License: CC-BY-NC 4.0 International \\
    \end{minipage}
  }
  \end{picture}
  \end{titlepage}
} % Assuming a local appendix setup
\usepackage[margin=1in]{geometry}
\usepackage{amsmath,amssymb,amsfonts,amsthm}
\usepackage{tcolorbox}
\usetikzlibrary{knots,intersections,decorations.pathreplacing,3d,calc,arrows.meta,positioning,decorations.pathmorphing}
\usepackage{pgfmath}
\usepackage{pgfplots}
\pgfplotsset{compat=1.18}
\usepackage{ulem}


% ==== Packages ====
\usepackage[T1]{fontenc}
\usepackage{lmodern}
\usepackage{microtype}

\geometry{margin=1in}
\usepackage{ bm, mathtools}
\usepackage{siunitx}
\sisetup{per-mode=symbol,round-mode=figures,round-precision=6}
\usepackage{physics}
\usepackage{upgreek}
\usepackage{graphicx}
\usepackage{booktabs}
\usepackage{hyperref}
\hypersetup{colorlinks=true, linkcolor=blue!60!black, citecolor=blue!60!black, urlcolor=blue!60!black}


% ===== Gauge sector macros =====
\newcommand{\Tr}{\mathrm{Tr}}
\newcommand{\ii}{\mathrm{i}}
% Gauge fields (adjoints; indices a=1..8, i=1..3)
\newcommand{\GsA}{G^a_{\mu\nu}}
\newcommand{\WsI}{W^i_{\mu\nu}}
\newcommand{\Bmn}{B_{\mu\nu}}



% ===============================
% Macros (canonicalized)
% ===============================

% swirl arrows (context-aware)
\newcommand{\swirlarrow}{%
    \mathchoice{\mkern-2mu\scriptstyle\boldsymbol{\circlearrowleft}}%
    {\mkern-2mu\scriptstyle\boldsymbol{\circlearrowleft}}%
    {\mkern-2mu\scriptscriptstyle\boldsymbol{\circlearrowleft}}%
    {\mkern-2mu\scriptscriptstyle\boldsymbol{\circlearrowleft}}%
}
\newcommand{\swirlarrowcw}{%
    \mathchoice{\mkern-2mu\scriptstyle\boldsymbol{\circlearrowright}}%
    {\mkern-2mu\scriptstyle\boldsymbol{\circlearrowright}}%
    {\mkern-2mu\scriptscriptstyle\boldsymbol{\circlearrowright}}%
    {\mkern-2mu\scriptscriptstyle\boldsymbol{\circlearrowright}}%
}


% Canonical symbols
\newcommand{\vswirl}{\mathbf{v}_{\swirlarrow}}
\newcommand{\vswirlcw}{\mathbf{v}_{\swirlarrowcw}}
\newcommand{\SwirlClock}{S_{(t)}^{\swirlarrow}}
\newcommand{\SwirlClockcw}{S_{(t)}^{\swirlarrowcw}}
\newcommand{\omegas}{\boldsymbol{\omega}_{\swirlarrow}}  % swirl vorticity
\newcommand{\vscore}{v_{\swirlarrow}}                    % shorthand: |v_swirl| at r=r_c
\newcommand{\vnorm}{\lVert \vswirl \rVert}               % swirl speed magnitude
\newcommand{\rhof}{\rho_{\!f}}                           % effective fluid density
\newcommand{\rhoE}{\rho_{\!E}}                           % swirl energy density
\newcommand{\rhom}{\rho_{\!m}}                           % mass-equivalent density
\newcommand{\rc}{r_c}                                    % string core radius (swirl string radius)
\newcommand{\FmaxEM}{F_{\mathrm{EM}}^{\max}}             % EM-like maximal force scale
\newcommand{\FmaxG}{F_{\mathrm{G}}^{\max}}               % G-like maximal force scale
\newcommand{\Lam}{\Lambda}                               % Swirl Coulomb constant
\newcommand{\Om}{\Omega_{\swirlarrow}}                   % swirl angular frequency profile
\newcommand{\alpg}{\alpha_g}                             % gravitational fine-structure analogue

% Policy: the golden constant is only allowed via hyperbolic functions.
\newcommand{\xig}{\operatorname{asinh}\!\left(\tfrac{1}{2}\right)}
\newcommand{\phig}{\exp(\xig)}
\newcommand{\phialg}{\bigl(1+\sqrt{5}\bigr)/2}
\newcommand{\xigold}{\tfrac{3}{2}\,\xig}
\newcommand{\GoldenDeclare}{%
    \textbf{Golden (hyperbolic)}:\ \(\ln\phi=\xig\), hence \(\phi=\phig\).
    \ \emph{(Equivalently, \(\phi=\phialg\); the algebraic form is derivative.)}%
}

% Theorem-like environments
\newtheorem{identity}{Identity}
\newtheorem{axiom}{Axiom}
\newtheorem{theorem}{Theorem}[section]
\newtheorem{lemma}[theorem]{Lemma}
\newtheorem{corollary}[theorem]{Corollary}
\newtheorem{definition}{Definition}[section]
\newtheorem{postulate}{Postulate}



%========================================================================================
% DOCUMENT START
%========================================================================================
\begin{document}

%========================================================================================
% TITLE PAGE
%========================================================================================

\titlepageOpen
\begin{abstract}
This Canon is the single source of truth for \emph{Swirl String Theory (SST)}: definitions, constants, boxed master equations, and notational conventions. It consolidates the core hydrodynamic and topological structure and integrates the Standard Model gauge sector as an emergent property of swirl dynamics. Canonical promotions in this version include:
\begin{itemize}
\item[\textbf{I}] The Swirl Coulomb constant $\Lam$ and hydrogen soft-core potential.
\item[\textbf{II}] The circulation–metric corollary and the Swirl Clock law.
\item[\textbf{III}] The Kelvin-compatible swirl Hamiltonian density and the swirl pressure law.
\item[\textbf{IV}] The emergence of Yang-Mills gauge fields from swirl-director elasticity.
\item[\textbf{V}] A parameter-free prediction for the Electroweak Symmetry Breaking (EWSB) scale from swirl energy density and topological weights.
\end{itemize}

\paragraph{Core Axioms (SST)}
    \begin{enumerate}
    \item \textbf{Swirl Medium:} Physics is formulated on $\mathbb{R}^3$ with absolute reference time. Dynamics occur in a frictionless, incompressible \emph{swirl condensate}, which serves as a universal substrate.
    \item \textbf{Swirl Strings (Circulation and Topology):} Particles and field quanta correspond to closed vortex filaments (\emph{swirl strings}). The circulation of the swirl velocity around any closed loop is quantized:
    \[
        \Gamma = \oint \vswirl \cdot d\boldsymbol{\ell} = n\,\kappa,\qquad n\in\mathbb{Z},\qquad \kappa = \frac{h}{m_{\text{eff}}}.
    \]
    Discrete quantum numbers (mass, charge, spin) track to the topological invariants of the swirl string (linking number, writhe, twist).
    \item \textbf{String-induced gravitation:} Macroscopic attraction emerges from coherent swirl flows and swirl-pressure gradients. The effective gravitational coupling $G_{\text{swirl}}$ is fixed by canonical constants.
    \item \textbf{Swirl Clocks:} Local proper-time rate depends on tangential swirl speed $v$, ticking slower by the factor $S_t=\sqrt{\,1-v^2/c^2\,}$ relative to an observer at rest in the medium.
    \item \textbf{Dual Phases (Wave–Particle):} Each swirl string has two limiting phases: an extended \emph{R-phase} (unknotted, wave-like) and a localized \emph{T-phase} (knotted, particle-like). Measurement corresponds to transitions between these phases.
    \item \textbf{Taxonomy:} Unknotted excitations are bosonic modes; chiral hyperbolic knots map to quarks; torus knots map to leptons. The detailed particle–knot dictionary is documented in the appendices.
    \end{enumerate}
\end{abstract}
\vfill
\titlepageClose

%================================================
\section{Classical Invariants and Swirl Quantization}
%================================================
    \label{sec:classical_invariants}
    Under Axiom 1 (inviscid, barotropic medium), the Euler equations yield standard vortex invariants. Kelvin's circulation theorem is central:
    \begin{equation}
    \frac{d\Gamma}{dt}=0, \qquad \Gamma=\oint_{\mathcal{C}(t)} \vswirl\cdot d\boldsymbol{\ell}. \label{eq:kelvin}
    \end{equation}
    This ensures the stability of swirl strings and quantizes their circulation (Axiom 2). The vorticity $\omegas = \nabla \times \vswirl$ is transported with the flow, and helicity $H = \int \vswirl \cdot \omegas \, dV$ is conserved up to reconnections.

    \begin{axiom}[Chronos–Kelvin Invariant]
    \label{ax:chronos-kelvin}
    For any thin, closed swirl loop with core radius $R(t)$, the following material invariant holds:
    \begin{equation}
    \boxed{\;
    \frac{D}{Dt}\!\Big(R^2\,\omega\Big)=0,
        \;} \qquad \text{equivalently,} \qquad
    \boxed{\;
    \frac{D}{Dt}\!\Big(
    \frac{c}{r_c}\,R^2 \sqrt{\,1-S_t^2\,}
    \Big)=0\,,
        \;}
    \label{eq:CK}
    \end{equation}
    where $\omega=\|\omegas\|$ on the loop and $S_t=\sqrt{\,1-(\omega r_c/c)^2\,}$ is the local Swirl Clock factor.
    \end{axiom}

    \subsection{Swirl Quantization Principle}
        \label{sec:swirl_quantization}
        By Axiom 2, the circulation of $\vswirl$ around any closed loop is quantized in units of $\kappa=h/m_{\text{eff}}$. Closed swirl filaments may form nontrivial knots and links, each topological class corresponding to a discrete excitation state:
        \[
            \Gamma = n\kappa \qquad \text{and} \qquad \text{Topology}(K)\in \{\text{trefoil knot},\; \text{figure-eight knot},\; \text{Hopf link},\;\dots\}\,.
        \]
        This joint discreteness of circulation and topology is the \emph{Swirl Quantization Principle}. It replaces canonical commutation relations with topological and integral constraints as the origin of quantum phenomena.

%===============================================================
\section{Canonical Constants and Effective Densities}
%===============================================================
    \label{sec:canonical_constants}
    \subsection*{Primary SST Constants (SI units)}
        \begin{itemize}
        \item \textbf{Swirl speed scale (core):} $\vscore = \num{1.09384563e6}\ \si{m/s}$ (tangential speed at $r=\rc$).
        \item \textbf{String core radius:} $\rc = \num{1.40897017e-15}\ \si{m}$.
        \item \textbf{Effective fluid density:} $\rhof = \num{7.0e-7}\ \si{kg/m^3}$ (a defined calibration constant).
        \item \textbf{Mass-equivalent density:} $\rhom = \num{3.8934358266918687e18}\ \si{kg/m^3}$.
        \item \textbf{EM-like maximal force:} $\FmaxEM = \num{2.9053507e1}\ \si{N}$.
        \item \textbf{Golden (hyperbolic):} $\ln\varphi=\operatorname{asinh}\!\left(\tfrac12\right)$, so $\varphi=e^{\operatorname{asinh}(1/2)}$.
        \end{itemize}

    \subsection*{Universal Constants Used}
        \begin{itemize}
        \item $c=\num{299792458}\ \si{m/s}$, \quad $t_p=\num{5.391247e-44}\ \si{s}$ (Planck time).
        \item Fine-structure constant: $\alpha \approx \num{7.29735256e-3}$.
        \end{itemize}

        \paragraph{Effective densities.} We define the swirl energy density $\rhoE$ and mass-equivalent density $\rhom$ from the effective fluid density $\rhof$:
            \[
                \rhoE \equiv \tfrac{1}{2}\,\rhof\,\vscore^2, \qquad
                \rhom \equiv \frac{\rhoE}{c^2}.
            \]
            The value of $\rhof$ is fixed by definition to calibrate swirl energetics against electromagnetism.

%================================================
\section{Canon Governance and Status Taxonomy}
%================================================
\label{sec:canon_governance}
A statement is \emph{canonical} if it is derivable from the Axioms and Definitions of SST, consistent with all prior canonical results, and satisfies checks for dimensional consistency, symmetry compliance, and recovery of classical limits. Numerical values for constants are \emph{Empirical Calibrations} used to anchor the theory but are not premises in proofs. Conjectures or unproven extensions are classified as \emph{Research Track} and are non-canonical.

%================================================
\section{Master Equations (Boxed Canonical Relations)}
%================================================
\label{sec:master_equations}

\subsection*{Hydrodynamics and Potential Theory}
    \begin{itemize}
    \item \textbf{Swirl Coulomb Potential:} The potential for a fundamental swirl string is regularized at the core:
    \[ \boxed{\,V_{\text{SST}}(r)=-\,\frac{\Lambda}{\sqrt{r^2+\rc^2}}\,,\qquad [\Lambda]=\mathrm{J\,m}\,}\,. \]
    Here, the Swirl Coulomb constant $\Lambda$ is defined by the core parameters:
    \[ \boxed{\,\Lambda = 4\pi\,\rhom\,\vscore^2\,\rc^4\,}\,. \]
    \item \textbf{Swirl Pressure Law:} From the radial component of the steady-state Euler equation:
    \[ \boxed{\,\frac{1}{\rhof}\frac{dp_{\text{swirl}}}{dr}=\frac{v_\theta(r)^2}{r}\,}\,. \]
    \item \textbf{Swirl Clock (Local Time Rate):} The local time dilation factor due to swirl velocity $v = \|\vswirl\|$ at $r=\rc$:
    \[ \boxed{\,\frac{dt_{\text{local}}}{dt_{\infty}} = \sqrt{\,1 - \frac{\|\vswirl\|^2}{c^2}\,}\quad (r=\rc)\,}\,. \]
    \end{itemize}

\subsection*{Gravitation and Mass}
    \begin{itemize}
    \item \textbf{Swirl–Gravity Coupling:} The gravitational constant emerges from core constants:
    \[ \boxed{\,G_{\text{swirl}} = \frac{\vscore\,c^5\,t_p^2}{2\,\FmaxEM\,\rc^2}\, \approx G_{\text{Newton}}}\,. \]
    \item \textbf{Topology–Driven Mass Law:} For a torus knot $T(p,q)$ with braid index $b$, genus $g$, number of components $n$, and dimensionless ropelength $\mathcal{L}_{\text{tot}}$:
    \[ \boxed{\,M\big(T(p,q)\big) = \left(\frac{4}{\alpha}\right) b^{-3/2}\,\varphi^{-\,g}\,n^{-1/\varphi} \left(\frac{1}{2}\,\rhof \vscore^2\right) \frac{\pi\,\rc^3\,\mathcal{L}_{\text{tot}}(T)}{c^2}\,}\,. \]
    \end{itemize}

\subsection*{Unified SST Lagrangian (Definitive Form)}
    \label{sec:lagrangian}
    The total dynamics of swirl and emergent gauge fields are described by:
    \[
        \boxed{\,\mathcal{L}_{\text{SST+Gauge}}
            =
            \underbrace{\frac{1}{2}\rhof\,\|\vswirl\|^2
            - \rhof\,\Phi_{\text{swirl}}
                + \lambda(\nabla\cdot\vswirl)
                + \chi_h\,\rhof\,(\vswirl\cdot\omegas)}_{\text{SST Hydrodynamics}}
            \;+\;
            \underbrace{\mathcal{L}_{\text{YM}}}_{\text{Gauge Fields}}
            \;+\;
            \underbrace{\mathcal{L}_{\text{Matter}}}_{\text{Gauge-charged Matter}}
            \,}\,.
    \]
    All terms have units of energy density ($\mathrm{J\,m^{-3}}$). The helicity coupling $\chi_h$ is dimensionless. $\mathcal{L}_{\text{YM}}$ and $\mathcal{L}_{\text{Matter}}$ are emergent terms described in Sec.~\ref{sec:gauge_core}.

%================================================
\section{The Standard Gauge Sector}
%================================================
\label{sec:gauge_core}
The gauge structure of the Standard Model emerges from the collective dynamics of swirl-string directors.

\begin{theorem}[Emergent Yang–Mills from Swirl Directors]
Let local swirl orientations be described by director frames $U_3(x)\!\in\!SU(3)$, $U_2(x)\!\in\!SU(2)$, and a phase $\vartheta(x)\!\in\!\mathbb R$. The kinetic energy of these directors (swirl elasticity) is given by:
\[ \mathcal L_{\rm dir}=\tfrac{\kappa_3}{2}\Tr(\partial_\mu U_3\partial^\mu U_3^\dagger) +\tfrac{\kappa_2}{2}\Tr(\partial_\mu U_2\partial^\mu U_2^\dagger) +\tfrac{\kappa_1}{2}(\partial_\mu\vartheta)^2. \]
Coarse-graining this Lagrangian over the swirl medium yields the effective Yang-Mills Lagrangian for the gauge fields $G_\mu, W_\mu, B_\mu$:
\[ \boxed{\ \mathcal L_{\rm YM}^{\rm eff}=-\tfrac14\sum_{i=1}^3 g_i^{-2}\,F^{(i)}_{\mu\nu}F^{(i)\,\mu\nu},\qquad g_i^{-2}=c_i\,\kappa_i,\ c_i>0\ } \]
where the squared inverse couplings $g_i^{-2}$ are proportional to the director stiffness constants $\kappa_i$.
\end{theorem}

\subsection{Knot-to-Representation Mapping}
    Quantum numbers are identified with topological invariants of the swirl strings.
    \begin{definition}[Hypercharge from Swirl Indices]
    For an oriented, framed knot $K$, let $s_3\!\in\!\{+1,0,-1\}$ be its color sign, $d_2\!\in\!\{0,1\}$ be its doublet indicator, and $\tau\!\in\!\{-1,0,+1\}$ be its twist sign. The hypercharge $Y$ is defined as:
    \[ \boxed{ Y(K)=\tfrac{1}{2}+\tfrac{2}{3}s_3(K)-d_2(K)-\tfrac{1}{2}\tau(K) } \]
    With the standard definition $Q=T_3+Y$, this map reproduces the electric charges of all Standard Model particles for one generation.
    \end{definition}

    \begin{theorem}[Per-Generation Anomaly Cancellation]
    The spectrum of left-chiral fermions generated by the knot-to-representation map is free of all gauge and mixed gravitational anomalies. For example, the $\mathrm{SU}(3)^2\mathrm{U}(1)$ and $\mathrm{SU}(2)^2\mathrm{U}(1)$ anomalies vanish:
    \[
        \sum_\alpha Y_\alpha\,T(R^{(\alpha)}_3)\,\dim R^{(\alpha)}_2=0,\quad
        \sum_\alpha Y_\alpha\,T(R^{(\alpha)}_2)\,\dim R^{(\alpha)}_3=0.
    \]
    \end{theorem}

\subsection{Coupling Constants and EWSB}
    The values of the gauge couplings and the EWSB scale are not free parameters but are determined by the fundamental swirl constants.

    \paragraph{Canonical Renormalization Scale and Couplings.}
        We define a canonical energy scale $\mu_\*$ and a dimensionless core modulus $\Sigma_{\rm core}$ from the swirl constants:
        \[ \mu_\* \equiv \frac{\hbar\,\vscore}{\rc} \approx 0.511~\mathrm{MeV}, \qquad \Sigma_{\rm core} \equiv \frac{\rhom\,\vscore^2\,\rc^2}{\FmaxEM} = \frac{1}{\pi}. \]
        The gauge couplings at this scale are determined by topological weights $W_i$ (derived from charge counting per generation):
        \[ \boxed{\ g_i^{-2}(\mu_\*) = \kappa_i\;\Sigma_{\rm core}\;W_i\ }, \]
        where $\kappa_i$ are geometric factors of order unity. This provides a first-principles calculation of the gauge couplings.

    \paragraph{Electroweak Symmetry Breaking.}
        The EWSB scale $v_\Phi$ is determined by the bulk swirl energy density $u_{\rm swirl} = \frac{1}{2}\rhof\vscore^2$ and the topological weights $W_i$.
        \begin{theorem}[EWSB Scale from Swirl Density]
        The vacuum expectation value $v_\Phi$ is given by the parameter-free relation:
        \[ \boxed{\ v_\Phi\ =\ u_{\rm swirl}^{1/4}\;\big(W_1 W_2 W_3\big)^{1/4}\ }. \]
        Using the canonical constants, this yields a prediction:
        \[ u_{\rm swirl}^{1/4} \approx 135.8~\mathrm{GeV}, \quad (W_1W_2W_3)^{1/4} \approx 1.912 \quad \Longrightarrow \quad \boxed{ v_\Phi^{\rm pred} \approx 259.5\ \mathrm{GeV}\ }. \]
        This value is within 5.4\% of the measured value of $246.22~\mathrm{GeV}$, with the small discrepancy expected to be resolved by higher-order corrections.
        \end{theorem}

        The standard electroweak mass relations remain canonical:
        \[ m_W=\tfrac12 g_2 v_\Phi,\qquad m_Z=\tfrac12\sqrt{g_2^2+g_1^2}\,v_\Phi,\qquad m_\gamma=0. \]


%================================================
\section{Wave–Particle Duality and Bosons in SST}
%================================================
\label{sec:wave_particle_duality}
\paragraph{Dual Phases.}
    An unknotted swirl string (R-phase) behaves as a delocalized wave, capable of interference. A knotted swirl string (T-phase) is a localized, particle-like soliton. The de Broglie relation $\lambda = h/p$ emerges from the requirement of wave coherence around a closed R-phase loop. Measurement is modeled as an interaction that forces a transition from the R-phase to the T-phase (collapse).

\paragraph{Unknot Bosons and Photons.}
    Unknotted swirl strings are intrinsically bosonic. The photon is identified with a delocalized, unknotted swirl oscillation propagating losslessly through the medium. The dynamics of these oscillations are described by a wave equation derived from a quadratic Lagrangian for a transverse swirl potential $\mathbf{a}(\mathbf{x}, t)$, which is mathematically equivalent to Maxwell's equations in vacuum.
    \[ \boxed{\,\partial_t^2 \mathbf{a} - c^2\,\nabla\times(\nabla\times \mathbf{a}) = 0, \qquad \nabla\cdot \mathbf{a}=0\,}\,. \]
    This establishes that photons are not localized "smoke rings" but extended transverse waves in the swirl condensate.

%================================================
% APPENDICES
%================================================
    \appendix
\section{Canon 4R: Research Extensions (Non-Canonical)}
The following relations are recorded for future investigation. They are dimensionally consistent but not yet promoted to canonical status.
\begin{itemize}
\item \textbf{EWSB from Anisotropy:} An older, deprecated model for the EWSB scale based on the UV core radius $\rc$:
\[ v_\Phi^{(\text{old})} = \left(\frac{\hbar c}{\rc}\right) \sqrt{\frac{\zeta_{\rm EW}}{\pi\,\lambda_\Phi}}. \]
This form is version-unstable and has been replaced by the density-topology law in Sec.~\ref{sec:gauge_core}.
\item \textbf{Swirl-Helicity and Chern-Simons Couplings:} Potential couplings between the swirl helicity term and topological terms in the gauge sector.
\item \textbf{Detailed Knot-to-Mass Spectrum:} A complete mapping from specific knot classes to the full particle mass spectrum, including binding energies.
\end{itemize}

\section{Critical Questions and Falsifiable Predictions}
\begin{itemize}
\item \textbf{EMF Quantization:} Are reconnection events observable as discrete flux impulses $\Delta\Phi = m \Phi^*$ in SQUID-like detectors? SST predicts yes.
\item \textbf{Lorentz Violation:} Does the absolute time postulate lead to observable Lorentz violation? SST predicts that the radiation sector has emergent Lorentz invariance, and any violation in the matter sector must be below current experimental bounds ($\delta c/c \lesssim 10^{-17}$).
\item \textbf{Proton Decay Selection Rules:} The knot-to-representation map may forbid certain proton decay channels based on topological conservation laws (e.g., net twist). Observing a forbidden channel would falsify this specific mapping.
\end{itemize}

\section{Glossary}
\begin{itemize}
\item \textbf{Absolute time (A-time):} The universal time parameter $t$ of the swirl condensate.
\item \textbf{Swirl Clock:} A local clock whose rate is slowed by local swirl intensity.
\item \textbf{R-phase vs. T-phase:} "Ring" phase (unknotted, wave-like) vs. "Torus-knot" phase (knotted, particle-like).
\item \textbf{Chirality:} Counter-clockwise (ccw) swirl corresponds to matter; clockwise (cw) corresponds to antimatter.
\end{itemize}

%================================================
% References
%================================================
\begin{thebibliography}{9}
    % Standard references would be included here.
\bibitem{Weinberg1967} S. Weinberg, \emph{A Model of Leptons}, Phys. Rev. Lett. 19, 1264 (1967).
\bibitem{PeskinSchroeder} M. E. Peskin and D. V. Schroeder, \emph{An Introduction to Quantum Field Theory}, Addison-Wesley (1995).
\bibitem{PDG2024} R.L. Workman et al. (Particle Data Group), \emph{Review of Particle Physics}, Prog. Theor. Exp. Phys. 2022, 083C01 (2022) and 2023 update.
\bibitem{ColemanWeinberg1973} S. Coleman and E. Weinberg, \emph{Radiative Corrections as the Origin of Spontaneous Symmetry Breaking}, Phys. Rev. D 7, 1888 (1973).
\bibitem{Witten1982} E. Witten, \emph{An SU(2) anomaly}, Phys. Lett. B 117, 324 (1982).
\end{thebibliography}

\end{document}
