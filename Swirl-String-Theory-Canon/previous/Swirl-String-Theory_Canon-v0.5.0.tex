%! Author = Omar Iskandarani
%! Title = Swirl String Theory (SST) Canon v0.5.0
%! Date = Oct 26, 2025
%! Affiliation = Independent Researcher, Groningen, The Netherlands
%! License = © 2025 Omar Iskandarani. All rights reserved. This manuscript is made available for academic reading and citation only. No republication, redistribution, or derivative works are permitted without explicit written permission from the author. Contact: info@omariskandarani.com
%! ORCID = 0009-0006-1686-3961
%! DOI = 10.5281/zenodo.17052966

\documentclass[11pt]{article}
\usepackage[margin=1in]{geometry}
\usepackage{amsmath,amssymb,amsfonts,amsthm}
\usepackage{graphicx}
\usepackage{tcolorbox}
\usepackage{booktabs}
\usepackage{hyperref}
\hypersetup{colorlinks=true, linkcolor=blue!60!black, citecolor=blue!60!black, urlcolor=blue!60!black}
\usepackage{physics}
\usepackage{siunitx}
\sisetup{per-mode=symbol}

%========================================================================================
% METADATA AND VERSIONING
%========================================================================================
\newcommand{\canonversion}{\textbf{v0.5.0}}
\newcommand{\papertitle}{Swirl String Theory (SST) Canon \canonversion}

%========================================================================================
% CUSTOM COMMANDS AND MACROS
%========================================================================================
% Swirl symbols
\newcommand{\vswirl}{\mathbf{v}_{\circ}}
\newcommand{\vnorm}{\lVert \vswirl \rVert}
\newcommand{\omegas}{\boldsymbol{\omega}_{\circ}}
\newcommand{\rc}{r_c}

% Densities
\newcommand{\rhof}{\rho_{\!f}}
\newcommand{\rhoE}{\rho_{\!E}}
\newcommand{\rhom}{\rho_{\!m}}
\newcommand{\rhocore}{\rho_{\mathrm{core}}}

% Forces and Constants
\newcommand{\FmaxEM}{F_{\mathrm{EM}}^{\max}}
\newcommand{\FmaxG}{F_{\mathrm{G}}^{\max}}
\newcommand{\Lam}{\Lambda}
\newcommand{\vscore}{v_{s}}

% Golden Ratio (Hyperbolic form)
\newcommand{\xig}{\operatorname{asinh}\!\left(\tfrac{1}{2}\right)}
\newcommand{\phig}{\exp(\xig)}

% Theorem-like environments
\newtheorem{axiom}{Axiom}
\newtheorem{theorem}{Theorem}[section]
\newtheorem{lemma}[theorem]{Lemma}
\newtheorem{corollary}[theorem]{Corollary}
\newtheorem{definition}{Definition}[section]
\newtheorem{postulate}{Postulate}

% Gauge sector macros
\newcommand{\GsA}{G^a_{\mu\nu}}
\newcommand{\WsI}{W^i_{\mu\nu}}
\newcommand{\Bmn}{B_{\mu\nu}}

%========================================================================================
% DOCUMENT START
%========================================================================================
\begin{document}

%========================================================================================
% TITLE PAGE
%========================================================================================
\title{\papertitle}
\author{Omar Iskandarani \\ \small{\textit{Independent Researcher, Groningen, The Netherlands}} \\ \small{\href{mailto:info@omariskandarani.com}{info@omariskandarani.com}} \\ \small{ORCID: \href{https://orcid.org/0009-0006-1686-3961}{0009-0006-1686-3961}}}
\date{October 26, 2025}
\maketitle

\begin{abstract}
This Canon is the single source of truth for Swirl String Theory (SST), version 0.5. It presents the complete axiomatic framework, definitive constants, master equations, and notational conventions of the theory. This version marks a major theoretical consolidation, elevating the topological-fluid-dynamic reinterpretation of the Standard Model gauge sector ($SU(3)\times SU(2)\times U(1)$) to full canonical status. This unification is achieved by establishing a formal homomorphism between the topological classes of swirl strings and the representation theory of elementary particles, with gauge fields and their interactions emerging as collective modes of the underlying swirl condensate. The Canon reaffirms the core SST axioms, including the postulate of a preferred foliation parametrized by absolute time, and systematizes the layered temporal ontology that reconciles this structure with observed relativistic phenomena. All terminology, symbols, and numerical constants have been standardized in accordance with the VAM-SST Rosetta, ensuring clarity, consistency, and reproducibility for all future work based on this framework.
\end{abstract}

\paragraph{Versioning} Semantic versions: vMAJOR.MINOR.PATCH. This file: \canonversion.\\
    Every paper or derivation must state the Canon version it depends on.

    \tableofcontents
    \newpage

%================================================
\section{Core Axioms and Foundational Principles}
%================================================
The formal system of Swirl String Theory is founded upon six core axioms. These axioms define the ontological substrate, the nature of physical entities, and the fundamental rules governing their dynamics. They are the immutable premises from which all canonical theorems of the theory are derived.

\begin{axiom}
Physics is formulated on a three-dimensional Euclidean space, $\mathbb{R}^{3}$, endowed with a preferred foliation parametrized by an absolute time coordinate. All physical dynamics occur within a frictionless, incompressible fluidic medium, termed the \emph{swirl condensate}. This condensate serves as a universal substrate, exhibiting Galilean symmetry in its own reference frame.
\end{axiom}

\begin{axiom}
All elementary particles and field quanta correspond to closed, knotted, or linked filaments of quantized vorticity within the swirl condensate, known as \emph{swirl strings}. The circulation of the swirl velocity, $\vswirl$, around any closed loop is quantized in integer multiples of a fundamental quantum, $\kappa$:
\[
    \Gamma = \oint \vswirl \cdot d\boldsymbol{\ell} = n\,\kappa, \qquad n\in\mathbb{Z}, \qquad \kappa = \frac{h}{m_{\text{eff}}}
\]
for an effective mass $m_{\text{eff}}$. The discrete quantum numbers of a particle—such as mass, charge, and spin—are direct manifestations of the topological invariants of its corresponding swirl string, including its knot class, linking number, writhe, and twist.
\end{axiom}

\begin{axiom}
Macroscopic gravitational attraction emerges as a collective phenomenon from coherent swirl flows and the pressure gradients they induce within the condensate. In the weak-field, low-velocity limit, the effective gravitational coupling constant, $G_{\text{swirl}}$, derived from the fundamental parameters of the condensate, is empirically equivalent to Newton's gravitational constant, $G_N$.
\end{axiom}

\begin{axiom}
The rate of local time evolution is dependent on the local intensity of the swirl. A clock comoving with the swirl flow at a tangential speed $||\mathbf{v}_{\phi}||$ experiences a slower rate of proper time passage relative to an observer in a region of quiescent condensate. This time dilation is governed by the Swirl Clock law: $S_t = \sqrt{1 - v^2/c^2}$. This principle is formalized through a layered temporal ontology that distinguishes between several distinct time parameters:
\begin{itemize}
\item \textbf{Absolute Time ($N$):} The global parameter of the preferred foliation, which serves as the universal, absolute time coordinate for the entire condensate.
\item \textbf{Observer Time ($T$):} The time measured by an external observer, typically situated far from strong swirl fields, represented by a scalar clock field $T(x)$.
\item \textbf{Swirl Clock ($S(t)$):} The internal phase accumulator of a swirl string, which tracks the cyclical evolution of its internal rotational dynamics.
\item \textbf{Kairos Event ($K$):} A discrete, non-continuous moment corresponding to a topological transition, such as a swirl string reconnection or a change in knot type. This represents the physical mechanism of quantum measurement or particle decay.
\end{itemize}
\end{axiom}

\begin{axiom}
Every swirl string can exist in one of two limiting phases. The \textbf{R-phase} ("Ring") is an extended, unknotted, or delocalized configuration of circulation that exhibits wave-like phenomena such as interference and diffraction. The \textbf{T-phase} ("Torus-knot") is a localized, knotted soliton that carries rest-mass and behaves as a discrete particle. Quantum measurement is the physical process of an interaction-driven transition between these phases: $R \to T$ (objective collapse) or $T \to R$ (de-localization), mediated by swirl radiation.
\end{axiom}

\begin{axiom}
The particle spectrum of the Standard Model is a direct consequence of the classification of swirl string topologies. Unknotted excitations behave as bosonic modes (e.g., gauge bosons). Chiral hyperbolic knots map to quarks, while torus knots map to leptons. Composite structures, such as linked knots, correspond to bound states like nuclei and molecules.
\end{axiom}

%================================================
\section{Canonical Constants and Densities}
%================================================
This section provides the single source of truth for all calibrated numerical values and defining relations within SST.

\subsection{Defining Relations for Densities}
    The theory distinguishes between several forms of density, related by canonical equalities. For a swirl string with a characteristic tangential velocity $\vnorm$, these are defined as:
    \begin{align}
    \rhoE &\equiv \tfrac{1}{2}\,\rhof\,\vnorm^2 \quad (\text{swirl energy density}) \\
    \rhom &\equiv \frac{\rhoE}{c^2} \quad (\text{mass-equivalent density})
    \end{align}
    The relationship between the microscopic core density ($\rhocore$) and the macroscopic effective fluid density ($\rhof$) is established through a coarse-graining coefficient, $K$, which depends on the core radius $\rc$ and the characteristic swirl speed $\vscore$ (tangential speed at $r=\rc$). The effective fluid density is then proportional to a coarse-grained angular rate of the swirl-string ensemble, $\Omega$:
    \begin{align}
    K &\equiv \frac{\rhocore\,\rc}{\vscore} \\
    \rhof &= K\,\Omega
    \end{align}

\subsection{Table of Canonical Constants (v0.5.0)}
    The following table lists the definitive numerical values for the primary constants of SST.
    \begin{table}[h!]
    \centering
    \caption{Primary SST Constants (SI units unless noted)}
    \begin{tabular}{@{}lll@{}}
    \toprule
    \textbf{Quantity} & \textbf{Symbol} & \textbf{Value} \\
    \midrule
    Characteristic Swirl Speed (core) & $\vscore$ & \num{1.09384563e6} \si{m/s} \\
    Swirl String Core Radius & $\rc$ & \num{1.40897017e-15} \si{m} \\
    Swirl String Core Density & $\rhocore$ & \num{3.8934358266918687e18} \si{kg/m^3} \\
    Background Effective Fluid Density & $\rhof^{\text{bg}}$ & \num{7.0e-7} \si{kg/m^3} \\
    Maximum EM-like Force & $\FmaxEM$ & \num{2.9053507e1} \si{N} \\
    Maximum Universal Force & $\FmaxG$ & \num{3.02563e43} \si{N} \\
    Speed of Light in Vacuum & $c$ & \num{299792458} \si{m/s} \\
    Fine-Structure Constant & $\alpha$ & \num{7.2973525693e-3} \\
    Planck Time & $t_p$ & \num{5.391247e-44} \si{s} \\
    Golden Constant (Hyperbolic) & $\phig$ & $\exp(\operatorname{asinh}(1/2))$ \\
    \bottomrule
    \end{tabular}
    \end{table}

%================================================
\section{Emergent Gauge Theory and Fundamental Interactions}
%================================================
With this version, the reinterpretation of the Standard Model gauge sector is elevated to full canonical status. Gauge interactions are not fundamental but are derived from the elastic and topological properties of the swirl condensate.

\subsection{The Knot-Representation Homomorphism}
    The correspondence between swirl string topology and particle identity is formalized through a canonical monoid homomorphism, $t$, from the set of knot classes, $\mathcal{K}$, to the representation ring of the Standard Model gauge group, $Rep(SU(3)\times SU(2)\times U(1))$.
    \
    This map assigns a specific gauge representation to each swirl string based on a set of computable topological indices.
    \begin{definition}
    For any oriented framed knot $K$ with color-sign $s_3 \in \{+1, 0, -1\}$, doublet indicator $d_2 \in \{0, 1\}$, and twist sign $\tau \in \{-1, 0, +1\}$, the hypercharge is defined as:
    \
    This formula reproduces the known electric charges for all Standard Model particles.
    \end{definition}

    \begin{theorem}[Per-Generation Anomaly Cancellation]
    For the canonical set of swirl string topologies corresponding to one complete generation of Standard Model fermions, as defined by the homomorphism $t$, all gauge ($SU(3)^2U(1)$, $SU(2)^2U(1)$, $U(1)^3$) and mixed gravitational ($grav^2U(1)$) anomalies are identically cancelled.
    \end{theorem}

\subsection{Emergent Yang-Mills Fields from Swirl Directors}
    The kinetic terms of the Yang-Mills Lagrangian arise from the "stiffness" of the swirl condensate. The local orientation of the swirl can be described by a set of director fields, collected into multi-director order parameters, $\mathcal{U}_3(x) \in SU(3)$ and $\mathcal{U}_2(x) \in SU(2)$.

    \begin{theorem}
    Let $U_3(x) \in SU(3)$, $U_2(x) \in SU(2)$, and $\vartheta(x) \in \mathbb{R}$ be swirl director fields. Define the connections:
    \
    With a director elasticity Lagrangian of the principal chiral form:
    \
    coarse-graining yields the effective Yang-Mills Lagrangian:
    \
    This establishes a canonical relationship between the observable gauge coupling constants ($g_i$) and the microscopic stiffness properties ($\kappa_i$) of the swirl condensate.
    \end{theorem}

\subsection{Canonical Couplings and Electroweak Symmetry Breaking}
    The gauge sector is now fully specified by canonical relations derived from the swirl condensate's properties, without free parameters.

    \paragraph{Canonical Scales and Couplings.}
        Define the dimensionless core swirl modulus $\Sigma_{\text{core}}$ and the canonical renormalization point $\mu_*$:
        \
        The gauge couplings at this scale are determined by topological weights $W_i$ (computable from knot invariants for each particle family) and group-geometric coefficients $\kappa_i$:
        \
        The running of the couplings follows the standard Renormalization Group equations.

    \paragraph{Electroweak Symmetry Breaking.}
        The electroweak symmetry breaking scale $v_\Phi$ is a collective phenomenon governed by the bulk swirl energy density, not a UV core length. The canonical density-topology law is:
        \
        Using the canonical constants and the topological weights for one SM generation ($W_3=2, W_2=2, W_1=10/3$), this yields a parameter-free prediction:
        \[
            v_\Phi^{\text{pred}} = (4.1877 \times 10^5~\mathrm{J/m^3})^{1/4} \times (\tfrac{40}{3})^{1/4} \approx 135.8~\mathrm{GeV} \times 1.912 \approx 259.5~\mathrm{GeV}
        \]
        This value is within 5.4\% of the empirical value ($246.22~\mathrm{GeV}$), with the residual expected to be accounted for by higher-order corrections. The standard mass relations for the electroweak bosons remain canonical:
        \

%================================================
\section{Master Equations of Swirl String Theory}
%================================================
This section presents the definitive, boxed master equations of Swirl String Theory in their canonical v0.5 form.

\subsection{The Unified SST Lagrangian ($\mathcal{L}_{\text{SST+Gauge}}$)}
    The complete dynamics of the swirl condensate and its emergent gauge fields are derived from the following unified Lagrangian density, which has units of energy density ($J/m^3$):
    \begin{equation}
    \boxed{
        \mathcal{L}_{\text{SST+Gauge}} = \underbrace{\frac{1}{2}\rhof\,\|\vswirl\|^2 + \lambda(\nabla\cdot\vswirl)}_{\text{Swirl Kinematics}} - \underbrace{\rhof\,\Phi_{\text{swirl}}(\mathbf{r},\omegas)}_{\text{Swirl Potential}} + \underbrace{\chi_h\,\rhof\,(\vswirl\cdot\omegas)}_{\text{Helicity Coupling}} + \underbrace{\mathcal{L}_{\text{YM}} + (D_\mu\Phi)^\dagger (D^\mu\Phi) - V(\Phi)}_{\text{Emergent Gauge \& Scalar Sector}}
    }
    \end{equation}

\subsection{The Topology-Driven Mass Law}
    The rest mass of a particle is determined by the energy stored in its corresponding swirl string. For a torus knot $T(p,q)$ with braid index $b$, Seifert genus $g$, number of components $n$, and dimensionless ropelength $\mathcal{L}_{\text{tot}}$, the invariant mass is:
    \begin{equation}
    \boxed{
        M(T(p,q)) = \left(\frac{4}{\alpha}\right) b^{-3/2}\,\phig^{-g}\,n^{-1/\phig} \left(\frac{1}{2}\rhocore\vscore^2\right) \frac{\pi\,\rc^3\,\mathcal{L}_{\text{tot}}(T)}{c^2}
    }
    \end{equation}

\subsection{The Swirl-Gravity Coupling Constant ($G_{\text{swirl}}$)}
    The effective Newtonian gravitational constant is derived from the primary constants of the theory:
    \begin{equation}
    \boxed{
        G_{\text{swirl}} = \frac{\vscore\,c^5\,t_p^2}{2\,\FmaxEM\,\rc^2}
    }
    \end{equation}
    Numerically, this evaluates to $G_{\text{swirl}} \approx 6.674 \times 10^{-11} \si{m^3.kg^{-1}.s^{-2}}$, matching the empirical value of $G_N$.

\subsection{The Swirl Clock Law (Local Time Dilation)}
    The rate of passage of local proper time ($dt_{\text{local}}$) for an observer comoving with a swirl flow is slowed relative to the absolute time of a distant observer ($dt_{\infty}$):
    \begin{equation}
    \boxed{
        \frac{dt_{\text{local}}}{dt_{\infty}} = \sqrt{1-\frac{||\mathbf{v}_{\phi}||^{2}}{c^{2}}}
    }
    \end{equation}

\subsection{The Swirl Pressure Law (Euler-SST Radial Balance)}
    For a steady, purely azimuthal swirl flow, the radial pressure gradient within the swirl condensate provides the necessary centripetal force. This canonical theorem relates the swirl pressure, $p_{\text{swirl}}$, to the local swirl velocity profile, $v_{\theta}(r)$:
    \begin{equation}
    \boxed{
        \frac{1}{\rhof}\frac{dp_{\text{swirl}}}{dr} = \frac{v_{\theta}(r)^2}{r}
    }
    \end{equation}

%================================================
\section{Advanced Canonical Doctrines}
%================================================
This section formalizes core interpretative principles of SST crucial for understanding quantum phenomena.

\subsection{Wave-Particle Duality and Objective Collapse}
    The dual nature of matter is a direct consequence of Axiom 5.
    \begin{itemize}
    \item \textbf{Duality:} The \textbf{R-phase} of a swirl string is a delocalized, wave-like state of circulation, which is the physical entity described by the Schrödinger equation. The \textbf{T-phase} is a compact, knotted, soliton-like state that corresponds to a localized particle.
    \item \textbf{Objective Collapse:} Quantum measurement is a physical process of objective collapse. An interaction between a swirl string in its R-phase and a macroscopic system triggers a deterministic but practically unpredictable instability, forcing a rapid transition to a stable, localized T-phase. This topological transition is a \textbf{Kairos Event} ($K$), which constitutes the "collapse of the wavefunction."
    \end{itemize}

\subsection{Quantum Entanglement as Topological Linkage}
    Quantum entanglement is explained as a direct consequence of topological conservation laws in the swirl condensate.
    \begin{itemize}
    \item \textbf{Mechanism:} An entangled state of two or more swirl strings corresponds to a single, unified fluid dynamic configuration where the constituent strings are topologically linked.
    \item \textbf{Non-Locality:} The correlation is not mediated by a signal but is an inherent property of the single, non-separable topological object. A measurement on one string imposes a constraint on the entire linked structure to conserve a global topological invariant (e.g., total linking number), instantaneously determining the state of the other string.
    \end{itemize}

%================================================
    \appendix
\section{Canon Governance and Versioning}
%================================================
\subsection{Formal System and Canonicality}
    The SST formal system is denoted $\mathcal{S}=(\mathcal{P},\mathcal{D},\mathcal{R})$, comprising the core axioms ($\mathcal{P}$), definitions ($\mathcal{D}$), and admissible inference rules ($\mathcal{R}$). A statement is \textbf{canonical} if it is a theorem or identity provable within $\mathcal{S}$. A statement is \textbf{empirical} if it asserts a measured value used to calibrate a symbol.

\subsection{Status Taxonomy}
    \begin{itemize}
    \item \textbf{Axiom/Postulate (Canonical):} A primitive assumption of SST.
    \item \textbf{Definition (Canonical):} Introduces a symbol by construction.
    \item \textbf{Theorem/Corollary (Canonical):} A proven consequence of axioms and definitions.
    \item \textbf{Calibration (Empirical):} Recommended numerical values for canonical symbols.
    \item \textbf{Research Track (Non-canonical):} Conjectures or alternatives pending proof or axiomatization.
    \end{itemize}

\subsection{Version History}
    \begin{table}[h!]
    \centering
    \begin{tabular}{@{}lll@{}}
    \toprule
    \textbf{Version} & \textbf{Date} & \textbf{Key Changes} \\
    \midrule
    0.4.1 & Sep 9, 2025 & Initial public consolidation. Gauge sector listed as Research Track. \\
    0.5.0 & Oct 26, 2025 & \textbf{Major Update:} Promotion of the gauge sector to full canonical status. \\
    & & Integration of layered temporal ontology. Formalization of advanced doctrines. \\
    \bottomrule
    \end{tabular}
    \end{table}

%================================================
\section{VAM-to-SST Concordance}
%================================================
This appendix provides a concordance for researchers familiar with the legacy Vortex Æther Model (VAM) literature.
\begin{table}[h!]
\centering
\caption{Terminology and Symbol Concordance}
\begin{tabular}{@{}lll@{}}
\toprule
\textbf{VAM (Legacy)} & \textbf{SST (House Style)} & \textbf{Meaning} \\
\midrule
æther, vortex æther & swirl condensate, medium & The fundamental fluidic substrate. \\
æther time & absolute time, foliation time ($N$) & The global time parameter. \\
vortex line/filament & swirl string & The topological excitations. \\
$\rho_x^{(\text{fluid})}$ & $\rhof$ & Effective fluid density. \\
$\rho_c, \rho^{(\text{mass})}$ & $\rhocore$ & Core density of a swirl string. \\
$C_c$ (tangential speed) & $\vscore$ & Characteristic swirl speed at core radius. \\
\bottomrule
\end{tabular}
\end{table}

%================================================
\section{Non-Canonical Research Extensions (Canon 5R)}
%================================================
This appendix serves as the designated repository for all speculative, unproven, or alternative ideas within the SST framework.

\subsection{Open Research Problems}
    \begin{itemize}
    \item \textbf{Derivation of the Born Rule:} A rigorous mathematical derivation of the Born rule from the ergodic dynamics of the swirl condensate is an open task.
    \item \textbf{No-Signaling Proof for Entanglement:} A formal proof is required to show that the topological linkage model of entanglement does not permit superluminal signaling.
    \item \textbf{Explicit Vertex Catalogue for Interactions:} The derivation of a complete catalogue of interaction vertices, analogous to Feynman diagrams, is a major ongoing research program.
    \end{itemize}

\subsection{Conjectural Relations}
    The following relations are dimensionally consistent and theoretically motivated but are not yet anchored to empirical data or derived from first principles.
    \begin{itemize}
    \item \textbf{Blackbody Swirl Temperature:} The potential connection between thermodynamic temperature and the statistical properties of a "gas" of swirl radiation modes.
    \item \textbf{Swirl-Helicity Chern-Simons Couplings:} The possibility of adding explicit Chern-Simons-like terms to the Unified Lagrangian to better capture parity-violating effects.
    \item \textbf{Quantitative Prediction of Coupling Constants:} A first-principles calculation of the condensate stiffness constants $\kappa_i$ from the fundamental SST parameters remains an unsolved problem.
    \end{itemize}

%================================================
% REFERENCES
%================================================
    \begin{thebibliography}{9}
    \bibitem{Moffatt1969} H. K. Moffatt. The degree of knottedness of tangled vortex lines. \textit{Journal of Fluid Mechanics}, 35(1):117-129, 1969.
    \bibitem{Kelvin1869} W. Thomson (Lord Kelvin). On vortex motion. \textit{Transactions of the Royal Society of Edinburgh}, 25:217-260, 1869.
    \bibitem{Batchelor1967} G. K. Batchelor. \textit{An Introduction to Fluid Dynamics}. Cambridge University Press, 1967.
    \bibitem{Finkelstein1968} D. Finkelstein and J. Rubinstein. Connection between spin, statistics, and kinks. \textit{Journal of Mathematical Physics}, 9(11):1762-1779, 1968.
    \bibitem{Jackson1999} J. D. Jackson. \textit{Classical Electrodynamics}. Wiley, 3rd edition, 1999.
    \bibitem{PeskinSchroeder1995} M. E. Peskin and D. V. Schroeder. \textit{An Introduction to Quantum Field Theory}. Westview, 1995.
    \bibitem{Volovik2003} G. E. Volovik. \textit{The Universe in a Helium Droplet}. Oxford, 2003.
    \bibitem{Weinberg1967} S. Weinberg. A model of leptons. \textit{Physical Review Letters}, 19:1264-1266, 1967.
    \bibitem{Iskandarani2025a} O. Iskandarani. \textit{Swirl String Theory (SST) Canon v0.4.1}. Zenodo, 2025.
    \bibitem{Iskandarani2025b} O. Iskandarani. \textit{SST Rosetta: VAM-to-SST Translation Guide}. Zenodo, 2025.
    \bibitem{Iskandarani2025c} O. Iskandarani. \textit{Quantum Mechanics and Quantum Gravity in the Vortex Æther Model}. Zenodo, 2025.
    \end{thebibliography}

\end{document}
