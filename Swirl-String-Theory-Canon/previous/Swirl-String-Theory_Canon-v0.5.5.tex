%! Author = Omar Iskandarani
%! Title = Swirl String Theory (SST) Canon v0.5.5
%! Date = Sept 9, 2025
%! Affiliation = Independent Researcher, Groningen, The Netherlands
%! License = © 2025 Omar Iskandarani. All rights reserved. This manuscript is made available for academic reading and citation only. No republication, redistribution, or derivative works are permitted without explicit written permission from the author. Contact: info@omariskandarani.com
%! ORCID = 0009-0006-1686-3961
%! DOI = 10.5281/zenodo.17052966 % Placeholder DOI, update upon finalization

\newcommand{\canonversion}{\textbf{v0.5.5}} % Semantic versioning: vMAJOR.MINOR.PATCH
\newcommand{\papertitle}{Swirl String Theory (SST) Canon \canonversion}
\newcommand{\paperdoi}{10.5281/zenodo.17052966}



%========================================================================================
% PACKAGES AND DOCUMENT CONFIGURATION
%========================================================================================
\documentclass[11pt]{article}
\usepackage{subfiles}
% sststyle.sty
\NeedsTeXFormat{LaTeX2e}
\ProvidesPackage{sststyle}[2025/07/01 SST unified style]



% === Draft Options ===
\newif\ifsstdraft
% \sstdrafttrue
\ifsstdraft
\RequirePackage{showframe}
\fi

% === Load Once ===
\RequirePackage{ifthen}
\newboolean{sststyleloaded}
\ifthenelse{\boolean{sststyleloaded}}{}{\setboolean{sststyleloaded}{true}

% === Page ===
\RequirePackage[a4paper, margin=2.5cm]{geometry}

% === Fonts ===
\RequirePackage[T1]{fontenc}
\RequirePackage[utf8]{inputenc}
\RequirePackage[english]{babel}
\RequirePackage{textgreek}
\RequirePackage{mathpazo}
\RequirePackage[scaled=0.95]{inconsolata}
\RequirePackage{helvet}

% === Math ===
\RequirePackage{amsmath, amssymb, mathrsfs, physics}
\RequirePackage{siunitx}
\sisetup{per-mode=symbol}

% === Tables ===
\RequirePackage{graphicx, float, booktabs}
\RequirePackage{array, tabularx, multirow, makecell}
\newcolumntype{Y}{>{\centering\arraybackslash}X}
\newenvironment{tighttable}[1][]{\begin{table}[H]\centering\renewcommand{\arraystretch}{1.3}\begin{tabularx}{\textwidth}{#1}}{\end{tabularx}\end{table}}
\RequirePackage{etoolbox}
\newcommand{\fitbox}[2][\linewidth]{\makebox[#1]{\resizebox{#1}{!}{#2}}}

% === Graphics ===
\RequirePackage{tikz}
\usetikzlibrary{3d, calc, arrows.meta, positioning}
\RequirePackage{pgfplots}
\pgfplotsset{compat=1.18}
\RequirePackage{xcolor}

% === Code ===
\RequirePackage{listings}
\lstset{basicstyle=\ttfamily\footnotesize, breaklines=true}

% === Theorems ===
\newtheorem{theorem}{Theorem}[section]
\newtheorem{lemma}[theorem]{Lemma}

% === TOC ===
\RequirePackage{tocloft}
\setcounter{tocdepth}{2}
\renewcommand{\cftsecfont}{\bfseries}
\renewcommand{\cftsubsecfont}{\itshape}
\renewcommand{\cftsecleader}{\cftdotfill{.}}
\renewcommand{\contentsname}{\centering \Huge\textbf{Contents}}

% === Sections ===
\RequirePackage{sectsty}
\sectionfont{\Large\bfseries\sffamily}
\subsectionfont{\large\bfseries\sffamily}

% === Bibliography ===


% === Links ===
\RequirePackage{hyperref}
\hypersetup{
    colorlinks=true,
    linkcolor=blue,
    citecolor=blue,
    urlcolor=blue,
    pdftitle={The Vortex \AE ther Model},
    pdfauthor={Omar Iskandarani},
    pdfkeywords={vorticity, gravity, \ae ther, fluid dynamics, time dilation, SST}
}
\urlstyle{same}
\RequirePackage{bookmark}

% === Misc ===
\RequirePackage[none]{hyphenat}
\sloppy
\RequirePackage{empheq}
\RequirePackage[most]{tcolorbox}
\newtcolorbox{eqbox}{colback=blue!5!white, colframe=blue!75!black, boxrule=0.6pt, arc=4pt, left=6pt, right=6pt, top=4pt, bottom=4pt}
\RequirePackage{titling}
\RequirePackage{amsfonts}
\RequirePackage{titlesec}
\RequirePackage{enumitem}

\AtBeginDocument{\RenewCommandCopy\qty\SI}

\pretitle{\begin{center}\LARGE\bfseries}
\posttitle{\par\end{center}\vskip 0.5em}
\preauthor{\begin{center}\large}
\postauthor{\end{center}}
\predate{\begin{center}\small}
\postdate{\end{center}}


\endinput
} % Assuming a local style file
% sstappendixsetup.sty

\newcommand{\titlepageOpen}{
  \begin{titlepage}
  \thispagestyle{empty}
  \centering
  \ifdefined\standalonechapter
  {\Large\bfseries \appendixtitle \par}
  \else
  {\Large\bfseries \papertitle \par}
    \fi
  \vspace{1cm}
  {\Large\itshape \textbf{Omar Iskandarani}\textsuperscript{\textbf{*}} \par}
  \vspace{0.5cm}
  {\today \par}
  \vspace{0.5cm}
}

% here comes abstract
\newcommand{\titlepageClose}{
  \vfill
  \raggedright % <-- fixes left alignment
  \null
  \begin{picture}(0,0)
  % Adjust position: (x,y) = (left, bottom)
  \put(0,-45){  % Shift 200pt left, 40pt down
    \begin{minipage}[b]{0.7\textwidth}
    \footnotesize % One step bigger than \tiny \scriptsize
    \renewcommand{\arraystretch}{1.0}
    \noindent\rule{\textwidth}{0.4pt} \\[0.5em]  % ← horizontal line
    \textsuperscript{\textbf{*}} Independent Researcher, Groningen, The Netherlands \\
    Email: \texttt{info@omariskandarani.com} \\
    ORCID: \texttt{\href{https://orcid.org/0009-0006-1686-3961}{0009-0006-1686-3961}} \\
    DOI: \href{https://doi.org/\paperdoi}{\paperdoi} \\
    License: CC-BY-NC 4.0 International \\
    \end{minipage}
  }
  \end{picture}
  \end{titlepage}
} % Assuming a local appendix setup
\usepackage[margin=1in]{geometry}
\usepackage{amsmath,amssymb,amsfonts,amsthm}
\usepackage{tcolorbox}
\usetikzlibrary{knots,intersections,decorations.pathreplacing,3d,calc,arrows.meta,positioning,decorations.pathmorphing}
\usepackage{pgfmath}
\usepackage{pgfplots}
\pgfplotsset{compat=1.18}
\usepackage{ulem}


% ==== Packages ====
\usepackage[T1]{fontenc}
\usepackage{lmodern}
\usepackage{microtype}

\geometry{margin=1in}
\usepackage{ bm, mathtools}
\usepackage{siunitx}
\sisetup{per-mode=symbol,round-mode=figures,round-precision=6}
\usepackage{physics}
\usepackage{upgreek}
\usepackage{graphicx}
\usepackage{booktabs}
\usepackage{hyperref}
\hypersetup{colorlinks=true, linkcolor=blue!60!black, citecolor=blue!60!black, urlcolor=blue!60!black}


% ===== Gauge sector macros =====
\newcommand{\Tr}{\mathrm{Tr}}
\newcommand{\ii}{\mathrm{i}}
% Gauge fields (adjoints; indices a=1..8, i=1..3)
\newcommand{\GsA}{G^a_{\mu\nu}}
\newcommand{\WsI}{W^i_{\mu\nu}}
\newcommand{\Bmn}{B_{\mu\nu}}



% ===============================
% Macros (canonicalized)
% ===============================

% swirl arrows (context-aware)
\newcommand{\swirlarrow}{%
	\mathchoice{\mkern-2mu\scriptstyle\boldsymbol{\circlearrowleft}}%
	{\mkern-2mu\scriptstyle\boldsymbol{\circlearrowleft}}%
	{\mkern-2mu\scriptscriptstyle\boldsymbol{\circlearrowleft}}%
	{\mkern-2mu\scriptscriptstyle\boldsymbol{\circlearrowleft}}%
}
\newcommand{\swirlarrowcw}{%
	\mathchoice{\mkern-2mu\scriptstyle\boldsymbol{\circlearrowright}}%
	{\mkern-2mu\scriptstyle\boldsymbol{\circlearrowright}}%
	{\mkern-2mu\scriptscriptstyle\boldsymbol{\circlearrowright}}%
	{\mkern-2mu\scriptscriptstyle\boldsymbol{\circlearrowright}}%
}


% Canonical symbols
\newcommand{\vswirl}{\mathbf{v}_{\swirlarrow}}
\newcommand{\vswirlcw}{\mathbf{v}_{\swirlarrowcw}}
\newcommand{\SwirlClock}{S_{(t)}^{\swirlarrow}}
\newcommand{\SwirlClockcw}{S_{(t)}^{\swirlarrowcw}}
\newcommand{\omegas}{\boldsymbol{\omega}_{\swirlarrow}}  % swirl vorticity
\newcommand{\vscore}{v_{\swirlarrow}}                    % shorthand: |v_swirl| at r=r_c
\newcommand{\vnorm}{\lVert \vswirl \rVert}               % swirl speed magnitude
\newcommand{\rhof}{\rho_{\!f}}                           % effective fluid density
\newcommand{\rhoE}{\rho_{\!E}}                           % swirl energy density
\newcommand{\rhom}{\rho_{\!m}}                           % mass-equivalent density
\newcommand{\rc}{r_c}                                    % string core radius (swirl string radius)
\newcommand{\FmaxEM}{F_{\mathrm{EM}}^{\max}}             % EM-like maximal force scale
\newcommand{\FmaxG}{F_{\mathrm{G}}^{\max}}               % G-like maximal force scale
\newcommand{\Lam}{\Lambda}                               % Swirl Coulomb constant
\newcommand{\Om}{\Omega_{\swirlarrow}}                   % swirl angular frequency profile
\newcommand{\alpg}{\alpha_g}                             % gravitational fine-structure analogue

% Policy: the golden constant is only allowed via hyperbolic functions.
\newcommand{\xig}{\operatorname{asinh}\!\left(\tfrac{1}{2}\right)}
\newcommand{\phig}{\exp(\xig)}
\newcommand{\phialg}{\bigl(1+\sqrt{5}\bigr)/2}
\newcommand{\xigold}{\tfrac{3}{2}\,\xig}
\newcommand{\GoldenDeclare}{%
	\textbf{Golden (hyperbolic)}:\ \(\ln\phi=\xig\), hence \(\phi=\phig\).
	\ \emph{(Equivalently, \(\phi=\phialg\); the algebraic form is derivative.)}%
}

% Theorem-like environments
\newtheorem{identity}{Identity}
\newtheorem{axiom}{Axiom}
\newtheorem{theorem}{Theorem}[section]
\newtheorem{lemma}[theorem]{Lemma}
\newtheorem{corollary}[theorem]{Corollary}
\newtheorem{definition}{Definition}[section]


%========================================================================================
% DOCUMENT START
%========================================================================================
\begin{document}

%========================================================================================
% TITLE PAGE
%========================================================================================

	\titlepageOpen

	\begin{abstract}
		This Canon is the single source of truth for \emph{Swirl String Theory (SST)}: all definitions, axioms, theorems, and empirical calibrations are presented with rigorous, axiomatic clarity. It unifies the core hydrodynamic, electromagnetic, and topological principles of the theory. This version (v0.6.0) updates the formal system by integrating pedagogical derivations from earlier versions while preserving the rigor of v0.5.5. In particular, it canonizes the following principles:

		\textbf{I. Foundational swirl invariants and quantization:} The Chronos--Kelvin circulation invariant and Swirl Quantization Principle unify classical vortex laws with quantum topology. The \emph{swirl Coulomb constant} $\Lambda$ is defined for the hydrogenic swirl potential, yielding the correct Bohr spectrum \cite{Batchelor1967,Saffman1992,Iskandarani2025Canon034,Iskandarani2025Hydrogen}.

		\textbf{II. Swirl--Electromagnetic unification:} Electromagnetism emerges as collective swirl dynamics. A changing swirl-string density produces an electromotive force (Faraday-law term), and \emph{photons} are identified as delocalized, unknotted swirl oscillations whose wave dynamics exactly reproduce Maxwell’s equations \cite{Jackson1999,Iskandarani2025MagneticVector,Iskandarani2025DoubleSlit}.

		\textbf{III. Standard Model from topology:} The $SU(3)\times SU(2)\times U(1)$ gauge structure and particle spectrum emerge from swirl-string director fields and knot topology. The \emph{weak mixing angle} $\theta_W$ and electroweak scale are derived from first principles (director stiffness ratios and bulk swirl energy), with a parameter-free prediction of the Higgs scale ($\sim 259~\si{GeV}$) \cite{Weinberg1967,PeskinSchroeder1995,Iskandarani2025Canon034}.

		\textbf{IV. Swirl gravitation mechanism:} Long-range gravitational attraction arises as a topological pressure effect in a flat medium. Chiral knotted strings (e.g., two hydrogen molecules modeled as trefoil knots) connected by a common swirl axis experience a persistent swirl clock gradient and mutual attraction, explaining molecular gravity without spacetime curvature (\emph{Hydrogen-Gravity theorem}) \cite{Iskandarani2025Hydrogen}.

		\textbf{V. Quantum measurement as dynamics:} Wave--particle duality is handled via two phases of swirl strings (extended R-phase vs.\ knotted T-phase). The R$\to$T (wave collapse) transition is governed by a \emph{dynamical rate law} proportional to ambient energy density, replacing an ad hoc wavefunction collapse postulate with a canonical kinetic rule \cite{Zurek2003,Iskandarani2025DoubleSlit}.

		\medskip
		\noindent\textbf{Core Axioms (SST).}
		\begin{enumerate}
			\item \textbf{Swirl Medium:} Physics is formulated on $\mathbb{R}^3$ with an absolute time. All dynamics occur in a frictionless, incompressible \emph{swirl condensate} (universal substrate).
			\item \textbf{Swirl Strings (Circulation \& Topology):} Particles and field quanta correspond to closed vortex filaments (``swirl strings'') in the condensate. The circulation around any closed loop is quantized:
			\[
				\Gamma = \oint_C \mathbf{v}_{\!\!\;\circ} \cdot d\boldsymbol{\ell} = n\,\kappa,\quad n\in\mathbb{Z},\quad \kappa=\frac{h}{m_{\rm eff}}\,,
			\]
			and discrete quantum numbers correspond to topological invariants (linking, twist, writhe) \cite{Onsager1949,Feynman1955,Iskandarani2025Canon034}.
			\item \textbf{String-Induced Gravitation:} Macroscopic attraction (gravity) emerges from coherent swirl flows and pressure gradients in the condensate. In the Newtonian limit, the effective coupling $G_{\rm swirl}$ is fixed by canonical constants such that $G_{\rm swirl}\approx G_{\rm N}$.
			\item \textbf{Swirl Clocks (Local Time):} A local proper-time rate depends on tangential swirl speed, with factor $S_t=\sqrt{1-v^2/c^2}$.
			\item \textbf{Dual Phases (Wave--Particle):} Each swirl string has two limiting phases: an extended \emph{R-phase} (unknotted, delocalized circulation) and a localized \emph{T-phase} (knotted). Quantum measurement corresponds to a dynamic transition between these phases.
			\item \textbf{Taxonomy (Particle--Knot Mapping):} Unknotted excitations correspond to bosonic fields; chiral hyperbolic knots correspond to quarks; torus knots correspond to leptons; linked composite knots represent bound states.
		\end{enumerate}
	\end{abstract}

	\titlepageClose

	\section{Canon Governance and Formal System}

	\textbf{Canonical Formal System.} SST is formulated as a formal system $S = (P, D, R)$ comprising axioms $P$, definitions $D$, and inference rules $R$. A statement is \emph{canonical} if it is derivable within $S$ and consistent with all prior canonical statements. The hierarchy:
	\begin{itemize}
		\item \textbf{Axiom (Canonical):} primitive assumption.
		\item \textbf{Definition (Canonical):} introduction of a symbol/concept (e.g.\ $\Lambda$).
		\item \textbf{Theorem/Corollary (Canonical):} proved within $S$ from axioms and theorems.
		\item \textbf{Calibration (Empirical):} numerical value for a canonical constant (not used as a premise).
		\item \textbf{Research Track (Non-canonical):} conjecture/extension not yet proven.
	\end{itemize}
	All canonical developments below adhere strictly to these rules. Derivations, dimensional analyses, and pedagogical explanations are deferred to appendices for clarity (cf.\ v0.3.4) \cite{Iskandarani2025Canon034}.

	\section{Classical Invariants and Swirl Quantization}

	Under Axiom~1 (inviscid incompressible medium), Euler’s equations yield standard vortex invariants:
	\begin{itemize}
		\item \textbf{Kelvin’s circulation theorem:} $\frac{d\Gamma}{dt}=0$ for any material loop co-moving with the fluid \cite{Batchelor1967,Saffman1992}.
		\item \textbf{Vorticity transport (Helmholtz law):} $\partial_t \boldsymbol{\omega}_{\!\!\;\circ} = \nabla \times (\mathbf{v}_{\!\!\;\circ}\times \boldsymbol{\omega}_{\!\!\;\circ})$ (vortex lines move with the flow) \cite{Helmholtz1858,Kelvin1869}.
		\item \textbf{Helicity conservation:} $H=\int \mathbf{v}_{\!\!\;\circ}\cdot\boldsymbol{\omega}_{\!\!\;\circ}\,dV$ is materially invariant up to reconnections \cite{Moffatt1969}.
	\end{itemize}

	\begin{axiom}[Chronos--Kelvin Invariant]\label{ax:CK}
	For any thin, closed swirl loop (core radius $R(t)$) carried with the flow (no reconnections or sources),
	\[
		\frac{D}{Dt}\Big(R^2\,\omega\Big)=0 \quad\Longleftrightarrow\quad
		\frac{D}{Dt}\Big(\frac{c}{r_c}\,R^2\sqrt{\,1 - S_t^2\,}\Big)=0,\quad S_t=\sqrt{1-(\omega r_c/c)^2}\,.
	\]
	This encapsulates Kelvin’s theorem with relativistic time dilation due to swirl motion (the ``swirl clock'' effect) \cite{Batchelor1967,Saffman1992,Iskandarani2025Canon034}.
	\end{axiom}

	\subsection{Swirl Quantization Principle}

	\textbf{Swirl Quantization Principle.} The joint discreteness of circulation and topology is the origin of quantum behavior in SST. A swirl string’s circulation $\Gamma$ is quantized in units of $\kappa$, and its allowed configurations are restricted to distinct knot classes \cite{Onsager1949,Feynman1955,Iskandarani2025Canon034}. This replaces canonical commutation relations by topological/integrality conditions.

	\section{Canonical Constants and Effective Densities}

	Primary SST constants (recommended calibrations):
	\begin{itemize}
		\item $v_{\!\!\;\circ} = 1.09385\times10^6~\si{m/s}$ (core tangential speed scale).
		\item $r_c = 1.40897\times10^{-15}~\si{m}$ (string core radius).
		\item $\rho_f = 7.00000\times10^{-7}~\si{kg/m^3}$ (effective fluid density).
		\item $\rho_m = 3.89344\times10^{18}~\si{kg/m^3}$ (mass-equivalent density).
		\item $F_{\rm EM}^{\max} = 2.90535\times10^7~\si{N}$ (EM-like maximal force).
		\item $F_{\rm G}^{\max} = 3.02563\times10^{43}~\si{N}$ (gravitational maximal force, reference).
		\item $G_{\!\!\;\circ}$ (dimensionless) -- swirl--EM transduction constant.
	\end{itemize}

	\textbf{Effective density derivation.} The swirl energy density is $\rho_E=\tfrac12\rho_f\,\|\mathbf{v}_{\!\!\;\circ}\|^2$, and mass-equivalent density $\rho_m=\rho_E/c^2$. Coarse-graining many thin vortex filaments in a volume leads to
	\[
		\rho_f \;=\; \rho_m\,\frac{r_c^2\,v_{\!\!\;\circ}}{\langle \omega_{\!\!\;\circ}\rangle}\,,
	\]
	relating $\rho_f$ to microscopic parameters and mean vorticity (see Appendix~A for derivation) \cite{Iskandarani2025Canon034}.

	\section{The Swirl--Electromagnetic Bridge}

	Electromagnetism is an emergent phenomenon of swirl dynamics.

	\begin{theorem}[Swirl-Induced Electromotive Force]
		A time-varying swirl areal density $\varrho_{\!\!\;\circ}$ acts as an effective source term in Faraday’s law:
		\[
			\nabla \times \mathbf{E} = -\,\partial_t \mathbf{B} \;-\; \mathbf{b}_{\!\!\;\circ}, \qquad \mathbf{b}_{\!\!\;\circ} = G_{\!\!\;\circ}\,\partial_t \varrho_{\!\!\;\circ}\,.
		\]
		Thus, reconnections or changes in vortex density generate electromotive force \cite{Jackson1999,Iskandarani2025MagneticVector}.
	\end{theorem}

	\begin{corollary}[Photon as Swirl Wave]
		Unknotted swirl excitations correspond to free electromagnetic radiation. Introduce a divergence-free swirl potential $\mathbf{a}$ such that
		$\mathbf{v}_{\!\!\;\circ}=\partial_t\mathbf{a}$, $\mathbf{b}_{\!\!\;\circ}=\nabla\times\mathbf{a}$, $\nabla\cdot\mathbf{a}=0$.
		With Lagrangian $L_{\rm wave}=\frac{\rho_f}{2}\|\mathbf{v}_{\!\!\;\circ}\|^2 - \frac{\rho_f c^2}{2}\|\mathbf{b}_{\!\!\;\circ}\|^2$, the Euler--Lagrange equation yields
		\[
			\partial_t^2 \mathbf{a} - c^2\,\nabla\times(\nabla\times \mathbf{a}) = 0,\quad \nabla\cdot\mathbf{a}=0,
		\]
		identical to free-space Maxwell in Coulomb gauge. Identifying $\mathbf{E}\propto\partial_t\mathbf{a}$, $\mathbf{B}\propto \nabla\times\mathbf{a}$ recovers all vacuum EM relations; unknotted R-phase excitations are photons \cite{Jackson1999,Iskandarani2025DoubleSlit}.
	\end{corollary}

	\section{Master Equations (Boxed Canonical Relations)}

	\textbf{Swirl Coulomb Potential:}
	\[
		V_{\rm SST}(r) = -\,\Lambda\,\sqrt{\,r^2 + r_c^2\,}\,, \qquad \Lambda = 4\pi\,\rho_m\,v_{\!\!\;\circ}^2\,r_c^4\,,
	\]
	recovering $-\Lambda/r$ for $r\gg r_c$ and the hydrogen spectrum \cite{Iskandarani2025Canon034,Iskandarani2025Hydrogen}.

	\textbf{Swirl Pressure Law (Euler radial balance):}
	\[
		\frac{1}{\rho_f}\frac{dp_{\rm swirl}}{dr} = \frac{v_\theta(r)^2}{\,r\,}\,,
	\]
	with $p_{\rm swirl}(r)=p_0+\rho_f v_0^2\ln(r/r_0)$ for flat rotation curves, providing centripetal force density for galaxies \cite{Batchelor1967,Saffman1992,Iskandarani2025Canon034}.

	\textbf{Swirl Clock (Local Time Dilation):}
	\[
		\frac{dt_{\rm local}}{dt_{\infty}} \;=\; \sqrt{\,1 - \frac{\|\mathbf{v}_{\!\!\;\circ}\|^2}{c^2}\,}\,.
	\]

	\textbf{Swirl Hamiltonian Density:}
	\[
		\mathcal{H}_{\rm SST} \;=\; \tfrac{1}{2}\rho_f\,\|\mathbf{v}_{\!\!\;\circ}\|^2 \;+\; \tfrac{1}{2}\rho_f\,r_c^2\,\|\boldsymbol{\omega}_{\!\!\;\circ}\|^2 \;+\; \lambda\,(\nabla\cdot\mathbf{v}_{\!\!\;\circ})\,,
	\]
	Kelvin-compatible energy density (extended form in Appendix~A) \cite{Batchelor1967,Saffman1992}.

	\textbf{Swirl--Gravity Coupling:}
	\[
		G_{\rm swirl} = \frac{v_{\!\!\;\circ}\,c^5\,t_P^2}{2\,F_{\rm EM}^{\max}\,r_c^2} \;\approx\; G_{\rm N}\,.
	\]

	\textbf{Topology--Driven Mass Law:}
	\[
		M(K)\;=\;\Big(\tfrac{4}{\alpha}\Big)^{\!b-3/2}\,\phi^{-g}\,n^{-1/\phi}\;\Big(\tfrac{1}{2}\rho_f v_{\!\!\;\circ}^2\Big)\,\frac{\pi\,r_c^3\,L_{\rm tot}(K)}{c^2}\,,
	\]
	relating mass to string length and topological invariants (research-grade; included for intuition) \cite{Iskandarani2025Canon034}.

	\section{The Standard Gauge Sector}

	\begin{theorem}[Emergent Yang--Mills from Swirl Directors]
		Elasticity of swirl director fields yields an effective Yang--Mills Lagrangian:
		\[
			\mathcal{L}_{\rm dir}\;\Longrightarrow\; \mathcal{L}_{\rm YM}^{\rm (eff)} \;=\; -\frac{1}{4}\sum_{i=1}^{3} g_i^{-2}\, F_{\mu\nu}^{(i)}\,F^{(i)\mu\nu}, \qquad g_i^{-2}\propto \kappa_i\,,
		\]
		with $SU(3)\times SU(2)\times U(1)$ gauge fields arising from director fluctuations \cite{ChoFaddeevNiemi1999,Iskandarani2025Canon034}.
	\end{theorem}

	\subsection{Knot-to-Representation Map and Particle Taxonomy}

	\textbf{Hypercharge from Topology.} Let $s_3(K)\in\{-1,0,+1\}$ (triality), $d_2(K)\in\{0,1\}$ (weak doublet flag), $\tau(K)\in\{\pm1\}$ (intrinsic twist). Define
	\[
		Y(K) \;=\; \frac{1}{2} + \frac{2}{3}s_3(K) \;-\; d_2(K) \;-\; \frac{1}{2}\tau(K)\,,
	\]
	reproducing SM hypercharges and, with $Q=T_3+\tfrac{1}{2}Y$, the observed charges \cite{Weinberg1967,PeskinSchroeder1995,Iskandarani2025Canon034}. The one-generation set cancels gauge anomalies (topological consistency).

	\subsection{Coupling Constants, EWSB, and the Weak Mixing Angle}

	\textbf{Canonical scale and couplings.} At $\mu=\hbar v_{\!\!\;\circ}/r_c\approx 0.511~\si{MeV}$, effective couplings satisfy $g_i^{-2}(\mu)=\kappa_i\,\Sigma_{\rm core}\,W_i$ (core modulus $\Sigma_{\rm core}=1/\pi$; $W_i$ topological weights) \cite{Iskandarani2025Canon034}.

	\begin{theorem}[Emergence of the Weak Mixing Angle]
		\[
			\tan^2\theta_W \;=\; \frac{g'^2}{g^2} \;=\; \frac{\kappa_2}{\kappa_1}\,,
		\]
		with $\kappa_{1,2}$ director stiffnesses for $U(1)_Y$ and $SU(2)_L$; hence $\theta_W$ is computable from topology \cite{Weinberg1967,PeskinSchroeder1995,Iskandarani2025Canon034}.
	\end{theorem}

	\textbf{EWSB scale.} From bulk swirl energy density $u_{\rm swirl}=\tfrac12\rho_f v_{\!\!\;\circ}^2$ and $(W_1W_2W_3)^{1/4}$,
	\[
		v_\Phi \;=\; u_{\rm swirl}^{1/4}\,(W_1 W_2 W_3)^{1/4} \;\approx\; 2.595\times10^2~\si{GeV}\,,
	\]
	a parameter-free prediction near the observed $246~\si{GeV}$ \cite{Iskandarani2025Canon034}.

	\section{Swirl Gravitation and the Hydrogen-Gravity Mechanism}

	\begin{theorem}[Swirl Gravity Mechanism in Flat Space]
		Chiral knotted strings generate quantized long-range circulation leading to mutual attraction. For a chiral knot $K$ (e.g.\ trefoil $3_1$) enclosing a straight axis, and a large loop $\mathcal{C}$ encircling this axis, Cauchy’s theorem applied to the analytic swirl potential $W(z)$ gives
		\[
			\oint_{\mathcal{C}} \mathbf{v}_{\!\!\;\circ}\cdot d\boldsymbol{\ell} \;=\; 2\pi i\,\mathrm{Res}(\partial_z W,0) \;=\; n\,\kappa\,,
		\]
		with $n$ the winding number (linking number). The locked circulation lowers pressure along the axis by $\Delta p = -\tfrac{1}{2}\rho_f \|\mathbf{v}_{\!\!\;\circ}\|^2$, yielding attraction between knots sharing the axis (e.g.\ two $H_2$ trefoils), all in flat space \cite{Iskandarani2025Hydrogen}.
	\end{theorem}

	\section{Wave--Particle Duality and Quantum Measurement}

	SST explains duality via R-phase (unknotted, delocalized) and T-phase (knotted, localized). de Broglie relations arise by phase closure on an R-loop ($\lambda=L/n$, $p=h/\lambda$) consistent with T-phase boosts.

	\begin{theorem}[R$\to$T Transition Dynamics]
		Quantum measurement (collapse) corresponds to a rapid R$\to$T phase transition induced by interactions:
		\[
			\Gamma_{R\to T} = C_{\rm int}\,\rho_{E,\rm int}\,f(\Delta K)\,,
		\]
		with $C_{\rm int}$ characteristic of the interaction, $\rho_{E,\rm int}$ the interacting field’s energy density, and $f(\Delta K)$ a monotone-decreasing function of knot complexity change. This yields a predictive, environment-dependent collapse kinetics \cite{Zurek2003,Iskandarani2025DoubleSlit}.
	\end{theorem}

	\begin{theorem}[Topological Spin--Statistics Connection]
		Unknotted swirl strings are bosons (integer spin), while knotted strings can carry half-integer spin, in line with Finkelstein--Rubinstein constraints on multi-valuedness under $2\pi$ rotations in configuration space \cite{FinkelsteinRubinstein1968}. Hence \emph{unknotted = boson}, \emph{knotted = fermion}.
	\end{theorem}

	\section{Unified SST Lagrangian (Definitive Form)}

	\[
		\boxed{\;
			\begin{aligned}
				\mathcal{L}_{\rm SST+Gauge} \;=\; &\underbrace{\frac{1}{2}\rho_f\,\|\mathbf{v}_{\!\!\;\circ}\|^2 \;-\; \rho_f\,\Phi_{\rm swirl} \;+\; \lambda\,(\nabla\cdot\mathbf{v}_{\!\!\;\circ}) \;+\; \chi_h\,\rho_f\,(\mathbf{v}_{\!\!\;\circ}\cdot\boldsymbol{\omega}_{\!\!\;\circ})}_{\text{SST Hydrodynamics}} \\[0.5ex]
				&+\; \underbrace{\mathcal{L}_{\rm YM}}_{\text{Gauge Fields}} \;+\; \underbrace{\mathcal{L}_{\rm Matter}}_{\text{Gauge-Charged Matter}}\,,
			\end{aligned}
			\;}
	\]
	where $\mathcal{L}_{\rm YM} = -\frac{1}{4}\sum_i g_i^{-2}F_{\mu\nu}^{(i)}F^{(i)\mu\nu}$. All terms carry units of energy density (J/m$^3$); $\lambda$ enforces incompressibility, $\chi_h$ is an optional helicity coupling \cite{Batchelor1967,Saffman1992,ChoFaddeevNiemi1999,Iskandarani2025Canon034}.

	\bigskip
	\noindent\textbf{Conclusion.} Swirl String Theory (v0.6.0) provides a coherent, testable framework where particles are knotted fluid vortices and forces emerge from hydrodynamic interactions. This document presented the formal axioms and theorems of SST. Appendices include derivations and consistency checks, and a glossary of terms and notation.

	\appendix

	\section{Derivations and Dimensional Checks}\label{app:deriv}

	\subsection{Swirl Hamiltonian Density Derivation}
	A Kelvin-compatible Hamiltonian density for vortex motion is
	\[
		\mathcal{H}_{\rm SST}[\mathbf{v}] \;=\; \frac{1}{2}\rho_f\,\|\mathbf{v}_{\!\!\;\circ}\|^2 \;+\; \frac{1}{2}\rho_f\,r_c^2\,\|\boldsymbol{\omega}_{\!\!\;\circ}\|^2 \;+\; \frac{1}{2}\rho_f\,r_c^4\,\|\nabla\boldsymbol{\omega}_{\!\!\;\circ}\|^2 \;+\; \lambda(\nabla\cdot\mathbf{v}_{\!\!\;\circ})\,,
	\]
	where the $r_c^2\|\omega\|^2$ term captures core rotational energy and $r_c^4\|\nabla\omega\|^2$ penalizes curvature (string tension). All terms are dimensionally J/m$^3$ \cite{Batchelor1967,Saffman1992}. The reduced two-term form appears in the main text.

	\subsection{Derivation of the Swirl Pressure Law}
	For steady, purely azimuthal flow $\mathbf{v}=v_\theta(r)\,\hat{\boldsymbol{\theta}}$, Euler’s radial equation $\rho_f (v_\theta^2/r)=-dp/dr$ yields
	\[
		\frac{1}{\rho_f}\frac{d p_{\rm swirl}}{dr} = \frac{v_\theta^2(r)}{r}\,,
	\]
	and integrating with $v_\theta\to v_0$ gives $p_{\rm swirl}(r)=p_0+\rho_f v_0^2\ln(r/r_0)$. Dimensions and interpretation are standard \cite{Batchelor1967,Saffman1992,Iskandarani2025Canon034}.

	\subsection{Additional Consistency Verifications}
	\textbf{Hydrogen numerics.} With $\Lambda=4\pi\rho_m v_{\!\!\;\circ}^2 r_c^4$ and calibrated constants, one finds $a_0=\hbar^2/(\mu\Lambda)\approx 5.29\times10^{-11}~\si{m}$ and $E_1=-\mu\Lambda^2/(2\hbar^2)\approx -13.6~\si{eV}$, matching hydrogen \cite{Iskandarani2025Canon034,Iskandarani2025Hydrogen}.

	\textbf{Dimensional check of $\Lambda$.} $[\rho_m v^2 r_c^4] = \text{kg/m}^3\times \text{m}^2/\text{s}^2\times \text{m}^4 = \text{kg\,m}^3/\text{s}^2 = \text{J\,m}$.

	\textbf{Swirl pseudo-metric.} In cylindrical coordinates, a steady swirl $v_\theta(r)$ gives an analogue line element $ds^2 = -(c^2 - v_\theta^2)dt^2 + 2 v_\theta r\,d\theta\,dt + dr^2 + r^2 d\theta^2 + dz^2$, indicating clock slowing by $\sqrt{1-v_\theta^2/c^2}$ (co-rotating frame) \cite{Iskandarani2025Canon034}.

	\section{SST Glossary and Notation Index}\label{app:glossary}

	\textbf{Absolute time (A-time):} Universal reference time $t$ for the swirl condensate. \\
	\textbf{Chronos time (C-time):} Time at infinity (no dilation). \\
	\textbf{Swirl Clock:} Local clock comoving with a swirl string, $dt_{\rm local}=S_t\,dt_\infty$. \\
	\textbf{R-phase vs.\ T-phase:} Unknotted delocalized circulation vs.\ knotted localized state. \\
	\textbf{String taxonomy:} Bosons = unknotted; leptons = torus knots; quarks = chiral hyperbolic; composites = links. \\
	\textbf{Chirality:} Handedness of swirl circulation (CCW $\to$ matter; CW $\to$ antimatter). \\
	\textbf{Circulation quantum $\kappa$:} $h/m_{\rm eff}$; $\Gamma=n\kappa$. \\
	\textbf{Swirl Coulomb constant $\Lambda$:} Defines $V_{\rm SST}$; reproduces hydrogen. \\
	\textbf{Swirl areal density $\varrho_{\!\!\;\circ}$:} Vortex cores per unit area; $\partial_t\varrho$ sources induction. \\
	\textbf{$G_{\!\!\;\circ}$:} Swirl--EM inductive coupling constant. \\
	\textbf{$v_{\!\!\;\circ}, \mathbf{v}_{\!\!\;\circ}, \boldsymbol{\omega}_{\!\!\;\circ}$:} Core speed scale; swirl velocity and vorticity fields. \\
	\textbf{$\rho_f, \rho_m$:} Effective fluid density; mass-equivalent density. \\
	\textbf{$G_{\rm swirl}$:} Swirl gravitational constant. \\
	\textbf{$\chi_h$:} Optional helicity coupling in $\mathcal{L}$. \\
	\textbf{$U_3, U_2, \vartheta$:} Director fields for $SU(3)$, $SU(2)$, and $U(1)$ sectors. \\
	\textbf{Knot invariants ($s_3,d_2,\tau,L_{\rm tot},b,g,\phi$):} Topological descriptors used in taxonomy and mass law. \\
	\textbf{Planck/core scales:} $t_P$, $\mu=\hbar v_{\!\!\;\circ}/r_c$.

	\begin{thebibliography}{99}

		\bibitem{Batchelor1967}
		G.~K. Batchelor, \emph{An Introduction to Fluid Dynamics}, Cambridge Univ.\ Press (1967).

		\bibitem{Saffman1992}
		P.~G. Saffman, \emph{Vortex Dynamics}, Cambridge Univ.\ Press (1992).

		\bibitem{Helmholtz1858}
		H. von Helmholtz, ``\"Uber Integrale der hydrodynamischen Gleichungen...'', \emph{J.\ Reine Angew.\ Math.} \textbf{55} (1858).

		\bibitem{Kelvin1869}
		W. Thomson (Lord Kelvin), ``On vortex motion'', \emph{Trans.\ R. Soc.\ Edinburgh} (1869).

		\bibitem{Moffatt1969}
		H.~K. Moffatt, ``The degree of knottedness of tangled vortex lines'', \emph{J.\ Fluid Mech.} \textbf{35}, 117 (1969).

		\bibitem{Onsager1949}
		L. Onsager, ``Statistical Hydrodynamics'', \emph{Nuovo Cimento} \textbf{6}, 279 (1949).

		\bibitem{Feynman1955}
		R.~P. Feynman, in \emph{Progress in Low Temperature Physics}, Vol.~1 (1955).

		\bibitem{Jackson1999}
		J.~D. Jackson, \emph{Classical Electrodynamics}, 3rd ed., Wiley (1999).

		\bibitem{Weinberg1967}
		S. Weinberg, ``A Model of Leptons'', \emph{Phys.\ Rev.\ Lett.} \textbf{19}, 1264 (1967).

		\bibitem{PeskinSchroeder1995}
		M.~E. Peskin and D.~V. Schroeder, \emph{An Introduction to Quantum Field Theory}, Addison--Wesley (1995).

		\bibitem{Zurek2003}
		W.~H. Zurek, ``Decoherence, einselection, and the quantum origins of the classical'', \emph{Rev.\ Mod.\ Phys.} \textbf{75}, 715 (2003).

		\bibitem{FinkelsteinRubinstein1968}
		D. Finkelstein and J. Rubinstein, ``Connection between spin and statistics'', \emph{J.\ Math.\ Phys.} \textbf{9}, 1762 (1968).

		\bibitem{ChoFaddeevNiemi1999}
		Y.~M. Cho, H. Khim, and P. Zhang; L.~D. Faddeev and A.~J. Niemi, papers on decompositions of non-Abelian gauge fields (1998--1999).

		\bibitem{Iskandarani2025Canon034}
		O. Iskandarani, \emph{Swirl String Theory (SST) Canon v0.3.4}, Zenodo (2025). DOI: 10.5281/zenodo.17014358.

		\bibitem{Iskandarani2025Hydrogen}
		O. Iskandarani, ``Long-Distance Swirl Gravity from Chiral Swirling Knots with Central Holes'' (2025).

		\bibitem{Iskandarani2025DoubleSlit}
		O. Iskandarani, ``Wave--Particle Duality in SST: Toroidal Circulation, Knot Collapse, and Photon-Induced Transitions'' (2025).

		\bibitem{Iskandarani2025MagneticVector}
		O. Iskandarani, ``Magnetic helicity in periodic domains: gauge conditions, existence of vector potentials, and periodic winding --- Rosetta to SST'' (2025).

	\end{thebibliography}

\end{document}