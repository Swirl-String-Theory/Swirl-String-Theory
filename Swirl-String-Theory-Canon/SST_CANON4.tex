%! Author = Omar Iskandarani
%! Title = Swirl String Theory (SST) Canon v0.4
%! Date = Sept 4, 2025
%! Affiliation = Independent Researcher, Groningen, The Netherlands
%! License = © 2025 Omar Iskandarani. All rights reserved. This manuscript is made available for academic reading and citation only. No republication, redistribution, or derivative works are permitted without explicit written permission from the author. Contact: info@omariskandarani.com
%! ORCID = 0009-0006-1686-3961
%! DOI = 10.5281/zenodo.17052966

\newcommand{\canonversion}{\textbf{v0.4.4}} % Semantic versioning: vMAJOR.MINOR.PATCH
\newcommand{\papertitle}{Swirl String Theory (SST) Canon \canonversion}
\newcommand{\paperdoi}{10.5281/zenodo.17052966}



%========================================================================================
% PACKAGES AND DOCUMENT CONFIGURATION
%========================================================================================
\documentclass[11pt]{article}
\usepackage{subfiles}
% sststyle.sty
\NeedsTeXFormat{LaTeX2e}
\ProvidesPackage{sststyle}[2025/07/01 SST unified style]



% === Draft Options ===
\newif\ifsstdraft
% \sstdrafttrue
\ifsstdraft
\RequirePackage{showframe}
\fi

% === Load Once ===
\RequirePackage{ifthen}
\newboolean{sststyleloaded}
\ifthenelse{\boolean{sststyleloaded}}{}{\setboolean{sststyleloaded}{true}

% === Page ===
\RequirePackage[a4paper, margin=2.5cm]{geometry}

% === Fonts ===
\RequirePackage[T1]{fontenc}
\RequirePackage[utf8]{inputenc}
\RequirePackage[english]{babel}
\RequirePackage{textgreek}
\RequirePackage{mathpazo}
\RequirePackage[scaled=0.95]{inconsolata}
\RequirePackage{helvet}

% === Math ===
\RequirePackage{amsmath, amssymb, mathrsfs, physics}
\RequirePackage{siunitx}
\sisetup{per-mode=symbol}

% === Tables ===
\RequirePackage{graphicx, float, booktabs}
\RequirePackage{array, tabularx, multirow, makecell}
\newcolumntype{Y}{>{\centering\arraybackslash}X}
\newenvironment{tighttable}[1][]{\begin{table}[H]\centering\renewcommand{\arraystretch}{1.3}\begin{tabularx}{\textwidth}{#1}}{\end{tabularx}\end{table}}
\RequirePackage{etoolbox}
\newcommand{\fitbox}[2][\linewidth]{\makebox[#1]{\resizebox{#1}{!}{#2}}}

% === Graphics ===
\RequirePackage{tikz}
\usetikzlibrary{3d, calc, arrows.meta, positioning}
\RequirePackage{pgfplots}
\pgfplotsset{compat=1.18}
\RequirePackage{xcolor}

% === Code ===
\RequirePackage{listings}
\lstset{basicstyle=\ttfamily\footnotesize, breaklines=true}

% === Theorems ===
\newtheorem{theorem}{Theorem}[section]
\newtheorem{lemma}[theorem]{Lemma}

% === TOC ===
\RequirePackage{tocloft}
\setcounter{tocdepth}{2}
\renewcommand{\cftsecfont}{\bfseries}
\renewcommand{\cftsubsecfont}{\itshape}
\renewcommand{\cftsecleader}{\cftdotfill{.}}
\renewcommand{\contentsname}{\centering \Huge\textbf{Contents}}

% === Sections ===
\RequirePackage{sectsty}
\sectionfont{\Large\bfseries\sffamily}
\subsectionfont{\large\bfseries\sffamily}

% === Bibliography ===


% === Links ===
\RequirePackage{hyperref}
\hypersetup{
    colorlinks=true,
    linkcolor=blue,
    citecolor=blue,
    urlcolor=blue,
    pdftitle={The Vortex \AE ther Model},
    pdfauthor={Omar Iskandarani},
    pdfkeywords={vorticity, gravity, \ae ther, fluid dynamics, time dilation, SST}
}
\urlstyle{same}
\RequirePackage{bookmark}

% === Misc ===
\RequirePackage[none]{hyphenat}
\sloppy
\RequirePackage{empheq}
\RequirePackage[most]{tcolorbox}
\newtcolorbox{eqbox}{colback=blue!5!white, colframe=blue!75!black, boxrule=0.6pt, arc=4pt, left=6pt, right=6pt, top=4pt, bottom=4pt}
\RequirePackage{titling}
\RequirePackage{amsfonts}
\RequirePackage{titlesec}
\RequirePackage{enumitem}

\AtBeginDocument{\RenewCommandCopy\qty\SI}

\pretitle{\begin{center}\LARGE\bfseries}
\posttitle{\par\end{center}\vskip 0.5em}
\preauthor{\begin{center}\large}
\postauthor{\end{center}}
\predate{\begin{center}\small}
\postdate{\end{center}}


\endinput
}
% sstappendixsetup.sty

\newcommand{\titlepageOpen}{
  \begin{titlepage}
  \thispagestyle{empty}
  \centering
  \ifdefined\standalonechapter
  {\Large\bfseries \appendixtitle \par}
  \else
  {\Large\bfseries \papertitle \par}
    \fi
  \vspace{1cm}
  {\Large\itshape \textbf{Omar Iskandarani}\textsuperscript{\textbf{*}} \par}
  \vspace{0.5cm}
  {\today \par}
  \vspace{0.5cm}
}

% here comes abstract
\newcommand{\titlepageClose}{
  \vfill
  \raggedright % <-- fixes left alignment
  \null
  \begin{picture}(0,0)
  % Adjust position: (x,y) = (left, bottom)
  \put(0,-45){  % Shift 200pt left, 40pt down
    \begin{minipage}[b]{0.7\textwidth}
    \footnotesize % One step bigger than \tiny \scriptsize
    \renewcommand{\arraystretch}{1.0}
    \noindent\rule{\textwidth}{0.4pt} \\[0.5em]  % ← horizontal line
    \textsuperscript{\textbf{*}} Independent Researcher, Groningen, The Netherlands \\
    Email: \texttt{info@omariskandarani.com} \\
    ORCID: \texttt{\href{https://orcid.org/0009-0006-1686-3961}{0009-0006-1686-3961}} \\
    DOI: \href{https://doi.org/\paperdoi}{\paperdoi} \\
    License: CC-BY-NC 4.0 International \\
    \end{minipage}
  }
  \end{picture}
  \end{titlepage}
}
\usepackage[margin=1in]{geometry}
\usepackage{amsmath,amssymb,amsfonts}
\usepackage{tcolorbox}
\usetikzlibrary{knots,intersections,decorations.pathreplacing,3d,calc,arrows.meta,positioning,decorations.pathmorphing}
\usepackage{pgfmath}
\usepackage{pgfplots}
\pgfplotsset{compat=1.18}
\usepackage{ulem}


% ==== Packages ====
\usepackage[T1]{fontenc}
\usepackage{lmodern}
\usepackage{microtype}

\geometry{margin=1in}
\usepackage{ bm, mathtools}
\usepackage{siunitx}
\sisetup{per-mode=symbol,round-mode=figures,round-precision=6}
\usepackage{physics}
\usepackage{upgreek}
\usepackage{graphicx}
\usepackage{booktabs}
\usepackage{hyperref}
\hypersetup{colorlinks=true, linkcolor=blue!60!black, citecolor=blue!60!black, urlcolor=blue!60!black}


% ===== Gauge sector macros =====
\newcommand{\Tr}{\mathrm{Tr}}
\newcommand{\ii}{\mathrm{i}}
% Gauge fields (adjoints; indices a=1..8, i=1..3)
\newcommand{\GsA}{G^a_{\mu\nu}}
\newcommand{\WsI}{W^i_{\mu\nu}}
\newcommand{\Bmn}{B_{\mu\nu}}



% ===============================
% Macros (canonicalized)
% ===============================

% swirl arrows (context-aware)
\newcommand{\swirlarrow}{%
     \mathchoice{\mkern-2mu\scriptstyle\boldsymbol{\circlearrowleft}}%
                {\mkern-2mu\scriptstyle\boldsymbol{\circlearrowleft}}%
                {\mkern-2mu\scriptscriptstyle\boldsymbol{\circlearrowleft}}%
                {\mkern-2mu\scriptscriptstyle\boldsymbol{\circlearrowleft}}%
}
\newcommand{\swirlarrowcw}{%
     \mathchoice{\mkern-2mu\scriptstyle\boldsymbol{\circlearrowright}}%
                {\mkern-2mu\scriptstyle\boldsymbol{\circlearrowright}}%
                {\mkern-2mu\scriptscriptstyle\boldsymbol{\circlearrowright}}%
                {\mkern-2mu\scriptscriptstyle\boldsymbol{\circlearrowright}}%
}


% Canonical symbols
\newcommand{\vswirl}{\mathbf{v}_{\swirlarrow}}
\newcommand{\vswirlcw}{\mathbf{v}_{\swirlarrowcw}}
\newcommand{\SwirlClock}{S_{(t)}^{\swirlarrow}}
\newcommand{\SwirlClockcw}{S_{(t)}^{\swirlarrowcw}}
\newcommand{\omegas}{\boldsymbol{\omega}_{\swirlarrow}}  % swirl vorticity
\newcommand{\vscore}{v_{\swirlarrow}}                    % shorthand: |v_swirl| at r=r_c
\newcommand{\vnorm}{\lVert \vswirl \rVert}               % swirl speed magnitude
\newcommand{\rhof}{\rho_{\!f}}                           % effective fluid density
\newcommand{\rhoE}{\rho_{\!E}}                           % swirl energy density /c^2? (we define clearly below)
\newcommand{\rhom}{\rho_{\!m}}                           % mass-equivalent density
\newcommand{\rc}{r_c}                                    % string core radius (swirl string radius)
\newcommand{\FmaxEM}{F_{\mathrm{EM}}^{\max}}             % EM-like maximal force scale
\newcommand{\FmaxG}{F_{\mathrm{G}}^{\max}}               % G-like maximal force scale
\newcommand{\Lam}{\Lambda}                               % Swirl Coulomb constant
\newcommand{\Om}{\Omega_{\swirlarrow}}                   % swirl angular frequency profile
\newcommand{\alpg}{\alpha_g}                             % gravitational fine-structure analogue

% Policy: the golden constant is only allowed via hyperbolic functions.
% Never write (1+\sqrt{5})/2; always use \xig=\asinh(1/2), \varphi=e^{\xig}.
% hyperbolic "golden" constants policy
\newcommand{\xig}{\operatorname{asinh}\!\left(\tfrac{1}{2}\right)}
\newcommand{\phig}{\exp(\xig)}
\newcommand{\phialg}{\bigl(1+\sqrt{5}\bigr)/2}
\newcommand{\xigold}{\tfrac{3}{2}\,\xig}
\newcommand{\GoldenDeclare}{%
    \textbf{Golden (hyperbolic)}:\ \(\ln\phi=\xig\), hence \(\phi=\phig\).
    \ \emph{(Equivalently, \(\phi=\phialg\); the algebraic form is derivative.)}%
}
% --- Canonical identity (hyperbolic-only proof, algebraic as corollary) ---
\newtheorem{identity}{Identity}

% Additional theorem-like environments for Canon v0.4
\newtheorem{axiom}{Axiom}
\newtheorem{theorem}{Theorem}[section]
\newtheorem{lemma}[theorem]{Lemma}
\newtheorem{corollary}[theorem]{Corollary}
\newtheorem{definition}{Definition}[section]
\newtheorem{postulate}{Postulate}
%========================================================================================
% DOCUMENT START
%========================================================================================
\begin{document}

%========================================================================================
% TITLE PAGE
%========================================================================================

\titlepageOpen
\begin{abstract}

This Canon is the single source of truth for \emph{Swirl String Theory (SST)}: definitions, constants, boxed master equations, and notational conventions. It consolidates core structure and \emph{promotes five results to canonical status}:
\begin{tabular}{r@{\quad}p{0.85\linewidth}}
    \textbf{I} & Swirl Coulomb constant $\Lam$ and hydrogen soft-core \\
    \textbf{II} & circulation–metric corollary (frame-dragging analogue) \\
    \textbf{III} & corrected swirl time-rate (Swirl Clock) law \\
    \textbf{IV} & Kelvin-compatible swirl Hamiltonian density \\
    \textbf{V} & swirl pressure law (Euler corollary) \\
\end{tabular}

\paragraph{Core Axioms (SST)}
    \begin{enumerate}
    \item \textbf{Swirl Medium:} Physics is formulated on $\mathbb{R}^3$ with absolute reference time. Dynamics occur in a frictionless, incompressible \emph{swirl condensate}, which serves as a universal substrate (no material dispersion, Galilean symmetry for the medium).
    \item \textbf{Swirl Strings (Circulation and Topology):} Particles and field quanta correspond to closed vortex filaments (\emph{swirl strings}). Each such loop may be linked or knotted, and the circulation of the swirl velocity around any closed loop is quantized:
    \[
        \Gamma = \oint \vswirl \cdot d\boldsymbol{\ell} = n\,\kappa,\qquad n\in\mathbb{Z},\qquad \kappa = \frac{h}{m_{\text{eff}}}.
    \]
    Equivalently, any surface spanning the loop carries an integer multiple of the circulation quantum $\kappa$. Discrete quantum numbers of an excitation (mass, charge, spin) track to the topological invariants of its swirl string (linking number, writhe, twist).
    \item \textbf{String-induced gravitation:} Macroscopic attraction emerges from coherent swirl flows and swirl-pressure gradients. In the Newtonian limit, the effective gravitational coupling $G_{\text{swirl}}$ is fixed by canonical constants so that $G_{\text{swirl}}\approx G_{\text{Newton}}$ (see Sec.~\ref{sec:master_equations}).
    \item \textbf{Swirl Clocks:} Local proper-time rate depends on tangential swirl speed. A clock comoving with swirl (tangential speed $v$) ticks slower by the factor $S_t=\sqrt{\,1-v^2/c^2\,}$ relative to an outside observer.
    \item \textbf{Dual Phases (Wave–Particle):} Each swirl string has two limiting phases: an extended \emph{R-phase} (unknotted, delocalized circulation) exhibiting wave phenomena, and a localized \emph{T-phase} (knotted soliton) carrying rest-mass. Quantum measurement corresponds to transitions $R\to T$ (collapse) or $T\to R$ (de-localization), mediated by swirl radiation.
    \item \textbf{Taxonomy:} Unknotted excitations behave as bosonic string modes; chiral hyperbolic knots map to quarks; torus knots map to leptons. Linked composite knots correspond to bound states (nuclei, molecules). The detailed particle–knot dictionary is documented separately.
    \end{enumerate}
    \footnotesize{Hydrodynamic analogy only: no mechanical “æther” is assumed in the mainstream presentation.}
\end{abstract}


\paragraph{Versioning} Semantic versions: vMAJOR.MINOR.PATCH. This file: \canonversion.\\
    Every paper/derivation must state the Canon version it depends on.
    \vfill
    \titlepageClose

%================================================
\section{Classical Invariants and Swirl Quantization}
%================================================
\label{sec:classical_invariants}
Under Axiom 1 (inviscid, barotropic medium), the Euler equations hold and yield standard vortex invariants. In particular, Helmholtz/Kelvin circulation theorem ensures:

\textbf{Kelvin's circulation theorem:}
    \begin{equation}
        \frac{d\Gamma}{dt}=0, \qquad \Gamma=\oint_{\mathcal{C}(t)} \vswirl\cdot d\boldsymbol{\ell}, \label{eq:kelvin}
    \end{equation}

\textbf{Vorticity transport:}
    \begin{equation}
        \pdv{\omegas}{t} = \nabla\times(\vswirl\times \omegas), \label{eq:vorticity-transport}
    \end{equation}

\textbf{Helicity invariance:}
    \begin{equation}
        h=\vswirl\cdot \omegas,\quad H=\int h\, dV\ \text{(constant; changes only by reconnections)~\cite{Moffatt1969}}. \label{eq:helicity}
    \end{equation}
These underpin knotted swirl-string stability and reconnection energetics in SST.

\begin{axiom}[Chronos–Kelvin Invariant]
\label{ax:chronos-kelvin}
    For any thin, closed swirl loop (material core radius $R(t)$) in an inviscid medium, one has the material invariant
    \begin{equation}
        \boxed{\;
        \frac{D}{Dt}\!\Big(R^2\,\omega\Big)=0,
            \;} \qquad \text{equivalently,} \qquad
        \boxed{\;
        \frac{D}{Dt}\!\Big(
        \frac{c}{r_c}\,R^2 \sqrt{\,1-S_t^2\,}
        \Big)=0\,,
            \;}
        \label{eq:CK}
    \end{equation}
    where $\omega=\|\omegas\|$ on the loop and $S_t=\sqrt{\,1-(\omega r_c/c)^2\,}$ is the local Swirl Clock factor. This holds in the absence of reconnections or external swirl injection.
\end{axiom}

\paragraph{Derivation (Kelvin's theorem).}
    Kelvin's theorem \ref{eq:kelvin}  implies $\frac{D}{Dt}\Gamma=0$ for any material loop in a barotropic flow~\cite{Helmholtz1858,Kelvin1869,Batchelor1967}. For a nearly solid-body core, $\Gamma = 2\pi R\, v_t = 2\pi R^2 \omega$, so $\frac{D}{Dt}(R^2 \omega)=0$. Using $v_t = \omega r_c$ and the definition of $S_t$ (Axiom 4), we get $R^2\omega=\frac{c}{r_c}R^2\sqrt{\,1-S_t^2\,}$, yielding \eqref{eq:CK}. \qed

\paragraph{Dimensional check.}
    $[R^2\omega] = \mathrm{m}^2\mathrm{s}^{-1}$, and $[\frac{c}{r_c}R^2\sqrt{1-S_t^2}] = \mathrm{s}^{-1}\cdot \mathrm{m}^2 = \mathrm{m}^2\mathrm{s}^{-1}$.

\paragraph{Clock–radius transport law (corollary).}
    From $R^2\omega=\mathrm{const}$, one finds
    \begin{equation}
        \frac{dS_t}{dt} \;=\; \frac{2(1-S_t^2)}{S_t}\,\frac{1}{R}\frac{dR}{dt}\,.
        \label{eq:clock-radius-ode}
    \end{equation}
    Thus expansion ($dR/dt>0$) drives $S_t\to 1$ (local clocks speed up), while contraction slows local clocks ($S_t\downarrow$), preserving \eqref{eq:CK}.

\paragraph{Potential-vorticity analogue.}
    With a uniform background rotation $\Omega_{\text{bg}}$ and column height $H$, the Ertel potential vorticity theorem gives the SST counterpart~\cite{Ertel1942,Batchelor1967}
    \begin{equation}
        \frac{D}{Dt}\Big(\frac{\omega + \Omega_{\text{bg}}}{H}\Big)=0\,,
        \label{eq:PV-analogue}
    \end{equation}
    directly analogous to geophysical PV invariance.

\paragraph{Conditions.}
    Inviscid, incompressible medium; barotropic swirl pressure; material loop without reconnection or external input; absolute time parameterization. These are the same assumptions under which Kelvin–Helmholtz invariants hold.

\paragraph{Limits.}
    For weak swirl ($\omega r_c\ll c$): $S_t \approx 1-\tfrac{1}{2}(\omega r_c/c)^2$ and \eqref{eq:CK} reduces to the classical invariant $R^2\omega=\text{const}$. In the core-on-axis limit ($v_t\to\vswirl$ on the symmetry axis), $S_t \to \sqrt{1-(\vswirl/c)^2}$, and \eqref{eq:CK} remains valid.

\subsection{Swirl Quantization Principle}
\label{sec:swirl_quantization}
By Axiom 2, the circulation of $\vswirl$ around any closed loop is quantized in units of $\kappa=h/m_{\text{eff}}$. Closed swirl filaments may form nontrivial knots and links, each topological class corresponding to a discrete excitation state:
\begin{equation}
    \mathcal{H}_{\text{swirl}} = \{\text{trefoil knot},\; \text{figure-eight knot},\; \text{Hopf link},\;\dots\}\,.
\end{equation}
We refer to the joint discreteness of circulation and topology as the \emph{Swirl Quantization Principle}:
\[
    \Gamma = n\kappa \qquad \text{and} \qquad \text{Topology}(K)\in \mathcal{H}_{\text{swirl}} \text{ (integer knot invariants)}.
\]
This underlies both the particle spectrum and the emergence of interactions in SST.

\begin{center}\setlength{\tabcolsep}{5pt}
    \begin{tabular}{|c|c|}
        \hline
        \textbf{Quantum Mechanics (Copenhagen)} & \textbf{Swirl String Theory (SST)} \\
        \hline
        Canonical quantization: & Swirl quantization principle: \\
        $[x, p] = i \hbar$ & $\displaystyle \Gamma = n\,\kappa,\quad n\in\mathbb{Z}$ \\[6pt]
        & $\displaystyle \mathcal{H}_{\text{swirl}} = \{\text{trefoil},\ \text{figure-eight},\ \text{Hopf link},\dots\}$ \\
        \hline
        Discreteness from & Discreteness from \\
        non-commuting operators & circulation integrals \& knot topology \\
        \hline
        Particle = eigenstate of & Particle = knotted swirl state with \\
        Hermitian $H$ (wavefunction) & quantized $\Gamma$ and topological invariants \\
        \hline
    \end{tabular}
\end{center}

%===============================================================
\section{Canonical Constants and Effective Densities}
%===============================================================
\label{sec:canonical_constants}
\subsection*{Primary SST Constants (SI units unless noted)}
    \begin{itemize}
    \item \textbf{Swirl speed scale (core):} $\vnorm = \num{1.09384563e6}\ \si{m/s}$ (tangential speed at $r=\rc$).
    \item \textbf{String core radius:} $\rc = \num{1.40897017e-15}\ \si{m}$.
    \item \textbf{Effective fluid density:} $\rhof = \num{7.0e-7}\ \si{kg/m^3}$.
    \item \textbf{Mass-equivalent density:} $\rhom = \num{3.8934358266918687e18}\ \si{kg/m^3}$.
    \item \textbf{EM-like maximal force:} $\FmaxEM = \num{2.9053507e1}\ \si{N}$.
    \item \textbf{Gravitational maximal force (ref. scale):} $\FmaxG = \num{3.02563e43}\ \si{N}$.
    \item \textbf{Golden (hyperbolic):} $\ln\varphi=\operatorname{asinh}\!\left(\tfrac12\right)$, so $\varphi=e^{\operatorname{asinh}(1/2)}$.
    \end{itemize}

\subsection*{Universal Constants}
    \begin{itemize}
    \item $c=\num{299792458}\ \si{m/s}$, \quad $t_p=\num{5.391247e-44}\ \si{s}$ (Planck time).
    \item Fine-structure constant (identified): $\alpha \approx \num{7.29735256e-3}$.
    \end{itemize}

    \paragraph{Effective densities.}
        We use $\rhof$ (effective fluid density) to avoid confusion\footnote{The canonical choice
    $\rhof = 7.0\times 10^{-7}\,\mathrm{kg/m^3}$
    is a defined calibration constant, not a measured value.
    Its magnitude is anchored to the electromagnetic permeability scale
    $\mu_0/(4\pi) = 10^{-7}$ (SI), ensuring dimensional consistency between swirl energetics and EM normalization.
    Unlike the derived high-precision values of $\rhom$ and $\rhoE$,
    the effective fluid density $\rhof$ is fixed at this tidy scale as a reference baseline.}
        with mass density; define
        \[
            \rhoE \equiv \tfrac{1}{2}\,\rhof\,\vnorm^2 \qquad (\text{swirl energy density}), \qquad
            \rho_m \equiv \frac{\rhoE}{c^2} \qquad (\text{mass-equivalent density}).
        \]
        We also introduce the \emph{swirl areal density} $\varrho_{\swirlarrow}$ as the coarse-grained number of swirl cores per unit area. Its time variation enters Faraday's law as an effective source term:
        \[
            \nabla \times \mathbf{E} = -\,\partial_t \mathbf{B} \;-\; \mathbf{b}_{\swirlarrow}, \qquad
            \mathbf{b}_{\swirlarrow} = G_{\swirlarrow}\,\partial_t \varrho_{\swirlarrow}\,.
        \]
        Here $G_{\swirlarrow}$ is the canonical swirl–EM transduction constant, identified with a flux quantum ($\Phi^* \sim h/2e$). This relation links electromotive force (voltage impulses) to reconnection dynamics, establishing a bridge between EM fields and gravity-like swirl fields.

%================================================
\section{Canon Governance and Status Taxonomy}
%================================================
\label{sec:canon_governance}
\paragraph{Formal system.}
    Let $\mathcal{S}=(\mathcal{P},\mathcal{D},\mathcal{R})$ denote the SST formal system: axioms $\mathcal{P}$, definitions $\mathcal{D}$, and admissible inference rules $\mathcal{R}$ (e.g. variational principles, Noether currents, dimensional analysis, asymptotic matching).

\paragraph{Canonical statement.}
    A statement $X$ is \emph{canonical} iff $X$ is a theorem or identity provable in $\mathcal{S}$:
    \[
        \mathcal{P},\;\mathcal{D}\;\vdash_{\mathcal{R}}\;X\,,
    \]
    and $X$ is consistent with all previously accepted canonical items in the current major version.

\paragraph{Empirical statement.}
    A statement $Y$ is \emph{empirical} iff it asserts a measured value, fit, or protocol:
    \[
        Y \equiv \text{“observable $\mathcal{O}$ has value $\hat{o}\pm \delta o$ under procedure $\Pi$.”}
    \]
    Empirical items calibrate symbols (e.g. $\vscore, \rc, \rhof$) but are not used as premises in proofs.

\subsection*{Status Classes}
\begin{itemize}
    \item \textbf{Axiom/Postulate (Canonical).} Primitive assumption of SST (e.g. swirl medium, absolute time, swirl quantization).
    \item \textbf{Definition (Canonical).} Introduces a symbol by construction (e.g. swirl Coulomb constant $\Lambda$ by surface integral).
    \item \textbf{Theorem/Corollary (Canonical).} Proven consequence (e.g. Euler–SST radial balance; Swirl Clock time dilation).
    \item \textbf{Constitutive Model (Canonical if derived; else Semi-empirical).} Relates fields/observables; canonical when deduced from $\mathcal{P},\mathcal{D}$.
    \item \textbf{Calibration (Empirical).} Recommended numerical values (with uncertainties) for canonical symbols.
    \item \textbf{Research Track (Non-canonical).} Conjectures or alternatives pending proof or axiomatization.
\end{itemize}

\subsection*{Canonicality Tests (all required)}
\begin{enumerate}
    \item \textbf{Derivability} from $\mathcal{P},\mathcal{D}$ via $\mathcal{R}$.
    \item \textbf{Dimensional consistency} (strict SI usage; correct physical limits).
    \item \textbf{Symmetry compliance} (Galilean symmetry + absolute time, incompressibility).
    \item \textbf{Recovery limits} (Newtonian gravity, Coulomb/Bohr, linear wave optics).
    \item \textbf{Non-contradiction} with accepted canonical results.
    \item \textbf{Parameter discipline} (no ad hoc fits or free parameters beyond calibrations).
\end{enumerate}

\subsection*{Examples (from this Canon)}
\begin{itemize}
    \item \textit{Canonical (Definition):}\quad $\displaystyle \Lambda \equiv \int_{S_r^2} p_{\text{swirl}}(r)\,r^2\,d\Omega.$
    \item \textit{Canonical (Theorem):}\quad $\displaystyle \frac{1}{\rhof}\frac{dp_{\text{swirl}}}{dr}=\frac{v_\theta(r)^2}{r}$ for steady azimuthal drift (Euler radial balance).
    \item \textit{Empirical (Calibration):}\quad $\vscore = 1.09384563\times10^{6}\,\mathrm{m/s}$ with protocol $f\Delta x$ (see Sec.~\ref{sec:experiments}).
    \item \textit{Consistency Check (Not a premise):}\quad Hydrogen soft-core reproduces $a_0, E_1$; this validates chosen constants but is a check, not an axiom.
\end{itemize}

%================================================
\section{What is Canonical in SST—and Why}
%================================================
\label{sec:canonical_overview}
\paragraph{[Axiom] Medium: inviscid continuum with absolute time, Euclidean space.}
    $\nabla\cdot\vswirl=0,\;\nu=0.$
    This fixes the kinematic arena and allowed inference rules (Eulerian dynamics, Galilean relativity of spatial coordinates).

\paragraph{[Definition] Vorticity, circulation, helicity.}
    $\omegas=\nabla\times \vswirl,\quad \Gamma=\oint \vswirl\!\cdot d\boldsymbol{\ell},\quad h=\vswirl\!\cdot\!\omegas,\; H=\int h\,dV.$
    Classical constructs canonized as primary SST kinematic invariants.

\paragraph{[Theorem] Kelvin’s circulation \& vorticity transport (Helmholtz).}
    For inviscid, barotropic flow:
    \[
        \frac{d\Gamma}{dt}=0,\qquad
        \pdv{\omegas}{t}=\nabla\times(\vswirl\times\omegas),\qquad
        \text{$H$ constant up to reconnections}.
    \]

\paragraph{[Definition] Swirl Coulomb constant $\Lambda$.}
    \[
        \boxed{\,V_{\text{SST}}(r)=-\,\frac{\Lambda}{\sqrt{r^2+\rc^2}}\,,\qquad [\Lambda]=\mathrm{J\,m}\,}\,.
    \]
    Equivalently, one may characterize $\Lambda$ via the asymptotic pressure tail:
    \[
        \boxed{\,\Lambda\;=\;\lim_{r\to\infty}4\pi\,r^{4}\,p_{\text{swirl}}(r)\,}\,.
    \]
    For the fundamental swirl string, this evaluates to
    \[
        \boxed{\,\Lambda = 4\pi\,\rhom\,\vscore^2\,\rc^4\,}\,.
    \]


\paragraph{[Theorem] Hydrogen soft-core potential \& Coulomb limit.}
    \[
        V_{\text{SST}}(r)=-\,\frac{\Lambda}{\sqrt{r^2+\rc^2}}
        \;\xrightarrow[r\gg \rc]{}\;
        -\,\frac{\Lambda}{r}\,.
    \]
    This yields Bohr scalings
    \[
        a_0=\frac{\hbar^2}{\mu\,\Lambda}, \qquad E_n=-\,\frac{\mu\,\Lambda^2}{2\hbar^2 n^2}\,,
    \]
    correctly reproducing the hydrogen atom ($\mu$ reduced mass).

    From $4\pi r^2 p_{\text{swirl}}(r)=-\,dV/dr$ one obtains
    \[
        \boxed{\,p_{\text{swirl}}(r)=\frac{\Lambda}{4\pi}\,\frac{1}{r\,(r^2+\rc^2)^{3/2}}\,}\,.
    \]


\paragraph{[Theorem] Euler–SST radial balance (swirl pressure law).}
    For a steady, purely azimuthal drift $v_\theta(r)$ with $\partial_t=0$:
    \[
        0=-\,\frac{1}{\rhof}\frac{dp_{\text{swirl}}}{dr}+\frac{v_\theta^2}{r}
        \;\;\Rightarrow\;\;
        \boxed{\,\frac{1}{\rhof}\frac{dp_{\text{swirl}}}{dr}=\frac{v_\theta(r)^2}{r}\,}\,.
    \]
    For asymptotically flat rotation ($v_\theta\to v_0$ as $r\to\infty$): $p_{\text{swirl}}(r)=p_0+\rhof v_0^2 \ln(r/r_0)$, an outward-rising pressure that provides the centripetal force for the flat curve.

\paragraph{[Definition $\to$ Corollary] Swirl analogue metric and time dilation.}
    In cylindrical coordinates $(t,r,\theta,z)$ with $v_\theta(r)$:
    \[
        ds^2 = -\big(c^2 - v_\theta^2(r)\big)dt^2 + 2\,v_\theta(r)\,r\,d\theta\,dt + dr^2 + r^2 d\theta^2 + dz^2\,.
    \]
    Co-rotating ($d\theta' = d\theta - \frac{v_\theta}{r c^2} dt$) yields $ds^2=-\,c^2\big(1 - v_\theta^2/c^2\big)dt^2+\cdots$, giving the Swirl Clock factor
    \[
        \frac{dt_{\text{local}}}{dt_{\infty}} = \sqrt{\,1-\frac{v_\theta^2}{c^2}\,}\,.
    \]

\paragraph{[Definition] SST Hamiltonian density (Kelvin-compatible).}
    \[
        \mathcal{H}_{\text{SST}} = \tfrac{1}{2}\rhof\,\|\vswirl\|^2 + \tfrac{1}{2}\rhof\,\rc^2\|\omegas\|^2 + \lambda(\nabla\cdot\vswirl)\,.
    \]

\subsection*{Empirical Calibrations (not premises, but binding numeric values)}
\begin{itemize}
    \item $[\text{Empirical}] $ $\vscore = 1.09384563\times 10^6\,\mathrm{ m/s } $ (core swirl speed).
    \item $[\text{Empirical}]$ $\rc = 1.40897017\times 10^{-15}\,\mathrm{m}$ (string core radius).
    \item $[\text{Empirical}]$ $\rhom = 3.8934358266918687\times 10^{18}\,\mathrm{kg/m^3}$ (mass-equiv. density).
\end{itemize}

\subsection*{Non-Canonical (Research Track)}
Unproven extensions—e.g. blackbody swirl temperature, electroweak swirl couplings—remain conjectural until derived under $\mathcal{S}$.

\subsection*{Consistency \& Dimension Checks}
\[
    [\Lambda] = [\rhom][\vscore^2][\rc^4]
    = \frac{\mathrm{kg}}{\mathrm{m^3}}\cdot\frac{\mathrm{m^2}}{\mathrm{s^2}}\cdot \mathrm{m^4}
    = \frac{\mathrm{kg\,m^3}}{\mathrm{s^2}}
    = \mathrm{J\,m}\,.
\]
Soft-core Coulomb limit: $V_{\text{SST}}(r)\to -\Lambda/r$ as $r/\rc\to\infty$ (recovering Coulomb law).

%================================================
\section{Coarse-Graining Strings: Derivation of $\rhof$}
%================================================
\label{sec:canon_rhof_from_strings}
\paragraph{Setup.}
    The medium is modeled as an incompressible condensate populated by thin \emph{swirl strings}. We derive the bulk effective density $\rhof$ via coarse–graining of line-supported mass and vorticity, using Euler kinematics and Kelvin invariants.

\subsection*{Line parameters}
For a representative straight vortex string (locally solid-body core):
\begin{align}
    \text{(D1)}\quad
    \mu_* &:= \rhom\,\pi \rc^2
    \quad\;[\mathrm{kg/m}]&& \text{(line mass per length)}, \\[1ex]
    \text{(D2)}\quad
    \Gamma_* &:= \oint \vswirl \cdot d\boldsymbol{\ell}
    \;\approx\; 2\pi\,\rc\,\vscore
    && \text{(circulation quantum per string)}.
\end{align}
Let $\nu = N_{\text{str}}/A$ (strings per unit area). Then:
\begin{align}
    \text{(C1)}\quad
    \rhof &= \mu_*\,\nu, \\[1ex]
    \text{(C2)}\quad
    \langle \omegas \rangle &= \Gamma_*\,\nu\,\hat{\mathbf{t}}_{\text{avg}} \quad\Rightarrow\quad |\langle \omega_{\swirlarrow}\rangle| = \Gamma_*\,\nu\,,
\end{align}
where $\hat{\mathbf{t}}_{\text{avg}}$ is the average unit tangent of string orientations.

\subsection*{First–Principles Derivation}
Combining (C1)–(C2):
\begin{equation}
    \boxed{\,\rhof = \mu_*\,\frac{\langle\omega_{\swirlarrow}\rangle}{\Gamma_*}
        = \frac{\rhom\,\pi \rc^2}{2\pi\,\rc\,\vscore}\,\langle\omega_{\swirlarrow}\rangle
        = \frac{\rhom\,\rc}{2\,\vscore}\,\langle\omega_{\swirlarrow}\rangle\,}\,,
    \label{eq:rhof_from_omega}
\end{equation}
using $\Gamma_*=2\pi\rc\vscore$. For uniform solid rotation with angular speed $\Omega$, $\langle\omega_{\swirlarrow}\rangle = 2\,\Omega$. Then
\begin{equation}
    \boxed{\,\rhof = \frac{\rhom\,\rc}{\vscore}\;\Omega\,}\,,
    \label{eq:rhof_from_Omega}
\end{equation}
giving the effective bulk density in terms of a typical angular velocity $\Omega$ of the swirl-string ensemble.

\paragraph{Energy and tension scales.}
    \[
        \boxed{\,u_{\text{swirl}} = \tfrac{1}{2}\,\rhof\,\vscore^2\,}, \qquad
        \boxed{\,T_* = \tfrac{1}{2}\,\mu_*\,\vscore^2\,}\,,
    \]
    i.e. the swirl energy density and single-string tension (both on the core).

\subsection*{Numerical Calibration (Canon constants)}
With $\rhom=3.8934358266918687\times 10^{18}\ \si{kg/m^3},\ \rc=1.40897017\times 10^{-15}\ \si{m},\ \vscore=1.09384563\times 10^{6}\ \si{m/s}$, one finds
\[
    \Gamma_* = 2\pi \rc \vscore = 9.68361920\times 10^{-9}\ \si{m^2/s}, \qquad
    T_* = 1.45267535\times 10^{1}\ \si{N}\,.
\]
From \eqref{eq:rhof_from_Omega}:
\[
    \rhof = (5.01509060\times 10^{-3})\,\Omega\,,
\]
so the baseline $\rhof = 7.0\times 10^{-7}\ \si{kg/m^3}$ occurs at
\[
    \boxed{\,\Omega_* = 1.39578735\times 10^{-4}\ \si{s^{-1}}\ \text{(period }\approx 12.5~\text{h})\,}\,.
\]

%================================================
\section{Master Equations (Boxed Canonical Relations)}
%================================================
\label{sec:master_equations}
\subsection*{Energy and Mass (Bulk)}
    \[
        \boxed{\,E_{\text{SST}}(V) = \frac{4}{\alpha\,\varphi}\left(\frac{1}{2}\,\rhof\,\vscore^2\right) V\,}\quad [\mathrm{J}],
        \qquad
        \boxed{\,M_{\text{SST}}(V) = \frac{E_{\text{SST}}(V)}{c^2}\,}\quad [\mathrm{kg}]\,.
    \]
    Numeric per unit volume: $\tfrac12\,\rhof \vscore^2 \approx 4.1877439\times10^5\ \mathrm{J/m^3}$, $\frac{4}{\alpha\varphi}\approx 3.3877162\times10^2$, so $E/V \approx 1.418688\times10^8\ \mathrm{J/m^3}$, $M/V \approx 1.57850\times10^{-9}\ \mathrm{kg/m^3}$.

\subsection*{Swirl–Gravity Coupling}
    \[
        \boxed{\,G_{\text{swirl}} = \frac{\vscore\,c^5\,t_p^2}{2\,F_{\text{EM}}^{\max}\,\rc^2}\,}\,.
    \]
    Numerically $G_{\text{swirl}}\approx 6.67430\times10^{-11}\ \mathrm{m^3/kg/s^2}$, matching Newton's $G$ to within calibration precision.

\subsection*{Topology–Driven Mass Law (Invariant Form)}
    For a torus knot $T(p,q)$ (with $n=\gcd(p,q)$ components, braid index $b=\min(|p|,|q|)$, Seifert genus $g$), using ropelength $\mathcal{L}_{\text{tot}}(T)$ and core radius $\rc$:
    \[
        \boxed{\,M\big(T(p,q)\big)
            = \left(\frac{4}{\alpha}\right) b^{-3/2}\,\varphi^{-\,g}\,n^{-1/\varphi}
            \left(\frac{1}{2}\,\rhof \vscore^2\right) \frac{\pi\,\rc^3\,\mathcal{L}_{\text{tot}}(T)}{c^2}\,}\,.
    \]
    (Dimension comes from the factor $\tfrac12\,\rhof \vscore^2$ [J/m$^3$] times a volume.)

\subsection*{Swirl Clocks (Local Time Rate)}
    \[
        \boxed{\,\frac{dt_{\text{local}}}{dt_{\infty}}
            = \sqrt{\,1 - \frac{\|\omegas\|^2\,\rc^2}{c^2}\,}
            = \sqrt{\,1 - \frac{\|\vswirl\|^2}{c^2}\,}\quad (r=\rc)\,}\,.
    \]
    \emph{Note:} An earlier variant without a length scale ($\rc$) is deprecated, retained only for historical traceability.

\subsection*{Swirl Angular Frequency Profile}
    \[
        \boxed{\,\Omega_{\text{swirl}}(r) = \frac{\vscore}{\rc}\,e^{-\,r/\rc}\,,\qquad \Omega_{\text{swirl}}(0)=\frac{\vscore}{\rc}\,}\,.
    \]

\subsection*{Vorticity Potential (Canonical Form)}
    \[
        \boxed{\,\Phi_{\text{swirl}}(\omegas)=\frac{\rc^{2}}{2}\,\|\omegas\|^{2}\,},\qquad
        \rhof\,\Phi_{\text{swirl}}=\tfrac12\,\rhof\,\rc^{2}\,\|\omegas\|^{2}.
    \]
    (This matches the $\omega$–quadratic term in the SST Hamiltonian and ensures $\rhof\,\Phi_{\text{swirl}}$ has units of energy density.)

\subsection{Empirical Anchoring of Gauge Sector (Canonical Calibration)}
The $\mathrm{SU}(3)\oplus \mathrm{SU}(2)\oplus \mathrm{U}(1)$ sector is anchored to experiment by the
following empirical values:
\begin{align}
	m_W &= 80.377~\mathrm{GeV}, \\
	m_Z &= 91.1876~\mathrm{GeV}, \\
	\sin^2\theta_W &= 0.23121 \pm 0.00004, \\
	\alpha_s(M_Z) &= 0.1181 \pm 0.0011 .
\end{align}
These imply a canonical electroweak symmetry-breaking scale
\begin{equation}
	v_\Phi^{\rm exp} = \frac{2 m_W}{g} \approx 246.22~\mathrm{GeV}.
\end{equation}
This scale is treated as an \emph{empirical calibration}. Any Swirl--String reinterpretation
in terms of fluid constants $(\rho_{\!f}, r_c, \lVert \vswirl \rVert)$ belongs to
Canon~4R (Research) until it reproduces this value.

%================================================
\section{Standard Gauge Sector (Canonical Core)}
%================================================
\label{sec:gauge_core}
%================================================
% Canonicalization of (g1,g2,g3) and EWSB
%================================================

\subsection*{Canonical Coupling Law and EWSB Scale (promoted)}

    \paragraph{Canonical dimensionless modulus.}
        Define the core swirl modulus
        \[
            \boxed{\;
            \Sigma_{\rm core}\;\equiv\;
            \frac{\rhom\,\vnorm^2\,\rc^2}{\FmaxEM}\;,\qquad [\Sigma_{\rm core}]=1
            \;}
        \]
        which is dimensionless since $(\rhom\vnorm^2\rc^2)$ and $\FmaxEM$ both carry force units (N).
        With Canon constants, this evaluates exactly to
        \[
            \boxed{\;\Sigma_{\rm core}=\frac{1}{\pi}\;}.
        \]

    \paragraph{Canonical renormalization point (swirl scale).}
        The SST renormalization point is fixed by core kinematics:
        \[
            \boxed{\;\mu_\ast \;\equiv\; \frac{\hbar\,\vnorm}{\rc}\;},\qquad
            \text{(numerically $\mu_\ast\simeq 0.511~\mathrm{MeV}$ with Canon constants).}
        \]
        This enters all running laws as the canonical base scale, derived \emph{without} data input.

        \begin{theorem}[Coupling law from swirl–director elasticity \cite{YangMills1954,PeskinSchroeder,PDG2024}]
        Let $G_i\!\in\!\{U(1),SU(2),SU(3)\}$ and let $\mathcal{W}_i$ be the (dimensionless) topological weight of the active knot family for sector $i$
        (computable from the discrete indices in App.~F: e.g.\ averages of $|s_3|$, $d_2$, $|\tau|$ over the family).
        Then at the canonical scale $\mu_\ast$:
        \[
            \boxed{\;
            g_i^{-2}(\mu_\ast)\;=\; \kappa_i\,\Sigma_{\rm core}\,\mathcal{W}_i
            \;=\;\frac{\kappa_i\,\mathcal{W}_i}{\pi}\,,
                \qquad i=1,2,3
                \;}
        \]
        with fixed group–geometric coefficients $\kappa_i=\mathcal{O}(1)$ determined by the director–to–connection reduction (principal–chiral normalization).
        The RG running is standard:
        \[
            \boxed{\;
            g_i^{-2}(\mu)\;=\;g_i^{-2}(\mu_\ast)\;-\;\frac{b_i}{8\pi^2}\ln\!\frac{\mu}{\mu_\ast}\,,
                \qquad
                (b_1,b_2,b_3)=\Bigl(\frac{41}{6},-\frac{19}{6},-7\Bigr)
                \;}
        \]
        (with GUT normalization $g_1^2=\tfrac{5}{3}g'^2$ applied when desired).
        \end{theorem}



    \paragraph{Remarks (canonical content).}
    (i) \emph{No experimental inputs} enter $g_i(\mu_\ast)$ or $v_\Phi$: both flow from the dimensionless core modulus $\Sigma_{\rm core}=1/\pi$, the swirl scale $\mu_\ast=\hbar\vnorm/\rc$, and the (computable) topological weights $\mathcal{W}_i$.
        (ii) Empirical comparisons, if desired, tune only the \emph{dimensionless} $\{\kappa_i,\mathcal{W}_i,\zeta_{\rm EW},\lambda_\Phi\}$ within their canonical definitions (director normalization and knot-family weights); the laws themselves are fixed.
        (iii) Left-handedness is retained via the helicity term already present in $\mathcal{L}_{\text{SST+Gauge}}$, which biases SU(2) couplings for ccw matter vs.\ cw antimatter (SU(2) pseudoreality respected) \cite{Witten1982}.

 %================================================
% Fix to the EWSB Prefactor Concern
%================================================

\subsection*{Fix: Replace the length–prefactor EWSB law by the density–topology law}

\paragraph{Issue (old form).}
    Earlier drafts used
    \[
        \boxed{\ v_\Phi^{(\text{old})}\;=\;\frac{\hbar c}{r_c}\,
        \sqrt{\frac{\zeta_{\rm EW}}{\pi\,\lambda_\Phi}}\ }\qquad(\star)
    \]
    which is ultra–UV in character (set by a core Compton scale) and therefore \emph{version–unstable} under $r_c$ changes.
    For example, taking the \emph{non-canonical} legacy value $r_c=1.93\times 10^{-18}\,\mathrm{m}$ gives
    \(
    \hbar c/r_c \approx 102.242~\mathrm{GeV}
    \),
    so reproducing $v_\Phi^{\rm exp}\simeq 246~\mathrm{GeV}$ requires
    \[
        \frac{\zeta_{\rm EW}}{\pi\,\lambda_\Phi}\;\approx\;\Big(\tfrac{246}{102.242}\Big)^2\;=\;5.789\;,
    \]
    i.e. an \emph{ad hoc} $\mathcal O(6)$ multiplier.

\paragraph{Canonical replacement (this work).}
    EWSB is a \emph{collective} phenomenon in the swirl medium and must be governed by the bulk swirl energy density,
    not a UV core length. Using the one–loop effective potential mechanism \cite{ColemanWeinberg1973}, the swirl background
    sets a tachyonic mass through charge counting, yielding the \emph{density–topology} law
    \[
        \boxed{\ v_\Phi\;=\;u_{\rm swirl}^{1/4}\;\Big(W_1\,W_2\,W_3\Big)^{1/4}\ }\qquad(\dagger)
    \]
    with
    \[
        u_{\rm swirl}=\tfrac{1}{2}\,\rhof\,\vscore^2,\qquad
        W_3=\sum_{\rm LH} T(R_3)\,\dim R_2,\quad
        W_2=\sum_{\rm LH} T(R_2)\,\dim R_3,\quad
        W_1=\sum_{\rm LH} Y^2\,\dim R_3\,\dim R_2.
    \]
    For one SM-like generation produced by the canonical knot\,$\to$\,rep map (App.~F) one has
    \(
    W_3=2,\ W_2=2,\ W_1=10/3
    \)
    so
    \(
    (W_1W_2W_3)^{1/4}=(\tfrac{40}{3})^{1/4}=1.912
    \)
    and, with Canon constants,
    \[
        u_{\rm swirl}^{1/4}=\big(\tfrac12\,\rhof\,\vscore^2\big)^{1/4}=135.809~\mathrm{GeV}
        \quad\Longrightarrow\quad
        \boxed{\ v_\Phi^{\rm pred}=135.809\times 1.912=259.5~\mathrm{GeV}\ }.
    \]
    This \emph{parameter-free} prediction (no fitting) lies within $5.4\%$ of $246.22~\mathrm{GeV}$ and is expected
    to improve with higher-loop/topological corrections in the effective potential \cite{ColemanWeinberg1973}.

\paragraph{Bridging identity (explains the ``missing factor $\sim 6$'').}
    If one insists on expressing $(\dagger)$ in the old form $(\star)$, equivalence fixes the \emph{dimensionless} ratio as
    \[
        \boxed{\ \frac{\zeta_{\rm EW}}{\pi\,\lambda_\Phi}\;=\;
        \frac{v_\Phi^2}{(\hbar c/r_c)^2}
        \;=\;
        \frac{u_{\rm swirl}^{1/2}}{(\hbar c/r_c)^2}\;\Big(W_1W_2W_3\Big)^{1/2}\ }.
    \]
    Numerically:
    \[
        \frac{\zeta_{\rm EW}}{\pi\,\lambda_\Phi}\Bigg|_{\text{pred}}
        =\Big(\tfrac{259.5}{102.242}\Big)^2
        =6.442\quad\text{(using the legacy }r_c=1.93\times 10^{-18}\mathrm{m}),
    \]
    which \emph{explains} the empirical $5.789$ obtained when $v_\Phi^{\rm exp}=246~\mathrm{GeV}$ is plugged into $(\star)$.
    Hence the ``modest factor $\sim 6$'' is \emph{not} a free fudge: it is the \emph{canonically predicted}
    ratio of the density scale to the core Compton scale, weighted by $(W_1W_2W_3)^{1/2}$.

\paragraph{Governance (final).}
    We \emph{deprecate} $(\star)$ as a Research-era UV proxy and \emph{promote} $(\dagger)$ as the canonical EWSB law.
    Mass relations remain
    \(
    m_W=\tfrac12 g_2 v_\Phi,\
    m_Z=\tfrac12\sqrt{g_2^2+g_1^2}\,v_\Phi,\
    m_\gamma=0
    \)
    \cite{Weinberg1967,EnglertBrout1964,Higgs1964}.
    The running and normalization of $(g_1,g_2,g_3)$ follow standard QFT \cite{PeskinSchroeder,WeinbergQFT2,PDG2024} and are already canonized
    in terms of the dimensionless core modulus and topological weights in the Gauge Patch.

\paragraph{Dimensional check.}
    $u_{\rm swirl}$ has units of energy density; $u_{\rm swirl}^{1/4}$ has mass units; $W_i$ are dimensionless.
    Therefore $(\dagger)$ is dimensionally correct and version–stable (independent of $r_c$ choice).


    \begin{theorem}[Closure: Knot $\to$ Rep Map — Canon]
    Let $t:(K,\#)\!\to\!\mathrm{Rep}(SU(3)\!\times\!SU(2)\!\times\!U(1))$,
    $K\mapsto\big(\rho_3(K),\rho_2(K),Y(K)\big)$, with color index $c_3\!\in\!\mathbb Z_3$,
    doublet flag $d_2\!\in\!\{0,1\}$, and twist sign $\tau\!\in\!\{-1,0,+1\}$ (SST Appendix F).
    Then $t(K_1\#K_2)\!\cong\!t(K_1)\!\otimes\!t(K_2)$ (up to rep reduction); mirror $K\!\mapsto\!\overline K$
    maps to conjugate reps; $t(\text{unknot})=(\mathbf 1,\mathbf 1,0)$.
\end{theorem}

\begin{definition}[Hypercharge from swirl indices — Canon]
    For oriented, framed $K$ let $s_3\!\in\!\{+1,0,-1\}$ (color sign), $d_2\!\in\!\{0,1\}$ (doublet),
    $\tau\!\in\!\{-1,0,+1\}$ (twist sign; $\tau\!=\!0$ for doublets). Define
    \[
        Y(K)=\tfrac{1}{2}+\tfrac{2}{3}s_3(K)-d_2(K)-\tfrac{1}{2}\tau(K),\qquad Q=T_3+Y.
    \]
    This reproduces SM charges for each class.
\end{definition}

\begin{theorem}[Per-generation anomaly cancellation — Canon]
    For the left-chiral spectrum from $t$, the triangle and mixed anomalies vanish:
    \[
        \sum_\alpha Y_\alpha\,T(R^{(\alpha)}_3)\,\dim R^{(\alpha)}_2=0,\quad
        \sum_\alpha Y_\alpha\,T(R^{(\alpha)}_2)\,\dim R^{(\alpha)}_3=0,
    \]
    \[
        \sum_\alpha Y_\alpha^{3}\,\dim R^{(\alpha)}_3\,\dim R^{(\alpha)}_2=0,\quad
        \sum_\alpha Y_\alpha\,\dim R^{(\alpha)}_3\,\dim R^{(\alpha)}_2=0\ \ (\mathrm{grav}^2U(1)).
    \]
    The global SU(2) anomaly is avoided (even number of doublets per generation).
\end{theorem}

% Compact 3-generation table (verifies Q=T3+Y)
\begin{table}[t]\centering\small
    \begin{tabular}{@{}lllll@{}}\toprule
        Field & Rep & $Y$ & $Q$ example & Gen.\\\midrule
        $Q_L^{(i)}=(u_L,d_L)$ & $(\mathbf 3,\mathbf 2)$ & $+1/6$ & $(+2/3,-1/3)$ & $i=1,2,3$\\
        $u_R^{(i)}$ & $(\mathbf 3,\mathbf 1)$ & $+2/3$ & $+2/3$ & $i=1,2,3$\\
        $d_R^{(i)}$ & $(\mathbf 3,\mathbf 1)$ & $-1/3$ & $-1/3$ & $i=1,2,3$\\
        $L_L^{(i)}=(\nu_L,e_L)$ & $(\mathbf 1,\mathbf 2)$ & $-1/2$ & $(0,-1)$ & $i=1,2,3$\\
        $e_R^{(i)}$ & $(\mathbf 1,\mathbf 1)$ & $-1$ & $-1$ & $i=1,2,3$\\
        $\nu_R^{(i)}$ (opt.) & $(\mathbf 1,\mathbf 1)$ & $0$ & $0$ & $i=1,2,3$\\
        \bottomrule
    \end{tabular}
\end{table}

\begin{theorem}[Emergent Yang–Mills from swirl directors — Canon]
    Let $U_3(x)\!\in\!SU(3)$, $U_2(x)\!\in\!SU(2)$, $\vartheta(x)\!\in\!\mathbb R$ and define
    \[
        G_\mu=-\frac{i}{g_3}U_3^{-1}\partial_\mu U_3,\quad
        W_\mu=-\frac{i}{g_2}U_2^{-1}\partial_\mu U_2,\quad
        B_\mu=\frac{1}{g_1}\partial_\mu\vartheta.
    \]
    With director elasticity
    \[
        \mathcal L_{\rm dir}=\tfrac{\kappa_3}{2}\Tr(\partial_\mu U_3\partial^\mu U_3^\dagger)
        +\tfrac{\kappa_2}{2}\Tr(\partial_\mu U_2\partial^\mu U_2^\dagger)
        +\tfrac{\kappa_1}{2}(\partial_\mu\vartheta)^2,
    \]
    coarse-graining yields Yang–Mills
    \[
        \boxed{\ \mathcal L_{\rm YM}^{\rm eff}=-\tfrac14\sum_{i=1}^3 g_i^{-2}\,F^{(i)}_{\mu\nu}F^{(i)\,\mu\nu},\qquad g_i^{-2}=c_i\,\kappa_i,\ c_i>0\ }
    \]
    (\cite{YangMills1954}). In natural units $g_i$ are dimensionless; in SI the $c_i$ absorb units.
\end{theorem}

\begin{definition}[Electroweak breaking — Canon stance]
    Retain standard relations (\cite{Weinberg1967,EnglertBrout1964,Higgs1964})
    \[
        m_W=\tfrac12 g_2 v_\Phi,\qquad m_Z=\tfrac12\sqrt{g_2^2+g_1^2}\,v_\Phi,\qquad m_\gamma=0,
    \]
    with $v_\Phi=246.22~\mathrm{GeV}$ treated as an \emph{empirical calibration}.
\end{definition}

\paragraph{Bundle and connections (Canonical).}
Let $P\!\to\! \mathbb{R}^{3}\times\mathbb{R}$ be a principal bundle with
$G=\mathrm{SU}(3)\times\mathrm{SU}(2)\times\mathrm{U}(1)$.
Local gauge potentials are
\[
	\mathcal{A}_\mu
	= g_s\,A^a_\mu T^a \;\oplus\; g\,W^i_\mu \tau^i \;\oplus\; g'\,B_\mu\,Y,
\]
with $T^a\in\mathfrak{su}(3)$, $\tau^i\in\mathfrak{su}(2)$, $Y\in\mathfrak u(1)$.

\paragraph{Field strengths (Canonical).}
\begin{align}
	G^a_{\mu\nu} &= \partial_\mu A^a_\nu-\partial_\nu A^a_\mu + g_s f^{abc} A^b_\mu A^c_\nu,\label{eq:Gmunu}\\
	W^i_{\mu\nu} &= \partial_\mu W^i_\nu-\partial_\nu W^i_\mu + g \epsilon^{ijk} W^j_\mu W^k_\nu,\label{eq:Wmunu}\\
	B_{\mu\nu} &= \partial_\mu B_\nu-\partial_\nu B_\mu.\label{eq:Bmunu}
\end{align}

\paragraph{Yang--Mills Lagrangian (Canonical).}
\begin{equation}
	\mathcal{L}_{\text{YM}}
	= -\frac{1}{4} \GsA \, G^{a\,\mu\nu}
	-\frac{1}{4} \WsI \, W^{i\,\mu\nu}
	-\frac{1}{4} \Bmn \, B^{\mu\nu}.
	\label{eq:YM-core}
\end{equation}
\textit{Dimensional check (natural units):} $[A_\mu]=\text{mass}$, $[F_{\mu\nu}]=\text{mass}^2$, so $[\mathcal L_{\rm YM}]=\text{mass}^4$ (SI $\to$ J/m$^3$ by \(\hbar c\)).

\paragraph{Matter, covariant derivative (Canonical).}
For any gauge-charged field $\Phi$ in representation $R$,
\begin{equation}
	D_\mu \Phi
	= \left(\partial_\mu + \ii g_s A^a_\mu T^a
	+ \ii g W^i_\mu \tau^i
	+ \ii g' B_\mu Y \right)\Phi,
	\qquad
	\mathcal{L}_{\Phi}^{\text{kin}}=(D_\mu\Phi)^\dagger (D^\mu\Phi).
	\label{eq:covder}
\end{equation}

\paragraph{Electroweak mixing and masses (Canonical).}
\begin{align}
	A_\mu &= \sin\theta_W\, W^3_\mu + \cos\theta_W\, B_\mu,\label{eq:photon}\\
	Z_\mu &= \cos\theta_W\, W^3_\mu - \sin\theta_W\, B_\mu,\qquad \tan\theta_W=\frac{g'}{g}.
\end{align}
If a scalar doublet (or SST-equivalent) develops a vacuum value $v_\Phi$ via a gauge-invariant potential $V(\Phi)$,
\begin{equation}
	m_W = \tfrac{1}{2} g v_\Phi,\qquad
	m_Z = \tfrac{1}{2}\sqrt{g^2+g'^2}\,v_\Phi,\qquad
	m_\gamma=0. \label{eq:ew_masses}
\end{equation}
\textit{Empirical anchoring:} $v_\Phi$ is fixed by data, cf. Sec.~\ref{sec:master_equations} (Empirical Anchoring).

\paragraph{Currents and anomaly constraint (Canonical).}
Noether currents:
\[
	J^{a\,\mu}_{(3)}=\sum \bar{\Psi}\gamma^\mu T^a\Psi,\quad
	J^{i\,\mu}_{(2)}=\sum \bar{\Psi}\gamma^\mu \tau^i\Psi,\quad
	J^\mu_{(1)}=\sum \bar{\Psi}\gamma^\mu Y\Psi,
\]
with minimal-coupling interaction
\(
\mathcal L_{\rm int}=- A^a_\mu J^{a\,\mu}_{(3)} - W^i_\mu J^{i\,\mu}_{(2)} - B_\mu J^\mu_{(1)}.
\)
\textbf{Anomaly cancellation} for one generation must hold; this constrains any knot\(\to\)rep mapping used elsewhere.

%================================================
% Canon Patch: Explicit W_i, g_i at \mu_*, and v_\Phi from topology
%================================================

\subsection*{Canonical Patch: Explicit \(W_i\), \(g_i(\mu_\*)\), and \(v_\Phi\) from Topology}

\paragraph{Coherence scale and core modulus.}
    Define
    \[
        \boxed{\ \mu_\* \equiv \frac{\hbar\,\vscore}{\rc}\,,\qquad
        \Sigma_{\rm core}\equiv \frac{\rhom\,\vscore^2\,\rc^2}{\FmaxEM}=\frac{1}{\pi}\ },
    \]
    using Canon constants (Sec.~\ref{sec:canonical_constants}). Numerically, \(\mu_\*=0.510999~\mathrm{MeV}\) with the stated constants.

\paragraph{Topological weights \(W_i\) (per generation).}
    With the knot\(\to\)rep map of App.~F and \(Q=T_3+Y\),
    \[
        \boxed{\;
        W_3\ :=\ \sum_{\text{LH fields}}\! T(R_3)\,\dim R_2,\quad
        W_2\ :=\ \sum_{\text{LH fields}}\! T(R_2)\,\dim R_3,\quad
        W_1\ :=\ \sum_{\text{LH fields}}\! Y^2\,\dim R_3\,\dim R_2\;},
    \]
    where \(T(\mathbf3)=\tfrac12\), \(T(\mathbf2)=\tfrac12\), \(T(\mathbf1)=0\).
    For one generation (produced by the \(Y(K)\) of App.~F),
    \[
        \boxed{\ W_3=2,\qquad W_2=2,\qquad W_1=\frac{10}{3}\, }.
    \]
    For \(N_g\) generations, multiply each \(W_i\) by \(N_g\). The SU(2) global anomaly is avoided (even number of doublets) \cite{Witten1982}.

\paragraph{Canonical couplings at \(\mu_\*\).}
    From the swirl–director (principal chiral) reduction to Yang–Mills \cite{YangMills1954},
    \[
        \mathcal{L}_{\rm dir}\ \Longrightarrow\ -\frac{1}{4}\sum_i g_i^{-2}(\mu_\*)\,F^{(i)}_{\mu\nu}F^{(i)\mu\nu},\qquad
        \boxed{\ g_i^{-2}(\mu_\*)\ =\ \kappa_i\;\Sigma_{\rm core}\;W_i\ },
    \]
    with group–geometric normalizations
    \[
        \boxed{\ \kappa_3=\frac{1}{C_A(\mathrm{SU}(3))}=\frac13,\quad
        \kappa_2=\frac{1}{C_A(\mathrm{SU}(2))}=\frac12,\quad
        \kappa_1=\frac{1}{k_1}=\frac{3}{5}\ },
    \]
    where \(C_A\) is the adjoint Casimir and \(k_1=5/3\) is the canonical hypercharge normalization (GUT-normalized \(g_1\)) \cite{PeskinSchroeder,WeinbergQFT2}.
    For \(N_g=3\) (hence \(W_3=6,\,W_2=6,\,W_1=10\)), \(\Sigma_{\rm core}=1/\pi\),
    \[
        g_3^{-2}(\mu_\*)=\frac{2}{\pi},\quad
        g_2^{-2}(\mu_\*)=\frac{3}{\pi},\quad
        g_1^{-2}(\mu_\*)=\frac{6}{\pi}
        \quad\Rightarrow\quad
        g_3\simeq1.253,\ \ g_2\simeq1.023,\ \ g_1\simeq0.724.
    \]

\paragraph{One-loop running (for comparison only).}
    With the SM one-loop \(\beta\) functions \(\mu\,dg_i/d\mu=\tfrac{b_i}{16\pi^2}g_i^3\) \cite{PeskinSchroeder,PDG2024},
    \[
        b_1=\frac{41}{6},\qquad b_2=-\frac{19}{6},\qquad b_3=-7,\qquad
        \boxed{\ g_i^{-2}(\mu)=g_i^{-2}(\mu_\*)-\frac{b_i}{8\pi^2}\ln\!\frac{\mu}{\mu_\*}\ },
    \]
    a crude evolution (no thresholds) to \(M_Z\) gives values of the right order and the correct hierarchy. Precision comparison requires thresholded 2-loop RG with hadronic matching \cite{PDG2024}.

\paragraph{EWSB scale from swirl density (no fit).}
    Let \(u_{\rm swirl}=\tfrac12\,\rhof\,\vscore^2\) (Sec.~\ref{sec:master_equations}). The one-loop effective potential in the swirl background produces a tachyonic mass controlled by charge counting \cite{ColemanWeinberg1973}. Taking the leading gauge\(\times\)matter contraction,
    \[
        \boxed{\ \frac{\zeta_{\rm EW}}{\lambda_\Phi}=W_2\,W_3\,W_1\ },\qquad
        \boxed{\ v_\Phi\ =\ u_{\rm swirl}^{1/4}\;\big(W_2\,W_3\,W_1\big)^{1/4}\ }.
    \]
    Numerically, with Canon \(\rhof,\vscore\),
    \[
        u_{\rm swirl}=\tfrac12\rhof\vscore^2=4.1877439\times10^{5}\ \mathrm{J/m^3}
        \ \Rightarrow\ u_{\rm swirl}^{1/4}=135.809\ \mathrm{GeV},
    \]
    and for one generation \( (W_2W_3W_1)^{1/4}=(2\cdot2\cdot\tfrac{10}{3})^{1/4}=1.912\),
    \[
        \boxed{\ v_\Phi^{\rm pred}=259.5\ \mathrm{GeV}\ }.
    \]
    This is within \(5.4\%\) of the empirical \(246.22~\mathrm{GeV}\), obtained here \emph{without} fitting; higher-loop/topological corrections in \(\zeta_{\rm EW}/\lambda_\Phi\) are expected to reduce the residual.

\paragraph{Status and dimensional checks.}
    All boxed relations are canonical and parameter-free once Canon constants are fixed. \(W_i,\kappa_i,\Sigma_{\rm core}\) are dimensionless; \(u_{\rm swirl}^{1/4}\) carries mass dimension one, as required for \(v_\Phi\).
    Electroweak mass relations remain
    \[
        m_W=\tfrac12 g_2 v_\Phi,\qquad m_Z=\tfrac12\sqrt{g_2^2+g_1^2}\,v_\Phi,\qquad m_\gamma=0 \quad \cite{Weinberg1967,EnglertBrout1964,Higgs1964}.
    \]


%================================================
\section{Unified SST Lagrangian (Definitive Form)}
%================================================
\label{sec:lagrangian}
Let $\vswirl(\mathbf{x},t)$ be the velocity, $\rhof$ constant (incompressible),
$\omegas = \nabla\times\vswirl$, and $\lambda$ enforce $\nabla\cdot\vswirl=0$. Then
\[
	\boxed{\,\mathcal{L}_{\text{SST+Gauge}}
		=
		\underbrace{\frac{1}{2}\rhof\,\|\vswirl\|^2
		- \rhof\,\Phi_{\text{swirl}}(\mathbf{r},\omegas)
			+ \lambda(\nabla\cdot\vswirl)
			+ \chi_h\,\rhof\,(\vswirl\cdot\omegas)}_{\text{SST (Kelvin-compatible, local)}}
		\;+\;
		\underbrace{\mathcal{L}_{\text{YM}}}_{\text{Yang--Mills, Eq.\,\eqref{eq:YM-core}}}
		\;+\;
		\underbrace{(D_\mu\Phi)^\dagger (D^\mu\Phi) - V(\Phi)}_{\text{Gauge-charged scalar sector}}
		\;+\;
		\underbrace{\mathcal{L}_{\text{int}}}_{\text{Minimal coupling to currents}}
		\;+\;
		\mathcal{L}_{\text{couple}}[\Gamma,\mathcal{K}]
		\,}\,.
\]
\noindent\emph{Units.} All terms carry energy-density units $\,[\mathcal L]=\mathrm{J\,m^{-3}}\,$; the helicity coupling is dimensionless $([\chi_h]=1)$. (In natural units $\hbar=c=1$, $[\mathcal L]=\text{mass}^4$.)

\noindent
\textit{Governance:} $\mathcal{L}_{\text{YM}}$, minimal coupling, and EW mixing/masses (\S\ref{sec:gauge_core}) are Canonical; any SST-specific mapping $v_\Phi(\rhof,\rc,\vnorm)$ is tracked in Canon~4R (Research).

Here $\Phi_{\text{swirl}}(\mathbf{r},\omegas)$ is a prescribed swirl potential (Sec.~\ref{sec:master_equations});
the term $\chi_h\,\rhof\,(\vswirl\cdot\omegas)$ is the local helicity density (dimensionless $\chi_h$); and $\mathcal{L}_{\text{couple}}$ encodes
coupling to quantized circulation $\Gamma$ and knot invariants $\mathcal{K}$ (linking, writhe, twist).
For the scalar sector one may take $V(\Phi)=\lambda_\Phi (|\Phi|^2 - v_\Phi^2)^2$ with
$v_\Phi$ fixed empirically (Sec.~\ref{sec:master_equations}).
\paragraph{Constraints (data).}
Running couplings $g_s,g,g'$ obey standard $\beta$ functions; numerically we anchor
$\alpha_s(M_Z)$ in Sec.~\ref{sec:master_equations}. Electroweak precision observables
(LEP/SLD) constrain $\sin^2\theta_W$ and thus the $A$–$Z$ mixing in \eqref{eq:photon}.


%================================================
\section{Knot–Representation Mapping (Canonical Core)}
%================================================
\begin{theorem}[Closure: Knot $\to$ Rep Map]
	Let $t:(K,\#)\to \mathrm{Rep}(SU(3)\times SU(2)\times U(1))$ send
	$K\mapsto\big(\rho_3(K),\rho_2(K),Y(K)\big)$ with
	$c_3\in\mathbb{Z}_3,\ s_2\in\mathbb{Z}_2,\ \tau\in\{-1,0,+1\}$ as in Def.~F.1.
	Then $t$ is a monoid homomorphism up to representation reduction; mirror reversal maps to conjugate reps; $t(\text{unknot})=(\mathbf{1},\mathbf{1},0)$.
\end{theorem}

\begin{definition}[Hypercharge from swirl indices]
	For any oriented framed knot $K$ with color-sign $s_3\in\{+1,0,-1\}$,
	doublet indicator $d_2\in\{0,1\}$ and twist sign $\tau\in\{-1,0,+1\}$,
	\[
		Y(K)=\tfrac{1}{2}+\tfrac{2}{3}s_3(K)-d_2(K)-\tfrac{1}{2}\tau(K),
		\qquad Q=T_3+Y .
	\]
\end{definition}

\begin{theorem}[Per-generation anomaly cancellation]
	With the left-chiral spectrum from $t$, the mixed and Abelian sums vanish:
	\[
		\sum_\alpha Y_\alpha\,T(R^{(\alpha)}_3)\,\dim R^{(\alpha)}_2=0,\quad
		\sum_\alpha Y_\alpha\,T(R^{(\alpha)}_2)\,\dim R^{(\alpha)}_3=0,
	\]
	\[
		\sum_\alpha Y_\alpha^3\,\dim R^{(\alpha)}_3\,\dim R^{(\alpha)}_2=0,\quad
		\sum_\alpha Y_\alpha\,\dim R^{(\alpha)}_3\,\dim R^{(\alpha)}_2=0\ (\text{grav}^2U(1)).
	\]
\end{theorem}

\begin{theorem}[Emergent Yang--Mills from swirl directors]
	Let $U(x)\in SU(3),\,V(x)\in SU(2),\,\theta(x)\in\mathbb{R}$ define
	\(
	A_\mu=\frac{i}{g_3}U^{-1}\partial_\mu U,\;
	W_\mu=\frac{i}{g_2}V^{-1}\partial_\mu V,\;
	B_\mu=\frac{1}{g_1}\partial_\mu\theta.
	\)
	Coarse-graining the director elasticity
	\(
	\mathcal{L}_{\rm dir}=\tfrac{\kappa_3}{2}\mathrm{Tr}[(\partial_\mu U)^\dagger(\partial_\mu U)]
	+\tfrac{\kappa_2}{2}\mathrm{Tr}[(\partial_\mu V)^\dagger(\partial_\mu V)]
	+\tfrac{\kappa_1}{2}(\partial_\mu\theta)^2
	\)
	yields
	\[
		\mathcal{L}_{\rm YM}^{\rm eff}=-\frac{1}{4}\sum_{i=1}^3 g_i^{-2}\,F^{(i)}_{\mu\nu}F^{(i)\,\mu\nu},
		\qquad g_i^{-2}=c_i\,\kappa_i,\ c_i>0.
	\]
\end{theorem}

\begin{definition}[EW breaking (SST stance)]
	Keep the canonical YM+Higgs form with $m_W=\tfrac{1}{2}g_2v_\Phi,\
	m_Z=\tfrac{1}{2}\sqrt{g_2^2+g_1^2}\,v_\Phi,\ m_\gamma=0$.
	The scale $v_\Phi$ is empirically anchored; any SST derivation of
	$v_\Phi(\rho_f,r_c,\|\mathbf{v}_{\!\boldsymbol{\circlearrowleft}}\|)$ remains Research-track until it reproduces $246.22$ GeV.
\end{definition}

%================================================
\section{Wave–Particle Duality in SST}
%================================================
\label{sec:wave_particle_duality}
The dual phases introduced in Axiom 5 formalize the wave–particle duality in SST. An unknotted swirl string in R-phase behaves as a coherent circulation wave (delocalized, diffraction-capable), whereas a knotted T-phase is localized and particle-like. We outline how standard quantum phenomena emerge from these phases:

\paragraph{de Broglie relation from circulation.}
    Consider a ring-like swirl string of radius $R$ carrying circulation $\Gamma = n h/m_e$ (assuming $m_{\text{eff}}=m_e$ for an electron). The tangential momentum is $p_\theta \approx m_e v_\theta$. Quantization gives $v_\theta = n \hbar/(m_e 2\pi R)$, hence
    \[
        p_\theta = \frac{n\,h}{2\pi R}\,.
    \]
    The de Broglie wavelength $\lambda = h/p_\theta$ follows as
    \[
        \lambda = \frac{2\pi R}{n}\,,
    \]
    i.e. the circumference $2\pi R$ is an integer multiple of the wavelength, consistent with wave coherence around the loop.

\paragraph{Interference and R→T collapse.}
    In a double-slit experiment, an electron's swirl string travels in R-phase through both slits as a distributed vortex loop. The intensity pattern arises from the phase difference of the two path segments around the loop, yielding interference fringes. No \emph{which-way} information is embedded in the R-phase itself. If a detection attempt (e.g. a photon scattering) forces a T-phase localization, the swirl string knots or collapses to one side, appearing particle-like at a single slit.

\paragraph{Photon-induced collapse (measurement).}
    A photon of frequency $\omega$ impinging on an R-phase swirl loop can deposit energy $\hbar\omega$. If this matches the gap $\Delta E_{\text{eff}}$ between the delocalized state and the nearest knotted state, it triggers
    \[
        \hbar\omega \approx \Delta E_{\text{eff}}\,,
    \]
    causing the loop to knot (transition to T-phase). Thus measurements involving photons inherently induce collapse by exciting the swirl into a localized mode.

\paragraph{Fringe visibility decay.}
    Environmental interactions cause gradual swirl-string collapse. If $\Gamma_{\text{collapse}}$ is the net knotting rate (transitions per second from R to T), the interference fringe visibility decays as
    \[
        V(t) = \exp(-\,\Gamma_{\text{collapse}}\,t)\,,
    \]
    analogous to decoherence with a coherence time $\tau_c = \Gamma_{\text{collapse}}^{-1}$. Low-noise experiments correspond to $\Gamma_{\text{collapse}}\to 0$ (long-lived R-phase), preserving interference, whereas any information leak ($\Gamma_{\text{collapse}}>0$) will diminish $V$.

%================================================
\section{Notation, Ontology, and Glossary}
%================================================
\label{sec:glossary}
\begin{itemize}
\item \textbf{Absolute time (A-time):} The universal time parameter $t$ of the swirl condensate (preferred foliation).
\item \textbf{Chronos time (C-time):} Time measured at infinity or far outside any swirl field ($dt_\infty$).
\item \textbf{Swirl Clocks:} Local proper-time scale factors set by $\|\omegas\|$ or $\|\vswirl\|$ (see Sec.~\ref{sec:lagrangian}); high swirl intensity (large $\omega$) slows down these clocks relative to A-time.
\item \textbf{R-phase vs. T-phase:} ``Ring'' phase (unknotted, extended) versus ``Torus-knot'' phase (knotted, localized). R-phase excitations superpose and interfere (bosonic behavior), while T-phase excitations manifest particle individuality (fermionic behavior via topological sign rules~\cite{FinkelsteinRubinstein1968}).
\item \textbf{String taxonomy:} Leptons are associated with torus knots; quarks with chiral hyperbolic knots; gauge bosons with unknots; neutrinos with linked loops (Hopf links). Family structure and conservation laws correspond to topological properties (e.g. genus, chirality, linking number).
\item \textbf{Chirality:} Counter-clockwise (ccw) swirl orientation corresponds to matter; clockwise (cw) corresponds to antimatter, through the sign of swirl–gravity interaction.
\end{itemize}

%================================================
\section{Unknot Bosons and Lossless Swirl Radiation}
%================================================
\label{sec:bosons_photons}
\paragraph{Postulate (Topological sector).}
    Let $\mathcal{U}$ denote an \emph{unknotted} closed swirl string ($\mathcal{H}=0$ Hopf invariant). Finkelstein–Rubinstein single-valuedness on multi-string configuration space enforces integer spin for $\mathcal{U}$~\cite{FinkelsteinRubinstein1968}:
    \[
        \boxed{\,\mathcal{U} \implies \text{bosonic sector}\,.}
    \]
    (Nontrivial knot classes supply the topological phase needed for half-integer spin.)

\paragraph{Field variables and lossless propagation.}
    Introduce a transverse swirl potential $\mathbf{a}(\mathbf{x},t)$ such that
    \[
        \mathbf{v} = \partial_t \mathbf{a}, \qquad
        \mathbf{b} = \nabla\times \mathbf{a}, \qquad \nabla\cdot \mathbf{a}=0\,.
    \]
    Consider the quadratic Lagrangian
    \[
        \mathcal{L}_{\text{wave}} = \frac{\rhof}{2}\,|\mathbf{v}|^2 - \frac{\rhof c^2}{2}\,|\mathbf{b}|^2\,,
    \]
    where $c$ is the observed luminal wave speed (from Axiom 1). The Euler–Lagrange equations give a lossless wave equation:
    \[
        \boxed{\,\partial_t^2 \mathbf{a} - c^2\,\nabla\times(\nabla\times \mathbf{a}) = 0, \qquad \nabla\cdot \mathbf{a}=0\,}\,,
    \]
    with conserved energy density $u$ and Poynting flux $\mathbf{S}$:
    \[
        u = \frac{\rhof}{2}\Big(|\mathbf{v}|^2 + c^2|\mathbf{b}|^2\Big), \qquad
        \mathbf{S} = \rhof c^2\,(\mathbf{v}\times \mathbf{b}), \qquad
        \partial_t u + \nabla\cdot \mathbf{S}=0\,,
    \]
    and momentum density $\mathbf{g}=\mathbf{S}/c^2$. Inviscid, dissipation-free background (Kelvin’s theorem) means no swirl circulation is lost; the waves propagate without attenuation~\cite{Batchelor1967,Saffman1992}.

\paragraph{Photon identification.}
    We identify electromagnetic field variables by the linear mapping
    \[
        \boxed{\,\mathbf{E} = \sqrt{\frac{\rhof}{\varepsilon_0}}\,\mathbf{v}, \qquad
        \mathbf{B} = \sqrt{\frac{\rhof}{\varepsilon_0}}\,\mathbf{b}\,}\,,
    \]
    yielding
    \[
        u = \frac{\varepsilon_0}{2}|\mathbf{E}|^2 + \frac{1}{2\mu_0}|\mathbf{B}|^2, \qquad
        \mathbf{S} = \frac{1}{\mu_0}\,(\mathbf{E}\times \mathbf{B}), \qquad
        \frac{1}{\varepsilon_0 \mu_0} = c^2\,,
    \]
    exactly the Maxwell energy–momentum in vacuum~\cite{Jackson1999}. Plane- and spherical-wave solutions of $\mathcal{L}_{\text{wave}}$ thus describe \emph{photons} as \emph{delocalized}, divergence-free swirl oscillations.

\paragraph{Quantization and single-photon amplitude.}
    Quantizing a cavity mode (volume $V$, frequency $\omega$) gives the standard one-photon field amplitude
    \[
        E_{\text{rms}}^{(1)} = \sqrt{\frac{\hbar\omega}{2\varepsilon_0 V}}\,,
    \]
    hence swirl velocity amplitude
    \[
        \boxed{\,v_{\text{rms}}^{(1)} = \sqrt{\frac{\hbar\omega}{2\,\rhof\,V}}\,}\,.
    \]
    For $\lambda = 532~\mathrm{nm}$ (green, $\omega=2\pi c/\lambda$) and $\rhof=7.0\times10^{-7}\,\mathrm{kg/m^3}$:
    \[
        V=1~\mathrm{mm}^3: \quad v_{\text{rms}}^{(1)} \approx 3.27\times10^{-2}\ \mathrm{m/s}\,,
    \]
    consistent with $E_{\text{rms}}^{(1)}$ and observed cavity QED couplings~\cite{HarocheRaimond2006,ScullyZubairy1997}.

\paragraph{Radiation from bound strings (``atoms'').}
    A localized bound swirl configuration with time-varying multipole moment $\mathbf{d}(t)$ launches outward transverse $\mathbf{a}$ waves. Far from the source ($r \gg$ source size), the solution is
    \[
        \mathbf{a}(\mathbf{x},t) \;\propto\; \frac{\mathbf{e}_\perp}{r}\,\Re\!\big(e^{i(kr-\omega t)}\big), \qquad k=\omega/c\,,
    \]
    with flux $\mathbf{S}=\rhof c^2\,(\mathbf{v}\times\mathbf{b})$ directed radially and $|\mathbf{S}|\propto r^{-2}$, ensuring constant power through spheres~\cite{Jackson1999}. Thus, \emph{atoms (knotted strings) emit concentric swirl waves; the lossless medium transmits them without attenuation.}

\paragraph{Exclusion of smoke-ring photons.}
    A localized vortex-ring (smoke ring) of core radius $\rc$ and energy $E_{\text{vr}}$ carrying momentum $p_{\text{vr}}$ cannot simultaneously satisfy $E_{\text{vr}}=\hbar\omega$ and $p_{\text{vr}}=\hbar k$ with subluminal core speed~\cite{Saffman1992,Batchelor1967}. Hence, \emph{photons in vacuum are not toroidal vortex rings}, but rather extended swirl modes as above.

\paragraph{Summary.}
    \[
        \boxed{\,\mathcal{U} \ (\text{unknot}) \implies \text{boson}; \qquad
        \text{photons} = \text{delocalized, lossless swirl waves launched by bound sources.}\,}
    \]

%===========================================================
\subsection{Photon as a Pulsed Unknot with Delocalized Circulation}
%===========================================================
\label{sec:photon_pulsed_unknot}
We can model the photon as a pulsed, unknotted swirl-string $K \cong S^1$ of radius $R$ (circumference $L=2\pi R$). Unlike massive particles (localized knots with core density $\rhom$), the photon has no rest-mass contribution ($\rhom=0$); its energy resides entirely in oscillatory swirl motion within the effective fluid $\rhof$.

\paragraph{Effective 1D action.}
    Define a transverse displacement $\xi(s,t)$ along the ring (parametrized by $s\in[0,L)$) with cross-sectional area $A_{\mathrm{eff}}=\pi w^2$. The photon's delocalized mode is described by
    \[
        S[\xi] = \frac{1}{2}\rhof A_{\mathrm{eff}} \int dt \int_0^L ds\,\Big[(\partial_t \xi)^2 - c^2 (\partial_s \xi)^2\Big]\,,
    \]
    yielding the wave equation
    \[
        \partial_t^2 \xi - c^2\,\partial_s^2 \xi = 0,\qquad \xi(s+L,t) = \xi(s,t)\,.
    \]

\paragraph{Normal modes.}
    Periodic boundary conditions give discrete wavenumbers
    \[
        k_m = \frac{2\pi m}{L},\qquad \omega_m = c\,k_m,\qquad m\in\mathbb{Z}_{>0}\,.
    \]
    A single-mode solution:
    \[
        \xi_m(s,t) = a_m \cos(k_m s - \omega_m t)\,.
    \]

\paragraph{Mode energy.}
    The time-averaged energy of mode $m$ is
    \[
        E_m = \frac{1}{2}\,\rhof A_{\mathrm{eff}} L\,\omega_m^2 a_m^2\,,
    \]
    which depends on the delocalized volume $A_{\mathrm{eff}}L$ rather than a massive core. Thus, photon energy is carried by the distributed swirl field, not a localized mass density.

\paragraph{Quantization.}
    Assigning energy $\hbar \omega_m$ to each mode yields amplitude
    \[
        a_m = \sqrt{\frac{\hbar}{\rhof A_{\mathrm{eff}} L \,\omega_m}}\,.
    \]
    For a photon of wavelength $\lambda$, set $R=\lambda/(2\pi)$ so $L=\lambda$, and choose $w\sim\lambda/(2\pi)$ (so $A_{\mathrm{eff}}=\pi w^2$):
    \[
        a = \sqrt{\frac{\hbar}{\rhof\,A_{\mathrm{eff}} \,L \,\omega}} = \sqrt{\frac{\hbar}{\rhof \,\pi w^2\, \lambda \,\omega}}\,,
    \]
    with $\omega = 2\pi c/\lambda$. For $\lambda = 500\,\mathrm{nm}$ and $\rhof = 7.0 \times 10^{-7}\,\mathrm{kg/m^3}$,
    \[
        a \approx 2.0 \times 10^{-12}\,\mathrm{m}\,, \qquad E = \hbar \omega \approx 3.97 \times 10^{-19}\,\mathrm{J}\;\; (2.48\,\mathrm{eV})\,.
    \]

\paragraph{Interpretation.}
    The photon is thus a \emph{pulsed unknot swirl-string}, with vanishing rest-mass density ($\rhom=0$) but finite distributed energy density
    \[
        \rhoE = \tfrac{1}{2}\rhof\!\Big((\partial_t \xi)^2 + c^2 (\partial_s \xi)^2\Big)
    \]
    integrated over its volume. It is neither pointlike nor bound to a core, but consists of a minimal swirl loop excited to launch delocalized waves—akin to a momentarily perturbed vortex ring radiating ripples in a fluid.

%================================================
\section{Canonical Checks (Verification in Practice)}
%================================================
\label{sec:canonical_checks}
\begin{enumerate}
\item \textbf{Dimensional analysis:} Verify SI consistency for every new term and equation introduced.
\item \textbf{Limiting cases:} Show that low-swirl limits ($\|\omegas\|\to 0$) recover classical mechanics and Maxwell electrodynamics; large-scale averages reproduce Newtonian gravity with $G_{\text{swirl}}$.
\item \textbf{Numerical evaluation:} Provide numeric factors using Canon constants (Sec.~\ref{sec:canonical_constants}) for any new formula. If new constants are needed, add them to Sec.~\ref{sec:canonical_constants} for consistency.
\item \textbf{Topology–quantum mapping:} Explicitly state which knot invariants correspond to which quantum numbers and how they are normalized (linking number $\leftrightarrow$ baryon number, etc.).
\item \textbf{Citations:} Cite any non-original constructs or standard results (e.g. Kelvin’s theorem, Planck’s law) using the provided BibTeX keys.
\end{enumerate}

%================================================
\section{Swirl Hamiltonian Density (Canonical Form)}
%================================================
\label{sec:hamiltonian}
Given effective density $\rhof$ and swirl vorticity $\omegas = \nabla\times \vswirl$, a Kelvin-compatible, dimensionally consistent Hamiltonian density is:
\begin{equation}
\mathcal{H}_{\text{SST}}[\vswirl] = \frac{1}{2}\rhof\,\|\vswirl\|^2 + \frac{1}{2}\rhof\,\rc^2\,\|\omegas\|^2 + \frac{1}{2}\rhof\,\rc^4\,\|\nabla\omegas\|^2 + \lambda\,(\nabla\cdot \vswirl)\,,
\label{eq:Hamiltonian_SST}
\end{equation}
which has units of energy density (J/m$^3$). The first term is kinetic energy of swirl motion; the second term $\sim \rc^2 \|\omegas\|^2$ represents core rotational energy (rest-mass analogue); the third term $\sim \rc^4\|\nabla\omegas\|^2$ penalizes curvature of the vortex filaments (string tension). In the limit $\rc\to 0$ or for spatially uniform vorticity, the higher-order terms vanish, reducing $\mathcal{H}_{\text{SST}}$ to the usual fluid kinetic energy density $\frac{1}{2}\rhof v^2$ with incompressibility constraint.

%================================================
\section{Swirl Pressure Law (Euler Corollary)}
%================================================
\label{sec:darkpressure}
For a steady, purely azimuthal flow ($v_r=v_z=0$, $\partial_t=0$), the radial component of the Euler momentum equation ($\rhof\,v_\theta^2/r = dp_{\text{swirl}}/dr$) provides a direct relationship for the swirl pressure gradient:
\begin{equation}
\frac{1}{\rhof}\frac{dp_{\text{swirl}}}{dr} = \frac{v_\theta(r)^2}{r}\,.
\label{eq:swirl_pressure_law}
\end{equation}
This is a canonical theorem derived directly from first principles. For a system exhibiting an asymptotically flat rotation curve where $v_\theta(r) \to v_0$ for large $r$, the pressure profile is found by integration:
\begin{equation}
p_{\text{swirl}}(r) = p_0 + \rhof v_0^2 \ln\left(\frac{r}{r_0}\right)\,.
\end{equation}
Here, $p_0$ is the pressure at a reference radius $r_0$. The resulting outward-rising pressure creates an inward-pointing force ($-\nabla p_{\text{swirl}}$), providing the centripetal acceleration required to maintain the flat rotation curve.

%================================================
\section{Experimental Protocols (Canon-Ready Tests)}
%================================================
\label{sec:experiments}
\subsection*{Universality of $\vscore = f\,\Delta x$ (multi-platform metrology)}
    \textit{(From \texttt{ExperimentalValidationOfVortexCoreTangentialVelocity.tex})}—In diverse systems (magnetized plasmas, superconducting vortices, optical ring modes, acoustic vortices), measure a natural frequency $f$ and a spatial period $\Delta x$ of a standing or traveling swirl mode. Verify:
    \begin{equation}
    \boxed{\,\vscore = f\,\Delta x \approx 1.09384563\times10^{6}\ \mathrm{m/s}\,}\,. \tag{X1}
    \end{equation}
    Achieve sub-ppm agreement across platforms; report mean and standard deviation. This confirms a universal quantum of circulation speed.

\subsection*{Swirl-induced gravitational potential}
    \textit{(From \texttt{ExperimentalValidationOfGravitationalPotential.tex})}—Infer $p_{\text{swirl}}(r)$ from centripetal balance (\S\ref{sec:darkpressure}) and compare predicted forces with measured thrust or buoyancy anomalies in shielded high-voltage/coil experiments (geometry: starship/Rodin coils). Ensure dimensional consistency and calibrate only via Canon constants.

%=========================================================
\section{Critical Questions Across SST Extensions}
%=========================================================
\label{sec:critical-questions}
We collect here several forward-facing questions for Swirl–String Theory (SST), posed as critical tests or extensions. Each is answered canonically, with experimental or theoretical implications.

\subsection*{1. Is EMF quantization observable?}
    If each reconnection or knotting event releases a discrete flux impulse $\Phi^*$, then
    \[
        \Delta \Phi_{\mathcal{C}} = \int_{\Sigma(\mathcal{C})} \Delta \mathbf{B} \cdot d\mathbf{S} = m\,\Phi^*,\quad m\in\mathbb Z,
    \]
    should appear as a \emph{quantized step} in a superconducting interferometer.
    For a pickup loop of inductance $L$,
    \[
        \Delta I=\frac{\Delta\Phi}{L},\qquad V_{\textrm ind}(t)=-\frac{d\Phi}{dt}.
    \]
    A reconnection of duration $\tau$ requires bandwidth $f_{\textrm BW}\gtrsim (2\pi\tau)^{-1}$. For $\tau\sim$ ns, this is 20--200 MHz, within modern SQUID ranges. If $\Phi^*\approx\Phi_0=h/2e$, steps are resolvable.

    \paragraph{Status.} Canonical prediction: \emph{yes}, observable as quantized steps; scale of $\Phi^*$ to be calibrated empirically.

\subsection*{2. Is $R\!\to\!T$ collapse deterministic or stochastic?}
Field equations (Euler + swirl coupling) are deterministic. However, topology change occurs at core separations $\sim r_c$ with extreme sensitivity to microstates. Effective model:
\[
    \dot K=F(K)+\sqrt{2D_{\textrm env}}\,\eta(t),\qquad \text{Vis}(t)=e^{-\Gamma_{\textrm env} t},\;\Gamma_{\textrm env}\propto D_{\textrm env}.
\]
Thus: deterministic instability in the $D_{\textrm env}\!\to 0$ limit; effectively stochastic under environmental noise.

\paragraph{Status.} Collapse is \emph{deterministic at the core level}, \emph{stochastic in practice}.

\subsection*{3. Can SST replace quantum measurement postulates?}
The dual-phase picture (delocalized $R$ vs localized $T$ states) suggests an objective collapse mechanism. Collapse is driven by swirl radiation and reconnections.
Key tasks:
\begin{itemize}
\item Derive the Born rule $P\sim|\psi|^2$ from ergodic measures on swirl phase space.
\item Ensure no-signaling under nonlocal correlations of linked knots.
\end{itemize}

\paragraph{Status.} Promising realist alternative, but \emph{derivation of Born and no-signaling remains open}.

\subsection*{4. How unique is the topological decomposition?}
Different knots $K$ can share
\[
    \mathcal M(K)=b(K)^{-3/2}\,\varphi^{-g(K)}\,n(K)^{-1/\varphi}\,L_{\textrm tot}(K).
\]
Thus mass alone is degenerate. Resolution requires:
\begin{itemize}
\item Helicity $H=\int \mathbf v\!\cdot\!\boldsymbol\omega\,dV$,
\item writhe/twist spectra and normal-mode eigenfrequencies,
\item stability (lifetime) and selection rules.
\end{itemize}

\paragraph{Status.} Unique particle identity emerges only when \emph{mass, helicity, and mode spectra} are jointly enforced.

\subsection*{5. Can the swirl Lagrangian generate interactions?}
Beyond mass, interaction terms may appear via extended couplings:
\[
    \mathcal L_{\textrm couple}\;+\;\lambda_{\textrm ch}\int (\mathbf v\!\cdot\!\boldsymbol\omega)(\nabla\!\cdot\!\mathbf a)\,dV
    +\;g_c \int \mathcal C(\mathcal K_1,\mathcal K_2)\,d\Sigma.
\]
These generate parity-odd (chiral) and contact vertices, reminiscent of Yukawa/weak interactions.

\paragraph{Status.} Plausible EFT tower; \emph{explicit vertex catalogue} is an open derivation.

\subsection*{6. Is the swirl condensate Lorentz-violating?}
SST posits absolute time (preferred foliation). Microscopic frame is Galilean.
However, the photon sector Lagrangian
\[
    \partial_t^2 \mathbf a - c^2 \nabla\times(\nabla\times \mathbf a)=0,\quad \nabla\cdot\mathbf a=0
\]
is exactly Lorentz-invariant. Residual anisotropies are suppressed operators of order $\epsilon^2$ with $\epsilon=\|\mathbf u_{\textrm drift}\|/c$.
Experimental bounds: $\delta c/c \lesssim 10^{-17}$--$10^{-21}$; SST must respect these.

\paragraph{Status.} \emph{Emergent Lorentz invariance} in radiation sector; matter sector constrained to high precision.


%================================================
\section{Limitations and Scope}
%================================================
\label{sec:limitations}

\begin{tcolorbox}[colback=yellow!6!white,colframe=yellow!40!black,title={\Large\warning~ Limitations}]
	\begin{itemize}
		\item \textbf{Speculative status.} SST remains theoretical; no direct experimental confirmations yet validate the swirl substratum or knot–particle correspondence.
		\item \textbf{Absolute time.} The postulated absolute time is philosophically and empirically contentious. While the radiation sector exhibits \emph{emergent} Lorentz invariance, the matter sector must satisfy stringent bounds on Lorentz violation; these constraints are under active review.
		\item \textbf{Gauge reinterpretation.} The mapping of knots to \(\mathrm{SU}(3)\oplus\mathrm{SU}(2)\oplus\mathrm{U}(1)\) representations is currently in the \emph{Research Track} (non-canonical). Promotion requires anomaly cancellation, correct hypercharge assignments, and empirical coupling fits.
		\item \textbf{Accessibility.} The Canon is technically dense and presumes familiarity with both vortex dynamics (Kelvin/Helmholtz) and Yang–Mills/electroweak theory.
	\end{itemize}
\end{tcolorbox}

\paragraph{Governance.}
Items flagged \emph{Research Track} are non-canonical per Sec.~\ref{sec:canon_governance}.
They are retained for traceability and future calibration; they are not used as premises in canonical proofs.


\subsection*{Experiment table}

\begin{table}[H]
\centering
\caption{Concrete experiments to test critical SST questions.}
\begin{tabular}{lllll}
\toprule
Objective & Observable & SST control & Expected scale & Note \\
\midrule
Flux impulse & $\Delta\Phi$ steps & reconnections & $\Phi^*\sim\Phi_0$? & SQUID; $f_{\textrm BW}$ ns--µs \\
Collapse rate & vis. $\sim e^{-\Gamma t}$ & env. coupling $D_{\textrm env}$ & tunable $\Gamma$ & Kramers escape analogue \\
Identity & spectra, $H$ & knot class $K$ & discrete $\Omega_n$ & inelastic spectroscopy \\
Interactions & scattering & contact twist/link & selection rules & EFT vertex test \\
Lorentz & $\delta c/c$ & drift $\mathbf u$ & $<10^{-17}$ & cavity/clock tests \\
\bottomrule
\end{tabular}
\end{table}

\paragraph{Governance Note.}
Definitions of $\varepsilon_*$, $\mathcal B[K]$, and $S_{\textrm comp}$ are Canonical.
Their \emph{interpretation as renormalizing} $g_{2,3}$ and the representation map
$t(K)\to(\mathrm{SU}(3),\mathrm{SU}(2),Y)$ are \emph{Research} until anomaly cancellation
and empirical coupling fits are demonstrated.

\section*{Canon 4R: Research Extensions (Non-Canonical)}
The following conjectural relations are recorded for future calibration. They are
dimensionally consistent but not yet anchored to empirical values.
\begin{itemize}
	\item $v_\Phi^{\rm SST} \sim \sqrt{\rho_{\!f}}\, r_c \lVert \vswirl \rVert / \hbar$
	\item Swirl-helicity $\times$ Chern–Simons couplings
	\item Knot $\to$ gauge representation map beyond anomaly checks
\end{itemize}

% ================== Begin Operational Kinematics ==================

% === Composite-knot correction (SST) ===
For a composite $K_1 \# K_2$,
\[
    u(K_1 \# K_2)
    = u(K_1) + u(K_2) - \Delta_u(K_1,K_2),
    \qquad \Delta_u \ge 0.
\]

Hence the barrier functional satisfies
\[
    \mathcal B[K_1 \# K_2]
    = \mathcal B[K_1] + \mathcal B[K_2] - \varepsilon_*\,\Delta_u.
\]

We define a dimensionless \emph{simplification index}:
\[
    S_{\textrm comp}(K_1,K_2)
    = \frac{\Delta_u}{u(K_1)+u(K_2)} \in [0,1).
\]

This index measures the degree to which composition
reduces the unknotting barrier.

\medskip

In SST taxonomy, the correction term
\[
    \Delta\mathcal B = \varepsilon_* \Delta_u
\]
acts as a \emph{nonlinear coupling},
analogous to the non-Abelian structure constants
in $\mathfrak{su}(3)\oplus\mathfrak{su}(2)\oplus\mathfrak{u}(1)$.

% ================== End Operational Kinematics ==================

% =========================================================
% SST: Invariant Mass from the Canonical Lagrangian
% =========================================================

\section*{Appendix C: Invariant Mass from the Canonical Lagrangian}

Starting from the schematic Lagrangian
\[
    \mathcal{L}_{\text{SST}}
    = \rhof\!\left(\tfrac{1}{2}\vswirl^2 - \Phi_{\text{swirl}}\right)
    + \tfrac{1}{4}F_{\mu\nu}F^{\mu\nu}
    + \big(\alpha C(K)+\beta L(K)+\gamma \mathcal{H}(K)\big)
    + \rhof \ln\!\sqrt{1-\tfrac{\|\boldsymbol\omega\|^2}{c^2}}
    + \Delta p(\text{swirl}),
\]
the \emph{mass sector} reduces, under the slender-tube approximation, to an invariant energy functional
\[
    E(K)= u\,V(K)\,\Xi_{\text{top}}(K),\qquad
    u=\tfrac{1}{2}\rho_{\text{core}}\;v_{\circlearrowleft}^{2},
\]
with $u$ the swirl energy density scale on the core, $V(K)$ the effective tube volume of the swirl string, and $\Xi_{\text{top}}(K)$ a dimensionless topological multiplier summarizing discrete combinatorial and contact/helicity corrections. In SST we adopt
\[
    V(K)\;=\;\pi r_c^2 \underbrace{\big(L_{\textrm phys}\big)}_{=\,r_c\,L_{\textrm tot}}
    \;=\;\pi r_c^3\,L_{\textrm tot},
\]
where $r_c$ is the core radius and $L_{\textrm tot}$ is the \emph{dimensionless ropelength}. The rest mass is $M=E/c^2$.

\paragraph{Canonical multiplier.}
    Guided by the EM coupling and SST’s discrete scaling rules, we take
    \[
        \Xi_{\text{top}}(K)=\frac{4}{\alpha_{\textrm fs}}\;b^{-3/2}\;\varphi^{-g}\;n^{-1/\varphi},
    \]
    where $b,g,n$ are the integer topology labels used in the Canon (e.g. torus index, layer, linkage count), $\alpha_{\textrm fs}$ is the fine-structure constant, and $\varphi$ the golden ratio. Collecting factors, the \textbf{invariant mass law} used in the code is
    \begin{equation*}
    \boxed{M(K)=\frac{4}{\alpha_{\textrm fs}}\;b^{-3/2}\;\varphi^{-g}\;n^{-1/\varphi}\;
    \frac{u\,\pi r_c^3 L_{\textrm tot}}{c^2},
        \qquad
        u=\tfrac{1}{2}\rho_{\text{core}}v_{\circlearrowleft}^2.
    }\label{eq:SST-invariant-mass}
    \end{equation*}

\paragraph{Leptons (solved $L_{\textrm tot}$).}
    For a lepton with labels $(b,g,n)$ and known mass $M_\ell^{\textrm(\exp)}$, invert \eqref{eq:SST-invariant-mass}:
    \[
        L_{\textrm tot}^{(\ell)} \;=\;
        \frac{M_\ell^{\textrm(\exp)}\,c^2}{\big(\tfrac{4}{\alpha_{\textrm fs}}\,b^{-3/2}\varphi^{-g}n^{-1/\varphi}\big)\,u\,\pi r_c^3}.
    \]

\paragraph{Baryons (exact closure).}
    Let the proton and neutron ropelengths be
    \[
        L_p=\lambda_b\,(2s_u+s_d)\,\mathcal S,\qquad
        L_n=\lambda_b\,(s_u+2s_d)\,\mathcal S,\qquad
        \mathcal S=2\pi^2\kappa_R,\;\;\kappa_R=2,
    \]
    with $(s_u,s_d)$ dimensionless sector weights and $\lambda_b$ a sector scale (set to $1$ in exact-closure).
    Imposing $M_p^{\textrm(\exp)}=M_p$ and $M_n^{\textrm(\exp)}=M_n$ in \eqref{eq:SST-invariant-mass} yields a \emph{linear} $2\times2$ system for $(s_u,s_d)$:
    \[
        \begin{bmatrix}
        2 & 1\\[2pt]
        1 & 2
        \end{bmatrix}
        \begin{bmatrix}
        s_u\\ s_d
        \end{bmatrix}

        =
        \frac{1}{K}
        \begin{bmatrix}
        M_p^{\textrm(\exp)}\\ M_n^{\textrm(\exp)}
        \end{bmatrix},
        \qquad
        K=\Big[\tfrac{4}{\alpha_{\textrm fs}}\,3^{-3/2}\,\varphi^{-2}\,3^{-1/\varphi}\Big]\frac{u\,\pi r_c^3\,\mathcal S}{c^2}.
    \]
    Solving gives
    \[
        s_u=\frac{2M_p^{\textrm(\exp)}-M_n^{\textrm(\exp)}}{3K},
        \qquad
        s_d=\frac{M_p^{\textrm(\exp)}}{K}-2s_u.
    \]

\paragraph{Composites (no binding).}
    For an atom with proton number $Z$ and neutron number $N$ (atomic mass includes $Z$ electrons),
    \[
        M_{\textrm atom}^{(\textrm pred)} = Z\,M_p+N\,M_n+Z\,M_e,\quad
        M_{\textrm mol}^{(\textrm pred)}=\sum_{\text{atoms}}M_{\textrm atom}^{(\textrm pred)}.
    \]
    Deviations from experiment in atoms/molecules correspond to \emph{binding energies} not included in this baseline (nuclear $\sim\!8\,{\textrm MeV}$ per nucleon; molecular $\sim{\textrm eV}$).

% ---------------------------------------------------------
\subsection{Benchmarks (exact\_closure mode)}
\label{sec:benchmarks-exact-closure}
The following table was generated by the Python file listed after it.
\emph{Errors in atoms/molecules = missing binding energy contribution, not model failure.}

\begin{table}[H]
    \centering
    \caption{Invariant-kernel mass benchmarks (exact\_closure). \emph{Errors in atoms/molecules = missing binding energy contribution, not model failure.}}
    \begin{tabular}{lccc}
        \toprule
        Species & Known mass (kg) & Predicted mass (kg) & Error (\%)\\
        \midrule
        electron e- & 9.109384e-31 & 9.109384e-31 & 0.0000\\
        muon $\mu$- & 1.883532e-28 & 1.883532e-28 & 0.0000\\
        tau $\tau$- & 3.167540e-27 & 3.167540e-27 & 0.0000\\
        proton p & 1.672622e-27 & 1.672622e-27 & 0.0000\\
        neutron n & 1.674927e-27 & 1.674927e-27 & 0.0000\\
        Hydrogen-1 atom & 1.673533e-27 & 1.673533e-27 & 0.0000\\
        Helium-4 atom & 6.646477e-27 & 6.689952e-27 & 0.6549\\
        Carbon-12 atom & 1.992647e-26 & 2.005276e-26 & 0.6330\\
        Oxygen-16 atom & 2.656017e-26 & 2.674532e-26 & 0.6980\\
        H$_2$ molecule & 3.367403e-27 & 3.347066e-27 & -0.6040\\
        H$_2$O molecule & 2.991507e-26 & 3.009885e-26 & 0.6139\\
        CO$_2$ molecule & 7.305355e-26 & 7.354340e-26 & 0.6704\\
        \bottomrule
    \end{tabular}\label{tab:benchmarks-exact-closure}
\end{table}

% ---------------------------------------------------------
\subsection*{Notes}
\begin{itemize}
    \item Elementary entries are exact by construction in exact\_closure mode (leptons solved from $L_{\textrm tot}$; $p,n$ from closure).
    \item Composite errors track omitted binding: nuclear $\mathcal O(10^{-3})$–$\mathcal O(10^{-2})$, molecular $\mathcal O(10^{-9})$.
\end{itemize}

% ---------------------------------------------------------

% === Unknotting barrier & gauge structure (SST) ===
\subsection*{From unknotting non-additivity to \(\mathfrak{su}(3)\oplus\mathfrak{su}(2)\oplus\mathfrak u(1)\) in SST}

\paragraph{Unknotting barrier functional.}
    Let \(K\subset\mathbb R^3\) be a closed swirl–string on a leaf \(\Sigma_t\).
    Define the per–crossing activation scale
    \[
        \varepsilon_*\;=\;\kappa\,\beta\,\rc\;+\;\frac{\kappa\pi}{2}\,\rhof\,\vnorm^2\,\rc^3,
        \quad \kappa=\mathcal O(1{-}10),
    \]
    and the barrier
    \[
        \mathcal B[K]\;=\;u(K)\,\varepsilon_*.
    \]
    For a connected sum \(K_1\#K_2\),
    \[
        u(K_1\#K_2)=u(K_1)+u(K_2)-\Delta_u(K_1,K_2),\qquad \Delta_u\ge 0,
    \]
    so that
    \[
        \mathcal B[K_1\#K_2]=\mathcal B[K_1]+\mathcal B[K_2]-\varepsilon_*\,\Delta_u.
    \]
    (Brittenham–Hermiller 2025 give explicit \(\Delta_u>0\) families for torus knots.)

\paragraph{Dimensionless simplification index.}
    Define
    \[
        S_{\textrm comp}(K_1,K_2):=\frac{\Delta_u(K_1,K_2)}{u(K_1)+u(K_2)}\in[0,1),
    \]
    and \(\Delta\mathcal B=\varepsilon_*\,\Delta_u\). Here \(S_{\textrm comp}\) is \emph{dimensionless}; \(\Delta\mathcal B\) has units of energy.

\paragraph{Discrete curvature on composition.}
    Let \((\mathcal K,\#)\) denote the knot configuration monoid. Non-additivity of \(u\) defines a discrete \(2\)-cocycle:
    \[
        \mathcal C(K_1,K_2):=\Delta_u(K_1,K_2).
    \]
    A minimal \emph{associator defect} (discrete curvature) for triples is
    \[
        \mathfrak{R}(K_1,K_2,K_3)\;:=\;
        \mathcal C(K_1,K_2)+\mathcal C(K_1\#K_2,K_3)
        -\mathcal C(K_2,K_3)-\mathcal C(K_1,K_2\#K_3).
    \]
    If \(\mathfrak{R}\equiv 0\) the composition is “flat” (effectively Abelian in the barrier metric); \(\mathfrak{R}\neq 0\) signals nontrivial curvature, the discrete analogue of non-Abelian structure.

\paragraph{Emergent gauge potentials from multi-director swirl.}
    Let \(\mathcal A=\big(A_\mu^{(0)},\,W_\mu^a,\,G_\mu^A\big)\) denote the swirl–gauge potentials for
    \(\mathfrak u(1)\oplus\mathfrak{su}(2)\oplus\mathfrak{su}(3)\).
    Introduce coupling functions driven by simplification statistics in a family \(\mathcal F\subset\mathcal K\):
    \[
        g_1^{-2}(\mathcal F)=g_{1,0}^{-2},\qquad
        g_2^{-2}(\mathcal F)=g_{2,0}^{-2}\Bigl[1+\lambda_2\,\langle S_{\textrm comp}\rangle_{\mathcal F}\Bigr],
        \qquad
        g_3^{-2}(\mathcal F)=g_{3,0}^{-2}\Bigl[1+\lambda_3\,\langle S_{\textrm comp}\rangle_{\mathcal F}\Bigr],
    \]
    with \(\lambda_{2,3}>0\) dimensionless and \(\langle\cdot\rangle_{\mathcal F}\) the family average.
    Thus nonzero simplification (\(\Delta_u>0\)) \emph{renormalizes} the non-Abelian sectors, while \(U(1)\) remains purely additive at leading order.

\paragraph{SST gauge Lagrangian (coupling by taxonomy).}
    With \(F_{\mu\nu}^{(0)},\,W_{\mu\nu}^a,\,G_{\mu\nu}^A\) the field strengths,
    \[
        \mathcal L_{\textrm gauge}
        =-\frac{1}{4}\frac{1}{g_1^2(\mathcal F)}\,F_{\mu\nu}^{(0)}F^{(0)\mu\nu}
        -\frac{1}{4}\frac{1}{g_2^2(\mathcal F)}\,W_{\mu\nu}^{a}W^{a\,\mu\nu}
        -\frac{1}{4}\frac{1}{g_3^2(\mathcal F)}\,G_{\mu\nu}^{A}G^{A\,\mu\nu},
    \]
    and the total selection functional acquires the barrier:
    \[
        \mathcal E_{\textrm tot}[K]=\alpha\,C(K)+\beta\,L(K)+\gamma\,\mathcal H(K)+\mathcal B[K],
        \quad
        \mathcal E_{\textrm tot}[K_1\#K_2]
        =\mathcal E_{\textrm tot}[K_1]+\mathcal E_{\textrm tot}[K_2]-\varepsilon_*\,\Delta_u.
    \]

\paragraph{Representation assignment (house mapping).}
    Adopt the Canon homomorphism \(t(K)=(L \bmod 3,\;S \bmod 2,\;\chi)\) to \(\mathfrak{su}(3)\oplus\mathfrak{su}(2)\oplus\mathfrak u(1)\):
    \[
        \begin{aligned}
        &\text{Color (SU(3)) index: } &&\mathbf 3 \text{ class determined by } (L \bmod 3),\\
        &\text{Weak (SU(2)) isospin: } &&\mathbf 2 \text{ / singlet by } (S \bmod 2),\\
        &\text{Hypercharge (U(1)): } &&Y\propto \chi\, \mathcal Q(K),
        \end{aligned}
    \]
    where \(\chi\in\{\pm1\}\) tracks chirality/mirror and \(\mathcal Q(K)\) is a chosen Abelian scalar (e.g.\ normalized circulation or writhe). This retains Abelian additivity for \(Y\) while allowing non-Abelian renormalization via \(\langle S_{\textrm comp}\rangle\).

\paragraph{Worked composite example.}
    For \(K=T(2,7)\) with \(u(K)=3\) and \(u(K\#\overline K)\le 5\),
    \[
        \Delta_u(K,\overline K)\ge 1,\qquad
        S_{\textrm comp}(K,\overline K)\ge \frac{1}{6}.
    \]
    Hence
    \[
        \Delta\mathcal B=\varepsilon_*\,\Delta_u\ge \varepsilon_*,
        \quad
        \frac{1}{g_{2,3}^2}\;\mapsto\;\frac{1}{g_{2,3}^2}\Bigl[1+\lambda_{2,3}\times\tfrac{1}{6}\Bigr]
        \ \ \text{for the family containing }K,\overline K.
    \]
    All quantities entering \(g_i(\mathcal F)\) are dimensionless (consistency check).

\paragraph{Physical interpretation.}
    Additive (\(U(1)\)) observables follow linear composition; subadditivity of \(u\) generates a discrete curvature that selectively enhances the non-Abelian sectors. Families with larger \(\langle S_{\textrm comp}\rangle\) act as stronger “non-Abelianizers” of the swirl–gauge dynamics.

% =========================
% Bibliography (non-original)
% =========================
    \begin{thebibliography}{9}
    \bibitem{BrittenhamHermiller2025}
    M.~Brittenham and S.~Hermiller,
    \emph{Unknotting number is not additive under connected sum},
    arXiv:2506.24088 (2025).
    \end{thebibliography}

%================================================
% Personas (unchanged logic, SST names)
%================================================
\section*{Appendix D: Persona Prompts}
\label{sec:personas}

\subsection*{Reviewer Persona}
    \scriptsize
    You are a peer reviewer for an SST paper. Use only the definitions and constants in the "SST Canon (\canonversion)".
    Check dimensional consistency, limiting behavior, and numerical validation. Flag any use of non-canonical
    constants or equations unless equivalence is proved. Demand explicit mapping from knot invariants (linking,
    writhe, twist) to claimed quantum numbers.

\subsection*{Theorist Persona}

    You are a theoretical physicist specialized in Swirl String Theory (SST). Base all reasoning on the attached
    "SST Canon (\canonversion)". Your task: derive the swirl-based Hamiltonian for [TARGET SYSTEM], use Sec.~\ref{sec:lagrangian},
    and verify the Swirl Clock law (Sec. \ref{sec:classical_invariants}). Provide boxed equations, dimensional checks, and a short numerical
    evaluation using the Canon constants.

\subsection*{Bridging Persona (Compare to GR/SM)}

    Work strictly within SST Canon (\canonversion). Compare [TARGET] to its GR/SM counterpart. Identify exact replacements
    (e.g., curvature → swirl), and show which terms reduce to Newtonian/Maxwellian limits. Include a correspondence
    table and any constraints needed for equivalence.


%================================================
% Session Kickoff Checklist
%================================================
    \normalsize
\section*{Appendix E: Session Kickoff Checklist}
\begin{enumerate}
    \item Start new chat per task; attach this Canon first.
    \item Paste a persona prompt (Sec.~\ref{sec:personas}).
    \item Attach only task-relevant papers/sources.
    \item State any corrections explicitly (they persist in the session).
    \item At end, record Canon deltas (if any) and bump version.
\end{enumerate}


%================================================
\section*{Appendix F: Dimensional Cross-Check for \texorpdfstring{$\mathcal L_{\text{SST+Gauge}}$}{L\_\{SST+Gauge\}}}
%================================================
\label{app:dim-check}

\noindent\textbf{Unit conventions.}
We present SI checks. Where convenient, we also note the natural-unit assignment
($\hbar=c=1$), with \([A_\mu]=\text{mass}\), \([F_{\mu\nu}]=\text{mass}^2\), ensuring \([\mathcal L]=\text{mass}^4\);
conversion to SI energy density uses \(\hbar c\).

\begin{center}
	\renewcommand{\arraystretch}{1.2}
	\begin{tabular}{@{}llcl@{}}
		\toprule
		\textbf{Term} & \textbf{Expression} & \textbf{Primary units} & \textbf{Check (SI)} \\
		\midrule
		Kinetic (swirl) &
		$\displaystyle \tfrac12\,\rhof\,\|\vswirl\|^2$ &
		$[\rhof]=\mathrm{kg\,m^{-3}},\ [\vswirl]=\mathrm{m\,s^{-1}}$ &
		$\mathrm{kg\,m^{-3}}\cdot \mathrm{m^2\,s^{-2}}=\mathrm{J\,m^{-3}}$ \\

		Swirl potential &
		$\displaystyle -\,\rhof\,\Phi_{\text{swirl}}(\mathbf r,\omegas)$ &
		$[\Phi_{\text{swirl}}]=\mathrm{m^2\,s^{-2}}$ &
		$\mathrm{kg\,m^{-3}}\cdot \mathrm{m^2\,s^{-2}}=\mathrm{J\,m^{-3}}$ \\

		Incompressibility constraint &
		$\displaystyle \lambda\,(\nabla\!\cdot\!\vswirl)$ &
		$[\nabla\!\cdot\!\vswirl]=\mathrm{s^{-1}}$ &
		$[\lambda]=\mathrm{J\,s\,m^{-3}}$ so product is $\mathrm{J\,m^{-3}}$ \\

		Helicity density (local) &
		$\displaystyle \chi_h\,\rhof\,(\vswirl\!\cdot\!\omegas)$ &
		$[\omegas]=\mathrm{s^{-1}}$ &
		$\chi_h$ dimensionless; $\mathrm{kg\,m^{-3}}\cdot \mathrm{m\,s^{-1}}\cdot \mathrm{s^{-1}}=\mathrm{J\,m^{-3}}$ \\

		Yang--Mills (gauge) &
		$\displaystyle -\tfrac14 \GsA G^{a\,\mu\nu}
		-\tfrac14 \WsI W^{i\,\mu\nu}
		-\tfrac14 \Bmn B^{\mu\nu}$ &
		$\hbar=c=1:\ [F]=\text{mass}^2$ &
		$[\mathcal L_{\rm YM}]=\text{mass}^4\ \Rightarrow\ \mathrm{J\,m^{-3}}$ via $\hbar c$ \\

		Scalar kinetic &
		$\displaystyle (D_\mu\Phi)^\dagger(D^\mu\Phi)$ &
		$\hbar=c=1:\ [D_\mu]=\text{mass}$,\ $[\Phi]=\text{mass}$ &
		$\text{mass}^4\ \Rightarrow\ \mathrm{J\,m^{-3}}$ \\

		Scalar potential &
		$\displaystyle V(\Phi)=\lambda_\Phi\big(|\Phi|^2-v_\Phi^2\big)^2$ &
		$\hbar=c=1:\ [V]=\text{mass}^4$ &
		$\text{mass}^4\ \Rightarrow\ \mathrm{J\,m^{-3}}$ \\

		Minimal coupling &
		$\displaystyle \mathcal L_{\text{int}}=-A^a_\mu J_{(3)}^{a\,\mu}-W^i_\mu J_{(2)}^{i\,\mu}-B_\mu J_{(1)}^\mu$ &
		$\hbar=c=1:\ [A_\mu]=\text{mass},\ [J^\mu]=\text{mass}^3$ &
		$\text{mass}^4\ \Rightarrow\ \mathrm{J\,m^{-3}}$ \\
		\bottomrule
	\end{tabular}
\end{center}


%================================================
\section*{Appendix F: Gauge Reinterpretation—Derivations and Checks}
\label{app:gauge_reinterpretation}
%================================================

\subsection*{F.1 Mathematical Closure: Knot $\to$ Representation Map}

\paragraph{Topological indices (computable).}
For an oriented, framed, connected swirl string $K$ (e.g.\ torus knot $T(p,q)$ or a hyperbolic knot), define three additive indices:
\begin{align*}
	c_3(K) &\in \mathbb{Z}_3
	\quad\text{(color class):}\quad
	c_3 := p \bmod 3,
	\quad
	K \mapsto \overline{K}\ \Rightarrow\ p\mapsto -p\ \Rightarrow\ c_3 \mapsto -c_3, \\
	s_2(K) &\in \mathbb{Z}_2
	\quad\text{(weak class):}\quad
	s_2 := q \bmod 2,
	\quad
	s_2(K_1\#K_2)= s_2(K_1)+s_2(K_2)\!\!\!\pmod{2}, \\
	\tau(K) &\in \{-1,0,+1\}
	\quad\text{(twist/sign class for singlets):}\quad
	\text{let } \mathrm{SL}(K)=\mathrm{Wr}(K)+\mathrm{Tw}(K) \text{ (Călugăreanu--White).}
\end{align*}
Define
\[
	\tau(K)=
	\begin{cases}
		0, & s_2(K)=1\quad(\text{SU(2) doublet, pre-split}),\\[3pt]
		\operatorname{sign}\!\big(\mathrm{SL}(K)\big)\in\{-1,+1\}, & s_2(K)=0\quad(\text{SU(2) singlet}).
	\end{cases}
\]
Mirror reversal $K\mapsto\overline{K}$ flips $\tau$.
All three indices are computable from a standard diagram (or directly from the torus pair $(p,q)$ with a chosen framing)~\cite{Calugareanu1961,White1969}.

\paragraph{Map to SM representations.}
\textbf{Color (SU(3)) from $c_3$.}
\[
	\rho_3(K)=
	\begin{cases}
		\mathbf{1}, & c_3=0,\\
		\mathbf{3}, & c_3=+1,\\
		\overline{\mathbf{3}}, & c_3=-1\ \ (\text{i.e.\ }c_3=2\ \text{mod }3).
	\end{cases}
\]
Introduce a color-sign $s_3(K)\in\{+1,0,-1\}$ by $s_3=+1$ for $\mathbf 3$, $s_3=0$ for singlet, $s_3=-1$ for $\overline{\mathbf 3}$.
\smallskip

\noindent
\textbf{Weak (SU(2)) from $s_2$.}
Doublet if $s_2=1$, singlet if $s_2=0$. Let $d_2(K)\in\{0,1\}$ be the doublet indicator.
\smallskip

\noindent
\textbf{Hypercharge (U(1)) in one closed form.}
\[
	\boxed{\quad
	Y(K)=\frac{1}{2}+\frac{2}{3}\,s_3(K)-d_2(K)-\frac{1}{2}\,\tau(K)\quad}
\]
Mirror sends $s_3\to -s_3$, $\tau\to -\tau$, so $Y$ conjugates accordingly. With $Q=T_3+Y$ this reproduces the known electric charges.

\paragraph{Sanity table (one LH generation, entirely via $(s_3,d_2,\tau)$).}
\begin{center}\small
\begin{tabular}{lcccc}
	\toprule
	field (LH Weyl) & $s_3$ & $d_2$ & $\tau$ & $Y$ \\
	\midrule
	$Q_L=(u_L,d_L)$ & $+1$ & $1$ & $0$ & $+1/6$ \\
	$u_R^c$ & $-1$ & $0$ & $+1$ & $-2/3$ \\
	$d_R^c$ & $-1$ & $0$ & $-1$ & $+1/3$ \\
	$L_L=(\nu_L,e_L)$ & $0$ & $1$ & $0$ & $-1/2$ \\
	$e_R^c$ & $0$ & $0$ & $-1$ & $+1$ \\
	$\nu_R^c$ (optional) & $0$ & $0$ & $+1$ & $0$ \\
	\bottomrule
\end{tabular}
\end{center}

\paragraph{Closure statements.}
\begin{lemma}[Monoid homomorphism to the rep ring]
	Let $t:\ (\mathcal{K},\#)\to R(\mathrm{SU}(3)\times\mathrm{SU}(2)\times\mathrm{U}(1))$ be the map $K\mapsto \big(\rho_3(K),\rho_2(K),Y(K)\big)$ defined above. Then:
	\begin{itemize}
		\item $c_3,s_2$ add mod their groups under $\#$; $\tau$ adds and is clamped in $\{-1,0,+1\}$ for singlets (doublets keep $\tau=0$).
		\item Mirror $K\mapsto\overline{K}$ maps conjugate reps; $d_2$ is unchanged (SU(2) is pseudoreal~\cite{Witten1982}).
		\item Disjoint union of components (links) is index-wise addition, acting as external tensor product.
	\end{itemize}
	Hence $t$ is a monoid homomorphism up to representation reduction.
\end{lemma}

\subsection*{F.2 Anomaly Freedom (per generation)}

Using the table above (sum over LH Weyl fields, i.e.\ include $u_R^c,d_R^c,e_R^c,\nu_R^c$ if present):
\begin{align*}
	\mathrm{SU}(3)^2\mathrm U(1):\quad
	&Q_L:\ 2\cdot Y\cdot T(\mathbf 3)=2\cdot \tfrac{1}{6}\cdot \tfrac{1}{2}=\tfrac{1}{6},\quad
	u_R^c:\ (-\tfrac{2}{3})\cdot \tfrac{1}{2}=-\tfrac{1}{3},\quad
	d_R^c:\ (+\tfrac{1}{3})\cdot \tfrac{1}{2}=+\tfrac{1}{6},\\[-2pt]
	&\text{sum }=0.\\[4pt]
	\mathrm{SU}(2)^2\mathrm U(1):\quad
	&Q_L:\ 3\cdot Y\cdot T(\mathbf 2)=3\cdot \tfrac{1}{6}\cdot \tfrac{1}{2}=\tfrac{1}{4},\quad
	L_L:\ (-\tfrac{1}{2})\cdot \tfrac{1}{2}=-\tfrac{1}{4},\ \ \text{sum }=0.\\[4pt]
	\mathrm U(1)^3,\ \mathrm{grav}^2\mathrm U(1):\quad
	&\text{standard SM sums vanish with these hypercharges, hence also vanish here~\cite{WeinbergQFT2,PeskinSchroeder}.}
\end{align*}
\begin{theorem}[Per-generation anomaly cancellation]
	For the knot$\to$rep map of \S F.1, all gauge and mixed anomalies cancel per generation.
\end{theorem}

\subsection*{F.3 Emergent Yang--Mills from Swirl Directors}

\paragraph{Multi-director order parameter and connections.}
Let $E(x)=[\mathbf e_1,\mathbf e_2,\mathbf e_3]\in\mathrm{SO}(3)$ be a triad of independent swirl directors; lift to $U(x)\in\mathrm{SU}(3)$ via a fixed embedding. Define the color connection
\[
	A_\mu := \frac{\ii}{g_3}\,U^{-1}\partial_\mu U \in \mathfrak{su}(3).
\]
Similarly, from a two-director subbundle $V(x)\in \mathrm{SU}(2)$ define
\[
	W_\mu := \frac{\ii}{g_2}\,V^{-1}\partial_\mu V \in \mathfrak{su}(2),
	\qquad
	B_\mu := \frac{1}{g_1}\,\partial_\mu \theta \in \mathfrak u(1)
\]
for a common condensate phase $\theta(x)$.

\paragraph{Director elasticity and coarse-graining.}
Consider the swirl-elastic Lagrangian (principal chiral form)\cite{Fradkin2013,Volovik2003,Cho1980,FaddeevNiemi1999}:
\[
    \mathcal L_{\text{dir}}
    =\frac{\kappa_3}{2}\,\Tr\!\big[(\partial_\mu U)^\dagger (\partial^\mu U)\big]
    +\frac{\kappa_2}{2}\,\Tr\!\big[(\partial_\mu V)^\dagger (\partial^\mu V)\big]
    +\frac{\kappa_1}{2}\,(\partial_\mu\theta)^2.
\]
Expanding $U=e^{\ii g_3 \epsilon}$ etc.\ and integrating out fast director modes at quadratic order yields an effective gauge-field kinetic energy:
\[
    \boxed{\quad
    \mathcal L_{\rm YM}^{\rm eff}
    = -\frac{1}{4}\,\frac{1}{g_3^2}\,G^a_{\mu\nu}G^{a\,\mu\nu}
    -\frac{1}{4}\,\frac{1}{g_2^2}\,W^i_{\mu\nu}W^{i\,\mu\nu}
    -\frac{1}{4}\,\frac{1}{g_1^2}\,B_{\mu\nu}B^{\mu\nu}\quad}
\]
with \emph{stiffness--coupling} relations
\[
    \frac{1}{g_i^2}=c_i\,\kappa_i,\qquad c_i>0\ \text{(normalization-dependent constants).}
\]
This realizes YM as \emph{emergent} from swirl textures (cf.\ emergent gauge fields in ordered media)\cite{Volovik2003,Fradkin2013,Zee2010,Wen2004}.

\subsection*{F.4 Empirical Correspondence (Lock-in Points)}
\begin{itemize}
    \item \textbf{Multiplet structure:} The table of \S F.1 reproduces SM multiplets per generation (color triplets/singlets, weak doublets/singlets).
    \item \textbf{Charges:} $Q=T_3+Y$ holds identically with $Y$ above (check each row).
    \item \textbf{Couplings from stiffness:} identify $\kappa_i=\zeta_i\,\rho_{\!f}\,\xi_i^2$ with correlation lengths $\xi_i$ and dimensionless $\zeta_i=O(1)$. Then
    \[
        \frac{1}{g_i^2}=c_i\zeta_i\,\rho_{\!f}\,\xi_i^2.
    \]
    Fit $\xi_i$ once at $\mu=M_Z$ to data ($g_1,g_2,g_3$) and record them as \emph{Empirical Calibrations}. Use standard SM $\beta$-functions for running\cite{PDG2024}. (For GUT-normalized $g_1$, fix $c_1$ accordingly.)
\end{itemize}

\subsection*{F.5 Symmetry Breaking and Chirality}

\paragraph{Order parameter as swirl doublet.}
Take a swirl doublet $\Phi$ and
\[
    \mathcal L_\Phi=(D_\mu \Phi)^\dagger (D^\mu \Phi)-\lambda_\Phi\big(|\Phi|^2-v_\Phi^2\big)^2,
    \qquad
    v_\Phi^2=\chi_\Phi\,\rho_{\!f}\,r_c^2\,\lVert \vswirl\rVert^2.
\]
Then the usual mass relations follow:
\[
    m_W=\tfrac12 g_2 v_\Phi,\qquad
    m_Z=\tfrac12\sqrt{g_2^2+g_1^2}\,v_\Phi,\qquad
    m_\gamma=0,
\]
with $v_\Phi$ empirically anchored\cite{Weinberg1967,EnglertBrout1964,Higgs1964}.

\paragraph{Left-handedness from helicity bias.}
The Kelvin/helicity term $\mathcal L_{\rm hel}=\eta\,(\mathbf v\!\cdot\!\boldsymbol\omega)$ produces, in linear response near the core, a chiral bias: the sign of $(\mathbf v\!\cdot\!\boldsymbol\omega)$ selects the left-chiral mode to couple to $W_\mu$ while the right-chiral is suppressed (SU(2) pseudoreality keeps $d_2$ unaffected)\cite{Witten1982}.
\begin{definition}[Weak-chirality bias]
In the radiation sector $\eta\!\to\!0$ (emergent Lorentz invariance), while in matter cores $\eta\!\neq\!0$ selects the left-handed weak current. This encodes parity violation without modifying the canonical YM structure.
\end{definition}

\subsection*{F.6 Falsifiable Predictions (Canon-Ready)}

\paragraph{P6.1 Running-ratio sum rule (coherence scale).}
If the director stiffnesses are isotropic at a swirl-coherence scale $\mu_\ast$,
\[
    \left.
    \frac{1}{g_3^2}:\frac{1}{g_2^2}:\frac{1}{g_1^2}
    \right|_{\mu_\ast}
    = 8:3:1\quad (\text{up to a common }c_i),
\]
then evolving down with SM $\beta$-functions gives a specific $\sin^2\theta_W(\mu)$ curve. Fit $\mu_\ast$ once; the \emph{shape} vs.\ $\ln\mu$ is then parameter-free and testable\cite{PDG2024}.

\paragraph{P6.2 Twist-parity selection rule for baryon decay.}
Let $\Delta\tau$ be the net change of the twist index across external lines. Local reconnections conserve $\sum \tau\ \text{mod }2$. Hence any effective operator with $\Delta\tau=1$ is forbidden in the SST EFT, excluding a family of proton-decay channels (catalogue explicitly). Observation of a forbidden channel falsifies this rule.

\paragraph{P6.3 Quantized EMF impulses vs.\ stiffness.}
The flux-impulse scale in reconnections scales with the doublet stiffness $\kappa_2$; therefore the step-height/bandwidth relation in SQUID-class pickup loops is predictive and testable with ns--$\mu$s bandwidth (see Canon Critical Questions and detector refs.)~\cite{Tinkham1996,ClarkeBraginski2004}.

\bigskip
\noindent\textit{Notes on citations.}
Călugăreanu--White: linking/twist/writhe decomposition for framed curves~\cite{Calugareanu1961,White1969}.
Pseudoreality/anomaly basics: ~\cite{Witten1982,WeinbergQFT2,PeskinSchroeder}.
Emergent gauge fields in ordered media and principal-chiral constructions: ~\cite{Volovik2003,Fradkin2013,Cho1980,FaddeevNiemi1999,Zee2010,Wen2004}.
EW breaking: ~\cite{Weinberg1967,EnglertBrout1964,Higgs1964}.
Running couplings: Particle Data Group\cite{PDG2024}.
Superconducting flux/EMF detection: ~\cite{Tinkham1996,ClarkeBraginski2004}.


%================================================
\section{Promotion Gate: Gauge Reinterpretation $\big(\mathfrak{su}(3)\oplus\mathfrak{su}(2)\oplus\mathfrak u(1)\big)$}
%================================================
\label{sec:gauge_promotion_gate}

\paragraph{Objective.}
Elevate the knot/topology reinterpretation of the Standard Model gauge sector from Research to Canon by meeting closure, anomaly, derivation, correspondence, breaking, and testability criteria.

\subsection*{(I) Closed Knot $\to$ Representation Map}
\begin{definition}[House map]
	A homomorphism
	\[
		t:\ \mathcal{K}\to \mathrm{Rep}\!\big(\mathrm{SU}(3)\times\mathrm{SU}(2)\times\mathrm{U}(1)\big),
		\qquad
		K\mapsto \big(R_3(K),R_2(K),Y(K)\big),
	\]
	is \emph{closed} if the following hold:
	\begin{enumerate}
		\item Composition: $t(K_1\#K_2)\cong t(K_1)\otimes t(K_2)$.
		\item Orientation/mirror: $t(\overline{K})\cong \overline{t(K)}$ (conjugate rep) and $Y(\overline{K})=-Y(K)$.
		\item Units: $t(\text{unknot})=(\mathbf{1},\mathbf{1},0)$.
		\item Parity of crossing number or genus induces $R_2\in\{\mathbf{2},\mathbf{1}\}$; $L\bmod 3$ induces $R_3\in\{\mathbf 3,\overline{\mathbf 3},\mathbf 1\}$.
	\end{enumerate}
\end{definition}

\begin{proposition}[Closure check]
	If (1)–(4) hold for a generating set of $\mathcal{K}$, closure extends to all composites by monoid generation.
\end{proposition}

\subsection*{(II) Anomaly Cancellation Equalities}
For the set of left-chiral excitations $\{\Psi_\alpha\}$ produced by $t$, the mixed and Abelian anomalies must vanish:
\begin{align}
	\mathcal A_{3^2-1} &\propto \sum_\alpha Y_\alpha\,T\!\big(R_3^\alpha\big)\,\dim R_2^\alpha = 0,\label{eq:A321}\\
	\mathcal A_{2^2-1} &\propto \sum_\alpha Y_\alpha\,T\!\big(R_2^\alpha\big)\,\dim R_3^\alpha = 0,\label{eq:A221}\\
	\mathcal A_{1^3} &\propto \sum_\alpha Y_\alpha^{\,3}\,\dim R_3^\alpha\,\dim R_2^\alpha = 0,\label{eq:A111}\\
	\mathcal A_{\mathrm{grav}^2-1} &\propto \sum_\alpha Y_\alpha\,\dim R_3^\alpha\,\dim R_2^\alpha = 0.\label{eq:Agrav}
\end{align}
Here $T(\mathbf 3)=\tfrac12$, $T(\mathbf 2)=\tfrac12$, and $T(\mathbf 1)=0$. In addition, the global SU(2) (Witten) anomaly requires an even number of $\mathrm{SU}(2)$ doublets:
\[
	\#\{\text{doublets}\}\ \equiv\ 0 \pmod 2.
\]

\subsection*{(III) Emergent Gauge Fields from Swirl Directors}
Let $\mathcal U_3(x)\in \mathrm{SU}(3)$ and $\mathcal U_2(x)\in \mathrm{SU}(2)$ be swirl \emph{director frames} built from orthonormal director fields of the multi-director condensate; let $\vartheta(x)\in \mathrm{U}(1)$. Define composite connections
\[
	G_\mu \equiv -\,\ii\,\mathcal U_3^{-1}\partial_\mu \mathcal U_3\in\mathfrak{su}(3),\quad
	W_\mu \equiv -\,\ii\,\mathcal U_2^{-1}\partial_\mu \mathcal U_2\in\mathfrak{su}(2),\quad
	B_\mu \equiv \partial_\mu \vartheta \in \mathfrak u(1).
\]
A gradient (stiffness) energy for the directors,
\[
	\mathcal L_{\text{dir}}=\frac{\kappa_3}{2}\Tr(\partial_\mu\mathcal U_3\,\partial^\mu\mathcal U_3^\dagger)
	+\frac{\kappa_2}{2}\Tr(\partial_\mu\mathcal U_2\,\partial^\mu\mathcal U_2^\dagger)
	+\frac{\kappa_1}{2}(\partial_\mu\vartheta)(\partial^\mu\vartheta),
\]
induces, after rewriting in terms of $(G_\mu,W_\mu,B_\mu)$ and adding the minimal gauge-covariant completion, the Yang–Mills sector
\[
	\mathcal L_{\text{YM}}=-\frac{1}{4 g_3^2}\,G_{\mu\nu}^a G^{a\,\mu\nu}
	-\frac{1}{4 g_2^2}\,W_{\mu\nu}^i W^{i\,\mu\nu}
	-\frac{1}{4 g_1^2}\,B_{\mu\nu}B^{\mu\nu},
	\quad
	g_i^{-2}\propto \kappa_i.
\]
\emph{Promotion criterion:} exhibit the explicit reduction $\mathcal L_{\text{dir}}\to\mathcal L_{\text{YM}}$ with $g_i$ expressed in Canon constants $(\rho_{\!f},r_c,\|\vswirl\|)$ up to an empirical factor fixed in Sec.~\ref{sec:master_equations}.

\subsection*{(IV) Multiplet and Charge Matching}
There must exist a finite set of knot classes $\{K\_\mathrm{gen}\}$ such that
\[
	t(K)\ \leftrightarrow\ \{(\mathbf 3,\mathbf 2)_{+1/6},\ (\overline{\mathbf 3},\mathbf 1)_{-2/3},\ (\overline{\mathbf 3},\mathbf 1)_{+1/3},\ (\mathbf 1,\mathbf 2)_{-1/2},\ (\mathbf 1,\mathbf 1)_{-1},\ (\mathbf 1,\mathbf 1)_{0}\}
\]
(up to right-handed conjugation), with electric charge $Q=T_3+Y$ reproduced exactly.

\subsection*{(V) Symmetry Breaking and Chirality}
Provide a swirl order parameter $\Phi$ (doublet under $\mathrm{SU}(2)$) and a gauge-invariant $V(\Phi)$ such that
\[
	\langle\Phi\rangle = \frac{v_\Phi}{\sqrt 2}\binom{0}{1},\qquad
	m_W=\tfrac12 g v_\Phi,\quad m_Z=\tfrac12\sqrt{g^2+g'^2}\,v_\Phi,\quad m_\gamma=0.
\]
Left-handedness must arise from a swirl–chirality selection rule (e.g., only ccw knots carry $R_2=\mathbf 2$), rendering right-handed excitations $\mathrm{SU}(2)$ singlets.

\subsection*{(VI) Falsifiable Prediction (required)}
Record at least one testable prediction traceable to swirl structure, e.g.:
\begin{itemize}
	\item A calculable shift in non-Abelian couplings due to the simplification index $\langle S_{\mathrm{comp}}\rangle$:
	$g_{2,3}^{-2}\mapsto g_{2,3}^{-2}\big[1+\lambda_{2,3}\langle S_{\mathrm{comp}}\rangle\big]$
	in families containing composites, yielding a measurable $\Delta\sigma/\sigma$ in channels with predominantly non-Abelian exchange.
	\item A required extra neutral lepton (knot class) to cancel \eqref{eq:A111} under your $Y(K)$, implying a sterile-like state with specific coupling absences.
\end{itemize}

\paragraph{Promotion Rule.}
If (I)–(VI) are satisfied and the numerical anchors (Sec.~\ref{sec:master_equations}) are met within experimental uncertainties, the Gauge Reinterpretation becomes Canonical (Sec.~\ref{sec:canon_governance}).

\subsection*{Reference Summary of Canonical Equations and Constants}
% Choose three integer invariants I_1,I_2,I_3 (e.g., writhe, twist, linking) and chirality χ=±1.
Y(K) = \alpha\,I_1(K) + \beta\,I_2(K) + \gamma\,I_3(K) + \delta\,\chi(K).
% Fit (α,β,γ,δ) to one generation’s Y targets; then verify anomalies (\ref{eq:A321})–(\ref{eq:Agrav}).
\#\{K:\ R_2(K)=\mathbf 2\}\ \equiv\ 0 \pmod 2 \quad \text{(per generation)}.
G_\mu = -\ii\,\mathcal U_3^{-1}\partial_\mu\mathcal U_3,\quad
W_\mu = -\ii\,\mathcal U_2^{-1}\partial_\mu\mathcal U_2,\quad
B_\mu = \partial_\mu\vartheta,
\quad
F_{\mu\nu} = \partial_\mu A_\nu - \partial_\nu A_\mu + [A_\mu,A_\nu].


\paragraph{Dimensional check.}
    $\Sigma_{\rm core}$ is dimensionless; hence $g_i$ are dimensionless. The RG term is dimensionless (logarithm) with the usual one-loop coefficients.
\paragraph{Non-canonical (Research).}
    The following proposed relation is \emph{not} a premise of the Canon; it is recorded for calibration and future derivation.

    \begin{theorem}[EWSB from swirl anisotropy (canonical scale and sign) \cite{Weinberg1967,EnglertBrout1964,Higgs1964}]
    Let the SU(2)-doublet order parameter $\Phi$ arise as the lowest scalar swirl mode. The swirl–helicity bias induces a negative mass term
    \[
        m_\Phi^2\;=\;-\,\zeta_{\rm EW}\,\Sigma_{\rm core}\,\Bigl(\frac{\hbar c}{\rc}\Bigr)^2,
        \qquad \zeta_{\rm EW}>0\ \text{dimensionless},
    \]
    and a quartic $V(\Phi)=\lambda_\Phi\big(|\Phi|^2-v_\Phi^2\big)^2$ with $\lambda_\Phi>0$ canonical.
    Then
    \[
        \boxed{\;
        v_\Phi
        \;=\;
        \frac{1}{\sqrt{\lambda_\Phi}}\,
        \sqrt{-m_\Phi^2}
        \;=\;
        \Bigl(\frac{\hbar c}{\rc}\Bigr)\,
        \sqrt{\frac{\zeta_{\rm EW}\,\Sigma_{\rm core}}{\lambda_\Phi}}
        \;=\;
        \Bigl(\frac{\hbar c}{\rc}\Bigr)\,
        \sqrt{\frac{\zeta_{\rm EW}}{\pi\,\lambda_\Phi}}
        \;}
    \]
    \[
        \boxed{\;
        m_W=\tfrac12 g_2\,v_\Phi,\qquad
        m_Z=\tfrac12\sqrt{g_2^2+g_1^2}\,v_\Phi,\qquad
        m_\gamma=0
        \;}
    \]
    so the entire EWSB scale and gauge masses are set by Canon constants $(\rc,\vnorm)$ and the \emph{dimensionless} swirl factors $(\zeta_{\rm EW},\lambda_\Phi)$.
    \end{theorem}

\paragraph{Notes for reviewers.}
\begin{itemize}
	\item \textbf{Helicity term:} since $[\vswirl\!\cdot\!\omegas]=\mathrm{m\,s^{-2}}$, multiplying by $\rhof$ gives $\mathrm{J\,m^{-3}}$, so $[\chi_h]=1$.
	\item \textbf{Constraint multiplier:} with $(\nabla\!\cdot\!\vswirl)$ in $\mathrm{s^{-1}}$, the Lagrange multiplier has units $[\lambda]=\mathrm{Pa\cdot s}=\mathrm{J\,s\,m^{-3}}$. In Euler–Lagrange equations, its \emph{spatial} gradient divided by time (or $\partial_t\lambda$) plays the role of a pressure field.
	\item \textbf{Swirl potential:} the canonical choice $\Phi_{\text{swirl}}=\frac{\vscore^{2}}{2\,\FmaxEM}(\omegas\!\cdot\!\mathbf r)$ has $[\Phi_{\text{swirl}}]=\mathrm{m^2\,s^{-2}}$ provided $\FmaxEM$ carries force units, consistent with Table entries.
	\item \textbf{Gauge block:} all gauge-sector rows are standard; in natural units they produce mass$^4$, hence energy density after restoring $\hbar,c$.
\end{itemize}

\noindent\textbf{Summary.} Each addend in $\mathcal L_{\text{SST+Gauge}}$ carries $\mathrm{J\,m^{-3}}$ in SI; coupling constants \((g_s,g,g',\chi_h)\) are dimensionless; the incompressibility multiplier has $[\lambda]=\mathrm{Pa\cdot s}$ under the present choice of constraint.

%================================================
% References
%================================================
\nocite{*}
\bibliographystyle{unsrt}
\bibliography{canon_swirl_string_theory}
\end{document}