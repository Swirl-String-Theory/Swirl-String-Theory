%! Author = Omar Iskandarani
%! Title = Swirl String Theory (SST) Canon v0.3
%! Date = Sept 1, 2025
%! Affiliation = Independent Researcher, Groningen, The Netherlands
%! License = © 2025 Omar Iskandarani. All rights reserved. This manuscript is made available for academic reading and citation only. No republication, redistribution, or derivative works are permitted without explicit written permission from the author. Contact: info@omariskandarani.com
%! ORCID = 0009-0006-1686-3961
%! DOI = 10.5281/zenodo.17014358

\newcommand{\canonversion}{\textbf{v0.3.4}} % Semantic versioning: vMAJOR.MINOR.PATCH
\newcommand{\papertitle}{Swirl String Theory (SST) Canon \canonversion}
\newcommand{\paperdoi}{10.5281/zenodo.17014358}

% ==== Swirl String Theory (SST) macros ====
% Context-aware subscript symbol; uses math styles, not \scriptsize
\newcommand{\swirlarrow}{%
	\mathchoice{\mkern-2mu\scriptstyle\boldsymbol{\circlearrowleft}}%
	{\mkern-2mu\scriptstyle\boldsymbol{\circlearrowleft}}%
	{\mkern-2mu\scriptscriptstyle\boldsymbol{\circlearrowleft}}%
	{\mkern-2mu\scriptscriptstyle\boldsymbol{\circlearrowleft}}%
}
\newcommand{\swirlarrowcw}{%
	\mathchoice{\mkern-2mu\scriptstyle\boldsymbol{\circlearrowright}}%
	{\mkern-2mu\scriptstyle\boldsymbol{\circlearrowright}}%
	{\mkern-2mu\scriptscriptstyle\boldsymbol{\circlearrowright}}%
	{\mkern-2mu\scriptscriptstyle\boldsymbol{\circlearrowright}}%
}

% Canonical symbols
\newcommand{\vswirl}{\mathbf{v}_{\swirlarrow}}
\newcommand{\vswirlcw}{\mathbf{v}_{\swirlarrowcw}}
\newcommand{\SwirlClock}{S_(t)^{\swirlarrow}}
\newcommand{\SwirlClockcw}{S_(t)^{\swirlarrowcw}}
\newcommand{\omegas}{\boldsymbol{\omega}_{\swirlarrow}}  % swirl vorticity
\newcommand{\vscore}{v_{\swirlarrow}}                    % shorthand: |v_swirl| at r=r_c
\newcommand{\vnorm}{\lVert \vswirl \rVert}               % swirl speed magnitude
\newcommand{\rhof}{\rho_{\!f}}                           % effective fluid density
\newcommand{\rhoE}{\rho_{\!E}}                           % swirl energy density /c^2? (we define clearly below)
\newcommand{\rhom}{\rho_{\!m}}                           % mass-equivalent density
\newcommand{\rc}{r_c}                                    % string core radius (swirl string radius)
\newcommand{\FmaxEM}{F_{\mathrm{EM}}^{\max}}             % EM-like maximal force scale
\newcommand{\FmaxG}{F_{\mathrm{G}}^{\max}}               % G-like maximal force scale
\newcommand{\Lam}{\Lambda}                               % Swirl Coulomb constant
\newcommand{\Om}{\Omega_{\swirlarrow}}                   % swirl angular frequency profile
\newcommand{\alpg}{\alpha_g}                             % gravitational fine-structure analogue

% Policy: the golden constant is only allowed via hyperbolic functions.
% Never write (1+\sqrt{5})/2; always use \xig=\asinh(1/2), \varphi=e^{\xig}.
\newcommand{\xig}{\operatorname{asinh}\!\left(\tfrac{1}{2}\right)} % base hyperbolic scale  "golden" constant is fundamentally hyperbolic.
\newcommand{\phig}{\exp(\xig)}                                     % golden from hyperbolic
\newcommand{\phialg}{\bigl(1+\sqrt{5}\bigr)/2}                     % algebraic echo (use sparingly)
\newcommand{\xigold}{\tfrac{3}{2}\,\xig}                           % "golden rapidity" scale

% --- Display helpers (optional) ---
\newcommand{\GoldenDeclare}{%
	\textbf{Golden (hyperbolic)}:\ \(\ln\phi=\xig\), hence \(\phi=\phig\).
	\ \emph{(Equivalently, \(\phi=\phialg\); this algebraic form is derivative.)}%
}
% --- Canonical identity (hyperbolic-only proof, algebraic as corollary) ---
\newtheorem{identity}{Identity}


%========================================================================================
% PACKAGES AND DOCUMENT CONFIGURATION
%========================================================================================
\documentclass[11pt]{article}
\usepackage{subfiles}
% sststyle.sty
\NeedsTeXFormat{LaTeX2e}
\ProvidesPackage{sststyle}[2025/07/01 SST unified style]



% === Draft Options ===
\newif\ifsstdraft
% \sstdrafttrue
\ifsstdraft
\RequirePackage{showframe}
\fi

% === Load Once ===
\RequirePackage{ifthen}
\newboolean{sststyleloaded}
\ifthenelse{\boolean{sststyleloaded}}{}{\setboolean{sststyleloaded}{true}

% === Page ===
\RequirePackage[a4paper, margin=2.5cm]{geometry}

% === Fonts ===
\RequirePackage[T1]{fontenc}
\RequirePackage[utf8]{inputenc}
\RequirePackage[english]{babel}
\RequirePackage{textgreek}
\RequirePackage{mathpazo}
\RequirePackage[scaled=0.95]{inconsolata}
\RequirePackage{helvet}

% === Math ===
\RequirePackage{amsmath, amssymb, mathrsfs, physics}
\RequirePackage{siunitx}
\sisetup{per-mode=symbol}

% === Tables ===
\RequirePackage{graphicx, float, booktabs}
\RequirePackage{array, tabularx, multirow, makecell}
\newcolumntype{Y}{>{\centering\arraybackslash}X}
\newenvironment{tighttable}[1][]{\begin{table}[H]\centering\renewcommand{\arraystretch}{1.3}\begin{tabularx}{\textwidth}{#1}}{\end{tabularx}\end{table}}
\RequirePackage{etoolbox}
\newcommand{\fitbox}[2][\linewidth]{\makebox[#1]{\resizebox{#1}{!}{#2}}}

% === Graphics ===
\RequirePackage{tikz}
\usetikzlibrary{3d, calc, arrows.meta, positioning}
\RequirePackage{pgfplots}
\pgfplotsset{compat=1.18}
\RequirePackage{xcolor}

% === Code ===
\RequirePackage{listings}
\lstset{basicstyle=\ttfamily\footnotesize, breaklines=true}

% === Theorems ===
\newtheorem{theorem}{Theorem}[section]
\newtheorem{lemma}[theorem]{Lemma}

% === TOC ===
\RequirePackage{tocloft}
\setcounter{tocdepth}{2}
\renewcommand{\cftsecfont}{\bfseries}
\renewcommand{\cftsubsecfont}{\itshape}
\renewcommand{\cftsecleader}{\cftdotfill{.}}
\renewcommand{\contentsname}{\centering \Huge\textbf{Contents}}

% === Sections ===
\RequirePackage{sectsty}
\sectionfont{\Large\bfseries\sffamily}
\subsectionfont{\large\bfseries\sffamily}

% === Bibliography ===


% === Links ===
\RequirePackage{hyperref}
\hypersetup{
    colorlinks=true,
    linkcolor=blue,
    citecolor=blue,
    urlcolor=blue,
    pdftitle={The Vortex \AE ther Model},
    pdfauthor={Omar Iskandarani},
    pdfkeywords={vorticity, gravity, \ae ther, fluid dynamics, time dilation, SST}
}
\urlstyle{same}
\RequirePackage{bookmark}

% === Misc ===
\RequirePackage[none]{hyphenat}
\sloppy
\RequirePackage{empheq}
\RequirePackage[most]{tcolorbox}
\newtcolorbox{eqbox}{colback=blue!5!white, colframe=blue!75!black, boxrule=0.6pt, arc=4pt, left=6pt, right=6pt, top=4pt, bottom=4pt}
\RequirePackage{titling}
\RequirePackage{amsfonts}
\RequirePackage{titlesec}
\RequirePackage{enumitem}

\AtBeginDocument{\RenewCommandCopy\qty\SI}

\pretitle{\begin{center}\LARGE\bfseries}
\posttitle{\par\end{center}\vskip 0.5em}
\preauthor{\begin{center}\large}
\postauthor{\end{center}}
\predate{\begin{center}\small}
\postdate{\end{center}}


\endinput
}
% sstappendixsetup.sty

\newcommand{\titlepageOpen}{
  \begin{titlepage}
  \thispagestyle{empty}
  \centering
  \ifdefined\standalonechapter
  {\Large\bfseries \appendixtitle \par}
  \else
  {\Large\bfseries \papertitle \par}
    \fi
  \vspace{1cm}
  {\Large\itshape \textbf{Omar Iskandarani}\textsuperscript{\textbf{*}} \par}
  \vspace{0.5cm}
  {\today \par}
  \vspace{0.5cm}
}

% here comes abstract
\newcommand{\titlepageClose}{
  \vfill
  \raggedright % <-- fixes left alignment
  \null
  \begin{picture}(0,0)
  % Adjust position: (x,y) = (left, bottom)
  \put(0,-45){  % Shift 200pt left, 40pt down
    \begin{minipage}[b]{0.7\textwidth}
    \footnotesize % One step bigger than \tiny \scriptsize
    \renewcommand{\arraystretch}{1.0}
    \noindent\rule{\textwidth}{0.4pt} \\[0.5em]  % ← horizontal line
    \textsuperscript{\textbf{*}} Independent Researcher, Groningen, The Netherlands \\
    Email: \texttt{info@omariskandarani.com} \\
    ORCID: \texttt{\href{https://orcid.org/0009-0006-1686-3961}{0009-0006-1686-3961}} \\
    DOI: \href{https://doi.org/\paperdoi}{\paperdoi} \\
    License: CC-BY-NC 4.0 International \\
    \end{minipage}
  }
  \end{picture}
  \end{titlepage}
}
\usepackage[margin=1in]{geometry}
\usepackage{amsmath,amssymb,amsfonts}
\usepackage{tcolorbox}
\usetikzlibrary{knots,intersections,decorations.pathreplacing,3d,calc,arrows.meta,positioning,decorations.pathmorphing}
\usepackage{pgfmath}
\usepackage{pgfplots}
\pgfplotsset{compat=1.18}
\usepackage{ulem}


% ==== Packages ====
\usepackage[T1]{fontenc}
\usepackage{lmodern}
\usepackage{microtype}

\geometry{margin=1in}
\usepackage{ bm, mathtools}
\usepackage{siunitx}
\sisetup{per-mode=symbol,round-mode=figures,round-precision=6}
\usepackage{physics}
\usepackage{upgreek}
\usepackage{graphicx}
\usepackage{booktabs}
\usepackage{hyperref}
\hypersetup{colorlinks=true, linkcolor=blue!60!black, citecolor=blue!60!black, urlcolor=blue!60!black}

%========================================================================================
% DOCUMENT START
%========================================================================================
\begin{document}

%========================================================================================
% TITLE PAGE
%========================================================================================

		\titlepageOpen
		\begin{abstract}

        This Canon is the single source of truth for \emph{Swirl String Theory (SST)}: definitions, constants, boxed master equations, and notational conventions. It consolidates core structure and \emph{promotes five results to canonical status}:
		\begin{tabular}{r@{\quad}p{0.85\linewidth}}
		   \textbf{I} & Swirl Coulomb constant $\Lam$ and hydrogen soft-core \\
		   \textbf{II} & circulation--metric corollary (frame-dragging analogue) \\
		   \textbf{III} & corrected swirl time-rate (Swirl Clock) law \\
		   \textbf{IV} & Kelvin-compatible swirl Hamiltonian density \\
		   \textbf{V} & swirl pressure law (Euler corollary) \\
		\end{tabular}

    \paragraph{Core Postulates (SST)}
    \begin{enumerate}
        \item \textbf{Swirl medium:} Physics is formulated on $\mathbb{R}^3$ with absolute reference time. Dynamics occur in an incompressible, inviscid \emph{swirl condensate}, which plays the role of a universal substrate.
        \item \textbf{Strings as swirls:} Particles and excitations correspond to closed, possibly linked or knotted \emph{swirl strings} with quantized circulation.
        \item \textbf{String-induced gravitation:} Macroscopic attraction emerges from coherent swirl fields and swirl-pressure gradients. The effective gravitational coupling $G_{\text{swirl}}$ is fixed by canonical constants.
        \item \textbf{Swirl clocks:} Local proper-time rate depends on tangential swirl velocity. Higher swirl density slows local clocks relative to the asymptotic frame.
        \item \textbf{Quantization from topology and circulation:} Discrete quantum numbers track directly to linking, writhe, twist, and circulation quantization of swirl strings.
        \item \textbf{Taxonomy:} Unknotted excitations behave as bosonic string modes; chiral hyperbolic knots map to quarks; torus knots map to leptons (taxonomy documented separately).
    \end{enumerate}
		\footnotesize{Hydrodynamic analogy only; no mechanical “æther” is assumed in the mainstream presentation.}
	\end{abstract}

		\vfill
		\paragraph{Versioning} Semantic versions: vMAJOR.MINOR.PATCH. This file: \canonversion.\\
		Every paper/derivation must state the Canon version it depends on.
		\titlepageClose
%================================================
    \section{Swirl Quantization Principle}
%================================================

    \subsection{Local Circulation Quantization}
    The circulation of the swirl velocity field around any closed loop is quantized:
    \begin{equation}
        \Gamma = \oint \vec{v}_{\text{swirl}} \cdot d\vec{\ell}
        = n \kappa,
        \qquad n \in \mathbb{Z},
    \end{equation}
    with circulation quantum
    \begin{equation}
        \kappa = \frac{h}{m_\text{eff}}.
    \end{equation}
    This parallels the Onsager--Feynman quantization condition in superfluids, but here is elevated to a fundamental postulate of the swirl condensate.

    \subsection{Topological Quantization}
    Closed swirl filaments may form knots and links. Each topological class corresponds to a discrete excitation state:
    \begin{equation}
        \mathcal{H}_\text{swirl}
        = \{ \text{trefoil}, \; \text{figure-eight}, \; \text{Hopf link}, \dots \}.
    \end{equation}
    Quantum numbers such as mass, charge, and chirality are encoded in the knot invariants (linking, twist, writhe).

    \subsection{Unified Principle}
    We define \emph{Swirl Quantization} as the joint discreteness of circulation and topology:
    \[
        \text{Swirl Quantization} \;\equiv\;
        \Big( \Gamma = n\kappa \Big)
        \;\; \cup \;\;
        \Big( \text{Knot spectrum } \mathcal{H}_\text{swirl} \Big).
    \]
    This principle underlies both the discrete particle spectrum and the emergence of fundamental interactions in Swirl String Theory.

    \begin{center}
        \begin{tabular}{|c|c|}
            \hline
            \textbf{Quantum Mechanics} & \textbf{Swirl String Theory} \\
            \hline
            Canonical Quantization: & Swirl Quantization Principle: \\
            $[x, p] = i \hbar$ & $\Gamma = n \kappa, \quad n \in \mathbb{Z}$ \\[6pt]
            & $\mathcal{H}_\text{swirl} =
            \{ \text{trefoil}, \; \text{figure-eight}, \; \text{Hopf link}, \dots \}$ \\
            \hline
            Discreteness arises from & Discreteness arises from \\
            operator commutators & circulation integrals and topology \\
            \hline
            Particles = eigenstates of & Particles = knotted swirl states with \\
            Hamiltonian operator & quantized circulation and invariants \\
            \hline
        \end{tabular}
    \end{center}


%===============================================================
%  Chronos–Kelvin Invariant (Canonical)
%===============================================================
    \section{Chronos–Kelvin Invariant (Canonical)}
    \addcontentsline{toc}{section}{Chronos–Kelvin Invariant (Canonical)}
    \label{sec:chronos_kelvin}

    \paragraph{Setting.}
    Consider a thin, material swirl loop (nearly solid–body core) of instantaneous material radius
    $R(t_{\swirlarrow})$ convected by an incompressible, inviscid medium. Let $\omega:=\|\boldsymbol{\omega}\|$ denote the
    vorticity magnitude on the loop and $r_c$ the canonical string radius. The local Swirl Clock is
    \begin{equation}
        S_t \;\equiv\; \frac{dt_{\text{local}}}{dt_\infty}
        \;=\;
        \sqrt{\,1-\frac{v_t^{\,2}}{c^2}\,}
        \;=\;
        \sqrt{\,1-\frac{\omega^2 r_c^2}{c^2}\,},\qquad v_t:=\omega r_c .
        \label{eq:SwirlClock-def}
    \end{equation}
    Material derivatives are taken with respect to absolute reference time:
    $\displaystyle \frac{D}{Dt_{\swirlarrow}}:=\frac{\partial}{\partial t_{\swirlarrow}}+\mathbf{v}\!\cdot\!\nabla$.

    \begin{theorem}[Chronos–Kelvin Invariant]
        For any such loop without reconnection or source terms, Kelvin’s theorem implies the
        material invariant
        \begin{equation}
            \boxed{\;
            \frac{D}{Dt_{\swirlarrow}}\!\left(R^2\,\omega\right)=0
            \;}
            \quad\Longleftrightarrow\quad
            \boxed{\;
            \frac{D}{Dt_{\swirlarrow}}\!\left(
                                    \frac{c}{r_c}\,R^2 \sqrt{1-S_t^2}
            \right)=0
            \;}
            \label{eq:CK}
        \end{equation}
    \end{theorem}

    \paragraph{Proof (one line).}
    Kelvin’s circulation theorem for an inviscid, barotropic medium gives
    $\displaystyle \frac{D}{Dt_{\swirlarrow}}\Gamma=0$ with $\Gamma=\oint \mathbf{v}\cdot d\boldsymbol{\ell}$ \cite{Helmholtz1858,Kelvin1869,Batchelor1967}.
    For a nearly solid–body core, $\Gamma=2\pi R\, v_t=2\pi R^2 \omega$; hence
    $\displaystyle \frac{D}{Dt_{\swirlarrow}}(R^2\omega)=0$.
    Using \eqref{eq:SwirlClock-def}, $R^2\omega=\tfrac{c}{r_c} R^2\sqrt{1-S_t^2}$, which yields \eqref{eq:CK}. \qed

    \paragraph{Dimensional consistency.}
    $[R^2\omega]=\text{m}^2\text{s}^{-1}$; and
    $\big[\tfrac{c}{r_c}R^2\sqrt{1-S_t^2}\big]=\text{s}^{-1}\cdot\text{m}^2=\text{m}^2\text{s}^{-1}$.

    \paragraph{Clock–radius transport law (corollary).}
    From $R^2\omega=\text{const}$ and \eqref{eq:SwirlClock-def},
    \begin{equation}
        \frac{dS_t}{dt_{\swirlarrow}} \;=\; \frac{2(1-S_t^2)}{S_t}\,\frac{1}{R}\frac{dR}{dt_{\swirlarrow}}
        \label{eq:clock-radius-ode}
    \end{equation}
    Hence expansion ($dR/dt_{\swirlarrow}>0$) pushes $S_t\!\to\!1$ (clocks speed up), while contraction
    slows clocks ($S_t\!\downarrow$), preserving \eqref{eq:CK}.

    \paragraph{PV analogue (optional).}
    With a uniform background rotation $\Omega_{\text{bg}}$ and column thickness $H$,
    the Ertel/PV structure gives the SST counterpart
    \begin{equation}
        \frac{D}{Dt_{\swirlarrow}}
        \left(
            \frac{\omega+\Omega_{\text{bg}}}{H}
        \right)=0,
        \label{eq:PV-analogue}
    \end{equation}
    the standard potential-vorticity conservation rewritten in SST terms \cite{Ertel1942,Batchelor1967}.

    \paragraph{Conditions (Canon).}
    Incompressible, inviscid medium; barotropic swirl pressure; material loop without reconnection or
    external injection; absolute reference time parametrization. These are the same hypotheses under which
    Kelvin/Helmholtz invariants hold.

    \paragraph{Limits.}
    Weak-swirl ($\omega r_c\!\ll\!c$): $S_t\simeq 1-\tfrac{1}{2}(\omega r_c/c)^2$ and \eqref{eq:CK} reduces to the
    classical $R^2\omega=\text{const}$. Core on-axis limit: $v_t\to\vswirl$ gives
    $S_t\to \sqrt{1-(\vswirl/c)^2}$, keeping \eqref{eq:CK} valid.

%================================================
% Foundational Identities
%================================================
\section{Foundational Identities}
\label{sec:foundational-identities}
    Let $\vb v$ be the swirl velocity ($\nabla\cdot \vb v=0$), $\bm{\omega}=\nabla\times \vb v$. For inviscid, barotropic flow \cite{Helmholtz1858,Kelvin1869,Batchelor1967,LandauLifshitz1987}:
    \begin{align}
    \textbf{Kelvin circulation:} & \quad \frac{d\Gamma}{dt}=0,\quad \Gamma=\oint_{\mathcal{C}(t)} \vb v\cdot d\vb \ell. \tag{F1} \label{eq:kelvin} \\
    \textbf{Vorticity transport:} & \quad \pdv{\bm{\omega}}{t} = \nabla\times(\vb v\times \bm{\omega}). \tag{F2} \label{eq:vorticity-transport}\\
    \textbf{Helicity:} & \quad h=\vb v\cdot \bm{\omega},\quad H=\int h, dV\ \text{(invariant up to reconnections)}.~\cite{Moffatt1969} \tag{F3}\label{eq:helicity}
    \end{align}
    These underpin knotted swirl string stability and reconnection energetics in SST.


%================================================
% Canonical Constants and Symbols
%================================================
\section{Canonical Constants and Symbols}
\label{sec:canonical_constants}
    \subsection*{Primary SST constants (SI unless noted)}
    \begin{itemize}
        \item Swirl speed scale (core): $\vnorm = \num{1.09384563e6}\ \si{m.s^{-1}}$ (evaluate at $r=\rc$).
        \item String (core) radius: $\rc = \num{1.40897017e-15}\ \si{m}$.
        \item Effective fluid density: $\rhof = \num{7.0e-7}\ \si{kg.m^{-3}}$.
        \item Mass-equivalent density: $\rhom = \num{3.8934358266918687e18}\ \si{kg.m^{-3}}$. % used in \Lambda
        \item EM-like maximal force: $\FmaxEM = \num{2.9053507e1}\ \si{N}$.
        \item Gravitational maximal force (reference scale): $\FmaxG = \num{3.02563e43}\ \si{N}$.
        \item Golden ratio: $\varphi = (1+\sqrt{5})/2 \approx \num{1.61803398875}$.
    \end{itemize}

    \subsection*{Universal constants}
    \begin{itemize}
        \item $c=\num{299792458}\ \si{m.s^{-1}}$, \quad $t_p=\num{5.391247e-44}\ \si{s}$.
        \item Fine-structure constant (identified): $\alpha \approx \num{7.2973525643e-3}$.
    \end{itemize}

    \paragraph{Effective densities (mainstream field-theory style).}
    \[
        \rho_f \equiv \text{effective fluid density},
	\]
        We use $\rhof$ to avoid confusion\footnote{The canonical choice
            $\rho_{\!f} = 7.0\times 10^{-7}\,\mathrm{kg\,m^{-3}}$
            is not a measured value but a calibration constant.
            Its magnitude is anchored to the electromagnetic permeability scale
            $\mu_0/(4\pi) = 10^{-7}$ in SI units, ensuring dimensional consistency
            between swirl energetics and EM normalization.
            Unlike the derived high-precision values of
            $\rho_{\!m}$ and $\rho_{\!E}$,
            the effective fluid density $\rho_{\!f}$ is defined
            at this tidy scale to serve as a reference baseline.}
            with mass density;
	\[
        \rho_E \equiv \tfrac12 \rho_f\, \vnorm^2\quad(\text{swirl energy density}),\qquad
        \rho_m \equiv \rho_E/c^2\quad(\text{mass-equivalent density}).
    \]

    \textbf{Note:} The local Python \texttt{constants\_dict} used in simulations must mirror these values exactly; papers should quote the Canon version.

    \subsection*{Swirl Areal Density and EM Coupling}

    In addition to the effective densities defined above, we introduce the swirl areal density
    \(\varrho_{\swirlarrow}\), defined as the coarse-grained number of swirl cores per unit area. Its time variation enters Maxwell’s law as an additional source term,
    \[
        \nabla \times \mathbf{E} = -\partial_t \mathbf{B} - \mathbf{b}_{\swirlarrow}, \qquad
        \mathbf{b}_{\swirlarrow} = G_{\swirlarrow}\,\partial_t \varrho_{\swirlarrow}.
    \]
    Here \(G_{\swirlarrow}\) is the canonical swirl--EM transduction constant, identified with a flux quantum \(\Phi^\ast \sim h/2e\). This relation links electromotive force (voltage impulses) and swirl reconnection dynamics, establishing the canonical bridge between EMF and gravity-like swirl fields.


%================================================
% Canon Governance & Status Taxonomy
%================================================
    \section{Canon Governance (Binding)}

    \subsection*{Definitions}
    \paragraph{Formal System.}
    Let \(\mathcal{S} = (\mathcal{P},\mathcal{D},\mathcal{R})\) denote the SST formal system:
    postulates \(\mathcal{P}\), definitions \(\mathcal{D}\), and admissible inference rules \(\mathcal{R}\)
    (variational derivation, Noether, dimensional analysis, asymptotic matching, etc.).

    \paragraph{Canonical statement.}
    A statement \(X\) is \emph{canonical} iff \(X\) is a theorem or identity provable in \(\mathcal{S}\):
    \[
        \mathcal{P},\mathcal{D}\ \vdash_{\mathcal{R}}\ X,
    \]
    and \(X\) is consistent with all previously accepted canonical items in the current major version.

    \paragraph{Empirical statement.}
    A statement \(Y\) is \emph{empirical} iff it asserts a measured value, fit, or protocol:
    \[
        Y \equiv \text{“observable } \mathcal{O} \text{ has value } \hat{o} \pm \delta o \text{ under procedure } \Pi\text{.”}
    \]
    Empirical items calibrate symbols (e.g., $\vscore$, $\rc$, $\rhof$) but are not premises in proofs.

    \subsection*{Status Classes}
    \begin{itemize}
        \item \textbf{Axiom / Postulate (Canonical).} Primitive assumptions of SST (e.g., incompressible, inviscid medium; absolute time; Euclidean space).
        \item \textbf{Definition (Canonical).} Introduces symbols by construction (e.g., swirl Coulomb constant \(\Lambda\) by surface-pressure integral).
        \item \textbf{Theorem / Corollary (Canonical).} Proven consequences (e.g., Euler–SST radial balance; Swirl Clocks time-scaling).
        \item \textbf{Constitutive Model (Canonical if derived; otherwise Semi-empirical).} Ties fields/observables; canonical when deduced from \(\mathcal{P},\mathcal{D}\).
        \item \textbf{Calibration (Empirical).} Recommended numerical values with uncertainties for canonical symbols.
        \item \textbf{Research Track (Non-canonical).} Conjectures or alternatives pending proof or axiomatization.
    \end{itemize}

    \subsection*{Canonicality Tests (all required)}
    \begin{enumerate}
        \item \textbf{Derivability} from \(\mathcal{P},\mathcal{D}\) via \(\mathcal{R}\).
        \item \textbf{Dimensional Consistency} (SI throughout; correct limits).
        \item \textbf{Symmetry Compliance} (Galilean + absolute time; foliation; incompressibility).
        \item \textbf{Recovery Limits} (Newtonian gravity, Coulomb/Bohr, linear waves).
        \item \textbf{Non-Contradiction} with accepted canonical theorems.
        \item \textbf{Parameter Discipline} (no ad-hoc fits).
    \end{enumerate}

    \subsection*{Examples (from current Canon)}
    \begin{itemize}
        \item \(\displaystyle \textit{Canonical (Definition):}\quad \Lambda \equiv \int_{S_r^2} p_{\text{swirl}}\,r^2\,d\Omega.\)
        \item \(\displaystyle \textit{Canonical (Theorem):}\quad \frac{1}{\rhof}\frac{dp_{\text{swirl}}}{dr}=\frac{v_\theta(r)^2}{r}\) for steady, azimuthal drift (Euler balance).
        \item \(\displaystyle \textit{Empirical (Calibration):}\quad \vscore=1.09384563\times10^{6}\,\mathrm{m\,s^{-1}}\) with procedure \(f\Delta x\).
        \item \(\displaystyle \textit{Consistency Check (Not a premise):}\) Hydrogen soft-core reproduces \(a_0,E_1\); validates choices but remains a check, not an axiom.
    \end{itemize}

    %! Canonical Scope and Rationale for SST
    \section{What is Canonical in SST—and Why}

    \paragraph{[Postulate] Incompressible, inviscid medium with absolute time and Euclidean space.}
    \(\nabla\!\cdot\!\vswirl=0,\ \nu=0.\)
    This fixes the kinematic arena and legal inference rules.

    \paragraph{[Definition] Vorticity, circulation, helicity.}
    \(\omegas=\nabla\times \vswirl,\quad \Gamma=\oint \vswirl\!\cdot d\boldsymbol{\ell},\quad h=\vswirl\!\cdot\!\omegas,\ H=\int h\,dV.\)
    Classical constructs canonized as primary SST kinematic invariants.

    \paragraph{[Theorem] Kelvin/vorticity transport/helicity invariants.}
    For inviscid, barotropic flow:
    \[
        \frac{d\Gamma}{dt}=0,\qquad
        \pdv{\omegas}{t}=\nabla\times(\vswirl\times\omegas),\qquad
        \text{$H$ invariant up to reconnections}.
    \]

    \paragraph{[Definition] Swirl Coulomb constant \(\Lambda\).}
    \[
        \boxed{\ \Lambda \equiv \int_{S_r^2} p_{\text{swirl}}(r)\, r^2\, d\Omega\ } \quad\Rightarrow\quad [\Lambda]=\mathrm{J\,m}=\mathrm{N\,m^2}.
    \]
    In SST Canon this evaluates symbolically to \( \Lambda=4\pi \rhom\, \vscore^{\,2}\, \rc^4\).

    \paragraph{[Theorem] Hydrogen soft-core potential and Coulomb recovery.}
    \[
        V_{\text{SST}}(r)=-\frac{\Lambda}{\sqrt{r^2+\rc^2}}
        \;\xrightarrow{r\gg \rc}\;
        -\frac{\Lambda}{r},
    \]
    yielding Bohr scalings
    \(a_0=\hbar^2/(\mu\Lambda)\), \(E_n=-\mu\Lambda^2/(2\hbar^2 n^2)\).

    \paragraph{[Theorem] Euler–SST radial balance (swirl pressure law).}
    For steady, purely azimuthal drift \(v_\theta(r)\),
    \[
        0=-\frac{1}{\rhof}\frac{dp_{\text{swirl}}}{dr}+\frac{v_\theta(r)^2}{r}
        \quad\Rightarrow\quad
        \boxed{\ \frac{1}{\rhof}\frac{dp_{\text{swirl}}}{dr}=\frac{v_\theta(r)^2}{r}\ }.
    \]
    For flat curves \(v_\theta\to v_0\): \(p_{\text{swirl}}(r)=p_0+\rhof v_0^2 \ln(r/r_0)\).

    \paragraph{[Definition \(\to\) Corollary] Effective swirl line element (analogue-metric form).}
    In \((t,r,\theta,z)\) with azimuthal drift \(v_\theta(r)\),
    \[
        ds^2=-(c^2-v_\theta^2)\,dt^2+2\,v_\theta r\,d\theta\,dt+dr^2+r^2d\theta^2+dz^2,
    \]
    co-rotating to \(ds^2=-c^2(1-v_\theta^2/c^2)dt^2+\cdots\), giving the Swirl Clock factor
    \(\displaystyle \frac{dt_{\text{local}}}{dt_\infty}=\sqrt{1-\frac{v_\theta^2}{c^2}}\).

    \paragraph{[Definition] SST Hamiltonian density (Kelvin-compatible).}
    \[
        \mathcal{H}_{\text{SST}}=\tfrac12\rhof\,\|\vswirl\|^2+\tfrac12\rhof\,\rc^2\|\omegas\|^2+\lambda(\nabla\!\cdot\!\vswirl).
    \]

    \subsection*{Empirical Calibrations (not premises, but binding numerically)}
    \begin{itemize}
        \item \([{\rm Empirical}]\) \(\vscore = 1.09384563\times 10^6\,\mathrm{m\,s^{-1}}\).
        \item \([{\rm Empirical}]\) \(\rc = 1.40897017\times 10^{-15}\,\mathrm{m}\).
        \item \([{\rm Empirical}]\) \(\rhom = 3.8934358266918687\times 10^{18}\,\mathrm{kg\,m^{-3}}\).
    \end{itemize}

    \subsection*{Non-Canonical (Research Track)}
    Blackbody via swirl temperature, EM/SST minimal coupling, etc., remain conjectural until proven under \(\mathcal{S}\).

    \subsection*{Consistency \& Dimension Checks (illustrative)}
    \[
        [\Lambda]=[\rhom][\vscore^2][\rc^4]
        =\frac{\mathrm{kg}}{\mathrm{m^3}}\cdot\frac{\mathrm{m^2}}{\mathrm{s^2}}\cdot\mathrm{m^4}
        =\frac{\mathrm{kg\,m^3}}{\mathrm{s^2}}
        =\mathrm{J\,m}.
    \]
    Soft-core Coulomb recovery: \(V_{\text{SST}}(r)\to -\Lambda/r\) as \(r/\rc\to\infty\).



%================================================
% Canon §X: Coarse-Graining Strings → 3D Effective Density
%================================================
    \section{Canonical Coarse–Graining of \texorpdfstring{$\rhof$} from a Swirl–String Bath}
    \label{sec:canon_rhof_from_strings}

    \paragraph{Scope.}
    The medium is modeled as an incompressible, inviscid fluid populated by thin \emph{swirl strings}. We derive the bulk effective fluid density \(\rhof\) via coarse–graining of line–supported mass and vorticity, relying on Euler kinematics and Kelvin–Helmholtz invariants.

    \subsection{Axioms and Definitions}
    A representative string carries:
    \begin{align}
        \text{(D1)}\quad
        \mu_\ast &\;\equiv\; \rhom\,\pi \rc^{\,2}
        \quad\;[\mathrm{kg/m}], \\[2mm]
        \text{(D2)}\quad
        \Gamma_\ast &\;\equiv\; \oint \vswirl\!\cdot\! d\boldsymbol{\ell}
        \;\simeq\; \kappa_\Gamma\, \rc\, \vscore,
        \qquad \kappa_\Gamma=2\pi \;\; \text{(near–solid–body core)}.
    \end{align}
    Let
    \(
    \nu \equiv N_{\text{str}}/A \ [\mathrm{m^{-2}}]
    \)
    be the areal string density. Then:
    \begin{align}
        \text{(C1)}\quad
        \rhof &= \mu_\ast\,\nu, \\[2mm]
        \text{(C2)}\quad
        \langle \omegas\rangle &= \Gamma_\ast\,\nu\,\hat{\mathbf{t}}_{\text{avg}}
        \ \Rightarrow\  |\langle\omega_s\rangle|=\Gamma_\ast\,\nu.
    \end{align}

    \subsection{First–Principles Derivation}
    Combining (C1)–(C2):
    \begin{equation}
        \boxed{\;
        \rhof
        \;=\; \mu_\ast\,\frac{\langle\omega_s\rangle}{\Gamma_\ast}
        \;=\; \frac{\rhom\,\pi \rc^{\,2}}{\kappa_\Gamma \rc \vscore}\,\langle\omega_s\rangle
        \;=\; \frac{\rhom\,\rc}{2\,\vscore}\,\langle\omega_s\rangle
        \;}
        \quad (\kappa_\Gamma=2\pi).
        \label{eq:rhof_from_omega}
    \end{equation}
    For uniform solid–body rotation \(\Omega\), \(\langle\omega_s\rangle=2\Omega\),
    \begin{equation}
        \boxed{\;
        \rhof
        \;=\; \frac{\rhom\,\rc}{\vscore}\;\Omega
        \;}
        \quad [\mathrm{kg/m^3}].
        \label{eq:rhof_from_Omega}
    \end{equation}

    \paragraph{Energy and tension scales.}
    \[
        \boxed{\, u_{\text{swirl}}=\tfrac12\,\rhof\,\vscore^2 \,},\qquad
        \boxed{\, T_\ast = \tfrac12\,\mu_\ast\,\vscore^2 \,}.
    \]

    \subsection{Numerical Calibration (SST Canonical Constants)}
    With
    \(
    \rhom=3.8934358266918687\times10^{18}\ \mathrm{kg/m^3},\
    \rc=1.40897017\times10^{-15}\ \mathrm{m},\
    \vscore=1.09384563\times10^{6}\ \mathrm{m/s}
    \),
    one finds
    \[
        \Gamma_\ast = 2\pi \rc \vscore
        = 9.68361920\times10^{-9}\ \mathrm{m^2/s},\quad
        T_\ast = 1.45267535\times10^{1}\ \mathrm{N}.
    \]
    From \eqref{eq:rhof_from_Omega},
    \[
        \rhof = \bigl(5.01509060\times10^{-3}\bigr)\,\Omega,
    \]
    so the Canon baseline \( \rhof=7.0\times10^{-7}\ \mathrm{kg/m^3}\) occurs at
    \[
        \boxed{\ \Omega_\ast = 1.39578735\times10^{-4}\ \mathrm{s^{-1}}\ (\text{period } \approx 12.5\ \mathrm{h})\ }.
    \]

%================================================
% Master Equations (Boxed, Definitive)
%================================================
    \section{Master Equations (Boxed, Definitive)}

    \subsection{Master Energy and Mass Formula (SST)}
    \[
        \boxed{\ E_{\text{SST}}(V) = \frac{4}{\alpha\,\varphi} \left( \frac{1}{2}\,\rhof\,\vscore^{2} \right) V\ }\quad [\text{J}],
        \qquad
        \boxed{\ M_{\text{SST}}(V) = \frac{E_{\text{SST}}(V)}{c^{2}} \ }\quad [\text{kg}].
    \]
    Numerics per unit volume:
    \(
    \tfrac12\rhof \vscore^2 \approx 4.1877439\times10^{5}\ \mathrm{J\,m^{-3}},
    \frac{4}{\alpha\varphi} \approx 3.3877162\times10^{2},
    \Rightarrow E/V \approx 1.418688\times10^{8}\ \mathrm{J\,m^{-3}},
    M/V \approx 1.57850\times10^{-9}\ \mathrm{kg\,m^{-3}}.
    \)

    \subsection{Swirl–Gravity Coupling}
    \[
        \boxed{\ G_{\text{string}} = \frac{\vscore\ c^{5}\ t_p^{2}}{2\,F_{\text{EM}}^{\max}\ \rc^{2}} \ }
    \]
    Numerically \( \approx 6.674302\times10^{-11}\ \mathrm{m^3\,kg^{-1}\,s^{-2}}\) with the Canon constants.

    \subsection{Topology–Driven Mass Law (invariant form)}
    Let \(T(p,q)\) be a torus knot/link, \(n=\gcd(p,q)\) components, braid index \(b(T)=\min(|p|,|q|)\), Seifert genus \(g(T)\) (with standard link adjustment). Using ropelength \(\mathcal{L}_{\rm tot}(T)\) and string core radius \(\rc\):
    \[
        \boxed{
            M\big(T(p,q)\big)
            =\left(\frac{4}{\alpha}\right)\,
            b(T)^{-3/2}\,
            \varphi^{-\,g(T)}\,
            n^{-1/\varphi}\,
            \left(\frac{1}{2}\rhof \vscore^2\right)\,
            \frac{\pi \rc^3\,\mathcal{L}_{\mathrm{tot}}(T)}{c^2}.
        }
    \]
    Dimensionality follows from the factor \(\tfrac12\rhof \vscore^2\) (J\,m\(^{-3}\)) times a volume.

    \subsection{Swirl Clocks (Local Time-Rate)}
        \label{sec:swirl_clocks}
    \[
        \boxed{\ \frac{dt_{\text{local}}}{dt_{\infty}}
            = \sqrt{1 - \frac{\lVert\omegas\rVert^{2}\,\rc^{2}}{c^{2}}}
            = \sqrt{1 - \frac{\lVert\vswirl\rVert^{2}}{c^{2}}}\ \ (r=\rc)\ }.
    \]
    \emph{Historical (deprecated) variant without a length scale is retained only for traceability.}

    \subsection{Swirl Angular Frequency Profile}
    \[
        \boxed{\ \Omega_{\text{swirl}}(r) = \frac{\vscore}{\rc}\, e^{-r/\rc}\ },
        \qquad
        \Omega_{\text{swirl}}(0)=\frac{\vscore}{\rc}.
    \]

    \subsection{Vorticity Potential (Canonical Form)}
    \[
        \Phi(\vec r,\omegas) = \frac{\vscore^{2}}{2\,F_{\text{EM}}^{\max}}\ \omegas\!\cdot\!\vec r.
    \]
    \textbf{Dimensional remark:} Use with the SST Lagrangian ensuring \(\rhof\Phi\) has energy density units.

%================================================
% Unified SST Lagrangian (Definitive Form)
%================================================
    \section{Unified SST Lagrangian (Definitive Form)}
    \label{sec:lagrangian}

    Let $\vswirl$ be the velocity, $\rhof$ constant (incompressible), $\omegas=\nabla\times\vswirl$, and $\lambda$ enforce incompressibility.
    \[
        \boxed{\
        \mathcal{L}_{\text{SST}} =
            \frac{1}{2}\rhof\,\lVert\vswirl\rVert^{2}
            - \rhof\,\Phi(\vec r,\omegas)
            + \lambda(\nabla\cdot\vswirl)
            + \eta\,\int (\vswirl\cdot\omegas)\,dV
            + \mathcal{L}_{\text{couple}}[\Gamma,\mathcal{K}]
            \ }.
    \]
    Here $\mathcal{L}_{\text{couple}}$ encodes coupling to quantized circulation $\Gamma$ and knot invariants $\mathcal{K}$ (linking, writhe, twist).

%================================================
% Notation, Ontology, Glossary
%================================================
    \section{Notation, Ontology, and Glossary}
    \begin{itemize}
        \item \textbf{Absolute time (A-time):} global time parameter of the medium.
        \item \textbf{Chronos Time (C-time):} asymptotic observer time ($dt_{\infty}$).
        \item \textbf{Swirl Clocks:} local clocks set by \(\lVert\omegas\rVert\) or \(\lVert\vswirl\rVert\) per Sec.~\ref{sec:lagrangian}.
        \item \textbf{String taxonomy:} leptons = torus knots; quarks = chiral hyperbolic knots; bosons = unknots; neutrinos = linked knots.
        \item \textbf{Chirality:} ccw $\leftrightarrow$ matter; cw $\leftrightarrow$ antimatter via swirl–gravity coupling.
    \end{itemize}


%================================================
% Photons as lossless swirl waves
%================================================

    \section{Unknot bosons and lossless swirl radiation}

    \paragraph{Postulate (Topological sector).}
    Let $\mathcal U$ denote an \emph{unknotted} closed swirl string (topological unknot, Hopf charge $\mathcal H=0$).
    Imposing Finkelstein--Rubinstein constraints for single-valued many-body wavefunctionals on the configuration space of closed strings yields \emph{integer spin} sectors for $\mathcal U$:
    \[
    \boxed{\ \mathcal U \;\Rightarrow\; \text{bosonic sector}\ }.
    \]
    (Nontrivial knot/link classes supply the sign structure needed for half-integer spin.)~\cite{FinkelsteinRubinstein1968}

    \paragraph{Field variables and lossless propagation.}
    Introduce a transverse swirl potential $\mathbf{a}(\mathbf{x},t)$ with
    \[
    \mathbf{v} \equiv \partial_t \mathbf{a},\qquad
    \mathbf{b} \equiv \nabla\times \mathbf{a},\qquad \nabla\cdot \mathbf{a}=0,
    \]
    and take the quadratic effective Lagrangian density
    \[
    \mathcal{L}_\text{swirl}
    =\frac{\rho_f}{2}\,|\mathbf{v}|^2-\frac{\rho_f c^2}{2}\,|\mathbf{b}|^2,
    \]
    with $\rho_f$ the effective (coarse-grained) density and $c$ the observed luminal wave speed.
    Euler–Lagrange equations give the \emph{lossless} wave equation
    \[
    \boxed{\ \partial_t^2 \mathbf{a}-c^2\,\nabla\times(\nabla\times \mathbf{a})=0,\qquad \nabla\cdot \mathbf{a}=0\ },
    \]
    with conserved energy density and flux
    \[
    u=\frac{\rho_f}{2}\Big(|\mathbf{v}|^2+c^2|\mathbf{b}|^2\Big),\qquad
    \mathbf{S}=\rho_f c^2\,\mathbf{v}\times \mathbf{b},\qquad
    \partial_t u+\nabla\cdot \mathbf{S}=0,
    \]
    and momentum density $\mathbf{g}=\mathbf{S}/c^2$ (Noether).
    Inviscid, incompressible background (Kelvin/Helmholtz) implies circulation is materially conserved; no viscous dissipation appears.~\cite{Batchelor1967,Saffman1992}

    \paragraph{Photon map (delocalized oscillatory circulation).}
    Identify electromagnetic fields by a constant rescaling
    \[
    \boxed{\ \mathbf{E}=\sqrt{\frac{\rho_f}{\varepsilon_0}}\,\mathbf{v},\qquad
    \mathbf{B}=\sqrt{\frac{\rho_f}{\varepsilon_0}}\,\mathbf{b}\ },
    \]
    so that
    \[
    u=\frac{\varepsilon_0}{2}|\mathbf{E}|^2+\frac{1}{2\mu_0}|\mathbf{B}|^2,\qquad
    \mathbf{S}=\frac{1}{\mu_0}\,\mathbf{E}\times \mathbf{B},\qquad
    \frac{1}{\varepsilon_0\mu_0}=c^2,
    \]
    exactly reproducing the Maxwell energy–momentum balance for radiation.~\cite{Jackson1999}
    Plane- and spherical-wave solutions of $\mathcal{L}_\text{swirl}$ thus realize \emph{photons} as \emph{delocalized}, time-periodic circulation modes.

    \paragraph{Quantization and single-photon amplitude.}
    Canonical quantization of a cavity mode with volume $V$ at frequency $\omega$ gives the standard one-photon field amplitude
    \[
    E_{\mathrm{rms}}^{(1)}=\sqrt{\frac{\hbar\omega}{2\varepsilon_0 V}},
    \]
    hence the swirl velocity amplitude
    \[
    \boxed{\ v_{\mathrm{rms}}^{(1)}=\sqrt{\frac{\hbar\omega}{2\rho_f V}}\ }.
    \]
    For example, with $\lambda=532\,\mathrm{nm}$ (green), $\omega=2\pi c/\lambda$, and $\rho_f=7.0\times 10^{-7}\,\mathrm{kg\,m^{-3}}$,
    \[
    V=1~\mathrm{mm}^3:\quad v_{\mathrm{rms}}^{(1)}\approx 3.27\times 10^{-2}\ \mathrm{m\,s^{-1}},
    \]
    consistent with $E_{\mathrm{rms}}^{(1)}$ and observed single-photon couplings in cavity QED.~\cite{HarocheRaimond2006,ScullyZubairy1997}

    \paragraph{Radiation from bound strings (``atoms'').}
    A localized bound swirl configuration with time-dependent multipole moment $\mathbf{d}(t)$ sources transverse $\mathbf{a}$, producing outward, concentric, divergence-free wavefronts. Far from the source ($r\gg$ size), the solution is
    \[
    \mathbf{a}(\mathbf{x},t)\ \propto\ \frac{\mathbf{e}_\perp}{r}\,\mathrm{Re}\!\left(e^{i(kr-\omega t)}\right),\qquad k=\omega/c,
    \]
    with Poynting flux $\mathbf{S}=\rho_f c^2\,\mathbf{v}\times\mathbf{b}$ radial and $|\mathbf{S}|\propto r^{-2}$, ensuring constant radiated power through spheres, as in Maxwell theory.~\cite{Jackson1999}
    Thus: \emph{atoms launch concentric swirling wave-fronts; the lossless foliation transmits them without attenuation.}

    \paragraph{Exclusion of smoke-ring photons.}
    Classical vortex-ring energetics $E_{\mathrm{vr}},P_{\mathrm{vr}}$ cannot simultaneously match $E=\hbar\omega$ and $p=\hbar k$ with causal core speeds for your $(\rho_f,r_c)$; hence localized unknot smoke-rings do not realize photons in vacuum.~\cite{Saffman1992,Batchelor1967}

    \paragraph{Summary.}
    \[
    \boxed{\ \mathcal U\ \text{(unknot)}\ \Rightarrow\ \text{boson};\quad
    \text{photons}=\text{delocalized, lossless swirl waves launched by bound sources.}\ }
    \]


%===========================================================
\subsection{Photon as a Pulsed Unknot with Delocalized Circulation}
%===========================================================

    We represent the photon as a delocalized circulation mode of an unknot swirl-string $K \cong S^1$, with radius $R$ and circumference $L=2\pi R$.  Unlike massive particles (localized knots with core density $\rhom$), the photon has no rest-mass contribution ($\rhom = 0$), and its energy is entirely carried by oscillatory delocalized swirl modes in the effective fluid density $\rhof$.

    \paragraph{Effective Action.}
    Introduce a transverse swirl-displacement field $\xi(s,t)$ defined along the ring coordinate $s \in [0,L)$, with tubular cross-sectional area $A_{\mathrm{eff}} = \pi w^2$. The delocalized photon mode is described by the effective 1D action
    \begin{equation*}
    S[\xi]
    = \frac{1}{2}\,\rhof A_{\mathrm{eff}} \int dt \int_0^L ds \,
    \Big[ \, (\partial_t \xi)^2 - c^2 (\partial_s \xi)^2 \, \Big],
    \end{equation*}
    which yields the wave equation
    \begin{equation*}
    \partial_t^2 \xi - c^2 \,\partial_s^2 \xi = 0,
    \qquad \xi(s+L,t) = \xi(s,t).
    \end{equation*}

    \paragraph{Normal Modes.}
    Periodic boundary conditions imply discrete wavenumbers
    \begin{equation*}
    k_m = \frac{2\pi m}{L},
    \qquad \omega_m = c\,k_m,
    \qquad m \in \mathbb{Z}_{\ge 1}.
    \end{equation*}
    A single mode solution is
    \begin{equation*}
    \xi_m(s,t) = a_m \cos\!\big(k_m s - \omega_m t\big).
    \end{equation*}

    \paragraph{Mode Energy.}
    The time-averaged energy of mode $m$ is
    \begin{equation*}
    E_m = \rhof \, A_{\mathrm{eff}} \, L \, \omega_m^2 \, a_m^2,
    \end{equation*}
    which depends on the delocalized volume $A_{\mathrm{eff}}L$ rather than a compact core ($r_c$). Thus the photon energy resides in the distributed swirl mode rather than a localized mass density.

    \paragraph{Quantization.}
    Assigning a quantum of energy $\hbar \omega_m$ to each mode yields the amplitude
    \begin{equation*}
        a_m = \sqrt{\frac{\hbar}{\rhof \, A_{\mathrm{eff}} \, L \, \omega_m}}.
    \end{equation*}
    For a photon of wavelength $\lambda$, we set
    \begin{equation*}
        R = \frac{\lambda}{2\pi}, \qquad
        L = \lambda, \qquad
        \omega = \frac{2\pi c}{\lambda}, \qquad
        w \sim \frac{\lambda}{2\pi}, \qquad
        A_{\mathrm{eff}} = \pi w^2.
    \end{equation*}

    \paragraph{Numerical Example.}
    For $\lambda = 500\,\mathrm{nm}$ and $\rhof = 7.0 \times 10^{-7}\,\mathrm{kg\,m^{-3}}$,
    we obtain
    \begin{align}
    a &\approx 2.0 \times 10^{-12}\,\mathrm{m},\\
    E &= \hbar \omega \;\approx\; 3.97 \times 10^{-19}\,\mathrm{J}\;\; (2.48\,\mathrm{eV}).
    \end{align}

    \paragraph{Interpretation.}
    The photon is therefore modeled as a \emph{pulsed unknot swirl-string}, with vanishing rest-mass density ($\rhom=0$), but finite delocalized energy density
    \begin{equation}
    \rhoE \;=\; \tfrac{1}{2}\rhof\!\left[(\partial_t \xi)^2
    + c^2 (\partial_s \xi)^2\right],
    \end{equation}
    integrated across the mode volume.
    It is neither purely localized nor purely delocalized, but consists of a minimal swirl core pulsed to launch delocalized circulation waves---precisely analogous to a pulsed rotor in water launching concentric swirling motions.



%================================================
% Canonical Checks
%================================================
    \section{Canonical Checks (What to Verify in Every Paper)}
    \begin{enumerate}
        \item Dimensional analysis on every new term/equation.
        \item Limits: low-swirl \(\lVert\omegas\rVert\!\to\!0\) recovers classical mechanics/EM; large-scale averages reproduce Newtonian gravity with \(G_{\text{string}}\).
        \item Numerics: provide prefactors using Canon constants; add any new constants to Sec.~\ref{sec:canonical_constants}.
        \item Explicit topology \(\leftrightarrow\) quantum mapping (which invariants, normalization).
        \item Cite any non-original constructs (BibTeX keys).
    \end{enumerate}


%================================================
% Swirl Hamiltonian Density (Canonical Form)
%================================================
        \section{Swirl Hamiltonian Density (Canonical Form)}
        \label{sec:hamiltonian}
        With the effective fluid density \(\rhof\), the swirl vorticity \(\omegas=\nabla\times \vswirl\), and a Lagrange multiplier \(\lambda\) to enforce incompressibility, a Kelvin-compatible, dimensionally normalized Hamiltonian density is given by:
        \begin{equation}
        \mathcal{H}_{\text{SST}}[\vswirl] = \frac{1}{2}\rhof\,\|\vswirl\|^2 + \frac{1}{2}\rhof\,\rc^2\,\|\omegas\|^2 + \frac{1}{2}\rhof\,\rc^4\,\|\nabla \omegas\|^2 + \lambda (\nabla\cdot \vswirl).
        \label{eq:Hamiltonian_SST}
        \end{equation}
        All terms carry units of energy density (J·m\(^{-3}\)). The first two terms represent the kinetic and rotational energy of the swirl, while the third term, proportional to the gradient of vorticity, corresponds to the energy associated with the curvature or "stiffness" of the swirl filaments. In the limit where the core radius \(\rc\to 0\) or for spatially uniform vorticity, this expression reduces to the simpler forms.

%================================================
% Swirl Pressure Law (Euler Corollary)
%================================================
        \section{Swirl Pressure Law (Euler Corollary)}
        \label{sec:darkpressure}
        For a steady, purely azimuthal drift velocity $v_\theta(r)$ with no radial flow, the radial component of the Euler momentum equation for an inviscid fluid provides a direct relationship for the swirl pressure gradient:
        \begin{equation}
        \frac{1}{\rhof}\frac{dp_{\text{swirl}}}{dr} = \frac{v_\theta(r)^2}{r}.
        \label{eq:swirl_pressure_law}
        \end{equation}
        This is a canonical theorem derived directly from first principles. For a system exhibiting an asymptotically flat rotation curve where $v_\theta(r) \to v_0$ for large $r$, the pressure profile is found by integration:
        \begin{equation}
        p_{\text{swirl}}(r) = p_0 + \rhof v_0^2 \ln\left(\frac{r}{r_0}\right).
        \end{equation}
        Here, $p_0$ is the pressure at a reference radius $r_0$. The resulting outward-rising pressure creates an inward-pointing force ($-\nabla p_{\text{swirl}}$), providing the centripetal acceleration required to maintain the flat rotation curve.


%================================================
% Experimental Protocols
%================================================
        \section{Experimental Protocols (Canon-ready)}
        \label{sec:experiments}

        \subsection*{Universality of $\vswirl = f\,\Delta x$ (metrology across platforms)}
            From \texttt{ExperimentalValidationOfVortexCoreTangientalVelocity.tex}: measure a natural frequency $f$ and a spatial step $\Delta x$ from standing/propagating modes; verify
            \begin{equation}
            \boxed{\ \vswirl = f\,\Delta x \approx \num{1.09384563e6}\ \si{m/s}\ } \tag{X1}
            \end{equation}
            Platforms: magnet/electret domains, laser interferometry on coil-bound modes, and acoustic analogues. Require ppm-level agreement; report mean and standard deviation across platforms.

        \subsection*{Swirl gravitational potential}
            From \texttt{ExperimentalValidationOfGravitationalPotential.tex}: infer $p_{\text{swirl}}(r)$ from centripetal balance (\S\ref{sec:darkpressure}) and compare predicted forces with measured thrust or buoyancy anomalies in shielded high-voltage/coil experiments (geometry: starship/Rodin coils). Ensure dimensional consistency and calibrate only via Canon constants.


%================================================
% END OF CANONICALS
%================================================

%================================================
% APPENDIX
%================================================
\appendix

%================================================
% Boxed Canon Equations (paste-ready)
%================================================
    \section*{Appendix A: Boxed Canon Equations (paste-ready)}
    \begin{enumerate}
        \item \textbf{Energy:} \fbox{$E_{\text{SST}} = \dfrac{4}{\alpha\varphi}\left(\dfrac{1}{2}\rhof \vscore^2\right)V$}
        \item \textbf{Mass:} \fbox{$M_{\text{SST}} = \dfrac{E_{\text{SST}}}{c^2}$}
        \item \textbf{$G$ coupling:} \fbox{$G_{\text{string}} = \dfrac{\vscore\, c^5 t_p^2}{2F_{\text{EM}}^{\max} \rc^2}$}
        \item \textbf{Swirl Clock:} \fbox{$\dfrac{dt_{\text{local}}}{dt_{\infty}}=\sqrt{1-\lVert\omegas\rVert^2 \rc^2/c^2}=\sqrt{1-\lVert\vswirl\rVert^2/c^2}$}
        \item \textbf{Swirl profile:} \fbox{$\Omega_{\text{swirl}}(r) = \dfrac{\vscore}{\rc}e^{-r/\rc}$}
    \end{enumerate}


%================================================
% Operational Kinematics (Non-fundamental)
%================================================

\section*{Appendix B: Operational Kinematics (Measurement Layer)}
\addcontentsline{toc}{section}{Operational Kinematics (Measurement Layer)}

		\paragraph{Axioms (signal layer, not fundamental).}
		(A1) In a local lab patch $\mathcal U$, lightlike signals follow $ds^2 = c^2 dt^2 - \|d\mathbf x - \mathbf u(\mathbf x)\,dt\|^2$
		with slowly varying drift $\mathbf u$ (PG form, kinematic ansatz).
		(A2) Clocks measure proper time via a \emph{factorized} rule
		\[
			d\tau \;=\; S_t(\Omega)\,\sqrt{1-\tfrac{v^2}{c^2}}\;dt,
		\]
		where $S_t(\Omega)\in(0,1]$ encodes swirl-clock (SST) and $\sqrt{1-v^2/c^2}$ is the purely kinematic SR term.
		(A3) Frequencies are compared by counting cycles along null paths (longitudinal, transverse, angle-resolved).

		\paragraph{Theorem O.1 (Rapidity composition, measurement form).}
		Let $\beta \equiv v/c$, $\xi \equiv \tanh^{-1}\!\beta$. Between co-moving leaves with drifts $\beta_1,\beta_2$ along a fixed axis,
		\[
			\xi_{\rm rel}=\xi_2-\xi_1,
			\qquad
			\beta_{\rm rel}=\tanh(\xi_{\rm rel})=\frac{\beta_2-\beta_1}{1-\beta_1\beta_2}.
		\]
		\emph{Sketch:} choose an orthonormal tetrad in $\mathcal U$; null calibration gives the standard Lorentz algebra in the tangent space; group composition is additive in $\xi$.

		\paragraph{Corollary O.2 (Doppler observables).}
		Longitudinal:
		\[
			\frac{f_{\rm obs}}{f_{\rm src}}=\sqrt{\frac{1-\beta}{1+\beta}} \;=\; e^{-\xi}.
		\]
		Transverse (pure time dilation):
		\[
			\frac{f_{\perp,{\rm obs}}}{f_{\rm src}}=\sqrt{1-\beta^2}=\frac{1}{\gamma}.
		\]
		Angle-resolved (optional):
		\[
			\frac{f_{\rm obs}}{f_{\rm src}}=\gamma\bigl(1-\beta\cos\theta\bigr).
		\]
		All are dimensionless; $\beta\to 0$ gives $f_{\rm obs}\to f_{\rm src}$.

		\paragraph{Corollary O.3 (Synchronization offset on the foliation).}
		For a baseline $\mathbf L$ at fixed $t$ with uniform drift $\mathbf u = v\,\hat{\mathbf x}$,
		\[
			\Delta t'=\gamma\!\left(\Delta t-\frac{v\,\Delta x}{c^2}\right),
			\qquad
			\Delta t'\big|_{\Delta t=0}=-\,\gamma\,\frac{v\,\Delta x}{c^2}
			\;\simeq\; -\,\frac{\mathbf u\!\cdot\!\mathbf L}{c^2}\quad (\beta\ll 1).
		\]

		\paragraph{Guardrails (to avoid interference with condensed sectors).}
		(I) Never multiply $S_t(\Omega)$ into the Doppler formulas; those calibrate the \emph{kinematic} factor only.
		(II) If using optical carriers inside media, replace $c\!\to\! c/n$ for \emph{signal} propagation, but keep clock $S_t(\Omega)$ from matter standards (e.g. atomic transitions).

		\paragraph{Dimensional checks.}
		$\xi$ and frequency ratios are dimensionless; $\Delta t'$ carries time units via $(v\,\Delta x)/c^2$.


% ================== End Operational Kinematics ==================

% =========================================================
% SST: Invariant Mass from the Canonical Lagrangian
% =========================================================

\section*{Appendix C: Invariant Mass from the Canonical Lagrangian}

        Starting from the schematic Lagrangian
        \[
            \mathcal{L}_{\text{SST}}
            = \rhof\!\left(\tfrac{1}{2}\vswirl^2 - \Phi_{\text{swirl}}\right)
            + \tfrac{1}{4}F_{\mu\nu}F^{\mu\nu}
            + \big(\alpha C(K)+\beta L(K)+\gamma \mathcal{H}(K)\big)
            + \rhof \ln\!\sqrt{1-\tfrac{\|\boldsymbol\omega\|^2}{c^2}}
            + \Delta p(\text{swirl}),
        \]
        the \emph{mass sector} reduces, under the slender-tube approximation, to an invariant energy functional
        \[
            E(K)= u\,V(K)\,\Xi_{\text{top}}(K),\qquad
            u=\tfrac{1}{2}\rho_{\text{core}}\;v_{\circlearrowleft}^{2},
        \]
        with $u$ the swirl energy density scale on the core, $V(K)$ the effective tube volume of the swirl string, and $\Xi_{\text{top}}(K)$ a dimensionless topological multiplier summarizing discrete combinatorial and contact/helicity corrections. In SST we adopt
        \[
            V(K)\;=\;\pi r_c^2 \underbrace{\big(L_{\rm phys}\big)}_{=\,r_c\,L_{\rm tot}}
            \;=\;\pi r_c^3\,L_{\rm tot},
        \]
        where $r_c$ is the core radius and $L_{\rm tot}$ is the \emph{dimensionless ropelength}. The rest mass is $M=E/c^2$.

        \paragraph{Canonical multiplier.}
            Guided by the EM coupling and SST’s discrete scaling rules, we take
            \[
                \Xi_{\text{top}}(K)=\frac{4}{\alpha_{\rm fs}}\;b^{-3/2}\;\varphi^{-g}\;n^{-1/\varphi},
            \]
            where $b,g,n$ are the integer topology labels used in the Canon (e.g. torus index, layer, linkage count), $\alpha_{\rm fs}$ is the fine-structure constant, and $\varphi$ the golden ratio. Collecting factors, the \textbf{invariant mass law} used in the code is
            \begin{equation*}
                \boxed{M(K)=\frac{4}{\alpha_{\rm fs}}\;b^{-3/2}\;\varphi^{-g}\;n^{-1/\varphi}\;
                \frac{u\,\pi r_c^3 L_{\rm tot}}{c^2},
                    \qquad
                    u=\tfrac{1}{2}\rho_{\text{core}}v_{\circlearrowleft}^2.
                    }\label{eq:SST-invariant-mass}
            \end{equation*}

        \paragraph{Leptons (solved $L_{\rm tot}$).}
            For a lepton with labels $(b,g,n)$ and known mass $M_\ell^{\rm(exp)}$, invert \eqref{eq:SST-invariant-mass}:
            \[
                L_{\rm tot}^{(\ell)} \;=\;
                \frac{M_\ell^{\rm(exp)}\,c^2}{\big(\tfrac{4}{\alpha_{\rm fs}}\,b^{-3/2}\varphi^{-g}n^{-1/\varphi}\big)\,u\,\pi r_c^3}.
            \]

        \paragraph{Baryons (exact closure).}
            Let the proton and neutron ropelengths be
            \[
                L_p=\lambda_b\,(2s_u+s_d)\,\mathcal S,\qquad
                L_n=\lambda_b\,(s_u+2s_d)\,\mathcal S,\qquad
                \mathcal S=2\pi^2\kappa_R,\;\;\kappa_R=2,
            \]
            with $(s_u,s_d)$ dimensionless sector weights and $\lambda_b$ a sector scale (set to $1$ in exact-closure).
            Imposing $M_p^{\rm(exp)}=M_p$ and $M_n^{\rm(exp)}=M_n$ in \eqref{eq:SST-invariant-mass} yields a \emph{linear} $2\times2$ system for $(s_u,s_d)$:
            \[
                \begin{bmatrix}
                2 & 1\\[2pt]
                1 & 2
                \end{bmatrix}
                \begin{bmatrix}
                s_u\\ s_d
                \end{bmatrix}
                =
                \frac{1}{K}
                \begin{bmatrix}
                M_p^{\rm(exp)}\\ M_n^{\rm(exp)}
                \end{bmatrix},
                \qquad
                K=\Big[\tfrac{4}{\alpha_{\rm fs}}\,3^{-3/2}\,\varphi^{-2}\,3^{-1/\varphi}\Big]\frac{u\,\pi r_c^3\,\mathcal S}{c^2}.
            \]
            Solving gives
            \[
                s_u=\frac{2M_p^{\rm(exp)}-M_n^{\rm(exp)}}{3K},
                \qquad
                s_d=\frac{M_p^{\rm(exp)}}{K}-2s_u.
            \]

        \paragraph{Composites (no binding).}
            For an atom with proton number $Z$ and neutron number $N$ (atomic mass includes $Z$ electrons),
            \[
                M_{\rm atom}^{(\rm pred)} = Z\,M_p+N\,M_n+Z\,M_e,\quad
                M_{\rm mol}^{(\rm pred)}=\sum_{\text{atoms}}M_{\rm atom}^{(\rm pred)}.
            \]
            Deviations from experiment in atoms/molecules correspond to \emph{binding energies} not included in this baseline (nuclear $\sim\!8\,{\rm MeV}$ per nucleon; molecular $\sim{\rm eV}$).

% ---------------------------------------------------------
        \subsection{Benchmarks (exact\_closure mode)}
        \label{sec:benchmarks-exact-closure}
        The following table was generated by the Python file listed after it.
        \emph{Errors in atoms/molecules = missing binding energy contribution, not model failure.}

        \begin{table}[H]
        \centering
        \caption{Invariant-kernel mass benchmarks (exact\_closure). \emph{Errors in atoms/molecules = missing binding energy contribution, not model failure.}}
        \begin{tabular}{lccc}
        \toprule
        Species & Known mass (kg) & Predicted mass (kg) & Error (\%)\\
        \midrule
        electron e- & 9.109384e-31 & 9.109384e-31 & 0.0000\\
        muon $\mu$- & 1.883532e-28 & 1.883532e-28 & 0.0000\\
        tau $\tau$- & 3.167540e-27 & 3.167540e-27 & 0.0000\\
        proton p & 1.672622e-27 & 1.672622e-27 & 0.0000\\
        neutron n & 1.674927e-27 & 1.674927e-27 & 0.0000\\
        Hydrogen-1 atom & 1.673533e-27 & 1.673533e-27 & 0.0000\\
        Helium-4 atom & 6.646477e-27 & 6.689952e-27 & 0.6549\\
        Carbon-12 atom & 1.992647e-26 & 2.005276e-26 & 0.6330\\
        Oxygen-16 atom & 2.656017e-26 & 2.674532e-26 & 0.6980\\
        H$_2$ molecule & 3.367403e-27 & 3.347066e-27 & -0.6040\\
        H$_2$O molecule & 2.991507e-26 & 3.009885e-26 & 0.6139\\
        CO$_2$ molecule & 7.305355e-26 & 7.354340e-26 & 0.6704\\
        \bottomrule
        \end{tabular}\label{tab:benchmarks-exact-closure}
        \end{table}

% ---------------------------------------------------------
        \subsection*{Notes}
        \begin{itemize}
        \item Elementary entries are exact by construction in exact\_closure mode (leptons solved from $L_{\rm tot}$; $p,n$ from closure).
        \item Composite errors track omitted binding: nuclear $\mathcal O(10^{-3})$–$\mathcal O(10^{-2})$, molecular $\mathcal O(10^{-9})$.
        \end{itemize}

% ---------------------------------------------------------


%================================================
% Personas (unchanged logic, SST names)
%================================================
\section*{Appendix D: Persona Prompts}
        \label{sec:personas}

        \subsection*{Reviewer Persona}
            \scriptsize
            You are a peer reviewer for an SST paper. Use only the definitions and constants in the "SST Canon (\canonversion)".
            Check dimensional consistency, limiting behavior, and numerical validation. Flag any use of non-canonical
            constants or equations unless equivalence is proved. Demand explicit mapping from knot invariants (linking,
            writhe, twist) to claimed quantum numbers.

        \subsection*{Theorist Persona}

            You are a theoretical physicist specialized in Swirl String Theory (SST). Base all reasoning on the attached
            "SST Canon (\canonversion)". Your task: derive the swirl-based Hamiltonian for [TARGET SYSTEM], use Sec.~\ref{sec:lagrangian},
            and verify the Swirl Clock law (Sec. \ref{sec:swirl_clocks}). Provide boxed equations, dimensional checks, and a short numerical
            evaluation using the Canon constants.


        \subsection*{Bridging Persona (Compare to GR/SM)}

            Work strictly within SST Canon (\canonversion). Compare [TARGET] to its GR/SM counterpart. Identify exact replacements
            (e.g., curvature → swirl), and show which terms reduce to Newtonian/Maxwellian limits. Include a correspondence
            table and any constraints needed for equivalence.


%================================================
% Session Kickoff Checklist
%================================================
            \normalsize
\section*{Appendix E: Session Kickoff Checklist}
        \begin{enumerate}
        \item Start new chat per task; attach this Canon first.
        \item Paste a persona prompt (Sec.~\ref{sec:personas}).
        \item Attach only task-relevant papers/sources.
        \item State any corrections explicitly (they persist in the session).
        \item At end, record Canon deltas (if any) and bump version.
        \end{enumerate}

%================================================
% References
%================================================
        \nocite{*}
        \bibliographystyle{unsrt}
        \bibliography{canon_swirl_string_theory}
\end{document}

