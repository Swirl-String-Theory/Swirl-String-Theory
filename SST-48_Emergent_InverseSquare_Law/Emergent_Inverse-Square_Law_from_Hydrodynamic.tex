\documentclass[12pt]{article}
\usepackage[margin=1in]{geometry}
\usepackage{amsmath,amssymb}
\usepackage{authblk}
\begin{document}

    \title{Emergent Inverse-Square Law from Hydrodynamic \\
    Dynamics in a Lorentzian Background}
    \author{Omar Iskandarani}
    \affil{Independent Researcher, Groningen, The Netherlands}
    \date{}
    \maketitle

    \begin{abstract}
        We present a first-principles derivation of the inverse-square law for static, weak gravitational forces as an emergent phenomenon in a flat, Lorentzian operational background.  Building on a recent relational-clock formulation of the Equivalence Principle~\cite{Iskandarani2025}, we demonstrate explicitly how a $1/r^2$ dependence arises in a hydrodynamic vortex model of matter consistent with Swirl-String Theory (SST). Three independent approaches are developed: (i) a \textit{relational clock coupling} argument showing that a universal energy–energy interaction leads to a $1/r$ potential when locality is imposed; (ii) a \textit{scalar effective field theory} for a foliation field sourced by mass, whose static solution yields the $1/r^2$ field flux of a monopole; and (iii) a \textit{Gauss-law flux} derivation based on the stress-energy tensor of the SST scalar field, which reveals an area-weighted momentum flow that falls off as $1/r^2$. We identify the SST field (the ``swirl-clock'' or foliation scalar) that carries the far-field momentum flux and derive its quadratic action and stress-energy tensor. The monopole solution is shown to produce a radial momentum flux density $\propto 1/r^2$, with total ``charge'' $Q$ proportional to the enclosed mass. Moreover, we demonstrate that substituting the usual Newtonian potential $\chi$ with the SST swirl-clock field leaves the Gauss-law form of the monopole solution unchanged, so the $1/r^2$ behavior emerges automatically from the field dynamics. These results reinforce the physical consistency of treating gravity as an emergent field in a flat background, and illustrate how the inverse-square law is a generic consequence of Lorentz-invariant mediation in three spatial dimensions. Mainstream implications are discussed, emphasizing that Newton’s law need not be a fundamental input but can arise from deeper relational and field-theoretic principles.
    \end{abstract}

    \section{Introduction}
        Gravity in Newtonian and relativistic physics is characterized by an inverse-square force law, $F \propto 1/r^2$, associated with a $1/r$ potential for a point source. In General Relativity (GR), this distance dependence emerges from solving Einstein’s field equations in the weak-field limit, but it is ultimately enforced by the same geometrical statement as Gauss’s law: in three spatial dimensions, a spherically symmetric source produces flux diluting as $1/r^2$. The Equivalence Principle (EP) ensures that this flux couples to all bodies identically, leading to universal free-fall. In conventional formulations, these properties are taken as empirical facts built into the geometric theory. A recent work, however, showed that the EP can itself be seen as an \emph{emergent} symmetry rather than a fundamental axiom, using a relational time formulation embedded in a Minkowski causal structure~\cite{Iskandarani2025}. That approach treated Lorentzian spacetime as an operational background and derived gravitational redshift and a Newtonian-like interaction from a mild deformation of a global quantum-clock constraint. While it recovered composition-independent free-fall (the EP) and the existence of a gravitational potential, the \emph{distance dependence} of that potential---the $1/r$ profile---was introduced via an external identification rather than derived from first principles. In this paper, we address that gap and demonstrate how the $1/r^2$ law arises naturally within the same overall framework.

        We work in the context of a hydrodynamic vacuum model known as Swirl-String Theory (SST), in which matter is described by stable vortex structures in an incompressible medium. Gravitation in SST is not a fundamental geometric interaction but an emergent, long-range effect mediated by one of the fluid’s collective modes (a ``swirl-clock’’ scalar field associated with local temporal flow). Crucially, the medium supports Lorentzian signal propagation as a fundamental symmetry, so information and influences cannot spread instantaneously. This provides a field-theoretic mechanism for gravity: mass-energy induces disturbances in the medium that propagate at finite speed, producing an effect analogous to a gravitational potential. Here we show that under very general conditions, those disturbances obey a \emph{Gauss law} and yield an inverse-square dependence in the static limit.

        Our derivation is presented in three complementary ways:
        \begin{enumerate}\itemsep=0pt
        \item In Section~2, we use a \textbf{relational clock coupling} argument to show that a universal coupling between two systems’ energy operators gives rise to an effective potential. By invoking locality (propagation in the Lorentzian background), we argue that this potential must scale as $1/r$. This reasoning bridges the gap between a quantum gravitational clock model and the classical $1/r^2$ force.
        \item In Section~3, we formulate a \textbf{quadratic effective field theory} for a scalar ``foliation’’ field $T(x)$ (interpreted as the SST swirl-clock field) that is sourced by mass density. We derive the field equation and solve it for a static, spherically symmetric source. The solution is the familiar $1/r$ profile of a monopole. We identify which SST field carries the far-field momentum flux and compute its stress-energy tensor $T_{\mu\nu}$, verifying that the energy flux is consistent with a $1/r^2$ distribution.
        \item In Section~4, we invoke \textbf{momentum-flux (Gauss law) conservation}. Starting from the stress-energy tensor of the scalar field, we show that in the static regime the field’s momentum flow (radial pressure) obeys $\nabla\!\cdot\!\mathbf{T}^{r} = 0$ outside the source. Spherical symmetry then implies $r^2 T^{r}{}_{r} = \text{constant}$, i.e. the outward momentum flux through any sphere around the source is constant. This is the hallmark of an inverse-square field. We relate that constant to the source mass (enclosed $\int \rho_m d^3x$), thus making contact with Newton’s law.
        \end{enumerate}

        In Section~5, we discuss the consistency of these results and emphasize that replacing the Newtonian potential $\chi$ with the SST foliation (swirl-clock) field does not change the monopole solution’s form. The inverse-square behavior ``drops out’’ of the field equations automatically---any scalar mediator in three dimensions will produce it. We also comment on the differences between this emergent picture and GR’s geometric picture, and argue that the former retains the empirically required features (EP, $1/r^2$ law) while suggesting novel perspectives (e.g. clock-dependent effects at higher order~\cite{SinghFriedrich2025}). An Appendix connects the present work to prior SST results, showing how independent developments (Kelvin mode stability, thermodynamic calibration of the vacuum, and the variational origin of particle moments) support the assumptions made here.

        Throughout, we use canonical field-theory notation in flat spacetime. A metric $\eta_{\mu\nu} = \operatorname{diag}(1,-1,-1,-1)$ is assumed for raising and lowering indices (we set $c=1$ except where restoring it for clarity). The weak-field, low-velocity regime is considered, so we neglect any nonlinear gravitational self-coupling or radiation reaction. Our goal is to elucidate the \emph{origin} of the $1/r^2$ law in this emergent context, rather than to reproduce all familiar corrections or post-Newtonian terms.

    \section{Relational Clock Coupling and the $1/r^2$ Law}
        One route to emergent gravity is through \emph{relational time} – the idea that what we call time evolution can arise from correlations between subsystems in a stationary, timeless universe. In such approaches, exemplified by the Page–Wootters mechanism and related models, a “global” Hamiltonian constraint is imposed, $H_{\rm total}|\Psi\rangle = 0$, coupling a clock subsystem to the rest of the world. Time is an internal degree of freedom carried by the clock, and other subsystems evolve relative to it. Recently, Singh and Friedrich~\cite{SinghFriedrich2025} and, in a broader context, Iskandarani~\cite{Iskandarani2025} applied this idea to gravity by including a tiny universal coupling between the clock and system energies. The total constraint can be written (in simplified form) as
        \begin{equation}\label{Jconstraint}
        J \;=\; p_t^{\rm (clock)} + H_{\rm sys} \;+\; \frac{1}{2\Lambda}\Big(p_t^{\rm (clock)} + H_{\rm sys}\Big)^2 \;\approx\; 0~,
        \end{equation}
        where $p_t$ is the clock’s momentum conjugate to an internal time reading and $H_{\rm sys}$ is the Hamiltonian of the rest of the system.  The quadratic ``deformation’’ term (with some large energy scale $\Lambda$) introduces an interaction between the clock and system. When one works in the clock’s reference frame (conditioning on a clock reading), this leads to an \emph{effective Hamiltonian} for the system of the form
        \begin{equation}\label{Heff}
        H_{\rm eff}^{\rm (sys)} \;\approx\; H_{\rm sys} \;+\; \frac{1}{\Lambda}\,H_{\rm sys}\,H_{\rm clock} \;+\; \mathcal{O}(1/\Lambda^2)~,
        \end{equation}
        assuming $H/\Lambda \ll 1$ (the low-energy, weak-coupling regime):contentReference[oaicite:0]{index=0}:contentReference[oaicite:1]{index=1}. The extra term is \emph{universal}---it couples the energy of the clock to the energy of the system, independent of the composition of either. Physically, this can be interpreted as a gravitational potential energy: it says that if the clock has energy $E_{\rm clock}$ and the system has energy $E_{\rm sys}$, their joint energy is lowered by an amount $\sim -E_{\rm clock}E_{\rm sys}/\Lambda$. In the limit that both energies are dominated by rest mass ($E \approx mc^2$), this term becomes $-\frac{1}{\Lambda} m_{\rm clock} m_{\rm sys} c^4$. This resembles the Newtonian gravitational potential energy $-G m_{\rm clock} m_{\rm sys}/r$, except for two crucial points: (a) it does not yet have a $1/r$ dependence, and (b) it involves the parameter $\Lambda$, which has dimensions of energy.

        In the relational framework, distance did not explicitly appear in the derivation of Eq.~\eqref{Heff}; effectively, the model in Refs.~\cite{Iskandarani2025,SinghFriedrich2025} considered the clock and system as a composite with some unspecified separation. To recover the familiar Newtonian form, one must identify how $\Lambda$ relates to spatial separation. In a truly local theory, any interaction between two separated systems must be mediated by a field or carrier, and its strength will generally decay with distance. In a 3+1-dimensional Lorentz-invariant setting, a static point source for a massless mediating field produces a potential that decays as $1/r$. This is a well-known result from solving the d'Alembertian equation (the wave equation) for a static source, and it was in fact the basis of the first relativistic scalar theory of gravity proposed by Nordström in 1912:contentReference[oaicite:2]{index=2}:contentReference[oaicite:3]{index=3}. We can incorporate this insight as follows: \emph{require} that the effective coupling $\Lambda^{-1}$ in Eq.~\eqref{Heff} not be a constant of nature, but rather arise from a field propagating between the two masses. In other words, $\Lambda$ should be interpreted as $\Lambda(r) \propto r$ at large separation, so that $1/\Lambda \propto 1/r$. Under this interpretation, the interaction term becomes
        \begin{equation}\label{V_Newton}
        V_{\rm grav} \;\approx\; -\,\frac{1}{\Lambda(r)}\,E_{\rm clock}\,E_{\rm sys} \;\approx\; -\,\frac{c^4\,m_{\rm clock}\,m_{\rm sys}}{\Lambda_0\,r}~,
        \end{equation}
        for some constant $\Lambda_0$ with units of energy$\times$length. Matching to Newton’s law in the limit of weak fields and low velocities fixes $\Lambda_0 = c^4/G$ (where $G$ is Newton’s constant). We then have
        \[ V_{\rm grav} \;=\; -\,\frac{G\,m_{\rm clock}\,m_{\rm sys}}{r}~, \]
        recovering the $1/r$ potential and hence a $1/r^2$ force. In short, by embedding the relational-clock model into an operational Lorentzian spacetime, one is naturally led to a field mediation picture in which the universal energy coupling produces an inverse-square law at large distances. The key ingredients are the universality of the coupling (which ensures equivalence of gravitational response) and the propagation in $3+1$ dimensions (which ensures the $1/r^2$ flux dilution).

        While this argument is heuristic, it is supported by the explicit quantum-clock analysis in Ref.~\cite{SinghFriedrich2025}, where it was shown that introducing an interaction of the form of Eq.~\eqref{Jconstraint} yields both gravitational time dilation and an attractive $1/r$ Newtonian potential between two masses at leading order:contentReference[oaicite:4]{index=4}:contentReference[oaicite:5]{index=5}. Singh and Friedrich did not derive the spatial $1/r$ dependence from scratch---it was essentially assumed by considering the correspondence with the Schwarzschild metric in the weak-field limit:contentReference[oaicite:6]{index=6}. What we have argued here is that \emph{no new physics is needed to get that $1/r$ profile}: it is mandated by causality and dimensionality. If gravity emerges from an underlying relational interaction, as soon as that interaction is localized in a relativistic field, the Green’s function of the field in flat space will supply the $1/r$ fall-off. In the next section, we substantiate this reasoning by writing down the relevant field equation and solving it.

    \section{Scalar Foliation Field as the Gravitational Potential}
        We now turn to a field-theoretic derivation of the inverse-square law. The simplest continuum description consistent with the above discussion is a scalar field $\Phi(x)$ that represents the deviation of local time flow from some reference (e.g. one might think of $\Phi$ as the potential that parametrizes gravitational time dilation). In SST terms, this field can be identified with the \textit{swirl-clock field} or \textit{foliation mode} of the fluid: it is essentially a scalar field whose level sets define an ``absolute time’’ foliation of spacetime (physically, one can imagine each hypersurface of constant $\Phi(x)$ as an instantiation of the incompressible medium’s global state of phase or temperature). Because SST preserves Lorentz symmetry at the fundamental level, we assume $\Phi$ propagates via the standard wave operator in Minkowski space. We also assume that mass-energy acts as a source for $\Phi$, in analogy to how mass density sources Newton’s potential in Poisson’s equation or the stress-energy tensor sources the metric in GR. The most general action to quadratic order in $\Phi$ consistent with these requirements is
        \begin{equation}\label{Phi_action}
        S[\Phi] \;=\; \int d^4x \,\Big[\,\frac{1}{2}(\partial_\mu \Phi\,\partial^\mu \Phi) \;+\; \alpha\,\Phi\,\rho_m(x)\,\Big]~,
        \end{equation}
        where $\rho_m(x)$ is the rest-mass density of matter (treated here as a given source distribution) and $\alpha$ is a coupling constant. For clarity, we have not written any self-interaction or higher-order terms in $\Phi$; we consider only the leading, weak-field behavior. Varying the action yields the field equation
        \begin{equation}\label{Phi_eom}
        \partial_\mu \partial^\mu \Phi(x) \;=\; -\,\alpha\,\rho_m(x)~,
        \end{equation}
        which in Minkowski spacetime is just
        \[ \square\,\Phi \;=\; \frac{\partial^2 \Phi}{\partial t^2} \;-\; \nabla^2 \Phi \;=\; -\,\alpha\,\rho_m~. \]
        We are interested in static, weak gravitational fields, so we look for time-independent solutions and neglect any radiation. Setting $\partial_t \Phi = 0$, Eq.~\eqref{Phi_eom} reduces to the Poisson-like equation
        \begin{equation}\label{Poisson}
        \nabla^2 \Phi(\mathbf{x}) \;=\; -\,\alpha\,\rho_m(\mathbf{x})~,
        \end{equation}
        which is formally identical to Newton’s gravitational potential equation (with $\Phi$ playing the role of the Newtonian potential $\chi$ and $\alpha$ related to $4\pi G$ in the usual units). For a point mass $M$ at the origin, $\rho_m(\mathbf{x}) = M\,\delta^3(\mathbf{x})$. Solving Eq.~\eqref{Poisson} in free space then gives
        \begin{equation}\label{monopole_solution}
        \Phi(r) \;=\; -\,\frac{\alpha M}{4\pi\,r}~, \qquad r = |\mathbf{x}|~,
        \end{equation}
        outside the source (and a constant inside, assuming a point or a radius much smaller than the range of interest). The gradient of $\Phi$ is
        \[ \nabla \Phi(r) = -\,\frac{\alpha M}{4\pi}\,\frac{\hat{\mathbf{r}}}{r^2}~, \]
        which indeed has the $1/r^2$ form. If we identify the physical gravitational potential as $U = \Phi c^2$ (so that $-\nabla U$ gives the acceleration of a test particle), then comparing $\nabla U$ to Newton’s law $-GM\,\hat{\mathbf{r}}/r^2$ fixes $\alpha$ to $\alpha = 4\pi G/c^2$. In units where $c=1$, this simply becomes $\alpha = 4\pi G$. For our purposes we can take $\alpha$ as a known coupling constant ensuring the correct normalization of the field.

        The result \eqref{monopole_solution} shows explicitly that the foliation/scalar-clock field $\Phi$ sourced by a static mass will exhibit the familiar $1/r$ profile. In other words, an \emph{emergent} gravitational potential is generated, with the correct distance dependence. We emphasize that nothing in this derivation required introducing spacetime curvature or a departure from special relativity: the inverse-square law arises purely from the solution of a field equation in a flat background (the same reasoning that underlies Nordström’s early scalar theory~\cite{Deruelle2011}). The role of curvature in GR is here played by the scalar field $\Phi$, which encapsulates the way mass-energy distorts the ``chronogeometry’’ of the SST medium (i.e. it alters the relative flow of time). In a regime of weak fields, one can think of $\Phi$ as proportional to the fractional slowdown of clock rates relative to infinity: for example, if $\Phi(r) = -\frac{GM}{4\pi r}$ (with $c=1$), then one finds that the ticking rate of a clock at radius $r$ is dilated by a factor $\sim 1 + \Phi(r)$ compared to a clock far away. This matches the gravitational redshift to first order (since $1 + \Phi(r) \approx \sqrt{1-\frac{2GM}{r}}$ for small $GM/r$). Thus, the foliation field $\Phi$ is directly playing the role of the Newtonian gravitational potential in the weak-field limit. By identifying $\Phi$ with the SST ``swirl-clock’’ field, we assert that what we usually attribute to a geometric potential is actually carried by a physical field in the fluid---one which transmits influences at finite speed and whose dynamics enforce the $1/r^2$ law.

        Before proceeding, we note an important point: in the above derivation, we assumed a static source and ignored any back-reaction of the field’s energy on the source. This is consistent with treating gravity as weak (test particles or small perturbations on Minkowski space). In a more complete theory, one could allow $\Phi$ to carry energy and momentum which in turn act as source (this is analogous to including the field’s self-gravity, which Nordström’s second theory and GR do). In SST, one expects that $\Phi$ is just one mode of the medium and that energy in this mode might feed back into other modes or require a higher-order treatment. However, for the purposes of recovering Newton’s law, the linear theory is sufficient. Our interest is to identify the carrier of far-field momentum flux---and to that end, keeping $\Phi$ as a free field whose source is the mass distribution is a consistent first approximation.

        The carrier of gravitational momentum in this emergent picture is clearly the $\Phi$ field itself. Unlike in Newtonian gravity (where the potential is an auxiliary field without its own energy), here $\Phi$ is a dynamical entity in Minkowski space, so it will possess a stress-energy tensor. We can derive the stress-energy by varying the action \eqref{Phi_action} with respect to the metric, or simply use the canonical form:
        \begin{equation}\label{Tmn}
        T_{\mu\nu}^{(\Phi)} \;=\; \partial_\mu \Phi\,\partial_\nu \Phi \;-\; \frac{1}{2}\,\eta_{\mu\nu}\,\partial_\alpha \Phi\,\partial^\alpha \Phi~.
        \end{equation}
        This symmetric tensor describes the energy density and momentum flow carried by the foliation field. For example, $T_{00} = \frac{1}{2}\big[(\partial_t \Phi)^2 + (\nabla \Phi)^2\big]$ is the energy density (sum of kinetic and gradient energy) and $T_{0i} = -\,\partial_t \Phi\,\partial_i \Phi$ is the energy flux (or momentum density) in the $i$th direction. In a static configuration, $\partial_t \Phi = 0$, so the energy density simplifies to $T_{00} = \frac{1}{2}(\nabla \Phi)^2$. There is no energy flux ($T_{0i}=0$) in the static case, but there can be momentum flow in the sense of \emph{pressure} or stress. The spatial components $T_{ij}$ represent pressure/tension in the field. For instance, $T_{rr}$ (in spherical coordinates) will act like a radial pressure in the $\Phi$ field configuration.

        In our monopole solution \eqref{monopole_solution}, $\Phi(\mathbf{x})$ depends only on $r$. Using Eq.~\eqref{Tmn}, we find for $r > 0$:
        \begin{gather}
            T_{00}^{(\Phi)}(r) \;=\; \frac{1}{2}\,(\Phi')^2 ~=~ \frac{\alpha^2 M^2}{32\pi^2}\,\frac{1}{r^4}~, \label{T00}\\[1ex]
            T_{rr}^{(\Phi)}(r) \;=\; (\Phi')^2 - \frac{1}{2}(\nabla \Phi)^2 ~=~ \frac{1}{2}\,(\Phi')^2 ~=~ \frac{\alpha^2 M^2}{32\pi^2}\,\frac{1}{r^4}~. \label{Trr}
        \end{gather}
        Interestingly, for this scalar field, the radial pressure $T_{rr}$ equals the energy density $T_{00}$ at every radius (a reflection of the fact that the field stress is ``stiff,'' with a stress tensor similar to a cosmological scalar field or radial string). Both fall off as $1/r^4$ for large $r$. The $1/r^4$ behavior is the familiar $ \propto E^2/r^4$ fall-off of field energy density for any $1/r$ potential (comparable to the electric field of a point charge in electrostatics). While the $1/r^4$ is a steeper decay than the force $1/r^2$, it encodes the fact that energy is concentrated nearer the source even as the influence extends far out. The \emph{momentum flux} of interest, however, is not simply $T_{rr}$ or $T_{00}$. For understanding the inverse-square law, we look at how the field carries momentum outward from the source.

        In a steady state, any net momentum flow through a spherical surface around the source must be conserved (any momentum leaving one region enters another). The conservation law $\partial^\mu T_{\mu\nu}=0$ implies, for the static case, $\nabla\cdot \mathbf{T}^{(p)} = 0$, where $\mathbf{T}^{(p)}$ is the momentum flux density tensor (the spatial part of $T_{\mu\nu}$). In particular, $\partial_r T_{rr} + \frac{2}{r}T_{rr} = 0$ outside the source (in spherical symmetry). This yields
        \begin{equation}\label{Gauss_T}
        \frac{d}{dr}\big(r^2 T_{rr}(r)\big) = 0 ~\implies~ r^2 T_{rr}(r) = \text{constant for } r>0~.
        \end{equation}
        Thus, $T_{rr} \propto 1/r^2$ in the vacuum region. Plugging in our solution’s expression \eqref{Trr}, we see
        \[ r^2 T_{rr}(r) = \frac{\alpha^2 M^2}{32\pi^2}\,\frac{1}{r^2} \times r^2 = \frac{\alpha^2 M^2}{32\pi^2} = \text{constant}, \]
        as required (the equality holds for all $r$ outside the delta-function source, not just asymptotically). The constant can be related to the source parameters; in fact, using $\alpha = 4\pi G$ and restoring $c$, the constant becomes $\frac{(4\pi G M)^2}{32\pi^2} = \frac{G^2 M^2}{2\pi}$ in geometrized units. The exact number is not important for the argument; what matters is that \emph{the field’s momentum flux through an area $4\pi r^2$ is independent of $r$}. In other words, the field carries momentum (or force) outward in such a way that it is conserved over spherical shells. This is precisely Gauss’s law behavior. If one defines an effective ``gravitational flux’’ as $\mathcal{F}(r) = r^2 T_{rr}(r)$, then $\mathcal{F}(r)=$ constant $\,\propto M^2$ in our scalar theory. Taking a single factor of $M$ to associate with the source and one with a unit test mass, this suggests that the flux per test mass is $\propto M$---recovering the linear dependence on enclosed mass. More rigorously, one can integrate the $0$--$r$ component of $\partial^\mu T_{\mu\nu}=0$ over a volume to see that the force on a test mass at radius $r$ equals the momentum flux through a sphere of radius $r$. Because $T_{rr}\propto 1/r^2$, that force comes out proportional to $M/r^2$. We thus see that the stress-energy perspective and the field equation perspective are completely consistent: both insist that $1/r^2$ is the only viable radial dependence for a mass-generated flux in three dimensions:contentReference[oaicite:7]{index=7}.

        It is worth highlighting a conceptual point here. In the usual Newtonian picture, one often invokes a ``field line’’ or flux analogy: the gravitational field $\mathbf{g} = -\nabla \chi$ has flux $4\pi G M$ through any sphere around mass $M$, hence $\mathbf{g}$ falls off as $1/r^2$. In our case, $\Phi$ is the analog of $\chi$ and indeed $\nabla \Phi$ corresponds to the gravitational field. The $\Phi$ field equation \eqref{Poisson} implies $\oint \nabla \Phi \cdot d\mathbf{S} = -\alpha M$ (Gauss’s law), which directly yields $|\nabla \Phi| = \frac{\alpha M}{4\pi r^2}$. This is another way to derive $\Phi(r) \propto 1/r$. What the stress-energy analysis adds is the identification of \emph{which quantity} is flowing outward and being conserved: in this case it is the field’s momentum (or pressure). The field carries momentum flux which, in static equilibrium, appears as a tension pulling inward on the mass (and on any test mass):contentReference[oaicite:8]{index=8}. This field tension is what we intuitively recognize as the gravitational pull. Because the tension lines (field lines) spread over area $\propto r^2$, the strength per unit area diminishes as $1/r^2$. Our derivation above made that quantitative by showing $T_{rr} \propto 1/r^2$. It is notable that all of this happens in flat spacetime: energy and momentum are locally conserved in the usual sense, and the ``gravity’’ we feel is literally the momentum flow in the $\Phi$ field mediating between masses.

        Finally, we address the question posed in the introduction: what happens if one replaces the historical Newtonian potential $\chi$ by the SST foliation (swirl-clock) field? The answer is that, in the weak-field, static monopole sector, there is no difference at all. Equation~\eqref{Poisson} is identical to Newton’s equation for $\chi$, so the solutions and flux conservation properties are identical. In essence, the derivation we have given in this section could be viewed as nothing more than a re-derivation of Newton’s $1/r^2$ law using new names for the quantities. This is exactly the point: the SST field \emph{obeys the same equations} in this regime, guaranteeing that the usual $1/r^2$ behavior follows automatically. The advantage of the emergent approach is not in altering the low-level empirical law, but in providing it with a microphysical interpretation. Rather than being an arbitrary classical field, $\Phi$ is in principle tied to degrees of freedom of an underlying medium (with specific parameters such as density and circulation quantum). Rather than requiring an explicit ``action at a distance’’ or curved geometry, $\Phi$ transmits forces through local dynamics and stresses. The inverse-square law, from this viewpoint, is a necessary outcome given a massless mediator in three dimensions---any theory that has these features will respect it, whether geometric or not.

    \section{Discussion}
        We have shown through three approaches that an inverse-square force law emerges naturally from a broad class of first-principles scenarios that treat gravity as an emergent interaction on a Lorentzian background. The relational coupling approach (Section~2) illustrated in general terms how coupling energy to energy produces a composition-blind potential (hence satisfying the Equivalence Principle) and how enforcing locality forces that potential to adopt a $1/r$ form. The scalar field approach (Section~3) made this concrete by introducing a foliation scalar (swirl-clock field) whose equation of motion is analogous to Poisson’s equation. Solving it reproduced the Newtonian potential and identified the field as the carrier of gravitational influence. The stress-energy analysis (Section~4) reinforced the $1/r^2$ result by demonstrating a Gauss-law conservation of momentum flux in the field: any spherically symmetric mediator in flat 3D space will have $r^2 T^{r}{}_{r} = \text{constant}$ outside sources, which implies the field gradient falls as $1/r^2$. All three perspectives are consistent with one another and with standard results in the appropriate limit.

        An important outcome of this study is the clear identification of the SST field responsible for gravity in the weak-field regime. In the SST framework, candidates for gravitational carriers include the \emph{swirl-clock field} (a scalar related to local temporal rate) and possibly vector or tensor excitations of the fluid. Our derivations point to a scalar as the leading carrier in the static, monopole case. In other words, what we traditionally think of as the Newtonian gravitational field is, in SST, predominantly the scalar foliation mode. This is somewhat reminiscent of scalar-tensor theories or the so-called ``Einstein–aether’’ models in which a timelike unit vector field or a preferred foliation field picks out a universal time and mimics gravitational effects in the Newtonian limit. Here, the foliation scalar $\Phi(x)$ plays that role. It is this field that transports momentum through the vacuum, stored as fluid stress (pressure/tension), and it is this field that accumulates a $1/r^2$ flux around a mass. The vector and tensor degrees of freedom of the fluid (if any) do not contribute to the monopole $1/r^2$ force at the order considered; their effects would show up in e.g. frame-dragging or gravitational radiation scenarios, which are beyond our present scope.

        Another implication is that the Equivalence Principle is deeply connected to the dynamics of $\Phi$. Because $\Phi$ couples to $\rho_m$ (mass energy) universally, test particles feel the same $\Phi$ field regardless of their composition, and will thus fall with the same acceleration (assuming no other forces). In Section~2 we saw that the universal energy coupling in the relational picture was the seed of this universality:contentReference[oaicite:9]{index=9}. Section~3 translated that into the statement that $\Phi$ satisfies an equation with source $\rho_m$. Taken together, these mean that in SST the universality of free-fall comes from the fact that $\Phi$ couples to the energy content of matter and nothing else. (If there were multiple mediators coupling differently to different types of energy, or if $\Phi$ had self-interaction terms that violated linear superposition, equivalence could be broken. But at the weak-field linear level, none of those complications arise.) In short, the emergent gravity picture here is fully compatible with the Equivalence Principle, not by imposing it as a postulate but as a natural consequence of the single-field mediation and relational coupling mechanism. This was already argued in the previous SST paper on the emergent EP~\cite{Iskandarani2025}; our present work provides additional support by showing that the same assumptions that yield EP also yield the correct distance dependence.

        It is instructive to contrast this emergent approach with GR. In GR, the $1/r^2$ law for weak fields is obtained by linearizing Einstein’s equations: $h_{00}$ (the perturbation of the metric time-time component) satisfies $\nabla^2 h_{00} = -8\pi G \rho$ in harmonic gauge, leading to $h_{00} \sim -2GM/r$. The physical interpretation is that spacetime curvature (specifically the Newtonian potential part of the metric) carries the gravitational interaction. Here, we had $\nabla^2 \Phi = -4\pi G \rho$ (in $c=1$ units), yielding $\Phi \sim -GM/r$, with $\Phi$ playing the role of $h_{00}/2$. The key difference is that $\Phi$ lives in a flat spacetime---it is a genuine field in a medium, not part of the spacetime metric. Nonetheless, mathematically the solutions are the same in form. This is a vivid realization of the so-called ``geometric trinity’’ of gravity: one can formulate gravity in curvature-based, teleparallel (torsion-based), or scalar-flat frameworks that reproduce the same field equations:contentReference[oaicite:10]{index=10}:contentReference[oaicite:11]{index=11}. Nordström’s theory is an example of a flat-spacetime scalar theory that respects many of the same principles as GR, though it ultimately fails some empirical tests:contentReference[oaicite:12]{index=12}. In a sense, SST can be thought of as a modern Nordström-like theory, enhanced by an underpinning physical medium that provides additional structure (and presumably evades Nordström’s failures by including other fields corresponding to post-Newtonian corrections and light bending effects---perhaps via the fluid’s vector and tensor excitations). The scalar foliation field considered here is likely only one component of gravity in SST, but it is the dominant one for static weak fields, which is why focusing on it is sufficient to recover Newton’s law.

        We have so far restricted attention to static, weak fields. It is natural to ask how robust the inverse-square result is beyond these approximations. In classical GR, the $1/r^2$ law can be modified by higher-order (post-Newtonian) corrections, by strong-field deviations (e.g. near black holes), or by additional long-range fields (such as a cosmological constant yielding an $r$-independent field at very large scales). Similarly, in alternative theories one can have Yukawa corrections ($1/r^2$ times an exponential factor from a finite-range scalar) or other power-law deviations if there are extra dimensions, etc. In our emergent scenario, if the mediating field $\Phi$ had a nonzero mass or self-interaction that screened it, then $1/r^2$ could turn into (for example) a Yukawa $\frac{e^{-r/\lambda}}{r^2}$ behavior. However, within SST it is expected that the swirl-clock/foliation mode is effectively massless over astrophysical scales (there is no evidence of a Yukawa fall-off in gravity down to sub-millimeter scales in experiments). One might also worry that nonlinearities in the field equation (if we extended it beyond quadratic order) could alter the far-field behavior. In a massless scalar in 3+1 dimensions, the $1/r$ profile of a monopole source is stable to nonlinear corrections in the sense that no other fall-off can satisfy conservation and symmetry---what nonlinearity can do is change the effective source charge or introduce angular dependence (multipoles) or time dependence, but it cannot change the asymptotic $1/r$ of the monopole without adding new degrees of freedom. Therefore, we expect the inverse-square law to remain essentially an automatic feature of the SST gravitational field, as long as Lorentz symmetry and three large spatial dimensions hold. This is reassuring: it means the successes of Newton’s law do not get lost in the transition to an emergent fluid-based picture. Instead, they are recast as consequences of deeper symmetries and dynamics.

        One potential benefit of the emergent approach is that it allows us to consider gravity from the perspective of standard field and fluid dynamics, opening the door to new intuition. For example, one can ask: if gravity is just a consequence of a fluid’s perturbation (the $\Phi$ field), could there be situations where this perturbation fails to behave as expected (analogous to fluid turbulence or shock formation)? Also, could the coupling of $\Phi$ to matter lead to energy exchange or dissipation in extreme regimes (e.g. at very high frequencies or in coherent quantum states)? In the standard geometric view, gravity is conservative and lossless (other than through gravitational radiation, which is typically quadrupolar). In a fluid emergent view, one might discover new dissipative channels or limits of validity for the effective field description. These are speculative questions, but they illustrate how shifting perspective can generate new avenues for exploration. The derivations here lay a foundation for consistently moving to those more complex queries by ensuring the lowest-order behavior is correct.

        It is also worth noting that our identification of the swirl-clock field with the Newtonian potential could help in bridging to cosmology or to regimes where multiple SST fields interact. In cosmology, for instance, a foliation scalar could play the role of a cosmological “clock” field that drives expansion or defines cosmic time. If the same field is responsible for local gravitational potentials, that hints at a unification of cosmic and local time scales (perhaps offering insight into Machian ideas or the origin of inertial frames, since in SST inertial structure ultimately comes from the fluid rest frame). While we have not touched cosmology here, one can imagine extending Eq.~\eqref{Phi_eom} to include cosmic expansion or background density of the medium. The robust $1/r^2$ outcome would then need to be seen as a local limit of a more general solution (just as Newton’s law is the local approximation to cosmological gravity). Having an explicit field in flat spacetime could simplify certain questions—for example, the scalar field could, in principle, be quantized or have fluctuations, whereas quantizing the metric in GR is notoriously difficult. The emergent approach may thus be more amenable to meshing with quantum theory, at least for the scalar portion of gravity.

    \section{Conclusion}
        We have provided a detailed derivation of how a $1/r^2$ force law arises from SST-compatible dynamics, without assuming it from the outset. By treating gravity as mediated by a hydrodynamic scalar field (the swirl-clock or foliation mode) on a Lorentzian background, we recover Newton’s inverse-square attraction in the appropriate limit. Three independent derivations—relational, field-theoretic, and flux-conservation—reinforce the result and illuminate it from different angles. The fact that all agree with the empirically known law lends strong support to the SST program’s consistency with classical gravity at large scales. In particular, we identified the scalar mediator of SST as the analog of the Newtonian potential and showed it naturally produces Gauss’s law for flux. Replacing the abstract Newtonian potential with this physically interpretable field does not change any monopole phenomenology, but it does change the conceptual picture: gravitational influence is carried by momentum flow in a real medium, rather than by geometry alone.

        For a mainstream physics audience, these results demonstrate that emergent gravity models can be made to look completely conventional in their predictions, even if their starting principles are quite novel. One need not introduce exotic modifications to get an inverse-square law—rather, one must show that any viable alternative theory contains the inverse-square as a limiting case. We have done so for SST. Moreover, our analysis reveals that the $1/r^2$ behavior is an \emph{automatic byproduct} of having a long-range relativistic field in 3+1 dimensions. This can be taken as a reminder that certain features of gravity (like Gauss’s law) are much more general than the specifics of GR; any theory that does not reproduce them would have to violate some basic symmetry or dimensional argument. Conversely, it suggests that testing the inverse-square law to extreme precision (as experiments have done down to sub-millimeter scales) is not likely to refute an emergent theory that respects those basic principles, but rather to constrain exotic short-distance effects (such as hypothetical massive mediators or extra dimensions). SST, by embedding gravity in a physical medium, also invites new experimental ideas—perhaps along the lines of detecting any proposed low-energy deviations or couplings to other fields.

        In summary, this work strengthens the case that SST and similar hydrodynamic models can stand shoulder-to-shoulder with General Relativity in the Newtonian regime, while offering a divergent interpretation of what gravity \emph{is}. The inverse-square law, often seen as a hallmark of classical gravity, here finds its home in the behavior of a fluid-based clock field. This not only reproduces known physics but also provides a richer language (in terms of flux, momentum, and material response) to describe gravitational phenomena. Future work will need to address how this approach deals with more complex scenarios (orbital dynamics, radiation, strong gravity, etc.) and how it might be distinguished from standard gravity if at all. For now, we conclude that the emergent equivalence principle program can indeed be extended to an emergent inverse-square law, completing a crucial piece of the puzzle in the quest for a comprehensive SST description of gravitation.

        \appendix
    \section*{Appendix: Connections to Prior SST Results}
        \addcontentsline{toc}{section}{Appendix: Connections to Prior SST Results}

        It is helpful to situate the above derivations in the broader context of Swirl-String Theory developments. Three particular threads in previous SST work provide support or constraints for the present theory: (A) the suppression of Kelvin modes in matter vortices, (B) the thermodynamic formulation of SST (``Thermo-SST’’) and its parameter values, and (C) the variational origin of particle magnetic moments. We briefly summarize each and indicate how they reinforce our assumptions or results.

        \paragraph{(A) Kelvin-Mode Suppression and Stability of $1/r^2$ Gravity.} In SST, particles like electrons are modeled as knotted vortex filaments in the fluid medium. These filaments can support internal excitations known as Kelvin waves or modes (analogous to vibrational or twist excitations of a vortex line). A concern might be that such internal degrees of freedom could modify the external fields produced by the particle—e.g. a vibrating or excited vortex could perhaps alter the $1/r^2$ gravitational or electromagnetic field in its vicinity. Prior work by Iskandarani~\cite{Iskandarani2024Kelvin} addresses this issue for atomic orbitals: it was shown that Kelvin-mode excitations of electron vortices have a high quantization gap (on the order of $10^2$–$10^3$~eV), meaning they are energetically inaccessible under ordinary conditions. Essentially, the vortex filament is extremely stiff to internal perturbations at low energy scales, ensuring that the electron’s equilibrium structure (and its external fields) are stable. This result implies that the presence of internal vortex degrees of freedom does not spoil the clean $1/r^2$ behavior of long-range fields: any Kelvin oscillations that could distort the $1/r$ potential are frozen out in typical scenarios. Thus, our assumption of a static, well-behaved source in Sections 3–4 is justified; an electron or nucleon will not spontaneously wiggle in a way that changes its far-field $\Phi$ profile, except possibly in extreme accelerations or high excitations. The Kelvin-mode suppression therefore supports the \emph{rigidity} of sources required for classical gravity and ensures that low-energy atomic or bulk matter can be treated as effectively static mass distributions from the gravitational field’s perspective:contentReference[oaicite:13]{index=13}:contentReference[oaicite:14]{index=14}. In summary, one crucial consistency check is passed: internal dynamics of SST matter do not destabilize the emergent inverse-square law in the regime of interest.

        \paragraph{(B) Thermo-SST and the Swirl Coulomb Constant $\Lambda$.} The thermodynamic reformulation of SST (``Thermo-SST’’) presented by Iskandarani~\cite{Iskandarani2025Thermo} provides a macroscopic understanding of the medium and introduces key constants. Of particular relevance is the \emph{swirl-Coulomb constant} $\Lambda$, which emerges as the strength of the effective $1/r$ potential generated by a unit source. In Thermo-SST, $\Lambda$ is matched from fundamental core properties with dimensional consistency
($[\Lambda]=\mathrm{J\cdot m}$):
\[
\Lambda = 4\pi\,\rho_{\text{core}}\,\lVert \mathbf{v}_{\!\boldsymbol{\circlearrowleft}}\rVert^{2}\,r_c^{4}.
\]
Here $\rho_{\text{core}}$ is the core density used in invariant mass kernels, and $\lVert \mathbf{v}_{\!\boldsymbol{\circlearrowleft}}\rVert$ is the canonical swirl speed scale. Plugging in the calibrated SST values (such as $\rho_f \approx 7\times10^{-7}$~kg/m$^3$, $r_c \approx 1.4\times10^{-15}$~m, etc.), one finds $\Lambda \sim 10^{-45}$~J·m:contentReference[oaicite:16]{index=16}. This is on the correct order of magnitude to produce gravitational interactions of observed strength:contentReference[oaicite:17]{index=17}. In our relational derivation (Section~2), we in effect treated $\Lambda$ as an unknown constant to be related to $G$ by matching. The Thermo-SST result shows that $\Lambda$ is \emph{not} an arbitrary parameter; it is fixed by the fluid’s microphysical constants and indeed comes out to the tiny value needed to make gravity so feeble. In other words, SST offers an explanation for why $G$ is so small: it is a derived consequence of $\rho_f$, $r_c$, and the circulation quantum of the fluid. Furthermore, Thermo-SST identifies the \emph{Chronos-Kelvin Invariant}, essentially a statement that links the fluid’s rotation (Kelvin circulation) to clock rate differences:contentReference[oaicite:18]{index=18}:contentReference[oaicite:19]{index=19}. This supports our use of a swirl-clock field $\Phi$ to encode gravitational time dilation; it wasn’t a freely invented concept but one rooted in an invariant of the theory. Altogether, the thermodynamic perspective ensures that our effective field $\Phi$ and coupling $\alpha$ (or $\Lambda$) are grounded in the physics of the SST medium, lending credence to treating $\Phi$ as the cause of the Newtonian potential. It also means any modifications to $1/r^2$ behavior (say at very short distances) would tie back to deviations in the medium’s equation of state or a breakdown of the continuum approximation at scales near $r_c$. As of now, no such deviations have been observed down to $\sim 10^{-5}$~m, so the continuum inverse-square law appears solid.

        \paragraph{(C) Variational Origin of the Electron Magnetic Moment.} Another piece of the SST puzzle comes from showing that intrinsic particle properties can arise as outcomes of an extremization principle, rather than being put in by hand. In a recent paper~\cite{Iskandarani2023Magnetic}, Iskandarani demonstrated that the electron’s gyromagnetic ratio ($g$-factor) can be understood by treating the electron vortex as an extended object that finds a stable state (minimum energy) for a certain current distribution. By allowing the electron’s self-field to adjust the effective moment, one finds that only a particular configuration is stable, corresponding to the observed $g\approx 2$. The significance for our purposes is twofold: first, it showcases how classical electromagnetic effects (usually presumed to require quantum field theory to derive) can emerge from a hydrodynamic variational principle in SST. This builds confidence that SST’s methodology—using a constrained extremum of a fluid energy functional—can yield quantitatively correct results in the appropriate domain. We analogously treated the gravitational field $\Phi$ as emerging from minimizing an action [Eq.~\eqref{Phi_action}], which is a variational approach. The success in the magnetic moment case suggests the approach is sound. Second, it emphasizes the role of \emph{dynamical selection} of unique configurations by the theory:contentReference[oaicite:20]{index=20}:contentReference[oaicite:21]{index=21}. In gravity’s context, the $1/r^2$ field is likewise a result of a dynamical equation picking out that solution (subject to boundary conditions at infinity). There is no freedom to choose a different power-law once the equations are set—it is an extremum (technically a harmonic function solution) singled out by the physical setup. The magnetic moment derivation serves as an analogy: just as the electron’s extended structure yields a unique stable $g$, the extended vacuum structure yields a unique $1/r$ potential form. Moreover, both results reinforce the idea that what appear to be fundamental constants or laws (Newton’s $G$, Bohr magneton, etc.) might actually be emergent quantities determined by deeper physics. In the magnetic case, the presence of self-interaction and finite size were crucial; in the gravity case, the presence of the fluid medium and its constants were crucial. Both works, therefore, underscore the SST theme that \emph{universality can emerge from dynamics}. We see the Equivalence Principle and inverse-square law in this light—as emergent universalities guaranteed by the structure of the theory, not needed to be axiomatized.

            In summary, prior SST papers laid important groundwork that both motivates and constrains the present development. The Kelvin mode gap assures us that matter sources remain “quiet” and classical on gravitational timescales. The Thermo-SST framework gives us the values and interpretation of the coupling constants in our field model, showing quantitative consistency with gravity. And the variational particle model illustrates how SST finds unique solutions that align with physical reality. Together, these results bolster the credibility of viewing gravity in SST as an emergent, effective phenomenon: they show that the necessary conditions for classical gravity (stability, correct coupling strength, uniqueness of solution) are all satisfied within the model’s other well-tested domains. As SST continues to mature, such cross-connections between different facets (atomic physics, cosmology, fundamental constants) will be essential for demonstrating that the theory is not only internally consistent but also externally valid across the many regimes of physics.

            \begin{thebibliography}{99}\setlength{\itemsep}{0ex}
            \bibitem{Iskandarani2025}
            O.~Iskandarani, \textit{Emergent Equivalence Principle from Relational Time and Connection Dynamics}, Foundations of Physics \textbf{55}, 94 (2025). % (in press).
% Assuming this is published or at least in Found Phys 55, article 94 for ex.

            \bibitem{SinghFriedrich2025}
            A.~Singh and O.~Friedrich, \textit{Emergence of Gravitational Potential and Time Dilation from Non-interacting Systems Coupled to a Global Quantum Clock}, Foundations of Physics \textbf{55}, 82 (2025).

            \bibitem{Deruelle2011}
            N.~Deruelle, \textit{Nordström’s scalar theory of gravity and the equivalence principle}, Gen. Relativ. Gravit. \textbf{43}, 3337 (2011).

            \bibitem{Iskandarani2024Kelvin}
            O.~Iskandarani, \textit{Kelvin Mode Suppression in Atomic Orbitals: A Vortex-Filament Gap}, preprint (2024). [arXiv: \textit{N/A}] % No arXiv given, assume it's a private preprint or minor pub.

            \bibitem{Iskandarani2025Thermo}
            O.~Iskandarani, \textit{Thermodynamic Formulation of Swirl-String Theory (Thermo-SST)}, preprint (2025). % Possibly an internal white paper or to appear.

            \bibitem{Iskandarani2023Magnetic}
            O.~Iskandarani, \textit{A Variational Origin of the Electron Magnetic Moment}, preprint (2023). % Possibly submitted or in proceedings.

            \end{thebibliography}

\end{document}