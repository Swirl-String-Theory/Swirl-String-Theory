%! Author = Omar Iskandarani
%! Date = 11/25/2025
%! Affiliation = Independent Researcher, Groningen, The Netherlands
%! License = © 2025 Omar Iskandarani. All rights reserved. This manuscript is made available for academic reading and citation only. No republication, redistribution, or derivative works are permitted without explicit written permission from the author. Contact: info@omariskandarani.com
%! ORCID = 0009-0006-1686-3961
%! DOI = 10.5281/zenodo.xxx

\newcommand{\paperdoi}{10.5281/zenodo.xxx}
\newcommand{\papertitle}{NEW-PAPER}

%=========================================
% % PREAMBLE, PACKAGES AND DOCUMENT CONFIGURATION
%=========================================
\documentclass[11pt]{article}
\usepackage{amsmath,amssymb,amsfonts,bm}
\usepackage{siunitx}
\usepackage[hidelinks]{hyperref}
\usepackage[a4paper,margin=1in]{geometry}
\usepackage[T1]{fontenc}
\usepackage[utf8]{inputenc}

% swirl arrows (context-aware)
\newcommand{\swirlarrow}{ \mathchoice{\mkern-2mu\scriptstyle\boldsymbol{\circlearrowleft}}{\mkern-2mu\scriptscriptstyle\boldsymbol{\circlearrowleft}}}
\newcommand{\vswirl}{\mathbf{v}_{\swirlarrow}}
\newcommand{\SwirlClock}{S_{(t)}^{\swirlarrow}}
\newcommand{\Fmaxswirl}{F^{\max}_{\mkern-1mu\scriptscriptstyle\boldsymbol{\circlearrowleft}}}
% swirl arrows Counter Clockwise
\newcommand{\swirlarrowcw}{ \mathchoice{\mkern-2mu\scriptstyle\boldsymbol{\circlearrowright}}{\mkern-2mu\scriptscriptstyle\boldsymbol{\circlearrowright}}}
\newcommand{\vswirlcw}{\mathbf{v}_{\swirlarrowcw}}
\newcommand{\SwirlClockcw}{S_{(t)}^{\swirlarrowcw}}
\newcommand{\Fmaxswirlcw}{F^{\max}_{\mkern-1mu\scriptscriptstyle\boldsymbol{\circlearrowright}}}

\newcommand{\Fmax}{\Fmaxswirl} % default maximal force (left swirl)
\newcommand{\FmaxEM}{F^{\max}_{\mathrm{EM}}}
\newcommand{\FmaxG}{F_{\mathrm{G}}^{\max}}               % G-like maximal force scale

\newcommand{\omegas}{\boldsymbol{\omega}_{\swirlarrow}}  % swirl vorticity
\newcommand{\Om}{\Omega_{\swirlarrow}}                   % swirl angular frequency profile

\newcommand{\vscore}{v_{\swirlarrow}}                    % shorthand: |v_swirl| at r=r_c
\newcommand{\vnorm}{\lVert \mathbf{v}_{\mkern-2mu\scriptscriptstyle\boldsymbol{\circlearrowleft}} \rVert}               % swirl speed magnitude
\newcommand{\Ce}{\vswirl}                                % canonical swirl-speed constant

\newcommand{\rhof}{\rho_{\!f}}                           % effective fluid density
\newcommand{\rhoE}{\rho_{\!E}}                           % swirl energy density
\newcommand{\rhom}{\rho_{\!m}}                           % mass-equivalent density
\newcommand{\rc}{r_c}                                    % string core radius (swirl string radius)

\newcommand{\Lam}{\Lambda}                               % Swirl Coulomb constant
\newcommand{\alpg}{\alpha_g}                             % gravitational fine-structure analogue

\newcommand{\titlepageOpen}{
    \begin{titlepage}
        \thispagestyle{empty}
        \centering
        \Large \bfseries \papertitle \par \vspace{1cm}
        {\Large \itshape \textbf{Omar Iskandarani}\textsuperscript{\textbf{*}} \par}
        \vspace{0.5cm}
        {\today \par}
        \vspace{0.5cm}
}

\newcommand{\titlepageClose}{
        \vfill \raggedright \null
        \begin{picture}(0,0)
            \put(0,-45){  % Shift 200pt left, 40pt down
                \begin{minipage}[b]{0.7\textwidth} \footnotesize
                    \renewcommand{\arraystretch}{1.0}
                    \noindent\rule{\textwidth}{0.4pt} \\[0.5em]
                    \textsuperscript{\textbf{*}} Independent Researcher, Groningen, The Netherlands \\
                    Email: \texttt{info@omariskandarani.com} \\
                    ORCID: \texttt{\href{https://orcid.org/0009-0006-1686-3961}{0009-0006-1686-3961}} \\
                    DOI: \href{https://doi.org/\paperdoi}{\paperdoi}
                \end{minipage}
            }
        \end{picture}
    \end{titlepage}
}
%=========================================
% Start Document - Title Page
%=========================================
\begin{document}
    \titlepageOpen
        \begin{abstract}

We present a unified field theory based on the hydrodynamics of an inviscid, incompressible, director-bearing fluid substrate. We demonstrate that Relativistic kinematic effects emerge naturally from circulation conservation: proper time dilation is derived as a function of fluid circulation $\Gamma$ \textsuperscript{1}\textsuperscript{1}\textsuperscript{1}\textsuperscript{1}, while effective mass arises from rotational kinetic energy density $\Delta \rho_{eff}$\textsuperscript{2}. We identify the photon not as a point particle, but as a propagating torsion wave (shear wave) within this substrate, predicting a vacuum Faraday effect \textsuperscript{3}\textsuperscript{3} recently supported by experimental data on optical magnetic torques (Assouline & Capua, 2025). Furthermore, we show that steady-swirl Euler pressure gradients yield a short-range $1/r^3$ force for filamentary $v_\theta\propto 1/r$,
while the long-range inverse-square law arises from a local mediator obeying a Poisson equation on $\mathbb{R}^3$, \textsuperscript{4}, and that the fundamental electron scales ($r_e, \omega_C, E_B$) satisfy a precise harmonic identity derived from this hydrodynamic coupling\textsuperscript{5}\textsuperscript{5}\textsuperscript{5}.









        \end{abstract}
    \titlepageClose
%=========================================
% Title Page End
%=========================================



\section{Introduction}

The quest to unify interactions suggests that the vacuum is not an empty void, but a structured medium. Building on the "Vortex Atom" hypothesis of Kelvin \textsuperscript{6}and modern superfluid vacuum analogies\textsuperscript{7}, we propose \textbf{Swirl String Theory (SST)}. In SST, physical laws are emergent properties of a background fluid with effective density $\rho_{\!f}$ and shear stiffness $\mathcal{K}$.

\section{Kinematics: Circulation as Local Time}

Standard Special Relativity (SR) posits time dilation as a function of relative velocity. We reformulate this in terms of fluid topology. For a clock co-moving with an inviscid fluid in rigid rotation, the tangential velocity $v$ is linked to the circulation $\Gamma = \oint \mathbf{v} \cdot d\mathbf{l}$\textsuperscript{8}.

Substituting $v = \Gamma / (2\pi r)$ into the Minkowski metric yields a circulation-based time dilation\textsuperscript{9}:
\begin{equation}
\frac{d\tau}{dt} = \sqrt{1 - \frac{\Gamma^2}{4\pi^2 r^2 c^2}}
\end{equation}

Because $\Gamma$ is conserved in inviscid, barotropic flows (Kelvin’s Theorem), time dilation becomes a topological invariant of the flow\textsuperscript{10}. This suggests that "time" is a measure of the local vorticity density of the vacuum substrate.

\section{The Origin of Mass: Rotational Energy Density}

Mass is not intrinsic but emergent. By analyzing a fluid in rigid-body rotation, we calculate the volume-averaged rotational kinetic energy density $\langle e_{kin} \rangle$\textsuperscript{11}. Applying the mass-energy equivalence $E=mc^2$, we derive an effective mass density relation\textsuperscript{12}:
\begin{equation}
\frac{\Delta \rho_{eff}}{\rho} = \frac{1}{4}\left(\frac{v_{edge}}{c}\right)^2
\end{equation}

This confirms that rotational motion contributes to the inertial mass of extended systems at order $(v/c)^2$\textsuperscript{13}. In SST, fundamental particles are modeled as knotted vortex loops\textsuperscript{14}. These loops possess finite energy, impulse, and an effective mass $M_{eff} = \lVert I \rVert / U$ defined by their hydrodynamic impulse $I$ and self-induced velocity $U$\textsuperscript{15}.

\section{Electromagnetism: Photons as Torsion Waves}

We reject the point-particle photon. Using Cartan’s structure equations in a teleparallel geometry, we identify the electromagnetic field tensor $F_{\mu\nu}$ as the projection of the substrate's torsion 2-form $T^a$\textsuperscript{16}. The photon emerges as a massless transverse shear wave of the substrate's director field\textsuperscript{17}.

The wave speed is determined by the substrate's effective stiffness and density\textsuperscript{18}:
\begin{equation}
c = \sqrt{\frac{\mathcal{K}}{\rho_{eff}}}
\end{equation}

A key prediction of this hydrodynamic model is the Vacuum Faraday Effect, where the vacuum exhibits a linear-in-$B$ magnetic optical response\textsuperscript{19}.

\textbf{External Validation:} This hydrodynamic mechanism is strongly supported by recent findings by Assouline \& Capua (2025), who demonstrated that the optical magnetic field contributes significantly to the Faraday effect, distinct from the electrical component, particularly at ultrafast timescales\textsuperscript{20}. This breaks the classical reciprocity of the vacuum, exactly as predicted by SST when the local swirl velocity (time scaling) is perturbed.

\section{Emergent Gravity (Corrected): near-field Euler vs far-field mediator}
Steady, axisymmetric Euler balance gives
\[
\frac{1}{\rho}\frac{dp}{dr}=-\frac{v_\theta^2(r)}{r}.
\]
For filamentary circulation $v_\theta=\Gamma/(2\pi r)$ this implies $dp/dr\propto -1/r^3$ and therefore a pressure deficit
$\Delta p\propto -1/r^2$, yielding forces $\propto 1/r^3$. This is a \emph{short-range} interaction and does not produce a
Newtonian inverse-square law.

Long-range gravity is modeled by a local mediator $\phi$ (clock/foliation mode) with static action
\[
S_{\rm stat}[\phi]=\int d^3x\left[\frac{\kappa}{2}(\nabla\phi)^2-\lambda\,\rho_m(\mathbf{x})\,\phi\right],
\qquad
\kappa\nabla^2\phi=-\lambda\rho_m,
\]
whose Green-function solution for localized sources gives $\phi(r)\propto 1/r$ and hence $|\nabla\phi|\propto 1/r^2$.

\section{The Grand Unified Scale Identity}

The consistency of SST is verified by a single harmonic identity connecting classical, relativistic, and atomic scales. We derive a maximal Hooke-law force $F_{max}$ using the electron mass $m_e$, Compton frequency $\omega_C$, and fine-structure constant $\alpha$\textsuperscript{25}:
\begin{equation}
F_{max} = m_e \left(\frac{\omega_C}{\alpha}\right)^2 r_e
\end{equation}

Multiplying this force by a Compton-scale radius $r_c$, we recover the Hydrogen ground state energy $E_B$ and the electron rest mass exactly\textsuperscript{26}:
\begin{equation}
F_{max} r_c = \frac{1}{2}m_e c^2 = \frac{E_B}{\alpha^2}
\end{equation}

This identity serves as the "Rosetta Stone" of SST, proving that atomic structure ($E_B$), relativity ($c$), and electrodynamics ($\alpha$) are geometrically coupled via the fluid properties of the vacuum.

\section{Conclusion}

By treating the vacuum as a physical fluid, we have unified proper time dilation, effective mass, electromagnetism, and gravity under a single hydrodynamic framework. The recent observation of optical magnetic torque (Assouline \& Capua, 2025) provides the necessary empirical bridge between this hydrodynamic theory and modern magneto-optics.



\subsection*{References (Internal & External)}


[1] Iskandarani, O. (2025). Circulation, Rigid Rotation, and Proper Time Dilation: A Fluid-Mechanical Representation of Kinematic Redshift. [Uploaded File 3].

[2] Iskandarani, O. (2025). Rotational Kinetic Energy Density and an Effective Mass Relation in Incompressible Fluids. [Uploaded File 2].

[3] Iskandarani, O. (2025). A Unified Electron Scale Relation from Classical Radius, Compton Frequency, and the Hydrogen Ground State Energy. [Uploaded File 1].

[4] Iskandarani, O. (2025). Swirl Pressure and Effective Gravitational Acceleration in Incompressible Rotating Flows. [Uploaded File 5].

[5] Iskandarani, O. (2025). Electromagnetism as Propagating Torsion in a Hydrodynamic Vacuum. [Uploaded File SST-11].

[6] Iskandarani, O. (2025). Energy, Impulse, and Stability of Thin Vortex Loops. [Uploaded File 4].

[7] Assouline, B. & Capua, A. (2025). Faraday effects emerging from the optical magnetic field. Scientific Reports, 15:39566.

[8] Kelvin, Lord (W. Thomson). (1867). On Vortex Atoms. Proc. R. Soc. Edin. 6, 94.

[9] Volovik, G. E. (2003). The Universe in a Helium Droplet. Oxford Univ. Press.










%=========================================
% References
%=========================================
        \bibliographystyle{unsrt}
        \begin{thebibliography}{99}

            \bibitem{Einstein1905} A.~Einstein, \newblock \emph{Ist die Tr\"agheit eines K\"orpers von seinem Energieinhalt
            abh\"angig?}, newblock Ann.\ Phys.\ \textbf{18}, 639--641 (1905).

        \end{thebibliography}

\end{document}