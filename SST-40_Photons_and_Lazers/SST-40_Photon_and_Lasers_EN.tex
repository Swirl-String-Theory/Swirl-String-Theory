%! Author = Omar Iskandarani
%! Title = Photons in Swirl–String Theory (SST): Kinematics on Swirl Strings and Plotting a Laser Beam
%! Date = Sept 4, 2025
%! Affiliation = Independent Researcher, Groningen, The Netherlands
%! ORCID = 0009-0006-1686-3961
%! DOI = 10.5281/zenodo.xxx
%========================================================================================
\newcommand{\paperdoi}{10.5281/zenodo.xxx}
%========================================================================================

%========================================================================================
% PACKAGES AND DOCUMENT CONFIGURATION
%========================================================================================
\documentclass[aps,prb,preprint,amsmath,amssymb]{revtex4-2} % switch to "reprint" for two-column look
\usepackage[utf8]{inputenc}
\usepackage[T1]{fontenc}
\usepackage{siunitx}
\usepackage{graphicx}
\usepackage{physics}
\usepackage{amsmath,amssymb,bm}
\usepackage[margin=1in]{geometry}
\usepackage[labelfont=bf]{caption}
\usepackage{tikz}
\usetikzlibrary{arrows.meta,positioning,fit}
% ---------- TikZ house styles (paste once) ----------
\tikzset{
    axis/.style   ={->,>=Latex,line width=0.6pt},
    thinline/.style={line width=0.6pt},
    bar/.style    ={draw,fill=gray!10,rounded corners=1pt},
    heater/.style ={draw,fill=red!15,rounded corners=1pt},
    sink/.style   ={draw,fill=blue!12,rounded corners=1pt},
    coilA/.style  ={thinline},
    coilB/.style  ={thinline,dashed},
    coilC/.style  ={thinline,dotted},
    lbl/.style    ={font=\footnotesize},
    note/.style   ={font=\scriptsize,align=center},
    box/.style    ={draw,rounded corners=2pt,fill=gray!8,inner sep=6pt},
    arr/.style    ={-Latex,line width=0.6pt},
}

\usepackage{pgfplots}\pgfplotsset{compat=1.18}
\usepackage[hidelinks]{hyperref}

% Safe \doi fallback (RevTeX usually defines \doi; this guards if not)
\providecommand{\doi}[1]{\href{https://doi.org/#1}{doi:\,#1}}

%-------------------------- SST minimal macro prelude (compile-safe) --------------------
% Vector swirl velocity + clock/vorticity symbols (aligned with Rosetta v0.6)
\newcommand{\vswirl}{\mathbf{v}_{\!\mkern-2mu\scriptstyle\boldsymbol{\circlearrowleft}}}
\newcommand{\omegas}{\boldsymbol{\omega}_{\!\mkern-2mu\scriptstyle\boldsymbol{\circlearrowleft}}}
\newcommand{\rc}{r_c}
\newcommand{\rhoF}{\rho_{f}}
\newcommand{\rhoE}{\rho_{E}}
\newcommand{\rhoM}{\rho_{\!m}}
% \vswirl = 1.09384563e6 m s^-1 ; \rc = 1.40897017e-15 m ; \rhoF = 7.0e-7 kg m^-3
%----------------------------------------------------------------------------------------

\title{\textbf{Photons in Swirl--String Theory (SST):\\
Kinematics on Swirl Strings and Plotting a Laser Beam}}
\author{Omar Iskandarani}
\affiliation{Independent Researcher, Groningen, The Netherlands}
\thanks{ORCID: 0009-0006-1686-3961, DOI: \paperdoi}
\date{\today}

\begin{document}
    \maketitle

    \begin{abstract}
        We model the photon as a one-dimensional, closed or open swirl string with phase
        $\phi(\mathbf{x},t)$ propagating helically along the string. Spin (circular polarization)
        corresponds to the handedness of the local swirl clock; optical orbital angular momentum (OAM)
        with topological charge $\ell$ is the phase winding in the transverse plane. For laser beams we use
        the (paraxial) Gaussian beam and its Laguerre--Gaussian extension to plot intensity and phase fields.
        All formulas are SI-dimensional and calibrated to the SST scale
        $\Omega_0=\lVert\vswirl\rVert/\rc$.
        In the \emph{Rosetta} mapping, the scalar phase mode plays the role of a Maxwell-like radiation
        sector with $\omega=\! ck$ in uniform backgrounds, while OAM corresponds to azimuthal phase winding.
    \end{abstract}

    \section{Kinematics: photon as a helical mode on a swirl string}
        Let $\mathbf{X}(s,t)$ denote the string centerline with arclength parameter $s$ and local tangent $\mathbf{t}$.
        A photon is modeled as a travelling phase wave on the string:
        \[
            \phi(\mathbf{x},t)=k z - \omega t + \ell\,\theta,\qquad k=\frac{2\pi}{\lambda}.
        \]
        where $k=2\pi/\lambda$, $\omega=2\pi f$, and $(r,\theta,z)$ are cylindrical coordinates along the propagation axis.
        \emph{Spin}/polarization is the local swirl-clock handedness (left/right), and \emph{OAM}
        is the integer winding $\ell\in\mathbb{Z}$ around the beam axis \cite{Allen1992,Siegman1986}.

        \paragraph{SST clock and energy density.}
            A convenient reference scale is
            \[
                \Omega_0=\frac{\lVert\vswirl\rVert}{\rc}\quad [\si{ s^{-1} }].
            \]
            Dimensional check: $[\vswirl]=\si{m\,s^{-1}}$, $[\rc]=\si{m}$, so $\Omega_0$ is a frequency.
            Numerically this recovers the electron Compton scale in the canonical calibration and is used as a
            normalization point.

    \section{Energy, momentum, and polarization}
    For a single photon, $E=\hbar\omega$ and $p=\hbar k$ (standard field theory).
    Within SST, the energy is associated with an effective string line energy.
    Without committing to microstructure, the operative identification is
    \[
        \boxed{E=\hbar\omega,\quad \mathbf{p}=\hbar \mathbf{k},\quad
        \text{spin}\;S=\pm\hbar\ \leftrightarrow\ \text{swirl-clock left/right}}
    \]
    where $+$ and $-$ correspond to left- and right-circular polarization, respectively.

    \section{Laser-beam model: Gaussian beam and LG modes}
    For a paraxial beam with waist $w_0$ at $z=0$ (\cite{Siegman1986}):
    \begin{align}
        w(z) &= w_0\sqrt{1+(z/z_R)^2},\qquad z_R=\frac{\pi w_0^2}{\lambda},\\
        R(z) &= z\!\left[1+\left(\frac{z_R}{z}\right)^2\right],\qquad
        \zeta(z)=\arctan\!\left(\frac{z}{z_R}\right).
    \end{align}
    The TEM$_{00}$ scalar field amplitude is
    \[
        E_{00}(r,z)=E_0\frac{w_0}{w(z)}\exp\!\Big(-\frac{r^2}{w(z)^2}\Big)
        \exp\!\Big(i kz - i\omega t + i\frac{k r^2}{2R(z)} - i\,\zeta(z)\Big),
    \]
    The intensity is $I=\tfrac12\epsilon_0 c\,|E|^2$ (units $\si{W\,m^{-2}}$).
    Laguerre--Gaussian (LG) with OAM $\ell$ and radial index $p$:
    \[
        E_{p}^{\ell}(r,\theta,z)=E_{00}\!\left(\frac{\sqrt{2}\,r}{w(z)}\right)^{|\ell|}
        L_p^{|\ell|}\!\!\left(\frac{2r^2}{w(z)^2}\right)e^{i\ell\theta}.
    \]
    For $\ell\neq 0$ there is an on-axis null; the ring maximum occurs at
    $r_{\max}(z)=w(z)\sqrt{|\ell|/2}$.

    \section{Numerical example and figures}
    Example parameters: \(\lambda=\SI{632.8}{nm}\), \(w_0=\SI{1.0}{mm}\) $\Rightarrow$
    \(z_R=\pi w_0^2/\lambda=\SI{4.9646}{m}\) and
    beam divergence \(\theta_{\mathrm{div}}=\lambda/(\pi w_0)=\SI{0.201}{mrad}\).

    \begin{figure}[h!]
        \centering
        \includegraphics[width=0.78\linewidth]{gaussian_beam_intensity.png}
        \caption{\textbf{Gaussian intensity (TEM$_{00}$) in the $z{=}0$ plane.}
        The map shows $I(r,0)\propto \exp(-2r^2/w_0^2)$ with a central peak and
        radial Gaussian decay. With $\lambda=\SI{632.8}{nm}$ and $w_0=\SI{1.0}{mm}$ this is
        the beam waist. The dimension of $I$ is $\si{W\,m^{-2}}$; relative scale shown.}
        \label{fig:gauss}
    \end{figure}

    \begin{figure}[h!]
        \centering
        \includegraphics[width=0.78\linewidth]{lg_phase_l1.png}
        \caption{\textbf{Phase field for LG$_0^{1}$ with topological charge $\ell{=}1$ at $z{=}0$.}
        The phase $\Phi(\theta)=\ell\,\theta$ winds by $2\pi$ around the axis and has a singular core
            (dark “vortex”): zero intensity on-axis and a ring maximum at
            $r_{\max}(0)=w_0/\sqrt{2}=\SI{0.707}{mm}$. This visualizes optical OAM.}
        \label{fig:lgphase}
    \end{figure}

    \begin{figure}[h!]
        \centering
        \includegraphics[width=0.78\linewidth]{beam_waist_vs_z.png}
        \caption{\textbf{Beam waist $w(z)$ versus $z$.}
        For $|z|\ll z_R$ the beam remains narrow; for $|z|\gg z_R$ one has $w(z)\approx |z|\theta_{\mathrm{div}}$
            with $\theta_{\mathrm{div}}=\lambda/(\pi w_0)$. The Rayleigh range $z_R=\SI{4.9646}{m}$
            marks the near- to far-field transition.}
        \label{fig:waist}
    \end{figure}

    \paragraph{Reading guide.}
        Fig.~\ref{fig:gauss} shows the TEM$_{00}$ intensity at the waist ($z=0$).
        Fig.~\ref{fig:lgphase} shows the $2\pi$ phase winding for $\ell=1$ (OAM, on-axis null).
        Fig.~\ref{fig:waist} confirms the known $w(z)$ scaling and the asymptotic divergence.

        Known limits: $w(0)=w_0$, far-field $z\gg z_R$ gives opening angle $\theta_{\rm div}\!=\!\lambda/(\pi w_0)$,
        and for $\ell\neq 0$ there is an axis null with ring maximum $r_{\max}(z)=w(z)\sqrt{|\ell|/2}$ \cite{Allen1992,Berry2001}.

    \section{Plotting recipe (algorithm)}
    \begin{enumerate}
        \item Choose \(\lambda\), \(w_0\); compute \(z_R=\pi w_0^2/\lambda\).
        \item Define a 2D grid in the $z=0$ plane; compute \(I(r,0)\propto e^{-2r^2/w_0^2}\) (TEM$_{00}$).
        \item For OAM, take phase \(\Phi=\ell\,\theta\) and (optionally) the LG envelope.
        \item For a longitudinal cut: plot \(w(z)\) and (optionally) \(r_{\max}(z)\).
    \end{enumerate}

    \section{Predictions (falsifiable) and edge cases}
    \textbf{P1 (spin↔swirl-clock).} Polarization helicity coincides one-to-one with the handedness of the local swirl clock;
    spin--to--orbital conversion under strong focusing produces steps in $\ell$ (test via forked interferograms \cite{Berry2001}).
    \textbf{P2 (OAM ring).} The on-axis null and ring radius \(r_{\max}\) follow the LG scaling;
    deviations at extreme focusing (non-paraxial) predict measurable phase modulations.
    \textbf{Edge cases.} Non-paraxial (\(w_0\!\sim\!\lambda\)), dispersive media, and the near field of structures
    (locally non-Gaussian) call for fully vectorial solutions.

    \section{Mainstream mapping (Rosetta-style quick dictionary)}
        The following items summarize the translation to standard EM/QO language, consistent with Rosetta v0.6:
        \begin{itemize}
            \item \textbf{Scalar radiation mode}: phase field obeys $\partial_t^2\theta-c^2\nabla^2\theta=0$ in uniform regions (luminal calibration).
            \item \textbf{Polarization}: spin $\pm\hbar$ maps to left/right circular polarization (handed swirl-clock).
            \item \textbf{OAM}: integer $\ell$ equals azimuthal phase winding; LG modes are standard paraxial solutions.
            \item \textbf{Energy/momentum}: $E=\hbar\omega$, $p=\hbar k$; intensity $I\propto |E|^2$ as in optics.
            \item \textbf{Analogue metric}: weak-field lensing/time-delay effects track gradients of the swirl energy fraction (see Rosetta).
        \end{itemize}

    \paragraph{Kid analogy.}
        A photon is like a tiny corkscrew ripple that travels along an invisible string:
        turning left or right sets polarization; adding an extra twist per loop yields OAM rings.


        \bibliographystyle{unsrt}
        \begin{thebibliography}{9}
            \bibitem{Siegman1986} A.~E.~Siegman, \emph{Lasers}, University Science Books (1986).
            \bibitem{Allen1992} L.~Allen \emph{et al.}, Phys. Rev. A \textbf{45}, 8185 (1992).
            \bibitem{Berry2001} M.~V.~Berry, J. Opt. A \textbf{6}, 259 (2004).
        \end{thebibliography}

\end{document}

