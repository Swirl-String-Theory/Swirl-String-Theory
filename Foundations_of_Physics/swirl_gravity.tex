
%! Author = Omar Iskandarani
%! Date = 11/17/2025
%! Affiliation = Independent Researcher, Groningen, The Netherlands
%! License = © 2025 Omar Iskandarani. All rights reserved. This manuscript is made available for academic reading and citation only. No republication, redistribution, or derivative works are permitted without explicit written permission from the author. Contact: info@omariskandarani.com
%! ORCID = 0009-0006-1686-3961
%! DOI = 10.5281/zenodo.xxx

\newcommand{\paperdoi}{10.5281/zenodo.xxx}
\newcommand{\papertitle}{Controlling Gravitational Behavior in the Swirl–String Theory Framework}

\documentclass[11pt]{article}
\usepackage{amsmath,amssymb,amsfonts,bm}
\usepackage{siunitx}
\usepackage[hidelinks]{hyperref}
\usepackage[a4paper,margin=1in]{geometry}
\usepackage[T1]{fontenc}
\usepackage[utf8]{inputenc}
\usepackage{microtype}
\usepackage{setspace}
\setstretch{1.1}

\newcommand{\titlepageOpen}{
    \begin{titlepage}
    \thispagestyle{empty}
    \centering
    \Large \bfseries \papertitle \par \vspace{1cm}
    {\Large \itshape \textbf{Omar Iskandarani}\textsuperscript{\textbf{*}} \par}
    \vspace{0.5cm}
    {\today \par}
    \vspace{0.5cm}
}

\newcommand{\titlepageClose}{
    \vfill \raggedright \null
    \begin{picture}(0,0)
    \put(0,-45){
        \begin{minipage}[b]{0.7\textwidth} \footnotesize
        \renewcommand{\arraystretch}{1.0}
        \noindent\rule{\textwidth}{0.4pt} \\[0.5em]
        \textsuperscript{\textbf{*}} Independent Researcher, Groningen, The Netherlands \\
        Email: \texttt{info@omariskandarani.com} \\
        ORCID: \texttt{\href{https://orcid.org/0009-0006-1686-3961}{0009-0006-1686-3961}} \\
        DOI: \href{https://doi.org/\paperdoi}{\paperdoi}
        \end{minipage}
    }
    \end{picture}
    \end{titlepage}
}

\begin{document}
    \titlepageOpen

    \begin{abstract}
        \noindent
        Swirl--String Theory (SST) models matter as quantized vortex filaments moving within a pervasive, inviscid ``swirl medium''.
        In this picture, what we call gravity arises from pressure deficits produced by the swirling cores rather than from spacetime curvature.
        This paper asks a practical question: \emph{if SST is the right effective description, what would ``controlling gravity'' look like within its rules?}
        I summarize the SST constants and bounds that fix the scale of any effect and then outline three classes of control:
        (i) modulating strength by adding or countering circulation;
        (ii) imposing directionality by aligning structures, phasing oscillations, or using bulk rotation; and
        (iii) attempting shielding, reflection or redirection by topological and dynamical means.
        For each, I discuss feasibility, energetic thresholds, and SST's built-in maximum-force limit.
        A key falsifiable prediction highlighted here is that any topological change to the swirl network that modifies gravitational coupling must co-emit a discrete electromagnetic impulse of fixed magnitude.
        The overall conclusion is conservative: macroscopic ``gravity control'' demands energy densities that push against SST's own limits, yet small, testable effects may be accessible in superfluid or Bose--Einstein--condensate analogs.
    \end{abstract}

    \titlepageClose

    \section*{1.\quad Background and Setup}

        Swirl--String Theory (SST) posits a flat background filled by an inviscid, incompressible swirl medium. Matter corresponds to topological vortex filaments (``swirl strings'') with core radius $r_c \sim 10^{-15}\,\mathrm{m}$ and quantized core swirl speed $v_{\circlearrowleft} \approx 1.09\times 10^6\,\mathrm{m\,s^{-1}}$.
        The ambient fluid mass density is very low ($\rho_f \approx 7\times 10^{-7}\,\mathrm{kg\,m^{-3}}$), but the vortex core stores a large mass-equivalent energy density $\rho_{\text{core}} \sim 3.9\times 10^{18}\,\mathrm{kg\,m^{-3}}$.
        The quantum of circulation is
        \begin{equation}
            \kappa \equiv 2\pi r_c\, v_{\circlearrowleft}.
        \end{equation}
        Gravitational attraction emerges as a pressure deficit created by these structures: co-rotating strings \emph{deepen} a shared pressure well, analogously to parallel vortices that pull together in a superfluid.

        For large separations ($r\gg r_c$), a softened ``swirl Coulomb'' potential captures the long-range behavior:
        \begin{equation}
            V_{\text{SST}}(r) \simeq - \frac{\Lambda}{r^2 + r_c^2}, \qquad
            |{\bf a}_g| \sim \left| \frac{d}{dr} \frac{\Lambda}{r^2+r_c^2} \right| \approx \frac{2\Lambda}{r^3}\quad (r\gg r_c),
        \end{equation}
        with strength set by a constant $\Lambda$ built from $(\rho_{\text{core}}, v_{\circlearrowleft}, r_c)$.
        The emergent coupling $G_{\text{swirl}}$ is calibrated to recover Newton's constant $G_N$ under ordinary conditions.
        SST also mirrors the general-relativistic upper bound on force via a maximum gravitational force $F_{\text{gr}}^{\max}\!\sim 3\times 10^{43}\,\mathrm{N}$; attempts to exceed it push the medium into nonlinearity or breakdown.

        \paragraph{Useful intuition.}
            When three quark-vortices merge into a ``baryon tube,'' the circulation adds ($\Gamma_{\text{baryon}} = 3\kappa$), increasing rim speed and deepening the core pressure deficit (read: larger rest mass).
            Likewise, two composite tubes that share an axis add their circulations; the combined pressure well between them is deeper than either alone.

    \section*{2.\quad Modulating Gravitational Strength}

    \subsection*{2.1\quad Strengthening attraction: additive circulation}
        Within SST, ``more linked circulation'' means a deeper pressure deficit and thus stronger attraction.
        Mechanisms include aligning multiple filaments, exciting higher-circulation modes, or coaxially \emph{sharing} a vortex core between composite structures.
        In effect, these increase $\Lambda_{\text{eff}}$ in $V_{\text{SST}}$ and raise $|{\bf a}_g| \propto \Lambda_{\text{eff}}/r^3$.

        \medskip
        \noindent\textit{Practical ceiling.}
        Macroscopic boosts require fantastically large $\Lambda_{\text{eff}}$, i.e., vast numbers of circulation quanta or core energy densities approaching the theory's limits.
        This quickly runs into the $F_{\text{gr}}^{\max}$ bound and the finite $v_{\circlearrowleft}$, so dramatic amplification at human scales is not realistic.

    \subsection*{2.2\quad Weakening attraction: opposing circulation}
        To suppress attraction, one seeks to \emph{undo} linking circulation.
        Counter-rotating (opposite-chirality) flow can raise local pressure---a ``hill'' that offsets the usual well.
        In fluid mechanics, vortex--antivortex pairs do not bind; in SST, carefully arranged opposite-chirality fields could reduce or, in principle, reverse the net pull in a region.
        A more global (and speculative) route would vary medium parameters such as $\rho_f$ or the effective swirl polarization; because $G_{\text{swirl}}$ derives from these, changing them changes the apparent strength of gravity.
        Both ideas face stability and energy constraints.

    \section*{3.\quad Directional and Anisotropic Control}

    SST is intrinsically directional because vortex lines have axes.
    Attraction is strongest when structures are coaxial and weaker off-axis.
    Three levers follow.

    \begin{itemize}
        \item \textbf{Alignment and focusing.} Arranging many filaments so their circulation is coherently directed can concentrate the field along a chosen axis, akin to a weak ``gravitational flashlight.''
        \item \textbf{Phased oscillations.} Small oscillations of the swirl network admit interference. A phased ring or array could weakly enhance the field on-axis and cancel in the plane, analogous to a phased antenna---but only at tiny amplitudes.
        \item \textbf{Bulk rotation.} Rapid rotation injects kinetic pressure that partially offsets the static deficit in the equatorial plane, yielding a mild polar--equatorial anisotropy (an SST cousin of frame dragging).
    \end{itemize}

    Define conceptually $G_{\text{swirl}}(\theta)$ as the effective coupling at polar angle $\theta$ relative to a chosen axis. In natural settings it is nearly isotropic, but purpose-built alignment could introduce a measurable (likely minute) angular dependence.

    \section*{4.\quad Shielding, Reflection, and Redirection}

    \paragraph{Shielding (screening).}
        A topological ``swirl cage''---a shell whose currents prevent linking between interior and exterior loops---could, in principle, cancel external fields inside, like a Faraday cage for pressure deficits.
        Realizing this likely requires opposite-chirality flows that mimic negative-mass behavior and would be metastable and energy intensive.

    \paragraph{Reflection and refraction.}
        Dynamic disturbances (the SST analog of gravitational waves) should partially reflect or refract at interfaces where the medium's effective stiffness or swirl polarization changes, by the usual impedance-mismatch logic.
        Rapidly rotating structures may also scatter perturbations, echoing superradiance analogies, though only weakly at accessible speeds.

    \paragraph{Redirection by flow.}
        Moving media drag signals (Fizeau-like effects). A large-scale circulating flow could gently deflect a background field around an object, acting as a very weak prism or mirror.
        Significant redirection would demand relativistic flow speeds, which are out of reach given $v_{\circlearrowleft}\!\ll c$.

    \section*{5.\quad Feasibility, Energy Scales, and Falsifiable Signatures}

    \paragraph{Energetics.}
        As a crude scaling, to counter Earth's $g\simeq 9.8\,\mathrm{m\,s^{-2}}$ over meter scales would require $\Lambda_{\text{eff}}$ orders of magnitude above the canonical value (think $10^{14}\,\mathrm{J\,m}$), far beyond practical means.
        The $F_{\text{gr}}^{\max}$ bound and finite core parameters together cap any realistic device well below macroscopic control.

    \paragraph{Electromagnetic co-signature.}
        SST ties gravity-like changes to electromagnetism: any \emph{topological} change in the swirl network (creating, annihilating, reconnecting loops) should produce a discrete electromagnetic impulse of fixed magnitude $\Delta\Phi=\pm \Phi_\star$ (comparable to a flux quantum), implying tiny, sharp voltage spikes (\,$\sim 10^{-3}$--$10^{-6}$\,V) when ``switching'' a configuration.
        This offers a clear experimental handle: gravity modulation without the EM blip would falsify the SST mechanism.

    \paragraph{Near-term tests.}
        Laboratory superfluids provide analogs. Co-rotating quantized vortices attract and can bind; engineered mixtures of opposite circulation in a condensate could display chirality-dependent weak interactions between test bodies.
        Looking for the predicted EM pulses during vortex reconnections or nucleation events is within current reach.

    \section*{6.\quad Conclusions}

    Within the rules of Swirl--String Theory, ``gravity control'' reduces to manipulating circulation and topology in an inviscid medium.
    Strengthening requires additive circulation; weakening needs carefully phased opposing flows or altered medium properties; directionality comes from alignment, phasing, and rotation; and shielding or redirection demand topological cages or strong flow gradients.
    All roads run into the same realities: steep energy requirements, stability issues, and an SST maximum-force bound.
    Nevertheless, the framework yields concrete, falsifiable signatures and small, testable effects in table-top analogs.
    That combination---clear limits plus clear predictions---is what makes the exercise scientifically useful.

    \bigskip
    \noindent\textbf{Notes on sources.} The ideas summarized here follow the SST canon (v0.5.10) and related notes on topological gravity via swirling knots.
    Where specific numbers are quoted (e.g., $r_c$, $v_{\circlearrowleft}$, $\rho_f$, $\rho_{\text{core}}$, $F_{\text{gr}}^{\max}$), they are the canonical values used for calibration in those documents.

    \bigskip
    \noindent\rule{\textwidth}{0.4pt}

    \bibliographystyle{unsrt}
    \begin{thebibliography}{99}
        \bibitem{Einstein1905} A.~Einstein, \emph{\"Uber einen die Erzeugung und Verwandlung des Lichtes betreffenden heuristischen Gesichtspunkt}, Ann.\ Phys.\ \textbf{17}, 132--148 (1905).
    \end{thebibliography}

\end{document}