\documentclass[11pt]{article}

\usepackage{amsmath,amssymb,amsfonts,bm}
\usepackage{siunitx}
\usepackage[hidelinks]{hyperref}
\usepackage[a4paper,margin=1in]{geometry}
\usepackage[T1]{fontenc}
\usepackage[utf8]{inputenc}

\title{Rotational Kinetic Energy Density and an Effective Mass Relation in Incompressible Fluids}
\author{Omar Iskandarani}
\date{\today}

\begin{document}
    \maketitle

    \begin{abstract}
        Kinetic energy contributes to inertia and gravitational mass through the relativistic relation
        $E = mc^2$.
        In extended media such as fluids, this contribution can be expressed as an effective mass density associated with internal motion.
        We consider an incompressible, inviscid Newtonian fluid undergoing rigid-body rotation in a finite cylinder and compute the volume-averaged rotational kinetic energy density.
        By associating this energy density with an effective mass density via $E = mc^2$ in the nonrelativistic limit, we obtain the closed-form relation
        \[
            \frac{\Delta \rho_{\mathrm{eff}}}{\rho} = \frac{1}{4}\left(\frac{v_{\mathrm{edge}}}{c}\right)^2,
        \]
        where $v_{\mathrm{edge}}$ is the tangential speed at the cylinder boundary and $\rho$ is the rest-mass density.
        The result provides a transparent classical example of how rotational motion modifies the mass density at order $(v/c)^2$.
        We discuss the connection with relativistic continuum mechanics and provide numerical estimates for laboratory and astrophysical regimes.
    \end{abstract}

    \section{Introduction}
        The equivalence between energy and mass, expressed by $E = mc^2$, implies that kinetic, field, and binding energies contribute to the inertia and gravitational mass of extended systems~\cite{Einstein1905,Tolman1934,LL_FieldTheory}.
        In the context of relativistic continuum mechanics, this statement is encoded in the stress--energy tensor $T^{\mu\nu}$: the total mass--energy is obtained by integrating $T^{00}$ over a spatial hypersurface, and $T^{00}$ includes both rest-mass and kinetic contributions~\cite{LL_FieldTheory,RezzollaZanotti2013}.

        While this viewpoint is standard in high-energy physics and general relativity, explicit examples in simple fluid configurations remain pedagogically useful.
        In particular, it is instructive to make the contribution of \emph{rotational} kinetic energy to an effective mass density quantitatively explicit in a setting where the flow field is analytically tractable.

        In this paper we analyze a canonical configuration from classical fluid mechanics~\cite{Batchelor1967,LandauLifshitzFM}: an incompressible, inviscid fluid in rigid-body rotation inside a finite cylinder.
        Within this model we:
        \begin{enumerate}
            \item compute the local and volume-averaged rotational kinetic energy density;
            \item define an effective mass density $\Delta\rho_{\mathrm{eff}}$ via $E = mc^2$ in the regime $v \ll c$;
            \item derive a compact expression for $\Delta\rho_{\mathrm{eff}}/\rho$ in terms of the edge speed $v_{\mathrm{edge}}$;
            \item discuss how this classical result fits within the framework of relativistic continuum mechanics.
        \end{enumerate}

        The derivation uses only incompressible Euler flow and the special-relativistic mass--energy relation.
        No modifications of Newtonian or relativistic theory are proposed.

    \section{Rigid-body rotation in an incompressible, inviscid fluid}
        \subsection{Flow configuration}
            We consider a Newtonian fluid of constant rest-mass density $\rho$ occupying a right circular cylinder of radius $R$ and height $L$.
            The fluid undergoes steady rigid-body rotation with constant angular velocity $\Omega$ about the $z$-axis.
            In cylindrical coordinates $(r,\theta,z)$, with $0 \le r \le R$, the velocity field is
            \begin{equation}
                \boldsymbol{v}(r) = \Omega r\,\hat{\boldsymbol{\theta}}.
                \label{eq:velocity-field}
            \end{equation}
            This flow is incompressible and inviscid:
            \begin{equation}
                \nabla\cdot\boldsymbol{v} = 0,
                \qquad
                \text{viscosity} = 0,
            \end{equation}
            and it satisfies the steady Euler equations with an appropriate pressure distribution~\cite{Batchelor1967,LandauLifshitzFM}.

        \subsection{Local kinetic energy density}
            The local kinetic energy density of the fluid is
            \begin{equation}
                e_{\mathrm{kin}}(r) = \frac{1}{2}\,\rho \lVert \boldsymbol{v}(r) \rVert^2
                = \frac{1}{2}\,\rho\,\Omega^2 r^2.
                \label{eq:local-ekin}
            \end{equation}
            This quantity is position-dependent and increases quadratically with radius.

        \subsection{Total rotational energy and volume-averaged energy density}
            The total rotational kinetic energy is obtained by integrating Eq.~\eqref{eq:local-ekin} over the fluid volume:
            \begin{equation}
                \begin{aligned}
                    E_{\mathrm{rot}}
                    &= \int_V e_{\mathrm{kin}}\,\mathrm{d}V \\
                    &= \int_0^L\!\mathrm{d}z \int_0^{2\pi}\!\mathrm{d}\theta \int_0^R
                    \frac{1}{2}\,\rho\,\Omega^2 r^2 \, r\,\mathrm{d}r \\
                    &= \frac{1}{2}\,\rho\,\Omega^2 (2\pi L) \int_0^R r^3\,\mathrm{d}r \\
                    &= \frac{\pi}{4}\,\rho\,\Omega^2\,L\,R^4.
                \end{aligned}
                \label{eq:Erot}
            \end{equation}
            The cylinder volume is $V = \pi R^2 L$, so the volume-averaged kinetic energy density is
            \begin{equation}
                \langle e_{\mathrm{kin}} \rangle
                = \frac{E_{\mathrm{rot}}}{V}
                = \frac{\frac{\pi}{4}\,\rho\,\Omega^2\,L\,R^4}{\pi R^2 L}
                = \frac{1}{4}\,\rho\,\Omega^2 R^2.
                \label{eq:avg-ekin}
            \end{equation}

            It is convenient to express this in terms of the edge speed
            \begin{equation}
                v_{\mathrm{edge}} := \Omega R.
                \label{eq:vedge-def}
            \end{equation}
            Then Eq.~\eqref{eq:avg-ekin} becomes
            \begin{equation}
                \langle e_{\mathrm{kin}} \rangle
                = \frac{1}{4}\,\rho\,v_{\mathrm{edge}}^2.
                \label{eq:avg-ekin-vedge}
            \end{equation}
            For comparison, the kinetic energy density at the boundary is
            \begin{equation}
                e_{\mathrm{kin}}(R)
                = \frac{1}{2}\,\rho\,v_{\mathrm{edge}}^2,
            \end{equation}
            so the volume average is exactly one half of the boundary value, reflecting the quadratic radial profile.

    \section{Effective mass density from $E = mc^2$}
        \subsection{Nonrelativistic limit and effective density}
            In special relativity, the total relativistic energy of a fluid element with rest-mass density $\rho$ and small velocity $v\ll c$ can be decomposed as~\cite{LL_FieldTheory,RezzollaZanotti2013}
            \begin{equation}
                \varepsilon \simeq \rho c^2 + \frac{1}{2}\rho v^2 + \cdots,
            \end{equation}
            where $\varepsilon$ is the total energy density in the local rest frame, and the ellipsis denotes higher-order terms in $v^2/c^2$ and internal energy contributions.
            To leading order in $v^2/c^2$, the kinetic part of the energy density can therefore be regarded as an \emph{effective mass density} via
            \begin{equation}
                \Delta\rho_{\mathrm{eff}}(\boldsymbol{x}) = \frac{e_{\mathrm{kin}}(\boldsymbol{x})}{c^2}.
                \label{eq:local-rhoeff}
            \end{equation}
            This interpretation is consistent with the structure of the stress--energy tensor for a perfect fluid~\cite{LL_FieldTheory}.

            For the rigidly rotating configuration considered here, we focus on the volume-averaged effective mass density
            \begin{equation}
                \Delta \rho_{\mathrm{eff}}
                := \frac{\langle e_{\mathrm{kin}} \rangle}{c^2}.
                \label{eq:avg-rhoeff-def}
            \end{equation}
            Inserting Eq.~\eqref{eq:avg-ekin-vedge} into Eq.~\eqref{eq:avg-rhoeff-def}, we obtain
            \begin{equation}
                \Delta \rho_{\mathrm{eff}}
                = \frac{1}{4 c^2}\,\rho\,v_{\mathrm{edge}}^2.
                \label{eq:avg-rhoeff}
            \end{equation}
            Dividing by the rest-mass density $\rho$ yields
            \begin{equation}
                \frac{\Delta \rho_{\mathrm{eff}}}{\rho}
                = \frac{1}{4}\left(\frac{v_{\mathrm{edge}}}{c}\right)^2.
                \label{eq:rhoeff-ratio}
            \end{equation}
            Thus, in the nonrelativistic regime, the rotational contribution to the mass density is second order in $v_{\mathrm{edge}}/c$, with a geometric coefficient $1/4$ specific to rigid-body rotation in a cylinder.

        \subsection{Dimensional check}
            The dimensions of Eq.~\eqref{eq:avg-rhoeff} are
            \[
                [\Delta \rho_{\mathrm{eff}}]
                = \frac{[\rho][v]^2}{[c]^2}
                = \frac{\text{kg}\,\text{m}^{-3}\,(\text{m/s})^2}{(\text{m/s})^2}
                = \text{kg}\,\text{m}^{-3},
            \]
            as required for a mass density.
            The ratio $\Delta\rho_{\mathrm{eff}}/\rho$ in Eq.~\eqref{eq:rhoeff-ratio} is dimensionless, as expected.

    \section{Numerical estimates}
        \subsection{Laboratory-scale example}
            Consider water with $\rho \approx \SI{1.0e3}{kg.m^{-3}}$, a cylinder of radius $R = \SI{0.10}{m}$, and angular velocity $\Omega = \SI{1.0e3}{s^{-1}}$, corresponding to $v_{\mathrm{edge}} = \SI{100}{m.s^{-1}}$.
            Then
            \begin{equation}
                \frac{\Delta \rho_{\mathrm{eff}}}{\rho}
                = \frac{1}{4}\left(\frac{100}{3.0\times 10^8}\right)^2
                \approx 3\times 10^{-14},
            \end{equation}
            and
            \begin{equation}
                \Delta \rho_{\mathrm{eff}} \sim 3\times 10^{-11}\,\mathrm{kg\,m^{-3}}.
            \end{equation}
            The effect is many orders of magnitude below typical experimental resolution in laboratory fluids.

        \subsection{Astrophysical order-of-magnitude}
            In astrophysical settings, rotational velocities can be relativistic.
            For example, in the inner regions of accretion flows or rapidly rotating compact stars, characteristic speeds may reach $v \sim 0.1c$ or higher~\cite{RezzollaZanotti2013}.
            If one naively substitutes $v_{\mathrm{edge}} = 0.1c$ into Eq.~\eqref{eq:rhoeff-ratio}, one finds
            \begin{equation}
                \frac{\Delta \rho_{\mathrm{eff}}}{\rho}
                \sim \frac{1}{4} (0.1)^2 = 2.5\times 10^{-3},
            \end{equation}
            already approaching the percent level.
            However, in such regimes a fully relativistic treatment of the fluid is required, and higher-order terms in $v^2/c^2$ as well as strong-gravity effects must be included.
            Equation~\eqref{eq:rhoeff-ratio} should therefore be regarded as illustrating the leading-order trend rather than providing a quantitatively accurate model for relativistic flows.

    \section{Relation to relativistic continuum mechanics}
        In relativistic hydrodynamics, a perfect fluid is described by the stress--energy tensor~\cite{LL_FieldTheory,RezzollaZanotti2013}
        \begin{equation}
            T^{\mu\nu} = (\varepsilon + p) u^\mu u^\nu + p g^{\mu\nu},
        \end{equation}
        where $\varepsilon$ is the total energy density in the fluid rest frame, $p$ is the pressure, and $u^\mu$ is the four-velocity.
        In the nonrelativistic limit and for small internal energy, one has
        \begin{equation}
            \varepsilon \simeq \rho c^2 + \frac{1}{2}\rho v^2 + \ldots,
        \end{equation}
        consistent with the decomposition used in Eq.~\eqref{eq:local-rhoeff}.

        The contribution of kinetic energy to the gravitational mass of an extended system can be derived by integrating $T^{00}$ over space in an appropriate frame~\cite{Tolman1934,LL_FieldTheory}.
        Our treatment effectively isolates the rotational part of this contribution for the specific case of rigid-body rotation in an incompressible fluid.
        Equation~\eqref{eq:rhoeff-ratio} can therefore be viewed as the nonrelativistic limit of the rotational piece of $T^{00}/c^2$, evaluated in a simple geometry.

    \section{Discussion and outlook}
        We have derived a compact relation between the rotational kinetic energy of an incompressible, inviscid fluid in rigid-body rotation and an associated effective mass density.
        The key steps are:
        \begin{enumerate}
            \item computation of the local kinetic energy density $e_{\mathrm{kin}}(r) = \frac{1}{2}\rho \Omega^2 r^2$;
            \item volume averaging over a finite cylinder, yielding $\langle e_{\mathrm{kin}} \rangle = \frac{1}{4}\rho v_{\mathrm{edge}}^2$;
            \item definition of an effective mass density via $E = mc^2$ in the nonrelativistic limit, resulting in
            $\Delta\rho_{\mathrm{eff}}/\rho = \frac{1}{4}(v_{\mathrm{edge}}/c)^2$.
        \end{enumerate}

        The analysis is fully contained within classical fluid mechanics and special relativity.
        It provides a transparent example of how rotational motion contributes to the mass density of an extended medium at order $(v/c)^2$.
        The coefficient $1/4$ is specific to rigid-body rotation in a cylinder and reflects the radial structure of the velocity field.
        For other velocity profiles or geometries, different geometric factors would appear, though the basic scaling $\Delta\rho_{\mathrm{eff}}/\rho \propto \langle v^2\rangle/c^2$ remains.

        Potential applications include:
        \begin{itemize}
            \item pedagogical demonstrations of mass--energy equivalence in continuum systems;
            \item benchmark problems for numerical schemes that couple incompressible fluid dynamics to relativistic mass--energy accounting in the low-velocity regime;
            \item conceptual comparisons with fully relativistic treatments of rotating fluids in astrophysical contexts.
        \end{itemize}
        Any attempt to attribute fundamental rest mass to internal rotational motion would require additional structural assumptions and a fully relativistic framework, and lies beyond the scope of the present work.

    \section*{Acknowledgments}
        The author thanks the authors of standard texts on fluid mechanics and relativistic field theory for providing the background material on which this analysis is based.

        \bibliographystyle{unsrt}
        \begin{thebibliography}{99}

            \bibitem{Einstein1905}
            A.~Einstein,
            \newblock \emph{Ist die Tr\"agheit eines K\"orpers von seinem Energieinhalt
            abh\"angig?},
            \newblock Ann.\ Phys.\ \textbf{18}, 639--641 (1905).

            \bibitem{Tolman1934}
            R.~C. Tolman,
            \newblock \emph{Relativity, Thermodynamics and Cosmology},
            \newblock Clarendon Press, Oxford (1934).

            \bibitem{LL_FieldTheory}
            L.~D. Landau and E.~M. Lifshitz,
            \newblock \emph{The Classical Theory of Fields},
            \newblock 4th ed., Butterworth--Heinemann, Oxford (1975).

            \bibitem{RezzollaZanotti2013}
            L.~Rezzolla and O.~Zanotti,
            \newblock \emph{Relativistic Hydrodynamics},
            \newblock Oxford University Press, Oxford (2013).

            \bibitem{Batchelor1967}
            G.~K. Batchelor,
            \newblock \emph{An Introduction to Fluid Dynamics},
            \newblock Cambridge University Press, Cambridge (1967).

            \bibitem{LandauLifshitzFM}
            L.~D. Landau and E.~M. Lifshitz,
            \newblock \emph{Fluid Mechanics},
            \newblock 2nd ed., Pergamon Press, Oxford (1987).

        \end{thebibliography}

\end{document}