%! Author = Omar Iskandarani
%! Date = 11/17/2025
%! Affiliation = Independent Researcher, Groningen, The Netherlands
%! License = © 2025 Omar Iskandarani. All rights reserved. This manuscript is made available for academic reading and citation only. No republication, redistribution, or derivative works are permitted without explicit written permission from the author. Contact: info@omariskandarani.com
%! ORCID = 0009-0006-1686-3961
%! DOI = 10.5281/zenodo.xxx

\newcommand{\paperdoi}{10.5281/zenodo.xxx}
\newcommand{\papertitle}{Gravitational Modulation in Swirl-String Theory}

%=========================================
% % PREAMBLE, PACKAGES AND DOCUMENT CONFIGURATION
%=========================================
\documentclass[11pt]{article}
\usepackage{amsmath,amssymb,amsfonts,bm}
\usepackage{siunitx}
\usepackage[hidelinks]{hyperref}
\usepackage[a4paper,margin=1in]{geometry}
\usepackage[T1]{fontenc}
\usepackage[utf8]{inputenc}

\newcommand{\titlepageOpen}{
    \begin{titlepage}
        \thispagestyle{empty}
        \centering
        \Large \bfseries \papertitle \par \vspace{1cm}
        {\Large \itshape \textbf{Omar Iskandarani}\textsuperscript{\textbf{*}} \par}
        \vspace{0.5cm}
        {\today \par}
        \vspace{0.5cm}
}

\newcommand{\titlepageClose}{
        \vfill \raggedright \null
        \begin{picture}(0,0)
            \put(0,-45){  % Shift 200pt left, 40pt down
                \begin{minipage}[b]{0.7\textwidth} \footnotesize
                    \renewcommand{\arraystretch}{1.0}
                    \noindent\rule{\textwidth}{0.4pt} \\[0.5em]
                    \textsuperscript{\textbf{*}} Independent Researcher, Groningen, The Netherlands \\
                    Email: \texttt{info@omariskandarani.com} \\
                    ORCID: \texttt{\href{https://orcid.org/0009-0006-1686-3961}{0009-0006-1686-3961}} \\
                    DOI: \href{https://doi.org/\paperdoi}{\paperdoi}
                \end{minipage}
            }
        \end{picture}
    \end{titlepage}
}
%=========================================
% Start Document - Title Page
%=========================================
\begin{document}
    \titlepageOpen

    \begin{abstract}

    \end{abstract}

    \titlepageClose
%=========================================
% Title Page End
%=========================================



\section{Introduction}

Swirl-String Theory (SST) is an emerging framework that reimagines fundamental particles and forces as manifestations of knotted \emph{vortex filaments} in an underlying incompressible fluid medium. Instead of treating gravity as curvature of spacetime, SST posits that gravitational effects arise from \textbf{swirling fluid dynamics} in flat space. In this picture, what we perceive as mass and gravity are generated by rotating “swirl strings” — fluid vortex structures whose circulation and topology produce long-range attraction. \textbf{Gravitational modulation} in SST refers to the idea that gravitational strength and related phenomena (attraction, time dilation, etc.) can be tuned or altered by changing the \emph{swirl characteristics} (such as circulation, velocity, or topology) of these vortex structures. This concept contrasts starkly with general relativity’s fixed geometric interpretation, suggesting instead a controllable, fluid-based gravity that is in principle testable in laboratory analogues\href{https://www.academia.edu/144408705/Long_Distance_Swirl_Gravity_from_Chiral_Swirling_Knots_with_Central_Holes#:~:text=composite%20tubes%20%28e,1%3A%20%28a}{academia.edu}\href{https://www.academia.edu/144408705/Long_Distance_Swirl_Gravity_from_Chiral_Swirling_Knots_with_Central_Holes#:~:text=crisp%20laboratory%20discriminator%20follows%20from,of%20radius%20rc%20supports%20a}{academia.edu}. Recent works have established key SST principles – including the Chronos-Kelvin time invariant and a \emph{Swirl Coulomb constant} $\Lambda$ – that lay the foundation for a fluid-mechanical unification of forces\href{https://www.academia.edu/144408705/Long_Distance_Swirl_Gravity_from_Chiral_Swirling_Knots_with_Central_Holes#:~:text=This%20Canon%20is%20the%20single,measurement%20via%20R%E2%86%94T%20phase%20transitions}{academia.edu}. In this paper, we explore how SST’s fluid vortex model gives rise to gravitational forces and how these forces might be modulated by altering swirl parameters. We review the emergence of gravity from pressure deficits in swirling cores, the information-theoretic (entropic) interpretation of swirl gravity, and the unique prediction of \emph{electromagnetic signals} that accompany dynamic gravitational changes in SST. Throughout, we use standard SST notation (see \cite{SSTCanon2025} for the canon of definitions) and SI units.


\section{Swirl-String Fundamentals and Vortex Gravity Mechanism}

\subsection{Core Structure of a Swirl String}

In SST, each elementary particle is modeled as a \textbf{swirl string}: a thin, tubular vortex filament with a well-defined core radius $r_c$ and a circulating superfluid flow. The core is characterized by a mass density $\rho_{\text{core}}$ (the density of the aether-like fluid composing the string) and supports a tangential swirl velocity $\mathbf{v}\textit{{!\boldsymbol{\circlearrowleft}}$ around its axis\href{https://www.academia.edu/144408705/Long_Distance_Swirl_Gravity_from_Chiral_Swirling_Knots_with_Central_Holes#:~:text=31%20,String%20Theory}{academia.edu}\href{https://www.academia.edu/144408705/Long_Distance_Swirl_Gravity_from_Chiral_Swirling_Knots_with_Central_Holes#:~:text=Topological%20Gravity%2C%20Entropic%20Force%2C%20Circulation,i%CE%A8%20in%20a%20simply%20connected}{academia.edu}. The swirl velocity profile typically resembles a Rankine vortex: nearly rigid rotation inside the core radius and $v}{!\boldsymbol{\circlearrowleft}} \propto 1/r$ decay outside the core, ensuring finite circulation. The circulation of each string is quantized. Kelvin's circulation theorem guarantees an invariant circulation quantum $\kappa$ for any closed loop co-moving with the fluid, analogous to quantized flux in superconductors. In fact, one can identify the fundamental circulation quantum with Planck’s constant (up to constants), establishing a bridge between fluid rotation and quantum theory. A canonical identification is $\kappa \equiv h = 2\pi,r_c,v_c,m_{\text{eff}}$, which yields an \emph{effective vortex mass} $m_{\text{eff}} = \frac{h}{2\pi,r_c,v_c}$ for a given core radius $r_c$ and characteristic core speed $v_c = |\mathbf{v}_{!\boldsymbol{\circlearrowleft}}|$\href{https://www.academia.edu/144408705/Long_Distance_Swirl_Gravity_from_Chiral_Swirling_Knots_with_Central_Holes#:~:text=theorem%20%C4%84xes%20circulation%20,plateaus%20at%20n%CE%BA%3B%20for%20loops}{academia.edu}\href{https://www.academia.edu/144408705/Long_Distance_Swirl_Gravity_from_Chiral_Swirling_Knots_with_Central_Holes#:~:text=,varying%20core%20areal}{academia.edu}. By plugging in $r_c$ and $v_c$ values consistent with known particle sizes and vortex speeds, this quantized vortex mass formula reproduces realistic particle masses to within remarkable accuracy in the earlier Vortex Aether Model framework \cite{VAMReform2025}. In SST, the same idea is carried forward: a particle’s rest mass originates from the rotational kinetic energy and topology of its swirl, rather than from a fundamental mass parameter. In essence, \textbf{the inertia and gravitational mass of a particle are emergent properties of its swirling aetheric core}.


Each swirl-string is topologically characterized by its knottedness and linking number. Chiral (non-self-congruent) knots carry nonzero topological helicity and serve as the building blocks of matter in SST\href{https://www.academia.edu/144408705/Long_Distance_Swirl_Gravity_from_Chiral_Swirling_Knots_with_Central_Holes#:~:text=Long,deepens%20the%20common%20pressure%20well}{academia.edu}\href{https://www.academia.edu/144408705/Long_Distance_Swirl_Gravity_from_Chiral_Swirling_Knots_with_Central_Holes#:~:text=without%20veri%C4%84ed%20linking%20would%20falsify,String%20Theory}{academia.edu}. For example, a trefoil vortex knot (the simplest chiral knot) can model an elementary fermion, whereas an achiral configuration (like a symmetric figure-eight knot) has canceling helicity and behaves very differently under gravity (as discussed later). Crucially, a swirl string’s topology (knot type, twist, linking) influences its gravitational behavior. The concept of \textbf{gravitational modulation} in SST largely hinges on this fact: by altering the swirl topology or parameters (for instance, merging multiple swirls or flipping their chirality), one modulates the depth of the induced “gravitational” potential well and hence the strength of attraction or repulsion.


\subsection{Pressure Deficit and Emergent Attraction}

The mechanism by which a swirl string generates gravity is fundamentally \emph{hydrodynamic}. A spinning vortex filament creates a region of reduced pressure along its core. Intuitively, the faster the fluid circulates, the lower the pressure in the core (by Bernoulli’s principle). In a fluid of density $\rho_{\text{core}}$, a circular flow speed $v_{!\boldsymbol{\circlearrowleft}}$ induces a pressure drop $\Delta p$ given (in the simplest incompressible, irrotational outside flow approximation) by the familiar relation:

\begin{equation*}

\Delta p ;=; -,\frac{1}{2},\rho_{\text{core}},v_{!\boldsymbol{\circlearrowleft}}^2~,

\end{equation*}

indicating that the core pressure is lower than the ambient pressure by an amount proportional to the kinetic energy density of the swirl. This pressure deficit acts analogously to a gravitational potential well in SST. Surrounding objects or fluid elements feel a force toward the low-pressure region. More formally, the Euler equation for the fluid (neglecting viscosity and assuming steady flow) gives $\rho_{\text{core}}\mathbf{a} = -\nabla p$. Thus a pressure gradient produces acceleration $\mathbf{a} = -\nabla p/\rho_{\text{core}}$. For the above $\Delta p$, the resulting acceleration (force per unit mass) is


gswirl≈−∇(Δpρcore)=−∇(−12ρcorev⁣↺2ρcore)=∇(v⁣↺22),
\mathbf{g}_{\text{swirl}} \;\approx\; -\,\nabla\Big(\frac{\Delta p}{\rho_{\text{core}}}\Big) \;=\; -\nabla\Big(\frac{-\frac{1}{2}\rho_{\text{core}}v_{\!\boldsymbol{\circlearrowleft}}^2}{\rho_{\text{core}}}\Big) \;=\; \nabla\Big(\frac{v_{\!\boldsymbol{\circlearrowleft}}^2}{2}\Big)~,
gswirl≈−∇(ρcoreΔp)=−∇(ρcore−21ρcorev↺2)=∇(2v↺2),

pointing inward toward the vortex core. Physically, test particles (or other vortices) are drawn into this low-pressure region, mimicking gravitational attraction. Two nearby swirl-strings will experience a mutual force drawing them together if their pressure wells overlap or if they are arranged to deepen a shared pressure region.


In fact, when two swirl strings share a common central axis (imagine two vortex tubes aligned end-to-end or coaxially), their circulations add \emph{linearly} (by Kelvin’s superposition principle) to give a combined circulation $\Gamma_{\text{total}}$. If each string carries circulation $\Gamma$, the composite system has $\Gamma_{\text{total}} = \Gamma_1 + \Gamma_2$. The \textbf{pressure well} associated with the composite tube is significantly deeper than that of either single vortex. As a concrete example, consider two proton-like swirl cores (each with circulation $\Gamma = 3\kappa$, as a proton’s three quark vortices combine to $3\kappa$) brought together along the same axis separated by an interface. They effectively form one continuous swirl tube with $\Gamma_{\text{total}} = 3\kappa + 3\kappa = 6\kappa$, doubling the circulation. The pressure deficit in the merged core is correspondingly larger, leading to a stronger long-range attraction than either proton’s individual field\href{https://www.academia.edu/144408705/Long_Distance_Swirl_Gravity_from_Chiral_Swirling_Knots_with_Central_Holes#:~:text=speeds%20the%20rim%20roughly%20threefold,11}{academia.edu}\href{https://www.academia.edu/144408705/Long_Distance_Swirl_Gravity_from_Chiral_Swirling_Knots_with_Central_Holes#:~:text=shared%20pressure%20well%20and%20yields,contribution%20following%20from%20additive%20circulation}{academia.edu}. This scenario provides a fluid-mechanical explanation for the cohesiveness of molecular bound states: \emph{neutral molecules (like H$_2$) experience mutual attraction in flat space due to the additive circulation of their internal swirl cores} \cite{London1930,Casimir1948}. In other words, the van der Waals/London forces between neutral atoms can be viewed in SST as a gravitational-like attraction emerging from shared vortex lines \cite{SSTHelicity2025}. The SST model thus seamlessly links what would traditionally be called “residual electromagnetic forces” to its unified swirl gravity: both result from the same pressure-well mechanism in the aether fluid.


It is important to note that this emergent gravity in SST operates entirely in a \emph{flat spacetime background}. The geometry of spacetime is not curved; instead, the geometry of the fluid flow determines the motion of masses. This aligns with entropic gravity viewpoints where gravity is not a fundamental interaction but an emergent, statistical effect \cite{Verlinde2011,Verlinde2017}. Here, however, the “entropy” has a concrete mechanical origin in vorticity and pressure. We can make this analogy precise. Define a \textbf{swirl entropy} field $S_{!\boldsymbol{\circlearrowleft}}(x^\mu)$ that increases with the local swirl areal density $\rho_{!\boldsymbol{\circlearrowleft}}$ (the vortex flux through a unit area). For instance, one may set

\begin{equation*}

S_{!\boldsymbol{\circlearrowleft}}(x^\mu) ;=; k_B \ln!\Bigg(1 + \frac{\rho_{!\boldsymbol{\circlearrowleft}}(x^\mu)}{\rho_0}\Bigg)~,

\end{equation*}

where $\rho_0$ is a constant scale. In the regime of small perturbations ($\rho_{!\boldsymbol{\circlearrowleft}} \ll \rho_0$), $S_{!\boldsymbol{\circlearrowleft}} \approx \frac{k_B}{\rho_0},\rho_{!\boldsymbol{\circlearrowleft}}$ up to an additive constant, so entropy gradients are essentially proportional to density gradients. Now, recall that in thermodynamic interpretations of gravity, a local entropy gradient $\nabla S$ in the presence of a temperature $T$ gives rise to an emergent force $F = T,\nabla S$ \cite{Padmanabhan2010,Bekenstein1973}. In SST’s fluid picture, the vortex-induced pressure deficit plays the role of an effective potential generating acceleration. We can equate the two pictures by identifying the gradient of $-\frac{1}{2}\rho_{\text{core}}v_{!\boldsymbol{\circlearrowleft}}^2$ with $T,\nabla S_{!\boldsymbol{\circlearrowleft}}$. Indeed, combining the earlier relations:

\begin{align*}

\Delta p ;&=; -,\frac{1}{2}\rho_{\text{core}},v_{!\boldsymbol{\circlearrowleft}}^2~, \

F_{\text{swirl}} &;=; -\nabla \Delta p ;\propto; \nabla!\big(\rho_{\text{core}},v_{!\boldsymbol{\circlearrowleft}}^2\big) ;\propto; \nabla S_{!\boldsymbol{\circlearrowleft}}~,

\end{align*}

one finds that \textbf{swirl-induced attraction is directly analogous to an entropic force}. In other words, regions of faster swirl (lower pressure, higher “swirl entropy”) pull things in, much as regions of high information density produce emergent gravity in Verlinde’s scenario\href{https://www.academia.edu/144408705/Long_Distance_Swirl_Gravity_from_Chiral_Swirling_Knots_with_Central_Holes#:~:text=%2811%29%20In%20SST%2C%20circulation,3%20Mass%20as}{academia.edu}\href{https://www.academia.edu/144408705/Long_Distance_Swirl_Gravity_from_Chiral_Swirling_Knots_with_Central_Holes#:~:text=order%20knots%20encode%20distinct%20energy,Savart%20dynamics}{academia.edu}. SST provides a concrete fluid microstructure underpinning this entropy: $S_{!\boldsymbol{\circlearrowleft}}$ counts the twisting of vortex flux lines, and its gradient is literally the spatial variation in swirl density. This insight, first elucidated by Iskandarani \cite{SSTCanon2025}, places SST’s explanation of gravity on a thermodynamic footing consistent with (but more physical than) other emergent gravity approaches \cite{Jacobson1995}.


\subsection{Time Dilation from Swirl Flows}

An object that resides in a deep swirl-induced pressure well experiences not only forces but also time dilation akin to gravitational redshift. SST incorporates a \emph{Chronos field} $T(x)$ (sometimes called the “swirl clock” field) that measures absolute aether time\href{https://www.academia.edu/144408705/Long_Distance_Swirl_Gravity_from_Chiral_Swirling_Knots_with_Central_Holes#:~:text=Topological%20Gravity%2C%20Entropic%20Force%2C%20Circulation,i%CE%A8%20in%20a%20simply%20connected}{academia.edu}\href{https://www.academia.edu/144408705/Long_Distance_Swirl_Gravity_from_Chiral_Swirling_Knots_with_Central_Holes#:~:text=circulation%20quantum%20%CE%BA,%CE%BA%20Status%2Flimits}{academia.edu}. Moving fluid or strong swirl reduces the rate of proper time for processes within that region relative to the far field (where the aether is stationary). Quantitatively, one can derive a time dilation relation from the energy of the swirling flow. For a single isolated swirl string with core tangential speed $v_{!\boldsymbol{\circlearrowleft}}$, the local time flow $d\tau$ is slowed according to

dτdt∞≈1−v⁣↺2c2,\frac{d\tau}{dt_{\infty}} \;\approx\; \sqrt{\,1 - \frac{v_{\!\boldsymbol{\circlearrowleft}}^2}{c^2}\,}~,dt∞dτ≈1−c2v↺2,

where $dt_{\infty}$ is time measured far from the vortex (where the aether is undisturbed) and $d\tau$ is local proper time in the vortex core frame. This relation resembles the special relativistic time dilation for a rotating frame. In the case of composite swirls, the effect is amplified. For example, a baryon (proton or neutron) consists of three knotted swirl sub-cores (“quark” vortices) merged into one tube. If each quark core has $v_c \approx 1.09\times 10^6~\text{m/s}$ (a typical vortex speed in this model), the combined core roughly triples the rim speed to $3v_c$ (since circulation adds)\href{https://www.academia.edu/144408705/Long_Distance_Swirl_Gravity_from_Chiral_Swirling_Knots_with_Central_Holes#:~:text=Kelvin%C5%A0s%20theorem%20,a%20single%20tube%20along%20%2Bz}{academia.edu}\href{https://www.academia.edu/144408705/Long_Distance_Swirl_Gravity_from_Chiral_Swirling_Knots_with_Central_Holes#:~:text=with%20reff%20%E2%89%88%20rc%20to,range%20attraction.%20This%20is}{academia.edu}. Plugging this into the time dilation formula:

\begin{equation*}

\frac{d\tau_{\text{baryon}}}{dt_{\infty}} ;=; \sqrt{,1 - \frac{(3v_c)^2}{c^2},}~,

\end{equation*}

one predicts a more pronounced local time slowdown for a proton’s core than for a single isolated swirl. Although $3v_c \ll c$ in practice (so the effect is small, on the order of $10^{-4}$ or less), this difference is conceptually important: \textbf{more massive swirl configurations exhibit slightly greater time dilation}, consistent with their deeper effective potential wells. In SST, what we call rest mass energy and gravitational redshift both stem from the kinetic energy of swirling aether. Thus, a heavier particle (with more circulation or more knotted substructure) not only has a larger gravitational pull but also a slower internal clock. This offers a physical intuition for why composite particles like baryons have larger rest masses (and associated slower decay rates for processes) than lighter single-vortex leptons, paralleling general relativity’s gravitational time dilation in a completely flat-space, fluid context\href{https://www.academia.edu/144408705/Long_Distance_Swirl_Gravity_from_Chiral_Swirling_Knots_with_Central_Holes#:~:text=with%20reff%20%E2%89%88%20rc%20to,range%20attraction.%20This%20is}{academia.edu}\href{https://www.academia.edu/144408705/Long_Distance_Swirl_Gravity_from_Chiral_Swirling_Knots_with_Central_Holes#:~:text=time%20dilation%2C%20consistent%20with%20the,connected%20by%20the%20same%20central}{academia.edu}.


\section{Topology-Dependent Modulation of Gravity}

A remarkable aspect of SST is that gravity can be \emph{polarity-dependent}. The strength and even sign of the induced force can change with the \textbf{chirality} (handedness) and linking of the swirl structures. Chiral swirls (knots with nonzero helicity) produce the attractive behavior discussed above. However, an \emph{achiral} configuration – one that is topologically the mirror image of itself (e.g., the figure-eight knot, which has zero net helicity) – does not generate the same kind of pressure well. In fact, achiral vortex knots tend to have a symmetry that prevents sustained pressure deficits; they do not \emph{synchronize} with the surrounding swirl phase and effectively float in the aether. Preliminary analyses suggest that such structures experience \textbf{negligible attraction and can even be repelled} from regions of space populated by normal (chiral) matter\href{https://www.academia.edu/144408705/Long_Distance_Swirl_Gravity_from_Chiral_Swirling_Knots_with_Central_Holes#:~:text=of%20these%20knotted%20structures,a%20topological%20origin%20for%20cosmic}{academia.edu}\href{https://www.academia.edu/144408705/Long_Distance_Swirl_Gravity_from_Chiral_Swirling_Knots_with_Central_Holes#:~:text=dynamically%20repelled%20from%20high,offers%20a%20compelling%20alternative%20to}{academia.edu}. In cosmological terms, this offers a fascinating possible explanation for the observed cosmic acceleration (usually attributed to dark energy). The SST reasoning is: the universe’s matter forms a chiral swirl network (e.g., knotted vortices in galaxies), which exerts an exclusion pressure on any residual achiral defects in the aether. Those achiral fluid elements are pushed outward, creating a small effective repulsive force on large scales \cite{AchiralExpulsion2025}. The calculated magnitude of this repulsion in SST’s toy models is below the currently observed dark energy density \cite{AchiralExpulsion2025}, but the qualitative trend is intriguing. As more achiral voids are expelled, the effect on cosmic expansion would diminish, potentially offering a transient acceleration mechanism. While these ideas are speculative, they underscore the core theme that \textbf{gravity in SST is a manifestly \emph{tunable}, topology-dependent phenomenon}. By rearranging the swirl configuration (for instance, twisting two vortex loops in opposite directions or altering a knot’s crossing number), one could in principle modulate the effective gravitational interaction – ranging from stronger attraction to a complete cancellation or even repulsion.


Another form of gravitational modulation comes from \textbf{dynamically changing the swirl}. If the circulation $\Gamma$ of a vortex string were somehow increased or decreased over time (say by merging vortices or letting one decay), the depth of the pressure well would change correspondingly, altering the gravitational force it exerts. In classical gravity, varying a mass would change the gravitational field smoothly; in SST, varying a swirl’s circulation or core size does the same for the emergent gravity. However, because circulation in a superfluid is quantized and conserved in a continuous process, such changes typically occur through discrete topological transitions (like vortex reconnections or knot topology changes). SST predicts \emph{unique signatures} of such processes, as we discuss next.


\subsection{Electromotive Signals from Swirl Topology Changes}

SST merges the concepts of gravity and electromagnetism via what is dubbed the \textbf{Swirl-Electromagnetic Bridge} \cite{SSTEMF2025}. In the canonical formulation, Maxwell–Faraday equations acquire additional source terms due to swirling aether flows. Specifically, a time-varying swirl density $\partial_t \rho_{!\boldsymbol{\circlearrowleft}}$ acts as a source of an \emph{effective electric field} curl, much like a changing magnetic field. The modified Faraday’s law in the presence of swirl is \cite{SSTEMF2025}:

\begin{equation*}

\nabla \times \mathbf{E} ;=; -,\frac{\partial \mathbf{B}}{\partial t} ;-; \mathbf{b}\textit{{!\boldsymbol{\circlearrowleft}}~, \qquad\text{where}\qquad

\mathbf{b}}{!\boldsymbol{\circlearrowleft}} ;\equiv; G_{!\boldsymbol{\circlearrowleft}},\frac{\partial \boldsymbol{\rho}\textit{{!\boldsymbol{\circlearrowleft}}}{\partial t}~.

\end{equation*}

Here $G}{!\boldsymbol{\circlearrowleft}}$ is a new universal constant (the \emph{swirl coupling constant}) that quantifies how efficiently changes in swirl density induce electric fields\href{https://www.academia.edu/144408705/Long_Distance_Swirl_Gravity_from_Chiral_Swirling_Knots_with_Central_Holes#:~:text=4,canonically%20normalized%20to%20a%20%C4%86ux}{academia.edu}\href{https://www.academia.edu/144408705/Long_Distance_Swirl_Gravity_from_Chiral_Swirling_Knots_with_Central_Holes#:~:text=and%20swirl%C5%B0gravity%20effects%20unify%20into,Integrating%20over%20a}{academia.edu}. In essence, whenever a swirl configuration changes with time, it will emit an \textbf{electromotive impulse}. These impulses are invariant in magnitude, $\Delta \Phi = \pm \Phi^\ast$, determined by fundamental constants (and the sign depends on the handedness of the topology change)\href{https://www.academia.edu/144408705/Long_Distance_Swirl_Gravity_from_Chiral_Swirling_Knots_with_Central_Holes#:~:text=equal,1%3A%20%28a}{academia.edu}\href{https://www.academia.edu/144408705/Long_Distance_Swirl_Gravity_from_Chiral_Swirling_Knots_with_Central_Holes#:~:text=crisp%20laboratory%20discriminator%20follows%20from,of%20radius%20rc%20supports%20a}{academia.edu}. The prediction is striking: if one were to, say, snap a vortex ring or cause two knotted vortices to exchange partners (changing the linking number by an integer), a sudden electrical signal would be produced in the surrounding space. This is a testable consequence of SST’s unification: gravitational events (changes in the effective mass distribution due to topology) have direct electromagnetic signatures.


Such electromotive signals provide a clear experimental handle on gravitational modulation. A “crisp laboratory discriminator” proposed by Iskandarani is to create controlled vortex topology changes in a superfluid or electromagnetic analog system and measure the induced voltage pulses \cite{SSTHelicity2025}. Because these impulses have a fixed quantized strength (related to $G_{!\boldsymbol{\circlearrowleft}}$ and the quantum of circulation), detecting them would confirm that the coupling between swirl dynamics and electromagnetism is real and that gravity can be manipulated via fluid dynamical means. It would also falsify the theory if the signals did \emph{not} appear as predicted. Notably, these signals are \emph{geometry-independent}: unlike electromagnetic induction in loops (which depends on area/orientation), the SST-induced impulse is topological and does not depend on the geometric details of the detection loop beyond linking with the vortex event\href{https://www.academia.edu/144408705/Long_Distance_Swirl_Gravity_from_Chiral_Swirling_Knots_with_Central_Holes#:~:text=equal,stevedore}{academia.edu}\href{https://www.academia.edu/144408705/Long_Distance_Swirl_Gravity_from_Chiral_Swirling_Knots_with_Central_Holes#:~:text=crisp%20laboratory%20discriminator%20follows%20from,of%20radius%20rc%20supports%20a}{academia.edu}. This robustness stems from the fact that the underlying cause is a change in linking number (an integer) of the swirl field with the detection loop. Either the loop links the changing vortex or it doesn’t, leading to a full impulse $\Phi^\ast$ or no impulse at all.


The connection between swirl dynamics and Maxwell’s equations also implies that one can, in principle, \textbf{drive gravitational changes using electromagnetic means}. For example, oscillating a charged object might perturb the aether and modulate a nearby vortex, or vice versa. While SST’s development in this direction is ongoing, the fundamental coupling $G_{!\boldsymbol{\circlearrowleft}}$ is set such that a unit change in swirl flux corresponds to a single flux quantum of electromagnetism (in SI units, $\Phi_0 = h/2e$ for superconductors, analogously $\Phi^\ast$ for swirl). This ensures consistency with known quantum induction phenomena while opening a door to \emph{controlling gravity by engineering electromagnetic environments} that influence swirl configurations.


\section{Conclusion and Outlook}

Swirl-String Theory offers a radically new perspective on gravity: one where \textbf{gravity is not a fixed geometry but a fluid-derived, adjustable force}. In this framework, the presence or absence of gravity, its strength, and even its sign are governed by the state of aetheric swirl flows. We have seen that a rotating superfluid filament (swirl string) naturally generates an attractive force via its pressure deficit, reproducing key features of Newtonian and relativistic gravity (at least in the weak-field limit) without invoking spacetime curvature. Moreover, by changing the circulation, number, or topology of these filaments, one can modulate the gravitational interaction. Two vortices merging deepen a shared potential well and enhance attraction; conversely, oppositely oriented or achiral vortices might cancel out attraction or produce a mild repulsion. This malleability suggests intriguing possibilities: for instance, developing \emph{“gravity control”} technologies by manipulating vortex configurations in laboratory superfluids or plasmas. While such applications remain speculative, the fundamental science points to testable differences from standard gravity. The predicted quantized electromotive impulses from topological transitions are a prime example — a window into “seeing” gravity in a new light via electrical measurements.


Crucially, SST remains consistent with known physics while extending it. In the regime of static, stable configurations, swirl gravity should mimic classical gravity (a $1/r^2$ attraction at distances large compared to $r_c$) with an appropriate effective coupling constant identified with Newton’s $G$. Indeed, the SST canon defines a swirl analog of Coulomb’s law for gravity, with $\Lambda$ playing the role of $4\pi G$ in the Poisson equation for the pressure field \cite{SSTCanon2025}. But beyond this static equivalence, SST provides richer insights: it integrates gravitational phenomena with quantum concepts (via circulation quantization and information-theoretic entropy) and with electromagnetic phenomena (via the swirl–EM bridge). It suggests that what we call a “gravitational wave” could simply be a vortex oscillation that inevitably carries an electromagnetic footprint. It also recasts cosmic mysteries like dark energy and dark matter in terms of fluid topology (e.g., achiral exclusions and vortex network structure) rather than exotic new fields, offering novel explanations that can be explored with both astrophysical observations and table-top fluid experiments.


In summary, gravitational modulation in Swirl-String Theory is not science fiction, but a logical consequence of its fluid-unified view of nature. Gravity emerges from swirl, and by mastering swirl we master gravity. Ongoing and future work will undoubtedly refine the quantitative predictions of SST (for example, deriving the precise gravitational constant from fluid microparameters, or computing the spectrum of allowed vortex-knot masses and comparing to the Standard Model). Experimental tests, such as measuring the proposed electromotive pulses or observing time dilation in superfluid analogs, will be critical. If validated, SST could herald a paradigm shift where controlling the “weather” of the quantum vacuum becomes the way we generate or nullify gravitational fields. At the very least, this theory provides a profound reminder that our classical separations between forces may be illusory — nature might be, at its heart, a single liquid tapestry of swirling, knotted flux from which space, time, matter, and gravity jointly emerge.


\begin{thebibliography}{99}

\bibitem{SSTCanon2025} O.~Iskandarani, \emph{Swirl-String Theory (SST) Canon v0.5.10}, 2025. (Core SST definitions, constants, and master equations.)

\bibitem{VAMReform2025} O.~Iskandarani, \emph{Quantum Mechanics and Quantum Gravity in the Vortex Aether Model: A Reformulation using Superfluid Vorticity and Topology}, 2025.

\bibitem{London1930} F.~London, Zur Theorie und Systematik der Molekularkräfte,'' \emph{Z.~Phys.} \textbf{63}, 245 (1930). \bibitem{Casimir1948} H.~B.~G.~Casimir and D.~Polder, The influence of retardation on the London–van der Waals forces,'' \emph{Phys.~Rev.} \textbf{73}(4), 360 (1948).

\bibitem{SSTHelicity2025} O.~Iskandarani, \emph{Long-Distance Swirl Gravity from Chiral Swirling Knots with Central Holes}, 2025.

\bibitem{Verlinde2011} E.~P.~Verlinde, On the origin of gravity and the laws of Newton,'' \emph{JHEP} \textbf{2011}(4), 29 (2011). \bibitem{Verlinde2017} E.~P.~Verlinde, Emergent gravity and the dark universe,'' \emph{SciPost Phys.} \textbf{2}, 016 (2017).

\bibitem{Jacobson1995} T.~Jacobson, Thermodynamics of spacetime: The Einstein equation of state,'' \emph{Phys.~Rev.~Lett.} \textbf{75}(7), 1260 (1995). \bibitem{Padmanabhan2010} T.~Padmanabhan, Thermodynamical aspects of gravity: New insights,'' \emph{Rep.~Prog.~Phys.} \textbf{73}(4), 046901 (2010).

\bibitem{Bekenstein1973} J.~D.~Bekenstein, Black holes and entropy,'' \emph{Phys.~Rev.~D} \textbf{7}(8), 2333 (1973). \bibitem{SSTEMF2025} O.~Iskandarani, Rotating-frame Unification in the SST Canon: From Swirl Density to Swirl-EMF, and a Canonical Derivation of $G_{!\circlearrowleft}$,'' 2025.

\bibitem{AchiralExpulsion2025} O.~Iskandarani, \emph{Milky Way as a Chiral Swirl Network – Exclusion of Achiral Knots (Dark Energy Implications)}, 2025.

\end{thebibliography}


\end{document}



%=========================================
% References
%=========================================
        \bibliographystyle{unsrt}
        \begin{thebibliography}{99}

            \bibitem{Einstein1905} A.~Einstein, \newblock \emph{Ist die Tr\"agheit eines K\"orpers von seinem Energieinhalt
            abh\"angig?}, newblock Ann.\ Phys.\ \textbf{18}, 639--641 (1905).

        \end{thebibliography}

\end{document}