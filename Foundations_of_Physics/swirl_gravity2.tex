
%! Author = Omar Iskandarani
%! Date = 11/17/2025
%! Affiliation = Independent Researcher, Groningen, The Netherlands
%! License = © 2025 Omar Iskandarani. All rights reserved. This manuscript is made available for academic reading and citation only. No republication, redistribution, or derivative works are permitted without explicit written permission from the author. Contact: info@omariskandarani.com
%! ORCID = 0009-0006-1686-3961
%! DOI = 10.5281/zenodo.xxx

\newcommand{\paperdoi}{10.5281/zenodo.xxx}
\newcommand{\papertitle}{Gravitational Modulation in Swirl–String Theory}

\documentclass[11pt]{article}
\usepackage{amsmath,amssymb,amsfonts,bm}
\usepackage{siunitx}
\usepackage[hidelinks]{hyperref}
\usepackage[a4paper,margin=1in]{geometry}
\usepackage[T1]{fontenc}
\usepackage[utf8]{inputenc}


% Convenience macros
\newcommand{\vcirc}{v_{\!\circlearrowleft}}
\newcommand{\rcore}{r_{\!c}}
\newcommand{\rhof}{\rho_{\!f}}
\newcommand{\rhocore}{\rho_{\text{core}}}
\newcommand{\kappaQ}{\kappa}
\newcommand{\Gswirl}{G_{\!\circlearrowleft}}
\newcommand{\LambdaS}{\Lambda}

% Title page helpers
\newcommand{\titlepageOpen}{
    \begin{titlepage}
    \thispagestyle{empty}
    \centering
    \Large \bfseries \papertitle \par \vspace{1cm}
    {\Large \itshape \textbf{Omar Iskandarani}\textsuperscript{\textbf{*}} \par}
    \vspace{0.5cm}
    {\today \par}
    \vspace{0.5cm}
}
\newcommand{\titlepageClose}{
    \vfill \raggedright \null
    \begin{picture}(0,0)
    \put(0,-45){
        \begin{minipage}[b]{0.7\textwidth} \footnotesize
        \renewcommand{\arraystretch}{1.0}
        \noindent\rule{\textwidth}{0.4pt} \\[0.5em]
        \textsuperscript{\textbf{*}} Independent Researcher, Groningen, The Netherlands \\
        Email: \texttt{info@omariskandarani.com} \\
        ORCID: \texttt{\href{https://orcid.org/0009-0006-1686-3961}{0009-0006-1686-3961}} \\
        DOI: \href{https://doi.org/\paperdoi}{\paperdoi}
        \end{minipage}
    }
    \end{picture}
    \end{titlepage}
}

\begin{document}
    \titlepageOpen

    \begin{abstract}
        \noindent
        Swirl–String Theory (SST) recasts matter as quantized vortex filaments in an inviscid background medium.
        Within this picture, gravitational attraction emerges from pressure deficits generated by swirling cores rather than spacetime curvature.
        This article develops a clear, testable notion of \emph{gravitational modulation}: adjusting perceived gravity by altering circulation, alignment, or topology of swirl structures.
        We (i) summarize the SST constants and the softened ``swirl Coulomb'' potential, (ii) show how additive or opposing circulation strengthens or weakens attraction, (iii) describe anisotropy via alignment, phasing, and bulk rotation, and (iv) outline predicted electromagnetic co-signatures of topological changes.
        We close with feasibility bounds and near-term analog experiments.
    \end{abstract}

    \titlepageClose

    \section{Introduction}
        Swirl–String Theory (SST) models elementary constituents as knotted or linked vortex filaments in a flat, incompressible medium.
        Mass and gravity, in this view, arise from the dynamics of the swirl rather than from curved spacetime.
        \emph{Gravitational modulation} then means altering the effective strength, sign, or directionality of this attraction by changing swirl parameters (circulation, density, orientation) or topology.
        Here we collect the minimal equations, lay out control levers consistent with SST, and emphasize falsifiable predictions.

    \section{SST Fundamentals and the Vortex–Gravity Mechanism}

        \subsection{Core structure and quantized circulation}
            Each particle is idealized as a vortex tube with core radius $\rcore \sim 10^{-15}\,\mathrm{m}$ and tangential core speed $\vcirc \approx 1.09\times 10^6\,\mathrm{m\,s^{-1}}$.
            The surrounding fluid has very low mass density $\rhof$, while the core stores a large mass-equivalent energy density $\rhocore$.
            Circulation is quantized:
            \begin{equation}
                \kappaQ \equiv 2\pi \rcore\, \vcirc,
            \end{equation}
            and Kelvin’s theorem keeps $\kappaQ$ fixed for material loops.
            Composite structures add circulation (e.g., three sub-cores in a baryon-like tube $\Rightarrow \Gamma=3\kappaQ$), deepening the associated pressure deficit and increasing effective rest mass.

        \subsection{Pressure deficit and emergent attraction}
            In steady, inviscid flow the Euler equation gives $\rho\,\mathbf{a} = -\nabla p$.
            Swirl creates a core pressure drop $\Delta p \simeq -\tfrac{1}{2}\rhocore \vcirc^2$, so
            \begin{equation}
                \mathbf{g}_{\text{swirl}} \;\approx\; -\,\nabla\!\left(\frac{\Delta p}{\rhocore}\right) \;=\; \nabla\!\left(\frac{\vcirc^2}{2}\right),
            \end{equation}
            directed toward the core.
            At distances $r\gg \rcore$, the long-range behavior is captured by a softened potential
            \begin{equation}
                V_{\text{SST}}(r) \simeq -\frac{\LambdaS}{r^2+\rcore^2}, \qquad
                |{\bf a}_g| \approx \frac{2\LambdaS}{r^3},
            \end{equation}
            with $\LambdaS$ set by core parameters $(\rhocore,\vcirc,\rcore)$ and calibrated so that the emergent coupling matches $G_N$ for ordinary conditions.

        \subsection{Clock effects from swirl}
            The same kinetic origin yields mild time dilation in deep pressure wells.
            To leading order, a local clock rate scales as
            \begin{equation}
                \frac{d\tau}{dt_\infty} \approx \sqrt{\,1-\frac{\vcirc^2}{c^2}\,},
            \end{equation}
            and increases in composite tubes where effective rim speeds grow with additive circulation.
            Although small for canonical $\vcirc \ll c$, the direction is consistent: deeper wells $\Rightarrow$ slower internal clocks.

    \section{Topology–Dependent Modulation}
        Because linking and chirality determine how pressure wells combine, gravity in SST is inherently topology sensitive.
        Aligned, like–handed structures \emph{add} circulation and strengthen attraction; carefully phased or opposite–handed structures can partially \emph{cancel} linking and weaken it.
        Achiral configurations (net helicity near zero) are predicted to couple weakly to the ambient swirl network, suggesting reduced attraction and, in special arrangements, effective repulsion.
        These ideas remain speculative but are natural within SST’s topological bookkeeping.

    \section{Directional Control and Anisotropy}
        Three levers enable anisotropy.
        \begin{itemize}
            \item \textbf{Alignment and focusing:} Coaxial alignment concentrates attraction along the common axis (a weak ``gravitational flashlight'').
            \item \textbf{Phased oscillations:} Small coherent oscillations of multiple sources interfere, enhancing on-axis and canceling off-axis responses (phased-array logic at tiny amplitudes).
            \item \textbf{Bulk rotation:} Rapid rotation injects kinetic pressure that slightly reduces equatorial attraction relative to polar directions (SST cousin of frame dragging).
        \end{itemize}
        Define an effective $G(\theta)$ relative to a chosen axis; in natural settings $G(\theta)$ is nearly isotropic, but engineered arrays could imprint a minute angular dependence.

    \section{Electromotive Co–Signatures of Topology Change}
        SST links swirl dynamics to electromotive effects: time–varying swirl density or changes in linking number act as sources of EMF.
        Topological transitions (creation, annihilation, reconnection) should emit discrete voltage impulses of fixed magnitude $\Delta\Phi=\pm\Phi_\ast$, set by universal constants and the handedness of the change.
        This prediction is a clean discriminator: gravity modulation via swirl that \emph{lacks} the EM impulse would falsify the mechanism.

    \section{Feasibility and Near–Term Tests}
        Energetic thresholds are steep: macroscopic changes to $g$ require enormous $\LambdaS_{\text{eff}}$ and quickly confront a maximum-force bound.
        Nevertheless, table-top analogs in superfluids or condensates can probe the qualitative levers:
        co-rotating vs.\ counter-rotating vortex interactions, engineered alignment, and searches for quantized EM impulses during vortex reconnections.

    \section{Conclusion}
        Within SST, ``controlling gravity'' reduces to controlling circulation and topology in an inviscid medium.
        Additive circulation strengthens attraction; opposing or achiral arrangements weaken it; alignment and phasing introduce directionality; and any topological switch should co-emit a quantized EM signal.
        The framework is energetically constrained yet experimentally falsifiable, with near-term probes available in fluid analog platforms.

        \bigskip
        \noindent\textbf{Notes on sources.} Constants and constructs (e.g.\ $\rcore$, $\vcirc$, $\rhof$, $\rhocore$, $\LambdaS$, maximum-force bounds) follow the SST canon and related notes on topological gravity and swirl–EM coupling.

        \bigskip
        \noindent\rule{\textwidth}{0.4pt}

        \appendix
        %========================================================
% SST GRAVITY MODULATION CHEAT-SHEET (CANON INSERT)
%========================================================

% --- SST macro prelude (local to this block) ---
        \newcommand{\rhoE}{\rho_{\!E}}
        \newcommand{\rhoM}{\rho_{\!m}}
        \newcommand{\vswirl}{\mathbf{v}_{\!\boldsymbol{\circlearrowleft}}}
        \newcommand{\vnorm}{\lVert \mathbf{v}_{\!\boldsymbol{\circlearrowleft}}\rVert}
        \newcommand{\rc}{r_c}

        \subsection{SST gravity modulation: summary}

            In Swirl--String Theory (SST), gravity is not a fundamental geometric field but an emergent pressure well of the swirl medium. The local swirl energy density
            \begin{equation}
                \rhoE \;=\; \frac{1}{2}\,\rhof\,\vnorm^{2},
                \qquad
                \rhoM \;=\; \frac{\rhoE}{c^{2}}
            \end{equation}
            acts as an effective source in a Newtonian/weak-field limit,
            \begin{equation}
                \nabla^{2} \Phi \;\approx\; 4\pi\,G_{\text{swirl}}\,\rhoM,
                \label{eq:SST-Poisson}
            \end{equation}
            with $G_{\text{swirl}}$ fixed by Canon so that $G_{\text{swirl}}\simeq G_{N}$ in ordinary regimes. Near a core, the swirl-induced pressure deficit
            \begin{equation}
                \Delta p \;\approx\; -\,\frac{1}{2}\,\rhocore\,\vnorm^{2}
            \end{equation}
            yields a ``gravitational'' acceleration
            \begin{equation}
                \mathbf{g}_{\text{SST}}
                \;=\; -\frac{1}{\rhocore}\,\nabla p
                \;\approx\; \nabla\!\left(\frac{\vnorm^{2}}{2}\right),
            \end{equation}
            so that strengthening gravity corresponds to increasing $\rhoE$ or steepening $\nabla\vnorm^{2}$, while weakening or inverting it corresponds to shallowing or overcompensating this pressure well (e.g.\ via opposite-chirality swirl).

            For a roughly uniform energy density $\rhoE$ in a spherical region of radius $R$, the induced surface acceleration is
            \begin{equation}
                g(R) \;\approx\; \frac{4\pi G_{N}}{3c^{2}}\,\rhoE\,R,
            \end{equation}
            so the energy density required to generate or cancel a target increment $g_{0}$ is
            \begin{equation}
                \rhoE \;\approx\; \frac{3c^{2}\,g_{0}}{4\pi G_{N}\,R}.
                \label{eq:rhoE-threshold}
            \end{equation}
            Even for $g_{0}\sim 0.01\,g$ and $R\sim1~\mathrm{m}$, Eq.~\eqref{eq:rhoE-threshold} gives $\rhoE\sim 10^{25}\,\mathrm{J/m^{3}}$ ($\rhoM\sim 10^{8}\,\mathrm{kg/m^{3}}$), i.e.\ white-dwarf-scale densities. This encodes the core Canon conclusion: \emph{any macroscopic gravity modulation in SST requires astrophysical energy densities in the swirl medium} and is therefore extremely constrained.

            Directional and topological effects enter as mild corrections. If internal swirl defines a symmetry axis, the effective coupling can be written
            \begin{equation}
                G_{\text{eff}}(\theta)
                \;=\; G_{\text{swirl}}\,f(\theta),
                \qquad
                f(\theta) \;=\; 1 + \delta f(\theta),
                \quad |\delta f(\theta)| \ll 1,
            \end{equation}
            with $f(\theta)$ encoding how much linking circulation $\Gamma_{\text{link}}$ is shared along angle $\theta$ relative to the swirl axis. Topological ``shielding'' or cancellation corresponds schematically to
            \begin{equation}
                \Gamma_{\text{link}}\neq 0
                \;\Rightarrow\;
                \Phi(r)\sim -\frac{\Lambda}{r^{2}+\rc^{2}},
                \qquad
                \Gamma_{\text{link}}\approx 0
                \;\Rightarrow\;
                \Phi(r)\approx 0 \quad (r\gg\rc),
            \end{equation}
            i.e.\ arranging swirl so that external loops see vanishing net linking (no long-range attraction). Such configurations are highly energetic and metastable, bounded above by the canonical maximum gravitational force $F_{\text{gr}}^{\max}$.

            Finally, any genuine SST gravity modulation via topology change is \emph{not} silent: the modified Faraday law
            \begin{equation}
                \nabla\times\mathbf{E}
                \;=\;
                -\,\frac{\partial\mathbf{B}}{\partial t}
                \;-\;
                \mathbf{b}_{\!\boldsymbol{\circlearrowleft}},
                \qquad
                \mathbf{b}_{\!\boldsymbol{\circlearrowleft}}
                \;=\;
                G_{\!\boldsymbol{\circlearrowleft}}\,
                \frac{\partial\boldsymbol{\rho}_{\!\boldsymbol{\circlearrowleft}}}{\partial t}
            \end{equation}
            implies, upon loop integration,
            \begin{equation}
                \oint \mathbf{E}\!\cdot d\boldsymbol{\ell}
                \;=\;
                -\,\frac{d\Phi_{B}}{dt}
                \;-\;
                \frac{d\Phi_{\!\boldsymbol{\circlearrowleft}}}{dt},
            \end{equation}
            and a discrete topology change in the swirl field yields a quantized jump
            \begin{equation}
                \Delta\Phi_{\!\boldsymbol{\circlearrowleft}}
                \;=\;
                \pm\,\Phi_{\ast}.
            \end{equation}
            Canon implication: \textbf{any nontrivial SST gravity control via swirl topology must be accompanied by a quantized electromotive impulse $\pm\Phi_{\ast}$}. Absence of such EM signatures rules out claimed gravity modulation within the SST framework.

% (Optional local bibliography if needed)
% \begin{thebibliography}{9}
% \bibitem{Jackson1999} J.~D.~Jackson, \emph{Classical Electrodynamics}, 3rd ed., Wiley (1999).
% \end{thebibliography}

        \bibliographystyle{unsrt}
        \begin{thebibliography}{99}
            \bibitem{SSTCanon2025} O.~Iskandarani, \emph{Swirl–String Theory (SST) Canon v0.5.10}, 2025.
            \bibitem{VAMReform2025} O.~Iskandarani, \emph{Quantum Mechanics and Quantum Gravity in the Vortex Aether Model}, 2025.
            \bibitem{London1930} F.~London, Zur Theorie und Systematik der Molekularkräfte, \emph{Z.~Phys.} \textbf{63}, 245 (1930).
            \bibitem{Casimir1948} H.~B.~G.~Casimir \& D.~Polder, The influence of retardation on the London–van der Waals forces, \emph{Phys.~Rev.} \textbf{73}, 360 (1948).
            \bibitem{SSTHelicity2025} O.~Iskandarani, \emph{Long–Distance Swirl Gravity from Chiral Swirling Knots with Central Holes}, 2025.
            \bibitem{Verlinde2011} E.~Verlinde, On the origin of gravity and the laws of Newton, \emph{JHEP} \textbf{2011}(4), 29 (2011).
            \bibitem{Verlinde2017} E.~Verlinde, Emergent gravity and the dark universe, \emph{SciPost Phys.} \textbf{2}, 016 (2017).
            \bibitem{Jacobson1995} T.~Jacobson, Thermodynamics of spacetime: The Einstein equation of state, \emph{Phys.~Rev.~Lett.} \textbf{75}, 1260 (1995).
            \bibitem{Padmanabhan2010} T.~Padmanabhan, Thermodynamical aspects of gravity: New insights, \emph{Rep.~Prog.~Phys.} \textbf{73}, 046901 (2010).
            \bibitem{Bekenstein1973} J.~D.~Bekenstein, Black holes and entropy, \emph{Phys.~Rev.~D} \textbf{7}, 2333 (1973).
            \bibitem{SSTEMF2025} O.~Iskandarani, Rotating–frame unification in the SST canon: From swirl density to swirl–EMF, 2025.
            \bibitem{AchiralExpulsion2025} O.~Iskandarani, \emph{Milky Way as a Chiral Swirl Network: Exclusion of Achiral Knots}, 2025.
        \end{thebibliography}

\end{document}