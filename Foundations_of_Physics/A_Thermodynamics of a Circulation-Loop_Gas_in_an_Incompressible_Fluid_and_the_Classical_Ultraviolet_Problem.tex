%! Author = Omar Iskandarani
%! Date = Nov 16, 2025
%! Affiliation = Independent Researcher, Groningen, The Netherlands
%! License = © 2025 Omar Iskandarani. All rights reserved. This manuscript is made available for academic reading and citation only. No republication, redistribution, or derivative works are permitted without explicit written permission from the author. Contact: info@omariskandarani.com
%! ORCID = 0009-0006-1686-3961
%! DOI = 10.5281/zenodo.xxx

\newcommand{\paperdoi}{10.5281/zenodo.xxx}
\newcommand{\papertitle}{Thermodynamics of a Circulation-Loop Gas in an Incompressible Fluid and the Classical Ultraviolet Problem}

%==========================================================
% PACKAGES AND DOCUMENT CONFIGURATION
%==========================================================
\documentclass[11pt]{article}

\usepackage{amsmath,amssymb,amsfonts,bm}
\usepackage{siunitx}
\usepackage[hidelinks]{hyperref}
\usepackage[a4paper,margin=1in]{geometry}
\usepackage[T1]{fontenc}
\usepackage[utf8]{inputenc}

\newcommand{\titlepageOpen}{
    \begin{titlepage}
        \thispagestyle{empty}
        \centering
        \Large \bfseries \papertitle \par \vspace{1cm}
        {\Large \itshape \textbf{Omar Iskandarani}\textsuperscript{\textbf{*}} \par}
        \vspace{0.5cm}
        {\today \par}
        \vspace{0.5cm}
}

% here comes abstract
\newcommand{\titlepageClose}{
        \vfill \raggedright \null
        \begin{picture}(0,0)
            \put(0,-45){  % Shift 200pt left, 40pt down
                \begin{minipage}[b]{0.7\textwidth} \footnotesize
                    \renewcommand{\arraystretch}{1.0}
                    \noindent\rule{\textwidth}{0.4pt} \\[0.5em]
                    \textsuperscript{\textbf{*}} Independent Researcher, Groningen, The Netherlands \\
                    Email: \texttt{info@omariskandarani.com} \\
                    ORCID: \texttt{\href{https://orcid.org/0009-0006-1686-3961}{0009-0006-1686-3961}} \\
                    DOI: \href{https://doi.org/\paperdoi}{\paperdoi}
                \end{minipage}
            }
        \end{picture}
    \end{titlepage}
}

\begin{document}
    \titlepageOpen

    \begin{abstract}
        We investigate a classical analogue model for thermal radiation based on an
        incompressible, inviscid fluid populated by thin circulation loops (vortex
        rings) of finite core radius. Treating closed circulation loops as localized,
        finite-energy structures with effective mass $M_{\mathrm{eff}}$, we construct a
        kinetic theory of a dilute loop gas in the canonical ensemble. The resulting
        partition function coincides with that of a classical ideal gas, and the
        equation of state $P V = N k_{\mathrm B} T$ follows in the standard way from
        the Helmholtz free energy. For an isotropic ensemble of ultra-relativistic or
        effectively massless excitations built from these loops, the relation
        $P = \tfrac{1}{3}u$ between pressure and energy density is recovered from the
        momentum flux tensor, relying only on three-dimensional isotropy and the
        relativistic relation $E = pc$.

        A finite core radius $a$ and an upper bound $v_{\max}$ on the tangential speed
        inside the cores imply a kinematic maximum internal frequency
        $\omega_{\max} = v_{\max}/a$. If electromagnetic normal modes are identified
        with mechanically realizable excitations of this medium, only modes with
        $\omega \leq \omega_{\max}$ are thermally accessible. The classical
        Rayleigh--Jeans integral is then automatically regulated, yielding a finite
        energy density $u \propto k_{\mathrm B}T\,\omega_{\max}^3$ without any
        modification of Maxwell's equations. The resulting cutoff Rayleigh--Jeans
        spectrum does not reproduce the $T^4$ scaling of the Planck law and therefore
        does not replace quantum statistics. Instead, it provides an explicit
        mechanical example in which microstructure and finite rotational frequencies
        eliminate the classical ultraviolet catastrophe at the level of microstate
        counting, while leaving intact the standard macroscopic equations of motion and
        equations of state.
    \end{abstract}


    \titlepageClose


    \section{Introduction}

        Classical thermodynamics and statistical mechanics provide a remarkably successful macroscopic
        description of fluids and radiation. The equation of state of an ideal gas,
        \begin{equation}
            P V = N k_{\mathrm B} T,
        \end{equation}
        and the relation between pressure and energy density for an isotropic radiation field,
        \begin{equation}
            P = \frac{1}{3} u(T),
        \end{equation}
        are textbook results derived from kinetic arguments and from the stress--energy tensor of
        relativistic fields~\cite{LandauLifshitz_StatPhys,Reif,LandauLifshitz_Fields}.
        At the same time, the classical Rayleigh--Jeans treatment of thermal radiation predicts
        $u(T) \propto T\,\omega_{\max}^2$, diverging as the high-frequency cutoff $\omega_{\max}\to\infty$.
        The resolution of this ``ultraviolet catastrophe'' by Planck's quantization of oscillator
        energies is one of the historical starting points of quantum theory~\cite{Planck1901,Rayleigh1900,Jeans1905}.

        In parallel with these developments, classical vortex dynamics has established that
        incompressible, inviscid fluids support localized vortical structures---thin vortex rings and
        more general closed vortex loops---with finite energy, impulse, and self-induced translational
        velocity~\cite{Batchelor1967,Saffman1992}. In recent work, the present author has summarized
        standard results for the energy and impulse of thin vortex loops and emphasized their
        ``particle-like'' attributes in purely classical Euler flow~\cite{Iskandarani2025_VortexLoops}.
        A closed filament of circulation $\Gamma$ and core size $a$ carries a finite kinetic energy
        $E(\Gamma,R,a)$, an impulse $\boldsymbol{I}$, and an effective mass
        $M_{\mathrm{eff}} = \|\boldsymbol{I}\|/U$, where $U$ is the self-induced velocity along the loop axis.
        The existence of such finite-energy, localized structures invites the question of how far
        one can push a classical ``vortex-particle'' picture within standard continuum mechanics.

        A second, complementary observation is that kinetic energy in extended media contributes to
        inertia and gravitational mass through the relativistic relation $E = mc^2$.
        While this statement is encoded in the stress--energy tensor of relativistic fluids,
        explicit examples in simple incompressible configurations remain pedagogically useful.
        In Ref.~\cite{Iskandarani2025_RotKE}, an incompressible, inviscid fluid undergoing rigid-body
        rotation in a finite cylinder was analyzed, and the volume-averaged rotational kinetic energy
        density $\langle e_{\mathrm{kin}}\rangle$ was related to an effective mass density
        $\Delta \rho_{\mathrm{eff}} = \langle e_{\mathrm{kin}}\rangle /c^2$ in the nonrelativistic limit.
        For that geometry one finds the compact result
        \begin{equation}
            \frac{\Delta \rho_{\mathrm{eff}}}{\rho}
            = \frac{1}{4}\left(\frac{v_{\mathrm{edge}}}{c}\right)^2,
        \end{equation}
        where $v_{\mathrm{edge}}$ is the tangential speed at the cylinder boundary and $\rho$ is the
        rest-mass density. This example shows, within standard special relativity, how rotational
        motion modifies the effective mass density of an incompressible medium at order $(v/c)^2$.

        The purpose of the present paper is to combine these two strands---finite-energy vortex loops
        and effective mass density from rotational kinetic energy---and to explore the thermodynamics
        of a dilute gas of vortex loops in an incompressible, inviscid fluid.
        Our analysis is entirely classical and remains within the framework of Eulerian fluid mechanics,
        special relativity, and textbook statistical mechanics. We show that:

        \begin{enumerate}
            \item A gas of weakly interacting, randomly oriented vortex loops with fixed circulation
            and core size admits a kinetic description in which the usual ideal-gas equation of
            state $PV = N k_{\mathrm B} T$ is recovered from the momentum flux, without any
            modification of Boltzmann's constant or of the definition of temperature.

            \item For an isotropic ensemble of ultra-relativistic or effectively massless excitations
            built from such loops, the standard relation $P = \tfrac{1}{3}u$ follows from the
            three-dimensional structure of the momentum flux tensor, in direct analogy with
            the radiation case.

            \item The mechanical spectrum of admissible loop configurations is effectively bounded from
            above in frequency by the combination of finite core radius, finite circulation, and
            a characteristic swirl speed, so that the classical Rayleigh--Jeans integral for the
            energy density remains finite when evaluated over this mechanically constrained mode
            set. No modification of Maxwell's equations or of the relativistic field equations is
            assumed; the regularization arises from the kinematics of the vortex degrees of freedom.
        \end{enumerate}

        From a thermodynamic standpoint, the main result is thus that a purely classical,
        incompressible fluid with vortex-loop microstructure can reproduce the familiar
        pressure--volume--temperature relations and the $P = \tfrac{1}{3}u$ relation for an isotropic
        radiation-like gas, while avoiding the ultraviolet divergence of the naive continuum mode
        counting. From a conceptual standpoint, the construction provides a concrete mechanical model
        in which the energy, pressure, and effective mass associated with internal rotational motion
        can be treated on the same footing as in relativistic continuum mechanics, without invoking
        any nonstandard dynamics or postulates beyond those already present in classical fluid
        mechanics and special relativity.

    \section{Vortex-loop gas and kinetic theory}
        \label{sec:vortex-gas}

        In this section we set up a minimal kinetic model for a dilute gas of vortex loops
        in an incompressible, inviscid fluid and derive its thermodynamic equation of state.
        The construction follows standard kinetic theory, with the only nonstandard ingredient
        being the identification of the microscopic ``particles'' with finite-energy vortex
        loops in Euler flow.

        \subsection{Vortex-loop kinematics and effective mass}

            We consider a homogeneous, incompressible, inviscid fluid of density $\rho_f$,
            described by the Euler equations
            \begin{equation}
                \frac{\partial \boldsymbol{v}}{\partial t}
                + (\boldsymbol{v}\cdot\nabla)\boldsymbol{v}
                = -\frac{1}{\rho_f}\nabla p,
                \qquad
                \nabla\cdot\boldsymbol{v} = 0,
                \label{eq:Euler}
            \end{equation}
            where $\boldsymbol{v}(\boldsymbol{x},t)$ is the velocity field and $p(\boldsymbol{x},t)$
            the pressure. Vorticity is defined as
            \begin{equation}
                \boldsymbol{\omega} = \nabla\times\boldsymbol{v}.
            \end{equation}
            We assume that vorticity is confined to thin tubes (vortex filaments) of core radius $a$,
            within which the tangential speed is bounded by a characteristic value $v_0$, and that the
            flow outside the cores is irrotational.

            Classical vortex dynamics shows that a closed vortex filament of circulation $\Gamma$
            and large-scale radius $R \gg a$ has finite kinetic energy $E(\Gamma,R,a)$, finite
            impulse $\boldsymbol{I}$, and a self-induced translational velocity $U$ along its symmetry
            axis~\cite{Batchelor1967,Saffman1992,Iskandarani2025_VortexLoops}. It is natural to define
            an effective mass
            \begin{equation}
                M_{\mathrm{eff}} \equiv \frac{\|\boldsymbol{I}\|}{U},
                \label{eq:Meff-def}
            \end{equation}
            so that the loop behaves kinematically like a particle of mass $M_{\mathrm{eff}}$ moving
            with velocity $U$ along its axis. In addition to this translational motion, the fluid
            inside and near the core carries internal rotational kinetic energy. In the nonrelativistic
            regime, the contribution of this internal energy to the effective mass density can be
            expressed, following Ref.~\cite{Iskandarani2025_RotKE}, as
            \begin{equation}
                \Delta \rho_{\mathrm{eff}} = \frac{\langle e_{\mathrm{kin}}\rangle}{c^2},
                \label{eq:Delta-rho-eff}
            \end{equation}
            where $\langle e_{\mathrm{kin}}\rangle$ is the volume-averaged rotational kinetic energy
            density and $c$ is the speed of light.

            In what follows, we treat each vortex loop as a structure with fixed circulation and
            core size, whose internal degrees of freedom are either frozen or accounted for as a
            fixed contribution to $M_{\mathrm{eff}}$. The centre-of-mass motion of the loops then
            admits a standard kinetic description.

        \subsection{Phase space and Hamiltonian}

            We consider $N$ well-separated vortex loops in a container of volume $V$.
            Let $\boldsymbol{x}_i$ and $\boldsymbol{p}_i$ denote the centre-of-mass position and
            momentum of the $i$-th loop ($i=1,\dots,N$). In the dilute limit we neglect mutual
            induction and other interactions between loops, so that the total Hamiltonian factorizes:
            \begin{equation}
                H = \sum_{i=1}^{N} \frac{\boldsymbol{p}_i^2}{2M_{\mathrm{eff}}}.
                \label{eq:H-nonrel}
            \end{equation}
            The phase space is the usual $6N$-dimensional space of coordinates and momenta,
            \begin{equation}
                \Gamma = \left\{ (\boldsymbol{x}_1,\dots,\boldsymbol{x}_N;\,
                             \boldsymbol{p}_1,\dots,\boldsymbol{p}_N) \right\},
            \end{equation}
            endowed with the Liouville measure
            \begin{equation}
                \mathrm{d}\mu(\Gamma)
                = \prod_{i=1}^{N} \mathrm{d}^3x_i\,\mathrm{d}^3p_i.
            \end{equation}
            Importantly, the incompressibility constraint $\nabla\cdot\boldsymbol{v}=0$ is imposed
            at the level of the underlying Euler flow and affects the detailed form of
            $M_{\mathrm{eff}}$ and the internal energy of each loop. It does not modify the
            Liouville measure or the Hamiltonian structure for the centre-of-mass coordinates,
            which remain those of a dilute gas of classical particles.

        \subsection{Canonical ensemble and partition function}

            We now place the vortex-loop gas in contact with a heat bath at temperature $T$ and
            describe it within the canonical ensemble. The canonical partition function is
            \begin{equation}
                Z_N(T,V)
                = \frac{1}{N!\,h^{3N}}
                \int \exp[-\beta H]\,
                \mathrm{d}^{3N}p\,\mathrm{d}^{3N}x,
                \qquad \beta = \frac{1}{k_{\mathrm B}T},
                \label{eq:ZN-def}
            \end{equation}
            where $h$ is Planck's constant and $k_{\mathrm B}$ Boltzmann's constant.
            Inserting the Hamiltonian~\eqref{eq:H-nonrel},
            \begin{equation}
                Z_N(T,V)
                = \frac{1}{N!\,h^{3N}}
                \prod_{i=1}^{N}
                \left[
                    \int_{V} \mathrm{d}^3x_i
                    \int_{\mathbb{R}^3} \exp\!\left(
                                                  -\beta\frac{\boldsymbol{p}_i^2}{2M_{\mathrm{eff}}}
                \right)\mathrm{d}^3p_i
                \right].
            \end{equation}
            The position integrals each yield a factor of $V$, and the momentum integrals are
            Gaussian:
            \begin{equation}
                \int_{\mathbb{R}^3}
                \exp\!\left(-\beta\frac{\boldsymbol{p}^2}{2M_{\mathrm{eff}}}\right)
                \mathrm{d}^3p
                = \left( 2\pi M_{\mathrm{eff}} k_{\mathrm B}T \right)^{3/2}.
            \end{equation}
            Hence
            \begin{equation}
                Z_N(T,V)
                = \frac{1}{N!}
                \left[
                    \frac{V}{\lambda_T^3}
                \right]^{\!N},
                \qquad
                \lambda_T
                = \frac{h}{\sqrt{2\pi M_{\mathrm{eff}} k_{\mathrm B}T}},
                \label{eq:ZN-final}
            \end{equation}
            where $\lambda_T$ is the thermal de Broglie wavelength associated with $M_{\mathrm{eff}}$.
            This is the standard partition function of a classical ideal gas with mass
            $M_{\mathrm{eff}}$; the vortex nature of the underlying structures is encoded entirely
            in the value of $M_{\mathrm{eff}}$ and in possible internal-state degeneracies, which
            we neglect here.

        \subsection{Ideal-gas equation of state}

            The Helmholtz free energy $F(T,V,N)$ is
            \begin{equation}
                F(T,V,N)
                = -k_{\mathrm B}T \ln Z_N(T,V)
                = -k_{\mathrm B}T
                \left[
                    -\ln N! + N\ln V - 3N\ln\lambda_T
                \right].
            \end{equation}
            The pressure is obtained in the usual way as
            \begin{equation}
                P
                = -\left(\frac{\partial F}{\partial V}\right)_{T,N}
                = k_{\mathrm B}T
                \left(\frac{\partial \ln Z_N}{\partial V}\right)_{T,N}
                = k_{\mathrm B}T\,\frac{N}{V}.
                \label{eq:ideal-gas}
            \end{equation}
            Thus the dilute vortex-loop gas obeys the ideal-gas law
            \begin{equation}
                P V = N k_{\mathrm B} T,
            \end{equation}
            with no modification of the equation of state due to the incompressibility of the underlying
            fluid. The incompressibility affects the value of $M_{\mathrm{eff}}$ through the
            kinetic energy of the core and the surrounding flow, but once $M_{\mathrm{eff}}$ is fixed,
            the centre-of-mass dynamics yields the same $P$--$V$--$T$ relation as for point particles.

        \subsection{Isotropic momentum flux and $P = \tfrac{1}{3}u$}

            We now turn to the relation between pressure and energy density for an isotropic ensemble
            of excitations. Consider a gas of particles (here, effective vortex-loop quasiparticles)
            with energies $E$ and momenta $\boldsymbol{p}$, moving with speed $v = |\boldsymbol{v}|$.
            In kinetic theory, the pressure on a plane normal to the $x$-axis is given by the
            momentum flux in the $x$-direction,
            \begin{equation}
                P
                = \frac{1}{V}
                \left\langle \sum_{i} p_{x,i} v_{x,i} \right\rangle,
            \end{equation}
            where the average is over all particles in the volume $V$ and $p_{x,i}$, $v_{x,i}$ are
            the $x$-components of the momentum and velocity of particle $i$.
            For an isotropic distribution, one can write
            \begin{equation}
                p_{x} v_{x} = (\boldsymbol{p}\cdot\boldsymbol{v})\cos^2\theta,
            \end{equation}
            where $\theta$ is the angle between $\boldsymbol{p}$ and the $x$-axis. In three dimensions
            the angular average of $\cos^2\theta$ over the unit sphere is
            \begin{equation}
                \langle \cos^2\theta \rangle
                = \frac{1}{4\pi}\int_0^{2\pi}\!\mathrm{d}\phi
                \int_0^\pi \cos^2\theta\,\sin\theta\,\mathrm{d}\theta
                = \frac{1}{3}.
            \end{equation}
            Thus,
            \begin{equation}
                P
                = \frac{1}{3V}
                \left\langle \sum_{i} \boldsymbol{p}_i\cdot\boldsymbol{v}_i \right\rangle.
            \end{equation}
            In the ultra-relativistic or effectively massless limit, one has $E_i = c\|\boldsymbol{p}_i\|$
            and $\|\boldsymbol{v}_i\|=c$, so that $\boldsymbol{p}_i\cdot\boldsymbol{v}_i = E_i$.
            The pressure therefore becomes
            \begin{equation}
                P
                = \frac{1}{3V}
                \left\langle \sum_{i} E_i \right\rangle
                = \frac{1}{3} u,
                \label{eq:P-u-1over3}
            \end{equation}
            where
            \begin{equation}
                u
                = \frac{1}{V}\left\langle \sum_{i} E_i \right\rangle
            \end{equation}
            is the energy density. This is the familiar relation for an isotropic gas of massless
            particles or photons~\cite{LandauLifshitz_StatPhys,LandauLifshitz_Fields}.
            In the present context, it applies to any regime in which the relevant vortex-loop
            excitations propagate at speeds close to $c$ and their energy is dominated by
            translational motion rather than rest energy.

            The crucial point is that Eq.~\eqref{eq:P-u-1over3} follows entirely from the
            three-dimensional isotropy of the momentum distribution and the kinematics of
            relativistic particles. It does not depend on the detailed microstructure of the
            underlying medium. The vortex-loop picture provides a concrete classical realization
            of such excitations within incompressible Euler flow; the relation $P = \tfrac{1}{3}u$
            then follows from standard kinetic arguments applied to these effective degrees of freedom.

            \medskip

            In summary, a dilute gas of vortex loops in an incompressible, inviscid fluid admits
            a standard kinetic description in which the ideal-gas law and the radiation-like
            relation $P = \tfrac{1}{3}u$ emerge in precisely the same way as for point particles
            and photons in conventional kinetic theory. The incompressibility of the underlying
            fluid constrains the internal structure and effective mass of the loops, but does not
            alter the macroscopic $P$--$V$--$T$ relations obtained from their centre-of-mass motion
            and isotropic momentum flux.


    \section{Mechanically allowed mode set and the classical ultraviolet problem}
        \label{sec:modes-UV}

        The derivation of the Rayleigh--Jeans law in a cavity rests on two ingredients:
        (i) the normal-mode structure of Maxwell's equations in a finite volume, and
        (ii) the assumption that each normal mode behaves as an independent harmonic oscillator
        carrying an average energy $k_{\mathrm B}T$ in the classical limit.
        The number of electromagnetic modes with angular frequency between $\omega$ and
        $\omega+\mathrm{d}\omega$ in a volume $V$ is~\cite{LandauLifshitz_Fields}
        \begin{equation}
            g_{\mathrm{EM}}(\omega)\,\mathrm{d}\omega
            = \frac{V}{\pi^2 c^3}\,\omega^2\,\mathrm{d}\omega,
            \label{eq:g-EM}
        \end{equation}
        so that the classical spectral energy density is
        \begin{equation}
            u_{\mathrm{RJ}}(\omega,T)\,\mathrm{d}\omega
            = \frac{\omega^2}{\pi^2 c^3} k_{\mathrm B}T\,\mathrm{d}\omega.
            \label{eq:u-RJ}
        \end{equation}
        The total energy density $u(T) = \int_0^\infty u_{\mathrm{RJ}}(\omega,T)\,\mathrm{d}\omega$
        diverges as $\omega^3$ at the upper limit, giving the well-known ultraviolet catastrophe
        in the purely classical field picture~\cite{Rayleigh1900,Jeans1905,Planck1901}.

        In the present framework, the electromagnetic field is regarded as a macroscopic,
        irrotational manifestation of the motion of an underlying incompressible fluid with
        circulation-loop microstructure. Maxwell's equations and their cavity eigenmodes are
        left intact at the macroscopic level. What changes is the interpretation of the
        \emph{microstates} underlying a given field configuration: instead of independent
        field oscillators, the microstates are mechanical configurations of circulation loops.
        We now show that the kinematics of these loops restricts the set of physically
        realisable high-frequency modes, effectively regularizing the Rayleigh--Jeans
        integral without any modification of Maxwell's equations.

        \subsection{Maximal rotational frequency from core size and tangential speed}

            As in Sec.~\ref{sec:vortex-gas}, we consider thin circulation loops (vortex rings)
            of core radius $a$ embedded in a homogeneous incompressible fluid of density $\rho_f$.
            Inside the core, the tangential speed $v_\theta(r)$ of the fluid elements is bounded
            by a characteristic value $v_0$, so that
            \begin{equation}
                v_\theta(r) \leq v_0 \quad \text{for } 0 \leq r \leq a.
            \end{equation}
            The associated angular frequency at radius $r$ is
            \begin{equation}
                \Omega(r) = \frac{v_\theta(r)}{r},
            \end{equation}
            so that the maximal angular frequency attainable within the core is bounded by
            \begin{equation}
                \Omega_{\max}
                = \max_{0<r\leq a} \frac{v_\theta(r)}{r}
                \lesssim \frac{v_0}{a}.
                \label{eq:Omega-max}
            \end{equation}
            This bound is purely kinematic: it expresses the fact that no fluid element can
            circulate around the core with tangential speed exceeding $v_0$ at a radius smaller
            than $a$. In particular, it implies a minimal time scale
            \begin{equation}
                \tau_{\min} \sim \frac{1}{\Omega_{\max}} \gtrsim \frac{a}{v_0},
            \end{equation}
            below which the fluid cannot respond to external forcing or internal excitations
            without violating the speed bound.

            In the present context, we will denote by
            \begin{equation}
                \omega_{\max} \equiv \Omega_{\max}
            \end{equation}
            the characteristic maximal angular frequency associated with internal rotational
            motion in the circulation-loop cores.
            This frequency provides a mechanical upper bound on how rapidly the substrate
            can be made to oscillate locally. Any putative field mode with $\omega \gg \omega_{\max}$
            would require fluid motions inside the cores with periods shorter than $\tau_{\min}$
            and hence tangential speeds above $v_0$, which are excluded by construction.

        \subsection{Mechanically realizable field modes}

            Maxwell's equations in vacuum (or in a homogeneous dielectric) admit harmonic
            solutions with arbitrarily large wave number $k$ and frequency
            $\omega = c k$.
            In the purely field-theoretic picture, nothing prevents one from populating modes
            with $k \to \infty$. In the present mechanical picture, however, such field
            configurations must be realizable as coarse-grained manifestations of the motion
            of the underlying fluid.

            We will not attempt to derive Maxwell's equations from the fluid model.
            Instead, we make the following conservative, kinematic assumption:

            \begin{quote}
                \emph{For a given cavity geometry and boundary conditions, we restrict attention
                to those electromagnetic normal modes whose spatial and temporal variations can
                be generated by configurations of circulation loops whose internal rotational
                frequencies do not exceed $\omega_{\max}$.}
            \end{quote}

            Operationally, this means that among the continuum of formal cavity eigenmodes,
            only those with angular frequency $\omega$ up to some effective cutoff
            \begin{equation}
                \omega \lesssim \omega_{\max}
                \label{eq:omega-cut}
            \end{equation}
            can actually be excited in thermal equilibrium with the mechanical substrate.
            Modes with $\omega \gg \omega_{\max}$ exist as mathematical solutions of the field
            equations, but there is no mechanical microstate in the circulation-loop gas that
            corresponds to their excitation at finite amplitude; they are therefore excluded
            from the thermodynamic counting of microstates.

            Equivalently, one may say that the \emph{effective} density of thermally accessible
            modes is modified from Eq.~\eqref{eq:g-EM} to
            \begin{equation}
                g_{\mathrm{eff}}(\omega)
                = g_{\mathrm{EM}}(\omega)\,f\!\left(\frac{\omega}{\omega_{\max}}\right),
                \label{eq:g-eff}
            \end{equation}
            where $f(x)$ is a dimensionless cutoff function satisfying
            \begin{equation}
                f(x) \to 1 \quad (x\ll 1),
                \qquad
                f(x) \to 0 \quad (x\gg 1),
                \label{eq:f-asymptotics}
            \end{equation}
            and decaying sufficiently rapidly as $x\to\infty$ to render the total energy finite.
            The detailed form of $f$ depends on the microphysics of the circulation-loop gas
            and on how electromagnetic excitations couple to it; for the present discussion,
            only the existence of such an $f$ with the stated asymptotics is required.

        \subsection{Modified Rayleigh--Jeans spectrum and finiteness of energy}

            With the mechanically constrained mode set~\eqref{eq:g-eff}, the classical
            Rayleigh--Jeans spectral energy density becomes
            \begin{equation}
                u_{\mathrm{eff}}(\omega,T)\,\mathrm{d}\omega
                = \frac{\omega^2}{\pi^2 c^3} k_{\mathrm B}T\,
                f\!\left(\frac{\omega}{\omega_{\max}}\right)\mathrm{d}\omega.
                \label{eq:u-eff}
            \end{equation}
            The total energy density is
            \begin{equation}
                u(T)
                = \int_0^\infty u_{\mathrm{eff}}(\omega,T)\,\mathrm{d}\omega
                = \frac{k_{\mathrm B}T}{\pi^2 c^3}
                \int_0^\infty \omega^2
                f\!\left(\frac{\omega}{\omega_{\max}}\right)\mathrm{d}\omega.
                \label{eq:u-T-integral}
            \end{equation}
            By changing variables $x = \omega/\omega_{\max}$, this becomes
            \begin{equation}
                u(T)
                = \frac{k_{\mathrm B}T}{\pi^2 c^3}\,\omega_{\max}^3
                \int_0^\infty x^2 f(x)\,\mathrm{d}x.
            \end{equation}
            Under the mild assumption that the integral
            \begin{equation}
                C_f \equiv \int_0^\infty x^2 f(x)\,\mathrm{d}x
            \end{equation}
            converges (which follows from the decay of $f$ as $x\to\infty$), we obtain
            \begin{equation}
                u(T)
                = \frac{k_{\mathrm B}T}{\pi^2 c^3}\,\omega_{\max}^3 C_f,
                \label{eq:u-T-finite}
            \end{equation}
            which is finite for all finite $T$ and scales linearly with temperature in the
            regime where equipartition holds. The ultraviolet divergence of the naive
            Rayleigh--Jeans integral is thus removed by the mechanical restriction of the
            mode set; the formal field-theoretic density of states~\eqref{eq:g-EM} is
            replaced, for thermodynamic purposes, by the effective density~\eqref{eq:g-eff}
            that accounts for the finite response time and finite core size of the substrate.

            As a simple illustration, consider the sharp-cutoff choice
            \begin{equation}
                f(x) =
                \begin{cases}
                    1, & 0 \leq x \leq 1,\\[3pt]
                    0, & x > 1,
                \end{cases}
            \end{equation}
            which encodes the assumption that modes with $\omega \leq \omega_{\max}$ are fully
            accessible, while those with $\omega > \omega_{\max}$ are inaccessible.
            In that case,
            \begin{equation}
                C_f = \int_0^1 x^2\,\mathrm{d}x = \frac{1}{3},
            \end{equation}
            and Eq.~\eqref{eq:u-T-finite} reduces to
            \begin{equation}
                u(T)
                = \frac{k_{\mathrm B}T}{3\pi^2 c^3}\,\omega_{\max}^3.
                \label{eq:u-RJ-cut-1}
            \end{equation}
            This is precisely the Rayleigh--Jeans result with a hard frequency cutoff
            at $\omega_{\max}$, now justified mechanically by the finite core size and maximal
            tangential speed of the circulation loops. For more realistic, smooth cutoff
            functions $f(x)$ the prefactor $C_f$ changes, but the qualitative conclusion
            remains: the energy density is finite and scales as $\omega_{\max}^3$.

        \subsection{Maxwell's equations and the role of the substrate}

            It is important to emphasize what has and has not been modified in this construction.
            At the macroscopic level:

            \begin{itemize}
                \item The form of Maxwell's equations and their normal-mode solutions in a cavity
                are left unchanged. The usual mode density~\eqref{eq:g-EM} remains valid
                as a statement about the spectrum of solutions to the field equations.

                \item The relations $P V = N k_{\mathrm B}T$ and $P = \tfrac{1}{3}u$ for the
                effective quasiparticles (Sec.~\ref{sec:vortex-gas}) are derived from
                standard kinetic theory and from the isotropy of the momentum distribution,
                without any modification of the field equations.
            \end{itemize}

            What \emph{does} change is the set of microstates that are admitted when one asks
            how a given electromagnetic field configuration is realized mechanically.
            In the usual classical treatment, each normal mode is assigned an independent
            harmonic-oscillator degree of freedom and hence contributes $k_{\mathrm B}T$ to
            the energy in the classical limit. In the present fluid-based model, the
            independent microstates are not field amplitudes but configurations of circulation
            loops obeying the kinematic constraints of incompressible Euler flow. These
            constraints imply a finite maximal rotational frequency $\omega_{\max}$ and
            thereby restrict the subset of cavity modes that can be populated in thermal
            equilibrium with the substrate.

            In this sense, the incompressible fluid with circulation-loop microstructure
            acts as a \emph{mechanical regulator} of the ultraviolet behavior of classical
            radiation. The Rayleigh--Jeans divergence does not arise, not because the
            field equations are altered, but because the assumption of an infinite set of
            independent harmonic oscillators is replaced by a finite-density set of
            mechanically realizable excitations whose internal time scales cannot be
            shorter than $1/\omega_{\max}$.

            From the thermodynamic point of view, the key result of this section is that
            the ultraviolet behavior of the classical radiation spectrum is rendered finite
            once the microstate space is defined in terms of circulation loops with finite
            core radius and bounded tangential speed. The subsequent sections will explore
            how this mechanical cutoff scale compares with known microscopic scales in
            high-energy physics and what, if any, observable consequences might arise in
            astrophysical or cosmological contexts.

    \section{Microphysical scale and numerical estimates}
        \label{sec:micro-scale}

        In Sec.~\ref{sec:modes-UV} we treated the electromagnetic field, in a purely kinematic
        sense, as being represented by a gas of mechanically admissible modes built from closed
        circulation loops (thin vortex rings) in an incompressible, inviscid medium. Each loop
        was assumed to have a finite core radius $a$ and a bounded tangential speed along the
        core. In this section we translate these mechanical assumptions into (i) an upper bound
        on the angular frequencies that can be realized by such loops and (ii) the corresponding
        ultraviolet cutoff for the Rayleigh--Jeans integral. Maxwell's equations and their
        normal modes are not modified; the change enters only through the identification of
        which modes are physically realizable given the microstructure.

        \subsection{Kinematic upper bound on angular frequency}

            Consider a single closed circulation loop with core radius $a$ embedded in an
            incompressible, inviscid fluid of density $\rho_f$. Let $r$ denote the local radius of
            curvature of the loop centreline, and let $v_\theta(r)$ be the local tangential speed of
            the circulating fluid inside the core. Locally, the motion is equivalent to rigid-body
            rotation in a small disc of radius $O(a)$, so that the angular frequency associated with
            this local rotation is
            \begin{equation}
                \omega_{\text{loc}}(r) = \frac{v_\theta(r)}{r}.
            \end{equation}
            We impose two mechanical conditions:
            \begin{enumerate}
                \item \emph{Finite core radius:}
                \begin{equation}
                    r \ge a,
                \end{equation}
                since the curvature radius of the centreline cannot be smaller than the core
                radius without leaving the thin-core regime.
                \item \emph{Bounded tangential speed:}
                \begin{equation}
                    v_\theta(r) \le v_{\max},
                \end{equation}
                where $v_{\max}$ is a characteristic upper bound, e.g.\ set by relativistic
                constraints $v_{\max}<c$ or by the stiffness of the underlying medium.
            \end{enumerate}
            Combining these two inequalities gives an upper bound on the local rotational frequency,
            \begin{equation}
                \omega_{\text{loc}}(r) = \frac{v_\theta(r)}{r}
                \le \frac{v_{\max}}{a}.
            \end{equation}
            Since any normal mode of the loop gas is built from superpositions of such local
            rotational degrees of freedom, it is natural to define the mechanically.allowed maximal
            angular frequency
            \begin{equation}
                \omega_{\max} := \frac{v_{\max}}{a}.
                \label{eq:omega-max-v-a}
            \end{equation}
            Equation~\eqref{eq:omega-max-v-a} is a purely kinematic consequence of finite core radius
            and bounded tangential speed in incompressible Euler flow; no assumption about
            electromagnetism has been used at this stage.

            For later comparison it is helpful to note that a closely related frequency scale already
            appears as a kinematic reference in rotating-flow benchmarks, where one defines
            \begin{equation}
                \Omega_0 \equiv \frac{C_e}{r_c}, \qquad
                E_0 = \hbar \Omega_0,
                \label{eq:Omega0-def}
            \end{equation}
            with $C_e$ a characteristic speed and $r_c$ a microscopic length scale.
            In the present notation, $\omega_{\max}$ plays the same structural role as $\Omega_0$
            if one identifies $v_{\max}\leftrightarrow C_e$ and $a\leftrightarrow r_c$.

        \subsection{Rayleigh--Jeans energy density with a mechanical cutoff}

            In the standard derivation of the Rayleigh--Jeans law, the electromagnetic field in a
            cavity is decomposed into normal modes; the number of modes per unit volume with angular
            frequency between $\omega$ and $\omega+\mathrm{d}\omega$ is
            \begin{equation}
                g(\omega)\,\mathrm{d}\omega
                = \frac{\omega^2}{\pi^2 c^3}\,\mathrm{d}\omega,
            \end{equation}
            where the factor of $2$ for polarization is already included in the prefactor.
            Equipartition assigns an average energy $k_{\mathrm B}T$ per mode, so that the spectral
            energy density in the classical Rayleigh--Jeans limit reads
            \begin{equation}
                u_{\mathrm{RJ}}(\omega,T)
                = g(\omega)\,k_{\mathrm B}T
                = \frac{k_{\mathrm B}T}{\pi^2 c^3}\,\omega^2.
                \label{eq:u-RJ-again}
            \end{equation}
            If all frequencies $0\le\omega<\infty$ are allowed, the total energy density
            \begin{equation}
                u_{\mathrm{tot}}^{(\mathrm{RJ})}(T)
                = \int_0^\infty u_{\mathrm{RJ}}(\omega,T)\,\mathrm{d}\omega
            \end{equation}
            diverges as $\omega^3$ at the upper limit: the classical ultraviolet catastrophe.

            In the circulation-loop picture, the \emph{mechanically realizable} modes are restricted
            by the bound~\eqref{eq:omega-max-v-a}. If one assumes that only modes with
            $\omega\le\omega_{\max}$ can be populated in thermal equilibrium with the substrate, the
            Rayleigh--Jeans spectrum is automatically cut off:
            \begin{equation}
                u_{\mathrm{tot}}^{(\mathrm{RJ,cut})}(T)
                = \int_0^{\omega_{\max}}
                \frac{k_{\mathrm B}T}{\pi^2 c^3}\,\omega^2\,\mathrm{d}\omega
                = \frac{k_{\mathrm B}T}{3\pi^2 c^3}\,\omega_{\max}^3.
                \label{eq:u-RJ-cut}
            \end{equation}
            Substituting Eq.~\eqref{eq:omega-max-v-a} gives
            \begin{equation}
                u_{\mathrm{tot}}^{(\mathrm{RJ,cut})}(T)
                = \frac{k_{\mathrm B}T}{3\pi^2 c^3}
                \left(\frac{v_{\max}}{a}\right)^3.
                \label{eq:u-RJ-cut-va}
            \end{equation}
            A dimensional check confirms that Eq.~\eqref{eq:u-RJ-cut-va} has the correct units:
            $[k_{\mathrm B}T] = \mathrm{J}$ and $[v_{\max}^3/(a^3 c^3)] = \mathrm{m}^{-3}$, so that
            $[u_{\mathrm{tot}}] = \mathrm{J\,m^{-3}}$ as required.

            More generally, one can write the effect of the mechanical constraint in terms of a
            dimensionless cutoff function $f(\omega/\omega_{\max})$,
            \begin{equation}
                u_{\mathrm{eff}}(\omega,T)
                = \frac{k_{\mathrm B}T}{\pi^2 c^3}\,\omega^2\,
                f\!\left(\frac{\omega}{\omega_{\max}}\right),
            \end{equation}
            with $f(x)\to 1$ for $x\ll 1$ and $f(x)\to 0$ for $x\gg 1$. The total energy density is
            then
            \begin{equation}
                u(T)
                = \int_0^\infty u_{\mathrm{eff}}(\omega,T)\,\mathrm{d}\omega
                = \frac{k_{\mathrm B}T}{\pi^2 c^3}\,\omega_{\max}^3
                \int_0^\infty x^2 f(x)\,\mathrm{d}x,
            \end{equation}
            which is finite provided the integral $\int_0^\infty x^2 f(x)\,\mathrm{d}x$ converges.
            The sharp cutoff used in Eq.~\eqref{eq:u-RJ-cut} corresponds to
            $f(x) = 1$ for $0\le x\le 1$ and $f(x)=0$ for $x>1$, giving
            \begin{equation}
                \int_0^\infty x^2 f(x)\,\mathrm{d}x
                = \int_0^1 x^2\,\mathrm{d}x = \frac{1}{3},
            \end{equation}
            and reproducing Eq.~\eqref{eq:u-RJ-cut-va}.

        \subsection{Numerical scales and relation to a kinematic reference frequency}

            The expression~\eqref{eq:u-RJ-cut-va} becomes quantitatively meaningful once concrete
            values for $v_{\max}$ and $a$ are specified. In the absence of a fully microscopic
            model, we keep these parameters symbolic but note that a natural identification is
            \begin{equation}
                v_{\max} \sim C_e, \qquad a \sim r_c,
            \end{equation}
            where $C_e$ and $r_c$ are a characteristic speed and length scale that already appear
            in rotating-flow benchmarks, via Eq.~\eqref{eq:Omega0-def}. With this choice,
            \begin{equation}
                \omega_{\max} \equiv \frac{v_{\max}}{a}
                \;\longrightarrow\;
                \Omega_0 = \frac{C_e}{r_c}, \qquad
                E_0 = \hbar\Omega_0.
            \end{equation}
            For illustrative numerical values one may adopt, following Ref.~[Iskandarani],
            \begin{equation}
                C_e \sim 10^6~\mathrm{m\,s^{-1}}, \qquad
                r_c \sim 10^{-15}~\mathrm{m},
            \end{equation}
            so that
            \begin{equation}
                \Omega_0 \sim \frac{10^6}{10^{-15}}~\mathrm{s^{-1}}
                \sim 10^{21}~\mathrm{s^{-1}},
            \end{equation}
            and
            \begin{equation}
                E_0 = \hbar\Omega_0
                \sim 10^{-13}~\mathrm{J}
                \sim 0.5~\mathrm{MeV}.
            \end{equation}
            Numerically, using CODATA values, this is of the same order as the electron rest energy
            $m_e c^2 \approx 0.511~\mathrm{MeV}$; this numerical proximity is used here only as a
            reference point for the scale of $E_0$.

            Substituting $\omega_{\max} = \Omega_0$ into Eq.~\eqref{eq:u-RJ-cut}, one finds
            \begin{equation}
                u_{\mathrm{tot}}^{(\mathrm{RJ,cut})}(T)
                = \frac{k_{\mathrm B}T}{3\pi^2 c^3}\,\Omega_0^3.
            \end{equation}
            For laboratory blackbodies at $T\sim 10^3$--$10^4~\mathrm{K}$, the characteristic
            frequencies contributing appreciably to the Planck spectrum lie far below $\Omega_0$,
            so the presence of such a mechanical cutoff would have no observable effect on
            standard thermal-radiation measurements.

        \subsection{Comparison with Planck's law and observational constraints}

            The exact blackbody spectrum is given by Planck's law,
            \begin{equation}
                u_{\mathrm{Planck}}(\omega,T)
                = \frac{\hbar}{\pi^2 c^3}
                \frac{\omega^3}{\exp(\hbar\omega/k_{\mathrm B}T)-1},
            \end{equation}
            with total energy density
            \begin{equation}
                u_{\mathrm{tot}}^{(\mathrm{Planck})}(T)
                = a_{\mathrm R} T^4, \qquad
                a_{\mathrm R} = \frac{\pi^2 k_{\mathrm B}^4}{15\hbar^3 c^3}.
            \end{equation}
            By contrast, the cutoff Rayleigh--Jeans result~\eqref{eq:u-RJ-cut-va} scales linearly
            with $T$ for fixed $\omega_{\max}$. No choice of $\omega_{\max}$ can reconcile this
            $T$-scaling with the empirically verified $T^4$ scaling over a broad range of
            temperatures. The mechanical cutoff therefore \emph{cannot} replace quantum statistics
            or Planck's law; it merely shows that, in any mechanical substrate with finite core
            radius and bounded rotational speeds, the classical equipartition argument does not
            lead to an ultraviolet divergence at the level of the microstates.

            From an observational standpoint, the absence of any high-frequency cutoff in
            laboratory, nuclear, and astrophysical photon spectra implies a lower bound
            \begin{equation}
                \omega_{\max} = \frac{v_{\max}}{a}
                \gg \omega_{\gamma,\max}^{\text{(obs)}},
            \end{equation}
            where $\omega_{\gamma,\max}^{\text{(obs)}}$ is the highest photon frequency reliably
            measured. This constraint translates into a bound on the ratio $v_{\max}/a$:
            either the core radius $a$ must be extremely small, or the mechanical speed
            scale $v_{\max}$ extremely large, or both.

            In summary, the mechanical mode restriction encoded in Eq.~\eqref{eq:omega-max-v-a}
            yields a natural ultraviolet cutoff $\omega_{\max} = v_{\max}/a$ and a finite
            Rayleigh--Jeans energy density, without altering Maxwell's equations or the
            macroscopic equations of state derived in Sec.~\ref{sec:vortex-gas}. The cutoff
            does not obviate the need for quantization to recover the Planck spectrum and its
            $T^4$ scaling, but it provides a concrete example of how a specific class of
            underlying rotational microphysics regularizes the classical ultraviolet catastrophe
            while remaining compatible, in principle, with standard electrodynamics at accessible
            frequencies.

    \section{Discussion and conclusions}
        \label{sec:discussion}

        \subsection{Summary of results}

            We have considered a classical, incompressible, inviscid fluid in which vorticity is
            confined to thin circulation loops (vortex rings) of finite core radius $a$. Within
            this framework, the main results of the paper are:

            \begin{enumerate}
                \item \emph{Vortex loops as quasiparticles and the ideal-gas law.}
                Treating closed circulation loops as localized, finite-energy structures with
                effective mass $M_{\mathrm{eff}}$ and centre-of-mass momentum $\boldsymbol{p}$,
                we showed that a dilute gas of such loops admits a standard kinetic description.
                In the canonical ensemble, the partition function reduces to that of a classical
                ideal gas, and the equation of state
                \begin{equation}
                    P V = N k_{\mathrm B} T
                \end{equation}
                follows from the usual derivative of the Helmholtz free energy with respect
                to volume. The incompressibility of the underlying medium affects the value of
                $M_{\mathrm{eff}}$ through the kinetic energy of the core and the surrounding
                flow but does not modify the macroscopic $P$--$V$--$T$ relation.

                \item \emph{Isotropic momentum flux and $P = \tfrac{1}{3}u$.}
                For an isotropic ensemble of ultra-relativistic or effectively massless
                excitations built from circulation loops, we recovered the standard relation
                between pressure and energy density,
                \begin{equation}
                    P = \frac{1}{3} u,
                \end{equation}
                by computing the momentum flux on a plane and using the three-dimensional
                angular average $\langle \cos^2\theta \rangle = 1/3$. This derivation depends
                only on the isotropy of the momentum distribution and the relativistic
                relation $E = pc$; it is independent of the detailed microstructure of the
                underlying fluid.

                \item \emph{Mechanical mode restriction and ultraviolet regularization.}
                By imposing a finite core radius $a$ and a bound $v_{\max}$ on the tangential
                speed of fluid elements within the core, we obtained a purely kinematic upper
                bound on the internal rotational frequency,
                \begin{equation}
                    \omega_{\max} = \frac{v_{\max}}{a}.
                \end{equation}
                If electromagnetic normal modes in a cavity are taken to be in one-to-one
                correspondence with mechanically realizable excitations of this medium, then
                only modes with $\omega \leq \omega_{\max}$ can be populated in thermal
                equilibrium. The classical Rayleigh--Jeans integral for the energy density,
                when restricted to this mechanically allowed mode set, becomes
                \begin{equation}
                    u_{\mathrm{tot}}^{(\mathrm{RJ,cut})}(T)
                    = \frac{k_{\mathrm B} T}{3\pi^2 c^3}\,\omega_{\max}^3,
                \end{equation}
                which is finite for all finite temperatures. More generally, introducing a
                smooth cutoff function $f(\omega/\omega_{\max})$ yields
                \begin{equation}
                    u(T)
                    = \frac{k_{\mathrm B}T}{\pi^2 c^3}\,\omega_{\max}^3
                    \int_0^\infty x^2 f(x)\,\mathrm{d}x,
                \end{equation}
                so that the ultraviolet divergence of the naive Rayleigh--Jeans spectrum is
                removed once the microstate space is defined in terms of circulation loops
                with finite core size and bounded rotational speeds.

                \item \emph{Consistency with Maxwell's equations.}
                Throughout, Maxwell's equations and their cavity eigenmodes were left
                unchanged at the macroscopic level. The density of modes
                $g(\omega) = \omega^2/(\pi^2 c^3)$ remains valid as a statement about the
                spectrum of solutions to the field equations. The modification enters only
                through the identification of which modes are physically realizable given the
                mechanical properties of the underlying medium; the ultraviolet behaviour of
                the spectrum is thus regularized at the level of microstate counting rather
                than by altering the field equations.
            \end{enumerate}

            Taken together, these results demonstrate that a classical incompressible medium with
            circulation-loop microstructure can reproduce the familiar ideal-gas and radiation
            equations of state and can regularize the ultraviolet behaviour of the classical
            Rayleigh--Jeans spectrum, entirely within the framework of Eulerian fluid mechanics,
            special relativity, and textbook statistical mechanics.

        \subsection{Relation to classical and quantum descriptions of radiation}

            It is important to distinguish clearly between three levels of description:

            \begin{enumerate}
                \item The \emph{macroscopic field description}, in which the electromagnetic field
                is governed by Maxwell's equations in a given background and admits a continuum
                of normal modes in a cavity, with density
                $g(\omega) = \omega^2/(\pi^2 c^3)$.

                \item The \emph{classical statistical description}, in which each normal mode is
                treated as an independent harmonic oscillator with average energy $k_{\mathrm B}T$
                in the high-temperature limit, leading to the Rayleigh--Jeans spectrum and its
                ultraviolet divergence.

                \item The \emph{quantum statistical description}, in which the energy of each mode
                is quantized in units $\hbar\omega$ and Bose--Einstein statistics yields
                Planck's law and the $T^4$ dependence of the blackbody energy density.
            \end{enumerate}

            The analysis in this paper modifies only the second of these points. We have shown that
            if one insists on assigning a classical mechanical substrate to the electromagnetic
            field---here, an incompressible medium with circulation loops of finite core radius and
            bounded tangential speeds---then the assumption of an infinite set of independent
            harmonic oscillators, each carrying $k_{\mathrm B}T$, is no longer tenable. Instead, the
            set of thermally accessible modes is limited by the kinematics of the medium, and the
            Rayleigh--Jeans integral is rendered finite without altering Maxwell's equations.

            At the same time, the mechanical cutoff introduced by $\omega_{\max} = v_{\max}/a$ does
            \emph{not} reproduce the correct temperature scaling of the blackbody energy density.
            For fixed $\omega_{\max}$, the cutoff Rayleigh--Jeans result scales as
            $u_{\mathrm{tot}}^{(\mathrm{RJ,cut})} \propto T$, whereas Planck's law yields
            $u_{\mathrm{tot}}^{(\mathrm{Planck})} \propto T^4$.\footnote{Here and throughout we
    assume the usual equilibrium photon gas in flat spacetime and neglect chemical
    potentials.} No choice of $\omega_{\max}$ can reconcile these scalings over a broad
            range of temperatures. The mechanical regularization discussed here therefore does not
            remove the need for quantization; it simply illustrates how a particular class of
            underlying microphysics prevents the classical ultraviolet catastrophe from arising at
            the level of the substrate.

            In this sense, the circulation-loop model should be viewed as a \emph{classical analogue}
            or toy model: it exhibits, in a purely mechanical setting, how a finite microstructure
            and a maximum internal frequency can regulate the ultraviolet behaviour of a classical
            field theory, while leaving intact the macroscopic equations of motion and the standard
            kinetic derivations of $PV = N k_{\mathrm B}T$ and $P = \tfrac{1}{3}u$.

        \subsection{Conceptual and pedagogical implications}

            Even if the actual microscopic origin of electromagnetism lies in quantum fields rather
            than classical incompressible media, explicit mechanical models can be conceptually
            useful. The circulation-loop construction offers several such benefits:

            \begin{itemize}
                \item It provides a concrete example in which \emph{internal motion} in an extended
                medium (here, rotational motion within vortex cores) contributes to effective
                mass density via $E = mc^2$. This complements textbook discussions of
                relativistic fluids by exhibiting a specific configuration in which the
                volume-averaged mass density is shifted by $\Delta\rho_{\mathrm{eff}}
                = \langle e_{\mathrm{kin}}\rangle/c^2$.

                \item It illustrates in a tangible way why the naive continuum assumption of
                infinitely many independent modes can be misleading. In the circulation-loop
                gas, the irrotational field outside the cores is not a collection of independent
                oscillators; it is constrained by the finite set of vortex degrees of freedom.
                The ultraviolet divergence of the Rayleigh--Jeans law is therefore seen as an
                artefact of overcounting microstates, rather than as an unavoidable pathology
                of classical physics.

                \item It connects the statistical mechanics of gases with the dynamics of vortical
                structures in fluids: the same circulation loops that serve as ``particles''
                in the kinetic theory also carry the rotational energy that underlies the
                effective mass density and the mechanical cutoff scale.
            \end{itemize}

            For teaching purposes, one might imagine a fluid analogue: a tank of water seeded with
            smoke-ring-like vortices, illuminated and tracked via particle-image velocimetry.
            As smaller and smaller scales are probed, one eventually reaches the minimal core
            size and maximal rotation rate of the rings. No matter how finely one tries to
            decompose the flow into ``modes'', there is a physical limit set by the underlying
            structures. This provides an intuitive picture of how ultraviolet cutoffs can arise
            from microstructure, independent of any quantum postulate.

        \subsection{Possible extensions and future work}

            The present work raises several avenues for further investigation:

            \begin{itemize}
                \item \emph{Detailed microphysical models.}
                We have treated the circulation loops in a schematic way, assuming a fixed
                core radius $a$, a fixed speed bound $v_{\max}$, and a simple relation
                $\omega_{\max} = v_{\max}/a$. A more detailed analysis could derive $a$,
                $v_{\max}$, and the effective cutoff function $f(\omega/\omega_{\max})$
                from a specific model of the medium, potentially including compressibility,
                elasticity, or additional internal degrees of freedom.

                \item \emph{Fluid analogues and laboratory tests.}
                In classical fluids, wave spectra in the presence of a bath of vortices often
                exhibit saturation or steepening at high wavenumber, reflecting the finite
                size of coherent structures. It would be interesting to design controlled
                experiments in which the energy spectrum of surface or internal waves is
                measured in a flow populated by vortices of known core size, to test how
                closely the observed saturation resembles the simple cutoff picture developed
                here.

                \item \emph{Coupling to electromagnetism and relativistic field theory.}
                We have deliberately avoided deriving Maxwell's equations from the underlying
                fluid dynamics. One possible extension is to embed the circulation-loop
                picture in a more general effective field theory, where the fluid degrees of
                freedom appear as an emergent sector coupled to electromagnetism. This could
                clarify whether the mechanical cutoff $\omega_{\max}$ can be related to
                known microscopic scales, such as the electron Compton frequency, without
                conflicting with precision tests of quantum electrodynamics.

                \item \emph{Connection to broader microstructural frameworks.}
                In separate work, circulation loops of the type considered here have been
                discussed under the name ``swirl strings'' as part of a broader attempt to
                represent matter and radiation in terms of structured rotational degrees of
                freedom in a continuous medium. The present paper can be viewed as the
                classical, incompressible limit of that wider programme, restricted to
                thermodynamic and ultraviolet questions. Exploring the full implications of
                such a framework lies beyond our present scope, but the circulation-loop
                gas studied here may serve as a useful benchmark sector.
            \end{itemize}

        \subsection{Final remarks}

            We have shown that a simple, classical model---an incompressible fluid populated by
            finite-core circulation loops---is sufficient to reproduce the ideal-gas and
            radiation equations of state and to regularize the classical ultraviolet behaviour
            of thermal radiation at the level of microstate counting. The ultraviolet catastrophe
            is thus seen not as an inevitable failure of classical reasoning, but as a consequence
            of assigning an infinite set of independent degrees of freedom to the field without
            regard to the mechanical limitations of any underlying medium. While the correct
            blackbody spectrum and its $T^4$ scaling undoubtedly require quantum statistics,
            the circulation-loop model provides a concrete example of how microstructure and
            finite rotational frequencies can render the classical ultraviolet problem moot,
            without modifying Maxwell's equations or the familiar thermodynamic relations that
            emerge from kinetic theory.

    \section*{Acknowledgments}
        The author thanks the authors of standard texts on fluid mechanics and relativistic field theory for providing the background material on which this analysis is based.

        \bibliographystyle{unsrt}
        \begin{thebibliography}{99}
            \bibitem{Einstein1905_Eequalsmc2}
            A.~Einstein,
            \newblock \emph{Ist die Tr\"agheit eines K\"orpers von seinem Energieinhalt abh\"angig?},
            \newblock Annalen der Physik \textbf{323}, 13, 639--641 (1905).
            \newblock \href{https://doi.org/10.1002/andp.19053231314}{10.1002/andp.19053231314}

            \bibitem{Planck1901}
            M.~Planck,
            \newblock \emph{{\"U}ber das Gesetz der Energieverteilung im Normalspectrum},
            \newblock Annalen der Physik \textbf{309}, 3, 553--563 (1901).
            \newblock \href{https://doi.org/10.1002/andp.19013090310}{10.1002/andp.19013090310}

            \bibitem{Rayleigh1900}
            Lord Rayleigh,
            \newblock \emph{Remarks upon the Law of Complete Radiation},
            \newblock Philosophical Magazine \textbf{49}, 539--540 (1900).
            \newblock \href{https://doi.org/10.1080/14786440009463878}{10.1080/14786440009463878}

            \bibitem{Jeans1905}
            J.~H.~Jeans,
            \newblock \emph{On the Partition of Energy between Matter and Ether},
            \newblock Philosophical Magazine \textbf{10}, 91--98 (1905).
            \newblock \href{https://doi.org/10.1080/14786440509463292}{10.1080/14786440509463292}

            \bibitem{LandauLifshitz_Fields}
            L.~D.~Landau and E.~M.~Lifshitz,
            \newblock \emph{The Classical Theory of Fields}, 4th ed.,
            \newblock Butterworth--Heinemann, 1975.

            \bibitem{LandauLifshitz_StatPhys}
            L.~D.~Landau and E.~M.~Lifshitz,
            \newblock \emph{Statistical Physics, Part 1}, 3rd ed.,
            \newblock Pergamon Press, 1980.

            \bibitem{Reif}
            F.~Reif,
            \newblock \emph{Fundamentals of Statistical and Thermal Physics},
            \newblock McGraw--Hill, 1965.

            \bibitem{Batchelor1967}
            G.~K.~Batchelor,
            \newblock \emph{An Introduction to Fluid Dynamics},
            \newblock Cambridge University Press, 1967.

            \bibitem{Saffman1992}
            P.~G.~Saffman,
            \newblock \emph{Vortex Dynamics},
            \newblock Cambridge University Press, 1992.

            \bibitem{Iskandarani2025_RotKE}
            Omar Iskandarani,
            \newblock \emph{Rotational Kinetic Energy Density and an Effective Mass Relation in Incompressible Fluids},
            \newblock (preprint, Zenodo), 2025.
            \newblock \href{https://doi.org/10.5281/zenodo.17619150}{10.5281/zenodo.17619150}

            \bibitem{Iskandarani2025_VortexLoops}
            Omar Iskandarani,
            \newblock \emph{Energy, Impulse, and Stability of Thin Vortex Loops in Incompressible, Inviscid Fluids},
            \newblock (preprint, Zenodo), 2025.
            \newblock \href{https://doi.org/10.5281/zenodo.17619189}{10.5281/zenodo.17619189}

        \end{thebibliography}

\end{document}