%! Author = Omar Iskandarani
%! Date = Nov 15, 2025
%! Affiliation = Independent Researcher, Groningen, The Netherlands
%! License = © 2025 Omar Iskandarani. All rights reserved. This manuscript is made available for academic reading and citation only. No republication, redistribution, or derivative works are permitted without explicit written permission from the author. Contact: info@omariskandarani.com
%! ORCID = 0009-0006-1686-3961
%! DOI = 10.5281/zenodo.17619170

\newcommand{\paperdoi}{10.5281/zenodo.17619170}

%========================================================================================
% PACKAGES AND DOCUMENT CONFIGURATION
%========================================================================================
\documentclass[11pt]{article}
\usepackage{amsmath,amssymb,amsfonts,bm}
\usepackage{siunitx}
\usepackage[hidelinks]{hyperref}
\usepackage{graphicx} % for figures
\usepackage{geometry}
\usepackage[utf8]{inputenc}
\usepackage[T1]{fontenc}
\geometry{margin=1in}

%=============== Notation shortcuts =================
\newcommand{\dd}{\mathrm{d}}
\newcommand{\Om}{\Omega}
\newcommand{\bk}{\boldsymbol{k}}
\newcommand{\br}{\boldsymbol{r}}
\newcommand{\ez}{\hat{\boldsymbol{z}}}
\newcommand{\er}{\hat{\boldsymbol{r}}}
\newcommand{\etheta}{\hat{\boldsymbol{\theta}}}
\newcommand{\bU}{\boldsymbol{u}}
\newcommand{\bW}{\boldsymbol{\omega}}
\newcommand{\bOm}{\boldsymbol{\Omega}}
\newcommand{\grad}{\boldsymbol{\nabla}}
\newcommand{\curl}{\boldsymbol{\nabla}\!\times}
\newcommand{\divg}{\boldsymbol{\nabla}\!\cdot}
\newcommand{\p}{\partial}

% Optional microscopic scales used only as kinematic reference
\newcommand{\Ce}{C_e}   % characteristic speed scale
\newcommand{\rc}{r_c}   % microscopic length scale

\title{Impulsive Axisymmetric Forcing in a Rotating Cylinder,\\
Reversible Swirl Response, and Skyrmionic Photon Fields:\\
Fluid Benchmarks and Fluid-Inspired Kinematic Hypotheses}
\author{Omar Iskandarani}
\affiliation{Independent Researcher, Groningen, The Netherlands}
\thanks{ORCID: 0009-0006-1686-3961, DOI: \paperdoi}
\date{\today}

\begin{document}
    \maketitle

% Revision notes:
% - Added \usepackage{graphicx} for figure support.
% - Clarified notation for surface location (z=H).
% - Defined swirl early.
% - Added suggestions for figure inclusion.
% - Improved transitions between classical and speculative sections.
% - Clarified speculative time-dilation mapping.
% - Explicitly connected Chern-number structure of skyrmionic and vorticity invariants.

    \begin{abstract}
        We analyze impulsive, axisymmetric forcing in a rotating cylinder and its delayed surface ``push--pull'' response using classical inertial-wave theory. Linear rotating Euler equations with standard boundary conditions yield an axisymmetric inertial-wave packet whose vertical group velocity $c_{g,z}$ quantitatively explains the observed arrival time and bipolar surface signature. We compute the reversible swirl (azimuthal velocity $u_\theta$) generated by vertical forcing and show that a symmetric up--down stroke leads to vanishing net rotation to leading order.

        In Part~II, we highlight the formal correspondence between the skyrmion charge of structured photon fields and a surface integral built from the direction field of fluid vorticity. This suggests a shared topological classification scheme and provides a testable consistency condition for any microscopic emission mechanism modeled on fluid analogues. Standard macroscopic equations are used throughout, with speculative hypotheses clearly isolated.
    \end{abstract}

    \section*{Clarifying Notes and Visual Suggestions}
        \begin{itemize}
            \item \textbf{Terminology}: We use the term ``swirl'' to mean the azimuthal component of the fluid velocity, $u_\theta$, in cylindrical coordinates.
            \item \textbf{Figures to Include} (suggested):
            \begin{enumerate}
                \item A schematic of slanted inertial-wave propagation in the rotating cylinder, showing beam angles, $c_{g,z}$, and the delayed arrival at $z=H$.
                \item A visualization of axial displacement $\xi(r,z)$ and the resulting swirl field $u_\theta(r,z)$, highlighting cyclonic/anticyclonic regions.
                \item A comparison figure: polarization wrapping on the Poincar\'e sphere for optical skyrmions versus a knotted vorticity direction field in physical space.
            \end{enumerate}
            \item \textbf{Equation Emphasis}: Equation \eqref{eq:omegaxi} is a key result linking vertical displacement gradients to vertical vorticity in the rotating frame; it is highlighted explicitly in the text.
        \end{itemize}

%=================================================================
    \section*{Part I: Rotating-Tank Impulse and Reversible Swirl}
%=================================================================

%=================================================================
        \subsection*{Notation card}
%=================================================================
            We use the following standard quantities:
            \begin{align*}
                \rho_{\mathrm{E}} &= \tfrac{1}{2}\,\rho\,\lVert\bm v\rVert^2
                &&\text{(kinetic-energy density)},\\
                \rho_{\mathrm{m}} &= \rho_{\mathrm{E}}/c^2
                &&\text{(mass-equivalent density)},\\
                \gamma(v) &= \frac{1}{\sqrt{1-v^2/c^2}}
                &&\text{(Lorentz factor)},\\
                \omega &= 2\Omega\,\frac{k_z}{k},\quad
                k &= \sqrt{k_r^2+k_z^2}
                &&\text{(inertial-wave dispersion in a uniformly rotating fluid)}.
            \end{align*}
            Here $\rho$ is the mass density, $c$ the vacuum speed of light, and $\gamma(v)$ appears only in the context of special relativity. Classical rotating-fluid results follow Greenspan, Batchelor, and Vallis~\cite{Greenspan1968,Batchelor1967,Vallis2017}.

%=================================================================
    \section{Set-up and observation}
%=================================================================
        Consider a vertical cylinder of radius $R$ and depth $H$, rotating at rate $\Omega$ about the vertical axis $z$. A bottom DC motor drives a three-fin impeller that, for a short interval, produces an axisymmetric, hollow-core swirl of radius $a_v\approx\SI{7.5}{mm}$ together with axial jetting. After a delay of order $H/c_{g,z}$, a two-lobed ``push--pull'' surface signal is observed on the rotation axis at the free surface $z=H$: a depression for the start impulse and a rise for the stop impulse.

        We first show that this delayed bipolar signal can be understood quantitatively as the arrival of an axisymmetric inertial-wave packet, launched by the impulsive forcing and propagating upward along the axis.

        \paragraph{Energy bookkeeping.}
            For later comparison with optical analogues, it is convenient to track the kinetic-energy density $\rho_{\mathrm{E}}=\tfrac{1}{2}\rho\lVert\bU'\rVert^2$ and the associated mass-equivalent density $\rho_{\mathrm{m}}=\rho_{\mathrm{E}}/c^2$ for the perturbation $\bU'$ (no new dynamics are implied).

%=================================================================
    \section{Linear rotating-wave framework}
%=================================================================
    In the bulk, small perturbations of a uniformly rotating, incompressible fluid admit \emph{inertial waves} with dispersion relation~\cite{Greenspan1968,Batchelor1967,Vallis2017}
    \begin{equation}
        \omega = 2\Omega\,\frac{k_z}{k},\qquad
        k=\sqrt{k_r^2+k_z^2}.
        \label{eq:disp}
    \end{equation}
    For axisymmetry ($m=0$) in a cylinder, the radial wavenumbers satisfy $k_r\approx\lambda_{0n}/R$, where $\lambda_{0n}$ are Bessel eigenvalues~\cite{Greenspan1968}. Group velocities follow from $\partial\omega/\partial k_i$:
    \begin{equation}
        c_{g,r}= -\frac{2\Omega\,k_z k_r}{k^3},\qquad
        c_{g,z}= \frac{2\Omega\,k_r^2}{k^3}.
        \label{eq:cg}
    \end{equation}
    The energy beams obey $\tan\alpha=|c_{g,r}|/c_{g,z}=k_z/k_r$.

    \paragraph{Arrival time.}
        Taking $c_{g,z}$ from \eqref{eq:cg}, an inertial-wave packet launched near the base arrives at the surface after
        \begin{equation}
            t_{\mathrm{arr}} \approx \frac{H}{c_{g,z}} = \frac{H\,k^3}{2\Omega\,k_r^2}.
            \label{eq:tarr}
        \end{equation}
        \textit{Numerics (bench-top).}
        For $R=\SI{0.075}{m}$, $H=\SI{0.30}{m}$, $\Omega\approx\SI{2}{rad/s}$, $k_r\approx3.83/R\approx\SI{51}{m^{-1}}$, $k_z\approx\pi/H\approx\SI{10.5}{m^{-1}}$, one finds $k\approx\SI{52.1}{m^{-1}}$, $c_{g,z}\approx\SI{7.4e-2}{m/s}$, and $t_{\mathrm{arr}}\approx\SI{4.1}{s}$. Units are consistent: $[H/c_{g,z}]=\mathrm{s}$.

    \paragraph{Analogy (10-year-old).}
        It is like tipping over a slanted row of dominoes from the bottom: the last domino at the top falls only after a clear delay.

%=================================================================
    \section{Impulse sign and the observed ``push--pull''}
%=================================================================
    A start impulse produces axial upwelling and a centrifugal reduction of pressure $p'$ in the hollow core, which yields a \emph{surface depression} on arrival. A stop impulse reverses the sign of the perturbation and gives a \emph{surface rise}. Hydrostatic surface coupling~\cite{Batchelor1967,Vallis2017} is
    \begin{equation}
        p'(z=H)+\rho g\,\eta=0
        \;\Rightarrow\;
        \eta=-\frac{p'(H)}{\rho g},
        \label{eq:hydro}
    \end{equation}
    where $\eta$ is the surface elevation at $z=H$. The observed two-lobed axial signal is thus naturally interpreted as the superposition of inertial-wave arrival and hydrostatic adjustment.

%=================================================================
    \section{Relation to vortex rings and jet starting vortices}
%=================================================================
    A short bursting jet from a submerged orifice in quiescent fluid sheds a starting vortex ring whose translational speed is approximately~\cite{Saffman1992}
    \begin{equation}
        U_{\mathrm{ring}}\approx\frac{\Gamma}{4\pi R_v}
        \left[
            \ln\!\left(\frac{8 R_v}{a_v}\right)-\tfrac{1}{4}
        \right],
        \label{eq:ring}
    \end{equation}
    where $\Gamma$ is circulation, $R_v$ ring radius, and $a_v$ core radius. In the present rotating-tank setting, the on-axis \emph{delayed} packet at the surface is dominated by the inertial-wave field, not by ballistic motion of a vortex ring, because the latter would arrive much sooner and would not exhibit the observed $\Omega$-scaling of $t_{\mathrm{arr}}$.

%=================================================================
    \section{What this \emph{is} and \emph{is not} an analogy to}
%=================================================================
    \subsection*{Photon analogy (limited)}
        A localized impulse produces a localized wave packet with a well-defined parity (sign) that propagates through the medium. However, inertial waves in a rotating fluid are anisotropic and dispersive [Eqs.~\eqref{eq:disp}--\eqref{eq:cg}], whereas free photons in vacuum are isotropic and nondispersive. Any analogy to photons is therefore qualitative and limited to the idea of a traveling packet carrying energy and momentum.

    \subsection*{Gravitational-wave analogy (limited)}
        Gravitational waves in general relativity are transverse, quadrupolar, and nondispersive (in vacuum). Axisymmetric inertial-wave responses in a rotating fluid are neither quadrupolar nor nondispersive; the group velocity depends on $(k_r,k_z)$ and $\Omega$. Again, any analogy is at the level of ``wave packet launched by an impulsive source'', not at the level of detailed polarization or propagation properties.

%=================================================================
    \section{A kinematic time-dilation hypothesis inspired by fluid analogues}
%=================================================================
    We now briefly introduce a kinematic hypothesis for time dilation, motivated by relating local fluid speeds to relativistic time rates. This section is speculative and not used in the derivation of the macroscopic fluid results above.

    \subsection{Speed-based time rate with a characteristic scale}
        Let $u(\br,t)$ be a local fluid speed. One may postulate a generalized time-rate relation
        \begin{equation}
            \frac{\dd \tau}{\dd t} = \sqrt{1-\frac{u^2}{\Ce^2}},
            \label{eq:legacy}
        \end{equation}
        where $C_e$ is a characteristic speed scale associated, for example, with a microscopic medium. Eq.~\eqref{eq:legacy} mimics the special-relativistic form but with $C_e$ in place of $c$.

    \subsection{Standard special relativity}
        In established special relativity, the proper-time rate of a pointlike clock moving at speed $v$ is
        \begin{equation}
            \frac{\dd \tau}{\dd t} = \sqrt{1-\frac{v^2}{c^2}},
            \label{eq:clock}
        \end{equation}
        with $c$ the invariant speed. For small $v/c$, both \eqref{eq:legacy} and \eqref{eq:clock} admit an even expansion in $v$, and the leading correction is quadratic in the speed. In the present work, Eq.~\eqref{eq:clock} is regarded as the correct description for actual clocks; Eq.~\eqref{eq:legacy} is kept only as a mathematical template for fluid-inspired kinematic hypotheses.

        \paragraph{Parity prediction.}
            Two opposite-sign swirl impulses reverse the sign of the induced surface angle (as in the push--pull signal), but any time-rate effect even in the speed (from expanding \eqref{eq:legacy} or \eqref{eq:clock}) does not reverse. This parity asymmetry may be used to design conceptual thought experiments but is not pursued further here.

%=================================================================
    \section{Falsifiable checks for the rotating-tank interpretation}
%=================================================================
    Within standard rotating-fluid theory, the proposed interpretation of the delayed surface signal implies several testable features:
    \begin{itemize}
        \item \textbf{Travel-time scaling:} The arrival time satisfies $t_{\mathrm{arr}}\propto\Omega^{-1}$ for fixed mode numbers $(k_r,k_z)$. The pair $(k_r,k_z)$ may be inferred from the beam angle or from measured cylinder modes~\cite{Greenspan1968}.
        \item \textbf{Bipolarity:} Start and stop impulses yield opposite signs of $\eta$ via \eqref{eq:hydro}, consistent with a depression followed by a rise (or vice versa).
        \item \textbf{Anisotropy:} Off-axis probes should detect beams at angle $\tan\alpha=k_z/k_r$, while an on-axis probe detects primarily the vertical arrival.
        \item \textbf{Dependence on background rotation:} As $\Omega\to 0$, we expect $c_{g,z}\propto\Omega\to 0$ and $t_{\mathrm{arr}}\to\infty$, so that the delayed bipolar signal disappears, consistent with experiments in a nonrotating tank.
    \end{itemize}

%=================================================================
    \section*{Conclusion of Part I}
%=================================================================
    An impulsive rotor in a rotating cylinder launches an axisymmetric inertial-wave packet whose vertical group velocity $c_{g,z}$ explains the observed delayed on-axis surface ``push--pull'' signal. The bipolar signature follows the impulse signs through hydrostatic coupling. The scaling $t_{\mathrm{arr}}\propto\Omega^{-1}$ and the disappearance of the delayed signal as $\Omega\to 0$ follow directly from standard rotating-fluid theory.

    \paragraph{Analogy (10-year-old).}
        Spinning a bucket and giving a quick push at the bottom sends a slow, slanted ``message wave'' up the water; by the time it reaches the surface, it makes a little dip or bump, depending on how you pushed.

%=================================================================
    \section{Reversible azimuthal response to axisymmetric vertical forcing}
%=================================================================
    We now consider how vertical forcing in a rotating cylinder produces azimuthal swirl, and how symmetry leads to nearly reversible angular response. The cylinder has radius $R$, height $H$, and contains an incompressible fluid of density $\rho$. The container rotates at angular velocity $\Omega$ about $z$. The base state is solid-body rotation in the rotating frame, with absolute vorticity $\bW_a^{(0)}=2\bOm$~\cite{Batchelor1967,Greenspan1968}.

    \subsection{Governing equations and vorticity production}
        In the rotating frame, the inviscid equations are
        \begin{align}
            \p_t \bU + (\bU\!\cdot\!\grad)\bU + 2\bOm\times\bU &= -\grad \Pi,
            \label{eq:NSrot}\\
            \divg \bU &= 0,
            \label{eq:incomp}
        \end{align}
        where $\Pi$ is the reduced pressure. Taking the curl gives the absolute-vorticity equation~\cite{Batchelor1967,Vallis2017}
        \begin{equation}
            \p_t \bW = \curl(\bU\times \bW_a),\qquad \bW_a=\bW+2\bOm.
            \label{eq:vortgen}
        \end{equation}
        Linearizing about the base state yields
        \begin{equation}
            \p_t \bW \approx 2(\bOm\!\cdot\!\grad)\bU.
            \label{eq:vortlin}
        \end{equation}
        For axisymmetric motions, the vertical vorticity component obeys
        \begin{equation}
            \boxed{\;\p_t \omega_z = 2\Omega\,\p_z w\;}
            \label{eq:key}
        \end{equation}
        with $w$ the vertical velocity. Introducing a vertical displacement $\xi$ with $w=\p_t\xi$ and integrating from rest gives the key result
        \begin{equation}
            \boxed{\;\omega_z(r,z,t) = 2\Omega\,\p_z \xi(r,z,t)\;}
            \label{eq:omegaxi}
        \end{equation}
        which directly links vertical displacement gradients to vertical vorticity.

    \subsection{From vertical vorticity to azimuthal velocity (swirl)}
        The azimuthal velocity $u_\theta$ (swirl) and vertical vorticity $\omega_z$ are related kinematically by
        \begin{equation}
            \omega_z=\frac{1}{r}\,\p_r\!\big(r\,u_\theta\big)
            \;\Rightarrow\;
            u_\theta(r,z,t)=\frac{1}{r}\int_0^r \omega_z(r',z,t)\,r'\,\dd r'.
            \label{eq:uth}
        \end{equation}
        For illustration, take a Gaussian displacement
        \begin{equation}
            \xi(r,z,t)=Z(t)\,\exp\!\left(-\frac{r^2+z^2}{a^2}\right),
        \end{equation}
        with amplitude $Z(t)$ and length scale $a$. Using \eqref{eq:omegaxi} and \eqref{eq:uth} then yields
        \begin{equation}
            u_\theta(r,z,t)= -\frac{2\Omega\,Z(t)\,z}{a^{2}}\,e^{-z^{2}/a^{2}}\,
            \frac{1-e^{-r^{2}/a^{2}}}{r}.
            \label{eq:uth_gauss}
        \end{equation}
        Below the forcing ($z<0$) the swirl is cyclonic, above it ($z>0$) anticyclonic. Near the axis, $1-e^{-r^2/a^2}\sim r^2/a^2$ ensures regularity.

    \subsection{Angle reversal and reversibility}
        The relative angular rate is $\dot{\theta}_{\mathrm{rel}}=u_\theta/r$. Over a time interval $[t_1,t_2]$,
        \begin{equation}
            \Delta \theta_{\mathrm{rel}}(r,z)=\int_{t_1}^{t_2}\frac{u_\theta}{r}\,\dd t.
        \end{equation}
        Because $\dot{\theta}_{\mathrm{rel}}$ is linear in $Z(t)$, a symmetric up--down cycle with zero mean displacement (for example, a stroke that raises then lowers the base with equal amplitude) yields
        \begin{equation}
            \boxed{\;\Delta \theta_{\mathrm{rel}}(r,z;\text{one period})=0\quad(\text{to leading order}).\;}
        \end{equation}
        Irreversibility arises only from viscosity (Ekman layers), quadratic advection (streaming), or near-resonant inertial waves~\cite{Greenspan1968}. To leading order in linear theory, the azimuthal swirl response is reversible.

        \paragraph{Analogy (10-year-old).}
            If you twist a bucket up and then twist it back down in exactly the opposite way, the water’s extra spin nearly cancels; only tiny friction effects stop it from being perfectly reversible.

%=================================================================
    \section*{Part II: Topological Structure of Skyrmionic Photon Fields and Fluid Analogies}
%=================================================================

    \subsection{Optical skyrmions and topological charge}
        Recent work on structured light has shown that single photons in suitably engineered microcavities can carry skyrmion-like polarization textures in momentum space, classified by an integer charge $N_{\mathrm{sk}}^{(\mathrm{ph})}$~\cite{Allen1992OAM,Walborn2010SPDCReview,Shen2024NatPhoton,Ma2025NanoPhotonSkyrmions}. In one common formulation, the Stokes vector $\vec{S}(\boldsymbol{k}_\perp)$ is normalized to a unit vector $\hat{\vec{S}}$, and the topological charge is
        \begin{equation}
            N_{\mathrm{sk}}^{(\mathrm{ph})}
            = \frac{1}{4\pi} \int d^{2}k_\perp\,
            \hat{\vec{S}}\cdot\big(\partial_{k_x}\hat{\vec{S}} \times \partial_{k_y}\hat{\vec{S}}\big)\in\mathbb{Z}.
            \label{eq:Nsk}
        \end{equation}
        This integer counts how many times the polarization state winds around the Poincar\'e sphere as transverse wavevector $\boldsymbol{k}_\perp$ spans the mode. Formally, it is a Chern-number-like invariant associated with the Berry curvature of the polarization bundle over momentum space.

    \subsection{Vorticity-based invariants in classical fluids}
        In classical vortex dynamics, related topological invariants appear in the study of knotted vorticity fields~\cite{Moffatt1969Helicity,Saffman1992}. Let $\hat{\boldsymbol{\omega}}=\boldsymbol{\omega}/\lVert\boldsymbol{\omega}\rVert$ be the unit vorticity direction, and let $\Sigma$ be a surface. One may define
        \begin{equation}
            H_{\mathrm{vortex}}[\hat{\boldsymbol{\omega}}|\Sigma]
            \equiv \frac{1}{4\pi}\int_{\Sigma}\!
            \hat{\boldsymbol{\omega}}\cdot\big( \partial_x \hat{\boldsymbol{\omega}} \times \partial_y \hat{\boldsymbol{\omega}} \big)\, dx \, dy.
            \label{eq:Hvortex}
        \end{equation}
        The integrand has the same structure as in \eqref{eq:Nsk}, but now built from the vorticity-direction field instead of the optical Stokes vector. It is thus natural to view both $N_{\mathrm{sk}}^{(\mathrm{ph})}$ and $H_{\mathrm{vortex}}$ as realizing the same underlying Chern-number structure in two different physical contexts.

        \paragraph{Formal correspondence.}
            The expressions for $N_{\mathrm{sk}}^{(\mathrm{ph})}$ and $H_{\mathrm{vortex}}$ are structurally identical. In fluid-analogue models of structured light, it is therefore natural to require that any effective description of emission from knotted vorticity configurations reproduce the same integer topological charge in the far-field optical texture, i.e. that $N_{\mathrm{sk}}^{(\mathrm{ph})}$ and $H_{\mathrm{vortex}}$ agree within a given mode family. In the present work, this is stated as a consistency condition rather than as a derived emission mechanism.

    \subsection{Projection from vorticity to optical modes}
    Suppose that on a surface $\Sigma$ the local polarization-driving quantity is described by a field
    \[
        \mathbf{p}(\mathbf{r})=p_0(\mathbf{r})\,\hat{\boldsymbol{\omega}}_\perp(\mathbf{r}),
    \]
    where $\hat{\boldsymbol{\omega}}_\perp$ is a unit vector derived from the vorticity direction projected into the plane of $\Sigma$. A far-field optical mode expansion in Laguerre--Gaussian functions $u_{p,\ell}$ with helicity $\sigma=\pm1$ may be written schematically as
    \begin{equation}
        A_{p\ell\sigma} = \int_{\Sigma}
        ( \hat{e}_\sigma^{*}\!\cdot\!\mathbf{P}_\perp\hat{\boldsymbol{\omega}}_\perp )\,
        u_{p,\ell}(\mathbf{r})\, e^{i\Phi(\mathbf{r})}\,\dd^2 r,
        \label{eq:modal_overlap}
    \end{equation}
    where $\mathbf{P}_\perp=\mathbf{I}-\hat{\mathbf{k}}\hat{\mathbf{k}}^\top$ projects transverse to the propagation direction $\hat{\mathbf{k}}$, and $\Phi(\mathbf{r})$ encodes accumulated phase. Computing Stokes parameters from the resulting field $\mathbf{E}$ leads back to a skyrmion charge of the form \eqref{eq:Nsk}.

    \paragraph{Interpretation.}
        Eq.~\eqref{eq:modal_overlap} sketches how a structured vorticity field on a surface could, in principle, seed a structured far-field optical texture with matching topological charge. This is intended as a formal mapping between two topological characterizations (vorticity-based and optical) rather than as a detailed microscopic emission model.

    \subsection{Orbital angular momentum and chirality}
    Per-photon orbital angular momentum (OAM) in spontaneous parametric down-conversion (SPDC) is known to obey additive selection rules of the form~\cite{Allen1992OAM,Walborn2010SPDCReview}
    \begin{equation}
        \ell_{\rm pump} = \sum_{j=1}^{n} \ell_j \quad (n\text{-photon channel}),
        \label{eq:oam_additivity}
    \end{equation}
    where $\ell_j$ denotes the OAM of each photon. In a fluid-analogue viewpoint, one may associate opposite senses of swirl (clockwise vs counterclockwise) with opposite helicities of the emitted light; this is fully consistent with the usual identification of optical helicity with circular polarization but adds no new selection rule beyond those already present in OAM-conserving nonlinear optics.

    \subsection{A kinematic reference scale}
    Finally, for dimensional comparison, introduce a kinematic frequency
    \begin{equation}
        \Omega_0 \equiv \frac{\Ce}{\rc},\qquad
        E_0=\hbar\Omega_0,
        \label{eq:Omega0}
    \end{equation}
    where $\Ce$ is a characteristic speed and $\rc$ a microscopic length scale. For illustrative numerical values $\Ce\sim 10^6\,\mathrm{m\,s^{-1}}$ and $\rc\sim 10^{-15}\,\mathrm{m}$, one finds $\Omega_0\sim 10^{21}\,\mathrm{s^{-1}}$ and $E_0\sim 0.5\,\mathrm{MeV}$, i.e.\ comparable to the electron rest energy $m_e c^2$. In the present work this is noted only as a suggestive coincidence; no claim is made that $E_0$ replaces $m_e c^2$ or that it arises from a specific underlying medium.

    \subsection{Minimal experimental roadmap}
    To the extent that fluid analogues of structured photon fields are of interest, several experimental directions suggest themselves:
    \begin{itemize}
        \item \textbf{Re-analysis of single-photon skyrmion data:} Express measured Stokes fields for single-photon skyrmions in terms of a reconstructed $\hat{\boldsymbol{\omega}}_\perp$ and check whether $N_{\mathrm{sk}}^{(\mathrm{ph})}$ can be mapped to a vorticity-based invariant of the form \eqref{eq:Hvortex}.
        \item \textbf{Rotating-fluid or elastic analogues:} Macroscopic systems with controlled vorticity or director fields (e.g.\ swirling flows, elastic or liquid-crystal textures) may be used to implement fields whose topology is directly controllable and then compared to corresponding structured optical modes.
        \item \textbf{Robustness tests:} Topological invariants such as $N_{\mathrm{sk}}^{(\mathrm{ph})}$ and $H_{\mathrm{vortex}}$ should be robust under smooth deformations and linear propagation; this can be tested experimentally by varying propagation distance and cavity tuning.
    \end{itemize}

    \paragraph{Analogy (10-year-old).}
        If you twist water into different kinds of knots and then shine light that copies those twists, the ``pattern of twists'' in the light can be counted, like counting how many times a shoelace wraps around itself.

%-----------------------------------------------------------------
    \section*{Global Conclusion}
%-----------------------------------------------------------------
    \paragraph{Part I.}
        We have shown that an impulsive rotor in a rotating cylinder launches an axisymmetric inertial-wave packet whose vertical group velocity explains the observed delayed on-axis surface ``push--pull'' signal. The scaling $t_{\mathrm{arr}}\propto\Omega^{-1}$ and the disappearance of the delayed signal as $\Omega\to 0$ follow directly from standard rotating-fluid theory. We also derived a linear relation between vertical displacement gradients and vertical vorticity, leading to a reversible azimuthal swirl response under symmetric up--down forcing.

    \paragraph{Part II.}
        We highlighted a formal correspondence between the skyrmion charge $N_{\mathrm{sk}}^{(\mathrm{ph})}$ of structured photon fields and a vorticity-based invariant $H_{\mathrm{vortex}}$ for knotted vorticity fields. Both invariants share the structure of a Chern number built from a unit vector field over a two-dimensional parameter space. This suggests a common topological language for comparing rotating-fluid analogues and structured-photon experiments. We did not propose a specific microscopic model of photon emission; instead, we formulated a consistency requirement: any fluid-inspired microscopic model that seeks to relate vorticity textures to structured photon fields should reproduce the same integer topological invariants that are already measured in single-photon skyrmion experiments.

%====================== Acknowledgments ======================
    \paragraph*{Acknowledgments.}
        Classical fluid and wave results are based on \cite{Batchelor1967,Greenspan1968,Saffman1992,Vallis2017,LandauFluids,Lighthill78}. Topological and OAM optics results follow \cite{Allen1992OAM,Ma2025NanoPhotonSkyrmions,Shen2024NatPhoton,Walborn2010SPDCReview,Moffatt1969Helicity}. Relativistic and causality constraints are discussed in \cite{Brillouin1960,Jackson1999,Abbott2017PRL,Abbott2017ApJL}.

%====================== Bibliography (BibTeX) ======================
        \begin{thebibliography}{99}

        \bibitem{Batchelor1967}
        G.~K.~Batchelor,
        \newblock \emph{An Introduction to Fluid Dynamics},
        \newblock Cambridge Univ.\ Press (1967).

        \bibitem{Greenspan1968}
        H.~P.~Greenspan,
        \newblock \emph{The Theory of Rotating Fluids},
        \newblock Cambridge Univ.\ Press (1968).

        \bibitem{Saffman1992}
        P.~G.~Saffman,
        \newblock \emph{Vortex Dynamics},
        \newblock Cambridge Univ.\ Press (1992).

        \bibitem{Vallis2017}
        G.~K.~Vallis,
        \newblock \emph{Atmospheric and Oceanic Fluid Dynamics} (2nd ed.),
        \newblock Cambridge Univ.\ Press (2017).

        \bibitem{LandauFluids}
        L.~D.~Landau and E.~M.~Lifshitz,
        \newblock \emph{Fluid Mechanics} (2nd ed.),
        \newblock Pergamon Press (1987).

        \bibitem{Lighthill78}
        M.~J.~Lighthill,
        \newblock \emph{Waves in Fluids},
        \newblock Cambridge Univ.\ Press (1978).

        \bibitem{Brillouin1960}
        L.~Brillouin,
        \newblock \emph{Wave Propagation and Group Velocity},
        \newblock Academic Press (1960).

        \bibitem{Jackson1999}
        J.~D.~Jackson,
        \newblock \emph{Classical Electrodynamics} (3rd ed.),
        \newblock Wiley (1999).

        \bibitem{Abbott2017PRL}
        B.~P.~Abbott \emph{et al.},
        \newblock \emph{{GW170817}: Observation of Gravitational Waves from a Binary Neutron Star Inspiral},
        \newblock Phys.\ Rev.\ Lett.\ \textbf{119}, 161101 (2017).

        \bibitem{Abbott2017ApJL}
        B.~P.~Abbott \emph{et al.},
        \newblock \emph{Multi-messenger Observations of a Binary Neutron Star Merger},
        \newblock Astrophys.\ J.\ Lett.\ \textbf{848}, L12 (2017).

        \bibitem{Allen1992OAM}
        L.~Allen, M.~W.~Beijersbergen, R.~J.~C.~Spreeuw, and J.~P.~Woerdman,
        \newblock \emph{Orbital angular momentum of light and the transformation of Laguerre--Gaussian laser modes},
        \newblock Phys.\ Rev.\ A \textbf{45}, 8185--8189 (1992).

        \bibitem{Moffatt1969Helicity}
        H.~K.~Moffatt,
        \newblock \emph{The degree of knottedness of tangled vortex lines},
        \newblock J.\ Fluid Mech.\ \textbf{35}, 117--129 (1969).

        \bibitem{Walborn2010SPDCReview}
        S.~P.~Walborn, C.~H.~Monken, S.~Padua, and P.~H.~Souto Ribeiro,
        \newblock \emph{Spatial correlations in parametric down-conversion},
        \newblock Phys.\ Rep.\ \textbf{495}, 87--139 (2010).

        \bibitem{Shen2024NatPhoton}
        Y.~Shen \emph{et al.},
        \newblock \emph{Topological textures of structured photons},
        \newblock Nat.\ Photon.\ (2024).

        \bibitem{Ma2025NanoPhotonSkyrmions}
        X.~Ma \emph{et al.},
        \newblock \emph{Single-photon optical skyrmions in spin--orbit--engineered microcavities},
        \newblock (journal details forthcoming) (2025).

        \end{thebibliography}

\end{document}