%! Author = Omar Iskandarani
%! Date = 12/5/2025
%! Affiliation = Independent Researcher, Groningen, The Netherlands
%! License = © 2025 Omar Iskandarani. All rights reserved. This manuscript is made available for academic reading and citation only. No republication, redistribution, or derivative works are permitted without explicit written permission from the author. Contact: info@omariskandarani.com
%! ORCID = 0009-0006-1686-3961
%! DOI = 10.5281/zenodo.17877206

\newcommand{\paperdoi}{10.5281/zenodo.17877206}
\newcommand{\papertitle}{Neutrino Chirality from Swirl–Clock Geometry:\\A First-Principles Derivation in Swirl–String Theory (SST)}

%=========================================
% % PREAMBLE, PACKAGES AND DOCUMENT CONFIGURATION
%=========================================
%===========================================================
% SST PAPER: Neutrino Chirality as a Swirl–Clock Phenomenon
%===========================================================

\documentclass[11pt]{article}

\usepackage{amsmath,amssymb,amsfonts,bm}
\usepackage{physics}
\usepackage{siunitx}
\usepackage[hidelinks]{hyperref}
\usepackage[a4paper,margin=1in]{geometry}
\usepackage[T1]{fontenc}
\usepackage[utf8]{inputenc}

% swirl arrows (context-aware)
\newcommand{\swirlarrow}{ \mathchoice{\mkern-2mu\scriptstyle\boldsymbol{\circlearrowleft}}{\mkern-2mu\scriptscriptstyle\boldsymbol{\circlearrowleft}}}
\newcommand{\vswirl}{\mathbf{v}_{\swirlarrow}}
\newcommand{\SwirlClock}{S_{(t)}^{\swirlarrow}}
\newcommand{\Fmaxswirl}{F^{\max}_{\mkern-1mu\scriptscriptstyle\boldsymbol{\circlearrowleft}}}
% swirl arrows Counter Clockwise
\newcommand{\swirlarrowcw}{ \mathchoice{\mkern-2mu\scriptstyle\boldsymbol{\circlearrowright}}{\mkern-2mu\scriptscriptstyle\boldsymbol{\circlearrowright}}}
\newcommand{\vswirlcw}{\mathbf{v}_{\swirlarrowcw}}
\newcommand{\SwirlClockcw}{S_{(t)}^{\swirlarrowcw}}
\newcommand{\Fmaxswirlcw}{F^{\max}_{\mkern-1mu\scriptscriptstyle\boldsymbol{\circlearrowright}}}

\newcommand{\Fmax}{\Fmaxswirl} % default maximal force (left swirl)
\newcommand{\FmaxEM}{F^{\max}_{\mathrm{EM}}}
\newcommand{\FmaxG}{F_{\mathrm{G}}^{\max}}               % G-like maximal force scale

\newcommand{\omegas}{\boldsymbol{\omega}_{\swirlarrow}}  % swirl vorticity
\newcommand{\Om}{\Omega_{\swirlarrow}}                   % swirl angular frequency profile

\newcommand{\vscore}{v_{\swirlarrow}}                    % shorthand: |v_swirl| at r=r_c
\newcommand{\vnorm}{\lVert \mathbf{v}_{\mkern-2mu\scriptscriptstyle\boldsymbol{\circlearrowleft}} \rVert}               % swirl speed magnitude
\newcommand{\Ce}{\vswirl}                                % canonical swirl-speed constant

\newcommand{\rhof}{\rho_{\!f}}                           % effective fluid density
\newcommand{\rhoE}{\rho_{\!E}}                           % swirl energy density
\newcommand{\rhom}{\rho_{\!m}}                           % mass-equivalent density
\newcommand{\rc}{r_c}                                    % string core radius (swirl string radius)

\newcommand{\Lam}{\Lambda}                               % Swirl Coulomb constant
\newcommand{\alpg}{\alpha_g}                             % gravitational fine-structure analogue

\newcommand{\rhoM}{\rho_{\!m}}



% Swirl–clock field (preferred foliation)
\newcommand{\uclk}{u^\mu}
\newcommand{\titlepageOpen}{
    \begin{titlepage}
        \thispagestyle{empty}
        \centering
        \Large \bfseries \papertitle \par \vspace{1cm}
        {\Large \itshape \textbf{Omar Iskandarani}\textsuperscript{\textbf{*}} \par}
        \vspace{0.5cm}
        {\today \par}
        \vspace{0.5cm}
}

\newcommand{\titlepageClose}{
        \vfill \raggedright \null
        \begin{picture}(0,0)
            \put(0,-45){  % Shift 200pt left, 40pt down
                \begin{minipage}[b]{0.7\textwidth} \footnotesize
                    \renewcommand{\arraystretch}{1.0}
                    \noindent\rule{\textwidth}{0.4pt} \\[0.5em]
                    \textsuperscript{\textbf{*}} Independent Researcher, Groningen, The Netherlands \\
                    Email: \texttt{info@omariskandarani.com} \\
                    ORCID: \texttt{\href{https://orcid.org/0009-0006-1686-3961}{0009-0006-1686-3961}} \\
                    DOI: \href{https://doi.org/\paperdoi}{\paperdoi}
                \end{minipage}
            }
        \end{picture}
    \end{titlepage}
}
%=========================================
% Start Document - Title Page
%=========================================
\begin{document}
    \titlepageOpen
    \begin{abstract}
        Neutrinos appear exclusively left-handed in low-energy interactions. In the Standard Model this is imposed by construction through the $SU(2)_L$ chiral gauge structure. In this paper we derive an alternative, purely geometric origin for neutrino chirality based on the swirl–clock structure of Swirl–String Theory (SST). A universal timelike field $u^\mu$, encoding the global foliation induced by swirl dynamics, couples axially to fermions. The coupling strength is determined by the canonical swirl-force scale $F_{\text{swirl}}^{\max}$, which itself arises from harmonic motion at the Compton frequency over the classical radius. Integrating out the right-chiral neutrino generates a low-energy effective theory with a single propagating chiral state—recovering the observed left-handed neutrino sector. Hydrogen's $3\times$ swirl-charge structure aligns with this preferred orientation, establishing a unified microscopic origin for matter chirality and the electroweak asymmetry. We extract experimental predictions and falsifiable signatures.
    \end{abstract}
    \titlepageClose
%=========================================
% Title Page End
%=========================================

        \section{Introduction}
            The persistent left-handedness of neutrinos is among the strongest observed violations of parity in nature. In the Standard Model, this phenomenon is implemented artificially: only left-chiral fermions transform under $SU(2)_L$, and right-chiral neutrinos are absent.

            Swirl–String Theory (SST), a hydrodynamic reinterpretation of particle physics where matter corresponds to knotted swirl strings embedded in a preferred foliation, provides an alternative pathway. SST postulates a global swirl–clock field $\uclk$ defining a timelike direction associated with the background swirl density. This structure breaks local Lorentz symmetry spontaneously, and allows new axial couplings to fermionic currents.

            This paper derives how such couplings naturally select a single neutrino chirality, linking the phenomenon to a universal oscillator scale derived from the electron.

%===========================================================
        \section{The Swirl–Clock Field and Preferred Foliation}

            SST introduces a unit timelike vector field
            \begin{equation}
                \uclk u_\mu = -1,
            \end{equation}
            representing the direction of ``vertical swirl'' in the fluid foliation. Matter excitations with swirl-clock orientation aligned with $\uclk$ are energetically favored; those anti-aligned are suppressed.

            Hydrogen's swirl-charge is
            \begin{equation}
                Q_{\text{swirl}}(H) = +3,
            \end{equation}
            reflecting the three swirl strings (proton constituents + electron) aligned along $\uclk$.

            We now show that the neutrino is the \emph{minimal} swirl-charged excitation obeying the same alignment rule, with chirality emerging dynamically.

%===========================================================
        \section{Axial Coupling of Neutrinos to the Swirl–Clock Field}

            Consider a Dirac neutrino field $\nu$. The most general leading-order coupling consistent with gauge invariance is an axial-current interaction:
            \begin{equation}
                \mathcal{L}_{\text{int}}
                =
                \frac{\lambda}{M_\ast}\,
                u_\mu\,\bar{\nu}\gamma^\mu\gamma^5\nu.
                \label{eq:axial}
            \end{equation}

            Write the chiral components
            \begin{equation}
                \nu_{L,R}
                = \frac{1\mp \gamma^5}{2}\nu.
            \end{equation}

            In the rest frame of the foliation, $u^\mu \simeq (1,0,0,0)$, so
            \begin{equation}
                \mathcal{L}_{\text{int}}
                =
                \frac{\lambda}{M_\ast}
                \left(\nu_L^\dagger\nu_L - \nu_R^\dagger\nu_R\right).
            \end{equation}

            Thus the two chiralities experience opposite ``chemical potentials'':
            \begin{equation}
                \mu_5 = \frac{\lambda}{M_\ast}.
            \end{equation}

            Low-energy dispersion relations become
            \begin{equation}
                E_{L,R}(\mathbf{p})
                = \lvert\mathbf{p}\rvert \mp \mu_5.
            \end{equation}

            If $\mu_5>0$, the right-chiral branch becomes energetically heavy and can be integrated out. The neutrino then appears \emph{purely left-handed}.

            To complete the mechanism, we must identify the scale $M_\ast$ from SST first principles.

%===========================================================
        \section{Swirl-Force Scale from Harmonic Motion of the Electron}

            Define the swirl-force scale using harmonic motion at the Compton frequency and the classical electron radius:
            \begin{equation}
                F_{\text{swirl}}^{\max}
                = m_e\left(\frac{\omega_C}{\alpha}\right)^2 r_e.
                \label{eq:Fmax}
            \end{equation}

            Here
            \[
                \omega_C = \frac{m_e c^2}{\hbar},
                \qquad
                r_e = \frac{e^2}{4\pi\varepsilon_0 m_e c^2}.
            \]

            Using the SST core radius $\rc$, define the swirl energy scale
            \begin{equation}
                E_{\text{swirl}}
                \equiv F_{\text{swirl}}^{\max}\rc.
            \end{equation}

            Substituting Eq.~\eqref{eq:Fmax} and canonical values,
            \begin{equation}
                E_{\text{swirl}}
                \approx 2.555\times 10^5~\text{eV}
                = \frac{1}{2}m_e c^2.
            \end{equation}

            Thus
            \begin{equation}
                M_\ast = \frac{E_{\text{swirl}}}{c^2}
                = \frac{1}{2} m_e.
            \end{equation}

            Inserting this into the axial coupling:
            \begin{equation}
                \mathcal{L}_{\text{int}}
                =
                \frac{2\lambda}{m_e}\,
                u_\mu\,\bar{\nu}\gamma^\mu\gamma^5\nu.
                \label{eq:final_axial}
            \end{equation}

            This shows that the neutrino's chiral asymmetry is anchored directly to the electron's Compton/Bohr structure.

%===========================================================
        \section{Hydrogen Alignment and the Swirl Selection Rule}

            Hydrogen carries swirl-charge $+3$:
            \[
                Q_{\text{swirl}}(H) = 3.
            \]

            The neutrino carries
            \[
                Q_{\text{swirl}}(\nu)=1,
            \]
            with orientation fixed by the sign of $\lambda$ in Eq.~\eqref{eq:final_axial}.

            Thus both hydrogen and the neutrino align with the same swirl-clock direction. The electroweak chiral asymmetry is therefore a consequence of a universal swirl selection rule:
            \[
                \text{Preferred chirality} \iff \text{alignment with } u^\mu.
            \]

            This reproduces the observed left-handed neutrino sector.

%===========================================================
        \section{Predictions and Falsifiable Signatures}

            \begin{enumerate}
                \item \textbf{Right-handed neutrino suppression.}
                The right-chiral branch is suppressed by $\mu_5 \propto \lambda/m_e$.
                Experiments searching for sterile right-handed neutrinos should observe an effective high mass gap.

                \item \textbf{Directional modulation.}
                If $\uclk$ has cosmological variation, ultra-high-energy neutrinos may show directional asymmetries.

                \item \textbf{Chiral amplification in gravitational wells.}
                The effective $\mu_5$ increases in regions where swirl density or foliation gradients increase (e.g., near neutron stars).

                \item \textbf{Hydrogen-dependent chirality lock-in.}
                Systems with coherent hydrogen swirl-charge (dense hydrogen phases, metallic hydrides) may exhibit enhanced parity-violating optical activity.
            \end{enumerate}

            These are specific to SST and distinguish it from Standard Model neutrino physics.

%===========================================================
        \section{Conclusion}

            We have shown that neutrino chirality emerges naturally in SST from an axial coupling to the swirl–clock field, with coupling strength determined purely by the fundamental swirl-force scale. This scale, in turn, is set by electron harmonic motion at the Compton frequency and connects directly to hydrogen's ground-state energetics.

            The result is a parsimonious explanation of the neutrino's left-handedness, not as an imposed gauge asymmetry but as a dynamical consequence of the swirl-geometry underlying spacetime and matter.

%===========================================================
            \begin{thebibliography}{9}

                \bibitem{Weinberg1967}
                S. Weinberg,
                \textit{A Model of Leptons},
                Phys. Rev. Lett. 19, 1264 (1967).

                \bibitem{Wu1957}
                C. S. Wu et al.,
                \textit{Experimental Test of Parity Conservation in Beta Decay},
                Phys. Rev. 105, 1413 (1957).

                \bibitem{Jacobson2001}
                T. Jacobson and D. Mattingly,
                \textit{Gravity with a Dynamical Preferred Frame},
                Phys. Rev. D 64, 024028 (2001).

            \end{thebibliography}

    \end{document}