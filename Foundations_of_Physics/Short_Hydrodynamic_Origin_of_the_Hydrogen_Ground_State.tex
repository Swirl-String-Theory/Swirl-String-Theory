%! Author = Omar Iskandarani
%! Date = 11/26/2025
%! Affiliation = Independent Researcher, Groningen, The Netherlands
%! License = © 2025 Omar Iskandarani. All rights reserved. This manuscript is made available for academic reading and citation only. No republication, redistribution, or derivative works are permitted without explicit written permission from the author. Contact: info@omariskandarani.com
%! ORCID = 0009-0006-1686-3961
%! DOI = 10.5281/zenodo.xxx

\newcommand{\paperdoi}{10.5281/zenodo.xxx}
\newcommand{\papertitle}{Hydrodynamic Origin of the Hydrogen Ground State}

%=========================================
% % PREAMBLE, PACKAGES AND DOCUMENT CONFIGURATION
%=========================================
\documentclass[11pt]{article}
\usepackage{amsmath,amssymb,amsfonts,bm}
\usepackage{siunitx}
\usepackage[hidelinks]{hyperref}
\usepackage[a4paper,margin=1in]{geometry}
\usepackage[T1]{fontenc}
\usepackage[utf8]{inputenc}
\usepackage{graphicx}

% swirl arrows (context-aware)
\newcommand{\swirlarrow}{ \mathchoice{\mkern-2mu\scriptstyle\boldsymbol{\circlearrowleft}}{\mkern-2mu\scriptscriptstyle\boldsymbol{\circlearrowleft}}}
\newcommand{\vswirl}{\mathbf{v}_{\swirlarrow}}
\newcommand{\SwirlClock}{S_{(t)}^{\swirlarrow}}
\newcommand{\Fmaxswirl}{F^{\max}_{\mkern-1mu\scriptscriptstyle\boldsymbol{\circlearrowleft}}}
% swirl arrows Counter Clockwise
\newcommand{\swirlarrowcw}{ \mathchoice{\mkern-2mu\scriptstyle\boldsymbol{\circlearrowright}}{\mkern-2mu\scriptscriptstyle\boldsymbol{\circlearrowright}}}
\newcommand{\vswirlcw}{\mathbf{v}_{\swirlarrowcw}}
\newcommand{\SwirlClockcw}{S_{(t)}^{\swirlarrowcw}}
\newcommand{\Fmaxswirlcw}{F^{\max}_{\mkern-1mu\scriptscriptstyle\boldsymbol{\circlearrowright}}}

\newcommand{\Fmax}{\Fmaxswirl} % default maximal force (left swirl)
\newcommand{\FmaxEM}{F^{\max}_{\mathrm{EM}}}
\newcommand{\FmaxG}{F_{\mathrm{G}}^{\max}}               % G-like maximal force scale

\newcommand{\omegas}{\boldsymbol{\omega}_{\swirlarrow}}  % swirl vorticity
\newcommand{\Om}{\Omega_{\swirlarrow}}                   % swirl angular frequency profile

\newcommand{\vscore}{v_{\swirlarrow}}                    % shorthand: |v_swirl| at r=r_c
\newcommand{\vnorm}{\lVert \mathbf{v}_{\mkern-2mu\scriptscriptstyle\boldsymbol{\circlearrowleft}} \rVert}               % swirl speed magnitude
\newcommand{\Ce}{\vswirl}                                % canonical swirl-speed constant

\newcommand{\rhof}{\rho_{\!f}}                           % effective fluid density
\newcommand{\rhoE}{\rho_{\!E}}                           % swirl energy density
\newcommand{\rhom}{\rho_{\!m}}                           % mass-equivalent density
\newcommand{\rc}{r_c}                                    % string core radius (swirl string radius)

\newcommand{\Lam}{\Lambda}                               % Swirl Coulomb constant
\newcommand{\alpg}{\alpha_g}                             % gravitational fine-structure analogue

\newcommand{\titlepageOpen}{
    \begin{titlepage}
        \thispagestyle{empty}
        \centering
        \Large \bfseries \papertitle \par \vspace{1cm}
        {\Large \itshape \textbf{Omar Iskandarani}\textsuperscript{\textbf{*}} \par}
        \vspace{0.5cm}
        {\today \par}
        \vspace{0.5cm}
}

\newcommand{\titlepageClose}{
        \vfill \raggedright \null
        \begin{picture}(0,0)
            \put(0,-45){  % Shift 200pt left, 40pt down
                \begin{minipage}[b]{0.7\textwidth} \footnotesize
                    \renewcommand{\arraystretch}{1.0}
                    \noindent\rule{\textwidth}{0.4pt} \\[0.5em]
                    \textsuperscript{\textbf{*}} Independent Researcher, Groningen, The Netherlands \\
                    Email: \texttt{info@omariskandarani.com} \\
                    ORCID: \texttt{\href{https://orcid.org/0009-0006-1686-3961}{0009-0006-1686-3961}} \\
                    DOI: \href{https://doi.org/\paperdoi}{\paperdoi}
                \end{minipage}
            }
        \end{picture}
    \end{titlepage}
}
%=========================================
% Start Document - Title Page
%=========================================
\begin{document}
    \titlepageOpen
        \begin{abstract}
    
        \end{abstract}
    \titlepageClose
%=========================================
% Title Page End
%=========================================
\section*{Circulation as a Fundamental Quantity in Physics and SST}

    Historically, the idea of \textit{circulation} as a primitive physical quantity dates back to 19th-century hydrodynamics. In 1858, Hermann von Helmholtz proved that vortex lines in an inviscid fluid move as indestructible tubes—their vorticity and circulation are conserved in time. A decade later, William Thomson (Lord Kelvin) formalized the conservation of circulation with his famous theorem:
    \begin{equation}
    \frac{d\Gamma}{dt} = 0
    \end{equation}
    for any material loop in a barotropic ideal fluid. Kelvin went further to propose that atoms might in fact be knotted vortex rings in a universal ether, each with a quantized circulation “charge”. This bold vortex atom model elevated the line-integral of velocity (circulation) to a candidate fundamental invariant of nature. The legacy of Helmholtz and Kelvin is a recognition that \textit{circulation can be treated as a basic building block}—a concept that Swirl String Theory (SST) adopts as a central postulate.

    In SST, matter is modeled as stable swirl strings (closed vortex filaments) in a pervasive incompressible medium. It is thus natural to take circulation (the strength of these vortices) as a primitive quantity. We seek a fundamental circulation constant, $\Gamma_0$, that characterizes the swirl medium. To pin down $\Gamma_0$ quantitatively, we start from a remarkable relativistic scale: the maximum electromagnetic force. Consider the ratio of an electron’s rest energy to its classical radius,
    \begin{equation}
    r_e = \frac{\alpha \hbar}{m_e c}
    \end{equation}
    This yields a force magnitude:
    \begin{equation}
    F_{\max} = \frac{m_e c^2}{r_e} = \frac{m_e^2 c^3}{\alpha \hbar} \approx 2.905 \times 10^1~\text{N}
    \end{equation}
    where $m_e$ is the electron mass, $c$ the speed of light, $\hbar$ Planck’s constant, and $\alpha \approx 1/137$ the fine-structure constant. This $F_{\max} \sim 29~\text{N}$ is an empirical force scale appearing in the SST canon (listed as the EM-sector maximum force). Physically, it represents the self-interaction force scale of an electron’s charge in classical electromagnetism, and in some formulations it’s viewed as an upper limit for force transmissible in that sector. In the context of the swirl medium, we posit that an equivalent force scale should emerge from the medium’s fundamental parameters. Dimensional analysis suggests that the combination $\rho_{!f} \Gamma_0^2$ (fluid density times circulation-squared) has units of force:
\begin{itemize}
        \item Dimensions check: $[\rho_{!f}\Gamma_0^2] = \left[\frac{M}{L^3}\right]\left[\frac{L^4}{T^2}\right] = \frac{M L}{T^2}$, the dimensions of force.
\end{itemize}

We therefore impose the relation
    \begin{equation}
    F_{\max} = \chi_F \rho_{!f} \Gamma_0^2
    \end{equation}
where $\rho_{!f}$ is the mass density of the swirl medium and $\chi_F$ is a dimensionless proportionality factor to be fixed by convention. Solving for the circulation constant gives
    \begin{equation}
    \Gamma_0 = \sqrt{\frac{F_{\max}}{\chi_F \rho_{!f}}}
    \end{equation}
    By choosing $\Gamma_0$ as a defining constant of the theory, we can absorb any order-unity factor $\chi_F$ into its value. In other words, $\Gamma_0$ is defined such that $\chi_F=1$, making it the exact circulation needed to reproduce the $29.05~\text{N}$ force when combined with the medium density. Using the calibrated SST density $\rho_{!f} = 7.0\times10^{-7}~\text{kg/m}^3$, we find:
\begin{itemize}
        \item Numerical value:
        \begin{equation}
        \Gamma_0 \approx \sqrt{\frac{29.05~\text{N}}{7.0\times10^{-7}~\text{kg/m}^3}} \approx 6.4\times10^3~\frac{\text{m}^2}{\text{s}}
        \end{equation}
\end{itemize}

    This fundamental circulation scale $\Gamma_0$ is on the order of $10^3~\text{m}^2/\text{s}$. Such a value may seem large compared to, say, the quantum of circulation in superfluid helium ($\sim 10^{-7}$~m$^2$/s), but it reflects the extremely low density of the vacuum medium in SST. Because $\rho_{!f}$ is so small, a much larger circulation is required to yield the same force—a direct consequence of the medium’s tenuousness. Indeed, $\rho_{!f}$ in SST is set fourteen orders below water’s density; intuitively, swirling such an “ultralight” medium demands a very strong circulation to accumulate significant inertia. The chosen $\Gamma_0$ ensures that $\rho_{!f}\Gamma_0^2$ matches a known physical scale ($29~\text{N}$), anchoring the theory’s single circulation constant to empirical reality. Moreover, this choice is consistent with particle physics scales: if we also take the core radius $r_c$ of a swirl string to be $1.4089\times10^{-15}$~m (a value calibrated to half the classical electron radius, as discussed below), then one finds
    \begin{equation}
    2 \rho_{!f} \Gamma_0^2 r_c \approx m_e c^2
    \end{equation}
    i.e., on the order of the electron’s rest energy ($\sim 0.511~\text{MeV}$). In essence, by fitting $\Gamma_0$ to $F_{\max}$ we simultaneously guarantee that the swirl energy available in one “quantum” of circulation (within a core-sized volume) is of the right order of magnitude for particle rest energies. This is a compelling consistency check: \textit{the circulation needed to hit the EM force limit, together with the tiny vacuum density and Fermi-scale core size, naturally encodes the rest-mass scale of matter}. No ad-hoc scales are introduced; $\Gamma_0$ is fixed by one physical requirement and then everything else falls into line.

    Having established $\Gamma_0$ as the circulation quantum of the swirl medium, we now examine the other two primitive constants that SST must posit: the medium’s mass density $\rho_{!f}$ and the vortex core length $r_c$. These serve as the natural complementary scales (of mass-density and length) alongside $\Gamma_0$ (of circulation) to span all relevant dimensions in the theory. Indeed, $(\Gamma_0, \rho_{!f}, r_c)$ can be viewed as an alternative fundamental trio in place of, say, $(c, \hbar, G)$—one tailored to a fluidic ontology:
\begin{itemize}
        \item \textbf{Swirl medium density} $\rho_{!f}$: This represents the inertial mass per volume of the vacuum condensate. It is an empirically calibrated constant in SST, set to $\rho_{!f}=7.0\times10^{-7}$~kg/m$^3$. Notably, this value corresponds to $1\times10^{-7}$ in SI units (mirroring the vacuum permeability $\mu_0/4\pi$) by design. The choice was made to anchor electromagnetic coupling in the swirl model—in effect, normalizing $\rho_{!f}$ such that the fluid’s response reproduces Coulombic strength. Physically, $\rho_{!f}$ is extremely small—about $10^{-19}$ times the density of water—reflecting a nearly void vacuum that nevertheless carries momentum and energy. Such a low density means the medium imparts only a feeble resistance to motion, which is why cosmic-scale circulation is needed to manifest substantial forces. In the SST Canon, $\rho_{!f}$ is treated as a fundamental constant (not derived from deeper theory) but its value is crucial for matching observed phenomena (e.g., atomic spectral lines, as it enters into formulas for wave impedance and force constants).
        \item \textbf{Core radius} $r_c$: This length scale sets the characteristic thickness of a swirl string—roughly the radius of the “tube” of circulating fluid that makes up a particle. SST takes $r_c$ to be on the order of a Fermi ($10^{-15}$~m), which is the size of a nucleon and coincidentally half the classical electron radius. In fact, the canonical value $r_c=1.40897\times10^{-15}$~m was chosen to calibrate the model’s mass predictions: by giving the smallest swirl (unknotted loop) a core of about $0.5 r_e$, the self-energy stored in its vortex matches the electron’s mass. The introduction of a finite $r_c$ is also essential for consistency—it regularizes the fluid vortex solutions by removing the singularity at $r=0$. In classical vortex dynamics, a finite core radius cuts off the $1/r$ velocity divergence, thereby capping the maximum swirl speed. The same is true in SST: $r_c$ provides an inner cutoff and ensures that the fluid velocity of a swirl string remains finite (and in practice sub-relativistic). Using the above values, the tangential speed at the core boundary is $v_{\text{core}} \sim \Gamma_0/(2\pi r_c)$. Plugging in $\Gamma_0 \approx 6.4\times10^3$~m$^2$/s and $r_c = 1.4\times10^{-15}$~m, we get $v_{\text{core}} \sim 1.1\times10^6$~m/s, about $3\times10^{-3}c$. This aligns with the characteristic swirl speed listed in the SST constants ($v \approx 1.09\times10^6$~m/s). Thus, the chosen $r_c$ guarantees that no point in the vortex approaches light-speed motion—a vital check on the theory’s internal consistency (no Lorentz violations at the mechanical level). Additionally, $r_c$ sets the scale of the smallest structures in the theory (the “thickness” of strings), playing a role analogous to the Planck length in quantum gravity or string length in string theory, albeit at a nuclear scale here.
\end{itemize}

    In summary, by starting from the electron’s force/mass scales and classical vortex principles, we justify the adoption of $(\Gamma_0, \rho_{!f}, r_c)$ as the primitive dimensional constants of SST. $\Gamma_0$ provides the link between circulation and force (anchoring the overall scale of vortex strength to known physics), while $\rho_{!f}$ and $r_c$ supply the natural density and length cutoffs to define the medium’s properties. \textit{All other dimensional quantities in SST can be constructed from these three:} for example, a characteristic energy scale can be formed as $\rho_{!f}\Gamma_0^2 r_c$ (on the order of $10^{-14}$~J, corresponding to hundreds of keV), and a characteristic time scale as $r_c^2/\Gamma_0$ (on the order of $10^{-30}$~s, which relates to high-frequency swirl excitations). In the next section, we formalize the circulation-based canon of SST by detailing these three primitive constants and showing how they underpin the theory’s postulates and benchmarks. This will lead directly into the section “Primitive dimensional constants,” where we enumerate $\Gamma_0$, $\rho_{!f}$, and $r_c$ and outline their defined values and roles in the SST framework.



%=========================================
% References
%=========================================
        \bibliographystyle{unsrt}
        \begin{thebibliography}{99}

            \bibitem{Einstein1905} A.~Einstein, \newblock \emph{Ist die Tr\"agheit eines K\"orpers von seinem Energieinhalt
            abh\"angig?}, newblock Ann.\ Phys.\ \textbf{18}, 639--641 (1905).

        \end{thebibliography}

\end{document}