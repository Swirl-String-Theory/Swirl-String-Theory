\documentclass[a4paper,10pt]{letter}

\usepackage[T1]{fontenc}
\usepackage[utf8]{inputenc}
\usepackage{lmodern}
\usepackage[hidelinks]{hyperref}
\usepackage{microtype}
\usepackage[margin=1in]{geometry}
\usepackage{amstext}

% Sender info
\signature{Omar Iskandarani\\
Independent Researcher, Groningen,\\ The Netherlands\\
ORCID: 0009-0006-1686-3961\\
Email: \href{mailto:info@omariskandarani.com}{info@omariskandarani.com}}
\address{Omar Iskandarani\\
Vinkenstraat 86A,\\
9713TK Groningen, \\The Netherlands}

\date{\today}

\begin{document}

    \begin{letter}{Editors\\\textit{Foundations of Physics}}
        \opening{Dear Editor,}

        I am pleased to submit the manuscript \textit{A Unified Electron Scale Relation from Classical Radius, Compton Frequency, and the Hydrogen Ground State Energy} for consideration in \textit{Foundations of Physics} as a Short Communication.

        The note presents a compact, algebraic identity linking three independently defined electron scales---the classical electron radius $r_e$, the Compton angular frequency $\omega_C$, and the hydrogen ground-state energy $E_B$. Using only textbook definitions and CODATA constants, a Hooke-law construction yields a maximal force $F_{\max}=m_e\!\left(\omega_C/\alpha\right)^2 r_e$ and identifies a Compton-scale radius for which $F_{\max}\,r_c=\tfrac{1}{2}m_ec^2=\tfrac{E_B}{\alpha^2}$. The derivation is strictly algebraic and dimensionally consistent, with a brief numerical check; no new dynamics or hypotheses are introduced.

        \textbf{Why \textit{Foundations of Physics}.}
        The result highlights a nontrivial consistency relation connecting classical electrodynamics, relativistic quantum mechanics, and atomic physics through fundamental constants. While modest in scope, it squarely addresses the journal's interest in the conceptual structure of physical theory and the interplay among established scales. The piece may also be useful pedagogically, as it exposes cross-domain structure typically taught in isolation.

        \textbf{Compliance.}
        The manuscript is original, not under consideration elsewhere, and all authorship/affiliation information is complete. Conflicts of interest: none. No external funding. Data and code are not applicable; all expressions derive from standard formulas and published constants (with full citations).

        Thank you for your consideration.

        \closing{Sincerely,}

    \end{letter}
\end{document}