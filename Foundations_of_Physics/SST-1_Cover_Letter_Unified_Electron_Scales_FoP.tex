\documentclass[a4paper,10pt]{letter}

\usepackage[T1]{fontenc}
\usepackage[utf8]{inputenc}
\usepackage{lmodern}
\usepackage[hidelinks]{hyperref}
\usepackage{microtype}
\usepackage[margin=1in]{geometry}
\usepackage{amstext}

% Sender info
\signature{Omar Iskandarani\\
Independent Researcher, Groningen,\\ The Netherlands\\
ORCID: 0009-0006-1686-3961\\
Email: \href{mailto:info@omariskandarani.com}{info@omariskandarani.com}}
\address{Omar Iskandarani\\
Vinkenstraat 86A\\
9713 TK Groningen\\
The Netherlands}

\date{\today}

\begin{document}

    \begin{letter}{Editors\\\textit{Foundations of Physics}}
        \opening{Dear Editor,}

        I hereby submit the manuscript \textit{On an Exact Identity Linking the Classical Electron Radius, Compton Frequency, and the Hydrogen Ground-State Energy} for consideration as a Short Communication in \textit{Foundations of Physics}.

        The note establishes a compact, purely algebraic identity relating three independently defined electron scales: the classical electron radius $r_e$, the Compton angular frequency $\omega_C$, and the hydrogen ground-state energy $E_B$. Using only textbook definitions and CODATA values for $m_e$, $\alpha$, $\hbar$, and $c$, a simple Hooke's-law oscillator with mass $m_e$, amplitude $r_e$, and frequency $\omega_\ast=\omega_C/\alpha$ defines a force scale
        \[
            F_{\max}=m_e\!\left(\tfrac{\omega_C}{\alpha}\right)^2 r_e,
        \]
        and shows that for a short Compton-scale length $r_c$ one obtains
        \[
            F_{\max} r_c=\tfrac{1}{2} m_e c^2=\tfrac{E_B}{\alpha^2}.
        \]
        The derivation is strictly algebraic and dimensionally consistent, and a brief numerical check using CODATA values is included. The manuscript is a short, self-contained note and does not introduce any new dynamics, interactions, or interpretational hypotheses.

        \textbf{Relevance to \textit{Foundations of Physics}.}
        The result highlights a nontrivial consistency relation intertwining classical electrodynamics, relativistic quantum relations, and atomic physics via fundamental constants. Although modest in scope, it speaks directly to the journal’s interest in conceptual structure and cross-domain coherence of physical theories. The communication may also serve a pedagogical role by exposing links between electron and hydrogen scales that are typically presented in isolation, and it can provide a clean benchmark identity for any theoretical framework in which these constants appear.

        \textbf{Availability and transparency.}
        A timestamped preprint has been deposited at Zenodo (DOI: \href{https://doi.org/10.5281/zenodo.17905207}{10.5281/zenodo.17905207}). The manuscript is original, not under consideration elsewhere, and all authorship and affiliation information is accurate. There are no conflicts of interest and no external funding. Data and code are not applicable; all expressions are derived from standard formulas and published constants with full citations.

        Thank you very much for considering this submission. I would be grateful if you would consider the manuscript for publication as a Short Communication in \textit{Foundations of Physics}.

        \closing{Sincerely,}

    \end{letter}
\end{document}