\documentclass[11pt]{article}

\usepackage{amsmath,amssymb,amsfonts,bm}
\usepackage{siunitx}
\usepackage[hidelinks]{hyperref}
\usepackage[a4paper,margin=1in]{geometry}
\usepackage[T1]{fontenc}
\usepackage[utf8]{inputenc}

\title{Swirl Pressure and Effective Gravitational Acceleration\\
in Incompressible Rotating Flows}
\author{Omar Iskandarani}
\date{\today}

\begin{document}
    \maketitle

    \begin{abstract}
        In incompressible, inviscid fluids, steady rotational motion generates radial pressure gradients that balance centripetal acceleration in the absence of external forces.
        In this paper we show that, for suitably chosen azimuthal velocity profiles, the associated pressure field can be written in the form of an effective gravitational potential whose radial derivative reproduces familiar Newtonian accelerations.
        Starting from the steady Euler equations for an axisymmetric swirling flow, we derive the radial balance
        \[
            \frac{1}{\rho}\frac{\mathrm{d}p}{\mathrm{d}r}
            = \frac{v_\theta^2(r)}{r} - g_r(r),
        \]
        and, in the absence of external gravity $g_r = 0$, identify an effective acceleration
        \(
        g_{\text{eff}}(r) = v_\theta^2(r)/r
        \)
        that acts on buoyant inclusions.
        For the azimuthal profile
        \(
        v_\theta(r) = \sqrt{GM/r}
        \),
        we obtain
        \(
        g_{\text{eff}}(r) = GM/r^2
        \)
        and a pressure field $p(r)$ whose contribution per unit mass, $p/\rho$, coincides (up to an additive constant) with the Newtonian potential $-GM/r$.
        We discuss the limitations of this analogy, emphasize that the resulting ``gravity'' is not universal but material-dependent, and relate the construction to analogue-gravity models based on moving media.
        The analysis is entirely within classical incompressible Euler flow and Newtonian gravity in the weak-field limit.
    \end{abstract}

    \section{Introduction}
        The equivalence between gravitational and inertial effects is a cornerstone of classical and relativistic physics~\cite{Einstein1916,MTW,Will2014}.
        In Newtonian mechanics, a gravitational field is described by a potential $\Phi(\bm{x})$ whose gradient gives the acceleration $\bm{g} = -\nabla\Phi$.
        In general relativity, gravitational effects are encoded in the spacetime metric, with Newtonian gravity recovered in the weak-field, slow-motion limit~\cite{MTW,LL_FieldTheory}.

        In parallel, classical fluid mechanics provides a rich set of rotating and swirling flows whose pressure fields support centripetal or centrifugal accelerations~\cite{Batchelor1967,LandauLifshitzFM,Acheson1990}.
        In steady axisymmetric rotation, radial pressure gradients balance the centripetal acceleration of fluid elements; small inclusions with density different from the ambient fluid experience net radial forces determined by these gradients.

        Analogue-gravity models exploit such moving media to mimic aspects of curved spacetime for perturbations such as sound waves or surface waves~\cite{Unruh1981,Visser1998,Barcelo2005}.
        In these models, an ``effective metric'' is derived from the background flow, leading to analogues of horizons and redshifts.

        The aim of this paper is more modest and sharply focused:
        \begin{enumerate}
            \item Starting from the steady incompressible Euler equations, we derive the radial force balance for axisymmetric swirling flows and identify a pressure-induced acceleration that can be written in the form of an effective gravitational field.
            \item We show that a specific choice of azimuthal velocity profile,
            \(
            v_\theta(r) = \sqrt{GM/r},
            \)
            leads to an effective acceleration $g_{\text{eff}}(r) = GM/r^2$ and a pressure field whose contribution per unit mass matches the Newtonian $-GM/r$ potential up to a constant.
            \item We discuss the physical limitations of this formal equivalence, in particular its dependence on material properties and the absence of a universal coupling analogous to the equivalence principle.
        \end{enumerate}
        The treatment remains strictly within the framework of classical incompressible fluid mechanics and Newtonian gravity.

    \section{Euler equations and radial balance in swirling flow}
        \subsection{Steady incompressible Euler equations}
            For an incompressible, inviscid fluid of constant density $\rho$, the Euler equations read~\cite{Batchelor1967,LandauLifshitzFM}
            \begin{align}
                \frac{\partial \bm{v}}{\partial t}
                + (\bm{v}\cdot\nabla)\bm{v}
                &= -\frac{1}{\rho}\nabla p + \bm{g}, \label{eq:Euler}\\
                \nabla\cdot\bm{v} &= 0, \label{eq:incompressible}
            \end{align}
            where $\bm{v}$ is the velocity field, $p$ the pressure, and $\bm{g}$ an external body-force field per unit mass (e.g.\ gravity).

            We consider steady flows, $\partial\bm{v}/\partial t = 0$, in which case Eq.~\eqref{eq:Euler} becomes
            \begin{equation}
            (\bm{v}\cdot\nabla)\bm{v}
            = -\frac{1}{\rho}\nabla p + \bm{g}.
            \label{eq:Euler-steady}
            \end{equation}

        \subsection{Axisymmetric swirling flow in cylindrical coordinates}
            Introduce cylindrical coordinates $(r,\theta,z)$ and consider an axisymmetric, purely azimuthal velocity field
            \begin{equation}
                \bm{v}(r) = v_\theta(r)\,\hat{\bm{\theta}},
                \qquad v_r = v_z = 0,
                \label{eq:swirl-ansatz}
            \end{equation}
            with no dependence on $\theta$ or $z$.
            This is a standard model for a swirling, incompressible flow with negligible meridional circulation~\cite{Batchelor1967,Acheson1990}.

            The convective acceleration $(\bm{v}\cdot\nabla)\bm{v}$ for this flow has only a radial component, corresponding to centripetal acceleration;
            one finds~\cite{Batchelor1967,LandauLifshitzFM}
            \begin{equation}
            (\bm{v}\cdot\nabla)\bm{v}
            = -\frac{v_\theta^2(r)}{r}\,\hat{\bm{r}}.
            \label{eq:centripetal}
            \end{equation}
            Substituting Eqs.~\eqref{eq:swirl-ansatz} and~\eqref{eq:centripetal} into Eq.~\eqref{eq:Euler-steady}, the radial component of the momentum equation becomes
            \begin{equation}
                -\frac{v_\theta^2(r)}{r}
                = -\frac{1}{\rho}\frac{\mathrm{d}p}{\mathrm{d}r}
                + g_r(r),
                \label{eq:radial-balance-general}
            \end{equation}
            where $g_r(r)$ is the radial component of $\bm{g}$.

            Equation~\eqref{eq:radial-balance-general} expresses the balance between centripetal acceleration, radial pressure gradient, and any external radial body force.
            It can be rewritten as
            \begin{equation}
                \frac{1}{\rho}\frac{\mathrm{d}p}{\mathrm{d}r}
                = \frac{v_\theta^2(r)}{r} - g_r(r).
                \label{eq:radial-balance}
            \end{equation}

    \section{Effective gravitational acceleration from swirl pressure}
        \subsection{Definition of an effective field}
            In the absence of external radial gravity, $g_r(r) = 0$, Eq.~\eqref{eq:radial-balance} reduces to
            \begin{equation}
                \frac{1}{\rho}\frac{\mathrm{d}p}{\mathrm{d}r}
                = \frac{v_\theta^2(r)}{r}.
                \label{eq:radial-no-gravity}
            \end{equation}
            From the perspective of a small inclusion of density $\rho_{\text{obj}}$ immersed in the fluid, the pressure gradient exerts a net radial force
            \begin{equation}
                F_r = -\int \frac{\partial p}{\partial r}\,\mathrm{d}A,
            \end{equation}
            where the integration is over the projected area.
            For a nearly spherical inclusion of volume $V$ and small density contrast $\Delta\rho = \rho - \rho_{\text{obj}}$, the leading-order radial acceleration can be written as
            \begin{equation}
                a_r \simeq -\frac{\Delta\rho}{\rho_{\text{obj}}}
                \frac{1}{\rho}\frac{\mathrm{d}p}{\mathrm{d}r},
            \end{equation}
            neglecting viscous effects and assuming the inclusion responds quasi-statically to the pressure gradient.

            This suggests defining an \emph{effective gravitational acceleration} associated with the swirl pressure as
            \begin{equation}
                g_{\text{eff}}(r) := \frac{v_\theta^2(r)}{r},
                \label{eq:g-eff-def}
            \end{equation}
            so that Eq.~\eqref{eq:radial-no-gravity} reads
            \begin{equation}
                \frac{1}{\rho}\frac{\mathrm{d}p}{\mathrm{d}r}
                = g_{\text{eff}}(r).
                \label{eq:radial-geff}
            \end{equation}
            Formally, one can introduce an effective potential $\Phi_{\text{eff}}(r)$ via
            \begin{equation}
                \Phi_{\text{eff}}(r) = -\frac{p(r)}{\rho} + \text{const.},
                \label{eq:phi-eff-def}
            \end{equation}
            so that
            \begin{equation}
                g_{\text{eff}}(r) = -\frac{\mathrm{d}\Phi_{\text{eff}}}{\mathrm{d}r}.
            \end{equation}
            Thus the pressure field in the swirling fluid plays the same mathematical role, for suitably defined test bodies, as a gravitational potential in Newtonian mechanics.

        \subsection{Choice of swirl profile for Newtonian $1/r^2$ acceleration}
            We now show that a particular swirl profile yields a Newtonian inverse-square effective acceleration.
            Consider the azimuthal velocity
            \begin{equation}
                v_\theta(r) = \sqrt{\frac{GM}{r}},
                \label{eq:vtheta-GM}
            \end{equation}
            where $G$ and $M$ are positive constants with the dimensions of the Newtonian gravitational constant and a mass, respectively.
            This profile is mathematically admissible as a steady solution of Eq.~\eqref{eq:radial-no-gravity} for an appropriate pressure field, although questions of stability and boundary conditions must be treated separately.

            Using Eq.~\eqref{eq:g-eff-def}, the effective acceleration is
            \begin{equation}
                g_{\text{eff}}(r) = \frac{v_\theta^2(r)}{r}
                = \frac{GM/r}{r}
                = \frac{GM}{r^2}.
                \label{eq:g-eff-1overr2}
            \end{equation}
            Thus the swirl-induced effective acceleration matches the radial dependence of Newtonian gravity outside a point mass $M$.

        \subsection{Effective potential and pressure field}
            Substituting Eq.~\eqref{eq:vtheta-GM} into Eq.~\eqref{eq:radial-no-gravity}, we obtain
            \begin{equation}
                \frac{1}{\rho}\frac{\mathrm{d}p}{\mathrm{d}r}
                = \frac{GM}{r^2}.
            \end{equation}
            Integrating with respect to $r$,
            \begin{equation}
                p(r) = -\rho \frac{GM}{r} + p_0,
                \label{eq:p-of-r}
            \end{equation}
            where $p_0$ is an integration constant.

            From Eq.~\eqref{eq:phi-eff-def}, the effective potential is
            \begin{equation}
                \Phi_{\text{eff}}(r)
                = -\frac{p(r)}{\rho} + \text{const.}
                = \frac{GM}{r} + \text{const.}
            \end{equation}
            Choosing the arbitrary constant appropriately, we can write
            \begin{equation}
                \Phi_{\text{eff}}(r) = -\frac{GM}{r},
            \end{equation}
            which is precisely the Newtonian gravitational potential of a point mass $M$ at the origin.

            Thus, at the level of the Euler equations and the radial pressure balance, a purely swirling incompressible flow with azimuthal profile~\eqref{eq:vtheta-GM} produces a pressure field whose contribution per unit mass has the same radial dependence as the Newtonian gravitational potential.

    \section{Free surface and effective potential in a rotating container}
        \subsection{Classical rotating-bucket paraboloid}
            A related and experimentally accessible example is the classical rotating-bucket experiment.
            Consider an incompressible, inviscid fluid rotating as a rigid body with angular velocity $\Omega$ about the vertical axis in a uniform gravitational field $\bm{g} = -g\,\hat{\bm{z}}$~\cite{LandauLifshitzFM,Acheson1990}.
            The velocity field is
            \begin{equation}
                \bm{v}(r) = \Omega r\,\hat{\bm{\theta}},
            \end{equation}
            and the steady Euler equations yield a pressure distribution such that the free surface $z = \eta(r)$ satisfies
            \begin{equation}
                g\,\eta(r) = \frac{\Omega^2 r^2}{2} + \text{const.},
            \end{equation}
            i.e.\ the free surface is a paraboloid.

            This configuration exemplifies how a combination of gravitational and centrifugal potentials determines the pressure field and the shape of the free surface.
            In the present context, we are interested in the complementary situation where external gravity is absent and the pressure field alone encodes an effective potential through Eq.~\eqref{eq:phi-eff-def}.

        \subsection{Analogy and limitations}
            In the swirl-induced effective-gravity construction, the role of the ``potential'' is played by $p/\rho$ rather than by a separate scalar field.
            The analogy is exact at the level of the Euler equation for the fluid and the motion of small buoyant inclusions, but it has important limitations:
            \begin{itemize}
                \item The force on an inclusion depends on the density contrast $\Delta\rho = \rho - \rho_{\text{obj}}$; there is no universal coupling independent of composition, in contrast with the equivalence principle in gravity~\cite{Will2014}.
                \item The effective potential arises from internal stresses in the medium rather than from spacetime geometry; it can be switched off by turning off the swirl.
                \item Stability and realizability of the required swirl profiles, such as Eq.~\eqref{eq:vtheta-GM}, are nontrivial and subject to various hydrodynamic instabilities~\cite{Acheson1990}.
            \end{itemize}
            These caveats emphasize that the ``gravitational equivalence'' discussed here is formal and context-dependent.

    \section{Relation to analogue gravity models}
        Analogue-gravity models interpret perturbations (e.g.\ sound waves) in moving media as propagating in an effective spacetime geometry determined by the background flow~\cite{Unruh1981,Visser1998,Barcelo2005}.
        For a barotropic, irrotational flow, the equation for linear acoustic perturbations can be recast as a covariant wave equation in a curved effective metric~\cite{Visser1998,Barcelo2005}.
        Horizons, redshifts, and other gravitational phenomena can then be mimicked in laboratory systems such as water channels, Bose--Einstein condensates, or optical media.

        The construction presented here differs in scope: we do not derive an effective metric for perturbations, but instead focus on the background pressure field and its role in determining the acceleration of small inclusions.
        Nevertheless, both approaches exploit the same underlying idea: the dynamics of fields or bodies in a structured medium can reproduce, at least formally, the effects of a gravitational potential or metric.

    \section{Discussion and outlook}
        We have shown that:
        \begin{enumerate}
            \item In an axisymmetric, purely swirling incompressible flow, the steady Euler equations lead to a radial balance relation, Eq.~\eqref{eq:radial-balance}, linking centripetal acceleration, pressure gradient, and external body forces.
            \item In the absence of external gravity, one can define an effective acceleration $g_{\text{eff}}(r) = v_\theta^2(r)/r$ and an effective potential $\Phi_{\text{eff}} = -p/\rho$ such that $g_{\text{eff}} = -\mathrm{d}\Phi_{\text{eff}}/\mathrm{d}r$.
            \item For the swirl profile $v_\theta(r) = \sqrt{GM/r}$, the effective acceleration takes the Newtonian inverse-square form $GM/r^2$, and the pressure field yields an effective potential identical in radial dependence to $-GM/r$.
        \end{enumerate}
        These results are obtained entirely within classical incompressible Euler flow and Newtonian gravity in the weak-field limit.
        They provide a precise sense in which the pressure field of a swirling fluid can act as an analogue of a gravitational potential for certain test bodies.

        At the same time, the analogy has clear limitations:
        the effective ``gravity'' is not universal, depends on material properties and boundary conditions, and does not arise from spacetime curvature.
        It is therefore best viewed as a useful conceptual and pedagogical tool, and as a starting point for more sophisticated analogue-gravity constructions, rather than as a replacement for gravitational theory.

        Future work within mainstream physics could address:
        \begin{itemize}
            \item stability and realizability of swirl profiles that mimic specific gravitational potentials;
            \item detailed trajectories of buoyant inclusions in such flows, including the role of viscosity and finite-size effects;
            \item connections between the present pressure-based analogy and acoustic-metric analogues, where perturbations propagate in an effective curved geometry.
        \end{itemize}

    \section*{Acknowledgments}
        The author thanks the authors of standard texts on fluid mechanics, gravity, and analogue models for providing the theoretical background on which this analysis is based.

        \bibliographystyle{unsrt}
        \begin{thebibliography}{99}

            \bibitem{Einstein1916}
            A.~Einstein,
            \newblock \emph{Die Grundlage der allgemeinen Relativit\"atstheorie},
            \newblock Ann.\ Phys.\ \textbf{49}, 769--822 (1916).

            \bibitem{MTW}
            C.~W. Misner, K.~S. Thorne, and J.~A. Wheeler,
            \newblock \emph{Gravitation},
            \newblock W.~H. Freeman, San Francisco (1973).

            \bibitem{LL_FieldTheory}
            L.~D. Landau and E.~M. Lifshitz,
            \newblock \emph{The Classical Theory of Fields},
            \newblock 4th ed., Butterworth--Heinemann, Oxford (1975).

            \bibitem{Will2014}
            C.~M. Will,
            \newblock \emph{The Confrontation between General Relativity and Experiment},
            \newblock Living Rev.\ Relativ.\ \textbf{17}, 4 (2014).

            \bibitem{Batchelor1967}
            G.~K. Batchelor,
            \newblock \emph{An Introduction to Fluid Dynamics},
            \newblock Cambridge University Press, Cambridge (1967).

            \bibitem{LandauLifshitzFM}
            L.~D. Landau and E.~M. Lifshitz,
            \newblock \emph{Fluid Mechanics},
            \newblock 2nd ed., Pergamon Press, Oxford (1987).

            \bibitem{Acheson1990}
            D.~J. Acheson,
            \newblock \emph{Elementary Fluid Dynamics},
            \newblock Oxford University Press, Oxford (1990).

            \bibitem{Unruh1981}
            W.~G. Unruh,
            \newblock \emph{Experimental Black-Hole Evaporation?},
            \newblock Phys.\ Rev.\ Lett.\ \textbf{46}, 1351--1353 (1981).

            \bibitem{Visser1998}
            M.~Visser,
            \newblock \emph{Acoustic black holes: Horizons, ergospheres, and Hawking radiation},
            \newblock Class.\ Quantum Grav.\ \textbf{15}, 1767--1791 (1998).

            \bibitem{Barcelo2005}
            C.~Barcel\'o, S.~Liberati, and M.~Visser,
            \newblock \emph{Analogue Gravity},
            \newblock Living Rev.\ Relativ.\ \textbf{8}, 12 (2005).

        \end{thebibliography}

\end{document}