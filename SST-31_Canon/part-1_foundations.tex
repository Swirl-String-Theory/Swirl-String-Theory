%========================================================================================
% PART I: FOUNDATIONS (Core Canon Spine)
%========================================================================================
\part{Foundations}

%===============================
% Sevenfold Genesis of the Swirling Cosmos
%===============================
\section{Sevenfold Genesis of the Swirling Cosmos}
\label{sec:sevenfold-genesis}

\noindent
\textbf{Canonical Cosmogony (SST · STC Mapped)}
This section compresses the sixteen stages of \emph{The Simplicity Codex}~\cite{Goldau2025_STC}
into seven logically complete emergence stages, consistent with the SST Canon~\canonversion{}
and the Lagrangian EFT~\cite{Iskandarani2025_Lagrangian}.
Each stage represents a parameter-free imprint of physical law onto the condensate.

\subsection*{Stage 1: Logical Substrate (Pre-Swirl Potential)}
A Haar-neutral, scale-free potential field encodes possible circulation states.
No time or space exist yet — only relational templates.
Global $\mathbb{Z}_2$ (chirality) and $\mathbb{Z}_3$ (triadic closure) symmetries are imprinted as pre-physical rules:
\[
\Gamma \mapsto -\Gamma,
\qquad
\Gamma_1 + \Gamma_2 + \Gamma_3 = 0 \pmod{2\pi}
\]
laying the groundwork for matter/antimatter duality and baryon triplet stability.
\hfill (STC Stages 1–3)

\subsection*{Stage 2: Big Condensation (First Manifestation)}
When the information complexity $\mathcal{I}_{\mathrm{swirl}}$ exceeds the Guardian threshold,
the swirl condensate forms as an incompressible, inviscid medium on $\mathbb{R}^3$ with absolute time $t$.
Primary constants lock in by resonance:
\[
\Gamma_0 = 2\pi r_c \vnorm,
\qquad
\tau_{\mathrm{beat}} = \frac{2\pi r_c}{\vnorm},
\qquad
\rho_{\!f} = \frac{\rho_{\!m} r_c}{\vnorm}\,\Omega
\]
marking the birth of physical time and the circulation quantum.
\hfill (STC Stage 8)

\textbf{Technical details:} See Section~\ref{canon58:classical-invariants} (Classical Invariants) and Section~\ref{canon58:rho_f} (Effective Medium) for derivations.

\subsection*{Stage 3: Tangible Chirality and Swirl-Time}
Knotted swirl strings appear, stabilized by circulation quantization $\Gamma = n\Gamma_0$.
The swirl clock
\[
S_t = \sqrt{1 - \frac{v^{2}}{c^{2}}}
\]
defines local proper time, with left-handed ($\circlearrowleft$) and right-handed ($\circlearrowright$) knots forming the basis for matter and antimatter.
\hfill (STC Stage 9)

\textbf{Technical details:} See Section~\ref{canon58:classical-invariants} (Classical Invariants) and Section~\ref{sec:cosmogony-seven} (Core Axioms) for the formal framework.

\subsection*{Stage 4: Topological Charges and Particle Spectrum}
Topological invariants of knots $(\mathrm{Lk}, \mathrm{Wr}, \mathrm{Tw})$ map to quantum numbers:
\[
Q(K) = T_3(K) + \frac{Y(K)}{2}
\]
with $Q$ the electric charge, $T_3$ weak isospin, and $Y$ hypercharge.
Fermion masses arise as soliton energies:
\[
m_{K} = \rho_{\!f} \vnorm^{2}\, \Vol_{\!\mathbb{H}}(K)\,\phi^{-2k}
\]
where $\Vol_{\!\mathbb{H}}(K)$ is the hyperbolic complement volume of $K$
and $\phi^{-2k}$ encodes Golden-layer suppression.
\hfill (STC Stage 10)

\textbf{Technical details:} See Section~\ref{sec:knot-taxonomy} (Knot Taxonomy) for the particle–knot mapping, and Section~\ref{canon58:hydro-grav} (Hydrogen–Gravity) for mass derivations.

\subsection*{Stage 5: Emergent Interactions (Gauge and Forces)}
Unknotted excitations of the condensate form the R-phase modes (photons, gluons, W/Z),
with interactions governed by the emergent gauge group:
\[
\mathfrak{g}_{\mathrm{swirl}} \;\simeq\;
\mathfrak{su}(3) \oplus \mathfrak{su}(2) \oplus \mathfrak{u}(1)
\]
and minimal coupling
\[
D_\mu = \nabla_\mu + i g_{\mathrm{sw}} W_{\mu}^{a}T^{a}\,.
\]
\hfill (STC Stage 11)

\textbf{Technical details:} See Section~\ref{canon58:swirl-em} (Swirl–EM Emergence) and Section~\ref{canon58:lagrangian} (Unified SST Lagrangian) for the formal derivation.

\subsection*{Stage 6: Geometric Closure and Constant Lock-In}
Global Gauss closure yields the $1/r^{2}$ force law:
\[
\nabla \cdot \vec{P}_{\mathrm{swirl}} = 0
\quad\Rightarrow\quad
F(r) \propto \frac{1}{r^{2}}
\]
and fixes $\pi$ geometrically.
The entire condensate enters a global resonance, locking all constants of nature.

\textbf{Technical details:} See Section~\ref{canon58:pressure} (Swirl Pressure Law) and Section~\ref{canon58:gauge-openers} (Gauge/EWSB Sector) for force laws and constant determinations.

\begin{tcolorbox}[title=Zero–Parameter Principle (Canonical),colframe=blue!75!black]
\textbf{Statement (Axiom):}
All dimensional constants of nature are determined by the condensate state, its circulation quantum,
and the allowed topological sectors. We take as primitive the circulation-based triplet
\[
(\Gamma_0,\rho_{\!f},r_c),
\]
where $\Gamma_0$ is the circulation quantum, $\rho_{\!f}$ the effective fluid density, and $r_c$ the electron-scale
core radius. All other dimensional quantities in SST (masses, charges, energies, forces) are derived
combinations of $(\Gamma_0,\rho_{\!f},r_c)$ and topology-dependent dimensionless factors.

The canonical swirl speed at the core boundary is not independent but given by
\[
\lVert \mathbf{v}_{\!\boldsymbol{\circlearrowleft}}\rVert
  = \chi_v\,\frac{\Gamma_0}{2\pi r_c},
\]
so that the former primitive set $(\lVert \mathbf{v}_{\!\boldsymbol{\circlearrowleft}}\rVert,r_c,\rho_{\!f})$
is just a reparametrization of $(\Gamma_0,\rho_{\!f},r_c)$.

\begin{aligned}
\text{Primary Scale:}\qquad &
\Gamma_0 \approx 6.4\times 10^{3}~\mathrm{m^2/s},
\qquad
\kappa_{\text{SST}} \equiv \Gamma_0 = 2\pi r_c \vnorm \\[4pt]
\text{Effective Density:}\qquad &
\rho_{\!f} = \frac{\rho_{\!m}\, r_c}{\vnorm}\, \Omega
\qquad (\text{coarse-grain rule}) \\[4pt]
\text{Mass Functional:}\qquad &
m_K = \rho_{\!f} \vnorm^{2}\, \Vol_{\!\mathbb{H}}(K)\,\phi^{-2k} \\[4pt]
\text{Gravitational Coupling:}\qquad &
G_{\mathrm{swirl}} = \frac{\vnorm c^{5} t_p^{2}}{2 F_{\mathrm{max}} r_c^{2}} \\[4pt]
\text{Fine-Structure Constant:}\qquad &
\alpha = \alpha_{\mathrm{DSI}}\!\bigl(\omega_{\mathrm{DSI}}\bigr),
\qquad
\omega_{\mathrm{DSI}} \approx 13.06
\end{aligned}

\textbf{Corollary:}
Once $\Gamma_0$, $r_c$, and $\rho_{\!f}$ are fixed by a single calibration
(e.g. $m_e$), the full mass spectrum and coupling strengths follow with no free parameters.
See STC Stages 12–13 (Gauss-closure \& Universal Resonance)~\cite{Goldau2025_STC}.
\end{tcolorbox}

\subsection*{Stage 7: Recursive Cosmos (Fractal Emergence)}
Composite knots (baryons, nuclei, atoms) satisfy $\mathbb{Z}_3$ closure,
1+12 isotropic shielding, and duality pairing.
Each stable composite becomes a new circulation source:
\[
\mathrm{Cluster} \;\Rightarrow\;
\mathrm{Meta\text{-}Knot} \;\Rightarrow\;
\mathrm{New\ Swirl\ Layer}
\]
seeding the next scale of complexity.
This recursion drives cosmic structure formation, yielding a fractal universe of knots within knots.
\hfill (STC Stages 14–16)


%========================================================================================
% CANON GOVERNANCE & STATUS TAXONOMY (Canonical)
%========================================================================================
    \section{Canon Governance and Status Taxonomy}\label{canon58:governance}
        \paragraph{Formal system.}
            Let $\mathcal{S}=(\mathcal{P},\mathcal{D},\mathcal{R})$ denote the SST formal system: axioms $\mathcal{P}$, definitions $\mathcal{D}$, and admissible inference rules $\mathcal{R}$ (variational principles, Noether currents, dimensional analysis, asymptotic matching).
        \paragraph{Canonical statement.}
            A statement $X$ is \emph{canonical} iff
            \[
                \mathcal{P},\mathcal{D}\vdash_{\mathcal{R}} X\,,
            \]
            and $X$ is consistent with accepted canon. % Check: [units N/A; limit → none]
        \paragraph{Empirical statement.}
            A statement $Y$ is \emph{empirical} iff it asserts a measured value or protocol:
            \[
                Y \equiv \text{``observable $\mathcal{O}$ has value $\hat{o}\pm\delta o$ under procedure $\Pi$.''}
            \]
% Check: [units ok; limit → none]
    \subsection*{Status Classes}
    \begin{itemize}
    \item \textbf{Axiom/Postulate (Canonical).} Primitive assumption of SST.
    \item \textbf{Definition (Canonical).} Introduces a symbol by construction.
    \item \textbf{Theorem/Corollary (Canonical).} Proven consequence within $\mathcal{S}$.
    \item \textbf{Constitutive Model.} Canonical if derived from $\mathcal{P},\mathcal{D}$; otherwise semi-empirical.
    \item \textbf{Calibration (Empirical).} Recommended numerical values for canonical symbols.
    \item \textbf{Research Track.} Conjectures or alternatives pending proof or axiomatization.
    \end{itemize}
    Items may be promoted or demoted between classes only upon satisfying or failing the Canonicality Tests.

    \subsection*{Canonicality Tests (all required)}
    \begin{enumerate}
    \item \textbf{Derivability} from $\mathcal{P},\mathcal{D}$ via $\mathcal{R}$.
    \item \textbf{Dimensional consistency} (strict SI usage; correct physical limits).
    \item \textbf{Symmetry compliance} (Galilean symmetry and incompressibility).
    \item \textbf{Recovery limits} (Newtonian gravity, Coulomb/Bohr, linear wave optics).
    \item \textbf{Non-contradiction} with accepted canonical results.
    \item \textbf{Parameter discipline} (no ad hoc fits beyond calibrations).
    \end{enumerate}
% --- CK corollary: add assumption note + cite ---
    \begin{tcolorbox}[title=Corollary: Clock--Radius Transport]
    \[
        \frac{dS_t}{dt}=\frac{2(1-S_t^2)}{S_t}\,\frac{1}{R}\frac{dR}{dt}.
    \]
    \textit{Assumption:} thin filament with local solid-body swirl $v_\theta\simeq \omega r$ evaluated at $r=r_c$, so that
    $S_t=\sqrt{1-(\omega r_c/c)^2}$ along the core \cite{Batchelor1967,Saffman1992}. % Canonical
    \end{tcolorbox}

% [STATUS: Canonical] [SOURCE: earlier Canon draft]
% [Knot taxonomy figure moved to Section~\ref{sec:knot-taxonomy} in Part II]

	\section{Core Axioms (SST)}
	\label{sec:cosmogony-seven}
	SST is built on a set of core axioms that establish its physical framework. These axioms, numbered below, are stated in plain language and form the starting postulates of the theory (they are considered \emph{canonical} by definition).
%================================================
% Axiom 0: Logical Substrate (Pre-Swirl Potential)
%================================================
    \begin{tcolorbox}[title=Axiom 0: Logical Substrate (Pre-Swirl Potential)]
    \label{axiom:logical-substrate}
    Before the emergence of space, time, or condensate, there exists a
    \emph{Haar-neutral}, scale-free state space $\mathcal{S}$ of possible circulation states
    $\{\Gamma_i\}$. This pre-physical substrate encodes only relational constraints:
    \[
        \Gamma \mapsto -\Gamma,
        \qquad
        \Gamma_1 + \Gamma_2 + \Gamma_3 = 0 \ (\mathrm{mod}\ 2\pi),
    \]
    representing a global $\mathbb{Z}_2$ \emph{chirality symmetry} and
    $\mathbb{Z}_3$ \emph{triadic closure}.
    No metric structure (no lengths, durations, or energies) is yet defined.
    This axiom specifies that:
    \begin{enumerate}
    \item Circulation states are allowed only in $\pm$ pairs (matter/antimatter duality).
    \item The sum of any three circulations must close to zero modulo $2\pi$, ensuring
    triplet stability (precursor to baryon confinement).
    \item Any potential $V[\Gamma]$ defined on $\mathcal{S}$ must satisfy
    $V[\Gamma]=V[-\Gamma]$ and $V[\Gamma_1,\Gamma_2,\Gamma_3] =
    V[\Gamma_1+\Gamma_2+\Gamma_3\ (\mathrm{mod}\ 2\pi)]$.
    \end{enumerate}
    This stage is purely ontological: it fixes the logical rule set within which the
    swirl condensate (Stage~2) will later form and evolve.
    \end{tcolorbox}

% Citation for STC mapping
    \noindent\textbf{Citation:}
    See \emph{The Simplicity Codex}~\cite{Goldau2025_STC} (Stages~1--3: Primordial Symmetry
    and Triadic Closure) for the information-theoretic basis of these constraints.

	\begin{enumerate}\itemsep 4pt
	\item \textbf{Swirl Medium (Absolute Space-Time):}\label{axiom:swirl-medium} Physics is formulated in Euclidean $\mathbb{R}^3$ space with an absolute time parameter. All dynamics occur in a frictionless, incompressible condensate called the \emph{swirl medium}, which acts as a universal substratum for motion (analogous to a perfect fluid with no viscosity or compressibility).
	\item \textbf{Swirl Strings (Circulation \& Topology):}\label{axiom:swirl-strings} Particles and field quanta correspond to closed vortex filaments ("swirl strings") in the medium. Each such filament may be knotted or linked. The circulation of the swirl velocity field $\vswirl$ around any closed loop $C$ is quantized in integer multiples of the circulation quantum $\Gamma_0$:
	\[
		\Gamma \;=\; \oint_{C} \vswirl \cdot d\ell \;=\; n\,\Gamma_0, \qquad n\in \mathbb{Z}\,,
	\]
	where $\Gamma_0$ is the primitive circulation quantum (approximately $6.4\times 10^{3}~\mathrm{m^2/s}$). In addition to circulation quantization, the allowed configurations of a swirl string are restricted to distinct knot topologies. Thus, discrete quantum numbers (e.g. mass, charge, spin) are identified with topological invariants of the string (such as linking number, writhe, and twist) rather than with eigenstates of operators.

	\textit{Rosetta remark.} In the linear mapping to conventional superfluid notation, the circulation quantum
	$\Gamma_0$ matches the Onsager–Feynman value $h/m_{\text{eff}}$ for the relevant excitation, but within SST
	we treat $\Gamma_0$ as primitive and regard $h$ as a derived quantity.
	\item \textbf{String-Induced Gravitation:}\label{axiom:string-gravitation} Macroscopic gravitational attraction emerges as an effective force resulting from coherent swirl flows and pressure gradients in the medium. In the non-relativistic limit, the effective gravitational coupling $G_{\text{swirl}}$ is fixed by canonical constants such that $G_{\text{swirl}} \approx G_N$ (Newton’s gravitational constant). In essence, what we perceive as gravity is a statistical effect of many swirl strings and their pressure fields rather than a fundamental spacetime curvature.
	\item \textbf{Swirl Clocks (Local Time Dilation):}\label{axiom:swirl-clocks} The local proper time in a region of the swirl medium depends on the swirl speed in that region. A clock comoving with a swirl string (tangential speed $v$) ticks slower than a clock at rest in the medium by the \emph{swirl clock factor}
	\[
		S_t \;=\; \sqrt{\,1 - \frac{v^2}{c^2}\,}\,,
	\]
	analogous to special relativistic time dilation. Higher swirl velocities (and thus higher local swirl energy density) cause deeper time dilation (slower clocks) relative to an observer at infinity.
	\item \textbf{Dual Phases (Wave–Particle Complementarity):}\label{axiom:dual-phases} Each swirl string has two limiting dynamical phases. In the \emph{R-phase} (“radiative” or \emph{wave-like} phase), the string is unknotted and its circulation is delocalized over an extended loop. In the \emph{T-phase} (“tangible” or \emph{particle-like} phase), the string is knotted and its circulation is localized, carrying rest-mass. Quantum wave–particle duality in SST is thus realized as the ability of a swirl string to transition between these two phases. A quantum measurement corresponds to a rapid transition from an R-phase state to a T-phase state ($R\to T$ “collapse”) or vice versa ($T\to R$ de-localization), typically accompanied by emission or absorption of small swirl excitations (swirl radiation).
    \item \textbf{Canonical Taxonomy (Particle–Knot Mapping):}\label{axiom:taxonomy}
    There is a one-to-one mapping between the topological class of a swirl string and the type of particle or field it represents.
    Delocalized R-phase excitations correspond to unknotted swirl strings and represent massless bosonic quanta — with photons realized as \emph{pulsed torsional oscillations} of the swirl director field (carrying helicity $\pm 1$) rather than static knots.
    Nontrivial torus knots correspond to leptons (e.g. the electron is represented by the trefoil $3_1$ knot).
    Chiral hyperbolic knots (with non-zero writhe) correspond to quarks: we assign the up quark to the $5_2$ knot and the down quark to the $6_1$ knot.
    Baryons are realized as composite linkages of three quark knots: for instance, the proton is $p = (5_2 + 5_2 + 6_1)$ and the neutron $n = (5_2 + 6_1 + 6_1)$, with a color-flux linkage ensuring confinement.
    Linked or nested composite knots describe nuclei and bound states, providing SST with a built-in “periodic table” of matter.

	\end{enumerate}



% [Sidebar: Knot taxonomy diagram -- illustrate unknotted loop (photon), trefoil knot (proton/quark), etc.]
        \begin{figure}[htbp]
        \centering
        \includegraphics[width=0.7\linewidth]{figures/3quarcks}
        \caption{Sst three knot 180 speed stagnation}
        \label{fig:sst-three-knot-180-speed-stagnation}
        \end{figure}

	These axioms define the ontological starting point of SST. The swirl medium (Axiom 1) provides the arena, swirl strings (Axiom 2) provide the basic degrees of freedom with quantized circulation and allowed topologies, and the remaining axioms posit how classical forces and quantum behaviors emerge from this framework (gravity from collective flows, time dilation from swirl motion, wave–particle dual phases, and a topological classification of particles).
    \newpage