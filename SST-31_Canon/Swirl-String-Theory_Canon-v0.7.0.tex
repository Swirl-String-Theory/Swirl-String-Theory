%! Author = Omar Iskandarani
%! Title = Swirl String Theory (SST) Canon v0.5
%! Date = Dec 12, 2025
%! Affiliation = Independent Researcher, Groningen, The Netherlands
%! License = © 2025 Omar Iskandarani. All rights reserved. This manuscript is made available for academic reading and citation only. No republication, redistribution, or derivative works are permitted without explicit written permission from the author. Contact: info@omariskandarani.com
%! ORCID = 0009-0006-1686-3961
%! DOI = 10.5281/zenodo.17899592
% NOTE(v0.6.1): Appendix B corrected (near-field Euler gives ~1/r^2; far-field 1/r from Poisson mediator).

\newcommand{\canonversion}{v0.7.0} % Semantic versioning: vMAJOR.MINOR.PATCH
\newcommand{\papertitle}{\textbf{Swirl String Theory (SST) Canon \canonversion:} \\
                         \Large Quantum Measurement, Time, Gravity, and Atomic Mass\\ from Topological Defects}
\newcommand{\paperdoi}{10.5281/zenodo.17899592}

%========================================================================================
% PACKAGES AND DOCUMENT CONFIGURATION
%========================================================================================
\documentclass[10pt,reprint,aps,onecolumn,nofootinbib]{revtex4-2}

% ====== minimal packages ======

\usepackage[utf8]{inputenc}
\usepackage[T1]{fontenc}
\usepackage{amsmath, amssymb, amsfonts}
\usepackage{geometry}
\usepackage{graphicx}
\usepackage{hyperref}
\usepackage{physics}
\usepackage{xcolor}
\usepackage{textcomp}
\usepackage{amsthm}

\usepackage{bm}
\usepackage{microtype}
\usepackage{tcolorbox}
\hypersetup{colorlinks=true,linkcolor=blue,citecolor=blue,urlcolor=blue}

% ==== Packages ====
\usepackage{lmodern}
\usepackage{booktabs}
\usepackage{tikz}
\usetikzlibrary{arrows.meta,positioning,calc,fit,decorations.pathmorphing}
% Tables and Figures
\usepackage{float}
\usepackage{import}

% ==========================
% PATCH 0: include utilities
% ==========================
\usepackage{appendix}

% ==========================
% PATCH 0b: theorem-like envs
% ==========================
\newtheorem{definition}{Definition}[section]
\newtheorem{proposition}{Proposition}[section]
\newtheorem{remark}{Remark}[section]

% Page geometry
\geometry{top=2.5cm, bottom=2.5cm, left=2.5cm, right=2.5cm}

% Hyperlink setup
\hypersetup{
    colorlinks=true,
    linkcolor=blue,
    citecolor=red,
    urlcolor=blue
}



% TOC Customization
\usepackage{tocloft}
\setcounter{tocdepth}{1}
\renewcommand{\cftsecfont}{\footnotesize}
\renewcommand{\cftsubsecfont}{\footnotesize\itshape}
\renewcommand{\cftsecleader}{\cftdotfill{5}}
% Simple compact TOC macro (tight line spacing)
%\renewcommand\cftsecpresnum{Chapter\space} % name before title
%\setlength{\cftsecnumwidth}{\widthof{Chapter XX: }}% calculate width to leave enough room
\newcommand{\compacttoc}{%
  \begingroup
  \setlength{\cftparskip}{0pt}%
  \setlength{\cftbeforesecskip}{0.1pt}%
  \setlength{\cftbeforesubsecskip}{0.1pt}%
  \renewcommand{\cftsecfont}{\scriptsize}%
  \renewcommand{\cftsubsecfont}{\scriptsize\itshape}%
  \renewcommand{\cftsecpagefont}{\scriptsize}%
  \renewcommand{\cftsubsecpagefont}{\scriptsize}%
  \setlength{\cftsecnumwidth}{3.2em}%
  \setlength{\cftsubsecindent}{1.0em}%
  \renewcommand{\cftdotsep}{2}%
  \setlength{\parskip}{0pt}%
  \linespread{0.70}\selectfont
  \tableofcontents
  \endgroup
}

% ===== Gauge sector macros =====
\newcommand{\Tr}{\mathrm{Tr}}
\newcommand{\ii}{\mathrm{i}}
% Gauge fields (adjoints; indices a=1..8, i=1..3)
\newcommand{\GsA}{G^a_{\mu\nu}}
\newcommand{\WsI}{W^i_{\mu\nu}}
\newcommand{\Bmn}{B_{\mu\nu}}
\sloppy
% ===============================
% Macros (canonicalized)
% ===============================
%=== SST macros (minimal set for this snippet) =========================

% swirl arrows (context-aware)
\newcommand{\swirlarrow}{ \mathchoice{\mkern-2mu\scriptstyle\boldsymbol{\circlearrowleft}}{\mkern-2mu\scriptscriptstyle\boldsymbol{\circlearrowleft}}}
\newcommand{\vswirl}{\mathbf{v}_{\!\boldsymbol{\circlearrowleft}}}
\newcommand{\SwirlClock}{S_{(t)}^{\swirlarrow}}
\newcommand{\Fmaxswirl}{F^{\max}_{\mkern-1mu\scriptscriptstyle\boldsymbol{\circlearrowleft}}}
% swirl arrows Counter Clockwise
\newcommand{\swirlarrowcw}{ \mathchoice{\mkern-2mu\scriptstyle\boldsymbol{\circlearrowright}}{\mkern-2mu\scriptscriptstyle\boldsymbol{\circlearrowright}}}
\newcommand{\vswirlcw}{\mathbf{v}_{\swirlarrowcw}}
\newcommand{\SwirlClockcw}{S_{(t)}^{\swirlarrowcw}}
\newcommand{\Fmaxswirlcw}{F^{\max}_{\mkern-1mu\scriptscriptstyle\boldsymbol{\circlearrowright}}}

\newcommand{\Fmax}{\Fmaxswirl} % default maximal force (left swirl)
\newcommand{\FmaxEM}{F^{\max}_{\mathrm{EM}}}
\newcommand{\FmaxG}{F_{\mathrm{G}}^{\max}}               % G-like maximal force scale

\newcommand{\omegas}{\boldsymbol{\omega}_{\swirlarrow}}  % swirl vorticity
\newcommand{\Om}{\Omega_{\swirlarrow}}                   % swirl angular frequency profile

\newcommand{\vscore}{v_{\swirlarrow}}                    % shorthand: |v_swirl| at r=r_c
\newcommand{\vnorm}{\lVert \mathbf{v}_{\mkern-2mu\scriptscriptstyle\boldsymbol{\circlearrowleft}} \rVert}               % swirl speed magnitude
\newcommand{\Ce}{\vswirl}                                % canonical swirl-speed constant


\newcommand{\rhof}{\rho_{\!f}}                           % effective fluid density
\newcommand{\rhoF}{\rho_{\!f}}
\newcommand{\rhoE}{\rho_{\!E}}                           % swirl energy density
\newcommand{\rhom}{\rho_{\!m}}                           % mass-equivalent density
\newcommand{\rhoM}{\rho_{\!m}}     % mass-equivalent density
\newcommand{\rc}{r_c}                                    % string core radius (swirl string radius)

\newcommand{\Lam}{\Lambda}                               % Swirl Coulomb constant
\newcommand{\alpg}{\alpha_g}                             % gravitational fine-structure analogue

\newcommand{\Golden}{\phi}
\newcommand{\GoldenSq}{\phi^{2}}


% Status tags (house style)
\newcommand{\statusResearch}{\textsf{[Research-track]}}
\newcommand{\statusCalibration}{\textsf{[Calibration]}}
\newcommand{\statusCanonical}{\textsf{[Canonical clarification]}}

\newcommand{\omegaVec}{\boldsymbol{\omega}}

\newcommand{\OmegaCore}{\Omega_{\mathrm{core}}}
\newcommand{\bg}{\mathrm{bg}}
\newcommand{\core}{\mathrm{core}}
\newcommand{\Vol}{\operatorname{Vol}}   % now \Vol_{\!\mathbb{H}}(K) works

% ===============================
% Policy: the golden constant is only allowed via hyperbolic functions.
\newcommand{\xig}{\operatorname{asinh}\!\left(\tfrac{1}{2}\right)}
\newcommand{\phig}{\exp(\xig)}
\newcommand{\phialg}{\bigl(1+\sqrt{5}\bigr)/2}
\newcommand{\xigold}{\tfrac{3}{2}\,\xig}
\newcommand{\GoldenDeclare}{%
    \textbf{Golden (hyperbolic)}:\ \(\ln\phi=\xig\), hence \(\phi=\phig\).
    \ \emph{(Algebraic form \(\phi=\phialg\) is equivalent.)}%
}

\newcommand{\vswirltext}{\mathbf{v}_{\mathrm{swirl}}}

% Misc
\newcommand{\dd}{\mathrm{d}}
\newcommand{\ee}{\mathrm{e}}
% Theorem-like environments (only if absent)
\makeatletter
\@ifundefined{theorem}{\newtheorem{theorem}{Theorem}}{}
\@ifundefined{corollary}{\newtheorem{corollary}{Corollary}}{}
\@ifundefined{definition}{\newtheorem{definition}{Definition}}{}
\@ifundefined{lemma}{\newtheorem{lemma}{Lemma}}{}
\makeatother

\newcommand{\rhocore}{\rho_{\text{core}}}
\newcommand{\clockfield}{\chi(x)}
\newcommand{\swirlfactor}{S_\circ(x)}



\begin{document}

	\title{\papertitle}
	\author{Omar Iskandarani}
	\affiliation{Independent Researcher, Groningen, The Netherlands}
    \thanks{ORCID: 0009-0006-1686-3961, DOI: \paperdoi}
	\date{\today}

    \begin{abstract}

        \noindent We present Version 0.7.0 of the Swirl-String Theory (SST) Canon. This release unifies three historically distinct phenomena---time, gravity, and mass---into a single hydrodynamic framework based on a frictionless, incompressible superfluid condensate. We resolve the "Problem of Time" in quantum mechanics by defining time as a relational observable (event count) derived from a conserved topological current $J^\mu$. We demonstrate that the scalar field mediating this clock synchronization satisfies a Poisson equation, naturally yielding the inverse-square law for gravity without assuming curved spacetime. Finally, we derive the invariant masses of stable particles (protons, electrons) as the integrated swirl energy of topological knots, strictly enforcing the separation between the vacuum fluid density ($\rho_f \sim 10^{-7}$ kg/m$^3$) and the core condensate density ($\rho_{\text{core}} \sim 10^{18}$ kg/m$^3$).

    \end{abstract}

        \maketitle
        \newpage
        %\tableofcontents
        \compacttoc


% ======================================================================
    \section{Executive Summary: The Hydrodynamic Unity}
% ======================================================================

        Current mainstream physics treats time as a background parameter, gravity as spacetime geometry, and mass as a coupling to the Higgs field. SST v0.7.0 proposes that these are emergent manifestations of a single substrate:

        \begin{enumerate}
            \item \textbf{Time} is the local counting of vortex events relative to the background flow.
            \item \textbf{Gravity} is the gradient in the density of these events (the clock field).
            \item \textbf{Mass} is the energy trapped within the topological defects (knots) that generate these events.
        \end{enumerate}

        By rigorously defining the \textit{Event Current} $J^\mu$ and the \textit{Clock Field} $\chi(x)$, we resolve the Pauli objection to the time operator and derive the $1/r^2$ gravitational force as a hydrodynamic entropy force.

% ======================================================================
    \section{Canonical Axioms (v0.7.0 Refined)}
% ======================================================================

        The theory is built upon three non-negotiable axioms.

        \subsection*{Axiom I: Swirl-Time \& Foliation}
            Time is not a universal parameter $t$, but a local physical field governed by the tangential swirl velocity $\mathbf{v}$. The local tick-rate $dt(x)$ relative to infinity is:
            \begin{equation}
                dt(x) = S_\circ(x) \, dt_\infty = \sqrt{1 - \frac{|\mathbf{v}|^2}{c^2}} \, dt_\infty
            \end{equation}
            where $c$ is the transverse wave speed of the medium.

        \subsection*{Axiom II: Incompressible Superfluid Vacuum}
            The universe is filled with a perfect, inviscid fluid defined by the Euler equations.
            \begin{itemize}
                \item \textbf{Vacuum Density:} $\rhof \approx 7.0 \times 10^{-7}$ kg m$^{-3}$.
                \item \textbf{Core Density:} $\rhocore \approx 3.89 \times 10^{18}$ kg m$^{-3}$ (inside vortex filaments).
                \item \textbf{Swirl Velocity Scale:} $|\vswirl| \approx 1.09 \times 10^6$ m s$^{-1}$.
            \end{itemize}

        \subsection*{Axiom III: Topological Matter (Rosetta Rule)}
            Stable particles are closed, knotted vortex filaments. Conserved quantum numbers (charge, spin, flavor) correspond to topological invariants (linking number, twist, knot type).

% ======================================================================
% NEW: Hydrodynamic closure (added; does not replace anything)
% ======================================================================
    \section{Hydrodynamic Equations of Motion and Topological Stability}
        \label{sec:hydroEOM}

        The axioms above become predictive only once the dynamical laws of the medium are
        made explicit. SST assumes an inviscid, incompressible condensate described by the
        Euler equations
        \begin{equation}
            \partial_t \mathbf{v} + (\mathbf{v}\cdot\nabla)\mathbf{v}
            = -\frac{1}{\rho}\nabla p,
            \qquad
            \nabla\cdot\mathbf{v}=0,
        \end{equation}
        together with vorticity $\boldsymbol{\omega}=\nabla\times\mathbf{v}$ evolution
        \begin{equation}
            \partial_t \boldsymbol{\omega} = \nabla\times(\mathbf{v}\times\boldsymbol{\omega})
        \end{equation}
        for barotropic flow.

        A cornerstone is Kelvin's circulation theorem: for a material loop $\mathcal{C}(t)$
        advected by the flow,
        \begin{equation}
            \frac{\mathrm{d}}{\mathrm{d}t}\oint_{\mathcal{C}(t)} \mathbf{v}\cdot \mathrm{d}\boldsymbol{\ell} = 0.
        \end{equation}
        This conservation law underwrites the stability of vortex filaments and their knots:
        in an ideal medium, circulation cannot continuously unwind without reconnection or
        non-ideal effects. In SST, this is the dynamical origin of particle-like persistence.

        \paragraph{Canonical note.}
            Whenever reconnection or dissipation is present, SST treats it as a controlled
            effective correction (e.g., coarse-grained or boundary-layer physics), not as the
            fundamental law of the condensate.

% ======================================================================
    \section{Thermodynamic Origin of Quantization (Canonical Thermodynamics Patch)}
    \label{sec:thermo_quant}
% ======================================================================

This chapter incorporates the thermodynamic sector of SST into Canon v0.7.0.
Its purpose is to provide a closed route from hydrodynamic state variables
(pressure, swirl energy, event density) to quantization conditions and
hydrogenic length/energy scales, without postulating quantization as an axiom.

\subsection{Thermodynamic State Variables and Swirl Temperature}

SST treats the condensate as a frictionless medium whose coarse-grained
macrostates admit thermodynamic potentials. We introduce an effective
swirl-temperature $T_{\mathrm{sw}}$ (an emergent measure of coarse-grained
strain/activation of swirl modes) and a Helmholtz-like functional
\begin{equation}
    F(r) \;=\; E(r) \;-\; T_{\mathrm{sw}}\,S(r),
\end{equation}
where $r$ is a coarse-grained orbital scale, $E$ is an effective swirl-energy
functional, and $S$ is the entropy associated with accessible swirl microstates.

\subsection{Free-Energy Extremum and Stable Orbital Radius}

A stable bound configuration corresponds to an extremum condition
\begin{equation}
    \frac{\mathrm{d}F}{\mathrm{d}r} = 0
    \qquad\Longleftrightarrow\qquad
    \frac{\mathrm{d}E}{\mathrm{d}r} = T_{\mathrm{sw}}\frac{\mathrm{d}S}{\mathrm{d}r}.
    \label{eq:Fextremum}
\end{equation}
This is the canonical SST quantization condition: discrete radii arise when
only discrete topological/adiabatic branches contribute to $S(r)$ under
Euler--Kelvin constraints.

\subsection{Core-to-Orbital Scale Bridge}

Canon v0.7.0 retains a two-scale bridge between the filament core scale $r_c$
and the hydrogenic orbital scale $a_0$ via the characteristic swirl speed
$|\mathbf{v}_{\circlearrowleft}|$. A canonical scaling relation is encoded as
\begin{equation}
    a_0
    \;=\;
    \frac{c^2}{2\,|\mathbf{v}_{\circlearrowleft}|^2}\,r_c,
    \label{eq:a0_from_rc}
\end{equation}
where $c$ is the transverse propagation speed of the medium (the relativistic
signal speed of small perturbations). Equation \eqref{eq:a0_from_rc} is used as
a structural constraint: the orbital scale is not arbitrary, but determined by
the ratio of propagation and swirl velocities times the core radius.

\subsection{Swirl-Clock Factor and Radial Dependence}

The Swirl-Clock field relates local tick-rate to local swirl kinematics:
\begin{equation}
    \swirlfactor(r)
    \;=\;
    \sqrt{1 - \frac{|\mathbf{v}(r)|^2}{c^2}}.
\end{equation}
In the thermodynamic orbital picture, excited configurations correspond to
larger characteristic radii and modified swirl profiles; this yields a
systematic trend in clock dilation and decay rates via available phase space.

\subsection{Thermodynamic Layering and Discrete Scale Invariance}

Canon v0.7.0 encodes a discrete-scale hierarchy (``golden layers'') that
organizes stable defect and orbital configurations. We write a minimal
canonical layering as
\begin{equation}
    \mathcal{E}_n \;\propto\; \phi^{-2n},
    \qquad
    \phi=\frac{1+\sqrt{5}}{2},
\end{equation}
and interpret $n\in\mathbb{Z}$ as a discrete scale index labeling
thermodynamic branches of the swirl condensate.

This structure is compatible with the selection principle of
Sec.~\ref{sec:knotSelection}: $\mathcal{E}_{\mathrm{eff}}[K]$ admits families
of minima related by discrete rescaling when core, curvature, and interaction
terms balance self-similarly.

\subsection{Operational Role in Canon v0.7.0}

The thermodynamic patch supplies:
\begin{itemize}
    \item a non-axiomatic mechanism for quantization (free-energy extremum),
    \item a core-to-orbital bridge \eqref{eq:a0_from_rc},
    \item a universal layering structure consistent with the golden hierarchy,
    \item a consistent interface to the TOA/clock sector (via $\swirlfactor$).
\end{itemize}

\subsection{Optional: verbatim include of the Thermodynamics paper}

If you have the LaTeX source of the thermodynamics paper, you can include it
verbatim instead of (or in addition to) the canonical summary above:
\begin{center}
\texttt{\textbackslash import\{./\}\{SST-Thermodynamics.tex\}}
\end{center}

% ======================================================================
    \section{Hydrogenic Orbitals as Swirl Equilibria (Canonical Orbitals Patch)}
    \label{sec:hydrogenic_orbitals}
% ======================================================================

This chapter integrates the hydrogenic orbitals program into the Canon. The
target is a reproducible chain:
\[
\text{swirl thermodynamics} \;\Rightarrow\; \text{discrete radii/energies}
\;\Rightarrow\; \text{standard semiclassical limits}.
\]

\subsection{Abe--Okuyama Quantum--Thermodynamic Isomorphism (Interface Tool)}

Canon v0.7.0 adopts a quantum--thermodynamic mapping as a \emph{methodological}
interface: quantum amplitudes are treated as effective encodings of a
thermodynamic ensemble of swirl microstates. In SST this is not a claim that
quantum mechanics is ``just'' thermodynamics, but a controlled correspondence
used to translate between:
\begin{itemize}
    \item hydrodynamic macrovariables (pressure, swirl temperature, entropy),
    \item effective quantum objects (phase, action, orbital spectra),
    \item operational predictions (transition lines, decay trends).
\end{itemize}

\subsection{Orbital Quantization from a Thermodynamic Extremum}

Using the free-energy extremum \eqref{eq:Fextremum}, discrete radii arise from
admissible entropy branches $S_n(r)$ consistent with Euler--Kelvin constraints:
\begin{equation}
    \frac{\mathrm{d}}{\mathrm{d}r}\Big(E(r)-T_{\mathrm{sw}}S_n(r)\Big)=0
    \quad\Rightarrow\quad r=r_n.
\end{equation}
The sequence $\{r_n\}$ defines the orbital ladder.

\subsection{Semiclassical Recovery (Bohr-like Limit)}

In the narrowband/weak-fluctuation limit, the orbital ladder reproduces the
standard large-$n$ scaling and classical time-of-flight trends. Canon v0.7.0
treats this as a required consistency check: SST quantization must recover
semiclassical limits when the thermodynamic coarse-graining scale is large
compared to $r_c$ and clock fluctuations are weak.

\subsection{Clock Coupling to Orbitals}

The clock field $\chi$ couples to orbitals through the Swirl-Clock factor
$\swirlfactor(r)$. This implies that orbital structure, decay rates, and
time-of-arrival statistics are not separate modules but a single coupled sector:
\begin{equation}
    \chi(r) \propto \ln \swirlfactor(r),
    \qquad
    \nabla^2\chi \propto \rho_{\mathrm{matter}}.
\end{equation}

\subsection{Optional: verbatim include of the Hydrogenic Orbitals paper}

If you have the LaTeX source, include it verbatim here:
\begin{center}
\texttt{\textbackslash import\{./\}\{SST-Hydrogenic\_Orbitals.tex\}}
\end{center}

% ======================================================================
    \section{The Unified Clock--Gravity Field}
% ======================================================================

    In previous versions, gravity and time were treated separately. In v0.7.0, they are unified via the scalar mediator field.

    \subsection{The Mediator is the Clock}
        We define the scalar field $\clockfield$ as the logarithmic gradients of the swirl dilation factor:
        \begin{equation}
            \chi(x) \propto \ln S_\circ(x)
        \end{equation}
        This field represents the local "density of time."

    \subsection{Emergence of the Inverse-Square Law}
        Matter (vortex knots) acts as a sink in the fluid pressure, sourcing the scalar field. In the static weak-field limit, the field satisfies the Poisson equation:
        \begin{equation}
            \nabla^2 \chi(x) = 4\pi G_{\text{eff}} \, \rho_{\text{matter}}(x)
        \end{equation}
        The Green's function solution in $\mathbb{R}^3$ is naturally the harmonic potential $\chi(r) \sim 1/r$. Thus, gravity is identified as the tendency of matter to migrate toward regions of slower time (lower swirl pressure), recovering the Newtonian limit without curvature.

% ======================================================================
    \section{Swirl-Clock Effective Field Theory (Khronon Sector)}
    \label{sec:clockEFT}
% ======================================================================

The scalar clock field $\chi(x)$ introduced above is not merely kinematic.
In Canon v0.7.0 it is promoted to a dynamical degree of freedom governed by
a low-energy effective field theory closely related to Einstein--\AE ther
and khronometric gravity, but reinterpreted hydrodynamically.

\subsection{Unit Timelike Foliation Vector}

Define the normalized foliation 4-vector
\begin{equation}
    u_\mu
    \;\equiv\;
    \frac{\nabla_\mu \chi}
    {\sqrt{g^{\alpha\beta}\nabla_\alpha\chi\,\nabla_\beta\chi}},
\end{equation}
which is hypersurface-orthogonal by construction. In SST, $u_\mu$ represents
the local rest frame of the condensate clock flow.

\subsection{Clock-Sector Action}

The most general diffeomorphism-invariant, second-order EFT for $u_\mu$
compatible with hypersurface orthogonality is
\begin{equation}
\begin{aligned}
S_\chi
=
\int d^4x \sqrt{-g}
\Big[
& c_1 (\nabla_\mu u_\nu)(\nabla^\mu u^\nu)
+ c_2 (\nabla_\mu u^\mu)^2 \\
& + c_3 (\nabla_\mu u_\nu)(\nabla^\nu u^\mu)
+ c_4 u^\mu u^\nu (\nabla_\mu u_\alpha)(\nabla_\nu u^\alpha)
\Big].
\end{aligned}
\end{equation}

In the strict khronon limit relevant for SST, $u_\mu$ derives entirely from
$\chi$ and no independent vector modes propagate.

\subsection{Gravitational-Wave Constraint}

Observations of GW170817 impose the constraint
\begin{equation}
    c_{13} \equiv c_1 + c_3 = 0,
\end{equation}
ensuring luminal propagation of tensor modes. Canon v0.7.0 adopts this
constraint as a consistency condition, leaving the clock sector compatible
with gravitational-wave observations.

\subsection{Interpretation in SST}

In Swirl-String Theory, this EFT does not describe a fundamental spacetime
preferred frame. Instead, it is the long-wavelength description of
collective excitations of the condensate clock flow.
The scalar mediator responsible for gravity and the relational time field
are therefore \emph{the same physical entity}.

% ======================================================================
    \section{Event Currents \& Covariant Measurement}
% ======================================================================

    To make "Time" a quantum observable, we introduce the Event Current.

    \subsection{The Conserved Current}
        We define a conserved 4-vector current $J^\mu$ representing the flow of physical events:
        \begin{equation}
            \partial_\mu J^\mu = 0
        \end{equation}
        A "measurement" is the integration of this flux over a detector's world-tube $\mathcal{W}$.

% ======================================================================
    \section{Relational Time-of-Arrival (TOA)}
% ======================================================================

    \subsection{Resolution of the Pauli Objection}
        Standard QM forbids a self-adjoint time operator $\hat{T}$ conjugate to a bounded Hamiltonian. We bypass this by defining Time of Arrival (TOA) relationally. The observable is not $t$, but the expectation value of the event count $N$ relative to a reference clock field $\chi$.

    \subsection{Microphysical Realization in SST}
        In SST, the abstract "events" are physically identified as the topological defects (vortex knots). The current $J^\mu$ is the topological current density:
        \begin{equation}
            J^\mu(x) = \sum_{k} \mathfrak{q}_k \int d\tau \, \dot{z}_k^\mu(\tau) \, \delta^{(4)}(x - z_k(\tau))
        \end{equation}
        where $z_k(\tau)$ describes the worldline of a knot center (e.g., a proton or electron) and $\mathfrak{q}_k$ is the topological charge. Time measurement is physically the counting of knot crossings through the vacuum foliation.

% ======================================================================
% MASS CHAPTER: EXPANDED (replaces the short version; not shrinking overall)
% ======================================================================
    \section{Atomic and Nuclear Masses from Swirl Energy (Expanded)}
    \label{sec:mass_expanded}

    This section completes the mass sector of Swirl-String Theory by providing a
    first-principles route from localized swirl kinetics in the core condensate to
    invariant particle masses. The goal is not a single heuristic formula but a
    hierarchy of approximations: (i) an exact definition, (ii) a slender-filament
    reduction, and (iii) a topological mass functional suitable for numerical
    evaluation.

    \subsection{Separation of Densities: Vacuum vs.\ Core}

        Canon v0.7.0 strictly separates the background vacuum density from the vortex-core
        condensate density:

        \begin{itemize}
            \item \textbf{Vacuum density (background medium):}
            \[
                \rhof \approx 7.0\times 10^{-7}\ \mathrm{kg\,m^{-3}}
            \]
            controlling wave propagation and large-scale hydrodynamics.
            \item \textbf{Core density (filament interior):}
            \[
                \rhocore \approx 3.893435827\times 10^{18}\ \mathrm{kg\,m^{-3}}
            \]
            controlling localized energy storage and inertial mass.
        \end{itemize}

        Using $\rhof$ in the core mass integral yields masses suppressed by roughly
        $25$ orders of magnitude; Canon v0.7.0 corrects this by construction.

    \subsection{Invariant Mass as Core-Localized Kinetic Swirl Energy}

        Let $K$ denote a closed, knotted vortex filament (a stable particle). SST defines
        the rest energy of $K$ as the kinetic swirl energy localized in its core volume
        $V_K$:
        \begin{equation}
            M(K)c^2
            \;\equiv\;
            \int_{V_K}
            \frac{1}{2}\,\rhocore
            \left|\mathbf{v}(\mathbf{r})\right|^2
            \,\mathrm{d}^3\mathbf{r}.
            \label{eq:mass_def_exact}
        \end{equation}
        This is the canonical mass definition: once $\mathbf{v}(\mathbf{r})$ is fixed by
        knot topology and core structure, $M(K)$ is fixed.

    \subsection{Velocity Field Determined by Topology}

        For a filament centerline $\mathbf{X}(s)$, the induced velocity field admits the
        Biot--Savart representation (in the standard slender-filament regime):
        \begin{equation}
            \mathbf{v}(\mathbf{r})
            =
            \frac{\Gamma}{4\pi}\oint_K
            \frac{\mathrm{d}\boldsymbol{\ell}\times(\mathbf{r}-\mathbf{r}')}{|\mathbf{r}-\mathbf{r}'|^3},
        \end{equation}
        where $\Gamma=\oint\mathbf{v}\cdot d\boldsymbol{\ell}$ is circulation. In SST,
        $\Gamma$ is treated as a topologically protected quantity (Kelvin theorem plus
        core regularization). The geometry of $\mathbf{v}$ is thus constrained by knot
        type and embedding.

    \subsection{Slender-Filament Reduction to a One-Dimensional Functional}

        For $r_c$ much smaller than local curvature radii, \eqref{eq:mass_def_exact}
        reduces to a line functional along the filament:
        \begin{equation}
            M(K)
            \;\approx\;
            \frac{1}{2c^2}\,
            \rhocore\,\Gamma^2\,\mathcal{L}(K)\,\Xi(K),
            \label{eq:mass_slender}
        \end{equation}
        where:
        \begin{itemize}
            \item $\mathcal{L}(K)\equiv L_K/\rc$ is the (dimensionless) ropelength of the knot,
            \item $\Xi(K)$ is a dimensionless geometry factor capturing near-field structure,
            writhe/twist distribution, and core profile effects.
        \end{itemize}
        In the minimal canonical approximation, $\Xi(K)\sim \mathcal{O}(1)$ and the main
        topological discriminator is $\mathcal{L}(K)$.

    \subsection{Discrete Mass Spectrum from Discrete Knot Complexity}

        The key physical point is that stable knots occupy discrete complexity classes.
        Consequently, the ropelength functional $\mathcal{L}(K)$ is not continuously tunable
        without changing topology, so masses become discrete:
        \begin{equation}
            K_1\neq K_2 \;\;\Rightarrow\;\; M(K_1)\neq M(K_2),
        \end{equation}
        modulo degeneracies that SST treats as symmetry/duality classes (chirality,
        orientation, linking).

    \subsection{Electron--Proton Hierarchy (Structural Explanation)}

        If the electron corresponds to the simplest stable chiral knot class and the proton
        to a composite/linked configuration, then
        \begin{equation}
            \frac{M_p}{M_e}\;\sim\;\frac{\mathcal{L}(K_p)\,\Xi(K_p)}{\mathcal{L}(K_e)\,\Xi(K_e)}.
        \end{equation}
        This produces a natural hierarchy (composites heavier than simples) without Higgs
        Yukawas. Canon v0.7.0 treats a full quantitative match as a numerical program:
        compute $\mathcal{L}(K)$ and $\Xi(K)$ from realistic filament embeddings.

    \subsection{Atomic Masses as Bound Multi-Knot Configurations}

        An atom is a stable multi-defect configuration:
        \[
            \{\text{nuclear composite knots}\} \;+\; \{\text{leptonic knots}\}
        \]
        bound by the unified clock--gravity field and transverse excitations (EM sector).
        Atomic masses follow from:
        \begin{equation}
            M_{\text{atom}}c^2
            =
            \sum_i M(K_i)c^2
            \;+\;E_{\text{bind}}(\text{configuration}),
        \end{equation}
        with $E_{\text{bind}}$ arising from interaction energy in the surrounding medium
        (overlap of swirl/pressure fields plus transverse-mode energy).

    \subsection{Why Vacuum Energy is Not Inertial Mass}

        The homogeneous background energy associated with $\rhof$ does not contribute to
        inertial mass because it is not localized, not tied to circulation, and does not
        carry the topological trapping that defines particles. In SST, only
        \emph{topologically trapped} core swirl energy contributes to $M$.

    \subsection{Summary of the Mass Sector}

        Mass in SST is:
        \begin{itemize}
            \item not fundamental,
            \item not a coupling constant,
            \item not generated by symmetry breaking,
            \item but the \emph{core-localized swirl kinetic energy} of topological defects.
        \end{itemize}

        This closes the mass sector at the canonical level and supplies the quantitative
        bridge needed for atomic/nuclear mass modeling.

% ======================================================================
    \section{Knot Selection and Stability Functional}
    \label{sec:knotSelection}
% ======================================================================

Not all topological knots correspond to stable particles.
Canon v0.7.0 introduces a variational selection principle that determines
which knot classes are dynamically realized.

\subsection{Effective Energy Functional}

For a candidate knot configuration $K$, define the effective energy
\begin{equation}
    \mathcal{E}_{\mathrm{eff}}[K]
    =
    \alpha\,C(K)
    + \beta\,L(K)
    + \gamma\,\mathcal{H}(K),
\end{equation}
where:
\begin{itemize}
    \item $C(K)$ is a crossing or self-contact measure (Biot--Savart energy),
    \item $L(K)$ is filament length (line tension contribution),
    \item $\mathcal{H}(K)$ is the helicity or linking invariant,
    \item $\alpha,\beta,\gamma$ are medium-dependent coefficients fixed by
    condensate parameters.
\end{itemize}

\subsection{Stability Criterion}

Physically realized particles correspond to local minima of
$\mathcal{E}_{\mathrm{eff}}[K]$ under smooth deformations that preserve topology.
Unstable knot classes either decay (via reconnection) or fail to localize
energy sufficiently to form particles.

\subsection{Relation to the Mass Functional}

The mass functional derived in Sec.~\ref{sec:mass_expanded} is recovered as the
dominant contribution to $\mathcal{E}_{\mathrm{eff}}$ once the knot geometry
is fixed at a stable minimum. Thus, mass is both an energetic and topological
quantity.

% ======================================================================
    \section{Discrete Scale Invariance and Golden-Layer Quantization}
    \label{sec:golden}
% ======================================================================

Beyond topological discreteness, Swirl-String Theory exhibits discrete
scale invariance in the organization of stable knot configurations.

\subsection{Golden-Layer Structure}

Empirically and numerically, stable filament configurations cluster into
energy layers separated by approximately constant scale ratios.
Canon v0.7.0 encodes this via a golden-layer factor:
\begin{equation}
    M_n \;\propto\; \phi^{-2n}, \qquad \phi=\frac{1+\sqrt{5}}{2},
\end{equation}
where $n\in\mathbb{Z}$ labels discrete scale layers.

\subsection{Origin}

This structure arises from:
\begin{itemize}
    \item scale competition between core radius $r_c$ and curvature radii,
    \item self-similar minimization of $\mathcal{E}_{\mathrm{eff}}[K]$,
    \item discrete topological branching under refinement.
\end{itemize}

\subsection{Physical Consequences}

Discrete scale invariance sharpens mass ratios, suppresses continuous
parameter drift, and stabilizes the particle spectrum across energy scales.
It provides an organizing principle linking atomic, nuclear, and potentially
subnuclear structure within a single condensate hierarchy.

% ======================================================================
% NEW: Quantum numbers (topology mapping) — added; no removal
% ======================================================================
    \section{Quantum Numbers as Topological Invariants}
    \label{sec:qn_topology}

    Axiom III states that conserved quantum numbers correspond to topological
    invariants. Canon v0.7.0 makes this explicit at the level needed for a complete
    unification narrative.

    \begin{itemize}
        \item \textbf{Charge}: corresponds to signed circulation $\Gamma$ (orientation
        of the filament and the sign convention for the induced flow).
        \item \textbf{Spin}: corresponds to intrinsic twist/writhe structure and the
        handedness (chirality) class of the knot embedding.
        \item \textbf{Particle--antiparticle}: corresponds to orientation reversal
        (knot time orientation) in the medium, consistent with the Rosetta rule.
    \end{itemize}

    These identifications are not merely interpretational: the invariants are conserved
    under the Euler--Kelvin dynamics, hence the associated quantum numbers are robust.

% ======================================================================
    \section{Neutrinos, Chirality, and Clock-Frame Couplings (Canonical Neutrinos Patch)}
    \label{sec:neutrinos}
% ======================================================================

This chapter integrates the neutrino/chirality sector into Canon v0.7.0. The
goal is to encode (i) a physically realized foliation frame, and (ii) a minimal,
testable coupling that selects chirality in the presence of the clock field.

\subsection{Unit Timelike Clock Frame}

From the clock/foliation scalar $\chi$ we define a unit timelike vector field
\begin{equation}
    u_\mu \equiv \frac{\nabla_\mu \chi}{\sqrt{g^{\alpha\beta}\nabla_\alpha\chi\,\nabla_\beta\chi}},
    \qquad
    u_\mu u^\mu = -1,
    \label{eq:unit_u}
\end{equation}
representing the local rest direction of the condensate clock flow.

This $u_\mu$ is the object that must appear in any fermionic sector that is
sensitive to foliation/clock structure.

\subsection{Minimal Axial Coupling for Neutrinos}

Canon v0.7.0 introduces the leading chirality-sensitive interaction between
a neutrino field $\nu$ and the clock frame $u_\mu$ as an axial coupling:
\begin{equation}
    \mathcal{L}_{\nu u}
    \;=\;
    \lambda_\nu\,u_\mu\,\bar{\nu}\gamma^\mu\gamma^5\nu,
    \label{eq:axial_coupling}
\end{equation}
where $\lambda_\nu$ is a coupling constant (to be bounded experimentally).

This term is:
\begin{itemize}
    \item Lorentz-covariant at the level of fields,
    \item foliation-sensitive through $u_\mu$,
    \item chirality-selective through $\gamma^5$,
    \item operationally testable via direction/time dependence in neutrino propagation.
\end{itemize}

\subsection{Consistency with the Clock EFT}

The $u_\mu$ used here is the same $u_\mu$ whose EFT is given in
Sec.~\ref{sec:clockEFT} (khronon sector). Therefore neutrino chirality,
clock propagation, and gravity constraints are unified into a single sector.

\subsection{Predictions and Falsifiers (Canon-Form)}

Canon v0.7.0 treats the neutrino sector as predictive only if it yields
falsifiable statements. Minimal examples include:
\begin{itemize}
    \item direction-dependent phase shifts correlated with $\nabla_\mu\chi$,
    \item controlled deviations from standard dispersion at extremely weak levels,
    \item coupling bounds from existing neutrino timing and oscillation datasets.
\end{itemize}

\subsection{Optional: verbatim include of the Neutrinos paper}

If you have the LaTeX source, include it verbatim here:
\begin{center}
\texttt{\textbackslash import\{./\}\{SST-Neutrinos.tex\}}
\end{center}

% ======================================================================
% NEW: Electromagnetism sector — added; no removal
% ======================================================================
    \section{Electromagnetism as Transverse Swirl Excitation}
    \label{sec:em_swirl}

    In addition to longitudinal pressure/clock modes (gravity sector), the condensate
    supports transverse excitations. Canon v0.7.0 treats these as the hydrodynamic
    origin of electromagnetic phenomena.

    A minimal kinematic correspondence is:
    \begin{equation}
        \mathbf{A} \propto \mathbf{v}_\perp,
        \qquad
        \mathbf{B}=\nabla\times\mathbf{A}\;\leftrightarrow\;\boldsymbol{\omega}_\perp,
    \end{equation}
    where $\mathbf{v}_\perp$ is the transverse component of swirl flow and
    $\boldsymbol{\omega}_\perp$ its associated vorticity.

    In the weak-excitation regime, Maxwell-like equations arise as effective field
    equations for transverse modes, while coupling to defects is mediated by
    circulation/twist degrees of freedom.

% ======================================================================
% NEW: Recovery of standard QM/QFT limits — added; no removal
% ======================================================================
    \section{Recovery of Standard Quantum Dynamics (Effective Limits)}
    \label{sec:qm_limits}

    SST is constructed to reproduce standard quantum dynamics as an effective limit.

    \subsection{Klein--Gordon and Schr\"odinger limits}

        For small perturbations around a stationary background, linearization yields
        relativistic scalar wave equations (Klein--Gordon type) for appropriate mode
        variables. In the slow-envelope / nonrelativistic limit, the Schr\"odinger
        equation emerges for coarse-grained amplitudes.

    \subsection{Interpretational closure}

        In SST, the wavefunction is not a fundamental object but an effective
        description of coherent mode structure in the condensate. Measurement
        statistics arise from event-counting of defect fluxes relative to the
        clock field.

% ======================================================================
    \section{Discussion: The Leap to v0.7.0}
% ======================================================================

    This version marks the transition from an interpretative model to a predictive physical theory. By patching the gravitational derivation (EFT Mediator) and anchoring the definition of time (Event Currents), SST now offers a closed-loop formalism where:
    \begin{itemize}
        \item \textbf{Gravity} is no longer an assumption but a derived Poisson effect.
        \item \textbf{Time} is strictly relational and compatible with Quantum Information principles.
        \item \textbf{Mass} is a calculable property of topology, not a free parameter.
    \end{itemize}

% ======================================================================
% NEW: Conclusion (tight; placed after Discussion as best practice)
% ======================================================================
    \section{Conclusion}
    \label{sec:conclusion}

    Canon v0.7.0 unifies time, gravity, mass, and measurement within a single
    hydrodynamic ontology. Time is defined operationally as a relational event count
    relative to a physical clock field; gravity is the Poisson-mediated structure of
    that same clock sector; and invariant masses are core-localized swirl kinetic
    energies trapped by topological defects. The framework is dynamically anchored
    in Euler--Kelvin invariants, admits effective limits reproducing standard quantum
    wave dynamics, and upgrades SST from interpretational narrative to a
    parameter-constrained, structurally predictive program.

    \appendix
    \section{Reference Constants}
    \begin{itemize}
        \item $\mathbf{v}_{\circlearrowleft} = 1.09384563 \times 10^6$ m s$^{-1}$
        \item $r_c = 1.40897017 \times 10^{-15}$ m
        \item $\rhocore = 3.893435827 \times 10^{18}$ kg m$^{-3}$
        \item $F_{\text{swirl}}^{\max} = 29.053507$ N
    \end{itemize}

% ======================================================================
    \section*{Appendix C: Relational Time-of-Arrival from Event Currents}
% ======================================================================

\import{../Foundations_of_Physics/sn-article-template/Relational_Time-of-Arrival_from_an_Event_Current_and_its_Continuum_Clock_Limit.tex}



%================================================
% References
%================================================

\bibliographystyle{unsrt}
\begin{thebibliography}{99}

    \bibliography{canon_swirl_string_theory}
    \bibitem{ThurstonNotes}
    W.~P.~Thurston,
    \newblock \emph{The Geometry and Topology of Three-Manifolds},
    \newblock Princeton Univ. Lecture Notes, 1979.

    \bibitem{NeumannZagier1985}
    W.~D.~Neumann and D.~Zagier,
    \newblock Volumes of hyperbolic three-manifolds,
    \newblock \emph{Topology} \textbf{24}(3):307--332, 1985. \url{https://doi.org/10.1016/0040-9383(85)90003-4}

    \bibitem{AdamsWeeks1992}
    C.~Adams, M.~Hildebrand, and J.~Weeks,
    \newblock Hyperbolic invariants of knots and links,
    \newblock \emph{Trans. Amer. Math. Soc.} \textbf{326}(1):1--56, 1992.

    \bibitem{Lewin1981}
    L.~Lewin,
    \newblock \emph{Polylogarithms and Associated Functions},
    \newblock North-Holland, 1981.

    \bibitem{KAtlas52}
    D.~Bar-Natan et al.,
    \newblock The Knot Atlas: entry \(5_2\),
    \newblock \url{https://katlas.org/wiki/5_2}.

    \bibitem{KAtlas61}
    D.~Bar-Natan et al.,
    \newblock The Knot Atlas: entry \(6_1\),
    \newblock \url{https://katlas.org/wiki/6_1}.

    \bibitem{Annala2025} T. Annala \emph{et al.}, ``Topologically protected vortex knots and links,'' \emph{Phys. Rev. Lett.}, 2025.
    \bibitem{Kleckner2016} D. Kleckner, L. Kauffman, W. Irvine, ``How superfluid vortex knots untie,'' \emph{Nat. Phys.} 12, 650–655 (2016).
    \bibitem{Ricca1996} R. Ricca, ``Applications of knot theory in fluid mechanics,'' \emph{Banach Center Publications}, Vol. 42 (1996).
    \bibitem{Purcell2025} D. Ibarra, D. Mathews, J. Purcell, ``On geometric triangulations of double twist knots,'' arXiv:2504.09901 (2025).
    \bibitem{Petersen2024} I. Petersen, A. Tsvietkova, ``Geometric structures and PSL$_2(\mathbb{C})$ representations of knot groups,'' \emph{Trans. AMS} (2024).


    \bibitem{SSTCanon05}
    Iskandarani, O.\ (2025).
    \emph{Swirl–String Theory Canon v0.5.8}.
    Internal manuscript (Canon).

    \bibitem{Rosetta05}
    Iskandarani, O.\ (2025).
    \emph{VAM–SST Rosetta v0.5}.
    Internal manuscript (Rosetta).

    \bibitem{IskandaraniTriad2025}
    O.~Iskandarani,
    ``The Hydrodynamic Triad: Unifying Gravity, Electromagnetism, and Quantum Mass
    via a Circulation-Based Vacuum Canon,''
    Zenodo (2025), DOI: 10.5281/zenodo.17728292.

    \bibitem{LandauLifshitzFM1987}
    Landau, L. D., \& Lifshitz, E. M.\ (1987).
    \emph{Fluid Mechanics} (2nd ed.). Pergamon.
    (Foundations of inviscid linearization and Bernoulli used in \eqref{eq:B5}.)

    \bibitem{MorseIngard1968}
    Morse, P. M., \& Ingard, K. U.\ (1968).
    \emph{Theoretical Acoustics}. Princeton University Press.
    (Standard monopole source \eqref{eq:B2} and far-field law \eqref{eq:B3}–\eqref{eq:B4}.)

    \bibitem{Pierce1989}
    Pierce, A. D.\ (1989/1991).
    \emph{Acoustics: An Introduction to Its Physical Principles and Applications} (2nd ed.). ASA.
    (Alternative derivations for \eqref{eq:B3}–\eqref{eq:B4}.)

    \bibitem{Westervelt1963}
    Westervelt, P. J.\ (1963).
    Parametric acoustic array.
    \emph{J. Acoust. Soc. Am.}, 35(4), 535–537.
    (Constitutive parametric pumping basis compatible with BASC inside $T$.)

    \bibitem{HamiltonBlackstock1998}
    Hamilton, M. F., \& Blackstock, D. T.\ (1998).
    \emph{Nonlinear Acoustics}. Academic Press.
    (Background on quadratic transduction and difference-frequency generation.)


    \bibitem{Einstein1905}
    A.~Einstein,
    \newblock Zur Elektrodynamik bewegter K{\"o}rper,
    \newblock {\em Annalen der Physik} \textbf{322}(10) (1905) 891--921.
    \newblock doi:10.1002/andp.19053221004.

    \bibitem{Minkowski1909}
    H.~Minkowski,
    \newblock Raum und Zeit,
    \newblock {\em Jahresbericht der Deutschen Mathematiker-Vereinigung} \textbf{18} (1909) 75--88.

    \bibitem{LevyLeblond1976}
    J.-M.~L{\'e}vy-Leblond,
    \newblock One more derivation of the Lorentz transformation,
    \newblock {\em American Journal of Physics} \textbf{44}(3) (1976) 271--277.
    \newblock doi:10.1119/1.10324.



    \bibitem{Batchelor1967}
    G.~K.~Batchelor, \emph{An Introduction to Fluid Dynamics} (Cambridge Univ. Press, 1967).
    \bibitem{Saffman1992}
    P.~G.~Saffman, \emph{Vortex Dynamics} (Cambridge Univ. Press, 1992).
    \bibitem{Onsager1949}
    L.~Onsager, ``Statistical Hydrodynamics,'' \emph{Nuovo Cimento} \textbf{6} (Suppl. 2), 279--287 (1949).
    \bibitem{Feynman1955}
    R.~P.~Feynman, ``Application of quantum mechanics to liquid helium,'' in \emph{Progress in Low Temperature Physics}, Vol.~1 (1955), pp.~17--53.

    \bibitem{Unruh1976}
    W.~G.~Unruh,
    ``Notes on black-hole evaporation,''
    \textit{Phys. Rev. D} \textbf{14}, 870--892 (1976).
    doi:10.1103/PhysRevD.14.870

    \bibitem{Crispino2008}
    L.~C.~B.~Crispino, A.~Higuchi, and G.~E.~A.~Matsas,
    ``The Unruh effect and its applications,''
    \textit{Rev. Mod. Phys.} \textbf{80}, 787--838 (2008).
    doi:10.1103/RevModPhys.80.787

    \bibitem{Barcelo2011}
    C.~Barcel\'o, S.~Liberati, and M.~Visser,
    ``Analogue gravity,''
    \textit{Living Rev. Relativ.} \textbf{14}, 3 (2011).
    doi:10.12942/lrr-2011-3

    \bibitem{KinslerAcoustics}
    L.~E.~Kinsler, A.~R.~Frey, A.~B.~Coppens, and J.~V.~Sanders,
    \textit{Fundamentals of Acoustics}, 4th ed.,
    Wiley, New York (2000).

    \bibitem{Deswal2025}
    A.~Deswal, N.~Arya, K.~Lochan, and S.~K.~Goyal,
    ``Time-Resolved and Superradiantly Amplified Unruh Effect,''
    \textit{Phys. Rev. Lett.} (2025), arXiv:2501.16219.

    \bibitem{GrossHaroche1982}
    M.~Gross and S.~Haroche,
    ``Superradiance: An essay on the theory of collective spontaneous emission,''
    \textit{Phys. Rep.} \textbf{93}, 301--396 (1982).
    doi:10.1016/0370-1573(82)90102-8

    \bibitem{Lochan2020}
    K.~Lochan, S.~Chakraborty, and T.~Padmanabhan,
    ``Detecting Acceleration-Enhanced Vacuum Fluctuations,''
    \textit{Phys. Rev. Lett.} \textbf{125}, 241301 (2020).
    doi:10.1103/PhysRevLett.125.241301

    \bibitem{WangBlencowe2021}
    H.~Wang and M.~P.~Blencowe,
    ``Coherently Amplifying Photon Production from Vacuum,''
    \textit{Commun. Phys.} \textbf{4}, 62 (2021).
    doi:10.1038/s42005-021-00576-9



    \bibitem{Zheng2025}
    H.~T.~Zheng, Y.~Zhou, Q.~Guo, and L.~Zhou,
    ``Enhancing Analog Unruh Effect via Superradiance,''
    \textit{Phys. Rev. Research} \textbf{7}, 013027 (2025).
    doi:10.1103/PhysRevResearch.7.013027

    \bibitem{Saha2025}
    S.~Saha, T.~Galley, and E.~Mart\'in-Mart\'inez,
    ``Emergence of Unruh Prethermalization in Many-Body Systems,''
    (2025), arXiv:2509.05816.

    \bibitem{Steinhauer2016}
    J.~Steinhauer,
    ``Observation of quantum Hawking radiation and its entanglement in an analogue black hole,''
    \textit{Nat. Phys.} \textbf{12}, 959--965 (2016).
    doi:10.1038/nphys3863

    \bibitem{Gooding2020}
    C.~Gooding, S.~Weinfurtner, and W.~G.~Unruh,
    ``Superradiant scattering from a hydrodynamic vortex,''
    \textit{Phys. Rev. D} \textbf{101}, 024050 (2020).
    doi:10.1103/PhysRevD.101.024050

    \bibitem{doCarmo-diff-geom-2016}
    M.~P.~do Carmo,
    \textit{Differential Geometry of Curves and Surfaces},
    revised and updated second edition,
    Dover Publications, Mineola, NY (2016).
% permalink: https://store.doverpublications.com/0486806995.html

    \bibitem{Ratcliffe-hyperbolic-2006}
    J.~G.~Ratcliffe,
    \textit{Foundations of Hyperbolic Manifolds}, 2nd ed.,
    Graduate Texts in Mathematics, Vol.~149,
    Springer, New York (2006).
    doi:10.1007/978-0-387-47322-5

    \bibitem{Thurston-3manifolds-1997}
    W.~P.~Thurston,
    \textit{Three-Dimensional Geometry and Topology, Vol.~1},
    Princeton Mathematical Series 35,
    Princeton University Press, Princeton, NJ (1997).
% permalink: https://press.princeton.edu/books/hardcover/9780691084219/three-dimensional-geometry-and-topology-volume-1

    \bibitem{Sornette1998}
    D.~Sornette,
    ``Discrete scale invariance and complex dimensions,''
    \emph{Physics Reports} \textbf{297}, 239--270 (1998).
    doi:10.1016/S0370-1573(97)00076-8.

    \bibitem{GluzmanSornette2002}
    S.~Gluzman and D.~Sornette,
    ``Log-periodic route to fractal functions,''
    \emph{Physical Review E} \textbf{65}, 036142 (2002).
    doi:10.1103/PhysRevE.65.036142.

    \bibitem{BaakeGrimm2013}
    M.~Baake and U.~Grimm,
    \emph{Aperiodic Order. Volume~1: A Mathematical Invitation},
    Encyclopedia of Mathematics and its Applications, Vol.~149
    (Cambridge University Press, Cambridge, 2013).
    doi:10.1017/CBO9781139025256.

    \bibitem{Moffatt1969}
    H.~K.~Moffatt,
    ``The degree of knottedness of tangled vortex lines,''
    \emph{Journal of Fluid Mechanics} \textbf{35}(1), 117--129 (1969).
    doi:10.1017/S0022112069000991.

    \bibitem{WangEtAl2025UnstableSingularities}
    Y.~Wang, M.~Bennani, J.~Martens, S.~Racani\`ere, S.~Blackwell,
    A.~Matthews, S.~Nikolov, G.~Cao-Labora, D.~S.~Park, M.~Arjovsky,
    D.~Worrall, C.~Qin, F.~Alet, B.~Kozlovskii, N.~Toma\v{s}ev,
    A.~Davies, P.~Kohli, T.~Buckmaster, B.~Georgiev, J.~G\'omez-Serrano,
    R.~Jiang, and C.-Y.~Lai,
    ``Discovery of Unstable Singularities,''
    arXiv:2509.14185 [math.AP] (2025).
    doi:10.48550/arXiv.2509.14185.


    \bibitem{AllenFeldman1993}
    P. B. Allen and J. L. Feldman,
    ``Thermal conductivity of disordered harmonic solids,''
    \textit{Physical Review B}
    \textbf{48}
    (1993), 12581.


    \bibitem{Buchert2000}
    Buchert, Thomas,
    ``On average properties of inhomogeneous fluids in general relativity: Dust cosmologies,''
    \textit{Gen. Relativ. Gravit.}
    \textbf{32}
    (2000), 105--125.
    doi: 10.1023/A:1001800617177

    \bibitem{Buchert2001}
    Buchert, Thomas,
    ``On average properties of inhomogeneous cosmologies,''
    \textit{Gen. Relativ. Gravit.}
    \textbf{33}
    (2001), 1381--1405.
    doi: 10.1023/A:1012061725841


    \bibitem{Englert1996}
    Englert, B.-G.,
    ``Fringe Visibility and Which-Way Information: An Inequality,''
    \textit{Phys. Rev. Lett.}
    \textbf{77}
    (1996), 2154--2157.
    doi: 10.1103/PhysRevLett.77.2154

    \bibitem{Goldau2025_STC}
    Pieter Goldau,
    ``The Simplicity Codex''
    (2025).
    Sixteen-stage parameter-free ontology, cited as STC
    doi: 10.5281/zenodo.17068210

    \bibitem{Hardy1963}
    R. J. Hardy,
    ``Energy-Flux Operator for a Lattice,''
    \textit{Physical Review}
    \textbf{132}
    (1963), 168.

    \bibitem{Iskandarani2025Canon034}
    Iskandarani, Omar,
    ``Swirl-String Theory (SST) Canon v0.3.4: Core Postulates, Constants, and Boxed Master Equations''
    (Zenodo, 2025).
    Single source of truth for SST symbols, constants, and canonical equations; required citation for dependent works.
    doi: 10.5281/zenodo.17014358

    \bibitem{Iskandarani2025Hydrogen}
    Iskandarani, Omar,
    ``Long-Distance Swirl Gravity from Chiral Swirling Knots with Central Holes''
    (Zenodo, 2025).
    doi: 10.5281/zenodo.17155855

    \bibitem{Iskandarani2025_Lagrangian}
    Iskandarani, Omar,
    ``Swirl-String Theory (SST) Lagrangian: Emergent Relativistic EFT with Preferred Foliation''
    (Zenodo, 2025).
    doi: 10.5281/zenodo.16956665

    \bibitem{Jackson1999}
    John David Jackson,
    \textit{Classical Electrodynamics}
    (3rd ed.,
    Wiley, 1999).

    \bibitem{Kelvin1869}
    W. Thomson (Lord Kelvin),
    ``On vortex motion,''
    \textit{Transactions of the Royal Society of Edinburgh}
    \textbf{25}
    (1869), 217--260.

    \bibitem{KhatiwadaQian2025}
    Khatiwada, P. and Qian, X.-F.,
    ``Wave-particle duality ellipse and application in quantum imaging with undetected photons,''
    \textit{Phys. Rev. Research}
    \textbf{7}
    (2025), 033033.
    doi: 10.1103/PhysRevResearch.7.033033

    \bibitem{PDG2024}
    Particle Data Group,
    ``Review of Particle Physics''
    (2024).

    \bibitem{Peierls1929}
    R. Peierls,
    ``Zur Theorie der spezifischen Wärme,''
    \textit{Annalen der Physik}
    \textbf{395}
    (1929), 1055.

    \bibitem{PeskinSchroeder1995}
    Michael E. Peskin and Daniel V. Schroeder,
    \textit{An Introduction to Quantum Field Theory}
    (Westview Press, 1995).

    \bibitem{Simoncelli2019Unified}
    M. Simoncelli et al.,
    ``Unified theory of thermal transport in crystals and disordered solids,''
    \textit{Nature Physics}
    \textbf{18}
    (2022), 1180.

    \bibitem{Weinberg1967}
    Weinberg, Steven,
    ``A Model of Leptons,''
    \textit{Physical Review Letters}
    \textbf{19}
    (1967), 1264--1266.
    doi: 10.1103/PhysRevLett.19.1264

    \bibitem{Zurek2003}
    Zurek, W. H.,
    ``Decoherence, einselection, and the quantum origins of the classical,''
    \textit{Rev. Mod. Phys.}
    \textbf{75}
    (2003), 715--775.
    doi: 10.1103/RevModPhys.75.715

    \bibitem{Gibbons2002_MaxTension}
    G.~W.~Gibbons,
    ``The Maximum Tension Principle in General Relativity,''
    \emph{Foundations of Physics} \textbf{32}, 1891--1901 (2002).
    doi:10.1023/A:1022370717626.

    \bibitem{Planck1900_Irreversible}
    M.~Planck,
    ``\"Uber irreversible Strahlungsvorg\"ange,''
    \emph{Annalen der Physik} \textbf{306}, 69--122 (1900).
    doi:10.1002/andp.19003060105.

    \bibitem{ClassicalElectronRadius}
    D.~J.~Griffiths,
    \emph{Introduction to Quantum Mechanics},
    Prentice--Hall, 1995, Sec.~10.3 (classical electron radius).



\end{thebibliography}

\end{document}