%! Author = Omar Iskandarani
%! Date = 2025-09-06

\documentclass[11pt]{article}

% Packages
\usepackage{amsmath,amssymb}
\usepackage{geometry}
\usepackage{graphicx}
\usepackage{hyperref}
\geometry{margin=1in}

% Macros (SST explicit symbols, no undefined macros in chat)
\newcommand{\vswirl}{\mathbf{v}_{\!\boldsymbol{\circlearrowleft}}}
\newcommand{\SwirlClock}{S_t^{\boldsymbol{\circlearrowleft}}}
\newcommand{\rhoF}{\rho_{\!f}}
\newcommand{\GammaC}{\Gamma_C}

\begin{document}

\title{Long-Distance Swirl Gravity from Chiral Swirling Knots with Central Holes}
\author{Omar Iskandarani}
\date{\today}
\maketitle

\begin{abstract}
We derive long-range gravitational attraction in Swirl--String Theory (SST) as a direct consequence of \emph{chiral swirling knots}---topological vortex filaments such as the trefoil ($3_1$), cinquefoil ($5_1$, $5_2$), and stevedore ($6_1$).
Each knot encloses a central rotational line, which acts as an anchor of circulation.
Using Cauchy's integral theorem, we show that the circulation measured around any loop enclosing this axis is quantized by the knot's winding number.
This quantization is expressed by the Swirl Clock $\SwirlClock$, and its persistence explains why neutral molecules (e.g.\ H$_2$) attract in otherwise flat space: their knots are connected via the same central swirl line extending beyond the equal-pressure boundary.
\end{abstract}

\section{Chiral Swirling Knots and Central Holes}
Consider a chiral knot $K$ embedded in $\mathbb{R}^3$, such as:
\[
3_1 \ (\text{trefoil}), \quad
5_1 \ (\text{cinquefoil torus}), \quad
5_2 \ (\text{cinquefoil twist}), \quad
6_1 \ (\text{stevedore}).
\]
Each knot can be parametrized on a torus with major radius $R$ and minor radius $r$.
The core tube of radius $r_c$ supports a tangential swirl velocity $\vswirl$, defining the \emph{Swirl Clock} $\SwirlClock$.

A defining feature is that all these knots possess a \emph{central hole} threaded by a straight axis (taken as the $z$-axis).
This axis is the ``fabric line'' of flat space: it is the singularity in the analytic swirl potential.

\section{Cauchy Integral and Circulation Quantization}
Let $C$ be a closed loop in the $x$--$y$ plane encircling the $z$-axis.
By Cauchy's integral theorem, for an analytic swirl potential $W(z)=\Phi+i\Psi$,
\begin{equation}
\oint_C \vswirl \cdot d\mathbf{l} =
\begin{cases}
0, & \text{if no singularity inside,}\\[4pt]
2\pi i\, \mathrm{Res}\!\left(\frac{dW}{dz}, z=0\right), & \text{if axis enclosed.}
\end{cases}
\end{equation}
In SST, the residue corresponds to the circulation quantum
\begin{equation}
\kappa = 2\pi\, v_c\, r_c,
\end{equation}
where $v_c$ is the swirl speed at the core boundary $r_c$.

If the knot winds around the axis $n$ times (its \emph{linking number}),
\begin{equation}
\GammaC = n \kappa.
\end{equation}
This is the Cauchy--Kelvin equivalence: long-distance swirl circulation is locked to integer multiples of $\kappa$ by topology.

\section{Swirl Clock and Chiral Gravity}
The Swirl Clock $\SwirlClock$ connects circulation to time dilation:
\begin{equation}
dt_{\text{local}} = dt_\infty \sqrt{1 - \frac{\lVert \vswirl \rVert^2}{c^2}}.
\end{equation}
Chirality (left- vs right-handed knotting) determines whether $\SwirlClock$ advances forward (matter) or backward (antimatter) relative to the background.

Thus, two chiral knots coupled through the same central line experience a net pressure gradient:
\begin{equation}
\Delta p = -\tfrac{1}{2}\rhoF \, \lVert \vswirl \rVert^2,
\end{equation}
which is transmitted outward along the axis and manifests as \emph{swirl gravity}.

\section{Hydrogen Molecules in Flat Space}
Even in Euclidean space (no curvature), two hydrogen molecules each represented by a chiral trefoil ($3_1$) have their central holes threaded by the same fabric axis.
Their swirl clocks couple via the shared circulation residue:
\[
\Gamma_{\text{H}_2-\text{H}_2} = \kappa_{(1)} + \kappa_{(2)}.
\]
This explains why neutral molecules attract at long range: the attraction is not from spacetime curvature but from \emph{quantized swirl circulation anchored to a shared axis}.

\section{Orientation and Stability of Knot Triads}
The interaction between multiple chiral knots is highly sensitive to their
relative orientation. Each knotted filament generates an anisotropic far-field
swirl, analogous to a dipole--quadrupole expansion in classical potential
theory. When three knots are placed on the vertices of an equilateral triangle
in the $x$--$y$ plane, the direction in which their ``active lobes'' point
determines whether the central swirl axes reinforce or cancel.

We investigated three candidate configurations:
\begin{enumerate}
    \item[(A)] The baseline orientation, where each knot faces inward toward
    the central stagnation point.
    \item[(B)] A mirror transformation of the central knot, flipping its
    chirality while keeping the others fixed.
    \item[(C)] A uniform $180^\circ$ rotation of all three knots about their
    own swirl axes.
\end{enumerate}
For each case, the Biot--Savart induced velocity at the swirl axis of knot
$i$ due to the other two knots,
\begin{equation}
\mathbf{v}_i \;=\; \sum_{j\neq i} \frac{\kappa}{4\pi}
\int_{\mathcal{C}_j} \frac{d\boldsymbol{\ell}\times
(\mathbf{x}_i-\mathbf{x}')}{\lVert \mathbf{x}_i-\mathbf{x}' \rVert^3},
\end{equation}
was evaluated numerically. In scenario~(A), the axis velocities
$\lVert\mathbf{v}_i\rVert$ were of order $10^{-11}$--$10^{-10}$~m/s,
non-uniform across the three knots. Mirroring the central knot (B) slightly
reduced the imbalance, but left one axis strongly driven. By contrast, in the
rotated configuration (C) the induced axis velocities dropped by nearly an
order of magnitude, to $\sim 10^{-12}$~m/s, and became more uniform across
the triad.

This analysis suggests that the $180^\circ$ rotated state is closer to a
\emph{relative equilibrium}---a configuration where each knot experiences
minimal external drive and the net Kelvin impulse
\begin{equation}
\mathbf{P} \;=\; \rhoF \sum_j \kappa \int_{\mathcal{C}_j}
\mathbf{x} \times d\boldsymbol{\ell}
\end{equation}
approaches zero. Physically, the rotation points the strongest swirl lobes
outward, away from the central axis, so that the weaker multipolar fields
cancel at long range. This prevents runaway attraction or collapse and yields
a stable bound triad.

Thus, the choice of $180^\circ$ rotation is not arbitrary: it emerges as the
orientation that minimizes axis--axis forcing and stabilizes the coupled swirl
system.

\section{Conclusion}
Long-distance gravitational attraction in SST is a manifestation of topological quantization: chiral knots with central holes enforce non-vanishing circulation residues along a central line.
This mechanism is general across knots $3_1, 5_1, 5_2, 6_1$, and directly explains molecular-scale attraction in flat space.

\bibliographystyle{plain}
\bibliography{swirlgravity}

\end{document}