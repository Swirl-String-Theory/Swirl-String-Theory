% tikzlibrarysstknots.code.tex
\ProvidesFile{tikzlibrarysstknots.code.tex}[2025/09/01 SST knot helpers]

\RequirePackage{tikz}
\usetikzlibrary{knots,hobby,calc,intersections,decorations.pathreplacing,shapes.geometric,spath3}

% ------- Shared styles (from your preamble) -------
\tikzset{
    knot diagram/every strand/.append style={ultra thick, black},
    every path/.style={black,line width=2pt},
    every node/.style={transform shape,knot crossing,inner sep=1.5pt},
    every knot/.style={line cap=round,line join=round,very thick},
    strand/.style={line cap=round,line join=round,line width=3pt,draw=black},
    over/.style={preaction={draw=white,line width=6.5pt}},
    sst/ring A/.style={draw=black, line width=3pt},
    sst/ring B/.style={draw=black,  line width=3pt},
    sst/ring C/.style={draw=black, line width=3pt},
}

% ------- Guides toggle -------
\newif\ifsstguides
\sstguidestrue

% ------- Helper: label & skeleton for points P1..Pn -------
\newcommand{\SSTGuidesPoints}[2]{% #1=basename (e.g. P), #2=last index
    \ifsstguides
    \foreach \i in {1,...,#2}{
        \fill[blue] (#1\i) circle (1.2pt);
        \node[blue,font=\scriptsize,above] at (#1\i) {\i};
    }
    \draw[gray!40, dashed] \foreach \i [remember=\i as \lasti (initially 1)] in {2,...,#2,1} { (#1\lasti)--(#1\i) };
    \fi
}

% ====================================================================
% LATEX SWIRL STRING THEORY
% KNOT LIBRARY
% ====================================================================
\newcommand{\SSTfourPanel}{% optional arg = flip list
%-------------------------------------------
% 4-panel cartoon (same scale / same view)
%-------------------------------------------
\begin{figure}[t]
\centering

% -------- helper macro: one panel --------------
% args: #1 = loop radius, #2 = ON/OFF style, #3 = caption text under panel
\newcommand{\OnePanel}[3]{%
    \begin{tikzpicture}[x=1cm,y=1cm,scale=0.60]
    % parameters (shared across all panels)
    \def\R{3.2}   % major radius (center of torus band)
    \def\r{1.1}   % half-thickness of torus band
    \def\tilt{0.55}% yscale to hint 3D tilt of torus
    \def\RL{#1}   % test loop radius

    % axes (same in all panels)
    \draw[axes,->] (-0.25,0) -- (6.8,0) node[below right] {$x$};
    \draw[axes,->] (0,-3.0) -- (0,3.0) node[left] {$y$};

    % central axis marker (z coming out of page)
    \fill (0,0) circle(2pt);
    \node[below left] at (0,0) {\footnotesize central axis ($z$)};

    % --- torus "band" with a little 3D hint (tilted annulus) ---
    % outer and inner tilted ellipses (yscale = tilt)
    \begin{scope}[yscale=\tilt]
    \draw[very thick, gray!40] (0,0) circle (\R+\r);
    \draw[very thick, gray!40] (0,0) circle (\R-\r);
    % faint fill between ellipses for depth cue
    \path[fill=gray!10,draw=none] (0,0) circle (\R+\r);
    \clip (0,0) circle (\R+\r);
    \fill[white] (0,0) circle (\R-\r);
    \end{scope}

    % --- schematic trefoil projection with over/under cues ---
    % front arc (solid), back arc (dashed + lighter) to hint 3D crossing
    % This is a clean cartoon, not exact geometry.
    % Back (far) lobe:
    \draw[knotline, dashed, opacity=0.35]
    plot[smooth cycle, tension=0.9]
    coordinates {
        ({\R+0.2},  0.10)
        ({\R-0.4},  0.95)
        ({\R+0.6},  0.85)
        ({\R+0.3}, -0.9)
        ({\R-0.9}, -1.05)
    };
    % Front (near) lobe with arrows (swirl direction)
    \draw[knotline]
    plot[smooth cycle, tension=0.9]
    coordinates {
        ({\R},     0.00)
        ({\R-0.6}, 1.00)
        ({\R+0.8}, 0.80)
        ({\R+0.4},-1.00)
        ({\R-0.8},-1.20)
    };
    % flow arrows along the visible lobe
    \foreach \t/\L in {0/0.35, 60/0.35, 180/0.35}{
        \draw[looparrow, blue!70]
        ({\R+0.7*cos(\t)},{0.1+0.9*sin(\t)}) --
        ++({0.7*cos(\t+25)*\L},{0.7*sin(\t+25)*\L});
    }

% --- test loop (the circulation contour) ---
    \draw[#2] (0,0) circle (\RL);

    % panel caption (under)
    \node at (3.0,-2.7) {\small #3};
    \end{tikzpicture}%
}

% --------- lay out four panels on one row ----------
\OnePanel{0.55}{testloopOFF}{(a)\; very small loop $\Rightarrow\ \Gamma=0$}\hspace{0.9cm}
\OnePanel{2.65}{testloopON}{(b)\; inside torus band $\Rightarrow\ \Gamma=-3\,\kappa$}\hspace{0.9cm}
\OnePanel{3.25}{testloopON}{(c)\; still inside band $\Rightarrow\ \Gamma=-3\,\kappa$}\hspace{0.9cm}
\OnePanel{4.60}{testloopOFF}{(d)\; far outside $\Rightarrow\ \Gamma\approx 0$}

\caption{Four identical-scale, top-down panels. The gray tilted annulus hints at the torus where the trefoil lives (donut seen at an angle).
The blue curve is a schematic trefoil; dashed parts are “behind” (3D cue).
The green \textbf{test loops} (b,c) lie \emph{within} the torus band and measure the plateau \(\Gamma=\mathrm{Lk}\cdot\kappa=\pm 3\,\kappa\) (sign by orientation).
Red loops (a,d) do not link the filament \(\Rightarrow \Gamma\approx 0\).}
\label{fig:fourpanel-3D-cartoon}
\end{figure}
}

\newcommand{\CirculationLoop}{

% ============================
% Figure 1: Trefoil + central axis + circulation loops
% ============================
\begin{figure}[t]
\centering
\begin{tikzpicture}[scale=1.05]
    % Parameters (schematic torus cross-section)
\def\R{3.2}   % major radius (to center of tube annulus)
\def\r{1.1}   % tube half-width in the drawing (annulus thickness/2)
\def\Rin{\R-\r}
\def\Rout{\R+\r}

% Coordinate axes (x-y plane view)
\draw[->,axisline] (-0.3,0) -- (6.8,0) node[below right] {$x$};
\draw[->,axisline] (0,-3.2) -- (0,3.2) node[left] {$y$};

% Central rotational axis (z-axis comes out-of-plane at (0,0))
\fill[black] (0,0) circle(2pt);
\node[below left] at (0,0) {central axis ($z$)};

% Annulus representing the torus "core circle" band (projection)
\draw[thick, gray!50] (0,0) circle (\Rin);
\draw[thick, gray!50] (0,0) circle (\Rout);
\node[labelbox,gray!50!black] at ({\R},-2.2) {torus annulus $[R-r,\,R+r]$};

% Circulation test loops (z=0 plane); inside annulus => plateau q=3
\foreach \RR in {2.1,2.6,3.2} {
    \draw[corecircle] (0,0) circle (\RR);
}
\node[labelbox] at (2.1,1.2) {$\Gamma \approx -3\,\kappa$};
\node[labelbox] at (2.6,-1.2) {$\Gamma \approx -3\,\kappa$};
\node[labelbox] at (3.2,1.2) {$\Gamma \approx -3\,\kappa$};

% Outside loop: ~0
\draw[corecircle] (0,0) circle (4.4);
\node[labelbox] at (4.4,-1.2) {$\Gamma \approx 0$};

% Schematic trefoil projection (simple stylized 3-lobe curve around core circle)
\begin{scope}
\draw[knotline]
plot[smooth cycle, tension=0.8]
coordinates {
    ({\R},0.0)
    ({\R-0.9},0.9)
    ({\R-0.1},1.6)
    ({\R+0.9},0.8)
    ({\R+0.6},-0.8)
    ({\R-0.4},-1.4)
};
% swirl direction arrows along the knot
\foreach \ang/\rad in {20/0.4,150/0.4,280/0.4}{
    \draw[swirlarrow, blue!65] ({\R+cos(\ang)*1.0},{sin(\ang)*1.0}) -- ++({cos(\ang+25)*\rad},{sin(\ang+25)*\rad});
}
\end{scope}

% Caption annotations
\node[align=left, labelbox] at (5.9,2.5) {%
    \small Trefoil $3_1$ (\emph{schematic})\\
    meridional winding $q=3$\\
    $\Rightarrow\ \Gamma=\mathrm{Lk}\cdot\kappa=-3\,\kappa$
};
\end{tikzpicture}
\caption{Trefoil knot with central axis and circulation loops in the $z{=}0$ plane.
Loops whose spanning disk intersects the filament within the torus annulus measure a constant plateau $\Gamma=\mathrm{Lk}\cdot\kappa=\pm 3\,\kappa$ (sign by orientation).
Loops outside the annulus do not enclose the filament and give $\Gamma\approx 0$.}
\label{fig:trefoil_axis_plateau}
\end{figure}

}


\newcommand{\Yjunction}{
% ============================
% Figure 2: Baryon composite tube (Y-junction -> single tube)
% ============================
\begin{figure}[t]
\centering
\begin{tikzpicture}[scale=1.0]
    % Central axis
\draw[axisline] (0,-2.2) -- (0,2.2);
\node[left] at (0,2.2) {$z$};

% Three incoming quark tubes (left side), colored
\draw[tube, red!70] (-5, 1.4) .. controls + (1,0) and +(-1,0.5) .. (-2, 0.6);
\draw[tube, green!70!black] (-5, 0.0) .. controls + (1,0) and +(-1,0.0) .. (-2, 0.0);
\draw[tube, blue!70] (-5,-1.4) .. controls + (1,0) and +(-1,-0.5) .. (-2,-0.6);

% Y-junction region
\draw[tube, gray!30,line cap=round] (-2,0.6) -- (-1.2,0.3);
\draw[tube, gray!30,line cap=round] (-2,0.0) -- (-1.2,0.0);
\draw[tube, gray!30,line cap=round] (-2,-0.6) -- (-1.2,-0.3);

% Merge into single composite tube (right side)
\draw[tube, black!80] (-1.2,0.0) .. controls +(0.8,0) and +(-0.8,0) .. (3.4,0.0);

% Swirl direction arrows on tubes
\foreach \x/\y in {-4.3/1.30, -3.6/1.00, -2.9/0.75}{
    \draw[vline, red!70] (\x,\y) -- ++(0.7,0.05);
}
\foreach \x/\y in {-4.3/0.00, -3.6/0.00, -2.9/0.00}{
    \draw[vline, green!70!black] (\x,\y) -- ++(0.7,0.00);
}
\foreach \x/\y in {-4.3/-1.30, -3.6/-1.00, -2.9/-0.75}{
    \draw[vline, blue!70] (\x,\y) -- ++(0.7,-0.05);
}
\foreach \x in {-0.8,0.2,1.2,2.2}{
    \draw[vline, black!80] (\x,0.0) -- ++(0.8,0.0);
}

% Labels
\node[align=left, labelbox] at (-4.6,2.0) {Three quark tubes\\[-2pt] $\Gamma_1=\Gamma_2=\Gamma_3=\kappa$};
\node[align=left, labelbox] at (2.3,0.9) {Composite tube\\[-2pt] $\Gamma_{\text{baryon}}=3\kappa$\\[-2pt] $v_{\theta,\mathrm{eff}}=3v_c$};
\node[align=left, labelbox] at (0.5,-1.7) {Additivity (Kelvin): $\ \Gamma=\sum_i \Gamma_i$};

% Central axis marker
\fill (0,0) circle (2pt);
\node[below right] at (0,0) {central axis};
\end{tikzpicture}


}



% ====================================================================
% 6_1: Down Quarck Stevedore
% Usage: \SSTdown
% ====================================================================
\newcommand{\SSTdown}[1][2,4,6,8]{% optional arg = flip list
    \begin{tikzpicture}[use Hobby shortcut]
        \coordinate (P1)  at ( 0,  1.5);
        \coordinate (P2)  at (-2,  2);
        \coordinate (P3)  at (-1.5, 0);
        \coordinate (P4)  at ( 1, -1.5);
        \coordinate (P5)  at (-1.5,-2);
        \coordinate (P6)  at (-2.5,-0.5);
        \coordinate (P7)  at (-1.5, 1);
        \coordinate (P8)  at ( 0,  3);
        \coordinate (P9)  at ( 1.5, 1);
        \coordinate (P10) at ( 2.5,-0.5);
        \coordinate (P11) at ( 1.5,-2);
        \coordinate (P12) at (-1, -1.5);
        \coordinate (P13) at ( 1.5, 0);
        \coordinate (P14) at ( 2,  2);
        \coordinate (P15) at ( 0,  1.5); % = P1

        \begin{knot}[
            consider self intersections,
            clip width=5pt, clip radius=3pt,
            ignore endpoint intersections=false,
            flip crossing/.list={#1}
        % draft mode=crossings % uncomment to see numbers
        ]
        \strand
        ([closed] P1)..(P2)..(P3)..(P4)..(P5)..(P6)..(P7)..(P8)%
        ..(P9)..(P10)..(P11)..(P12)..(P13)..(P14)..(P15);
        \end{knot}
        \SSTGuidesPoints{P}{15}
    \end{tikzpicture}%
}

% ====================================================================
% 6_1: Down Quarck Stevedore
% Usage: \SSTdownRound
% ====================================================================
\newcommand{\SSTroundDown}[1][2,4,6,8,10,12,14,16]{% optional arg = flip list
        \begin{tikzpicture}[use Hobby shortcut]
        \coordinate (P1)  at ( 0,  2);
        \coordinate (P2)  at (-2,  2);
        \coordinate (P3)  at (-1.5, -1);
        \coordinate (P4)  at ( 0.5, -2);
        \coordinate (P5)  at (-1.5,-2);
        \coordinate (P6)  at (-2.5,-0.5);
        \coordinate (P7)  at (-2, 1);
        \coordinate (P8)  at ( 0,  3);
        \coordinate (P9)  at ( 2, 1);
        \coordinate (P10) at ( 2.5,-0.5);
        \coordinate (P11) at ( 1.5,-2);
        \coordinate (P12) at (-0.5, -2);
        \coordinate (P13) at ( 1.5, -1);
        \coordinate (P14) at ( 2,  2);
        \coordinate (P15) at ( 0,  2); % = P1

        \begin{knot}[
            consider self intersections,
            clip width=5pt, clip radius=3pt,
            ignore endpoint intersections=false,
            flip crossing/.list={#1}
        % draft mode=crossings % uncomment to see numbers
        ]
        \strand
        ([closed] P1)..(P2)..(P3)..(P4)..(P5)..(P6)..(P7)..(P8)%
        ..(P9)..(P10)..(P11)..(P12)..(P13)..(P14)..(P15);
        \end{knot}
        \SSTGuidesPoints{P}{15}
        \end{tikzpicture}
}
% ====================================================================
% 5_2: Up Quark
% Usage: \SSTup
% ====================================================================
\newcommand{\SSTup}[1][2,4,6,8]{% optional arg = flip list
    \begin{tikzpicture}[use Hobby shortcut]
        \coordinate (P1) at ( 2.0,  2.0);
        \coordinate (P2) at ( 1.8,  0.0);
        \coordinate (P3) at (-2.3, -1.0);
        \coordinate (P4) at ( 0.5,  1.0);
        \coordinate (P5) at (-2.0,  2.0);
        \coordinate (P6) at (-1.8,  0.0);
        \coordinate (P7) at ( 2.3, -1.0);
        \coordinate (P8) at (-0.5,  1.0);
        \coordinate (P9) at ( 2.0,  2.0); % = P1

        \begin{knot}[
            consider self intersections,
            clip width=5pt, clip radius=3pt,
            ignore endpoint intersections=false,
            flip crossing/.list={#1}
        % draft mode=crossings
        ]
        \strand
        ([closed] P1)..(P2)..(P3)..(P4)..(P5)..(P6)..(P7)..(P8)..(P9);
        \end{knot}
        \SSTGuidesPoints{P}{9}
    \end{tikzpicture}%
}

% ====================================================================
% 5_2: Up Quark
% Usage: \SSTupRound
% ====================================================================
\newcommand{\SSTroundUp}[1][2,4,6,8,10,12,14,16,18]{% optional arg = flip list
\begin{tikzpicture}[use Hobby shortcut]
\coordinate (P1) at (2, -1.5);
\coordinate (P2) at (1.5, 1);
\coordinate (P3) at (0, 2);
\coordinate (P4) at (-2, 1);
\coordinate (P5) at (-1, -1.5);
\coordinate (P6) at (0.5, -2);
\coordinate (P7) at (-1.25, -2.25);
\coordinate (P8) at (-2, -1.5);
\coordinate (P9) at (-1.5, 1);
\coordinate (P10) at (0, 2);
\coordinate (P11) at (2, 1);
\coordinate (P12) at (1, -1.5);
\coordinate (P13) at (-0.5, -2);
\coordinate (P14) at (1.25, -2.25);
\coordinate (P15) at (2, -1.5); % = P1

\begin{knot}[
    consider self intersections,
    clip width=5pt, clip radius=3pt,
    ignore endpoint intersections=false,
    flip crossing/.list={#1}
% draft mode=crossings
]
\strand
([closed] P1)..(P2)..(P3)..(P4)..(P5)..(P6)..(P7)..(P8)..(P9)..(P10)..(P11)..(P12)..(P13)..(P14)..(P15);
\end{knot}
\SSTGuidesPoints{P}{15}
\end{tikzpicture}
}
% ====================================================================
% Usage: \SSTSeptfoil
% ====================================================================
\newcommand{\SSTSeptfoil}{%
    \begin{tikzpicture}[use Hobby shortcut]
        \path[spath/save=7-1]
        ([closed]90:2) foreach \k in {1,...,7} {
            .. (90-360/7+\k*720/7:1.5) .. (90+\k*720/7:2)
        } (90:2);
        \tikzset{
            every spath component/.style={draw},
            spath/knot={7-1}{15pt}{1,3,...,15}
        }
    \end{tikzpicture}
}

% ====================================================================
% Usage: \\SSTSinquefoil
% ====================================================================
\newcommand{\SSTSincfoil}{%
    \begin{tikzpicture}
        \begin{knot}[
            consider self intersections=true,
    %        draft mode=crossings,
            flip crossing/.list={2,4},
            only when rendering/.style={
                show curve controls
            }
        ]
        \strand (2,0) .. controls +(0,1.0) and +(54:1.0) .. (144:2) .. controls +(54:-1.0) and +(18:-1.0) .. (-72:2) .. controls +(18:1.0) and +(162:-1.0) .. (72:2) .. controls +(162:1.0) and +(126:1.0) .. (-144:2) .. controls +(126:-1.0) and +(0,-1.0) .. (2,0);
        \end{knot}
    \end{tikzpicture}
}
% ====================================================================
% FigureEightDark: two circles, centers at (±D, 0), each radius = R (in cm)
% Usage: \SSTFigureEightDark
% Example: \SSTFigureEightDark
% ====================================================================
\newcommand{\SSTFigureEightDark}{%
    \begin{tikzpicture}[use Hobby shortcut]
        % ===== Up Quark knot (5_2): control points =====
        \coordinate (P1) at (-2.0,  -2.0);  % start / close
        \coordinate (P2) at (-2.0,  2.0);
        \coordinate (P3) at ( 1, -0.5);
        \coordinate (P4) at (-1, -0.5);
        \coordinate (P5) at ( 2.0,  2.0);
        \coordinate (P6) at ( 2.0, -2.0);
        \coordinate (P7) at (-1,  0.5);
        \coordinate (P8) at ( 1,  0.5);
        \coordinate (P9) at (-2.0, -2.0);  % = P1 (smooth closed path)

        \begin{knot}[
            consider self intersections,
            % draft mode=crossings,              % show crossing indices while tuning
            ignore endpoint intersections=false,
            clip width=5pt, clip radius=3pt,
            flip crossing/.list={2,4}        % same over/under pattern as your snippet
        ]
        \strand
        ([closed] P1)..(P2)..(P3)..(P4)..(P5)..(P6)..(P7)..(P8)..(P9);
        \end{knot}

    \end{tikzpicture}
}



% ====================================================================
% Hopf link: two circles, centers at (±D, 0), each radius = R (in cm)
% Usage: \SSTHopfLink[<flip list>]{<R cm>}{<D cm>}
% Example: \SSTHopfLink[2]{2}{1}
% ====================================================================
\newcommand{\SSTHopfLink}[3][2]{%
    \def\R{#2}\def\D{#3}%
    \begin{tikzpicture}
        \begin{knot}[
            consider self intersections,
            clip width=5pt, clip radius=3pt,
            ignore endpoint intersections=false,
            flip crossing/.list={#1}
        % draft mode=crossings
        ]
        \strand[sst/ring A] (\D,0)   circle[radius=\R cm];
        \strand[sst/ring B] (-\D,0)  circle[radius=\R cm];
        \end{knot}%
    \end{tikzpicture}%
}

% ====================================================================
% Borromean rings: three circles at vertices of an equilateral triangle
% Horizontal half-spacing = D (cm), so centers at:
%   ( +D, 0 ), ( -D, 0 ), ( 0, sqrt(3)*D )
% All radii = R (cm)
% Usage: \SSTBorromean[<flip list>]{<R cm>}{<D cm>}
% Example: \SSTBorromean[3,4]{2}{1}
% ====================================================================
\newcommand{\SSTBorromean}[3][3,4]{%
    \def\R{#2}\def\D{#3}%
    \begin{tikzpicture}
        \pgfmathsetmacro{\H}{\D*1.7320508075688772} % sqrt(3)*D
        \begin{knot}[
            consider self intersections,
            clip width=5pt, clip radius=3pt,
            ignore endpoint intersections=false,
            flip crossing/.list={#1}
        % draft mode=crossings
        ]
        \strand[sst/ring A] ( \D, 0)  circle[radius=\R cm];
        \strand[sst/ring B] (-\D, 0)  circle[radius=\R cm];
        \strand[sst/ring C] ( 0, \H)  circle[radius=\R cm];
        \end{knot}%
    \end{tikzpicture}%
}

% =========================
% Figure-8 knot (4_1)
% Usage: \SSTFigureEight[<flip list>]
% Default flip list matches your snippet: {1,3,5,7}
% =========================
\newcommand{\SSTFigureEight}[1][1,3,5,7]{%
    \begin{tikzpicture}[use Hobby shortcut]
        \begin{knot}[
            consider self intersections,
            clip width=6pt,
            clip radius=3pt,
            ignore endpoint intersections=false,
            flip crossing/.list={#1}
        % draft mode=crossings, % uncomment while tuning
        ]
        \strand
        ([closed]0,0)
        .. ( 1.5,  1.0)
        .. ( 0.5,  2.0)
        .. (-0.5,  1.0)
        .. ( 0.5,  0.0)
        .. ( 0.0, -0.5)
        .. (-0.5,  0.0)
        .. ( 0.5,  1.0)
        .. (-0.5,  2.0)
        .. (-1.5,  1.0)
        .. ( 0.0,  0.0);
        \end{knot}
        % small spacer to avoid clipping, matches your snippet
        \path (0,-.7);
    \end{tikzpicture}%
}

% =========================
% Trefoil (3_1), alternating 3 sides red / 3 sides blue
% This reproduces your "3 sides blue, 3 side red" using Hobby blanks + reuse.
% Usage: \SSTTrefoilAltColors
% (Scale from the surrounding tikzpicture if needed.)
% =========================
\newcommand{\SSTTrefoilAltColors}{%
    \begin{tikzpicture}[
        use Hobby shortcut,
        every path/.style={
            line width=1mm,
            white,
            double=red,
            double distance=.5mm
        }
    ]
        \def\nfoil{3}
        % base Hobby path with "soft" blanks at alternating arcs
        \draw ([closed]0,2)
        \foreach \k in {1,...,\nfoil} {
            .. ([blank=soft]90+360*\k/\nfoil-180/\nfoil:-.5)
            .. (90+360*\k/\nfoil:2)
        };
        % reuse the previous path with inverted blanks, drawn in blue
        \draw[
            use previous Hobby path={invert soft blanks,disjoint},
            double=blue
        ];
    \end{tikzpicture}%
}

% =========================
% Trefoil (3_1) via spath3 "knot" post-processor (two-color components)
% This reproduces your "1 side blue, 1 side red" approach.
% Usage: \SSTTrefoilSpath[<gap>][<flip list>]
% Defaults: gap=8pt, flips={1,3,5}
% =========================
\newcommand{\SSTTrefoilSpath}[2][8pt]{%
    \begin{tikzpicture}
        \begingroup
        \def\Gap{#1}%
        \def\Flips{#2}%
        % Saved spath named 'trefoil' (local to this group)
        \path[spath/save=trefoil]
        ([closed]90:2)
        \foreach \k in {1,...,3} {
            .. (-30+\k*240:.5) .. (90+\k*240:2)
        } (90:2);
        % Now render it as a knot with the requested gap + crossing flips
        \tikzset{spath/knot={trefoil}{\Gap}{\Flips}}
        \endgroup
    \end{tikzpicture}%
}