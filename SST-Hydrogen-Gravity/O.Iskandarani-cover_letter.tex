\documentclass[a4paper,10pt]{letter}

\usepackage[T1]{fontenc}
\usepackage[utf8]{inputenc}
\usepackage{lmodern}
\usepackage[hidelinks]{hyperref}
\usepackage{microtype}
\usepackage[margin=1in]{geometry}
\usepackage{amstext}

% Sender info
\signature{Omar Iskandarani\\
Independent Researcher, Groningen, The Netherlands\\
ORCID: 0009-0006-1686-3961\\
Email: \href{mailto:info@omariskandarani.com}{info@omariskandarani.com}}
\address{Omar Iskandarani\\
Vinkenstraat 86A,\\
9713TK Groningen, The Netherlands}

\date{\today}

\begin{document}

\begin{letter}{Editors\\\textit{Foundations of Physics}}
\opening{Dear Editor,}

I am pleased to submit the manuscript \textit{Long-Distance Swirl Gravity from Chiral Swirling Knots with Central Holes} for consideration in \textit{Foundations of Physics}. The work proposes a flat-space, topological mechanism for long-range attraction within Swirl–String Theory (SST) and advances concrete, falsifiable predictions that connect conceptual foundations to laboratory tests.

Conceptually, we derive integer plateaus of circulation (tracked by a “Swirl Clock”) via Cauchy’s integral theorem and Kelvin’s invariant; show that additivity at Y-junctions yields composite baryon tubes with \(\Gamma_{\text{baryon}}=3\kappa\), setting a local time-dilation scale; and provide a sharp experimental discriminator from the swirl–EM correspondence, where topology changes generate geometry-independent, quantized electromotive impulses of fixed magnitude \(\Delta\Phi=\pm\Phi_\star\) (sign set by chirality). At galactic scales, a saturated swirl “tail” contributes a radius-independent term to \(v_\phi(r)\), offering a falsifiable analogue of flattened rotation curves within this model.

\textbf{Relationship to prior work and presentation.} This manuscript builds on a formal framework (SST) developed in public technical reports available on Zenodo. The full derivations and symbolic mappings are included in referenced appendices, but the current manuscript is self-contained, testable, and focused on experimental predictions. All persistent DOIs and the code/data archive are cited in the “Code and Data Availability” section.

\textbf{Why \textit{Foundations of Physics}.} The paper addresses foundational questions about the origin of long-range attraction, the role of topology versus curvature, and the interface between information-theoretic and fluid descriptions—each paired with explicit failure modes—aligning with the journal’s mission.

\textbf{Compliance.} The manuscript is original, not under consideration elsewhere, and all authorship/affiliation information is complete. Conflicts of interest: none. Data and code are available via the cited Zenodo record(s).

Thank you for your consideration.

\closing{Sincerely,}

\end{letter}
\end{document}
