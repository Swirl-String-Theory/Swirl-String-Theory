%! Author = Omar Iskandarani
%! Date = 1/27/2026
%! Affiliation = Independent Researcher, Groningen, The Netherlands
%! License = © 2025 Omar Iskandarani. All rights reserved. This manuscript is made available for academic reading and citation only. No republication, redistribution, or derivative works are permitted without explicit written permission from the author. Contact: info@omariskandarani.com
%! ORCID = 0009-0006-1686-3961
%! DOI = 10.5281/zenodo.xxx

\newcommand{\paperdoi}{10.5281/zenodo.18388716}
\newcommand{\papertitle}{Swirl Holographic Principle (SHP)}

%=========================================
% % PREAMBLE, PACKAGES AND DOCUMENT CONFIGURATION
%=========================================
\documentclass[11pt]{article}
\usepackage{amsmath,amssymb,amsfonts,bm}
\usepackage{siunitx}
\usepackage[hidelinks]{hyperref}
\usepackage[a4paper,margin=1in]{geometry}
\usepackage[T1]{fontenc}
\usepackage[utf8]{inputenc}

% swirl arrows (context-aware)
\newcommand{\swirlarrow}{\mkern-2mu\scriptscriptstyle\boldsymbol{\circlearrowleft}}
\newcommand{\vswirl}{\mathbf{v}_{\mkern-2mu\scriptscriptstyle\boldsymbol{\circlearrowleft}}}
\newcommand{\SwirlClock}{S_{(t)}^{\mkern-2mu\scriptscriptstyle\boldsymbol{\circlearrowleft}}}
\newcommand{\Fmaxswirl}{F^{\max}_{\mkern-1mu\scriptscriptstyle\boldsymbol{\circlearrowleft}}}
\newcommand{\Fmax}{F^{\max}_{\mkern-1mu\scriptscriptstyle\boldsymbol{\circlearrowleft}}} 
\newcommand{\FmaxEM}{F^{\max}_{\mathrm{EM}}}
\newcommand{\FmaxG}{F_{\mathrm{G}}^{\max}}               % G-like maximal force scale
\newcommand{\vscore}{v_{\swirlarrow}}                    % shorthand: |v_swirl| at r=r_c
\newcommand{\vnorm}{\lVert \mathbf{v}_{\mkern-2mu\scriptscriptstyle\boldsymbol{\circlearrowleft}} \rVert}  % swirl speed magnitude
\newcommand{\rhoF}{\rho_{\!f}}\newcommand{\rhof}{\rho_{\!f}}     % effective fluid density
\newcommand{\rhoE}{\rho_{\!E}}\newcommand{\rhoe}{\rho_{\!E}}                           % swirl energy density
\newcommand{\rhoM}{\rho_{\!m}}\newcommand{\rhom}{\rho_{\!m}}                           % mass-equivalent density
\newcommand{\omegas}{\boldsymbol{\omega}_{\swirlarrow}}  % swirl vorticity
\newcommand{\Om}{\Omega_{\swirlarrow}}                   % swirl angular frequency profile
\newcommand{\rc}{r_c}                                    % string core radius (swirl string radius)


\newcommand{\titlepageOpen}{
    \begin{titlepage}
        \thispagestyle{empty}  \centering
        \Large \bfseries \papertitle \par \vspace{1cm}
        {\Large \itshape \textbf{Omar Iskandarani}\textsuperscript{\textbf{*}} \par} \vspace{0.5cm}
        {\today \par}  \vspace{0.5cm}
}

\newcommand{\titlepageClose}{
        \vfill \raggedright \null
        \begin{picture}(0,0)
            \put(0,-45){  % Shift 200pt left, 40pt down
                \begin{minipage}[b]{0.7\textwidth} \footnotesize
                    \renewcommand{\arraystretch}{1.0} \noindent\rule{\textwidth}{0.4pt} \\[0.5em]
                    \textsuperscript{\textbf{*}} Independent Researcher, Groningen, The Netherlands \\
                    Email: \texttt{info@omariskandarani.com} \\
                    ORCID: \texttt{\href{https://orcid.org/0009-0006-1686-3961}{0009-0006-1686-3961}} \\
                    DOI: \href{https://doi.org/\paperdoi}{\paperdoi}
                \end{minipage}
            }
        \end{picture}
    \end{titlepage}
}
%=========================================
% Start Document - Title Page
%=========================================
\begin{document}
    \titlepageOpen
        \begin{abstract}
    
        \end{abstract}
    \titlepageClose


%==================== SST macro prelude (minimal) ====================

    \providecommand{\rc}{r_c}
%====================================================================

    \section{Foundational Postulate: Swirl Holographic Principle (SHP)}

        Let $V\subset\mathbb{R}^3$ be a bounded domain with smooth boundary $\partial V$.
        Let $\vswirl(t,\mathbf{x})$ be a divergence-free swirl field in $V$ evolving in the incompressible inviscid class
        (without reconnection/dissipation in the interval of interest).
        Define the tangential boundary trace $\mathbf{v}_\tau=\vswirl-(\vswirl\cdot\mathbf{n})\mathbf{n}$ on $\partial V$,
        equivalently boundary loop circulations $\Gamma_C(t)=\oint_C \vswirl\cdot d\boldsymbol{\ell}$ for loops $C\subset\partial V$.
        Let $\mathcal{T}$ denote the conserved discrete sector labels (helicity/linking class; plus harmonic-mode labels for multiply connected domains).

        \paragraph{Postulate (SHP).}
            For each time $t$, the bulk state in $V$ is fully specified by boundary swirl data and sector labels:
            \begin{equation}
                \big(\vswirl(t,\cdot)\ \text{in}\ V\big)\ \equiv\
                \mathcal{R}\!\left[\mathbf{v}_\tau(t,\cdot)\big|_{\partial V},\ \mathcal{T}\right],
            \end{equation}
            for a well-defined reconstruction map $\mathcal{R}$ (Green/Hodge reconstruction).
            Thus there are no independent bulk degrees of freedom beyond those encoded on $\partial V$ and $\mathcal{T}$.

    \section{Specialization I: General Relativity as Constraint Holography}

    In $3+1$ (ADM) form, GR uses $(h_{ij},\pi^{ij})$ as canonical data on a hypersurface $\Sigma$.
    Bulk admissibility is restricted by the Hamiltonian and momentum constraints,
    \begin{equation}
        \mathcal{H}(h,\pi)=0,\qquad \mathcal{H}_i(h,\pi)=0,
    \end{equation}
    so consistent boundary data (e.g. induced boundary metric $\gamma_{ab}$ and/or extrinsic curvature data) plus global sector labels
    (topology and conserved charges) determine the admissible interior by solving the constraint system.
    This is the GR analogue of SHP: boundary data + constraints + sector labels $\Rightarrow$ bulk.

    \section{Specialization II: AdS/CFT as On-shell Boundary Functional Reconstruction}

    In AdS/CFT, boundary values of bulk fields act as sources $J$ for boundary operators $\mathcal{O}$.
    The on-shell bulk functional depends on boundary sources,
    \begin{equation}
        Z_{\mathrm{bulk}}[J]\ \approx\ \exp\!\left(-S_{\mathrm{on\text{-}shell}}[J]\right),
    \end{equation}
    and boundary correlators follow from functional differentiation,
    \begin{equation}
        \langle \mathcal{O}(x)\rangle_J = \frac{\delta S_{\mathrm{on\text{-}shell}}[J]}{\delta J(x)}.
    \end{equation}
    Given $J$ (boundary data), the bulk solution is reconstructed by solving bulk equations of motion with $\phi|_{\partial}=J$,
    i.e. a reconstruction map $\mathcal{R}$ in the AdS variable set. This is the functional-quantum specialization of SHP.

    %==============================================================================
% SST-63: Swirl Holographic Principle (Sections 1--3)
%==============================================================================

    \section{Introduction: Why Holography Is Inevitable in SST}

    Holographic behavior---the determination of bulk physics by boundary data---is
    often presented as a surprising or uniquely quantum-gravitational feature.
    In Swirl--String Theory (SST), by contrast, holography is a direct and unavoidable
    consequence of classical mechanics under three assumptions: (i) incompressibility,
    (ii) inviscid evolution, and (iii) conservation of topological sector.
    These assumptions eliminate independent bulk degrees of freedom and enforce
    global constraints that bind the interior dynamics to boundary data.

    In incompressible inviscid media, vorticity is materially advected and circulation
    is conserved (Kelvin--Helmholtz theorems). As a result, admissible bulk configurations
    are those consistent with boundary circulation and conserved topological labels
    (linking, twist, and writhe). SST adopts these mechanical facts as fundamental.
    The central claim of this work is therefore not that holography exists, but that
    it is \emph{inevitable} once the above constraints are imposed.

    This paper formalizes that claim as a foundational postulate---the
    \emph{Swirl Holographic Principle} (SHP)---and provides a theorem-level statement
    showing that the bulk SST state is uniquely determined by boundary swirl data
    together with discrete sector labels. We then show that General Relativity (GR)
    and AdS/CFT arise as specializations of the same boundary-determination logic,
    with different choices of dynamical variables and constraints.

    \section{Foundational Postulate: Swirl Holographic Principle (SHP)}

    Let $V\subset\mathbb{R}^3$ be a bounded domain with smooth boundary $\partial V$.
    Let the SST bulk state be represented by a divergence-free swirl field
    $\mathbf{v}_{\!\boldsymbol{\circlearrowleft}}(t,\mathbf{x})$ evolving in the
    incompressible, inviscid class,
    \begin{align}
        \partial_t \mathbf{v}_{\!\boldsymbol{\circlearrowleft}}
        + (\mathbf{v}_{\!\boldsymbol{\circlearrowleft}}\!\cdot\nabla)
        \mathbf{v}_{\!\boldsymbol{\circlearrowleft}}
        &= -\nabla p,
        \\
        \nabla\cdot\mathbf{v}_{\!\boldsymbol{\circlearrowleft}} &= 0,
    \end{align}
    with impermeable boundary condition
    $\mathbf{v}_{\!\boldsymbol{\circlearrowleft}}\!\cdot\mathbf{n}|_{\partial V}=0$.
    Define the tangential boundary trace
    \begin{equation}
        \mathbf{v}_\tau :=
        \mathbf{v}_{\!\boldsymbol{\circlearrowleft}}
        - (\mathbf{v}_{\!\boldsymbol{\circlearrowleft}}\!\cdot\mathbf{n})\mathbf{n}
        \quad \text{on } \partial V,
    \end{equation}
    equivalently the boundary circulation functional
    $\Gamma_C(t)=\oint_C \mathbf{v}_{\!\boldsymbol{\circlearrowleft}}\cdot
    d\boldsymbol{\ell}$ for all smooth loops $C\subset\partial V$.

    Let $\mathcal{T}$ denote the discrete topological sector label of the flow,
    consisting of helicity class and linking/twist/writhe invariants, assumed
    conserved (no reconnection or dissipation in the interval of interest).

    \paragraph{Postulate (SHP).}
        For each time $t$, the bulk SST state in $V$ is fully specified by the pair
        \begin{equation}
            \big(\mathbf{v}_\tau(t,\cdot)\big|_{\partial V},\;\mathcal{T}\big).
        \end{equation}
        Equivalently, there exists a reconstruction map $\mathcal{R}$ such that
        \begin{equation}
            \mathbf{v}_{\!\boldsymbol{\circlearrowleft}}(t,\cdot)
            \;=\;
            \mathcal{R}\!\left[
                             \mathbf{v}_\tau(t,\cdot)\big|_{\partial V},\;
                             \mathcal{T}
            \right],
        \end{equation}
        up to a pressure gauge. No independent bulk degrees of freedom exist beyond
        those encoded on the boundary and in the discrete sector labels.

    \paragraph{Physical meaning.}
        Boundary swirl data fix all admissible vorticity fluxes through interior surfaces,
        while $\mathcal{T}$ selects the unique topological realization consistent with
        those fluxes. The interior is therefore constrained, not autonomous.

    \section{Boundary Determination Theorem (Euler/Hodge Form)}

    We now state the precise theorem underlying SHP in the classical Euler regime.

    \paragraph{Theorem (Boundary Determination of Bulk Swirl State).}
        Let $V\subset\mathbb{R}^3$ be smooth and simply connected.
        Assume $\mathbf{v}_{\!\boldsymbol{\circlearrowleft}}(t,\cdot)\in H^s(V)$ with
        $s>5/2$, impermeable boundary conditions, and inviscid evolution without
        reconnection so that $\mathcal{T}$ is conserved.
        Then for each $t$ in the local existence interval, the solenoidal bulk field
        $\mathbf{v}_{\!\boldsymbol{\circlearrowleft}}(t,\cdot)$ is uniquely determined
        by $\mathbf{v}_\tau(t,\cdot)|_{\partial V}$ and $\mathcal{T}$, up to pressure gauge.

    \paragraph{Sketch of proof.}
        For divergence-free vector fields with $\mathbf{v}\cdot\mathbf{n}=0$, Hodge
        decomposition yields
        \begin{equation}
            \mathbf{v}_{\!\boldsymbol{\circlearrowleft}}
            = \nabla\times\mathbf{A} + \mathbf{h},
        \end{equation}
        where $\mathbf{h}$ is harmonic (curl-free and divergence-free).
        In simply connected $V$, $\mathbf{h}=0$; otherwise, $\mathbf{h}$ is finite-dimensional
        and fixed by circulation around nontrivial cycles, which are included in
        $\mathcal{T}$.

        The vector potential satisfies a Poisson problem
        $-\Delta\mathbf{A}=\boldsymbol{\omega}$ with gauge and boundary conditions,
        so that $\mathbf{v}_{\!\boldsymbol{\circlearrowleft}}=\nabla\times\mathbf{A}$
        is uniquely determined by the vorticity field $\boldsymbol{\omega}$.

        By Stokes' theorem, boundary circulations determine vorticity fluxes through
        all interior surfaces anchored on $\partial V$,
        \begin{equation}
            \oint_{\partial S}\mathbf{v}_{\!\boldsymbol{\circlearrowleft}}\cdot
            d\boldsymbol{\ell}
            =
            \int_S \boldsymbol{\omega}\cdot d\mathbf{S}.
        \end{equation}
        Conservation of the topological sector $\mathcal{T}$ restricts admissible
        interior rearrangements of vorticity. Together, these conditions uniquely fix
        $\boldsymbol{\omega}$ and hence the bulk field
        $\mathbf{v}_{\!\boldsymbol{\circlearrowleft}}$.

    \paragraph{Consequence.}
        The Euler equations act as constraint equations: the bulk is the unique
        configuration compatible with boundary swirl data and conserved topology.
        This completes the proof sketch and establishes SHP at theorem level.

        %==============================================================================
    \section{Specialization I: General Relativity as Constraint Holography}
%==============================================================================

    We now show that General Relativity (GR) arises as a direct specialization of the
    Swirl Holographic Principle when the dynamical variables are changed from a swirl
    velocity field to a spacetime metric, and the mechanical constraints are replaced
    by geometric ones. No additional holographic assumptions are required.

    \subsection{ADM formulation and boundary determination}

        In the $3+1$ (ADM) formulation of GR, spacetime is foliated by spacelike
        hypersurfaces $\Sigma_t$ with induced metric $h_{ij}$ and conjugate momentum
        $\pi^{ij}$, related to the extrinsic curvature $K_{ij}$. The Einstein equations
        split into evolution equations and constraint equations:
        \begin{align}
            \mathcal{H}(h,\pi) &= 0 \qquad \text{(Hamiltonian constraint)}, \\
            \mathcal{H}_i(h,\pi) &= 0 \qquad \text{(momentum constraints)}.
        \end{align}

        These constraints eliminate independent bulk degrees of freedom. Admissible bulk
        geometries are precisely those compatible with:
        \begin{itemize}
            \item boundary data (induced boundary metric and/or extrinsic curvature),
            \item global sector labels (topology and conserved charges),
            \item the constraint equations.
        \end{itemize}

        This is mathematically equivalent to the SST statement that incompressibility and
        Kelvin--Helmholtz invariants restrict admissible interior swirl configurations.

    \subsection{Exact structural correspondence with SHP}

        The correspondence between SST and GR is summarized as follows:

        \begin{center}
            \begin{tabular}{lll}
                \toprule
                \textbf{SST} & & \textbf{GR (ADM)} \\
                \midrule
                $\nabla\cdot \mathbf{v}_{\!\boldsymbol{\circlearrowleft}} = 0$
                & $\longleftrightarrow$ &
                $\nabla_\mu T^{\mu\nu} = 0$ \\[6pt]

                Boundary tangential swirl $\mathbf{v}_\tau$
                & $\longleftrightarrow$ &
                Boundary induced metric $\gamma_{ab}$ \\[6pt]

                Kelvin circulation invariants
                & $\longleftrightarrow$ &
                Momentum constraints $\mathcal{H}_i=0$ \\[6pt]

                Topological sector $\mathcal{T}$
                & $\longleftrightarrow$ &
                Topology + ADM charges \\[6pt]

                Reconstruction map $\mathcal{R}$
                & $\longleftrightarrow$ &
                Solution of Einstein constraints \\[6pt]
                \bottomrule
            \end{tabular}
        \end{center}

        In both theories, the bulk is \emph{not free}. It is fixed by boundary data and
        constraints.

    \subsection{Flat spacetime as a special SST limit}

        In GR, flat spacetime corresponds to:
        \[
            R_{\mu\nu\rho\sigma} = 0,
            \qquad
            K_{ij} = 0,
            \qquad
            \pi^{ij} = 0,
        \]
        up to coordinate transformations.

        In SST, the corresponding limit is:
        \[
            \nabla\cdot \mathbf{v}_{\!\boldsymbol{\circlearrowleft}} = 0,
            \qquad
            \nabla\times \mathbf{v}_{\!\boldsymbol{\circlearrowleft}} = 0,
            \qquad
            \mathbf{v}_{\!\boldsymbol{\circlearrowleft}} = \text{constant}.
        \]

        Thus, \emph{flat spacetime is not characterized by incompressibility alone}, but
        by the absence of vorticity and swirl gradients. This mirrors the GR statement
        that conservation laws do not imply flatness; curvature vanishes only when
        additional geometric invariants vanish.

    \subsection{Gravity as constrained interior structure}

        In GR, gravitational attraction and time dilation arise from curvature induced by
        stress--energy, subject to $\nabla_\mu T^{\mu\nu}=0$.

        In SST, gravitational effects arise from vorticity and pressure gradients in an
        otherwise divergence-free medium:
        \begin{equation}
            \nabla p = \rho_{\!f}\,\mathbf{v}_{\!\boldsymbol{\circlearrowleft}}
            \times \boldsymbol{\omega},
            \qquad
            \boldsymbol{\omega} = \nabla\times\mathbf{v}_{\!\boldsymbol{\circlearrowleft}}.
        \end{equation}

        The key point is structural: in both theories, gravity is \emph{not} produced by
        sources that violate conservation, but by constrained internal structure of the
        field itself. The role played by curvature tensors in GR is played by vorticity
        and swirl-induced pressure gradients in SST.

    \subsection{Interpretation}

        General Relativity is therefore a geometric specialization of the Swirl
        Holographic Principle:

        \begin{quote}
            \emph{Boundary data, global sector labels, and constraint equations uniquely
            determine the admissible bulk.}
        \end{quote}

        SST provides a mechanical realization of this logic, while GR provides a geometric
        one. The holographic character of GR is thus not fundamental but inherited from a
        more general constraint-holographic structure shared by both theories.

        %==============================================================================
    \section{Specialization II: AdS/CFT as On-Shell Boundary Reconstruction}
%==============================================================================

    We now show that the AdS/CFT correspondence arises as a quantum-field-theoretic
    specialization of the Swirl Holographic Principle. In this setting, the boundary
    data are not mechanical swirl variables but field-theoretic sources, and the
    reconstruction map is implemented by solving bulk equations of motion with fixed
    boundary conditions and evaluating the on-shell action.

    \subsection{Boundary sources and bulk reconstruction}

        In the AdS/CFT framework, bulk fields $\Phi$ (including the metric and matter
        fields) are defined on an asymptotically Anti--de Sitter spacetime. Their boundary
        values $\Phi|_{\partial}$ act as sources $J$ for operators $\mathcal{O}$ in a
        lower-dimensional conformal field theory (CFT) living on the boundary.

        The defining relation is the Gubser--Klebanov--Polyakov--Witten (GKPW) prescription:
        \begin{equation}
            Z_{\mathrm{bulk}}[J]
            \;\approx\;
            \exp\!\left(-S_{\mathrm{bulk}}^{\mathrm{on\text{-}shell}}[J]\right),
        \end{equation}
        where $S_{\mathrm{bulk}}^{\mathrm{on\text{-}shell}}[J]$ is the bulk action evaluated
        on the unique solution of the bulk equations of motion satisfying
        $\Phi|_{\partial}=J$.

        Thus, given boundary data $J$, the admissible bulk configuration is obtained by a
        reconstruction map:
        \begin{equation}
            \Phi_{\mathrm{bulk}}
            \;=\;
            \mathcal{R}_{\mathrm{AdS}}[J],
        \end{equation}
        where $\mathcal{R}_{\mathrm{AdS}}$ denotes solution of the bulk field equations
        subject to boundary conditions.

    \subsection{Exact correspondence with SHP}

        The structural equivalence with SHP is immediate:

        \begin{center}
            \begin{tabular}{lll}
                \toprule
                \textbf{SHP (SST)} & & \textbf{AdS/CFT} \\
                \midrule
                Boundary swirl data $\mathbf{v}_\tau$ & $\longleftrightarrow$ & Boundary sources $J$ \\[6pt]

                Topological sector $\mathcal{T}$ & $\longleftrightarrow$ & Choice of CFT vacuum / sector \\[6pt]

                Reconstruction map $\mathcal{R}$ & $\longleftrightarrow$ & Bulk EOM + boundary conditions \\[6pt]

                Bulk observables & $\longleftrightarrow$ & Functional derivatives of $S_{\text{on-shell}}$ \\[6pt]
                \bottomrule
            \end{tabular}
        \end{center}

        In both theories, the bulk configuration is not independently specifiable: it is
        the unique solution compatible with boundary data and global sector labels.

    \subsection{Why AdS geometry is required}

        In AdS/CFT, the negative curvature of AdS spacetime ensures that bulk fields are
        elliptically controlled by boundary data. This mathematical property guarantees
        well-posedness of the boundary value problem and uniqueness of the reconstructed
        bulk.

        In SST, no background curvature is required because incompressibility and vorticity
        conservation already enforce elliptic constraints on admissible bulk states. The
        role played by AdS geometry in quantum gravity is thus played by mechanical
        constraints in SST.

    \subsection{On-shell action as boundary functional}

        A central feature of AdS/CFT is that all physical information about the bulk is
        contained in a boundary functional:
        \begin{equation}
            S_{\mathrm{bulk}}^{\mathrm{on\text{-}shell}}
            =
            S_{\mathrm{bulk}}^{\mathrm{on\text{-}shell}}[J].
        \end{equation}
        Expectation values of boundary operators follow from functional differentiation:
        \begin{equation}
            \langle \mathcal{O}(x) \rangle_J
            =
            \frac{\delta S_{\mathrm{bulk}}^{\mathrm{on\text{-}shell}}[J]}{\delta J(x)}.
        \end{equation}

        This is the quantum-field analogue of SHP: the bulk action itself is a functional of
        boundary data, and all bulk observables are boundary-generated.

    \subsection{Interpretation}

        AdS/CFT therefore represents a functional, quantum realization of constraint
        holography. The Swirl Holographic Principle captures the same logic at the
        mechanical level:

        \begin{quote}
            \emph{Boundary data and global sector labels uniquely determine the admissible
            bulk state.}
        \end{quote}

        SST does not reproduce AdS/CFT dynamically, nor does it require quantum gravity.
        Rather, AdS/CFT appears as a special case of the same boundary-determination
        principle applied to quantum fields in a curved background. In this sense,
        holography is not a uniquely quantum phenomenon but a general consequence of
        constraint-dominated dynamics.

        %==============================================================================
    \section{Clock Holography, Falsifiers, and Minimal Experiments}
%==============================================================================

    The Swirl Holographic Principle is not a metaphysical claim; it produces concrete,
    falsifiable consequences. In SST, time, gravity, and bulk dynamics inherit their
    structure from boundary swirl data and conserved topology. This section states the
    key predictions and the minimal conditions under which SHP must fail.

    \subsection{Clock holography}

        In SST, the local clock rate is a functional of swirl intensity. A representative
        form used throughout SST is
        \begin{equation}
            dt_{\mathrm{local}}
            =
            dt_{\infty}
            \sqrt{1 - \frac{\lVert \boldsymbol{\omega} \rVert^2}{c^2}},
            \qquad
            \boldsymbol{\omega}=\nabla\times\mathbf{v}_{\!\boldsymbol{\circlearrowleft}}.
        \end{equation}

        By SHP, the bulk vorticity field $\boldsymbol{\omega}(t,\cdot)$ is uniquely
        determined by boundary swirl data and the conserved sector $\mathcal{T}$.
        Consequently, the bulk clock field is also boundary-determined.

        \paragraph{Clock holography (statement).}
            Given fixed $\mathcal{T}$, the entire bulk distribution of clock rates in $V$ is a
            functional of boundary swirl data:
            \begin{equation}
                dt_{\mathrm{local}}(t,\mathbf{x})
                =
                \mathcal{C}\!\left[
                                 \mathbf{v}_\tau(t,\cdot)\big|_{\partial V}
                \right].
            \end{equation}

            This result has no direct analogue in GR or AdS/CFT: in SST, time dilation is not
            merely influenced by boundary conditions but \emph{encoded} by them.

    \subsection{Topology as protected information}

    The discrete sector labels $\mathcal{T}$ play the role of protected information.
    Helicity and linking numbers cannot change under ideal evolution and therefore act
    as robust memory degrees of freedom.

    \paragraph{Prediction.}
        Two systems driven with identical boundary swirl data but prepared in different
        topological sectors $\mathcal{T}_1\neq\mathcal{T}_2$ will exhibit distinct bulk
        responses (clock-rate distributions, pressure fields, mode spectra), despite
        indistinguishable boundary forcing.

        This is a sharp discriminator between SST and purely metric theories, where
        topology without curvature has no dynamical effect.

    \subsection{Controlled breakdown of holography}

    SHP is exact only under ideal conditions. Its breakdown is itself predictive.

    \paragraph{Breakdown mechanisms.}
        \begin{enumerate}
            \item \textbf{Reconnection}: topological sector $\mathcal{T}$ changes.
            \item \textbf{Dissipation}: viscosity destroys Kelvin--Helmholtz invariants.
            \item \textbf{Compressibility}: $\nabla\cdot\mathbf{v}_{\!\boldsymbol{\circlearrowleft}}\neq 0$.
        \end{enumerate}

    \paragraph{Observable signature.}
        When SHP fails, boundary data no longer uniquely predict the interior. The failure
        manifests as:
        \begin{itemize}
            \item loss of reproducibility under identical boundary driving,
            \item broadband excitation and energy leakage,
            \item breakdown of clock holography (clock-rate noise not traceable to boundary data).
        \end{itemize}

        This gives SST a well-defined regime of validity rather than unfalsifiable scope.

    \subsection{Minimal experimental tests}

    The following tabletop-scale experiments are sufficient to probe SHP qualitatively,
    without invoking exotic physics.

    \paragraph{Test 1: Boundary-only control.}
        Prepare a closed toroidal or cavity system with minimal interior access. Modulate
        boundary swirl phase and circulation while holding geometry fixed. SST predicts
        measurable changes in interior resonant structure or clock proxies, despite no
        direct interior forcing.

    \paragraph{Test 2: Sector discrimination.}
        Prepare two configurations with identical boundary driving but different linking
        or knot classes. SST predicts distinct interior responses; GR and standard EM do not.

    \paragraph{Test 3: Breakdown signature.}
        Increase driving until reconnection-like events occur. SST predicts a sharp
        transition from boundary-determined to boundary-indeterminate interior behavior,
        accompanied by dissipative signatures.

    \subsection{Summary of falsifiable content}

    The Swirl Holographic Principle predicts:
    \begin{itemize}
        \item boundary determination of bulk dynamics in ideal regimes,
        \item boundary encoding of clock-rate fields,
        \item topologically protected bulk information,
        \item controlled and observable breakdown under reconnection or dissipation.
    \end{itemize}

    Any observation of sustained bulk degrees of freedom independent of boundary data
    and sector labels in an incompressible inviscid regime would falsify SHP and, by
    extension, SST in its present form.



    \begin{thebibliography}{99}

        \bibitem{ArnoldKhesin1998}
        V.~I.~Arnold and B.~A.~Khesin (1998),
        \textit{Topological Methods in Hydrodynamics},
        Springer,
        doi:10.1007/978-1-4612-0639-0.

        \bibitem{MajdaBertozzi2002}
        A.~J.~Majda and A.~L.~Bertozzi (2002),
        \textit{Vorticity and Incompressible Flow},
        Cambridge University Press,
        doi:10.1017/CBO9780511613204.

        \bibitem{ArnowittDeserMisner1962}
        R.~Arnowitt, S.~Deser, and C.~W.~Misner (1962),
        \textit{The Dynamics of General Relativity},
        in \textit{Gravitation: An Introduction to Current Research}, Wiley.
        (permalink: https://arxiv.org/abs/gr-qc/0405109)

        \bibitem{GibbonsHawking1977}
        G.~W.~Gibbons and S.~W.~Hawking (1977),
        \textit{Action Integrals and Partition Functions in Quantum Gravity},
        Phys. Rev. D \textbf{15}, 2752,
        doi:10.1103/PhysRevD.15.2752.

        \bibitem{York1972}
        J.~W.~York (1972),
        \textit{Role of Conformal Three-Geometry in the Dynamics of Gravitation},
        Phys. Rev. Lett. \textbf{28}, 1082,
        doi:10.1103/PhysRevLett.28.1082.

        \bibitem{Maldacena1998}
        J.~M.~Maldacena (1998),
        \textit{The Large N Limit of Superconformal Field Theories and Supergravity},
        Adv. Theor. Math. Phys. \textbf{2}, 231,
        doi:10.1023/A:1026654312961; arXiv:hep-th/9711200.

        \bibitem{GubserKlebanovPolyakov1998}
        S.~S.~Gubser, I.~R.~Klebanov, and A.~M.~Polyakov (1998),
        \textit{Gauge Theory Correlators from Non-Critical String Theory},
        Phys. Lett. B \textbf{428}, 105,
        doi:10.1016/S0370-2693(98)00377-3; arXiv:hep-th/9802109.

        \bibitem{Witten1998}
        E.~Witten (1998),
        \textit{Anti De Sitter Space and Holography},
        Adv. Theor. Math. Phys. \textbf{2}, 253,
        doi:10.4310/ATMP.1998.v2.n2.a2; arXiv:hep-th/9802150.

    \end{thebibliography}




\end{document}