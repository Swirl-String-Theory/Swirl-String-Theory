\documentclass[a4paper,10pt]{letter}

\usepackage[T1]{fontenc}
\usepackage[utf8]{inputenc}
\usepackage{lmodern}
\usepackage[hidelinks]{hyperref}
\usepackage{microtype}
\usepackage[margin=1in]{geometry}

% Sender info
\signature{Omar Iskandarani\\
Independent Researcher, Groningen,\\ The Netherlands\\
ORCID: 0009-0006-1686-3961\\
Email: \href{mailto:info@omariskandarani.com}{info@omariskandarani.com}}

\address{Omar Iskandarani\\
Vinkenstraat 86A\\
9713 TK Groningen\\
The Netherlands}

\date{\today}

\begin{document}

    \begin{letter}{Editors\\
    \textit{Chaos: An Interdisciplinary Journal of Nonlinear Science}}

        \opening{Dear Editors,}

        I am pleased to submit the manuscript \textit{``Delay-induced mode discreteness in nonlinear ring systems''} for consideration as a Research Article in \textit{Chaos: An Interdisciplinary Journal of Nonlinear Science}.

        \textbf{Summary.}
        The manuscript demonstrates that discrete, mode-like structures in circulating nonlinear systems with long delay can arise purely from deterministic pattern-forming dynamics. Using a minimal scalar delay-differential equation, we show that modulational instability organizes the circulating field into plateau-like domains separated by drifting fronts. These multistable square-wave states behave as discrete circulation modes, even though the underlying model is continuous and low-dimensional. The delay time acts as a pseudo-spatial coordinate, so that the dynamics reduce to a spatio-temporal pattern-formation problem in the long-delay limit.

        \textbf{What is new.}
        \begin{itemize}\setlength\itemsep{0.3em}
            \item \emph{Classical origin of mode discreteness:} Discrete circulation modes emerge from delay-induced modulational instability without microscopic quantization assumptions.
            \item \emph{Unified spatio-temporal picture:} Long-delay dynamics, front propagation, and pattern selection are combined into a single physical framework.
            \item \emph{Multistability and snaking:} Square-wave families organize into collapsed-snaking structures controlled by Bykov $T$-points.
            \item \emph{Topological persistence:} Cantori and turnstiles provide a dynamical explanation for the robustness of circulating states.
        \end{itemize}

        \textbf{Relevance to \textit{Chaos}.}
        The work directly addresses core topics of the journal—nonlinear dynamics, delay-differential equations, pattern formation, and bifurcation theory—and applies them to physical systems such as optical cavities, electronic delay loops, and microwave resonators.

        \textbf{Scope and transparency.}
        The manuscript clearly distinguishes schematic illustrations from reproducible numerical results and specifies the long-delay regime $(\tau/\varepsilon \gg 1)$ in which the analysis applies. The submission is original, not under consideration elsewhere, and there are no conflicts of interest or external funding.

        Thank you for considering this manuscript. I would be grateful for the opportunity to have it reviewed in \textit{Chaos}.

        \closing{Sincerely,}

    \end{letter}

\end{document}