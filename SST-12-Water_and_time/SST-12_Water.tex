% !TEX program = pdflatex
\documentclass[12pt]{article}
\usepackage{amsmath,amssymb,amsfonts,bm}
\usepackage{siunitx}
\usepackage[hidelinks]{hyperref}
\usepackage{geometry}
\usepackage[utf8]{inputenc}
\usepackage[T1]{fontenc}
\geometry{margin=1in}

%==================== SST Rosetta Macros (self-contained) ====================
% Canon: Swirl-String-Theory_Canon-v0.5.10; Rosetta: SST-Rosetta-v0.6
% 1:1 translation preserving sections/equations; SST terms carry mainstream glosses inline.

% ---- Core SST symbols ----
\newcommand{\vSwirl}{\mathbf{v}_{\!\boldsymbol{\circlearrowleft}}}   % characteristic swirl speed (SST)
\newcommand{\rc}{r_{c}}                                                 % core radius (SST)
\newcommand{\rhof}{\rho_{\!f}}                                        % effective fluid density
\newcommand{\rhoE}{\rho_{\!E}}                                        % swirl energy density
\newcommand{\rhoM}{\rho_{\!m}}                                        % mass-equivalent density (\rho_E/c^2)
\newcommand{\St}{S_{\!t}}                                              % Swirl Clock factor
\newcommand{\Omegazero}{\Omega_{0}}                                    % vSwirl/rc scale
\newcommand{\Ce}{C_e}                                                   % legacy symbol (Rosetta: maps to |\vSwirl|)

% ---- Mainstream shortcuts (kept from original) ----
\newcommand{\dd}{\mathrm{d}}
\newcommand{\Om}{\Omega}
\newcommand{\bk}{\boldsymbol{k}}
\newcommand{\br}{\boldsymbol{r}}\newcommand{\ez}{\hat{\boldsymbol{z}}}
\newcommand{\er}{\hat{\boldsymbol{r}}}\newcommand{\etheta}{\hat{\boldsymbol{\theta}}}
\newcommand{\bU}{\boldsymbol{u}}\newcommand{\bW}{\boldsymbol{\omega}}\newcommand{\bOm}{\boldsymbol{\Omega}}
\newcommand{\grad}{\boldsymbol{\nabla}}\newcommand{\curl}{\boldsymbol{\nabla}\!\times}\newcommand{\divg}{\boldsymbol{\nabla}\!\cdot}
\newcommand{\p}{\partial}

% ---- Canonical identities (Rosetta) ----
%  \rhoE = 1/2 \, \rhof \, \|\bm v\|^2; \qquad \rhoM = \rhoE/c^2; \qquad \Omegazero = \|\vSwirl\|/\rc.

%=============== Title and metadata =================
\title{Impulsive Axisymmetric Forcing in a Rotating Cylinder,\\
Reversible Swirl Response, and Skyrmionic Photon Emission:\\
Fluid Benchmarks and Fluid-Inspired Kinematic Hypotheses\\[2pt]
\large (SST Canon + Rosetta 1:1 Translation with Mainstream Glosses)}
\author{Omar Iskandarani}
\date{2025}

\begin{document}
    \maketitle

    \begin{abstract}
        We present a one-to-one Swirl--String Theory (SST) translation of a rotating-tank impulse experiment and skyrmionic-photon framework. All sections/equations from the original are preserved, with SST terms added via Rosetta mapping and mainstream physics glosses inline. Macrodynamics remain standard rotating Euler; SST introduces (i) bookkeeping of swirl energy/mass densities and (ii) the canonical \emph{Swirl Clock} $\dd\tau/\dd t=\sqrt{1-v^2/c^2}$ alongside the legacy $\sqrt{1-u^2/\Ce^2}$ hypothesis (kept as an explicit, testable variant). Dimensional checks and numerics use $\|\vSwirl\|=1.09384563\times10^6\,\mathrm{m\,s^{-1}}$, $\rc=1.40897017\times10^{-15}\,\mathrm{m}$, $\rhof=7.0\times10^{-7}\,\mathrm{kg\,m^{-3}}$.
    \end{abstract}

%=================================================================
    \section*{Rosetta card (SST $\leftrightarrow$ mainstream)}
%=================================================================
        \vspace{-6pt}
        \begin{align*}
            \rhoE &= \tfrac{1}{2}\,\rhof\,\|\bm v\|^2 &&\text{(mainstream: kinetic-energy density)},\\
            \rhoM &= \rhoE/c^2 &&\text{(mainstream: mass from energy)},\\
            \Omegazero &\equiv \|\vSwirl\|/\rc &&\text{(SST characteristic frequency)},\\
            \St &= \sqrt{1-\,v^2/c^2} &&\text{(Swirl Clock; mainstream: Lorentz factor)}.
        \end{align*}
        \noindent Distinct roles: $\Omega$ (lab rotation) controls inertial waves; $\Omegazero$ is an SST material scale derived from $\vSwirl$ and $\rc$.

%=================================================================
    \section{Set-up and observation}
%=================================================================
        Consider a vertical cylinder of radius $R$ and depth $H$, rotating at rate $\Omega$ about $z$. A bottom DC motor drives a three-fin impeller that briefly produces a hollow-core vortex of radius $a_v\approx\SI{7.5}{mm}$ and axial jetting. After a delay $\sim H/c_{g,z}$, a two-lobed "push--pull" on-axis surface signal appears at $z=H$.

        \paragraph{SST map.} Energy in the packet that reaches the surface is tracked as $\rhoE=\tfrac{1}{2}\,\rhof\,\|\bU'\|^2$ and mass-equivalent density $\rhoM=\rhoE/c^2$ (bookkeeping only).

%=================================================================
    \section{Linear rotating-wave framework}
%=================================================================
    In the bulk, small perturbations admit \emph{inertial waves} with dispersion\cite{Greenspan1968,Batchelor1967,Vallis2017}
    \begin{equation}
        \omega \,=\, 2\Omega\,\frac{k_z}{k},\qquad k=\sqrt{k_r^2+k_z^2}. \label{eq:disp}
    \end{equation}
    For axisymmetry ($m=0$) in a cylinder, $k_r\approx\lambda_{0n}/R$ (Bessel eigenvalues)\cite{Greenspan1968}. Group velocities follow from $\partial\omega/\partial k_i$:
    \begin{equation}
        c_{g,r}= -\,\frac{2\Omega\,k_z k_r}{k^3},\qquad
        c_{g,z}= \frac{2\Omega\,k_r^2}{k^3}. \label{eq:cg}
    \end{equation}
    Beams satisfy $\tan\alpha=|c_{g,r}|/c_{g,z}=k_z/k_r$.

    \paragraph{Arrival time.} With $c_{g,z}$ as above,
        \begin{equation}
            t_{\mathrm{arr}} \approx \frac{H}{c_{g,z}} = \frac{H\,k^3}{2\Omega\,k_r^2}. \label{eq:tarr}
        \end{equation}
        \textit{Numerics (bench-top).} $R=\SI{0.075}{m}$, $H=\SI{0.30}{m}$, $\Omega\approx\SI{2}{rad/s}$, $k_r\approx3.83/R\approx\SI{51}{m^{-1}}$, $k_z\approx\pi/H\approx\SI{10.5}{m^{-1}}$. Then $k\approx\SI{52.1}{m^{-1}}$, $c_{g,z}\approx\SI{7.4e-2}{m/s}$, $t_{\mathrm{arr}}\approx\SI{4.1}{s}$. Units: $[H/c_{g,z}]=\mathrm{s}$.

    \paragraph{Analogy (10-year-old).} A tilted line of dominos falls in a slanted path; it reaches the top after a short delay.

%=================================================================
    \section{Impulse sign and the observed "push--pull"}
%=================================================================
    Start impulse: axial upwelling and centrifugal low $p'$ in the hollow core yield a \emph{surface depression} on arrival. Stop impulse reverses the sign and gives a \emph{surface rise}. Hydrostatic surface coupling\cite{Batchelor1967,Vallis2017}:
    \begin{equation}
        p'(z=0)+\rho g\,\eta=0\;\Rightarrow\;\eta=-\,\frac{p'(0)}{\rho g}. \label{eq:hydro}
    \end{equation}

%=================================================================
    \section{Relation to vortex rings and jet starting vortices}
%=================================================================
    A short burst sheds a starting ring with translational speed in quiescent fluid\cite{Saffman1992}
    \begin{equation}
        U_{\mathrm{ring}}\approx\frac{\Gamma}{4\pi R_v}\Big[\ln\!\Big(\frac{8 R_v}{a_v}\Big)-\tfrac{1}{4}\Big]. \label{eq:ring}
    \end{equation}
    In rotation, the on-axis \emph{delayed} packet at the surface is dominated by the inertial-wave field, not by ring ballistic motion.

%=================================================================
    \section{What this \emph{is} and \emph{is not} an analogy to}
%=================================================================
    \subsection*{Photon analogy (limited)}
        Localized impulse $\Rightarrow$ localized packet with clear parity. Inertial waves are anisotropic and dispersive (\eqref{eq:disp}--\eqref{eq:cg}); free photons in vacuum are isotropic and nondispersive. Analogy is qualitative.

    \subsection*{Gravitational-wave analogy (limited)}
        GR waves are transverse, quadrupolar, nondispersive (in vacuum). Axisymmetric inertial-wave response is neither quadrupolar nor nondispersive; $c_g$ depends on $(k_r,k_z)$ and $\Omega$.

%=================================================================
    \section{A proposed microscopic interpretation inspired by fluid analogs (VAM)}
%=================================================================
    \textbf{Legacy rule (kept as testable hypothesis).} Let $u(\br,t)$ be local swirl speed. Postulate
    \begin{equation}
        \frac{\dd \tau}{\dd t} = \sqrt{1-\frac{u^2}{\Ce^2}}, \label{eq:legacy}
    \end{equation}
    with $\Ce$ a characteristic speed. (Rosetta: $\Ce\mapsto\|\vSwirl\|$.)

    \textbf{Canonical SST (Swirl Clock).} SST uses
    \begin{equation}
        \frac{\dd \tau}{\dd t} = \St = \sqrt{1-\frac{v^2}{c^2}}\quad (v=|u_\theta|\;\text{locally}), \label{eq:clock}
    \end{equation}
    recovering Lorentz kinematics.

    \paragraph{Parity prediction.} Two opposite-sign swirl impulses flip surface angle, but any time-rate effect even in speed (\eqref{eq:legacy} or \eqref{eq:clock} expansion) does not reverse.

%=================================================================
    \section{Falsifiable checks}
%=================================================================
    \begin{itemize}
        \item \textbf{Travel-time scaling:} $t_{\mathrm{arr}}\propto\Omega^{-1}$; infer $(k_r,k_z)$ from beam angle or cylinder modes\cite{Greenspan1968}.
        \item \textbf{Bipolarity:} Start/stop impulses give opposite $\eta$ via \eqref{eq:hydro}.
        \item \textbf{Anisotropy:} Off-axis probes detect beams at angle $\tan\alpha=k_z/k_r$; axis receives vertical arrival.
        \item \textbf{Time-rate parity:} Two "clock tracers" at the surface (on-axis vs off-axis reference); bound even-in-speed time drift.
    \end{itemize}

%=================================================================
    \section*{Conclusion (Part I)}
%=================================================================
    An impulsive rotor in a rotating cylinder launches an axisymmetric inertial-wave packet whose $c_{g,z}$ yields a delayed on-axis surface response of the observed magnitude. The two-lobed signal follows the impulse signs. A speed-based time-rate rule, whether legacy \eqref{eq:legacy} or canonical \eqref{eq:clock}, produces even-in-speed effects that do not reverse.

%=================================================================
    \section{Why the Surface "Push--Pull" Requires Background Rotation}
%=================================================================
    \paragraph{Observation.} The delayed bipolar surface signal occurs only for nonzero $\Omega$; for $\Omega=0$ there is no reproducible delayed response.

    \subsection{Mechanism: Coriolis restoring force and inertial-wave beams}
    Linear rotating Euler equations\cite{Batchelor1967}:
    \begin{equation}
        \partial_t \boldsymbol{u} + 2\boldsymbol{\Omega}\!\times\!\boldsymbol{u} = -\nabla \pi,\qquad \nabla\!\cdot\!\boldsymbol{u}=0. \label{eq:rotEuler}
    \end{equation}
    They admit inertial waves with dispersion and group velocity (\eqref{eq:disp}--\eqref{eq:cg}). Arrival time \eqref{eq:tarr}. Hydrostatic coupling \eqref{eq:hydro}.

    \subsection{Scaling explaining the "off" state at $\Omega=0$}
    As $\Omega\to0$: $\omega\to0$, $c_{g,z}\propto\Omega\to0$, $t_{\mathrm{arr}}\to\infty$. Geostrophic scaling yields $\eta\propto\Omega$; viscous damping also grows (Ekman). Hence the delayed bipolar signature vanishes.

    \subsection{Testable predictions vs $\Omega$}
    \begin{align}
        t_{\mathrm{arr}}(\Omega)&\propto\Omega^{-1},\\
        \eta_{\max}(\Omega)&\propto\Omega.
    \end{align}
    Dimensional checks: $[H/c_{g,z}]=\mathrm{s}$; $[p'/\rho g]=\mathrm{m}$.

%=================================================================
    \section{Hypothesized Compressional Signaling Branch (Fluid-Inspired, Microscopic)}
%=================================================================
    \paragraph{Summary.} Add a longitudinal compressional channel with speed $c_{\mathrm{P}}\gg c$ that couples (ultra-weakly) to isotropic stress. Macrodynamics unchanged.

    \subsection{Minimal linear model}
    Relax strict incompressibility and let $p'=B_*\,\rho'/\rho_*$; linearization gives\cite{LandauFluids,Lighthill78}
    \begin{equation}
        \partial_t^2 \rho' - c_{\mathrm{P}}^{\,2}\,\nabla^2 \rho' = 0,\qquad c_{\mathrm{P}}=\sqrt{\frac{B_*}{\rho_*}}. \label{eq:pwave}
    \end{equation}
    Dimension: $[c_{\mathrm{P}}]=\mathrm{m\,s^{-1}}$.

    \subsection{Field-theoretic sketch (source coupling)}
    Introduce scalar $\Phi$ with action
    \begin{equation}
        S_\Phi=\int\!dt\,d^3x\;\tfrac{1}{2}\big[\kappa_\Phi^{-1}(\partial_t\Phi)^2-\Lambda_\Phi^2(\nabla\Phi)^2\big]+\int\!dt\,d^3x\;\varepsilon\,\Phi\,\mathcal{T},
    \end{equation}
    where $\mathcal{T}$ is isotropic stress trace. Euler–Lagrange equation
    \begin{equation}
        \partial_t^2\Phi - C_*^{\,2}\nabla^2\Phi = \varepsilon\,\kappa_\Phi\,\mathcal{T},\qquad C_*\equiv\Lambda_\Phi\sqrt{\kappa_\Phi}. \label{eq:Phi}
    \end{equation}
    Identifying $\rho'\propto\partial_t\Phi$ maps $C_*\to c_{\mathrm{P}}$ and \eqref{eq:pwave}.

    \subsection{Numerical checks with supplied parameters}
    Using $\rho_*=7.0\times10^{-7}\,\mathrm{kg\,m^{-3}}$, $B_*\approx3.49924562\times10^{35}\,\mathrm{J\,m^{-3}}$:
    \begin{equation}
        c_{\mathrm{P}}=\sqrt{B_*/\rho_*}=7.0703057319\times10^{20}\,\mathrm{m\,s^{-1}}. \label{eq:cPnumber}
    \end{equation}
    Ratios: $c_{\mathrm{P}}/c\approx2.36\times10^{12}$; for $L=\SI{10}{m}$, $t=L/c_{\mathrm{P}}\approx1.4\times10^{-20}\,\mathrm{s}$. Compatibility: GW170817 constrains $v_{\mathrm{GW}}\approx c$\cite{Abbott2017PRL,Abbott2017ApJL}; a distinct, trace-coupled channel can evade provided coupling is tiny.

    \subsection{Falsifiable protocol}
    Two baselines $L_1<L_2$; earliest correlated arrivals $t_1,t_2$ (after excluding EM/acoustic paths) obey
    \begin{equation}
        c_{\mathrm{P}} \gtrsim \frac{L_2-L_1}{t_2-t_1}. \label{eq:cpbound}
    \end{equation}
    Null at resolution $\delta t$ yields $c_{\mathrm{P}}\gtrsim (L_2-L_1)/\delta t$.

%=================================================================
    \section{Replacing the Working Fluid by a VAM-Like Superfluid Medium}
%=================================================================
    \paragraph{Aim.} Consider a hypothetical inviscid superfluid with microscopic swirl. Macroscopically, keep rotating Euler; microscopically, two branches: transverse (inertial-like) and longitudinal (compressional).

    \subsection{Transverse inertial-like branch with effective background rate}
    Define $\Omega_*\equiv\Ce/\ell_*$. Then
    \begin{equation}
        \omega = 2\Omega_*\,\frac{k_z}{k},\qquad c_{g,z}=\frac{2\Omega_*\,k_r^2}{k^3} \sim 2\,\Ce\,\frac{R}{\ell_*}. \label{eq:disp-aether}
    \end{equation}
    Arrival:
    \begin{equation}
        t_{\mathrm{arr}}\sim \frac{H\,\ell_*}{2\,\Ce R}. \label{eq:arrive-aether}
    \end{equation}
    \textit{Numbers.} $R=\SI{7.5}{cm}$, $H=\SI{30}{cm}$, $\Ce=1.09384563\times10^6\,\mathrm{m\,s^{-1}}$:
    (i) $\ell_*=R$: $t\sim H/(2\Ce)\approx1.37\times10^{-7}\,\mathrm{s}$; (ii) $\ell_*=\rc$: $t\sim2.6\times10^{-21}\,\mathrm{s}$. Units: $[H\ell_*/(\Ce R)]=\mathrm{s}$.

    \subsection{Longitudinal compressional branch}
    As in \eqref{eq:pwave}, $c_{\mathrm{P}}=\sqrt{B_*/\rho_*}$ is kinematic front speed; detection requires ultra-weak coupling to avoid conflicts with known tests\cite{Brillouin1960,Jackson1999}.

%=================================================================
    \section*{Part II: Reversible Azimuthal Response to Axisymmetric Vertical Forcing}
%=================================================================
    \paragraph{Setting.} Cylinder radius $R$, height $H$, fluid density $\rho$. Container rotates at $\Omega$; base state at rest in rotating frame; absolute vorticity $\bW_a^{(0)}=2\bOm$\cite{Batchelor1967,Greenspan1968}.

    \subsection{Governing equations and vorticity production}
    Inviscid rotating-frame equations:
    \begin{align}
        \p_t \bU + (\bU\!\cdot\!\grad)\bU + 2\bOm\times\bU &= -\grad \Pi,\label{eq:NSrot}\\
        \divg \bU &= 0.\label{eq:incomp}
    \end{align}
    Curl gives absolute-vorticity equation\cite{Batchelor1967,Vallis2017}
    \begin{equation}
        \p_t \bW = \curl(\bU\times \bW_a),\qquad \bW_a=\bW+2\bOm. \label{eq:vortgen}
    \end{equation}
    Linearizing about $\bW_a^{(0)}$ yields
    \begin{equation}
        \p_t \bW \approx 2(\bOm\!\cdot\!\grad)\,\bU. \label{eq:vortlin}
    \end{equation}
    Axisymmetry $\Rightarrow$ vertical component
    \begin{equation}
        \boxed{\;\p_t \omega_z = 2\Omega\,\p_z w.\;} \label{eq:key}
    \end{equation}
    Introduce displacement $\xi$ with $w=\p_t\xi$; integrate from rest:
    \begin{equation}
        \boxed{\;\omega_z(r,z,t) = 2\Omega\,\p_z \xi(r,z,t).\;} \label{eq:omegaxi}
    \end{equation}

    \subsection{From vertical vorticity to azimuthal velocity}
    Kinematic relation
    \begin{equation}
        \omega_z=\frac{1}{r}\,\p_r\!\big(r\,u_\theta\big) \;\Rightarrow\; u_\theta(r,z,t)=\frac{1}{r}\int_0^r \omega_z(r',z,t)\,r'\,\dd r'. \label{eq:uth}
    \end{equation}
    \textit{Gaussian kernel.} For $\xi=Z(t)\,\exp(-(r^2+z^2)/a^2)$,
    \begin{equation}
        u_\theta(r,z,t)= -\,\frac{2\Omega\,Z(t)\,z}{a^{2}}\,e^{-z^{2}/a^{2}}\,\frac{1-e^{-r^{2}/a^{2}}}{r}. \label{eq:uth_gauss}
    \end{equation}
    Sign: cyclonic below ($z<0$), anticyclonic above ($z>0$). Near-axis regularity: $1-e^{-r^2/a^2}\sim r^2/a^2$.

    \subsection{Angle reversal and reversibility}
    Relative angular rate $\dot{\theta}_\text{rel}=u_\theta/r$. Over a stroke $[t_1,t_2]$,
    \begin{equation}
        \Delta \theta_\text{rel}(r,z)=\int_{t_1}^{t_2}\frac{u_\theta}{r}\,\dd t.\end{equation}
    Because $\dot{\theta}_\text{rel}$ is linear in $Z(t)$, a symmetric up–down cycle with zero mean displacement yields
    \begin{equation}
        \boxed{\;\Delta \theta_\text{rel}(r,z;\text{one period})=0\quad(\text{to leading order}).\;}
    \end{equation}
    Deviations arise from viscosity (Ekman), quadratic advection (streaming), or near-resonant inertial waves\cite{Greenspan1968}.

    \subsection{A fluid-inspired kinematic time hypothesis (legacy variant)}
    Define $L\equiv \tfrac{1}{2}\,r_e$ with classical electron radius $r_e= e^2/(4\pi\epsilon_0 m_e c^2)\approx2.82\times10^{-15}\,\mathrm{m}$. With $\omega=2u_\theta/\Ce$, define
    \begin{equation}
        \alpha_f \equiv \frac{\omega L}{c}= \frac{r_e}{c\,\Ce}\,u_\theta,\qquad \frac{\dd \tau}{\dd t}=\sqrt{1-\alpha_f^2}.
    \end{equation}
    For $\alpha_f\ll1$, $\dd\tau/\dd t\approx1-\tfrac{1}{2}\alpha_f^2$. Quadratic parity: angle reverses; time deficit does not.

%=================================================================
    \section*{Part III: Skyrmionic Photon Emission from Knotted Swirl Sources}
%=================================================================
    \subsection{From optical skyrmions to VAM/SST topology}
        Optical skyrmions built from LG modes with opposite circular polarizations carry integer topological charge via Stokes field\cite{Ma2025NanoPhotonSkyrmions,Shen2024NatPhoton,Allen1992OAM}:
        \begin{equation}
            N_{\mathrm{sk}}^{(\mathrm{ph})} = \frac{1}{4\pi} \int d^{2}k_\perp \, \hat{\vec{S}}\cdot\big(\partial_{k_x}\hat{\vec{S}} \times \partial_{k_y}\hat{\vec{S}}\big)\in\mathbb{Z}. \label{eq:Nsk}
        \end{equation}
        Let $\hat{\boldsymbol{\omega}}=\boldsymbol{\omega}/\|\boldsymbol{\omega}\|$; define vortex charge on $\Sigma$ by
        \begin{equation}
            H_{\mathrm{vortex}}[\hat{\boldsymbol{\omega}}|\Sigma] \equiv \frac{1}{4\pi}\int_{\Sigma}\!\hat{\boldsymbol{\omega}}\cdot\big( \partial_x \hat{\boldsymbol{\omega}} \times \partial_y \hat{\boldsymbol{\omega}} \big)\, dx \, dy. \label{eq:Hvortex}
        \end{equation}
        \textbf{Topological inheritance.} \; $N_{\mathrm{sk}}^{(\mathrm{ph})} = H_{\mathrm{vortex}}[\hat{\boldsymbol{\omega}}|\Sigma]$.

    \subsection{Projection law: from swirl to Stokes}
        Let $\mathbf{p}(\mathbf{r})=p_0(\mathbf{r})\,\hat{\boldsymbol{\omega}}_\perp(\mathbf{r})$ on $\Sigma$. Far-field Jones amplitude in mode space (LG basis $u_{p,\ell}$, helicity $\sigma=\pm1$):
        \begin{equation}
            A_{p\ell\sigma} = \int_{\Sigma} ( \hat{e}_\sigma^{\!*}\!\cdot\!\mathbf{P}_\perp\hat{\boldsymbol{\omega}}_\perp )\, u_{p,\ell}(\mathbf{r})\, e^{i\Phi(\mathbf{r})}\,\dd^2 r, \label{eq:modal_overlap}
        \end{equation}
        with $\mathbf{P}_\perp=\mathbf{I}-\hat{\mathbf{k}}\hat{\mathbf{k}}^\top$ and a phase $\Phi$. Stokes field computed from $\mathbf{E}$ yields \eqref{eq:Nsk}.

    \subsection{VAM/SST radiative vertex: OAM additivity and chirality}
        Per-photon OAM in SPDC conserves $\ell$ additively\cite{Kopf2025OAMConservation,Walborn2010SPDCReview}; adopt
        \begin{equation}
            \ell_{\rm src} = \sum_{j=1}^{n} \ell_j \quad (n\text{-photon channel}). \label{eq:oam_additivity}
        \end{equation}
        Swirl chirality sets photon helicity: ccw $\Rightarrow \sigma=+1$, cw $\Rightarrow \sigma=-1$.

    \subsection{Frequency and energy scale}
        \begin{equation}
            \Omegazero \equiv \frac{\Ce}{\rc},\qquad E_0=\hbar\Omegazero. \label{eq:Omega0}
        \end{equation}
        Numerically (given values): $\Omegazero\simeq7.77\times10^{20}\,\mathrm{s^{-1}}$, $E_0\simeq0.511\,\mathrm{MeV}$.

        \paragraph{Quantized line set.} Let
            \begin{equation}
                \omega_{m\ell} = m\,\Omegazero + \delta\omega_\ell,\qquad E_{m\ell}=\hbar\omega_{m\ell}= m\,E_0 + \delta E_\ell. \label{eq:omegamell}
            \end{equation}

    \subsection{Predictions}
    \begin{itemize}
        \item \textbf{P1 (topological spectroscopy):} $N_{\mathrm{sk}}^{(\mathrm{ph})} = H_{\mathrm{vortex}}$ within a fixed mode family.
        \item \textbf{P2 (chirality--helicity):} flip swirl chirality $\Rightarrow$ flip photon helicity; OAM additivity \eqref{eq:oam_additivity} intact.
        \item \textbf{P3 (robustness):} $N_{\mathrm{sk}}^{(\mathrm{ph})}$ invariant under smooth, linear propagation.
        \item \textbf{P4 (Purcell tuning):} LDOS changes reshape texture without changing integer charge.
        \item \textbf{P5 (line assignments):} $m=1$ near $m_e c^2$; absence of a strong ladder at $n\times0.511\,\mathrm{MeV}$ constrains $\delta E_\ell$ and higher-$m$ couplings.
    \end{itemize}

%-----------------------------------------------------------------
    \section{Minimal experimental roadmap}
%-----------------------------------------------------------------
    \subsection{E1: Re-analyze single-photon skyrmion data}
        Express measured Stokes fields via a reconstructed $\hat{\boldsymbol{\omega}}_\perp$; test $N_{\mathrm{sk}}^{(\mathrm{ph})}=H_{\mathrm{vortex}}$, and OAM/helicity control.

    \subsection{E2: Swirl-based emitters at non-optical frequencies}
        Macroscopic swirl sources coupled via electro-/magneto-/piezo-optic effects should reproduce topological mapping independent of absolute energy scale.

    \subsection{E3: High-energy cross-checks}
        Consistency with electron mass/QED precision and absence of extra stable lines near multiples of $0.511\,\mathrm{MeV}$.

%-----------------------------------------------------------------
    \section*{Global Conclusion}
%-----------------------------------------------------------------
    \paragraph{Part I.} Inertial-wave packet with $c_{g,z}$ explains delayed axial "push--pull"; $t_{\mathrm{arr}}\propto\Omega^{-1}$ and $\eta_{\max}\propto\Omega$. Time-rate hypotheses are even in speed.
    \paragraph{Part II.} Vertical forcing produces opposite-sign vorticity above/below; linear reversibility gives zero net angle over a symmetric cycle.
    \paragraph{Part III.} Skyrmionic photon textures inherit vortex topology; OAM additivity and chirality--helicity mapping follow; $\Omegazero=\Ce/\rc$ sets a kinematic scale with numerical validation.

    \paragraph{Analogy (10-year-old).} Knots in a river can flick patterns into light; how they’re tied (knot type) decides the pattern, not how loudly you splash.

%====================== Acknowledgments ======================
    \paragraph*{Acknowledgments.} Fluid/wave results: \cite{Batchelor1967,Greenspan1968,Saffman1992,Vallis2017,LandauFluids,Lighthill78}. Topological/OAM optics: \cite{Allen1992OAM,Ma2025NanoPhotonSkyrmions,Shen2024NatPhoton,Walborn2010SPDCReview,Moffatt1969Helicity}. Relativistic/causality constraints: \cite{Brillouin1960,Jackson1999,Abbott2017PRL,Abbott2017ApJL}.

%====================== Bibliography (BibTeX) ======================
        \begin{filecontents}[overwrite]{refs.bib}
            @book{Batchelor1967,title={An Introduction to Fluid Dynamics},author={Batchelor, G. K.},year={1967},publisher={Cambridge Univ. Press}}
            @book{Greenspan1968,title={The Theory of Rotating Fluids},author={Greenspan, H. P.},year={1968},publisher={Cambridge Univ. Press}}
            @book{Saffman1992,title={Vortex Dynamics},author={Saffman, P. G.},year={1992},publisher={Cambridge Univ. Press}}
            @book{Vallis2017,title={Atmospheric and Oceanic Fluid Dynamics},edition={2},author={Vallis, G. K.},year={2017},publisher={Cambridge Univ. Press}}
            @book{LandauFluids,title={Fluid Mechanics},author={Landau, L. D. and Lifshitz, E. M.},edition={2},year={1987},publisher={Pergamon Press}}
            @book{Lighthill78,title={Waves in Fluids},author={Lighthill, M. J.},year={1978},publisher={Cambridge Univ. Press}}
            @book{Brillouin1960,title={Wave Propagation and Group Velocity},author={Brillouin, L.},year={1960},publisher={Academic Press}}
            @book{Jackson1999,title={Classical Electrodynamics},author={Jackson, J. D.},edition={3},year={1999},publisher={Wiley}}
            @article{Abbott2017PRL,title={GW170817: Observation of Gravitational Waves from a Binary Neutron Star Inspiral},author={Abbott, B. P. and others},journal={Phys. Rev. Lett.},volume={119},number={16},pages={161101},year={2017},doi={10.1103/PhysRevLett.119.161101}}
            @article{Abbott2017ApJL,title={Multi-messenger Observations of a Binary Neutron Star Merger},author={Abbott, B. P. and others},journal={Astrophys. J. Lett.},volume={848},pages={L12},year={2017},doi={10.3847/2041-8213/aa91c9}}
            @article{Allen1992OAM,title={Orbital angular momentum of light and the transformation of Laguerre--Gaussian laser modes},author={Allen, L. and Beijersbergen, M. W. and Spreeuw, R. J. C. and Woerdman, J. P.},journal={Phys. Rev. A},volume={45},pages={8185--8189},year={1992},doi={10.1103/PhysRevA.45.8185}}
            @article{Moffatt1969Helicity,title={The degree of knottedness of tangled vortex lines},author={Moffatt, H. K.},journal={J. Fluid Mech.},volume={35},pages={117--129},year={1969},doi={10.1017/S0022112069000991}}
            @article{Walborn2010SPDCReview,title={Spatial correlations in parametric down-conversion},author={Walborn, S. P. and Monken, C. H. and Padua, S. and Souto Ribeiro, P. H.},journal={Phys. Rep.},volume={495},pages={87--139},year={2010},doi={10.1016/j.physrep.2010.06.003}}
            @article{Shen2024NatPhoton,title={Topological textures of structured photons},author={Shen, Y. and others},journal={Nat. Photon.},year={2024}}
            @article{Ma2025NanoPhotonSkyrmions,title={Single-photon optical skyrmions in spin--orbit--engineered microcavities},author={Ma, X. and others},journal={(journal details forthcoming)},year={2025}}
            @article{Kopf2025OAMConservation,title={Per-photon orbital-angular-momentum conservation in down-conversion},author={Kopf, T. and others},journal={(venue forthcoming)},year={2025}}
        \end{filecontents}

        \bibliographystyle{unsrt}
        \bibliography{refs}

\end{document}