\documentclass[a4paper,10pt]{letter}

\usepackage[T1]{fontenc}
\usepackage[utf8]{inputenc}
\usepackage{lmodern}
\usepackage[hidelinks]{hyperref}
\usepackage{microtype}
\usepackage[margin=1in]{geometry}

% Sender info
\signature{Omar Iskandarani\\
Independent Researcher, Groningen,\\ The Netherlands\\
ORCID: 0009-0006-1686-3961\\
Email: \href{mailto:info@omariskandarani.com}{info@omariskandarani.com}}

\address{Omar Iskandarani\\
Vinkenstraat 86A\\
9713 TK Groningen\\
The Netherlands}

\date{\today}

\begin{document}

    \begin{letter}{Editors\\
    \textit{Chaos: An Interdisciplinary Journal of Nonlinear Science}}

    Dear Editors of \textit{Chaos},

    Please consider the manuscript entitled
    ``Delay-Induced Mode Selection in Circulating Feedback Systems''
    for publication in \textit{Chaos}.

    This work investigates how finite circulation delays in closed feedback loops give rise to discrete, stability-selected phase-locked states within a minimal classical framework. Using a reduced phase-oscillator description, we show that temporal nonlocality alone—without invoking spatial standing-wave boundary conditions or microscopic quantization—enforces a ladder of stable operating frequencies through dynamical stability constraints.

    For transparency, we note that a separate manuscript by the same author is currently under review at \textit{Chaos}, entitled
    ``Delay-Induced Pattern Formation as a Route to Mode Discreteness in Nonlinear Ring Systems.''
    The two works address distinct physical questions at different descriptive levels:

    \begin{itemize}
        \item The manuscript currently under review focuses on spatiotemporal pattern formation, where delay induces extended structures and mode discreteness emerges from the organization of spatial degrees of freedom in nonlinear ring systems.
        \item In contrast, the present manuscript deliberately eliminates spatial structure and develops a minimal effective phase description, demonstrating that discrete operating states arise already at the level of a reduced temporal model through delay-induced stability selection.
    \end{itemize}

    Accordingly, the present work emphasizes:
    \begin{itemize}
        \item an effective constitutive interpretation of delay as a macroscopic ingredient,
        \item stability-based mode selection rather than pattern formation,
        \item connections to homogenization and bianisotropic effective theories, where hidden nonlocality legitimizes emergent macroscopic structure.
    \end{itemize}

    There is no overlap in figures, derivations, or text between the two manuscripts, and each is intended to stand independently for readers interested in complementary aspects of delay-induced dynamics.

    We believe this contribution will be of interest to the \textit{Chaos} readership working on delayed feedback systems, nonlinear oscillators, phase-locked loops, and effective descriptions of nonlocal dynamics.

    Thank you for your consideration.

    \closing{Sincerely,}

    \end{letter}

\end{document}