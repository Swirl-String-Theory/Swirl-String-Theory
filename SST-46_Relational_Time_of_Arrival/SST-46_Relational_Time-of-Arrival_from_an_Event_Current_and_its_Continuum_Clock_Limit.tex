% =====================================================================
% Foundations of Physics (FoP) -- sn-jnl submission version
% Paper: Relational Time-of-Arrival from an Event Current and its Continuum Clock Limit
% Author: Omar Iskandarani
% Notes:
% - Single .tex file (no \input)
% =====================================================================

% \documentclass[referee,pdflatex,sn-mathphys-num]{sn-jnl} % optional for review
\documentclass[pdflatex,sn-mathphys-num]{sn-jnl}

% =========================
% Core packages (lean)
% =========================
\usepackage[T1]{fontenc}
\usepackage[utf8]{inputenc}
\usepackage{lmodern}
\usepackage{microtype}
\usepackage{graphicx}
\usepackage{booktabs}
\usepackage{amsmath,amssymb,amsfonts}
\usepackage{amsthm}
\usepackage{mathtools}
\usepackage{bm}
\usepackage{bbm}

% NOTE: \usepackage{physics}.
% If you rely on \qty, \dv, \pdv, etc., uncomment; otherwise leave disabled.
% \usepackage{physics}

% =========================
% Theorem environments
% =========================
\theoremstyle{thmstylethree}
\newtheorem{definition}{Definition}[section]
\newtheorem{lemma}{Lemma}[section]

\theoremstyle{thmstyleone}
\newtheorem{proposition}{Proposition}[section]

\theoremstyle{thmstyletwo}
\newtheorem{remark}{Remark}[section]

\numberwithin{equation}{section}

% =========================
% Macros (SST)
% =========================
\newcommand{\dd}{\mathrm{d}}
\newcommand{\Sig}{\Sigma}
\newcommand{\Jev}{j^\mu_{\mathrm{ev}}}
\newcommand{\Nev}{N_{\mathrm{ev}}}
\newcommand{\F}{\mathcal{F}}
\newcommand{\cO}{\mathcal{O}}
\newcommand{\eps}{\varepsilon}

% QFT macros
\newcommand{\W}{W}
\newcommand{\Gret}{G_{\mathrm{ret}}}
\newcommand{\Gplus}{G^{+}}
\newcommand{\Kzero}{K_{0}}

% =========================
% Metadata
% =========================
\newcommand{\papertitle}{Relational Time-of-Arrival as a Covariant Field Observable: From Event Currents to a Continuum Clock Limit}
\newcommand{\papershorttitle}{Relational Time-of-Arrival}
\newcommand{\paperemail}{info@omariskandarani.com}
\newcommand{\paperlocation}{Groningen}
\newcommand{\paperkeywords}{relational time, time of arrival, quantum time observables, covariant observables, event currents, quantum field theory}
\newcommand{\paperdoi}{10.5281/zenodo.18050158}
\newcommand{\paperorcid}{0009-0006-1686-3961}



\begin{document}

    \title[\papershorttitle]{\papertitle}

    \author*[1]{\fnm{Omar} \sur{Iskandarani}}\email{\paperemail}

    \affil*[1]{ \orgname{Independent Researcher},
        \orgaddress{\city{\paperlocation}, \country{The Netherlands}}\\
        \today
    }


    \abstract{%
        Time-of-arrival (TOA) in quantum theory sits at an awkward intersection: Pauli-type arguments obstruct a self-adjoint operator canonically conjugate to a semibounded Hamiltonian, while relativistic localization makes "the" time parameter detector- and frame-dependent. We propose a covariant alternative that treats TOA as a relational field observable rather than an operator. The construction uses two conserved ingredients: a matter current $J^\mu$ that records detector crossings through a world-tube $\Sigma$, and an event-count current $j^\mu_{\mathrm{ev}}$ whose integral defines a discrete clock $N_{\mathrm{ev}}$. Coarse-graining $j^\mu_{\mathrm{ev}}$ over a scale $\ell$ yields an infrared clock field $T(x)$, and TOA becomes the flux through $\Sigma$ conditioned on the local clock reading $T|_{\Sigma}$. In a $1+1$-dimensional example with a narrowband Klein--Gordon wavepacket and a gapped quadratic clock EFT, the TOA density reduces to a convolution of the semiclassical flux profile with the clock readout distribution. This predicts a calculable arrival-time broadening set by the clock variance, while massive clock correlations impose an exponentially suppressed envelope on contributions that would require spacelike clock correlations. The result is a TOA observable independent of detector microphysics up to the geometry of $\Sigma$ and the conserved flux $J^\mu$, with a controlled UV-to-IR matching between event counts and continuum time.
    }


    \keywords{\paperkeywords}

    \maketitle


% ======================================================================
    \section*{Notation summary}

    \begin{center}
    \begin{tabular}{ll}
    $J^\mu$ & conserved matter current (detector flux) \\
    $j^\mu_{\mathrm{ev}}$ & conserved event-count current (clock sector) \\
    $N_{\mathrm{ev}}$ & discrete event-count clock \\
    $T(x)$ or $\chi(x)$ & continuum clock field (IR limit) \\
    $\Sigma$ & detector world-tube \\
    $\mathcal{F}$ & detector flux functional $n_\mu J^\mu$ \\
    $(\cdot)_\ell$ & coarse-graining over spacetime scale $\ell$
    \end{tabular}
    \end{center}

% ======================================================================
    \section{Introduction}
% ======================================================================
        Time is usually supplied to quantum theory as an external parameter. Operationally, however, time is read from physical degrees of freedom, and in relativistic settings those degrees of freedom are local. This mismatch is visible already in nonrelativistic quantum mechanics: Pauli's argument suggests that a self-adjoint time operator canonically conjugate to a Hamiltonian bounded from below cannot exist in general \cite{Pauli1980}. In quantum field theory the situation is sharper, because localization, causality, and detector modelling enter from the start \cite{BuschGiannitrapani1995}.

        Many successful TOA constructions therefore adopt explicitly operational definitions: POVM-based prescriptions such as Kijowski's \cite{Kijowski1974} and later developments \cite{Giannitrapani1997}, as well as detector models and measurement couplings \cite{BuschGiannitrapani1995,ColosiOeckl2025TimeOperatorTOA}. These approaches produce useful arrival-time statistics, but typically tie the observable to a chosen clock variable, a preferred slicing, or detector microphysics.

        In this paper we take a different starting point. We define TOA as a \emph{relational field observable}: detector crossings are binned by a \emph{physical clock field} constructed from a conserved event-count current. Concretely, we introduce (i) a conserved matter current $J^\mu$ that determines flux through a detector world-tube $\Sigma$, and (ii) a conserved (or approximately conserved) current $j^\mu_{\mathrm{ev}}$ whose integral counts localized events and thereby defines a discrete clock. A controlled coarse-graining over a scale $\ell$ produces an infrared clock field $T(x)$, which plays the role of a continuum time variable on scales $\gg \ell$.

        The payoff is twofold. First, the TOA observable is manifestly covariant and does not require postulating a self-adjoint time operator. Second, the clock sector carries dynamics: an effective field theory for $T$ fixes correlation scales, which then enter arrival-time statistics in a parameter-controlled way.

        \paragraph{Relation to existing time-of-arrival formalisms.}
            The TOA literature offers several workable definitions—POVM prescriptions, explicit detector models, and path-integral or histories-based constructions. Our goal is not to compete with their operational success, but to isolate a covariant object that can be stated directly at the field level. The key move is to treat the clock as part of the physics: TOA is defined by conditioning flux through $\Sigma$ on a clock reading built from an event current. In that sense, “detector independence” is not achieved by abstracting away measurement, but by specifying the minimal conserved structures that any detector model must reduce to in the appropriate limit.


            Rather than revising measurement postulates or postulating a generalized time operator, we keep the standard field-theoretic setting and change only what is taken to be primitive: TOA is defined as a relational field-functional built from conserved currents and a physical clock sector. The absence of a universal time operator is then not a gap to be patched, but a cue that TOA should be formulated relationally through correlations between physical degrees of freedom.


        \paragraph{Scope and assumptions.}
        Throughout this work, the event current $j^\mu_{\mathrm{ev}}$ is treated as a
        \emph{physical but model-agnostic} structure. It may be fundamental (e.g.\ a
        primitive detection or interaction current) or emergent (e.g.\ a coarse-grained
        description of localized spacetime processes such as interaction vertices,
        defect crossings, or excitation worldlines). The approximate conservation
        $\nabla_\mu j^\mu_{\mathrm{ev}}\approx 0$ is expected to hold whenever the relevant
        events are robust under smooth deformations on the scales of interest, with
        violations suppressed by the ultraviolet resolution scale.

        The coarse-graining map $(\cdot)_\ell$ is assumed to be a local, causal smoothing
        operation over spacetime regions of characteristic size $\ell$, preserving
        current conservation in the infrared. No specific kernel is fixed; admissible
        choices include covariant averaging, block-spin–type maps, or detector-induced
        temporal binning, provided they respect locality and current conservation at
        scales $\gg\ell$.

% ======================================================================
    \section{Detector World-Tubes and Matter Flux}
% ======================================================================

        We work on a four-dimensional Lorentzian spacetime $(\mathcal{M},g_{\mu\nu})$.
        A detector is modeled as a timelike world-tube $\Sig \subset \mathcal{M}$,
        with intrinsic coordinates $\sigma^a$ ($a=1,2,3$), induced metric $h_{ab}$,
        and future-directed unit normal $n_\mu$.

        Let $J^\mu$ denote a conserved matter current,
        \begin{equation}
            \nabla_\mu J^\mu = 0 .
        \end{equation}

        \begin{definition}[Matter flux]
            The matter flux through $\Sig$ is
            \begin{equation}
                \F(\sigma) \equiv n_\mu J^\mu\big(X(\sigma)\big).
            \end{equation}
        \end{definition}

        The total registered flux is
        \begin{equation}
            \mathcal{N} = \int_\Sig \dd^3\sigma\,\sqrt{|h|}\,\F(\sigma).
        \end{equation}

% ======================================================================
    \section{Event Currents and Discrete Clocks}
% ======================================================================

        \begin{definition}[Event current]
            An \emph{event current} is a vector field $\Jev$ satisfying
            \begin{equation}
                \nabla_\mu \Jev \approx 0 ,
            \end{equation}
            where equality may be exact or hold up to controlled corrections.
        \end{definition}

        \begin{remark}[Concrete example]
        In a relativistic QFT scattering experiment, $j^\mu_{\mathrm{ev}}$ may represent the effective flux of detector-registered interaction events, i.e.\ a spacetime current whose integral counts localized vertex/crossing events relevant to the measurement.
        \end{remark}

        \begin{definition}[Event count]
            The associated event count through a hypersurface $\Sig$ is
            \begin{equation}
                \Nev(\Sig) \equiv \int_\Sig \dd\Sigma_\mu\,\Jev .
            \end{equation}
        \end{definition}

        \begin{proposition}
            If $\nabla_\mu \Jev = 0$, then $\Nev(\Sig)$ is invariant under smooth
            deformations of $\Sig$ that preserve its boundary.
        \end{proposition}

        The discrete-event clock is
        \begin{equation}
            T_{\mathrm{disc}} \equiv \Nev .
        \end{equation}

        \paragraph{Conceptual hierarchy.}
        The logical structure of the construction may be summarized schematically as
        \begin{center}
        \begin{tabular}{c c c}
        event current $j^\mu_{\mathrm{ev}}$
        & $\longrightarrow$ &
        discrete clock $T_{\mathrm{disc}} = N_{\mathrm{ev}}$ \\
        & &
        $\downarrow$ coarse-graining \\
        & &
        continuum clock field $T(x)$
        \end{tabular}
        \end{center}
        The event current is the fundamental object; discrete and continuous clocks are
        its ultraviolet and infrared descriptions, respectively.

% ======================================================================
    \section{Coarse-Graining and the Continuum Clock Limit}
% ======================================================================

        Let $(\cdot)_\ell$ denote coarse-graining over a spacetime scale $\ell$.
        Define a scalar clock field $T(x)$ by
        \begin{equation}
            \partial_\mu T(x) \equiv \frac{1}{\nu_0}\,\big(j_{\mathrm{ev},\mu}(x)\big)_\ell ,
            \label{eq:clockgradient}
        \end{equation}
        where $\nu_0$ has dimensions (events per unit time).

        \begin{lemma}
            If $\nabla_\mu \Jev =0$, then $T(x)$ is locally well-defined up to an
            additive constant; global obstructions correspond to nontrivial topology.
        \end{lemma}

        \begin{proposition}[IR equivalence]
            At scales $\gg \ell$, event counts and clock differences match:
            \begin{equation}
                \Nev(\Sig)\approx \nu_0 \int_\Sig \dd\Sigma_\mu\,\partial^\mu T
                = \nu_0\,\Delta T.
            \end{equation}
        \end{proposition}

% ======================================================================
    \section{Clock-Sector Effective Field Theory}
% ======================================================================

        Write the clock field as $T(x)=t+\tau(x)$ around a background $T_0=t$.
        A minimal stable quadratic EFT is
        \begin{equation}
            S_\tau = \frac{1}{2}\int \dd^4x
            \left[
                (\partial_t\tau)^2
            - c_\tau^2(\nabla\tau)^2
            - \mu_\tau^2 \tau^2
            \right].
            \label{eq:clockEFT}
        \end{equation}
        The parameter $\mu_\tau$ sets a correlation scale in space and time.


        In Appendix~\ref{app:prop}, $\mu_\tau$ is shown to control the exponential suppression envelope of spacelike clock correlations and the corresponding early-arrival bounds.

        \begin{remark}[Correlation scale]
        The mass parameter $\mu_\tau$ sets the characteristic temporal and spatial
        correlation scale of the clock field. As shown explicitly in
        Appendix~\ref{app:prop}, the same parameter controls the exponential
        suppression envelope of spacelike clock correlations and, consequently,
        bounds early-arrival contributions in the TOA distribution.
        \end{remark}

% ======================================================================
    \section{Relational Time-of-Arrival Observable}
% ======================================================================

        \begin{definition}[Relational TOA distribution]
            The TOA probability density with respect to the clock field $T$ is
            \begin{equation}
                p(\Theta) \equiv
                \frac{1}{\mathcal{N}}
                \int_\Sig \dd^3\sigma\,\sqrt{|h|}\,
                \F(\sigma)\,
                \delta\!\big(T_\Sig(\sigma)-\Theta\big),
                \label{eq:TOA}
            \end{equation}
            where $T_\Sig(\sigma)\equiv T(X(\sigma))$.
        \end{definition}

    \paragraph{Why this is nontrivial.}
        Equation~\eqref{eq:TOA} is not a relabeling of an external time coordinate. The clock $T(x)$ is a dynamical field (or effective field) with its own correlators, fixed by the clock-sector EFT. Those correlators enter TOA statistics through the averaging in Eq.~\eqref{eq:pThetaAvg} and lead to quantitative modifications—e.g.\ broadened arrival-time distributions and an $\mu_\tau$-controlled envelope for contributions that would require spacelike clock correlations (Appendix~\ref{app:prop}).

        This definition is covariant and depends on detector microphysics only
        through the geometry of $\Sig$ and the flux functional $\F$.

% ======================================================================
    \section{Semiclassical Limit}
% ======================================================================

        In the semiclassical regime of sharply peaked wavepackets and weak clock
        fluctuations,
        \begin{equation}
            p(\Theta)\to \delta(\Theta-\Theta_{\mathrm{cl}}),
        \end{equation}
        where $\Theta_{\mathrm{cl}}$ matches the classical time-of-flight.

% ======================================================================
    \section{Worked Example in $1+1$ Dimensions}
% ======================================================================

        We now give an explicit $1+1$D computation (wavepacket + clock correlator).
        This section is intended as a concrete template; the same steps extend to
        $3+1$D with transverse integrals.

        \subsection{Setup: detector worldline and matter wavepacket}

            Work in Minkowski space with coordinates $(t,x)$ and metric $\eta=\mathrm{diag}(1,-1)$.
            Model the detector as the timelike worldline at fixed $x=x_D$:
            \begin{equation}
                \Sig:\quad X^\mu(\sigma)=(t=\sigma,\ x=x_D).
            \end{equation}
            For a $1+1$D detector ``world-tube'', the induced measure reduces to $\dd\sigma=\dd t$.

            Take a complex Klein--Gordon field $\psi$ of mass $m$ with conserved $U(1)$ current
            \cite{Peskin1995,Weinberg1995}:
            \begin{equation}
                J^\mu = \frac{i}{2m}\left(\psi^\ast \partial^\mu \psi - \psi \partial^\mu \psi^\ast\right),
                \qquad \partial_\mu J^\mu = 0.
                \label{eq:KGcurrent}
            \end{equation}
            In $1+1$D, the spatial flux at the detector is $J^x(t,x_D)$.
            We take the incoming direction to be $+x$ and define the registered flux as
            \begin{equation}
                \F(t)\equiv J^x(t,x_D)\,\mathbbm{1}_{\{J^x\ge 0\}},
            \end{equation}
            where the indicator restricts to incoming flux (a standard operational step
            in TOA treatments \cite{Kijowski1974,Giannitrapani1997}).

            Let the positive-frequency wavepacket be
            \begin{equation}
                \psi(t,x)=\int_{-\infty}^{+\infty}\frac{\dd k}{\sqrt{2\pi}}\,
                a(k)\,e^{-i\omega_k t + i k x},
                \qquad \omega_k=\sqrt{k^2+m^2}.
                \label{eq:wavepacket}
            \end{equation}
            Choose $a(k)$ narrowly peaked at $k_0>0$ with width $\Delta k\ll k_0$.
            Then the packet propagates with group velocity
            \begin{equation}
                v_g=\frac{\partial \omega_k}{\partial k}\Big|_{k_0}=\frac{k_0}{\omega_{k_0}}\in(0,1).
            \end{equation}

        \subsection{Semiclassical approximation for the flux}

            Under the narrowband approximation, $J^x$ reduces to the familiar form
            \begin{equation}
                J^x(t,x)\approx v_g\,|\psi(t,x)|^2,
                \label{eq:Japprox}
            \end{equation}
            up to corrections of order $\cO((\Delta k/k_0)^2)$.
            A Gaussian $a(k)$ gives an approximately Gaussian arrival profile at $x=x_D$:
            \begin{equation}
                \F(t)\approx \F_0\,\exp\!\left[-\frac{(t-t_{\mathrm{cl}})^2}{2\sigma_t^2}\right],
                \qquad
                t_{\mathrm{cl}}=\frac{x_D-x_0}{v_g},
                \label{eq:fluxGaussian}
            \end{equation}
            where $x_0$ is the initial packet center and $\sigma_t\sim 1/(v_g\Delta k)$.
            Normalize $\mathcal{N}=\int \dd t\,\F(t)$.

        \subsection{Clock field on the detector and stationary correlator}

            Take the continuum clock field
            \begin{equation}
                T(t,x)=t+\tau(t,x).
            \end{equation}
            The relational TOA density \eqref{eq:TOA} becomes in $1+1$D
            \begin{equation}
                p(\Theta)=\frac{1}{\mathcal{N}}\int_{-\infty}^{+\infty}\dd t\,
                \F(t)\,
                \Big\langle \delta\!\big(t+\tau(t,x_D)-\Theta\big)\Big\rangle_{\tau},
                \label{eq:pThetaAvg}
            \end{equation}
            where $\langle\cdot\rangle_\tau$ denotes averaging over clock fluctuations.

            Assume the clock sector is in a stationary Gaussian state induced by the quadratic EFT
            \eqref{eq:clockEFT}. Then the single-point random variable $\tau(t,x_D)$ is Gaussian,
            with mean $0$ and variance
            \begin{equation}
                \sigma_\tau^2 \equiv \langle \tau(t,x_D)^2\rangle = C(0,0),
            \end{equation}
            where the two-point function is
            \begin{equation}
                C(\Delta t,\Delta x)\equiv \langle \tau(t,x)\tau(t+\Delta t,x+\Delta x)\rangle .
            \end{equation}

            In $1+1$D, for a gapped relativistic scalar ($c_\tau=1$) at equal position $\Delta x=0$,
            the stationary correlator decays exponentially in $|\Delta t|$:
            \begin{equation}
                C(\Delta t,0)\propto e^{-\mu_\tau|\Delta t|}
                \quad\text{for}\quad |\Delta t|\gg \mu_\tau^{-1},
                \label{eq:Cexp}
            \end{equation}
            with proportionality factor depending on UV regularization (Appendix~\ref{app:prop}).

        \subsection{Closed-form TOA distribution: convolution formula}

            For a stationary Gaussian variable $X=\tau(t,x_D)$ with variance $\sigma_\tau^2$,
            \begin{equation}
                \left\langle \delta(t+X-\Theta)\right\rangle_X
                =\frac{1}{\sqrt{2\pi\sigma_\tau^2}}
                \exp\!\left[-\frac{(\Theta-t)^2}{2\sigma_\tau^2}\right].
            \end{equation}
            Therefore \eqref{eq:pThetaAvg} becomes the convolution
            \begin{equation}
                p(\Theta)=\frac{1}{\mathcal{N}}
                \int_{-\infty}^{+\infty}\dd t\,
                \F(t)\,
                \frac{1}{\sqrt{2\pi\sigma_\tau^2}}
                \exp\!\left[-\frac{(\Theta-t)^2}{2\sigma_\tau^2}\right].
                \label{eq:convolution}
            \end{equation}

            If $\F(t)$ is Gaussian as in \eqref{eq:fluxGaussian}, the convolution is again Gaussian:
            \begin{equation}
                p(\Theta)=\frac{1}{\sqrt{2\pi(\sigma_t^2+\sigma_\tau^2)}}
                \exp\!\left[-\frac{(\Theta-t_{\mathrm{cl}})^2}{2(\sigma_t^2+\sigma_\tau^2)}\right].
                \label{eq:pGaussian}
            \end{equation}
            Thus the clock sector produces a \emph{predictive broadening}:
            \begin{equation}
                \mathrm{Var}(\Theta)=\sigma_t^2+\sigma_\tau^2,
                \qquad
                \sigma_\tau^2=C(0,0).
            \end{equation}

            \paragraph{Falsifiability.}
            Given an experimentally or numerically specified flux profile $\mathcal{F}(t)$
            and a measured or modeled clock variance $\sigma_\tau^2$, the convolution
            formula~\eqref{eq:convolution} predicts a unique TOA distribution without
            additional fitting parameters. Deviations from the predicted variance
            $\mathrm{Var}(\Theta)=\sigma_t^2+\sigma_\tau^2$ or from the expected crossover
            behavior at $\Delta t\sim\mu_\tau^{-1}$ would directly falsify the assumed clock
            sector or its effective field description.

            \begin{remark}[Gaussian tails vs exponential envelopes]
                The Gaussian form of $p(\Theta)$ arises from the \emph{local, one-point}
                statistics of the clock readout at the detector. Exponential behavior enters
                instead through the \emph{spacetime propagation} of clock correlations: early-
                arrival contributions requiring spacelike correlations are bounded by an
                exponential \emph{envelope} set by $\mu_\tau^{-1}$, derived in
                Appendix~\ref{app:prop}. These two effects are conceptually distinct and
                govern different regimes of the TOA distribution.
            \end{remark}

        \subsection{Discrete-event ultraviolet corrections}

            If the UV clock is an event-count $T_{\mathrm{disc}}=\Nev$, then at time resolution
            $\Delta t \lesssim \ell$ one expects renewal-type corrections:
            \begin{equation}
                p(\Theta)\;=\;p_{\mathrm{IR}}(\Theta)\;+\;\delta p_{\mathrm{UV}}(\Theta),
            \end{equation}
            where $p_{\mathrm{IR}}$ is the continuum prediction (e.g. \eqref{eq:pGaussian}) and
            $\delta p_{\mathrm{UV}}$ carries integer-step features and non-Gaussianity that vanish
            upon coarse-graining $\Delta t\gg \ell$.

        \medskip
        \noindent
        \textit{Bridge to a concrete realization.}
        So far the discussion has been model-independent: $J^\mu$ and $j^\mu_{\mathrm{ev}}$ are treated as effective conserved structures. The next section simply records one explicit realization in which $j^\mu_{\mathrm{ev}}$ is generated by defect worldlines; none of the preceding definitions rely on that choice.

    \section{Example Realization: Defect Worldlines and a Foliation Clock}
        \label{sec:SSTrealization}

        The relational TOA construction above is model-independent: it requires only
        (i) a conserved matter flux $J^\mu$ through the detector world-tube $\Sigma$
        and (ii) a conserved (or approximately conserved) event current
        $j^\mu_{\mathrm{ev}}$ whose integral defines the ultraviolet event-count clock.
        Here we record one concrete realization in a hydrodynamic/defect setting
        motivated by Swirl-String Theory (SST), presented strictly as an example.

        \subsection{Defect (knot-center) worldline current}

            Consider $N$ stable, localized defects with center worldlines $z_k^\mu(\tau)$.
            Define the distributional defect current
            \begin{equation}
                j^\mu_{\mathrm{ev}}(x)
                \;\equiv\;
                \sum_{k=1}^{N} q_k \int \dd\tau\,
                \frac{\dd z_k^\mu}{\dd\tau}\,
                \delta^{(4)}\!\big(x-z_k(\tau)\big),
                \label{eq:defectCurrent}
            \end{equation}
            where $q_k\in\mathbb{Z}$ is an integer-valued defect charge.
            In the absence of creation/annihilation events in the bulk, this current is
            conserved in the distributional sense,
            \begin{equation}
                \nabla_\mu j^\mu_{\mathrm{ev}}(x)=0,
            \end{equation}
            and the associated event count through a detector world-tube $\Sigma$ is
            \begin{equation}
                N_{\mathrm{ev}}(\Sigma)\equiv\int_{\Sigma}\dd\Sigma_\mu\, j^\mu_{\mathrm{ev}}.
            \end{equation}
            Operationally, $N_{\mathrm{ev}}(\Sigma)$ counts (with multiplicity $q_k$) the
            number of defect worldlines intersecting the detector region.

        \subsection{Continuum clock field from coarse-graining}

            At scales large compared to a coarse-graining length $\ell$, define a smooth
            clock field $\chi(x)$ by
            \begin{equation}
                \partial_\mu \chi(x)
                \;\equiv\;
                \frac{1}{\nu_0}\,\big(j_{\mathrm{ev},\mu}(x)\big)_\ell,
                \label{eq:chiFromEvents}
            \end{equation}
            where $(\cdot)_\ell$ denotes coarse-graining and $\nu_0$ is a conversion factor.
            This yields the continuum-clock (IR) description of the event-count (UV) clock.

        \subsection{Foliation interpretation}

            If the macroscopic medium admits a hypersurface-orthogonal timelike congruence
            $u_\mu$ (a foliation), then one may parametrize it by a scalar potential $\chi$
            via
            \begin{equation}
                u_\mu \;=\;\frac{\nabla_\mu \chi}{\sqrt{g^{\alpha\beta}\nabla_\alpha \chi\,\nabla_\beta \chi}}.
                \label{eq:uFromChi}
            \end{equation}
            In this interpretation, the TOA observable $p(\Theta)$ defined in
            Eq.~\eqref{eq:TOA} becomes the distribution of detector flux binned by the
            physical clock reading $\chi$ on $\Sigma$.

        \subsection{Interpretational note (optional)}

            In SST language one may view the localized defects as ``topological matter
            excitations'' in the effective medium. The integer charges $q_k$ can then be
            interpreted as quantized defect labels. Classical hydrodynamic analogies (e.g.,
            linking/helicity-based interpretations) are deferred to Discussion to avoid
            placing additional structure into the formal construction.


% ======================================================================
    \section{Discussion and Outlook (Analogy Only)}
% ======================================================================

    \paragraph{What the event current represents (and what it does not).}
        In the main text, the event current $j^\mu_{\mathrm{ev}}$ is introduced as an (approximately) conserved quantity whose integral counts \emph{localized, robust events} relevant to a detection process. At this level of generality it is deliberately agnostic: the ``events'' might be particle crossings of a detector region, interaction vertices effectively registered by an apparatus, defect/excitation worldlines intersecting $\Sigma$, or other countable processes whose total number is stable under smooth deformations of $\Sigma$ that keep the boundary fixed.

        This is the only structural assumption the formalism needs: that on the scales of interest the relevant events can be summarized by a current with $\nabla_\mu j^\mu_{\mathrm{ev}}\approx 0$ (exactly or with controlled violations). Section~\ref{sec:SSTrealization} provides one concrete realization in terms of defect worldlines in a hydrodynamic medium, but the relational TOA definition itself does not depend on adopting that ontology. Put differently, the construction is compatible with many microscopic stories; it only requires that they reduce to an effective event-count description on appropriate scales.

        \medskip
        \noindent
        \textbf{Classical analogy (optional intuition, not an assumption).}
        The comparison with helicity in ideal fluid dynamics is meant as a guide to intuition rather than a derivation. In fluids, conserved integrals can often be understood geometrically in terms of linking and knotting of flow structures \cite{Moffatt1969Helicity,ArnoldKhesin1998,Saffman1992}. By analogy, one may picture $j^\mu_{\mathrm{ev}}$ as a flux of ``crossing events'' that are topologically protected or otherwise robust, so that a discrete count becomes a smooth clock field under coarse-graining. None of the results above require this picture; it merely suggests one way in which a conserved event current might arise in an effective description.

        \medskip
        \noindent
        \textbf{Predictive content and where it enters.}
        Once a clock sector is specified, its correlators $C(\Delta t,\Delta x)$ become physical input rather than interpretational decoration. In the simplest (Gaussian) setting they determine the amount of arrival-time broadening through local clock variance, while massive clock correlations also supply an exponential envelope for contributions that would require correlations across a spacelike separation (Appendix~\ref{app:prop}). At sufficiently fine time resolution, the ultraviolet event-count description can additionally produce discrete/renewal features that are washed out in the infrared after coarse-graining.

    \subsection*{Idealized measurement scenarios}

    A minimal scenario is a relativistic wavepacket experiment in which the detector couples (directly or effectively) to a clock degree of freedom with a finite correlation time. The observable consequence is that the TOA statistics need not coincide with those obtained from the matter wavepacket alone. In the Gaussian clock limit discussed in Sec.~6, the predicted broadening
    \begin{equation}
        \mathrm{Var}(\Theta)=\sigma_t^2+\sigma_\tau^2
    \end{equation}
    would appear as an excess arrival-time spread beyond the semiclassical flux width $\sigma_t$.

    From a practical standpoint, the framework is well-suited to numerical study. Given a flux profile $\mathcal{F}(t)$ and a model (or measurement-informed ansatz) for the clock correlator $C(\Delta t,\Delta x)$, the TOA density follows directly from the convolution formula, Eq.~\eqref{eq:convolution}. This makes the clock-sector parameters (e.g.\ $c_\tau$ and $\mu_\tau$ in the quadratic EFT) quantitatively testable in toy models and in lattice or continuum simulations.

% ======================================================================
    \section{Conclusion}
% ======================================================================

    We defined time-of-arrival (TOA) as a \emph{relational}, manifestly covariant observable: detector crossings are binned neither by an externally supplied time coordinate nor by postulating a universal time operator, but by a \emph{physical clock} evaluated on the detector world-tube $\Sigma$, within the class of clocks that admit a (conserved or approximately conserved) event current. The construction uses two ingredients:

    \smallskip
    \noindent
    (i) a conserved matter current $J^\mu$ whose contraction with the unit normal defines the detector flux $\mathcal{F}=n_\mu J^\mu$, and

    \noindent
    (ii) an event current $j^\mu_{\mathrm{ev}}$ whose integral defines a discrete event-count clock.

    \smallskip
    A controlled coarse-graining over a scale $\ell$ then yields an infrared clock field $T(x)$ through
    \begin{equation}
        \partial_\mu T(x)=\frac{1}{\nu_0}\big(j_{\mathrm{ev},\mu}(x)\big)_\ell,
    \end{equation}
    so that discrete counts and continuum time arise as UV and IR descriptions of the same underlying structure.

    The resulting TOA density,
    \begin{equation}
        p(\Theta)=
        \frac{1}{\mathcal{N}}
        \int_{\Sigma}\dd^3\sigma\,\sqrt{|h|}\,
        \mathcal{F}(\sigma)\,
        \delta\!\big(T_\Sigma(\sigma)-\Theta\big),
    \end{equation}
    depends on detector microphysics only through the geometry of $\Sigma$ and the conserved flux $J^\mu$. Because the definition never requires a self-adjoint operator conjugate to the Hamiltonian, it sidesteps Pauli-type obstructions at the outset. In the semiclassical regime it reproduces the usual time-of-flight notion.

    In the worked $1+1$D example, a narrowband incoming wavepacket together with a stationary Gaussian clock sector reduces $p(\Theta)$ to a convolution of the semiclassical flux profile with the local clock readout distribution, producing a concrete prediction: arrival-time broadening
    \begin{equation}
        \mathrm{Var}(\Theta)=\sigma_t^2+\sigma_\tau^2,
        \qquad
        \sigma_\tau^2=\langle\tau^2\rangle,
    \end{equation}
    with $\sigma_\tau^2$ fixed by the clock sector. Beyond this local effect, Appendix~\ref{app:prop} shows that massive clock correlators are exponentially suppressed at spacelike separations with envelope scale $\mu_\tau^{-1}$ (up to algebraic prefactors), providing a controlled bound on early-arrival contributions without implying superluminal signaling.

    Relative to mainstream TOA formalisms, the point here is not that detector models or POVMs fail operationally, but that the relational definition isolates a covariant field-functional whose dependence on measurement details is pushed into explicit, modelable structures: conserved currents and a clock sector with its own dynamics. This reframes detector dependence as a UV$\rightarrow$IR matching problem—event counts to continuum time—making TOA a covariant, predictive field observable.

    \subsection*{Next steps}
        Immediate extensions include: (i) a $3+1$D worked example for finite-size detectors to quantify geometric dependence; (ii) non-Gaussian clock dynamics beyond the quadratic EFT to predict higher-cumulant signatures in $p(\Theta)$; and (iii) a systematic treatment of renewal-type UV corrections at temporal resolution $\Delta t\lesssim \ell$ and their crossover to the continuum limit. In particular, a numerical study of Eq.~\eqref{eq:convolution} as $\mu_\tau$ is varied would directly probe the predicted exponential suppression envelope and the crossover from discrete event-count statistics to continuum clock time.

% ======================================================================
        \appendix
    \section*{Appendix}
% ======================================================================
    \section{Exponential Suppression from the Clock Propagator}
        \label{app:prop}
% ======================================================================

        This appendix derives the exponential suppression scale governing \emph{acausal/spacelike}
        clock correlations and the resulting exponential \emph{envelope} on early-arrival
        contributions. The derivation uses standard massive-scalar correlators and Bessel
        function asymptotics \cite{Peskin1995,Weinberg1995,GradshteynRyzhik}.

        \subsection{Clock correlators from the quadratic EFT}

            Consider the quadratic clock action in flat $1+1$D with $c_\tau=1$:
            \begin{equation}
                S_\tau=\frac12\int \dd^2x\left[(\partial_t\tau)^2-(\partial_x\tau)^2-\mu_{\tau}^{\,2}\tau^2\right].
            \end{equation}
            The Wightman function $\Gplus(x)$ for a free massive scalar is a Lorentz-invariant
            function of $s^2=(\Delta t)^2-(\Delta x)^2$:
            \begin{equation}
                \Gplus(\Delta t,\Delta x)=\langle \tau(t,x)\tau(t+\Delta t,x+\Delta x)\rangle.
            \end{equation}

            For spacelike separations $(\Delta x)^2-(\Delta t)^2>0$, standard results give
            (see, e.g., representations in \cite{Weinberg1995} and integral tables):
            \begin{equation}
                \Gplus(\Delta t,\Delta x)
                =\frac{1}{2\pi}\Kzero\!\left(\mu_\tau\,\sqrt{(\Delta x)^2-(\Delta t)^2}\right)
                \quad\text{(spacelike)}.
                \label{eq:WightmanK0}
            \end{equation}
            While conventions vary by factors and $i0^+$ prescriptions, the key fact is that the
            spacelike decay is controlled by $\Kzero$.

        \subsection{Bessel asymptotics and exponential decay}

            For large argument $z\gg 1$, the modified Bessel function satisfies
            \begin{equation}
                \Kzero(z)\sim \sqrt{\frac{\pi}{2z}}\,e^{-z}\left[1+\cO\!\left(\frac{1}{z}\right)\right].
                \label{eq:K0asym}
            \end{equation}
            Therefore, for spacelike separations with invariant distance
            \begin{equation}
                \rho \equiv \sqrt{(\Delta x)^2-(\Delta t)^2}\gg \mu_\tau^{-1},
            \end{equation}
            Eq.~\eqref{eq:WightmanK0} yields the exponential suppression
            \begin{equation}
                \Gplus(\Delta t,\Delta x)\;\propto\;
                \frac{e^{-\mu_\tau \rho}}{\sqrt{\mu_\tau \rho}}\,
                \left[1+\cO\!\left((\mu_\tau\rho)^{-1}\right)\right].
                \label{eq:spacelikeExp}
            \end{equation}

            \begin{proposition}[Acausal envelope scale]
                Any TOA contribution requiring clock correlations between detector readout and
                source data across a spacelike interval $\rho$ is exponentially suppressed with
                envelope scale $\mu_\tau^{-1}$ as in \eqref{eq:spacelikeExp}.
            \end{proposition}

            \begin{remark}
                This mechanism is directly analogous to standard QFT statements: massive fields do not
                transmit strong correlations outside the light cone; the residual correlations decay as
                $e^{-\mu \rho}$ (up to power-law prefactors). This is a correlator statement and does
                not imply superluminal signaling.
            \end{remark}

        \subsection{Implication for early-arrival probabilities}

            To connect to TOA, consider a source preparing a wavepacket at $(t_0,x_0)$ and a detector
            at $(t,x_D)$. ``Early arrival'' relative to a classical causal time-of-flight means that
            the detector registers significant probability at $(t,x_D)$ even when, under the
            semiclassical causal picture, the relevant support would require correlations across a
            spacelike interval. In any model where such early-arrival probability is controlled by
            a clock-sector correlator linking those regions, Eq.~\eqref{eq:spacelikeExp} implies an
            exponential bound of the form
            \begin{equation}
                p_{\mathrm{early}}(\Delta)\;\lesssim\;A(\Delta)\,\exp\!\big(-\mu_\tau\,\rho(\Delta)\big),
                \label{eq:earlyBound}
            \end{equation}
            where $\Delta$ parameterizes the ``degree of earliness'' and $\rho(\Delta)$ is the
            corresponding spacelike separation required by that early-arrival event, with prefactor
            $A(\Delta)$ at most polynomial in $\rho$ for large $\rho$.

            This yields a controlled, parameter-fixed exponential envelope governed by $\mu_\tau$.

        \subsection{Retarded Green's function and locality}

            For completeness, the retarded propagator solves
            \begin{equation}
            (\Box+\mu_\tau^2)\,\Gret(x)=\delta^{(2)}(x),
            \qquad \Gret(x)=0\ \text{for}\ t<0.
            \end{equation}
            Local detector couplings to $\tau$ then inherit causal support from $\Gret$, while
            spacelike Wightman correlations remain exponentially suppressed as above. This
            distinguishes \emph{correlation tails} from \emph{signal propagation}, consistent with
            standard relativistic QFT \cite{Peskin1995,Weinberg1995}.

% ======================================================================

% =========================
% Back matter (FoP / Springer)
% =========================
        \backmatter

        \bmhead{Acknowledgements}
        Not applicable.

        \bmhead{Declarations}

    \subsection*{Funding}
        Not applicable.

    \subsection*{Competing interests}
        The author declares no competing interests.

    \subsection*{Ethics approval}
        Not applicable.

    \subsection*{Consent to participate}
        Not applicable.

    \subsection*{Consent for publication}
        Not applicable.

    \subsection*{Data availability}
        All data generated or analyzed during this study are included in this published article.

    \subsection*{Code availability}
        Not applicable.

    \subsection*{Author contributions}
        O.I. conceived the study, developed the analysis, performed the derivations, and wrote the manuscript.

% =========================
% References
% =========================
    \begin{thebibliography}{99}

        \bibitem{Pauli1980}
        W.~Pauli,
        \textit{General Principles of Quantum Mechanics},
        Springer (1980).

        \bibitem{BuschGiannitrapani1995}
        P.~Busch and G.~Giannitrapani,
        The time--energy uncertainty relation,
        \textit{Int.\ J.\ Theor.\ Phys.} \textbf{34}, 2149 (1995).
        DOI: 10.1007/BF00670728.

        \bibitem{Kijowski1974}
        J.~Kijowski,
        On the time operator in quantum mechanics and the Heisenberg uncertainty relation,
        \textit{Rep.\ Math.\ Phys.} \textbf{6}, 361 (1974).
        DOI: 10.1016/0034-4877(74)90051-5.

        \bibitem{Giannitrapani1997}
        G.~Giannitrapani,
        Positive-operator-valued measures for arrival times and time operators,
        \textit{Int.\ J.\ Theor.\ Phys.} \textbf{36}, 1575 (1997).
        DOI: 10.1007/BF02435766.

        \bibitem{ColosiOeckl2025TimeOperatorTOA}
        D.~Colosi and R.~Oeckl,
        A Time Operator and the Time-Of-Arrival Problem in Quantum Field Theory,
        \textit{Foundations of Physics} \textbf{55}, 56 (2025).
        DOI: 10.1007/s10701-025-00866-x.

        \bibitem{Peskin1995}
        M.~E.~Peskin and D.~V.~Schroeder (1995),
        \textit{An Introduction to Quantum Field Theory},
        Addison-Wesley.

        \bibitem{Weinberg1995}
        S.~Weinberg (1995),
        \textit{The Quantum Theory of Fields, Vol.\ I: Foundations},
        Cambridge University Press.
        DOI: 10.1017/CBO9781139644167.

        \bibitem{GradshteynRyzhik}
        I.~S.~Gradshteyn and I.~M.~Ryzhik,
        \textit{Table of Integrals, Series, and Products},
        7th ed., Academic Press (2007).
        (Asymptotics and identities for modified Bessel functions.)

        \bibitem{Moffatt1969Helicity}
        H.~K.~Moffatt,
        The degree of knottedness of tangled vortex lines,
        \textit{J.\ Fluid Mech.} \textbf{35}, 117 (1969).
        DOI: 10.1017/S0022112069000991.

        \bibitem{ArnoldKhesin1998}
        V.~I.~Arnold and B.~A.~Khesin,
        \textit{Topological Methods in Hydrodynamics},
        Springer (1998).
        DOI: 10.1007/978-1-4612-0639-0.

        \bibitem{Saffman1992}
        P.~G.~Saffman,
        \textit{Vortex Dynamics},
        Cambridge University Press (1992).
        DOI: 10.1017/CBO9780511624063.

        \bibitem{GambiniPullin2021}
        R.~Gambini and J.~Pullin,
        Relational time and quantum clocks,
        \textit{Phys.\ Rev.\ D} \textbf{104}, 025003 (2021).
        DOI: 10.1103/PhysRevD.104.025003.

        \bibitem{HohnSmith2020}
        P.~A.~H{\"o}hn and A.~R.~H.~Smith,
        Quantum reference frames and time,
        \textit{Phys.\ Rev.\ D} \textbf{101}, 125011 (2020).
        DOI: 10.1103/PhysRevD.101.125011.

        \bibitem{Giovannetti2015}
        V.~Giovannetti, S.~Lloyd, and L.~Maccone,
        Quantum time,
        \textit{Nature Physics} \textbf{11}, 1 (2015).
        DOI: 10.1038/nphys3527.

    \end{thebibliography}

\end{document}