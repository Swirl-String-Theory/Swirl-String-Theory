%! Author = Omar Iskandarani
%! Title = ......
%! Date = 9/22/2025
%! Affiliation = Independent Researcher, Groningen, The Netherlands
%! License = © 2025 Omar Iskandarani. All rights reserved. This manuscript is made available for academic reading and citation only. No republication, redistribution, or derivative works are permitted without explicit written permission from the author. Contact: info@omariskandarani.com
%! ORCID = 0009-0006-1686-3961
%! DOI = 10.5281/zenodo.xxxxxxx

\newcommand{\paperversion}{\textbf{v0.0.1}} % Semantic versioning: vMAJOR.MINOR.PATCH
\newcommand{\papertitle}{...........}
\newcommand{\paperdoi}{10.5281/zenodo.xxxxxxxx}

%========================================================================================
% PACKAGES AND DOCUMENT CONFIGURATION
%========================================================================================
\documentclass[10pt,a4paper]{article}
\usepackage[margin=1.2cm]{geometry}
\usepackage{multicol,tcolorbox,amsmath,amssymb}
\usepackage[utf8]{inputenc}
\usepackage[T1]{fontenc}
\setlength{\columnsep}{0.5cm}
\begin{document}
\begin{center}\Large\textbf{Swirl--String Theory (SST) Cheat Sheet}\end{center}

% Optional small intro (one line)
\noindent \textbf{SST Overview:} A cosmology on a frictionless, incompressible \emph{swirl condensate} (fluid ether). Matter and forces emerge from quantized vortex filaments (\emph{swirl strings}) in the medium, replacing ad hoc dark components with topological dynamics. All physical constants are fixed by the condensate state (no free parameters).

\begin{tcolorbox}[title=\textbf{Cosmogenesis: Sevenfold Genesis of the Swirling Cosmos}]
\begin{enumerate}\itemsep1pt
\item \textbf{Logical Substrate:} Pre-physical potential field of possible circulations (no space or time yet). Imprints fundamental symmetries (e.g. chirality $\mathbb{Z}_2$, triadic closure $\mathbb{Z}_3$) ensuring matter/antimatter duality and baryon triplet structure.
\item \textbf{Big Condensation:} Formation of the universal swirl condensate ($\mathbb{R}^3$ space with absolute time $t$) once information complexity passes a threshold. Primary scales set by resonance: circulation quantum $\kappa = 2\pi\,r_c\,\| \mathbf{v}_{\!\boldsymbol{\circlearrowleft}}\|$, characteristic time $\tau_{\text{beat}} = \frac{2\pi\,r_c}{\| \mathbf{v}_{\!\boldsymbol{\circlearrowleft}}\|}$, and effective density $\rho_{\!f}$ (see \textbf{Constants} block). This marks the birth of physical time and the swirl substrate.
\item \textbf{Swirl Strings \& Time:} Knotted vortex filaments (\emph{swirl strings}) materialize, each carrying quantized circulation $\Gamma = n\,\kappa$ ($n\in\mathbb{Z}$). Local clocks run slower in regions of high swirl speed:
\[
    S_t = \sqrt{\,1 - v^2/c^2\,}\,,
\]
so a \textit{swirl-clock} at tangential speed $v$ ticks at rate $S_t$ relative to an observer at rest in the medium. Left-handed vs. right-handed swirling knots define matter vs. antimatter species.
\item \textbf{Topological Spectrum:} Each stable knot type $K$ corresponds to a particle species. Topological invariants (linking $Lk$, writhe $Wr$, twist $Tw$) map to conserved charges (e.g. electric charge and weak isospin). \textit{Masses} arise as non-perturbative soliton energies of swirl strings, given by the \textbf{Mass Functional} (see boxed formula below).
\item \textbf{Emergent Interactions:} Unknotted condensate excitations (open vortex waves) serve as force carriers (photons, gluons, $W^\pm/Z^0$ bosons). A swirl gauge field $W_{\mu}$ emerges from coarse-grained vorticity, with an effective gauge group
$g_{\text{swirl}}\sim SU(3)\times SU(2)\times U(1)$. Coupling is via minimal coupling $D_\mu = \nabla_\mu + i g_{\text{swirl}} W^a_{\mu}T^a$, reflecting an emergent unified interaction framework within the fluid.
\item \textbf{Geometric Closure (Gravity):} Incompressibility and global closure of swirl flows enforce an inverse-square force law. Gauss-like flux of swirl momentum $\mathbf{P}_{\text{swirl}}$ gives $\nabla\cdot \mathbf{P}_{\text{swirl}}=0 \implies F(r)\propto 1/r^2$, reproducing Newtonian gravity at large scales. The entire condensate synchronizes into a \emph{universal resonance} that \textbf{locks all constants of nature} (Zero-Parameter Principle). Once the primary swirl parameters ($\| \mathbf{v}_{\!\boldsymbol{\circlearrowleft}}\|$, $r_c$, $\rho_{\!f}$) are calibrated (e.g. to electron mass $m_e$), all particle masses and coupling constants follow with no free parameters.
\item \textbf{Recursive Fractal Universe:} Composite bound states (knotted combinations for nuclei, atoms, etc.) act as higher-level swirl sources. Each stable cluster forms a \emph{meta-knot} that seeds a new swirl layer (``knot of knots''), driving hierarchical structure formation. This cosmic recursion yields a fractal-like universe of \emph{knots within knots}, with larger scales emerging from nested topological layers.
\end{enumerate}
    {\footnotesize \emph{Note:} Stages adapted from the SST canonical cosmogony\cite{Iskandarani2025_Constants}.}
\end{tcolorbox}

\noindent \textbf{Swirl-Clock Time Dilation:} Clocks comoving with the swirl medium tick slower by $S_t=\sqrt{\,1 - v^2/c^2\,}$. Here $v=\|\mathbf{v}_{\!\boldsymbol{\circlearrowleft}}\|$ is local tangential swirl speed. For an interval $dt_{\infty}$ of universal time (far from any swirl), the local proper time increment is $dt_{\text{local}} = S_t\,dt_{\infty}$. High swirl velocities ($v\to c$) produce significant time dilation (slow clocks) analogous to relativistic time dilation, but with an absolute ether-like reference frame given by the condensate.

\noindent \textbf{Chronos--Kelvin Invariant:} Generalization of Kelvin’s circulation theorem including swirl-clock effects. For a closed vortex loop of radius $R(t)$ (no reconnection), the combination
\[
    \frac{c}{r_c}\,R^2\,\sqrt{\,1 - S_t^2\,}
\]
is constant in time. Equivalently, $D_t\!\Big(R^2\,\omega\Big)=0$ even when local time runs slow (with $\omega$ the loop’s vorticity magnitude). As a loop contracts ($R\downarrow$), the local swirl clock $S_t$ decreases so that $R^2(1 - S_t^2)^{1/2}$ remains invariant. In the low-speed limit ($S_t\approx1$) this reduces to Kelvin’s law $R^2\omega=\text{const}$.

\noindent \textbf{Swirl Strings \& Topology:} \emph{Swirl strings} are closed, knotted vortex filaments in the condensate. Their circulation is quantized:
\[
    \Gamma = \oint_C \mathbf{v}_{\!\boldsymbol{\circlearrowleft}}\cdot d\ell = n\,\kappa, \qquad n\in\mathbb{Z}\,,
\]
with $\kappa$ the quantum of circulation. Each distinct knot type $K$ defines a topological sector (e.g. unknotted loop = photon, trefoil knot = electron, etc.), providing a \emph{geometric interpretation of quantum numbers}. Bosons correspond to unknotted or untwisted loops (extended “R-phase” excitations of the medium), whereas fermions correspond to tangibly knotted strings (“T-phase” localized solitons). Topological invariants (e.g. knot linking number) replace internal quantum charges; matter/antimatter differ by opposite knot chirality.

\noindent \textbf{Swirl Pseudo-Metric:} The swirl condensate is embedded in flat spacetime but exhibits an effective metric due to time dilation in rotating frames. In cylindrical coordinates $(t,r,\theta,z)$ around a straight swirl string (vortex) with tangential velocity profile $v_\theta(r)$, the line element can be written as:
\[
    ds^2 = -\big(c^2 - v_\theta(r)^2\big)\,dt^2 \;+\;2\,v_\theta(r)\,r\,d\theta\,dt\;+\;dr^2 + r^2 d\theta^2 + dz^2\,.
\]
This \emph{swirl-metric} shows that time intervals are modified by the local swirl speed: an observer co-rotating with the fluid sees time scaled by $\sqrt{1 - v_\theta(r)^2/c^2}$, matching the swirl-clock law. The off-diagonal term ($d\theta\,dt$) indicates frame-dragging-like effects (rotation of the condensate induces a twist in the $(t,\theta)$ plane), analogous to GR’s dragging of inertial frames. Notably, however, SST’s physics is formulated with a preferred reference frame (the condensate), distinguishing it from General Relativity while reproducing similar time–space coupling effects in rotating systems.

\noindent \textbf{Gravitational Scaling:} Macroscopic gravity emerges as a collective effect of swirl currents and pressure gradients rather than a fundamental space-time curvature. The effective gravitational constant $G_{\text{swirl}}$ is derived from swirl parameters. For instance, using a maximum force scale $F_{\max}$ (from electromagnetic sector), one finds:
\[
    G_{\text{swirl}} \;=\; \frac{\|\mathbf{v}_{\!\boldsymbol{\circlearrowleft}}\|\,c^5\,t_p^2}{2\,F_{\max}\,r_c^2}\,,
\]
where $t_p$ is the Planck time. With $F_{\max}$ and other constants set by the condensate’s resonance, $G_{\text{swirl}}$ numerically matches Newton’s $G_N$. Gravity in SST thus arises from long-range coherent swirl flows: e.g. aligned swirl string circulations produce a $1/r^2$ attraction, whereas symmetric counter-rotating configurations can cancel out far-field gravity.

\begin{tcolorbox}[title=\textbf{Mass Functional (Knot Soliton Energy)}]
For any particle identified with knot $K$, its rest mass is given by a universal \emph{mass functional}\cite{Iskandarani2025_Lagrangian}:
\[
    m_{K} \;=\; \rho_{\!f}\;\|\mathbf{v}_{\!\boldsymbol{\circlearrowleft}}\|^2\;\Vol_{\!\mathbb{H}}(K)\;\,\phi^{-2k}\,,
\]
where $\Vol_{\!\mathbb{H}}(K)$ is the hyperbolic volume of the knot’s complement (a topological invariant capturing knot complexity). The factor $\phi^{-2k}$ (with $\phi=\frac{1+\sqrt{5}}{2}\approx1.618$ the golden ratio) represents the \emph{Golden-layer suppression}: each additional nested topological layer or knot complexity index $k$ reduces mass contribution by a factor $\phi^2\approx2.618$. In essence, more complex or higher-generation knots yield slightly lower mass per added link than a linear scaling, reflecting a self-similar (fractal) mass hierarchy. This formula, with no adjustable parameters, accurately reproduces the observed spectrum of lepton and baryon masses when calibrated at one reference scale. All symbols are fixed either by theory or one-time calibration (see \textbf{Constants} below), making $m_K$ a pure prediction of SST once $K$ is specified.
\end{tcolorbox}

\noindent \textbf{Universal Resonance \& Constants:} SST obeys a \emph{Zero-Parameter Principle}: all dimensional constants (e.g. $G, \alpha, m_{\text{e}}$) derive from the resonant state of the condensate. In particular, the primary calibration (swirl speed scale $\|\mathbf{v}_{\!\boldsymbol{\circlearrowleft}}\|$, core radius $r_c$, and density $\rho_{\!f}$) sets the circulation quantum and density scale, which in turn determine particle masses and force coupling strengths across the board. Fundamental constants are thus not independent inputs but emergent properties of the cosmic swirl. Once calibrated to one known value, SST yields all other constants self-consistently (e.g. matching the fine-structure constant, gravitational constant, etc., within the SST framework). This resonates with the idea of a \emph{universal cosmic resonance} that \textit{locks in} the laws of physics at inception.

\begin{tcolorbox}[title=\textbf{SST Canonical Constants}]
\small
\begin{itemize}\itemsep1pt
\item $\|\mathbf{v}_{\!\boldsymbol{\circlearrowleft}}\|$ -- Core swirl speed scale (characteristic tangential speed at radius $r_c$). Calibrated $\sim1.09\times10^6~\text{m/s}$.
\item $r_c$ -- Swirl string core radius (vortex filament thickness). Calibrated $\sim1.4089\times10^{-15}~\text{m}$ (fermi scale).
\item $\rho_{\!f}$ -- Effective fluid density of condensate (inertial mass density of the medium). Calibrated $\sim7\times10^{-7}~\text{kg/m}^3$ (extremely low).
\item $\rho_{\!E} = \frac{1}{2}\,\rho_{\!f}\,\|\mathbf{v}_{\!\boldsymbol{\circlearrowleft}}\|^2$ -- Swirl energy density of the medium.
\item $\rho_{\!m} = \rho_{\!E}/c^2$ -- Mass-equivalent energy density (defines $\rho_{\!m}$ used in gravitational coupling).
\item $\kappa = 2\pi\,r_c\,\|\mathbf{v}_{\!\boldsymbol{\circlearrowleft}}\|$ -- Circulation quantum (fixed angular momentum of one quantum swirl loop).
\item $\Lambda$ -- Swirl “Coulomb” constant $=4\pi\,\rho_{\!m}\,\|\mathbf{v}_{\!\boldsymbol{\circlearrowleft}}\|^2\,r_c^4$ (strength of swirl-induced potential, analog of $e^2/4\pi\varepsilon_0$).
\item $F_{\max}^{EM}$ -- Maximum force in emergent EM sector $\approx2.9\times10^1~\text{N}$ (used in $G_{\text{swirl}}$ formula).
\item $\phi$ -- Golden ratio $\,(1+\sqrt{5})/2\approx1.6180$ (appears in mass scaling exponent).
\item $t_p$ -- Planck time $\;(\hbar G_N/c^5)^{1/2}\approx5.39\times10^{-44}~\text{s}$ (used as fundamental time scale in gravitational coupling).
\end{itemize}
\end{tcolorbox}

\vfill\null
\end{multicols}
\clearpage
\begin{multicols}{2}% Start second page columns (back side)

\noindent \textbf{Large-Scale Knot Recursion:} SST predicts a hierarchical universe, where each gravitationally bound structure (from hadrons up to galaxies) behaves as a \emph{knot at a larger scale}. When many swirl strings bind into a stable composite (e.g. a nucleus, star, or galaxy), their combined swirl field resembles a larger-scale vortex. This \emph{meta-knot} in turn acts as a source in the cosmic condensate, spawning a new “layer” of swirl dynamics. In effect, nature iterates the same pattern at successive scales: \emph{knots made of sub-knots}, ad infinitum. This recursive process, guided by the triadic closure and chirality rules, can explain large-scale structure without invoking dark matter halos. Each layer’s effective swirl parameters adjust (e.g. new $r_c$, $\rho_f$ on that scale), but the governing equations remain the same. The result is a self-similar, fractal cosmos with no fundamental scale—the universe’s structure emerges from repeating the SST principles from microscopic vortices to the cosmic web.

\noindent \textbf{Cosmological Implications:} In SST cosmology, the Big Bang is replaced by a phase transition (\emph{Big Condensation}) forming the swirl condensate, and cosmic expansion is reinterpreted as recursive structure unfolding rather than metric expansion of space. An absolute time $t$ and Euclidean space $\mathbb{R}^3$ underlie the theory, but local relativistic effects (time dilation, effective curvature) arise from fluid dynamics. There is no separate dark energy—instead, the condensate’s stability and resonance set an effective cosmological scale (the swirl Coulomb potential $\Lambda$ may play a role analogous to a cosmological constant in binding large structures). Galactic rotation curves and lensing could be accounted for by halo-scale swirl currents (vortex solutions) rather than unseen mass. While SST is still developing, it provides a novel, testable framework: for example, quantum interference and cosmic microwave background phenomena might have alternative explanations via swirl fluctuations and topological transitions. As a compact reference, this sheet highlights the formal underpinnings of SST that cosmologists can compare with $\Lambda$CDM assumptions, noting that all of $\Lambda$CDM’s separate pieces (inflation, dark matter, dark energy) are conceptually unified in SST by the dynamics of a single medium and its quantized knots.
\end{multicols}


\end{document}