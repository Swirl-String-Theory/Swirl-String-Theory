%! Author = Omar Iskandarani
%! Title = ......
%! Date = 9/20/2025
%! Affiliation = Independent Researcher, Groningen, The Netherlands
%! License = © 2025 Omar Iskandarani. All rights reserved. This manuscript is made available for academic reading and citation only. No republication, redistribution, or derivative works are permitted without explicit written permission from the author. Contact: info@omariskandarani.com
%! ORCID = 0009-0006-1686-3961
%! DOI = 10.5281/zenodo.xxxxxxx

\newcommand{\paperversion}{\textbf{v0.0.1}} % Semantic versioning: vMAJOR.MINOR.PATCH
\newcommand{\papertitle}{...........}
\newcommand{\paperdoi}{10.5281/zenodo.xxxxxxxx}

%========================================================================================
% PACKAGES AND DOCUMENT CONFIGURATION
%========================================================================================
\documentclass[floatfix,aps,onecolumn,nofootinbib]{revtex4-2}

\usepackage{amsmath,amssymb,amsfonts}
\usepackage{bm}
\usepackage{physics}
\usepackage{microtype}
\usepackage{tcolorbox}
\usepackage{hyperref}
\hypersetup{colorlinks=true,linkcolor=blue,citecolor=blue,urlcolor=blue}

\usepackage{graphicx}
\usepackage{float}
\usepackage{xcolor}
\usepackage{enumitem}
\usepackage{caption}
\usepackage{mathtools}

\usepackage{tikz}
\usetikzlibrary{knots,intersections,decorations.pathreplacing}
\usetikzlibrary{3d, calc, arrows.meta, positioning}
\usetikzlibrary{decorations.pathmorphing}

\usepackage{pgfmath, pgfplots}
\pgfplotsset{compat=1.18} % or version you have
\usepackage{titlesec}
\usepackage{ulem}

\usepackage[T1]{fontenc}
\usepackage{lmodern}
\usepackage{booktabs}
\usepackage[utf8]{inputenc}



% ==== Swirl String Theory (SST) macros ====
% Context-aware subscript symbol; uses math styles, not \scriptsize
\newcommand{\swirlarrow}{%
    \mathchoice{\mkern-2mu\scriptstyle\boldsymbol{\circlearrowleft}}%
    {\mkern-2mu\scriptstyle\boldsymbol{\circlearrowleft}}%
    {\mkern-2mu\scriptscriptstyle\boldsymbol{\circlearrowleft}}%
    {\mkern-2mu\scriptscriptstyle\boldsymbol{\circlearrowleft}}%
}
\newcommand{\swirlarrowcw}{%
    \mathchoice{\mkern-2mu\scriptstyle\boldsymbol{\circlearrowright}}%
    {\mkern-2mu\scriptstyle\boldsymbol{\circlearrowright}}%
    {\mkern-2mu\scriptscriptstyle\boldsymbol{\circlearrowright}}%
    {\mkern-2mu\scriptscriptstyle\boldsymbol{\circlearrowright}}%
}

% Canonical symbols
\newcommand{\vswirl}{\mathbf{v}_{\swirlarrow}}
\newcommand{\vswirlcw}{\mathbf{v}_{\swirlarrowcw}}
\newcommand{\SwirlClock}{S_{(t)}^{\swirlarrow}}
\newcommand{\SwirlClockcw}{S_{(t)}^{\swirlarrowcw}}
\newcommand{\omegas}{\boldsymbol{\omega}_{\swirlarrow}}  % swirl vorticity
\newcommand{\vscore}{v_{\swirlarrow}}                    % shorthand: |v_swirl| at r=r_c
\newcommand{\vnorm}{\lVert \vswirl \rVert}               % swirl speed magnitude
\newcommand{\rhoF}{\rho_{\!f}}                           % effective fluid density
\newcommand{\rhoE}{\rho_{\!E}}                           % swirl energy density /c^2? (we define clearly below)
\newcommand{\rhom}{\rho_{\!m}}                           % mass-equivalent density
\newcommand{\rc}{r_c}                                    % string core radius (swirl string radius)
\newcommand{\FmaxEM}{F_{\mathrm{EM}}^{\max}}             % EM-like maximal force scale
\newcommand{\FmaxG}{F_{\mathrm{G}}^{\max}}               % G-like maximal force scale
\newcommand{\Lam}{\Lambda_{\swirlarrow}}                               % Swirl Coulomb constant
\newcommand{\Om}{\Omega_{\swirlarrow}}                   % swirl angular frequency profile
\newcommand{\alpg}{\alpha_g}                             % gravitational fine-structure analogue


\begin{document}
    \title{\papertitle}
    \author{Omar Iskandarani}
    \thanks{Independent Researcher, Groningen, The Netherlands\\
    info@omariskandarani.com \\
    ORCID: \href{https://orcid.org/0009-0006-1686-3961}{0009-0006-1686-3961} \\
    DOI: \href{https://doi.org/\paperdoi}{\paperdoi} \\ 
    version: \paperversion}
    \date{\today}

    \begin{abstract}
        \vspace*{-0.5em}
        \section*{\centering Abstract}
        \vspace*{-1em}


✨ \textit{“Wist je dat ongeveer 85\% van het universum onzichtbaar lijkt te zijn? Astronomen noemen dit dark matter, een mysterieuze substantie die sterrenstelsels bij elkaar zou houden. Maar misschien bestaat die ‘donkere materie’ helemaal niet. Mijn onderzoek suggereert dat zwaartekracht zelf uit microscopische draaikolken in deeltjes voortkomt. Stel je voor dat elk proton en neutron kleine knopen van onzichtbare draaikolken zijn, zoals wervelingen in een rivier. Hun verborgen draai vormt de lijm die massa aantrekt — zonder dat we een onbekende materie hoeven te veronderstellen. Misschien is dark matter dus geen nieuwe substantie, maar een effect van structuren die we al kennen, diep in de bouwstenen van materie.”}

    \begin{abstract}


		\vspace{1ex}\noindent
		\emph{Keywords:} SST
    \end{abstract}
    \maketitle


\subsection*{🧭 Korte uitleg (Chronos vs Kairos)}

In de klassieke Griekse filosofie:

\begin{itemize}
    \item \textbf{Chronos}: de meetbare, lineaire tijd. ``De klok die tikt.'' In jouw Canon vertaalt dat zich naar $\tau$ (proper time) of absolute tijd $N$.
    \item \textbf{Kairos}: het ``geschikte moment'' of een kwalitatieve sprong in tijd. In dynamische systemen wordt dit vaak opgevat als een bifurcatiepunt waar de gewone (Chronos) evolutie tijdelijk onderbroken wordt door een topologische sprong.
\end{itemize}

In SST betekent dit: de Swirl Clock $S_t^{\circlearrowleft}$ kan soms niet meer vloeiend evolueren, maar ervaart een \textit{fase-breuk} --- bijvoorbeeld wanneer $|\omega|^2$ een kritieke waarde overschrijdt. Dat moment is een Kairos.


            \subsection{Kairos-bifurcaties in Swirl Time (\emph{Research})}
                \label{sec:kairos}

                \paragraph{Claim.}
                    Naast de continue voortgang van Chronos-tijd $\tau$ en de cyclische Swirl Clock $\SwirlClock$ bestaan kritieke drempels (\emph{Kairos momenten}) waar de tijdsevolutie een bifurcatie/sprong vertoont.

                \paragraph{Mapping (Rosetta).}
                    \emph{Chronos} $\to$ lokale proper time $\tau$ (en absolute tijd $N$);
                    \emph{Kairos} $\to$ topologische fase-sprong in $\SwirlClock$ wanneer een kritische swirl-excitatiedrempel wordt overschreden.
                    Alle grootheden in SST-notatie: $\rhoF$, $\rc$, $\SwirlClock$.

                \paragraph{Kernvergelijking (dimensioneel consistent).}
                    We verankeren de karakteristieke draaisnelheid aan de kwantumschaal via
                    \[
                        \omega \;=\; \alpha\,\omega_C,
                        \qquad
                        \omega_C \;=\; \frac{m_e c^2}{\hbar},
                    \]
                    en postuleren de Kairos-drempel als
                    \begin{equation}
                    \boxed{\;\;\omega^2 \;\gtrsim\; \frac{c^2}{\rc^2}\;}\,.
                    \label{eq:kairos-threshold}
                    \end{equation}

                \paragraph{Schwarzian-correctie in de tijdsactie.}
                    De effectieve evolutie van de lokale tijd nemen we als
                    \begin{equation}
                    \frac{d\tau}{dN}
                    \;=\;
                    \sqrt{1-\frac{\vnorm^2}{c^2}}
                    \;+\;
                    \varepsilon\,\{\SwirlClock,\,N\},
                    \qquad
                    \{\SwirlClock,\,N\}
                    =
                    \frac{\SwirlClock''' }{\SwirlClock'}-\frac{3}{2}\!\left(\frac{\SwirlClock''}{\SwirlClock'}\right)^{\!2},
                    \label{eq:schwarzian}
                    \end{equation}
                    waar de Schwarzian-term de niet-lineaire gevoeligheid vangt die, nabij \eqref{eq:kairos-threshold}, een sprong in de fase van $\SwirlClock$ kan induceren.

                \paragraph{Mini-numeriek voorbeeld (met Canon-constanten).}
                    Met $c=2.9979\times10^8\,\mathrm{m/s}$, $\hbar=1.0546\times10^{-34}\,\mathrm{J\,s}$,
                    $m_e=9.1094\times10^{-31}\,\mathrm{kg}$, $\alpha=7.297\times10^{-3}$ en $\rc=1.40897\times10^{-15}\,\mathrm{m}$:
                    \[
                        \omega_C \approx 7.76\times10^{20}\ \mathrm{s^{-1}},\quad
                        \omega=\alpha\omega_C \approx 5.67\times10^{18}\ \mathrm{s^{-1}},
                    \]
                    \[
                        \omega^2 \approx 3.21\times10^{37}\ \mathrm{s^{-2}},
                        \qquad
                        \frac{c^2}{\rc^2}\approx 4.53\times10^{46}\ \mathrm{s^{-2}},
                        \qquad
                        \frac{\omega^2}{c^2/\rc^2}\approx 7.1\times10^{-10}.
                    \]
                    Dus zonder aanvullende mechanismen wordt de drempel niet gehaald (theoretische grens \emph{in situ}).

                \paragraph{Verzachtingslemma’s (bereikbaarheid van Kairos).}
                    \begin{itemize}
                    \item \textbf{Lemma A (Fractale versterking, link met $D_{\mathrm{swirl}}$).}
                    Voor multischalige coherentie vervangt
                    $\displaystyle \frac{c^2}{\rc^2}\to \frac{c^2}{\rc^2}\!\left(\frac{\rc}{r_{\mathrm{eff}}}\right)^{3-D_{\mathrm{swirl}}}$,
                    met $2.6\!\lesssim\!D_{\mathrm{swirl}}\!\lesssim\!2.9$ en $r_{\mathrm{eff}}\!>\!\rc$,
                    waardoor de drempel effectief daalt.

                    \item \textbf{Lemma B (Knooppakket-coherentie).}
                    Voor $n$ fase-gelockte knopen:
                    $\displaystyle \omega_{\mathrm{eff}}^2 \;\simeq\; n\,\xi(n)\,\omega^2$,
                    met $\xi(n)=1-\beta\log n$ (jouw Canon-coherentiesuppressie).
                    Voor matige $n$ (\emph{meso}-coherentie) kan $\omega_{\mathrm{eff}}^2$ de drempel kruisen.

                    \item \textbf{Lemma C (Resonante pomp via Schwarzian).}
                    In \eqref{eq:schwarzian} kan $\varepsilon$ lokaal toenemen bij fase-locking ($\SwirlClock'$ groot, $\SwirlClock''$ klein),
                    waardoor de effectieve drempel tijdelijk verlaagt en een Kairos-sprong triggert.
                    \end{itemize}

                \paragraph{Falsificeerders \& minimale proef.}
                    \emph{Falsificeer} door afwezigheid van elke niet-analytische fase-sprong in $\SwirlClock$ onder gecontroleerde resonante pomp (BEC/fluïde analoog) bij parameters die de lemmas voorzien.
                    \emph{Minimale proef}: ring-condensaat met aangedreven knoopconfiguratie;
                    scan pompsterkte (controle op $\varepsilon$) en $n$ (koppeling);
                    zoek hysterese/sprongen in frequentie-lock van $\SwirlClock$.


% Document
\begin{document}




\bibliographystyle{unsrt}
\bibliography{reference}
\end{document}