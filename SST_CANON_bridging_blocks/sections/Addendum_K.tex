%! Author = Omar Iskandarani
%! Title = ......
%! Date = .....
%! Affiliation = Independent Researcher, Groningen, The Netherlands
%! License = © 2025 Omar Iskandarani. All rights reserved. This manuscript is made available for academic reading and citation only. No republication, redistribution, or derivative works are permitted without explicit written permission from the author. Contact: info@omariskandarani.com
%! ORCID = 0009-0006-1686-3961
%! DOI = 10.5281/zenodo.xxxxxxx

% =====================================================================
% SST Canon Addendum — Bridging Blocks
% Version tag: v0.3.1+2025-08-27 (conforms to Canon v0.3.1)
% Persona: Bridging (does not modify Core Postulates)
% =====================================================================

% === Metadata ===
\newcommand{\papertitle}{....}
\newcommand{\paperdoi}{10.5281/zenodo.xxxxxxxx}


\ifdefined\standalonechapter\else
% Standalone mode
\documentclass[11pt]{article}
% sststyle.sty
\NeedsTeXFormat{LaTeX2e}
\ProvidesPackage{sststyle}[2025/07/01 SST unified style]



% === Draft Options ===
\newif\ifsstdraft
% \sstdrafttrue
\ifsstdraft
\RequirePackage{showframe}
\fi

% === Load Once ===
\RequirePackage{ifthen}
\newboolean{sststyleloaded}
\ifthenelse{\boolean{sststyleloaded}}{}{\setboolean{sststyleloaded}{true}

% === Page ===
\RequirePackage[a4paper, margin=2.5cm]{geometry}

% === Fonts ===
\RequirePackage[T1]{fontenc}
\RequirePackage[utf8]{inputenc}
\RequirePackage[english]{babel}
\RequirePackage{textgreek}
\RequirePackage{mathpazo}
\RequirePackage[scaled=0.95]{inconsolata}
\RequirePackage{helvet}

% === Math ===
\RequirePackage{amsmath, amssymb, mathrsfs, physics}
\RequirePackage{siunitx}
\sisetup{per-mode=symbol}

% === Tables ===
\RequirePackage{graphicx, float, booktabs}
\RequirePackage{array, tabularx, multirow, makecell}
\newcolumntype{Y}{>{\centering\arraybackslash}X}
\newenvironment{tighttable}[1][]{\begin{table}[H]\centering\renewcommand{\arraystretch}{1.3}\begin{tabularx}{\textwidth}{#1}}{\end{tabularx}\end{table}}
\RequirePackage{etoolbox}
\newcommand{\fitbox}[2][\linewidth]{\makebox[#1]{\resizebox{#1}{!}{#2}}}

% === Graphics ===
\RequirePackage{tikz}
\usetikzlibrary{3d, calc, arrows.meta, positioning}
\RequirePackage{pgfplots}
\pgfplotsset{compat=1.18}
\RequirePackage{xcolor}

% === Code ===
\RequirePackage{listings}
\lstset{basicstyle=\ttfamily\footnotesize, breaklines=true}

% === Theorems ===
\newtheorem{theorem}{Theorem}[section]
\newtheorem{lemma}[theorem]{Lemma}

% === TOC ===
\RequirePackage{tocloft}
\setcounter{tocdepth}{2}
\renewcommand{\cftsecfont}{\bfseries}
\renewcommand{\cftsubsecfont}{\itshape}
\renewcommand{\cftsecleader}{\cftdotfill{.}}
\renewcommand{\contentsname}{\centering \Huge\textbf{Contents}}

% === Sections ===
\RequirePackage{sectsty}
\sectionfont{\Large\bfseries\sffamily}
\subsectionfont{\large\bfseries\sffamily}

% === Bibliography ===


% === Links ===
\RequirePackage{hyperref}
\hypersetup{
    colorlinks=true,
    linkcolor=blue,
    citecolor=blue,
    urlcolor=blue,
    pdftitle={The Vortex \AE ther Model},
    pdfauthor={Omar Iskandarani},
    pdfkeywords={vorticity, gravity, \ae ther, fluid dynamics, time dilation, SST}
}
\urlstyle{same}
\RequirePackage{bookmark}

% === Misc ===
\RequirePackage[none]{hyphenat}
\sloppy
\RequirePackage{empheq}
\RequirePackage[most]{tcolorbox}
\newtcolorbox{eqbox}{colback=blue!5!white, colframe=blue!75!black, boxrule=0.6pt, arc=4pt, left=6pt, right=6pt, top=4pt, bottom=4pt}
\RequirePackage{titling}
\RequirePackage{amsfonts}
\RequirePackage{titlesec}
\RequirePackage{enumitem}

\AtBeginDocument{\RenewCommandCopy\qty\SI}

\pretitle{\begin{center}\LARGE\bfseries}
\posttitle{\par\end{center}\vskip 0.5em}
\preauthor{\begin{center}\large}
\postauthor{\end{center}}
\predate{\begin{center}\small}
\postdate{\end{center}}


\endinput
}
% sstappendixsetup.sty

\newcommand{\titlepageOpen}{
  \begin{titlepage}
  \thispagestyle{empty}
  \centering
  \ifdefined\standalonechapter
  {\Large\bfseries \appendixtitle \par}
  \else
  {\Large\bfseries \papertitle \par}
    \fi
  \vspace{1cm}
  {\Large\itshape \textbf{Omar Iskandarani}\textsuperscript{\textbf{*}} \par}
  \vspace{0.5cm}
  {\today \par}
  \vspace{0.5cm}
}

% here comes abstract
\newcommand{\titlepageClose}{
  \vfill
  \raggedright % <-- fixes left alignment
  \null
  \begin{picture}(0,0)
  % Adjust position: (x,y) = (left, bottom)
  \put(0,-45){  % Shift 200pt left, 40pt down
    \begin{minipage}[b]{0.7\textwidth}
    \footnotesize % One step bigger than \tiny \scriptsize
    \renewcommand{\arraystretch}{1.0}
    \noindent\rule{\textwidth}{0.4pt} \\[0.5em]  % ← horizontal line
    \textsuperscript{\textbf{*}} Independent Researcher, Groningen, The Netherlands \\
    Email: \texttt{info@omariskandarani.com} \\
    ORCID: \texttt{\href{https://orcid.org/0009-0006-1686-3961}{0009-0006-1686-3961}} \\
    DOI: \href{https://doi.org/\paperdoi}{\paperdoi} \\
    License: CC-BY-NC 4.0 International \\
    \end{minipage}
  }
  \end{picture}
  \end{titlepage}
}
\begin{document}

  % === Title page ===
  \titlepageOpen

  \begin{abstract}


  \end{abstract}

  \titlepageClose
  \fi

  \ifdefined\standalonechapter
  \section{\papertitle}
  \else
  \fi
% ============= Begin of content ============


% =====================================================================
% SST Canon Addendum — Intersecting D-brane PS & Theta-Texture Bridge
% Version tag: v0.3.1+2025-08-27-K (bridging-only; does not modify Core)
% =====================================================================

  \section*{Addendum K: Intersecting D-brane PS \& $\vartheta$-texture bridge (bridging)}

      \subsection*{K1. Scope and payoff}
          A Pati–Salam (PS) construction on a Type IIA $T^6/(\mathbb Z_2\times \mathbb Z_2)$ orientifold with intersecting D6-branes yields
          (i) rank-3 Yukawa matrices built from products of Jacobi $\vartheta$-functions,
          (ii) a selection rule $i{+}j{+}k\equiv 0 \pmod{3}$ for trilinears, and
          (iii) textures equivalent to $\Delta(27)$ that naturally give near-tribimaximal lepton mixing.
          This block translates those ingredients into the SST/VAM language (swirl projectors, EMH selector) and records usable kernel forms.

      \subsection*{K2. $\vartheta$-function Yukawa kernel (canonical form)}
          For three two-tori, the trilinear Yukawa couplings factorize as
          \begin{align}
          Y_{ijk}
          &= \mathfrak h\,q_u\,\sigma_{abc}\;
          \prod_{r=1}^3
          \vartheta\!\left[\begin{smallmatrix}\delta^{(r)}\\[2pt]\phi^{(r)}\end{smallmatrix}\right]\!(\kappa^{(r)}),\\[4pt]
          \vartheta\!\left[\begin{smallmatrix}\delta\\[2pt]\phi\end{smallmatrix}\right]\!(\kappa)
          &\equiv
          \sum_{\ell\in\mathbb Z}
          \exp\!\Big\{\pi i\,(\delta{+}\ell)^2 \kappa\;+\;2\pi i\,(\delta{+}\ell)\phi\Big\},
          \end{align}
          with input parameters (for each torus) $\delta^{(r)},\phi^{(r)},\kappa^{(r)}$ determined by intersection numbers,
          relative brane shifts $\varepsilon^{(r)}_{a,b,c}$, and Kähler data.
          Define the \emph{total shift}
          \[
              \varepsilon\;\equiv\;
              \frac{I_{ab}\,\varepsilon_c+I_{ca}\,\varepsilon_b+I_{bc}\,\varepsilon_a}{I_{ab}I_{bc}I_{ca}}.
          \]
          Focusing on the first torus (the others provide an overall factor), choose counting map $s^{(1)}=-\,i$.
          Then a generic family mass matrix \emph{decomposes} into odd/even Higgs-VEV sectors:
          \begin{align}
          M \;\sim\;
          &\underbrace{\begin{pmatrix}
          A\,v_1 & B\,v_3 & C\,v_5\\
          C\,v_3 & A\,v_5 & B\,v_1\\
          B\,v_5 & C\,v_1 & A\,v_3
          \end{pmatrix}}_{\text{odd-VEV block }(v_{1,3,5})}
          \;+\;
          \underbrace{\begin{pmatrix}
          E\,v_4 & F\,v_6 & D\,v_2\\
          D\,v_6 & E\,v_2 & F\,v_4\\
          F\,v_2 & D\,v_4 & E\,v_6
          \end{pmatrix}}_{\text{even-VEV block }(v_{2,4,6})},
          \end{align}
          where $v_k\equiv \langle H_{k+1}\rangle$ and the six coefficients are $\vartheta$-combinations
          \[
              \begin{aligned}
              &A\equiv \vartheta\!\left[\begin{smallmatrix}\varepsilon\\[-2pt]0\end{smallmatrix}\right]\!(6J/\alpha')\!,\quad
              B\equiv \vartheta\!\left[\begin{smallmatrix}\varepsilon+\tfrac13\\[-2pt]0\end{smallmatrix}\right]\!(6J/\alpha')\!,\quad
              C\equiv \vartheta\!\left[\begin{smallmatrix}\varepsilon-\tfrac13\\[-2pt]0\end{smallmatrix}\right]\!(6J/\alpha')\!,\\
              &D\equiv \vartheta\!\left[\begin{smallmatrix}\varepsilon+\tfrac16\\[-2pt]0\end{smallmatrix}\right]\!(6J/\alpha')\!,\quad
              E\equiv \vartheta\!\left[\begin{smallmatrix}\varepsilon+\tfrac12\\[-2pt]0\end{smallmatrix}\right]\!(6J/\alpha')\!,\quad
              F\equiv \vartheta\!\left[\begin{smallmatrix}\varepsilon-\tfrac16\\[-2pt]0\end{smallmatrix}\right]\!(6J/\alpha')\!.
              \end{aligned}
          \]

          \paragraph{Symmetry loci (shift $\varepsilon$).}
              At $\varepsilon=0$ or $\tfrac12$: $B=C$ and $D=F$; additionally $A=1,\ E=0$ at $\varepsilon=0$ and $A=0,\ E=1$ at $\varepsilon=\tfrac12$.
              Thus the odd-VEV block ($v_{1,3,5}$) or even-VEV block ($v_{2,4,6}$) can dominate by choice of $\varepsilon$.

  \subsection*{K3. $\Delta(27)$ texture emergence}
      At the symmetry loci above, $M$ takes the $\Delta(27)$-type form
      \[
          M\;\sim\;
          \begin{pmatrix}
          f_1 v_1 & f_2 v_3 & f_2 v_2\\
          f_2 v_3 & f_1 v_2 & f_2 v_1\\
          f_2 v_2 & f_2 v_1 & f_1 v_3
          \end{pmatrix},
      \]
      leading to (near-)tribimaximal lepton mixing when the even-VEV block dominates.
      This supplies a top-down realization of the projector textures adopted in Addendum~A4.

  \subsection*{K4. PS breaking and hypercharge (for cross-matching Addendum F)}
      Brane-splitting on torus 1:
      $a\rightarrow(a_1,a_2)=(6,2)$, $c\rightarrow(c_1,c_2)=(2,2)$ breaks
      $SU(4)_C\times SU(2)_L\times SU(2)_R$ to
      $SU(3)_C\times SU(2)_L\times U(1)_{I_3^R}\times U(1)_{B-L}$,
      then to hypercharge $U(1)_Y$ via vevs of vectorlike states with
      \[
          Q_Y=\tfrac16\bigl(Q_{a_1}-3Q_{a_2}-3Q_{c_1}+3Q_{c_2}\bigr).
      \]
      Gauge couplings unify at the string scale with a calculable holomorphic kinetic function; this dovetails with Safe PS (Addendum~F).

  \subsection*{K5. SST/VAM dictionary}
      \begin{itemize}
      \item \textbf{Shift parameter} $\varepsilon\;\longleftrightarrow\;
      \emph{swirl phase-offset}$ $\varepsilon_{\text{ae}}\equiv \Phi_{\rm swirl}/(2\pi)$ between the three family projectors:
      tuning $\varepsilon_{\text{ae}}=0$ (quarks) favors odd-VEV dominance and small quark mixing;
      $\varepsilon_{\text{ae}}=\tfrac12$ (leptons) favors even-VEV dominance and large lepton mixing.
      \item \textbf{Odd/even VEV blocks} $\longleftrightarrow$
      \emph{helicity-parity sectors} in the knotted-vortex map: odd (co-rotating) vs.\ even (counter-rotating) excitations.
      \item \textbf{$\Delta(27)$ locus} $\longleftrightarrow$
      projector basis of Addendum~A4 with a fixed phase $\Delta\phi$; the $\vartheta$-kernel supplies the discrete
      phase structure without inserting ad hoc symmetries.
      \end{itemize}

  \subsection*{K6. Minimal working recipe (quarks vs.\ leptons)}
      \begin{enumerate}
      \item Take $\varepsilon_{\text{ae}}^{(q)}=0$ and arrange $v^{U,D}_{1,3,5}\gg v^{U,D}_{2,4,6}$ (odd-dominance). Then CKM arises from small off-diagonals $B,C,D,F\ll A$ at the $\varepsilon=0$ locus.
      \item Take $\varepsilon_{\text{ae}}^{(\ell)}=\tfrac12$ and arrange $v^{U,\ell}_{2,4,6}\gg v^{U,\ell}_{1,3,5}$ (even-dominance). Then the lepton sector inherits the $\Delta(27)$ pattern and near-tribimaximal mixing.
      \item Enforce the consistency constraint on shifts $\varepsilon_u+\varepsilon_\ell=\varepsilon_d+\varepsilon_\nu$ and map to Addendum~J lemmas (EMH selector for Cabibbo; EJA targets for mass ratios).
      \end{enumerate}

  \subsection*{K7. Dimensional and limit checks}
      \begin{itemize}
      \item $Y_{ijk}$ and $\vartheta$-coefficients are dimensionless; Higgs $v_k$ carry mass dimension 1; entries of $M$ have mass dimension 1.
      \item Limits: $\varepsilon\to 0,\tfrac12$ recover symmetry-enhanced textures; turning off even (odd) VEVs isolates the quark (lepton) template.
      \end{itemize}

  \subsection*{K8. Predictive hooks}
      \begin{itemize}
      \item \textbf{Mixing angles (illustrative):} near-tribimaximal PMNS with nonzero $\theta_{13}$ emerges naturally once the even block dominates.
      \item \textbf{PS gauge unification:} string-scale unification compatible with Addendum~F running; hypercharge embedding fixed as above.
      \end{itemize}

  \section*{References (Bib\TeX)}
  \begin{verbatim}
@article{Mayes2019IntersectingDBranePS,
  author        = {Van E. Mayes},
  title         = {All Fermion Masses and Mixings in an Intersecting D-brane World},
  year          = {2019},
  eprint        = {1902.00983},
  archivePrefix = {arXiv},
  primaryClass  = {hep-ph}
}
  \end{verbatim}



% ============== End of content =============

% === Bibliography (only for standalone) ===
  \ifdefined\standalonechapter\else
  \bibliographystyle{unsrt}
  \bibliography{../bridging_blocks_references}
\end{document}
\fi