%! Author = Omar Iskandarani
%! Title = ......
%! Date = .....
%! Affiliation = Independent Researcher, Groningen, The Netherlands
%! License = © 2025 Omar Iskandarani. All rights reserved. This manuscript is made available for academic reading and citation only. No republication, redistribution, or derivative works are permitted without explicit written permission from the author. Contact: info@omariskandarani.com
%! ORCID = 0009-0006-1686-3961
%! DOI = 10.5281/zenodo.xxxxxxx

% =====================================================================
% SST Canon Addendum — Bridging Blocks
% Version tag: v0.3.1+2025-08-27 (conforms to Canon v0.3.1)
% Persona: Bridging (does not modify Core Postulates)
% =====================================================================

% === Metadata ===
\newcommand{\papertitle}{....}
\newcommand{\paperdoi}{10.5281/zenodo.xxxxxxxx}


\ifdefined\standalonechapter\else
% Standalone mode
\documentclass[11pt]{article}
% sststyle.sty
\NeedsTeXFormat{LaTeX2e}
\ProvidesPackage{sststyle}[2025/07/01 SST unified style]



% === Draft Options ===
\newif\ifsstdraft
% \sstdrafttrue
\ifsstdraft
\RequirePackage{showframe}
\fi

% === Load Once ===
\RequirePackage{ifthen}
\newboolean{sststyleloaded}
\ifthenelse{\boolean{sststyleloaded}}{}{\setboolean{sststyleloaded}{true}

% === Page ===
\RequirePackage[a4paper, margin=2.5cm]{geometry}

% === Fonts ===
\RequirePackage[T1]{fontenc}
\RequirePackage[utf8]{inputenc}
\RequirePackage[english]{babel}
\RequirePackage{textgreek}
\RequirePackage{mathpazo}
\RequirePackage[scaled=0.95]{inconsolata}
\RequirePackage{helvet}

% === Math ===
\RequirePackage{amsmath, amssymb, mathrsfs, physics}
\RequirePackage{siunitx}
\sisetup{per-mode=symbol}

% === Tables ===
\RequirePackage{graphicx, float, booktabs}
\RequirePackage{array, tabularx, multirow, makecell}
\newcolumntype{Y}{>{\centering\arraybackslash}X}
\newenvironment{tighttable}[1][]{\begin{table}[H]\centering\renewcommand{\arraystretch}{1.3}\begin{tabularx}{\textwidth}{#1}}{\end{tabularx}\end{table}}
\RequirePackage{etoolbox}
\newcommand{\fitbox}[2][\linewidth]{\makebox[#1]{\resizebox{#1}{!}{#2}}}

% === Graphics ===
\RequirePackage{tikz}
\usetikzlibrary{3d, calc, arrows.meta, positioning}
\RequirePackage{pgfplots}
\pgfplotsset{compat=1.18}
\RequirePackage{xcolor}

% === Code ===
\RequirePackage{listings}
\lstset{basicstyle=\ttfamily\footnotesize, breaklines=true}

% === Theorems ===
\newtheorem{theorem}{Theorem}[section]
\newtheorem{lemma}[theorem]{Lemma}

% === TOC ===
\RequirePackage{tocloft}
\setcounter{tocdepth}{2}
\renewcommand{\cftsecfont}{\bfseries}
\renewcommand{\cftsubsecfont}{\itshape}
\renewcommand{\cftsecleader}{\cftdotfill{.}}
\renewcommand{\contentsname}{\centering \Huge\textbf{Contents}}

% === Sections ===
\RequirePackage{sectsty}
\sectionfont{\Large\bfseries\sffamily}
\subsectionfont{\large\bfseries\sffamily}

% === Bibliography ===


% === Links ===
\RequirePackage{hyperref}
\hypersetup{
    colorlinks=true,
    linkcolor=blue,
    citecolor=blue,
    urlcolor=blue,
    pdftitle={The Vortex \AE ther Model},
    pdfauthor={Omar Iskandarani},
    pdfkeywords={vorticity, gravity, \ae ther, fluid dynamics, time dilation, SST}
}
\urlstyle{same}
\RequirePackage{bookmark}

% === Misc ===
\RequirePackage[none]{hyphenat}
\sloppy
\RequirePackage{empheq}
\RequirePackage[most]{tcolorbox}
\newtcolorbox{eqbox}{colback=blue!5!white, colframe=blue!75!black, boxrule=0.6pt, arc=4pt, left=6pt, right=6pt, top=4pt, bottom=4pt}
\RequirePackage{titling}
\RequirePackage{amsfonts}
\RequirePackage{titlesec}
\RequirePackage{enumitem}

\AtBeginDocument{\RenewCommandCopy\qty\SI}

\pretitle{\begin{center}\LARGE\bfseries}
\posttitle{\par\end{center}\vskip 0.5em}
\preauthor{\begin{center}\large}
\postauthor{\end{center}}
\predate{\begin{center}\small}
\postdate{\end{center}}


\endinput
}
% sstappendixsetup.sty

\newcommand{\titlepageOpen}{
  \begin{titlepage}
  \thispagestyle{empty}
  \centering
  \ifdefined\standalonechapter
  {\Large\bfseries \appendixtitle \par}
  \else
  {\Large\bfseries \papertitle \par}
    \fi
  \vspace{1cm}
  {\Large\itshape \textbf{Omar Iskandarani}\textsuperscript{\textbf{*}} \par}
  \vspace{0.5cm}
  {\today \par}
  \vspace{0.5cm}
}

% here comes abstract
\newcommand{\titlepageClose}{
  \vfill
  \raggedright % <-- fixes left alignment
  \null
  \begin{picture}(0,0)
  % Adjust position: (x,y) = (left, bottom)
  \put(0,-45){  % Shift 200pt left, 40pt down
    \begin{minipage}[b]{0.7\textwidth}
    \footnotesize % One step bigger than \tiny \scriptsize
    \renewcommand{\arraystretch}{1.0}
    \noindent\rule{\textwidth}{0.4pt} \\[0.5em]  % ← horizontal line
    \textsuperscript{\textbf{*}} Independent Researcher, Groningen, The Netherlands \\
    Email: \texttt{info@omariskandarani.com} \\
    ORCID: \texttt{\href{https://orcid.org/0009-0006-1686-3961}{0009-0006-1686-3961}} \\
    DOI: \href{https://doi.org/\paperdoi}{\paperdoi} \\
    License: CC-BY-NC 4.0 International \\
    \end{minipage}
  }
  \end{picture}
  \end{titlepage}
}
\begin{document}

  % === Title page ===
  \titlepageOpen

  \begin{abstract}


  \end{abstract}

  \titlepageClose
  \fi

  \ifdefined\standalonechapter
  \section{\papertitle}
  \else
  \fi
% ============= Begin of content ============

% =====================================================================
% Addendum B: Asymptotically Safe Gravity as a Quark–Mass Splitter
% Source: Eichhorn & Held (2018), arXiv:1803.04027
% Persona: Bridging (does not modify Core Postulates)
% =====================================================================

  \section*{Addendum B: ASG-Induced Yukawa Splitting and Charge Selection}

      \subsection*{B1. Setup and one-loop structure with gravity-induced anomalous dimensions}
          We introduce dimensionless, approximately scale-constant (in the trans–cutoff regime) anomalous-dimension parameters $f_g\!\ge\!0$ and $f_y\!\ge\!0$ to encode universal gravity effects on gauge and Yukawa sectors, respectively. The gauge beta functions are modeled as
          \begin{equation}
          \beta_{g_i} \;=\; \frac{b_{0,i}}{16\pi^2}\,g_i^3 \;-\; f_g\, g_i,
          \qquad i\in\{3,2,Y\},
          \end{equation}
          with $b_{0,3}=-7$, $b_{0,2}=-19/6$, and
          \begin{equation}
          b_{0,Y} \;=\; \frac{1}{6}\Big(19 + 36\,(Y_b^2 + 2Y_Q^2 + Y_t^2)\Big),
          \end{equation}
          where $Y_t, Y_b, Y_Q$ denote the (model-normalized) hypercharges of the RH top, RH bottom, and LH doublet. A nontrivial Abelian fixed point appears at
          \begin{equation}\tag{B1}
          g_{Y,*}^2 \;=\; \frac{16\pi^2}{b_{0,Y}}\, f_g \, .
          \end{equation}

          The Yukawa beta functions (suppressing family indices) in the trans–cutoff regime (with $f_g, f_y$ effectively constant) read
          \begin{equation}\tag{B2}
          \beta_{y_{t(b)}} \;=\; \frac{y_{t(b)}}{16\pi^2}
          \Big(\tfrac{3}{2}y_{b(t)}^2 + \tfrac{9}{2}y_{t(b)}^2 - \tfrac{9}{4}g_2^2 - 8g_3^2\Big)
          \;-\; f_y\,y_{t(b)}
          \;-\; \frac{3y_{t(b)}}{16\pi^2}\,(Y_Q^2+Y_{t(b)}^2)\,g_Y^2 .
          \end{equation}
          At the interacting UV fixed point with $g_2\!\to\!0$, $g_3\!\to\!0$ and $g_Y\!\to\!g_{Y,*}$, Eqs.~(B1)–(B2) yield nonzero Yukawa fixed points $y_{t,*},y_{b,*}$.

      \subsection*{B2. Fixed-point relation and its generalization}
          Eliminating $f_y$ between the $t$ and $b$ equations at the fixed point gives the \emph{universal difference law}
          \begin{equation}\tag{B3}
          y_{t,*}^2 \;-\; y_{b,*}^2 \;=\; (Q_t^2 - Q_b^2)\, g_{Y,*}^2 ,
          \end{equation}
          using $Y_t=Q_t$, $Y_b=Q_b$, $Y_Q=Q_t-\tfrac12$ and the normalization $Q=T_3+Y$. For the Standard-Model charges $Q_t=\tfrac{2}{3}$, $Q_b=-\tfrac{1}{3}$, Eq.~(B3) reduces to
          \begin{equation}\tag{B3-SM}
          \boxed{\, y_{t,*}^2 - y_{b,*}^2 \;=\; \tfrac{1}{3}\, g_{Y,*}^2 \,}\, .
          \end{equation}
          Thus $y_{t,*}\!>\!y_{b,*}$ is enforced by hypercharge alone. IR running (below the gravity-dominated regime) preserves the inequality, thereby generating $M_t\!>\!M_b$ without ad hoc Yukawa hierarchies.

      \subsection*{B3. Consequences and selection rules (SST interpretation)}
          \begin{itemize}
          \item \textbf{Charge selection:} Varying $Q_b/Q_t$ away from $-\tfrac12$ spoils the simultaneous retrodiction of $(g_Y,\,y_t,\,y_b)$. In SST, this acts as a \emph{topological selection rule}: the emergent hypercharge map from swirl–string data must fix $Q_b/Q_t=-1/2$.
          \item \textbf{Universality of gravity–induced anomalous dimensions:} The observed $(g_Y, y_t, y_b)$ favor gauge-group \emph{independence} of $f_g$. In SST bridging language: the swirl microstructure that induces antiscreening couples \emph{universally} to all gauge sectors.
          \item \textbf{IR robustness:} Once set by (B3) at the UV fixed point, Abelian screening (the $-\,g_Y^2$ term) drives $y_t/y_b$ further from unity, maintaining $y_t>y_b$ down to the electroweak scale.
          \end{itemize}

      \subsection*{B4. Canon Lemma (bridging): ASG quark–mass splitter}
          \begin{quote}
          \textbf{Lemma.} In any SST bridging EFT where (i) a universal antiscreening term induces an Abelian fixed point (B1) and (ii) Yukawa flows include gravity-induced damping as in (B2), the UV fixed point implies the exact relation (B3). For SM charges, (B3-SM) holds and enforces $M_t\!>\!M_b$ after IR evolution.
          \end{quote}

          \paragraph{Dimensional/limit checks.} All quantities in (B1)–(B3) are dimensionless. The limit $g_{Y,*}\!\to\!0$ (no fixed point) collapses (B3) to the trivial $y_{t,*}=y_{b,*}$. The result is independent of the absolute UV scale so long as the fixed-point regime exists.

  \subsection*{B5. SST mapping and practical use}
      \begin{itemize}
      \item \textbf{Hypercharge from swirl topology:} Implement the emergent U(1) map so that $Q_t, Q_b$ (hence $Y$) arise from discrete swirl invariants; enforce $Q_b/Q_t=-1/2$ as a canon constraint (bridging) consistent with (B3).
      \item \textbf{Numerical pipeline:} Calibrate $f_g$ from the IR value of $g_Y$ using (B1), then use (B3) plus IR running to fit $y_b$ with a single $f_y$. This mirrors the two-parameter retrodiction in the ASG analysis and provides a compact SST flavour module.
      \item \textbf{Universality test:} Penalize any SST scenario that requires $f_g$ to differ between $SU(3)$ and $SU(2)$ to hit $(g_Y,y_t,y_b)$, in line with the universality hint.
      \end{itemize}

  \section*{References (Bib\TeX)}
  \begin{verbatim}
@article{EichhornHeld2018ASGQuarkSplit,
  author        = {Astrid Eichhorn and Aaron Held},
  title         = {Mass difference for charged quarks from asymptotically safe quantum gravity},
  journal       = {Phys. Rev. Lett. (preprint version on arXiv)},
  year          = {2018},
  eprint        = {1803.04027},
  archivePrefix = {arXiv},
  primaryClass  = {hep-th}
}
@article{GrossWilczek1973,
  author  = {David J. Gross and Frank Wilczek},
  title   = {Ultraviolet Behavior of Non-Abelian Gauge Theories},
  journal = {Phys. Rev. Lett.},
  volume  = {30},
  pages   = {1343},
  year    = {1973}
}
@article{Politzer1973,
  author  = {H. David Politzer},
  title   = {Reliable Perturbative Results for Strong Interactions?},
  journal = {Phys. Rev. Lett.},
  volume  = {30},
  pages   = {1346},
  year    = {1973}
}
@article{GellMannLow1954},
  author  = {M. Gell-Mann and F. Low},
  title   = {Quantum electrodynamics at small distances},
  journal = {Phys. Rev.},
  volume  = {95},
  pages   = {1300},
  year    = {1954}
}
  \end{verbatim}




% ============== End of content =============

% === Bibliography (only for standalone) ===
  \ifdefined\standalonechapter\else
  \bibliographystyle{unsrt}

\end{document}
\fi