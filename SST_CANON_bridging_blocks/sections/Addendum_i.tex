%! Author = Omar Iskandarani
%! Title = ......
%! Date = .....
%! Affiliation = Independent Researcher, Groningen, The Netherlands
%! License = © 2025 Omar Iskandarani. All rights reserved. This manuscript is made available for academic reading and citation only. No republication, redistribution, or derivative works are permitted without explicit written permission from the author. Contact: info@omariskandarani.com
%! ORCID = 0009-0006-1686-3961
%! DOI = 10.5281/zenodo.xxxxxxx

% =====================================================================
% SST Canon Addendum — Bridging Blocks
% Version tag: v0.3.1+2025-08-27 (conforms to Canon v0.3.1)
% Persona: Bridging (does not modify Core Postulates)
% =====================================================================

% === Metadata ===
\newcommand{\papertitle}{....}
\newcommand{\paperdoi}{10.5281/zenodo.xxxxxxxx}


\ifdefined\standalonechapter\else
% Standalone mode
\documentclass[11pt]{article}
% sststyle.sty
\NeedsTeXFormat{LaTeX2e}
\ProvidesPackage{sststyle}[2025/07/01 SST unified style]



% === Draft Options ===
\newif\ifsstdraft
% \sstdrafttrue
\ifsstdraft
\RequirePackage{showframe}
\fi

% === Load Once ===
\RequirePackage{ifthen}
\newboolean{sststyleloaded}
\ifthenelse{\boolean{sststyleloaded}}{}{\setboolean{sststyleloaded}{true}

% === Page ===
\RequirePackage[a4paper, margin=2.5cm]{geometry}

% === Fonts ===
\RequirePackage[T1]{fontenc}
\RequirePackage[utf8]{inputenc}
\RequirePackage[english]{babel}
\RequirePackage{textgreek}
\RequirePackage{mathpazo}
\RequirePackage[scaled=0.95]{inconsolata}
\RequirePackage{helvet}

% === Math ===
\RequirePackage{amsmath, amssymb, mathrsfs, physics}
\RequirePackage{siunitx}
\sisetup{per-mode=symbol}

% === Tables ===
\RequirePackage{graphicx, float, booktabs}
\RequirePackage{array, tabularx, multirow, makecell}
\newcolumntype{Y}{>{\centering\arraybackslash}X}
\newenvironment{tighttable}[1][]{\begin{table}[H]\centering\renewcommand{\arraystretch}{1.3}\begin{tabularx}{\textwidth}{#1}}{\end{tabularx}\end{table}}
\RequirePackage{etoolbox}
\newcommand{\fitbox}[2][\linewidth]{\makebox[#1]{\resizebox{#1}{!}{#2}}}

% === Graphics ===
\RequirePackage{tikz}
\usetikzlibrary{3d, calc, arrows.meta, positioning}
\RequirePackage{pgfplots}
\pgfplotsset{compat=1.18}
\RequirePackage{xcolor}

% === Code ===
\RequirePackage{listings}
\lstset{basicstyle=\ttfamily\footnotesize, breaklines=true}

% === Theorems ===
\newtheorem{theorem}{Theorem}[section]
\newtheorem{lemma}[theorem]{Lemma}

% === TOC ===
\RequirePackage{tocloft}
\setcounter{tocdepth}{2}
\renewcommand{\cftsecfont}{\bfseries}
\renewcommand{\cftsubsecfont}{\itshape}
\renewcommand{\cftsecleader}{\cftdotfill{.}}
\renewcommand{\contentsname}{\centering \Huge\textbf{Contents}}

% === Sections ===
\RequirePackage{sectsty}
\sectionfont{\Large\bfseries\sffamily}
\subsectionfont{\large\bfseries\sffamily}

% === Bibliography ===


% === Links ===
\RequirePackage{hyperref}
\hypersetup{
    colorlinks=true,
    linkcolor=blue,
    citecolor=blue,
    urlcolor=blue,
    pdftitle={The Vortex \AE ther Model},
    pdfauthor={Omar Iskandarani},
    pdfkeywords={vorticity, gravity, \ae ther, fluid dynamics, time dilation, SST}
}
\urlstyle{same}
\RequirePackage{bookmark}

% === Misc ===
\RequirePackage[none]{hyphenat}
\sloppy
\RequirePackage{empheq}
\RequirePackage[most]{tcolorbox}
\newtcolorbox{eqbox}{colback=blue!5!white, colframe=blue!75!black, boxrule=0.6pt, arc=4pt, left=6pt, right=6pt, top=4pt, bottom=4pt}
\RequirePackage{titling}
\RequirePackage{amsfonts}
\RequirePackage{titlesec}
\RequirePackage{enumitem}

\AtBeginDocument{\RenewCommandCopy\qty\SI}

\pretitle{\begin{center}\LARGE\bfseries}
\posttitle{\par\end{center}\vskip 0.5em}
\preauthor{\begin{center}\large}
\postauthor{\end{center}}
\predate{\begin{center}\small}
\postdate{\end{center}}


\endinput
}
% sstappendixsetup.sty

\newcommand{\titlepageOpen}{
  \begin{titlepage}
  \thispagestyle{empty}
  \centering
  \ifdefined\standalonechapter
  {\Large\bfseries \appendixtitle \par}
  \else
  {\Large\bfseries \papertitle \par}
    \fi
  \vspace{1cm}
  {\Large\itshape \textbf{Omar Iskandarani}\textsuperscript{\textbf{*}} \par}
  \vspace{0.5cm}
  {\today \par}
  \vspace{0.5cm}
}

% here comes abstract
\newcommand{\titlepageClose}{
  \vfill
  \raggedright % <-- fixes left alignment
  \null
  \begin{picture}(0,0)
  % Adjust position: (x,y) = (left, bottom)
  \put(0,-45){  % Shift 200pt left, 40pt down
    \begin{minipage}[b]{0.7\textwidth}
    \footnotesize % One step bigger than \tiny \scriptsize
    \renewcommand{\arraystretch}{1.0}
    \noindent\rule{\textwidth}{0.4pt} \\[0.5em]  % ← horizontal line
    \textsuperscript{\textbf{*}} Independent Researcher, Groningen, The Netherlands \\
    Email: \texttt{info@omariskandarani.com} \\
    ORCID: \texttt{\href{https://orcid.org/0009-0006-1686-3961}{0009-0006-1686-3961}} \\
    DOI: \href{https://doi.org/\paperdoi}{\paperdoi} \\
    License: CC-BY-NC 4.0 International \\
    \end{minipage}
  }
  \end{picture}
  \end{titlepage}
}
\begin{document}

  % === Title page ===
  \titlepageOpen

  \begin{abstract}


  \end{abstract}

  \titlepageClose
  \fi

  \ifdefined\standalonechapter
  \section{\papertitle}
  \else
  \fi
% ============= Begin of content ============



% =====================================================================
% SST Canon Addendum — Unified Flavor–CP–Modular Symmetry (Narain → SST)
% Version tag: v0.3.1+2025-08-27-I (conforms to Canon v0.3.1)
% Persona: Bridging (does not modify Core Postulates)
% Source: Baur–Nilles–Trautner–Vaudrevange (2019), arXiv:1901.03251
% =====================================================================

  \section*{Addendum I: Unified Flavor–CP–Modular Symmetry (Narain → SST) (bridging)}

      \subsection*{I1. Unified symmetry via outer automorphisms (Narain picture)}

          \textbf{Statement.} In heterotic orbifolds, the \emph{unified flavor group} $\mathcal G_{\mathrm{UFG}}$ arises from the \emph{outer automorphisms} of the Narain space group $\widehat S_{\mathrm{Narain}}$. It \emph{contains and unifies} (i) traditional non-Abelian flavor, (ii) modular/T-duality symmetries, and (iii) CP-like transformations. The group is \emph{moduli-dependent} and can be enhanced on lines/points in moduli space.

          \paragraph{Lattice-basis automorphism.}
              For $g=(\widehat\Theta^{\,k},\widehat N)\in\widehat S_{\mathrm{Narain}}$, an outer automorphism $\widehat h=(\widehat\Sigma,\widehat T)\notin\widehat S_{\mathrm{Narain}}$ acts by
              $g\mapsto \widehat h\,g\,\widehat h^{-1}\in\widehat S_{\mathrm{Narain}}$ with consistency
              \[
                  \widehat\Sigma\,\widehat\Theta^{\,k}\,\widehat\Sigma^{-1}=\widehat\Theta^{\,k'},
                  \qquad
                  \bigl(\mathbb 1-\widehat\Sigma\,\widehat\Theta^{\,k}\,\widehat\Sigma^{-1}\bigr)\widehat T\in\mathbb Z^{2D}.
              \]
              When $\widehat\Sigma$ leaves the compactification moduli invariant, it contributes to the \emph{traditional} flavor group at that moduli value.

          \paragraph{Local flavor groups.}
              Fields localized at distinct fixed points/sectors can feel \emph{different} subgroups (\emph{local flavor groups}). This supplies a controlled sector-dependent flavor/CP structure.

  \subsection*{I2. $T^2/\mathbb Z_3$ exemplar and enhancement loci}

      Let the Kähler modulus be $T=b+i\tfrac{\sqrt3}{2}\,r$. Generically one obtains a non-Abelian flavor group $\Delta(54)$ generated by two lattice translations and a symmetric rotation.

      \subsubsection*{I2.1 Generic region}
          At generic $\langle T\rangle$, the generators $A=(\mathbb 1,\widehat T_1)$, $B=(\mathbb 1,\widehat T_2)$, and $C=(\widehat S_{\mathrm{rot}}(\pi),0)$ close to $\Delta(54)$. (Space-group selection rules embed into this closure.)

      \subsubsection*{I2.2 Vertical lines $b=\tfrac{n_B}{2}$ (CP unification)}
          For integer $n_B$, a left–right symmetric reflection $D(n_B)=(\widehat S_{\mathrm{refl}}(\tfrac{2\pi}{6}),0)$ preserves the Narain lattice and acts on the modulus as
          \[
              T\;\longmapsto\; n_B - T.
          \]
          On $\operatorname{Re}T=\tfrac{n_B}{2}$ this yields an extra $\mathbb Z_2$ intertwining barred/unbarred twisted sectors, promoting CP from an outer automorphism of $\Delta(54)$ to an \emph{inner} operation inside the enhanced group. The unified flavor group enhances to $\mathrm{SG}(108,17)$.

      \subsubsection*{I2.3 Black semicircles $|T|=1$}
          On $|T|=1$ ($\mathrm{Im}\,T>0$), an asymmetric reflection $E=(\widehat A_{\mathrm{refl}},0)$ is unbroken, giving another enhancement to a group isomorphic to $\mathrm{SG}(108,17)$ (a different extension of $\Delta(54)$).

      \subsubsection*{I2.4 Intersections}
          At intersections of enhancement loci: two-line crossings (e.g.\ $b=0,\ r=2/\sqrt3$) give $\mathrm{SG}(216,87)$; triple crossings (e.g.\ $b=\tfrac12,\ r=1$) give $\mathrm{SG}(324,39)$. These realize points where CP is exactly conserved inside $\mathcal G_{\mathrm{UFG}}$, breaking spontaneously when displaced in moduli.

  \subsection*{I3. SST/VAM adapter (bridging)}

      \paragraph{Dictionary.}
          Replace the Narain lattice by a \emph{swirl lattice} (a discrete lattice of circulation/winding and \ae{}-phase translations) with an effective “swirl modulus” $T_{\text{ae}}$ (dimensionless). Outer automorphisms of the swirl lattice unify: (i) discrete swirl flavor (knot classes), (ii) \ae{}-modular transformations (duality of discrete swirl periods), and (iii) a CP-like reflection. \emph{Local flavor groups} correspond to distinct localization of knot excitations in the swirl lattice.

      \paragraph{CP-in-UFG selection rule.}
          If a sector’s $T_{\text{ae}}$ lies on a CP-enhancement locus (analogue of $\operatorname{Re}T=\tfrac{n}{2}$ or $|T|=1$), then CP is \emph{inner} and can be exact at that scale; away from the locus, CP is spontaneously broken by $T_{\text{ae}}$ displacement.

      \paragraph{Sector differentiation.}
          Assign $T_{\text{ae}}$ values for $u,d,\ell,\nu$ on \emph{different} loci (“local flavoring”): e.g.\ quarks near generic $\Delta(54)$ (small mixing) while leptons near an enhancement arc (large mixing). This dovetails with Addenda C (EW kernels), D (EJA ladders), and E (rotating rank-one).

  \subsection*{I4. Practical canon rules}

  \begin{itemize}
  \item \textbf{Rule I.1 (Swirl–Narain closure).} Build the sector’s discrete symmetry as the closure of its swirl translations $(A,B)$ and one reflection/rotation $(C,D,E)$ determined by $T_{\text{ae}}$.
  \item \textbf{Rule I.2 (CP bookkeeping).} Mark CP as \emph{inner} if $T_{\text{ae}}$ lies on enhancement loci; else tag as \emph{outer} with spontaneous breaking controlled by $\delta T_{\text{ae}}$.
  \item \textbf{Rule I.3 (Local flavor).} Permit $\mathcal G_u\neq\mathcal G_d\neq\mathcal G_\ell\neq\mathcal G_\nu$. Record their residuals after Wilson-line–like swirl translations (“swirl Wilson lines”).
  \end{itemize}

  \subsection*{I5. Consistency and limits}

  \begin{itemize}
  \item Dimensions: $T,\ T_{\text{ae}}$ are dimensionless; all group operations are algebraic.
  \item Limits: moving off enhancement loci reduces $\mathrm{SG}(108,17)\to \Delta(54)$; triple points flow to $\mathrm{SG}(216,87)$ or $\Delta(54)$ under small deformations.
  \item Compatibility: The EMH selector (Addendum H) and R2M2 geometry (Addendum E) can be imposed \emph{within} a given local flavor group.
  \end{itemize}

  \section*{References (Bib\TeX)}
  \begin{verbatim}
@article{BaurNillesTrautnerVaudrevange2019,
  author        = {Alexander Baur and Hans Peter Nilles and Andreas Trautner
                   and Patrick K. S. Vaudrevange},
  title         = {Unification of Flavor, CP, and Modular Symmetries},
  year          = {2019},
  eprint        = {1901.03251},
  archivePrefix = {arXiv},
  primaryClass  = {hep-th}
}
@article{ChenEtAl2014CPFiniteGroups,
  author  = {M.-C. Chen and M. Fallbacher and K. T. Mahanthappa
             and M. Ratz and A. Trautner},
  title   = {CP Violation from Finite Groups},
  journal = {Nucl. Phys. B},
  volume  = {883},
  pages   = {267--305},
  year    = {2014},
  doi     = {10.1016/j.nuclphysb.2014.03.023}
}
@article{KobayashiEtAl2007StringyFlavor,
  author  = {T. Kobayashi and H. P. Nilles and F. Pl{\"o}ger
             and S. Raby and M. Ratz},
  title   = {Stringy origin of non-Abelian discrete flavor symmetries},
  journal = {Nucl. Phys. B},
  volume  = {768},
  pages   = {135--156},
  year    = {2007},
  doi     = {10.1016/j.nuclphysb.2007.01.018}
}
@article{LercheLuestWarner1989,
  author  = {W. Lerche and D. L{\"u}st and N. P. Warner},
  title   = {Duality Symmetries in N=2 Landau-Ginzburg Models},
  journal = {Phys. Lett. B},
  volume  = {231},
  pages   = {417--424},
  year    = {1989},
  doi     = {10.1016/0370-2693(89)90686-2}
}
  \end{verbatim}


% ============== End of content =============

% === Bibliography (only for standalone) ===
  \ifdefined\standalonechapter\else
  \bibliographystyle{unsrt}
  \bibliography{../bridging_blocks_references}
\end{document}
\fi