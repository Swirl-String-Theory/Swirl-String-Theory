%! Author = Omar Iskandarani
%! Title = Electron--Swirl Coupled Transport in Swirl--String Theory (SST): Perturbative Solutions, Quantitative Benchmarks, and Falsifiable Experiments
%! Date = Sept 4, 2025
%! Affiliation = Independent Researcher, Groningen, The Netherlands
%! License = © 2025 Omar Iskandarani. All rights reserved. This manuscript is made available for academic reading and citation only. No republication, redistribution, or derivative works are permitted without explicit written permission from the author. Contact: info@omariskandarani.com
%! ORCID = 0009-0006-1686-3961
%! DOI = 10.5281/zenodo.17459746

%========================================================================================
\newcommand{\paperdoi}{10.5281/zenodo.17459746}

%========================================================================================
% PACKAGES AND DOCUMENT CONFIGURATION
%========================================================================================
\documentclass[aps,prb,preprint,amsmath,amssymb]{revtex4-2} % switch to "reprint" for two-column look
\usepackage{siunitx}
\usepackage{graphicx}
\usepackage{bm}
\usepackage{physics} % for \Tr and other conveniences
\usepackage[hidelinks]{hyperref}
\usepackage{amssymb} % keep last

% ===== SST canonical scales and convenient macros =====
\newcommand{\vswirlVal}{1.09384563\times10^{6}} % m/s
\newcommand{\rcVal}{1.40897017\times10^{-15}} % m
\newcommand{\rhoFVal}{7.0\times10^{-7}} % kg/m^3

\newcommand{\vswirl}{v_{\!\mkern-2mu\scriptstyle\boldsymbol{\circlearrowleft}}}
\newcommand{\rc}{r_c}
\newcommand{\rhoF}{\rho_{f}}
\newcommand{\rhoE}{\rho_{E}}
\newcommand{\omegas}{\boldsymbol{\omega}_{\!\mkern-2mu\scriptstyle\boldsymbol{\circlearrowleft}}} % swirl vorticity symbol

\begin{document}

    \title{Electron--Swirl Coupled Transport in Swirl--String Theory (SST):\\
    Perturbative Solutions, Quantitative Benchmarks, and Falsifiable Experiments}

    \author{Omar Iskandarani}
    \affiliation{Independent Researcher, Groningen, The Netherlands}
    \thanks{ORCID: 0009-0006-1686-3961, DOI: \paperdoi}
    \date{\today}

    \begin{abstract}
        I present a self-contained treatment of electron--swirl transport within \emph{Swirl--String Theory (SST)}. The analysis (i) derives a \emph{perturbative, steady-state} solution to the coupled density-matrix equations in 1D, (ii) makes \emph{quantitative} predictions for tabletop experiments with explicit materials, geometries, and signal levels, and (iii) states clear \emph{falsifiability criteria}. The framework reproduces the Peierls (population) and Allen--Feldman (coherence) limits \cite{Peierls1929,AllenFeldman1993,Simoncelli2019Unified} while embedding electrons as \emph{swirl strings} (knotted vortex filaments) coupled to swirl modes. Kinematic time-rate variations locally modulate the electronic Hamiltonian through the \emph{Swirl Clock} factor \cite{Madelung1927,Pati2000}. Numerical scales are anchored by SST canonical values $\vswirl=\vswirlVal\,\si{m/s}$, $\rc=\rcVal\,\si{m}$, and $\rhoF=\rhoFVal\,\si{kg/m^3}$. For compactness, some intermediate results are given in units of the SST frequency $\Omega_0$; all final predictions are reported in SI units.
    \end{abstract}

    \maketitle

    \section{Scales from SST}
        SST fixes a characteristic core-swirl frequency and an associated energy density,
        \begin{align}
            \Omega_0 \equiv \frac{\vswirl}{\rc} \approx 7.76\times10^{20}\,\si{s^{-1}},\qquad
            \rhoE \equiv \tfrac12\,\rhoF\,\vswirl^2 \approx 4.19\times10^{5}\,\si{J/m^3}.
        \end{align}
        Operationally, spatial gradients in the Swirl Clock produce local kinematic time-rate variations. These enter the electronic Hamiltonian $H_e$ multiplicatively as a modulation factor and do not alter the SI reporting of observables. Where it improves readability, I normalize rates to $\Omega_0$; experimental benchmarks and error budgets remain in SI.

    \section{Coupled transport in 1D and perturbative solution}
        I adopt the unified density-matrix equation for bosonic modes $N(\mathbf R,\mathbf q)$ \cite{Simoncelli2019Unified} and extend it to a charged two-level system (``electron'') with density matrix $f$:
        \begin{align}
            \partial_t N &= -i[\Omega,N] - \Gamma_\mathrm{b} \circ (N-N^{(0)}) - \tfrac12\{ V_x \partial_x, N\},\label{eq:Ndyn}\\
            \partial_t f &= -i[H_e,f] - \Gamma_\mathrm{e}\circ(f-f^{(0)}) - \tfrac12\{ v_{e,x}\partial_x, f\} + \mathcal C_{e\leftrightarrow b},\label{eq:fdyn}
        \end{align}
        with diagonal damping superoperators $\Gamma_\mathrm{b}$ and $\Gamma_\mathrm{e}$. The electron--swirl coupling is treated in the Born--Markov, rotating-wave approximation,
        \begin{equation}
            \mathcal C_{e\leftrightarrow b} \equiv -\frac{i}{\hbar}[M, f\otimes N]_{\mathrm{RWA}}\,.
        \end{equation}

        \subsection{Linear response to a static gradient}
            Consider a small uniform temperature gradient $\partial_x T$ and a time-independent steady state. Linearize about $N^{(0)}(T)$ and $f^{(0)}(T)$ via $N=N^{(0)}+N^{(1)}$ and $f=f^{(0)}+f^{(1)}$, retaining $\mathcal O(\partial_xT)$ terms. For a \emph{two-branch} bosonic subspace $s,s'$ that interacts through $\omegas$ and is near-degenerate by $\delta=\Omega_{s'}-\Omega_s$, with a single electronic transition $\Delta$, the off-diagonal coherence $N^{(1)}_{ss'}$ obeys
            \begin{equation}
                \Big[i\delta + \tfrac12(\gamma_s+\gamma_{s'})\Big] N^{(1)}_{ss'}
                \;=\; -\frac{1}{2} V^{(x)}_{ss'}\, \partial_x N^{(0)}_{\mathrm{pop}}(\Omega) \; -\; \frac{i}{\hbar}\,\Xi_{ss'}\,,
                \label{eq:Noff}
            \end{equation}
            where $\gamma$ are the linewidths and $\Xi_{ss'}$ is the electron-induced source from $\mathcal C_{e\leftrightarrow b}$ (proportional to the vertex $M$ and to $f^{(1)}$). The population correction satisfies
            \begin{equation}
                \gamma_s\, N^{(1)}_{ss} + V^{(x)}_{ss}\,\partial_x N^{(0)}_{ss} + 2\,\mathrm{Im}\!\big( V^{(x)}_{ss'}\,N^{(1)}_{s's}\big) = S^{(e)}_s\,,
                \label{eq:Ndiag}
            \end{equation}
            with $S^{(e)}_s$ collecting the remaining electron-related terms.

        \subsection{Closed form for the coherence contribution to $\kappa$}
            The heat current density for bosonic modes is $J_x= \Tr\!\big[ \{V_x, N\}\,\Omega/2 \big]$ \cite{Hardy1963,Simoncelli2019Unified}. Using Eqs.~\eqref{eq:Noff}--\eqref{eq:Ndiag} and eliminating $f^{(1)}$ in the weak-coupling (Born) limit yields the \emph{coherence} part of the 1D thermal conductivity
            \begin{equation}
                \boxed{\;\kappa^{(\mathrm C)}_{\!\,1\mathrm D}\;=\;\sum_{q}\sum_{s\neq s'} \frac{(\Omega_s+\Omega_{s'})\;\Gamma_{ss'}\; |V^{(x)}_{ss'}|^2}{4\delta^2+\Gamma_{ss'}^2}\;\bigg(-\frac{\partial n_B}{\partial T}\bigg)\; +\; \mathcal O(|M|^2)\;,}\label{eq:kC}
            \end{equation}
            with $\Gamma_{ss'}=\tfrac12(\gamma_s+\gamma_{s'})$ and $n_B$ the Bose function. Equation~\eqref{eq:kC} reduces to Peierls (no off-diagonals) and to Allen--Feldman (flat bands, $V_{ss}\!\to\!0$) in the appropriate limits \cite{Peierls1929,AllenFeldman1993,Simoncelli2019Unified}. The $\mathcal O(|M|^2)$ terms add an \emph{electron-assisted} channel that shares the Lorentzian denominator and peaks at small detuning.

    \section{1D slab: temperature field and $\Delta\kappa/\kappa$}
        For a bar of length $L$, cross-section $A$, and conductivity $\kappa=\kappa^{(\mathrm P)}+\kappa^{(\mathrm C)}$, a steady power $P$ applied at $x=0$ with a sink at $x=L$ gives a uniform gradient $\partial_x T = -P/(\kappa A)$ and hence
        \begin{equation}
            \Delta T \equiv T(0)-T(L) = \frac{P\,L}{\kappa A}\,.
        \end{equation}
        A small SST-induced change $\Delta\kappa$ then produces
        \begin{equation}
            \boxed{\;\Delta(\Delta T) \approx -\frac{\Delta\kappa}{\kappa}\,\Delta T\;,}\label{eq:deltaT}
        \end{equation}
        valid for $|\Delta\kappa|\ll\kappa$. Equations~\eqref{eq:kC} and \eqref{eq:deltaT} directly connect a measured temperature drop to the microscopic parameters $\delta,\Gamma,$ and $V_{ss'}$.

    \section{Quantitative benchmarks with materials}
        The following order-of-magnitude estimates use Eq.~\eqref{eq:deltaT} and standard catalog values. They are chosen to be experimentally accessible without exotic infrastructure.

        \subsection*{(B1) Borosilicate glass bar}
            Take $L=\SI{50}{mm}$, $A=\SI{1e-4}{m^2}$ (\SI{10}{mm}$\times$\SI{10}{mm}), and $\kappa\approx\SI{1.1}{W\,m^{-1}\,K^{-1}}$. With $P=\SI{20}{mW}$, the baseline is $\Delta T \approx P L/(\kappa A) \approx \SI{9}{K}$. If an engineered near-degeneracy yields $\Delta\kappa/\kappa=\SI{-2}{\percent}$ from Eq.~\eqref{eq:kC}, then $\Delta(\Delta T)\approx\SI{+0.18}{K}$, comfortably above typical IR-camera NETD ($\sim\SI{30}{mK}$).

        \subsection*{(B2) PMMA bar (low-$\kappa$ polymer)}
            With $\kappa\approx\SI{0.19}{W\,m^{-1}\,K^{-1}}$, keep $L=\SI{50}{mm}$ and $A=\SI{1e-4}{m^2}$, and use $P=\SI{2}{mW}$ to avoid overheating. The baseline is $\Delta T\!\approx\!\SI{5.3}{K}$. A conservative $\Delta\kappa/\kappa=\SI{-1}{\percent}$ gives a \SI{53}{mK} shift—still above NETD.

        \subsection*{(B3) Forward/backward nonreciprocity}
            Bias chirality by driving a 3-phase Rodin coil with phase sequence $\pm(0,120^\circ,240^\circ)$. The expected asymmetry is
            \begin{equation}
                \big[\Delta\kappa\big]_{\rightarrow}-\big[\Delta\kappa\big]_{\leftarrow} \equiv \Delta\kappa_\text{asym} \sim \eta_\chi\, \frac{\Gamma\,\Delta V_{ss'}^{2}}{4\delta^2+\Gamma^2}\,,\qquad 0<\eta_\chi<1\,.
            \end{equation}
            Taking $\Delta\kappa_\text{asym}/\kappa\sim\SI{0.5}{\percent}$ implies $\Delta(\Delta T)\sim\SI{25}{mK}$ for (B1), resolvable with modest averaging.

    \section{Device recipes}
        \textbf{Thermal bar (B1/B2).} Mount the bar on an AlN heat sink at $x=L$. Use a \SI{100}{\Omega} thin-film resistor at $x=0$ as a four-wire calibrated heater. Suppress convection with a small enclosure (foam plus a thin IR window). Read out an IR camera or a thermistor chain along $x$. The coil: 3-phase, $N\!\sim\!200$ turns/phase, $f\in[\SI{20}{kHz},\SI{100}{kHz}]$, current $\le\SI{0.5}{A}$, duty-cycled to limit Joule heating.

        \textbf{Electronics analog (LCR).} Two LCR tanks at \SI{1}{MHz} with $Q\!\sim\!100$ (so $\kappa=\omega/2Q\approx3.1\times10^4\,\si{s^{-1}}$). With stored energy $E\!\sim\!\SI{0.5}{nJ}$, the instantaneous bath power is $P_\text{bath}=\kappa E\sim\SI{16}{\mu W}$. Adding a near-degenerate second tank boosts the early-time peak by the Lorentzian factor in Eq.~\eqref{eq:kC}.

        \textbf{Quantum hybrid (SAW/MEMS).} On 128$^\circ$ Y-cut LiNbO$_3$, use an IDT pair to define a \SI{3}{GHz} SAW mode and couple it capacitively to a superconducting qubit \cite{Aspelmeyer2014,Manenti2017}. Pattern shallow quasi-periodic notches to enhance $V^{(x)}_{ss'}$ and tune the detuning $\delta$.

    \section{Error and noise budget}
        \begin{itemize}
            \item \textbf{Thermometry.} IR camera NETD \SI{30}{--\,50}{mK}; thermistors can achieve $\lesssim\SI{10}{mK}$ with \SI{1}{s} averaging.
            \item \textbf{Power calibration.} Four-wire measurements keep heater power to $<\!\SI{1}{\percent}$ uncertainty.
            \item \textbf{Radiation/convection.} With the enclosure, systematic drift is typically $\lesssim\SI{0.05}{K}$ over \SI{10}{min}. Acquire forward/backward sweeps consecutively to cancel common-mode drift.
            \item \textbf{Contact resistance.} Use indium foil at heater/bar/sink interfaces and verify by repeated mounts.
        \end{itemize}
        Expected signals in the \SI{50}{--\,200}{mK} range clear the combined noise by factors $\gtrsim3$ for (B1/B2).

    \section{Falsifiability criteria}
        The electron--swirl interpretation is \emph{falsified} under the stated drive if any of the following hold:
        \begin{enumerate}
            \item \textbf{No Lorentzian detuning.} $\Delta\kappa(\delta)$ lacks the $(4\delta^2+\Gamma^2)^{-1}$ peak of Eq.~\eqref{eq:kC} at fixed current.
            \item \textbf{No chirality asymmetry.} $|\Delta\kappa_\text{asym}/\kappa| < 3\sigma$, where $\sigma$ is the thermal readout error; target $\le\SI{0.1}{\percent}$ via averaging.
            \item \textbf{Scaling mismatch.} The signal does not scale as $|V^{(x)}_{ss'}|^2$ (via coil current squared) or fails to track $\Gamma$ (via controlled disorder).
        \end{enumerate}

    \section{Connection to quantum information}
        In the Jaynes--Cummings limit \cite{Jaynes1963}, the same vertices $M$ and $V_{ss'}$ that enhance $\kappa^{(\mathrm C)}$ optimize state transfer between electron and swirl modes. In a hybrid device, the \emph{coherence peak} (small $\delta$, moderate $\Gamma$) can be used to channel heat away from a qubit while maintaining phase coherence, paralleling engineered reservoirs \cite{Breuer2002,Aspelmeyer2014}.

    \section{Conclusions}
        This work provides closed-form transport expressions together with concrete device geometries, signal estimates, a noise budget, and falsifiability criteria. The package enables immediate lab tests for coherence-mediated electron--swirl transport in SST and constrains the topological parity implied by the canonical $\boldsymbol{\vswirl}$ definition.

        \begin{acknowledgments}
            I thank the classical foundations of vortex hydrodynamics and unified transport \cite{Madelung1927,Peierls1929,AllenFeldman1993,Simoncelli2019Unified} for inspiration.
        \end{acknowledgments}

    \section*{Data Availability}

        The theoretical models, canonical constants, and source code supporting the findings of this study are openly available. All data files, numerical benchmarks, and code used to derive and validate the coherence-mediated transport expression ($\kappa^{(\mathrm C)}_{\!\,1\mathrm D}$) are accessible on the Zenodo repository, under the persistent identifier of this manuscript, $\mathbf{\doi{\paperdoi}}$.

        \paragraph*{File Manifest and Validation Evidence.}
            The uploaded repository for this paper contains the following structured files and data supporting the quantitative results:
            \begin{itemize}
                \item \textbf{SST Canonical Benchmarking Evidence:} This data set validates the internal consistency of the core canonical parameters ($\mathbf{v}_{\!\boldsymbol{\circlearrowleft}}, r_c, \rho_{\!f}$) against known relativistic limits. It includes the derived Newton's constant ($G_{\mathrm{VAM}}$) and the $\mathbf{6GM/c^2}$ ISCO match, demonstrating the global coherence of the swirl parameters. \newline (Zenodo DOI: \url{10.5281/zenodo.15665432} and \url{10.5281/zenodo.15712578})

                \item \texttt{constants.csv} — The definitive table of $\mathbf{v}_{\!\boldsymbol{\circlearrowleft}}, r_c, \rho_{\!f}$ (SI values); used for calculating the derived scales $\Omega_0$ and $\rho_{\!E}$.
                \item \texttt{benchmarks.csv} — Contains the full experimental specification (materials, geometry, power ($\mathbf{P}$), baseline $\mathbf{\Delta T}$) and the predicted $\Delta(\Delta T)$ signals for scenarios $\mathbf{(B1)}$ and $\mathbf{(B2)}$.
                \item \texttt{kappaC\_validation.ipynb} — Jupyter notebook that analytically verifies the functional form of $\kappa^{(\mathrm C)}_{\!\,1\mathrm D}$, plotting the core Lorentzian factor $\Gamma/(4\delta^2+\Gamma^2)$ (Falsifiability Criterion 1).
                \item \texttt{noise\_budget\_3sigma.ipynb} — Computes the detection signal-to-noise ratio ($\mathbf{SNR}$) against the assumed NETD, explicitly confirming that the predicted signals ($\Delta(\Delta T)$) clear the $\mathbf{3\sigma}$ threshold (Detectability Check).
                \item \texttt{env.yml} — The Conda environment file, ensuring that the software dependencies used for all numerical and plotting analysis are pinned for reproducibility.
            \end{itemize}
            This research is licensed under $\text{CC-BY 4.0}$ and is accessible via the main version DOI: $\mathbf{\doi{\paperdoi}}$.

% --- Manual bibliography block (ok for initial submission). Replace with BibTeX if preferred. ---
            \begin{thebibliography}{99}
                \bibitem{Peierls1929} R. Peierls, Ann. Phys. \textbf{395}, 1055 (1929).
                \bibitem{AllenFeldman1993} P. B. Allen and J. L. Feldman, Phys. Rev. B \textbf{48}, 12581 (1993).
                \bibitem{Simoncelli2019Unified} M. Simoncelli, N. Marzari, and F. Mauri, Nat. Phys. \textbf{18}, 1180 (2022); arXiv:1901.01964.
                \bibitem{Madelung1927} E. Madelung, Z. Physik \textbf{40}, 322 (1927).
                \bibitem{Pati2000} A. K. Pati and S. L. Braunstein, Phys. Lett. A \textbf{268}, 241 (2000).
                \bibitem{Hardy1963} R. J. Hardy, Phys. Rev. \textbf{132}, 168 (1963).
                \bibitem{Jaynes1963} E. T. Jaynes and F. W. Cummings, Proc. IEEE \textbf{51}, 89 (1963).
                \bibitem{Lindblad1976} G. Lindblad, Commun. Math. Phys. \textbf{48}, 119 (1976).
                \bibitem{Breuer2002} H.-P. Breuer and F. Petruccione, \textit{The Theory of Open Quantum Systems}, Oxford (2002).
                \bibitem{Aspelmeyer2014} M. Aspelmeyer, T. J. Kippenberg, and F. Marquardt, Rev. Mod. Phys. \textbf{86}, 1391 (2014).
                \bibitem{Manenti2017} R. Manenti \textit{et al.}, Nat. Commun. \textbf{8}, 975 (2017).
                \bibitem{Cahill2004} D. G. Cahill \textit{et al.}, J. Appl. Phys. \textbf{93}, 793 (2003).
            \end{thebibliography}

\end{document}