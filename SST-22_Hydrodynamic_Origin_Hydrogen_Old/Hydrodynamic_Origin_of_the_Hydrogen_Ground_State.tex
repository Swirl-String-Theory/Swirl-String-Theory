%! Author = Omar Iskandarani
%! Date = 11/26/2025
%! Affiliation = Independent Researcher, Groningen, The Netherlands
%! License = © 2025 Omar Iskandarani. All rights reserved. This manuscript is made available for academic reading and citation only. No republication, redistribution, or derivative works are permitted without explicit written permission from the author. Contact: info@omariskandarani.com
%! ORCID = 0009-0006-1686-3961
%! DOI = 10.5281/zenodo.xxx

\newcommand{\paperdoi}{10.5281/zenodo.18388662}
\newcommand{\papertitle}{Hydrodynamic Origin of the Hydrogen Ground State}

%=========================================
% % PREAMBLE, PACKAGES AND DOCUMENT CONFIGURATION
%=========================================
\documentclass[11pt]{article}
\usepackage{amsmath,amssymb,amsfonts,bm}
\usepackage{siunitx}
\usepackage[hidelinks]{hyperref}
\usepackage[a4paper,margin=1in]{geometry}
\usepackage[T1]{fontenc}
\usepackage[utf8]{inputenc}
\usepackage{graphicx}

% swirl arrows (context-aware)
\newcommand{\swirlarrow}{ \mathchoice{\mkern-2mu\scriptstyle\boldsymbol{\circlearrowleft}}{\mkern-2mu\scriptscriptstyle\boldsymbol{\circlearrowleft}}}
\newcommand{\vswirl}{\mathbf{v}_{\swirlarrow}}
\newcommand{\SwirlClock}{S_{(t)}^{\swirlarrow}}
\newcommand{\Fmaxswirl}{F^{\max}_{\mkern-1mu\scriptscriptstyle\boldsymbol{\circlearrowleft}}}
% swirl arrows Counter Clockwise
\newcommand{\swirlarrowcw}{ \mathchoice{\mkern-2mu\scriptstyle\boldsymbol{\circlearrowright}}{\mkern-2mu\scriptscriptstyle\boldsymbol{\circlearrowright}}}
\newcommand{\vswirlcw}{\mathbf{v}_{\swirlarrowcw}}
\newcommand{\SwirlClockcw}{S_{(t)}^{\swirlarrowcw}}
\newcommand{\Fmaxswirlcw}{F^{\max}_{\mkern-1mu\scriptscriptstyle\boldsymbol{\circlearrowright}}}

\newcommand{\Fmax}{\Fmaxswirl} % default maximal force (left swirl)
\newcommand{\FmaxEM}{F^{\max}_{\mathrm{EM}}}
\newcommand{\FmaxG}{F_{\mathrm{G}}^{\max}}               % G-like maximal force scale

\newcommand{\omegas}{\boldsymbol{\omega}_{\swirlarrow}}  % swirl vorticity
\newcommand{\Om}{\Omega_{\swirlarrow}}                   % swirl angular frequency profile

\newcommand{\vscore}{v_{\swirlarrow}}                    % shorthand: |v_swirl| at r=r_c
\newcommand{\vnorm}{\lVert \mathbf{v}_{\mkern-2mu\scriptscriptstyle\boldsymbol{\circlearrowleft}} \rVert}               % swirl speed magnitude
\newcommand{\Ce}{\vswirl}                                % canonical swirl-speed constant

\newcommand{\rhof}{\rho_{\!f}}                           % effective fluid density
\newcommand{\rhoE}{\rho_{\!E}}                           % swirl energy density
\newcommand{\rhom}{\rho_{\!m}}                           % mass-equivalent density
\newcommand{\rc}{r_c}                                    % string core radius (swirl string radius)

\newcommand{\Lam}{\Lambda}                               % Swirl Coulomb constant
\newcommand{\alpg}{\alpha_g}                             % gravitational fine-structure analogue

\newcommand{\titlepageOpen}{
    \begin{titlepage}
        \thispagestyle{empty}
        \centering
        \Large \bfseries \papertitle \par \vspace{1cm}
        {\Large \itshape \textbf{Omar Iskandarani}\textsuperscript{\textbf{*}} \par}
        \vspace{0.5cm}
        {\today \par}
        \vspace{0.5cm}
}

\newcommand{\titlepageClose}{
        \vfill \raggedright \null
        \begin{picture}(0,0)
            \put(0,-45){  % Shift 200pt left, 40pt down
                \begin{minipage}[b]{0.7\textwidth} \footnotesize
                    \renewcommand{\arraystretch}{1.0}
                    \noindent\rule{\textwidth}{0.4pt} \\[0.5em]
                    \textsuperscript{\textbf{*}} Independent Researcher, Groningen, The Netherlands \\
                    Email: \texttt{info@omariskandarani.com} \\
                    ORCID: \texttt{\href{https://orcid.org/0009-0006-1686-3961}{0009-0006-1686-3961}} \\
                    DOI: \href{https://doi.org/\paperdoi}{\paperdoi}
                \end{minipage}
            }
        \end{picture}
    \end{titlepage}
}
%=========================================
% Start Document - Title Page
%=========================================
\begin{document}
    \titlepageOpen
        \begin{abstract}
    
        \end{abstract}
    \titlepageClose
%=========================================
% Title Page End
%=========================================
%======================================================================
% Minimal SST macro prelude
%======================================================================
    \newcommand{\vswirl}{\mathbf{v}_{\!\boldsymbol{\circlearrowleft}}}
    \newcommand{\rhoF}{\rho_{\!f}}
    \newcommand{\rhoE}{\rho_{\!E}}
    \newcommand{\rhoM}{\rho_{\!m}}
    \newcommand{\rc}{r_c}

%======================================================================
    \section{Primitive Circulation-Based Canon}
        \label{sec:canon_circulation}
%======================================================================

        \subsection{Primitive dimensional constants}

            We take as primitive the following three dimensional quantities:
            \begin{align}
                \Gamma_0 &:\ \text{circulation quantum of a single swirl string},
                &
                [\Gamma_0] &= L^2 T^{-1},
                \label{eq:Gamma0_primitive}
                \\
                \rhoF &:\ \text{effective fluid density},
                &
                [\rhoF] &= M L^{-3},
                \label{eq:rhoF_primitive}
                \\
                \rc &:\ \text{core radius (electron-scale reference length)},
                &
                [\rc] &= L.
                \label{eq:rc_primitive}
            \end{align}
            All other dimensional quantities in SST are defined as derived combinations of
            $(\Gamma_0,\rhoF,\rc)$ and dimensionless coefficients.

        \subsection{Kinematic axiom: Kelvin circulation theorem}

            We assume an incompressible, inviscid, barotropic fluid with no external
            body forces. The velocity field $\mathbf{v}(\mathbf{x},t)$ obeys the
            Euler equations, and the circulation
            \begin{equation}
                \Gamma(C,t) = \oint_{C(t)} \mathbf{v}\cdot d\boldsymbol{\ell}
            \end{equation}
            around any material contour $C(t)$ advected by the flow is conserved:
            \begin{equation}
                \frac{D\Gamma}{Dt} = 0.
                \label{eq:Kelvin_circulation}
            \end{equation}
            Equation~\eqref{eq:Kelvin_circulation} is the kinematic backbone of SST,
            encoding the frozen-in character of swirl strings.
% (Helmholtz~\cite{Helmholtz1858}, Thomson/Kelvin~\cite{Thomson1869},
% Batchelor~\cite{Batchelor1967}, Saffman~\cite{Saffman1992}.)

        \subsection{Quantization axiom: circulation quanta}

            Swirl strings are modeled as thin-tube regions of concentrated vorticity.
            For each closed string $K$ we assign an integer circulation quantum $n_K$,
            and postulate
            \begin{equation}
                \Gamma_K = \oint_{K} \mathbf{v}\cdot d\boldsymbol{\ell}
                = n_K \Gamma_0, \qquad n_K \in \mathbb{Z}.
                \label{eq:Gamma_quantization}
            \end{equation}
            This is the circulation analogue of the quantized vortex condition in
            superfluids,
            \begin{equation}
                \Gamma_n = n\kappa, \qquad \kappa = \frac{h}{m},
            \end{equation}
            as established by Onsager and Feynman.\footnote{%
        See Onsager~\cite{Onsager1949}, Feynman~\cite{Feynman1955},
        and Donnelly~\cite{Donnelly1991} for quantized circulation in
        superfluid helium.
    }
            In SST, $\Gamma_0$ is taken as a universal topological constant.

        \subsection{Derived swirl speed scale}

            We define a canonical swirl speed at the core boundary of a reference
            string by
            \begin{equation}
                \lVert\vswirl\rVert
                \equiv \chi_v \frac{\Gamma_0}{2\pi\rc},
                \label{eq:vswirl_from_Gamma}
            \end{equation}
            where $\chi_v$ is a dimensionless geometrical factor that encodes the
            deviation from a pure Rankine vortex profile. Dimensional consistency is
            manifest:
            \([ \Gamma_0 / \rc ] = L^2T^{-1}/L = LT^{-1}.\)

            For a string $K$ with circulation quantum $n_K$, the characteristic swirl
            speed scales as
            \begin{equation}
                \lVert\vswirl(K)\rVert \sim |n_K|\,
                \lVert\vswirl\rVert.
            \end{equation}

        \subsection{Derived force (tension) and energy scales}

            The effective force or tension scale associated with swirl strings is
            defined by combining the fluid density and circulation:
            \begin{equation}
                F_{\text{swirl}}^{\max}
                \equiv \chi_F\,\rhoF\,\Gamma_0^2,
                \label{eq:Fmax_from_Gamma}
            \end{equation}
            with $\chi_F$ dimensionless. The dimensions are
            \(
            [\rhoF\,\Gamma_0^2] = (ML^{-3})(L^4T^{-2})
            = ML T^{-2}.
            \)
            Equation~\eqref{eq:Fmax_from_Gamma} replaces the earlier postulation of
            $F_{\text{swirl}}^{\max}$ as primitive; in this canon, it is a derived
            quantity.

            A characteristic energy scale for the core region is defined as
            \begin{equation}
                E_{\text{core}}
                \equiv \chi_E\,\rhoF\,\Gamma_0^2 \rc,
                \label{eq:Ecore_from_Gamma}
            \end{equation}
            which has dimensions of energy $[E] = ML^2T^{-2}$.
            This can be interpreted as the integrated rotational kinetic energy
            density of a minimally structured core.

        \subsection{Electron as reference knot}

            We identify the electron with a specific knot type $K_e$ and circulation
            quantum $n_e$. The electron rest energy sets
            \begin{equation}
                E_{\text{core}}(K_e)
                = m_e c^2,
                \label{eq:electron_energy_match}
            \end{equation}
            with
            \begin{equation}
                E_{\text{core}}(K_e)
                = \Xi_e\,E_{\text{core}}
                = \Xi_e\,\chi_E \rhoF \Gamma_0^2 \rc.
            \end{equation}
            Here $\Xi_e$ is a dimensionless factor encoding the detailed geometry of
            $K_e$ (tube length, curvature, knot type). Equation
            \eqref{eq:electron_energy_match} constrains the combination
            $\Xi_e \chi_E \rhoF \Gamma_0^2 \rc$ to reproduce $m_e c^2$.

            Similarly, the electron-scale relations involving the classical radius,
            Compton frequency, and hydrogen ground-state energy select a consistent
            set of dimensionless parameters
            \(
            \{\chi_v,\chi_F,\chi_E,\Xi_e,\dots\}
            \)
            given $(\Gamma_0,\rhoF,\rc)$.

        \subsection{Swirl energy density and effective mass density}

            Locally, the swirl energy density is taken as
            \begin{equation}
                \rhoE = \frac{1}{2}\rhoF \lVert\vswirl\rVert^2,
                \label{eq:rhoE_swirl}
            \end{equation}
            and the associated mass-equivalent density is
            \begin{equation}
                \rhoM = \frac{\rhoE}{c^2}.
            \end{equation}
            Substituting Eq.~\eqref{eq:vswirl_from_Gamma} into
            Eq.~\eqref{eq:rhoE_swirl} yields
            \begin{equation}
                \rhoE = \frac{1}{2}\rhoF
                \left(\chi_v \frac{\Gamma_0}{2\pi \rc}\right)^2
                = \frac{\chi_v^2}{8\pi^2}\,
                \rhoF\,\frac{\Gamma_0^2}{\rc^2}.
            \end{equation}

        \subsection{Gravitational coupling from $(\Gamma_0,\rhoF,\rc)$}

            The swirl-based gravitational coupling $G_{\text{swirl}}$ is taken to be
            a dimensionally consistent combination of the primitive constants and
            the speed of light:
            \begin{equation}
                G_{\text{swirl}}
                = \lambda_G\,
                \frac{\Gamma_0^{\,a}\,\rhoF^{\,b}\,\rc^{\,d}}{c^{\,e}},
                \label{eq:Gswirl_circulation_general}
            \end{equation}
            with integers (or rationals) $(a,b,d,e)$ chosen such that
            \(
            [G_{\text{swirl}}] = M^{-1}L^3T^{-2}.
            \)
            The dimensionless coefficient $\lambda_G$ is then determined by matching
            $G_{\text{swirl}}$ to Newton's constant $G$.
            In this formulation, gravity emerges as a coarse-grained response to the
            circulation-controlled swirl energy density $\rhoE$.

        \subsection{Knot--particle dictionary in circulation language}

            For a general knot $K$, we assign:

            \begin{itemize}
                \item circulation quantum $n_K$ via Eq.~\eqref{eq:Gamma_quantization},
                \item geometric invariants: tube length $L(K)$, crossing number $C(K)$,
                helicity $\mathcal{H}(K)$,
                \item physical attributes: mass $m_K$, spin $s_K$, charge $q_K$.
            \end{itemize}

            The effective mass is modeled as
            \begin{equation}
                m_K = \frac{1}{c^2}
                \left[\alpha\,C(K) + \beta\,L(K) + \gamma\,\mathcal{H}(K)\right]
                \rhoF\,\Gamma_0^2 \rc,
            \end{equation}
            where $(\alpha,\beta,\gamma)$ are dimensionless coefficients and we have
            factored out the canonical energy scale $\rhoF\Gamma_0^2 \rc$.
            Additional discrete factors (e.g.\ golden-layer weights) can be included
            as multiplicative corrections to the bracketed term.
            This realizes a circulation-based mass functional in which $\Gamma_0$
            and $\rhoF$ provide the dimensional backbone and $K$'s topology selects
            the dimensionless multipliers.

    \begin{figure}[t]
        \centering
        \includegraphics[width=0.9\linewidth]{SST_energy_spectrum}
        \caption{%
            \textbf{SST energy spectrum as hydrodynamic flow regimes.}
            The classical Bohr energy levels of hydrogen are reinterpreted in
            Swirl--String Theory (SST) as distinct hydrodynamic flow regimes of
            an incompressible background medium rather than as abstract
            ``shells'' in configuration space.
            \textbf{Left:} The usual bound states $n=1,2,3,\dots$ are organised
            as a ladder in which each level is characterised by an orbital fluid
            speed
            \(
            v_n = \alpha c / n
            \),
            with $\alpha$ the fine-structure constant and $c$ the speed of
            light. As $n$ increases, the orbital swirl weakens and the effective
            flow speed decreases, $v_n \to 0$ as $n\to\infty$, corresponding to
            ionisation.
            \textbf{Right:} The ground state $n=1$ is shown as a geometric
            stability boundary of the vacuum ``engine''. The green orbit marks
            the innermost stable laminar flow, at which the electron circulates
            with speed
            \(
            v_1 = \alpha c \approx 2.19\times 10^6\ \mathrm{m/s}.
            \)
            The red inner region indicates radii for which a hypothetical orbit
            would require $v>\alpha c$. In SST, such a regime lies beyond the
            Mach limit of the vacuum texture: the flow would become
            hydrodynamically unstable (turbulent or singular), and no stationary
            laminar swirl-string configuration exists. The Bohr ground state
            thereby acquires a hydrodynamic interpretation: the electron cannot
            ``fall into the nucleus'' because the medium admits no stable
            laminar orbit at radii smaller than $n=1$.
        }
        \label{fig:SST_energy_spectrum}
    \end{figure}




%======================================================================
        \subsection{Hydrodynamic Origin of the Hydrogen Ground State}
            \label{sec:SST_hydrogen_groundstate}
%======================================================================

            In conventional quantum mechanics the discrete spectrum of the Hydrogen
            atom,

            \begin{equation}
                E_n = -\,\frac{13.6~{\rm eV}}{n^2},
                \qquad n=1,2,3,\ldots,
            \end{equation}

            is regarded as a purely spectral property of the Coulomb Hamiltonian.
            Within Swirl--String Theory (SST), the same spectrum is reinterpreted as
            a hierarchy of \emph{stationary incompressible flow regimes} sustained
            by the orbital swirl structure of the electron string around the
            protonic core.

%----------------------------------------------------------------------
            \subsubsection*{Orbital swirl velocity as the principal quantum number}
%----------------------------------------------------------------------

                The Bohr orbital velocity,

                \begin{equation}
                    v_n = \frac{\alpha c}{n},
                \end{equation}

                is taken to represent the coarse--grained swirl speed of the electron
                string along a circular streamline at radius

                \begin{equation}
                    r_n = \frac{n^2 a_0}{1},
                \end{equation}

                with $a_0$ the Bohr radius. As $n$ increases, the swirl becomes
                progressively weaker and more diffuse. In the limit $n\to\infty$,
                $v_n\to 0$ and the flow approaches the unbound (ionised) regime.

                The principal quantum number labels discrete \emph{laminar} flow
                patterns supported by the medium.

%----------------------------------------------------------------------
            \subsubsection*{The electron-scale constraint}
%----------------------------------------------------------------------

                Independently, the SST electron-scale derivation relates the core radius
                $r_c$, the swirl speed $\lVert \mathbf{v}_{\!\boldsymbol{\circlearrowleft}}
        \rVert$, and the swirl energy density $\rho_{\!E}$ via

                \begin{equation}
                    \rho_{\!E}
                    = \frac{1}{2}\rho_{\!f}
                    \lVert \mathbf{v}_{\!\boldsymbol{\circlearrowleft}} \rVert^{2}.
                \end{equation}

                Using the canonical swirl speed value established in the SST Canon,

                \begin{equation}
                    \lVert \mathbf{v}_{\!\boldsymbol{\circlearrowleft}}\rVert
                    \approx 1.0938\times 10^{6}~{\rm m/s}
                    \approx \frac{1}{2}\alpha c,
                \end{equation}

                we identify the internal vorticity scale as exactly half the vacuum Mach
                limit. This factor of $1/2$ is characteristic of the dipole topology
                of the vortex loop.

%----------------------------------------------------------------------
            \subsubsection*{Ground-state stability from a hydrodynamic speed limit}
%----------------------------------------------------------------------

                Combining the orbital relation $v_n = \alpha c / n$ with the
                electron-scale constraint reveals a hydrodynamic interpretation of the
                Bohr ground state.

                At $n=1$, the orbital velocity reaches the vacuum limit:

                \begin{equation}
                    v_1 = \alpha c = 2 \lVert \mathbf{v}_{\!\boldsymbol{\circlearrowleft}}\rVert.
                \end{equation}

                This velocity $v_1 = \alpha c$ represents the \emph{maximum laminar
    translation speed} permitted by the vacuum flow texture (the transverse
                Mach limit). Thus the $n=1$ state sits at the boundary between
                admissible laminar flow and a regime in which the required flow speed
                would exceed the vacuum stability limit.

                For any hypothetical ``sub-Bohr'' value $n<1$, the orbital velocity
                would satisfy

                \begin{equation}
                    v_n = \frac{\alpha c}{n} > \alpha c,
                \end{equation}

                forcing the flow into a non-laminar (turbulent or singular) regime
                analogous to a sonic boom. In SST such a configuration cannot sustain
                a stationary swirl string, and therefore \emph{no bound state exists
    for $r < a_0$}.

                The ground state is not imposed by abstract quantisation but arises
                dynamically as the innermost stable laminar flow configuration permitted
                by the fluid properties of the vacuum.


                %======================================================================
        \subsection{Hydrodynamic Derivation of the Rydberg Constant}
            \label{sec:SST_rydberg}
%======================================================================

            In the SST framework, the ionization of the Hydrogen atom corresponds to
            the acceleration of the electron vortex string from its stable ground-state
            orbit ($n=1$) to the unbound vacuum flow regime ($n \to \infty$).

            The energy required for this transition—the Rydberg energy $E_{Ry}$—is
            identifiable not as an electrostatic potential difference, but as the
            \emph{kinetic energy} of the electron vortex traveling at the vacuum
            stability limit.

%----------------------------------------------------------------------
            \subsubsection*{Rydberg Energy as Vacuum Kinetic Limit}
%----------------------------------------------------------------------

                From the preceding section, the ground-state orbital velocity $v_1$ is
                defined by the transverse Mach limit of the vacuum:

                \begin{equation}
                    v_1 = \alpha c.
                    \label{eq:v1_mach}
                \end{equation}

                The classical kinetic energy $T_1$ of the electron mass $m_e$ moving
                at this limit is:

                \begin{equation}
                    T_1 = \frac{1}{2} m_e v_1^2 = \frac{1}{2} m_e (\alpha c)^2.
                \end{equation}

                In standard theory, the Rydberg energy is defined as $hcR_\infty$.
                Equating the hydrodynamic kinetic energy to the spectral energy yields:

                \begin{equation}
                    hcR_\infty = \frac{1}{2} m_e \alpha^2 c^2.
                \end{equation}

%----------------------------------------------------------------------
            \subsubsection*{SST Substitution: The Swirl Velocity Relation}
%----------------------------------------------------------------------

                We now substitute the SST canonical relation between the fine structure
                constant and the intrinsic swirl velocity, $\alpha c = 2 \mathbf{v}_{\!\boldsymbol{\circlearrowleft}}$:

                \begin{equation}
                    hcR_\infty = \frac{1}{2} m_e (2 \mathbf{v}_{\!\boldsymbol{\circlearrowleft}})^2
                    = 2 m_e \mathbf{v}_{\!\boldsymbol{\circlearrowleft}}^2.
                \end{equation}

                Solving for the Rydberg constant $R_\infty$:

                \begin{equation}
                    R_\infty = \frac{2 m_e \mathbf{v}_{\!\boldsymbol{\circlearrowleft}}^2}{hc}.
                    \label{eq:SST_Rydberg}
                \end{equation}

%----------------------------------------------------------------------
            \subsubsection*{Physical Interpretation}
%----------------------------------------------------------------------

                Equation \eqref{eq:SST_Rydberg} provides a purely kinematic definition
                of the Rydberg constant. It states that the fundamental wavenumber of
                atomic spectroscopy is determined by the ratio of the \emph{vortex
    swirl energy} ($m_e \mathbf{v}_{\!\boldsymbol{\circlearrowleft}}^2$) to the \emph{action-speed product} ($hc$).

                Specifically, $R_\infty$ represents the spatial frequency of a wave
                associated with a vortex loop accelerating to twice its intrinsic spin
                velocity. The factor of 2 arises from the geometry of the loop: the
                coherent translation of a dipole structure requires twice the energy
                of a monopole flow of equivalent velocity.

                Thus, in SST, spectral lines are not transitions between abstract
                probability clouds, but are acoustic resonance shifts caused by the
                deceleration of the electron knot from its maximum laminar speed
                ($\alpha c$) to lower harmonic velocities ($v_n = \alpha c / n$).


                %======================================================================
        \subsection{Hydrodynamic Derivation of the Compton Wavelength}
            \label{sec:SST_compton}
%======================================================================

            The Compton wavelength $\lambda_c$ defines the fundamental length scale of
            quantum interaction for a particle of mass $m_e$. In SST, this emerges
            from the helical geometry of the vortex string trajectory.

            We interpret $\lambda_c$ as the \emph{longitudinal spatial period} (or pitch)
            of the vortex filament as it translates at the speed of light $c$,
            governed by the internal gearing ratio $\alpha$.

%----------------------------------------------------------------------
            \subsubsection*{The Geometric Pitch Relation}
%----------------------------------------------------------------------

                Standard electrodynamics establishes the relationship between the
                Classical Electron Radius ($r_e$), the Fine Structure Constant ($\alpha$),
                and the Compton Wavelength ($\lambda_c$):

                \begin{equation}
                    r_e = \alpha \frac{\lambda_c}{2\pi}.
                \end{equation}

                In the SST Canon, the geometric Core Radius $r_c$ is exactly half the
                classical radius ($r_e = 2r_c$), reflecting the dipole (loop) topology
                of the knot. Substituting $r_e = 2r_c$:

                \begin{equation}
                    2r_c = \alpha \frac{\lambda_c}{2\pi} \quad \implies \quad
                    \lambda_c = \frac{4\pi r_c}{\alpha}.
                    \label{eq:lambda_geo}
                \end{equation}

                This equation states that the Compton wavelength is the circumference of
                the vortex core ($2\pi r_c$) amplified by the inverse Mach number ($1/\alpha$)
                and a topological factor of 2.

%----------------------------------------------------------------------
            \subsubsection*{SST Substitution: The Helical Pitch Formula}
%----------------------------------------------------------------------

                We now substitute the SST canonical definition of $\alpha$ derived from
                the swirl velocity ($\alpha = 2 \mathbf{v}_{\!\boldsymbol{\circlearrowleft}} / c$) into Eq.~\eqref{eq:lambda_geo}:

                \begin{equation}
                    \lambda_c = \frac{4\pi r_c}{(2 \mathbf{v}_{\!\boldsymbol{\circlearrowleft}} / c)}
                    = \frac{2\pi r_c c}{\mathbf{v}_{\!\boldsymbol{\circlearrowleft}}}.
                    \label{eq:SST_Compton}
                \end{equation}

%----------------------------------------------------------------------
            \subsubsection*{Numerical Verification}
%----------------------------------------------------------------------

                Using the canonical values:
                \begin{itemize}
                    \item $r_c \approx 1.409 \times 10^{-15}$ m
                    \item $c \approx 3.00 \times 10^8$ m/s
                    \item $\mathbf{v}_{\!\boldsymbol{\circlearrowleft}} \approx 1.094 \times 10^6$ m/s
                \end{itemize}

                \begin{equation}
                    \lambda_c \approx \frac{2\pi (1.409 \times 10^{-15})(2.998 \times 10^8)}{1.094 \times 10^6}
                    \approx 2.426 \times 10^{-12} \, \text{m}.
                \end{equation}

                This matches the CODATA value for the Compton wavelength of the electron
                ($2.42631 \times 10^{-12}$ m).

%----------------------------------------------------------------------
            \subsubsection*{Physical Interpretation: The Vacuum Screw}
%----------------------------------------------------------------------

                Equation \eqref{eq:SST_Compton} reveals the mechanical nature of mass
                transport in the vacuum.
                The term $\frac{2\pi r_c}{\mathbf{v}_{\!\boldsymbol{\circlearrowleft}}}$ represents the \emph{period of one internal rotation}
                of the vortex core. Multiplying by $c$ gives the distance traveled
                during one rotation.

                Thus, the electron behaves as a \textbf{Self-Propelling Screw}:
                \begin{enumerate}
                    \item It spins internally at speed $\mathbf{v}_{\!\boldsymbol{\circlearrowleft}}$.
                    \item It moves forward at speed $c$.
                    \item The "thread pitch" of this motion is exactly $\lambda_c$.
                \end{enumerate}

                Mass, in this view, is the resistance to changing this pitch. A shorter
                wavelength (higher mass) implies a "tighter" screw thread that requires
                more energy to accelerate.
%=========================================
% References
%=========================================
        \bibliographystyle{unsrt}
        \begin{thebibliography}{99}

            \bibitem{Einstein1905} A.~Einstein, \newblock \emph{Ist die Tr\"agheit eines K\"orpers von seinem Energieinhalt
            abh\"angig?}, newblock Ann.\ Phys.\ \textbf{18}, 639--641 (1905).

        \end{thebibliography}

\end{document}