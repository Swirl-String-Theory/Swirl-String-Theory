%! Author = mr
%! Date = 8/27/2025

% Preamble
\documentclass[11pt]{article}

% Packages
\usepackage{amsmath}

% Document
\begin{document}

\section{Mapping Standard Model Particles to SST Knot Structures}

    In Swirl–String Theory (SST), each elementary particle is represented by a quantized knotted vortex loop in an underlying {\ae}ther-like fluid. The knot type and its symmetries encode the particle’s quantum numbers (mass, charge, chirality, \emph{etc.}) \cite{Iskandarani2025Canon,Iskandarani2025Lagrangian}. The canon adopts the following taxonomy:
    \begin{itemize}
    \item \textbf{Unknotted loops} $\rightarrow$ bosons (force carriers).
    \item \textbf{Torus knots} $\rightarrow$ leptons (charged states are chiral torus knots).
    \item \textbf{Chiral hyperbolic knots} $\rightarrow$ quarks.
    \item \textbf{Amphichiral (mirror-symmetric) knots} $\rightarrow$ neutral ``dark'' states (e.g., neutrinos or other non-EM–interacting excitations).
    \end{itemize}
    Below we map the SM content—three generations of quarks and leptons, plus gauge bosons—onto their SST vortex–knot representatives, following the canonical rules and the helicity/mass analyses.

    \subsection{Leptons: Torus Knots and Amphichiral Hyperbolic Knots}

        \paragraph{Charged leptons ($e^-$, $\mu^-$, $\tau^-$).}
            Each charged lepton is identified with a prime \textbf{torus knot} $T(2,2k{+}1)$, reflecting broken mirror symmetry (consistent with nonzero electric charge). The canonical assignments are:
            \[
                e^- \leftrightarrow 3_1\quad(\text{trefoil}),\qquad
                \mu^- \leftrightarrow 5_1\quad(\text{cinquefoil}),\qquad
                \tau^- \leftrightarrow 7_1\quad(\text{septfoil}),
            \]
            with $5_1=T(2,5)$ and $7_1=T(2,7)$ \cite{Iskandarani2025Canon,KnotAtlas}. Increasing winding/twist raises the solitonic energy, tracking the $e\!:\!\mu\!:\!\tau$ mass hierarchy. In the SST invariant-mass model, $(e,\mu)$ calibrate the parameters; $\tau$ then follows once the knot length and layer are fixed \cite{Iskandarani2025Mass}. These knots are reversible (two orientations, particle/antiparticle) but not mirror-symmetric—consistent with a unique charge sign.

        \paragraph{Neutral leptons ($\nu_e,\nu_\mu,\nu_\tau$).}
            Electrically neutral, weakly interacting states are assigned to \textbf{amphichiral hyperbolic knots}, i.e., non-torus knots identical to their mirror image. The canonical pins are
            \[
                \nu_e \leftrightarrow 4_1\quad(\text{figure-eight}),\qquad
                \nu_\mu \leftrightarrow 6_3,\qquad
                \nu_\tau \leftrightarrow 8_3,
            \]
            all amphichiral with high discrete symmetry \cite{KnotAtlas}. In the SST helicity classifier, such amphichiral controls yield $a_{\rm SST}\!\approx\!-0.5$ \cite{Iskandarani2025Helicity}, matching the absence of EM charge and the near-perfect mirror balance. Higher crossing number modestly increases hyperbolic volume and thus mass, aligning with the observed neutrino hierarchy.

\subsection{Quarks: Chiral Hyperbolic Knots (Confined Vortices)}

    Quarks are represented by \textbf{chiral, non-torus (hyperbolic) knots}. Chirality and geometric complexity (crossings, writhe) correlate with larger solitonic energy and reduced stability, mirroring the SM mass/lifetime trends \cite{Iskandarani2025Helicity}.

    \paragraph{Up quark ($u$).}
        The canonical representative is $6_2$ \cite{Iskandarani2025Helicity}. It sits near the amphichiral band—its helicity index $a_{\rm SST}\!\approx\!-0.490$—indicating mild chirality. The hyperbolic complement volume anchors hadronic scaling, with
        \[
            \mathcal{V}_{6_2}=2.8281.
        \]

    \paragraph{Down quark ($d$).}
        Mapped to $7_4$, which exhibits stronger chiral asymmetry with $a_{\rm SST}\!\approx\!-0.523$ \cite{Iskandarani2025Helicity}. Its larger hyperbolic volume,
        \[
            \mathcal{V}_{7_4}=3.1639,
        \]
        supports a higher mass contribution than $u$ \cite{Iskandarani2025Mass}. In nucleons, the triplet of quark knots (e.g., $u\!u\!d$ for $p$; $d\!d\!u$ for $n$) forms a linked, metastable configuration consistent with confinement; the total hadronic scale follows from the sum of constituent volumes/topology \cite{Iskandarani2025Mass}.

    \paragraph{Second and third generations ($c,s,t,b$).}
        Higher generations arise from increased knot complexity (additional layers/winding or higher-crossing chiral analogs of the first-generation representatives) \cite{Iskandarani2025Mass}. For example, a chiral 8-crossing knot in the $u$-family (e.g., $8_{19}$) is a natural charm candidate, while a strongly chiral 9-crossing analog fits the strange sector \cite{KnotAtlas}. The top and bottom correspond to still higher-writhe, high-volume chiral knots (order $\geq 10$ crossings), with the top’s extreme energy rendering it only barely metastable (rapid weak decay).

\subsection{Gauge Bosons and Related Excitations: Unknots and Links}

\paragraph{Photon ($\gamma$).}
    A \textbf{trivial knot} (unknot): a closed loop supporting massless swirl-waves—no chirality, no charge—consistent with the photon’s quantum numbers \cite{Iskandarani2025Canon}.

\paragraph{W and Z bosons.}
    Massive, short-lived excitations interpreted as transiently twisted/linked vortex loops. Charged $W^\pm$ correspond to small chiral deformations; the neutral $Z$ to an achiral excitation. Their lack of topological protection matches their brief lifetimes \cite{Iskandarani2025Lagrangian}.

\paragraph{Gluons ($g$).}
    Modeled as flux-tube excitations linking quark knots—Hopf-linked loops or twisted bridges confined within hadrons—capturing color confinement phenomenology \cite{Iskandarani2025Lagrangian}.

\paragraph{Higgs ($H^0$).}
    A radially excited unknot or a high-symmetry amphichiral knot (e.g., $12a_{1202}$) acting as a scalar mode \cite{KnotAtlas}.

\subsection{Exotic and Composite Constructions}

SST admits richer topologies:
\begin{itemize}
\item \textbf{Hopf link} (two linked unknots): mesonic analogs (quark–antiquark).
\item \textbf{Borromean rings} (three mutually linked loops): baryonic analogs.
\item \textbf{High-crossing amphichiral knots} (e.g., $12a_{1202}$, $15_{331}$): candidates for neutral dark states or sterile neutrinos \cite{KnotAtlas}.
\end{itemize}

\paragraph{Summary.}
    Within the SST canon, topology provides a one-to-one map to particle sectors: torus knots for charged leptons; amphichiral hyperbolic knots for neutrinos; chiral hyperbolic knots for quarks (with complexity tracking generation and mass); and unknotted/linked loops for gauge and composite excitations \cite{Iskandarani2025Canon,Iskandarani2025Helicity,Iskandarani2025Mass}.


\end{document}